\newcount\wordfflag
\wordfflag=0
\newtoks\prontok
\newwrite\allentries
\immediate\openout\allentries=dictionarywords
\newwrite\likeentries
\immediate\openout\likeentries=duplicatewords

\def\mylaunch#1{\pdflaunchlink{a_scripts/#1.sh}{L}\;\;\pdflaunchlink{abatchfiles/#1.bat}{W}}
\newcommand{\pdflaunchlink}[2]{%
\pdfstartlink attr{/Border [0 0 0]} user{/Subtype /Link /A << %
/S /Launch /F (#1) >>}%
\pdfliteral{0 1 0 0 k}%
{#2}\pdfliteral{0 0 0 1 k}\pdfendlink%
}

\def\general#1#2{\hypertarget{#1}\break{}\textcolor{red}{\rm\bfseries #2}\global\wordfflag=0}

\def\word{\futurelet\new\macA}

\def\macA{\ifx\new[\let\next=\wordfirst 
\else
  \ifx\new(\let\next=\wordparfirst
  \else
  \let\next=\wordsecond
  \fi
\fi 
\next}



%%Addanothertarget -- > #1 ---> target
%%                      #2 ---> not needed
\def\addanothertarget#1#2{%
\hyperdef{A}{#1}{}
\hypertarget{#1}{}
\immediate\write\allentries{"#1",}%
}


%%wordwsas -- > #1 ---> hypertarget
%%              #2 ---> material to be typeset(very complex)
%%		#3 ---> see also data
%% #1 will appear in the words list(dictionarywords.tex)
%% words with this macros don't have speech, because of special case.
%% This macro is generaly used for the comma seperated entries. 
%% ex: (co-opt, coopt) (Z,z) (K,k)
\def\wordwsas#1#2#3{%
\hyperdef{A}{#1}{}%
\hypertarget{#1}\break{#3}%
\hfil\break\textcolor{red}{{\rm\bfseries #2}}%
\immediate\write\allentries{"#1",}%
\global\wordfflag=1}

%%wordwosas -- > #1 ---> hypertarget
%%               #2 ---> data to be typeset(in some cases it is
%%               very complex for example see word cooperate in letter C
%% #1.sh and #1.bat for speech.
%% #1 will appear in the words list.              
\def\wordwosas#1#2{%
\hyperdef{A}{#1}{}%
\hypertarget{#1}\break{}%
\hfil\break\textcolor{red}{{\rm\bfseries #2}}%
\immediate\write\allentries{"#1",}%
\prontok={#1}\global\wordfflag=0}

%%wordwosasf -- > #1 ---> hypertarget
%%                #2 ---> data to be typeset
%% words with this macro don't have speech
%% #1 will appear in the list of words
%% This macro is for foreign words
\def\wordwosasf#1#2{%
\hyperdef{A}{#1}{}%
\hypertarget{#1}\break{}%
\hfil\break\textcolor{red}{{\enbi #2}}%
\immediate\write\allentries{"#1",}%
\global\wordfflag=1}


%%wordsidentical -- > #1 ---> hypertarget 
%%                    #2 ---> data to be typeset.
%% #1 will appear in the list of words. 
%% #2.sh and #2.bat for speech		      
\def\wordsidentical#1#2{%
\hyperdef{A}{#1}{}%
\hypertarget{#1}\break{}%
\hfil\break\textcolor{red}{{\rm\bfseries #2}}%
\immediate\write\allentries{"#1",}%
\prontok={#2}\global\wordfflag=0}


%%wordspecial -- > #1(#2) --- > hypertarget
%%                 #3     --- > superscript for word
%%                 #4     --- > see also data
%% #1(#2) will appear in the words list.
%% #1 will be typeset with #3 as superscript.
%% #1.sh and #2.bat for speech.
\def\wordspecial#1#2#3#4{%
\hyperdef{A}{#1(#2)}{}%
\hypertarget{#1(#2)}\break{{\rm\footnotesize\engit{\footnotesize See also #4}
{}}}\hfil\break\textcolor{red}{\textcolor{superscript}{$^{#3}$}{\rm\bfseries #1}}%
\immediate\write\allentries{"#1(#2)",}%
\prontok={#1}\global\wordfflag=0}

%%wordfirst -- > #1(#2) --- > hypertarget
%%               #3     --- > word to be typeset with #2 as superscript 
%%i.e #1 and #3 are same.  
%% #1(#2) will appear in the words list.
%% #1 will appear in duplicatewords.tex
%% #1 will be used for speech
\def\wordfirst[#1(#2)]#3{%
\hyperdef{A}{#1(#2)}{}%
\hypertarget{#1(#2)}\break{{\rm\footnotesize\engit{\footnotesize See also} {\input #1(#2)
}}}\hfil\break\textcolor{red}{\textcolor{superscript}{$^{#2}$}{\rm\bfseries #3}}%
\immediate\write\allentries{"#1(#2)",}%
\immediate\write\likeentries{#1}
\prontok={#1}\global\wordfflag=0}

%%wordparfirst -- > #1 #2(#3) --- > hypertarget
%%                  #4        --- > word to be typeset with #3 as superscript
%% i.e #1\ #2 and #4 are identical.  
%% #1\ #2(#3) will appear in the list of words.
%% #1\ #2 will appear in duplicatewords.tex
%% #1-#2 will be used for speech (#1-#2.sh #1-#2.bat)
\def\wordparfirst(#1 #2[#3])#4{%
\hyperdef{A}{#1 #2(#3)}{}%
\hypertarget{#1 #2(#3)}\break{{\rm\footnotesize\engit{\footnotesize See also} {\input #1-#2(#3)
}}}\hfil\break\textcolor{red}{\textcolor{superscript}{$^{#3}$}{\rm\bfseries #4}}%
\immediate\write\allentries{"#1 #2(#3)",}%
\immediate\write\likeentries{#1 #2}
\prontok={#1-#2}\global\wordfflag=0}

%%wordsecond -- > #1 --- > hypertarget
%% #1 will appear in the words list.
%% #1 will be used for speech.
%% #1 will be typeset.               
\def\wordsecond#1{%
\hyperdef{A}{#1}{}%
\hypertarget{#1}\break{\textcolor{red}{\rm\bfseries #1}}
\immediate\write\allentries{"#1",}\prontok={#1}
\global\wordfflag=0}

%%wordwithhyphen  -- > #2 --- > hypertarget and word to be typeset.
%% #1 will be used for the speech(#1.sh and #1.bat)
%% #1 will appear in the words list.
%% This macro is used for the words which starts with hyphen ex: -cele 
\def\wordwithhyphen#1#2{%
\hyperdef{A}{#1}{}%
\hypertarget{#1}\break{\textcolor{red}{\rm\bfseries #2}}
\immediate\write\allentries{"#1",}
\prontok={#1}\global\wordfflag=0}

%%wordnospeech -- > #1 --- > hypertarget
%%                  #2 --- > word to be typeset.
%% #1 will appear in the list of words.
%% No speech   		     
\def\wordnospeech#1#2{%
\hyperdef{A}{#1}{}%
\hypertarget{#1}\break{\textcolor{red}{\rm\bfseries #2}}
\immediate\write\allentries{"#1",}}

%%wordRemoveSpace -- > #2 --- > hypertarget
%%                     #2 --- > word to be typeset
%% #2 will appear in the words list.
%% #1 for speech (#1.sh #1.bat)
\def\wordRemoveSpace#1#2{%
\hyperdef{A}{#2}{}%
\hypertarget{#2}\break{\textcolor{red}{\rm\bfseries #2}}
\immediate\write\allentries{"#2",}
\prontok={#1}\global\wordfflag=0}


%%fiveargs -- > #1\ #2\ #3(#4) --- > hypertarget
%%              #5             --- > word to be typeset with #4 as superscript
%% i.e #1\ #2\ #3 and #5 are identical
%% #1\ #2\ #3(#4) will appear in the words list.
%% #1\ #2\ #3 will appear in the duplicatewords.tex
%% #1-#2-#3 for speech (#1-#2-#3.sh and #1-#2-#3.bat)
\def\fiveargs#1 #2 #3(#4)#5{%
\hyperdef{A}{#1 #2 #3(#4)}{}%
\hypertarget{#1 #2 #3(#4)}\break{{\rm\footnotesize\engit{\footnotesize See also} {\input #1-#2-#3(#4)
}}}\hfil\break\textcolor{red}{\textcolor{superscript}{$^{#4}$}{\rm\bfseries #5}}
\immediate\write\allentries{"#1 #2 #3(#4)",}
\immediate\write\likeentries{#1 #2 #3}
\prontok={#1-#2-#3}\global\wordfflag=0}


%% Following are the macros which are used to typeset foreign
%% words, and for foreign words we are leaving speech part.


\def\wordf{\global\wordfflag=1\futurelet\new\macAf}

\def\macAf{\ifx\new[\let\next=\wordfirstf 
\else
  \ifx\new(\let\next=\wordparfirstf
  \else
  \let\next=\wordsecondf
 \fi
\fi\next}


%%wordfirstf  -- > #1(#2) --- > hypertarget
%%                 #3     --- > word to be typeset with #2 as superscript
%% i.e. #1 and #3 are identical.
%% #1(#2) will appear in the words list
%% #1 will appear in the duplicatewords.tex
%% No speech   
\def\wordfirstf[#1(#2)]#3{%
\hyperdef{A}{#1(#2)}{}%
\hypertarget{#1(#2)}\break{{\rm\footnotesize\engit{\footnotesize See also} {\input #1(#2)
}}}\hfil\break\textcolor{red}{\textcolor{superscript}{$^{#2}$}{\enbi #3}}%
\immediate\write\allentries{"#1(#2)",}%
\immediate\write\likeentries{#1}}


%%wordparfirstf  -- > #1 #2(#3) --- > hypertarget
%%                    #4        --- > word to be typeset with #3 as
%%                    superscript
%% #1\ #2 and #4 are identical
%% #1\ #2(#3) will appear in the words list.
%% #1\ #2 will appear in duplicatewords.tex 
%% No speech
\def\wordparfirstf(#1 #2[#3])#4{%
\hyperdef{A}{#1 #2(#3)}{}%
\hypertarget{#1 #2(#3)}\break{{\rm\footnotesize\engit{\footnotesize See also} {\input #1-#2(#3)
}}}\hfil\break\textcolor{red}{\textcolor{superscript}{$^{#3}$}{\enbi #4}}
\immediate\write\allentries{"#1 #2(#3)",}%
\immediate\write\likeentries{#1 #2}}

%%fiveargsf -- > #1 #2 #3(#4) --- > hypertarget
%%               #5           --- > word to be typeset with #4 as
%%               superscript
%% i.e. #1\ #2\ #3  and #5 are identical
%% #1\ #2\ #3(#4) will appear in the words list.
%% #1\ #2\ #3 will appear in duplicatewords.tex 
%% No speech   
\def\fiveargsf#1 #2 #3(#4)#5{%
\hyperdef{A}{#1 #2 #3(#4)}{}%
\hypertarget{#1 #2 #3(#4)}\break{{\rm\footnotesize\engit{\footnotesize See also} {\input #1-#2-#3(#4)
}}}\hfil\break\textcolor{red}{\textcolor{superscript}{$^{#4}$}{\enbi #5}}
\immediate\write\allentries{"#1 #2 #3(#4)",}
\immediate\write\likeentries{#1 #2 #3}
\global\wordfflag=1}

%%wordsecondf  -- > #1 --- > hypertarget and also word to be typeset.
%% #1 will appear in the words list.
%% No speech   
\def\wordsecondf#1{
\hyperdef{A}{#1}{}%
\hypertarget{#1}\break{\textcolor{red}{\enbi #1}}
\immediate\write\allentries{"#1",}}

\def\pron#1{\quad
\ifx?#1
\else
  \ifnum\wordfflag=1
  {\bf #1} 
  \global\wordfflag=0
  \else
  {\bf #1\,}\global\edef\storetemp{\the\prontok}$^{\mylaunch{\storetemp}}$
  \fi
\fi}

\def\gl#1{\quad\textcolor{colorgl}{#1}\par\nobreak}

\long\def\dictmeaning#1{\par\narrower #1\par}
\def\nhang{\hangindent\parindent}
\def\ntextindent#1{\indent\llap{#1\enspace}\ignorespaces}
\def\numb{\par\nhang\ntextindent}
\def\num#1{\ifnum#1=1
\numb{\hfil\vadjust{\vskip\parskip}\break\textcolor{numcolor}{\bfseries #1.}}
\else
\numb{\hfil\vadjust{\vskip\parskip}\break\textcolor{numcolor}{\bfseries #1.}}
\fi}
\def\eng#1{{\rm #1}}
\def\engb#1{{\enb #1}}
\def\engb#1{{\enb #1}}
\def\engit#1{{\eni #1}}
\def\enbi{\fontfamily{cmr}\fontseries{bx}\fontshape{it}\fontsize{12pt}{12pt}\selectfont}
\def\engbit#1{{\enbi{#1}}}
\def\eni{\fontfamily{cmr}\fontshape{it}\fontsize{12pt}{12pt}\selectfont}
\def\enb{\fontfamily{cmr}\fontseries{bx}\fontsize{12pt}{12pt}\selectfont}
\def\alnum#1{\hangindent=32pt\hangafter=1\textcolor{numcolor}{\enb{#1.}}}
\long\def\bentry{\begingroup\noindent}
\def\eentry{\par\endgroup\smallbreak}


