\bentry
\wordRemoveSpace{G,-g}{G, g}
\pron{jiV} 
\gl{\nA}
\expl{}
\bmng
\bnum
\num{1} iMgilxVSf vaNRmAleya ELaneya akaSxra. 
\numi{2} (\saM) 
\banum
\alnum{a} paMcami; iMgilxVSf savxragArxmadalilx aidaneya savxra (``pa''). 
\alnum{b} adakekx saMvAdiyAda AdhAra savxra: paMcami shurxti. 
\eanum
\numie
\enum
\emng
\eentry

\bentry
\wordnospeech{G}{G}
\pron{?}
\gl{\saMkiSx}
\expl{}
\bmng
\bnum
\num{1} \eng{gauss.} 
\num{2} \eng{giga.} 
\num{3} \eng{gravitational constant.} 
\num{4} (\ame, \ashi) sAvira DAlarfgaLu yA pwMDfgaLu 
\enum
\emng
\eentry

\bentry
\wordnospeech{g}{g} 
\pron{?}
\gl{\saMkiSx}
\bmng
\bnum
\num{1} \eng{gelding.} 
\num{2} \eng{gram(s).} 
\num{3} \eng{acceleration due to gravity.} 
\num{4} \eng{gravity.} 
\enum
\emng
\eentry

\bentry
\wordnospeech{GA}{GA} 
\pron{?}
\gl{\saMkiSx}
\expl{}
\bmng
 (\ame) \eng{Georgia.} 
\emng
\eentry

\bentry
\wordnospeech{Ga}{Ga} 
\pron{?}
\gl{\saMkeV}
\expl{}
\bmng
 (\ravi) \eng{gallium.} 
\emng
\eentry

\bentry
\wordnospeech{Ga.}{Ga.}
\pron{?}
\gl{\saMkiSx}
\expl{}
\bmng
(\ame) \eng{Georgia.} 
\emng
\eentry

\bentry
\word{gab} 
\pron{gAYXbf} 
\gl{\nA}
\expl{}
\bmng
(\AmA) (vaqthA) mAtu; haraTe; gapipx. 
\emng

\noindent
\gl{\nuga}
\bmng
\hyperdef{G}{gab nuga(1)}{} 
\hypertarget{gab nuga(1)}{} 
\bnum
\numi{1} \eng{gift of the gab} 
\banum
\alnum{a} vAkfshakitx;BASaNashakitx; vAkapxTutavx; vAgimxte; mAtanADuva shakitx; 
\alnum{b} vAcALate; mAtALitana; haraTe malalxtana. 
\eanum
\numie
\num{2} \eng{stop your gab} haraTe nililxsu; bAyimucucx; mAtu sAkumADu. 
\enum
\emng
\eentry

\bentry
\word{gabardine} 
\pron{gAYXbaDiRVnf} 
\gl{\nA}
\expl{}
\bmng
\bnum
\numi{1} gAYXbaDiRVnf: 
\banum
\alnum{a} \kanmu\ uNeNxya yA hatitxya, mUle neyegxya, nuNupAda gaTiTx baTeTx. 
\alnum{b} A baTeTxyiMda tayArisida uDupu. 
\eanum
\numie
\num{2} maLeyaMgigaLanunx mADalu baLasuva baTeTx, arive. 
\enum
\emng
\eentry

\bentry
\word{gabber}
\pron{gAYXbarf}
\gl{\nA}
\bmng
 haraTemalalx yA haraTemalilx; bAyibaDuka yA bAyibaDuki. 
\emng
\eentry

\bentry
\word[gabble(1)]{gabble}
\pron{gAYXbflf}
\gl{\sakirx}
\bmng
\bnum
\num{1} loTaguTuTx; baDabaDisu; baDabaDane athaRvAgadaMte mAtanADu (\akirx\ saha.) 
\num{2} beVgabeVga gaTiTxyAgi Odu yA mAtanADu. 
\enum
\emng

\noindent
\gl{\akirx}
\bmng
(\kanmu\ OduvAga) gaTiTxyAgi beVgabeVgane Odu. 
\emng
\eentry

\bentry
\word[gabble(2)]{gabble}
\pron{gAYXbflf}
\gl{\nA}
\bmng
baDabaDike; loTaguTuTxvike; athaRvAgadaMte baDabaDane mAtanADuvudu. 
\emng
\eentry

\bentry
\word{gabbler}
\pron{gAYXbalxrf}
\gl{\nA}
\bmng
 athaRvAgadaMte baDabaDane mAtanADuvava. 
\emng
\eentry

\bentry
\word{gabbro}
\pron{gAYXborxV}
\gl{\nA}
\expl{(\bava\ \eng{gabbros}.) }
\bmng
(\BUvi) gArxYXneYTf matutx DAlareYTfgaLanunx hoVluva, oMdu terana parxtAyxmilxVya aginxshile. 
\emng
\eentry

\bentry
\word{gabbroic}
\pron{gAYXbArxikf}
\gl{\gu}
\expl{}
\bmng
(\BUvi) gAYXboVrx shileya yA adakekx saMbaMdhisida. 
\emng
\eentry

\bentry
\word{gabbroid}
\pron{gAYXbArxyfDx}
\gl{\gu}
\expl{}
(\BUvi) \bmng
 gAYxbArxyfDx; gAYxbArxBa; gArxYxborxV shileyanunx hoVluva. 
\emng
\eentry

\bentry
\word{gabby}
\pron{gAYXbi}
\gl{\gu}
\expl{(\tara\ \eng{gabbier}, \tama\ \eng{gabbiest}).}
\bmng
(\AmA) vAcALiyAda; mAtALiyAda; haraTemalalxnAda. 
\emng
\eentry

\bentry
\word{gabelle}
\pron{gabelf}
\gl{\nA}
\expl{}
\bmng
(\ca) terige; kara (\sA\ videVshiV terige, \kanmu\ phArxnisxna kArxMtipUvaR kAlada upipxna terige). 
\emng
\eentry

\bentry
\word{gaberdine}
\pron{gAYxbaDiRVnf}
\gl{\nA}
\bmng
\bnum
\num{1} (\ca) kapani; meVludagale (\kanmu\ yehUdayxrU BikuSxkarU hAkikoLuLxva, saDilavAgiyU udadxvAgiyU iruva, oraTu baTeTxya meVluDupu). 
\num{2} \eng{gabardine} padada rUpAMtara. 
\num{3} gaTiTx huriya baTeTx. 
\num{4} uDupu; hodike; kavaca; (meYge) rakaSxNe. 
\enum
\emng
\eentry

\bentry
\word{gabfest}
\pron{gAYXbfphesfTx}
\gl{\nA}
\expl{}
\bmng
(\ame) (\ashi) 
\bnum
\num{1} haraTe guMpu; mAtigAgi seVrida kUTa. 
\num{2} diVGaR saMBASaNe; bahaLa kAla eLeda saMBASaNe, mAtukate, saMBASaNAkUTa. 
\enum
\emng
\eentry

\bentry
\word{gabion}
\pron{geVbianf}
\gl{\nA}
\bmng
 hodarukaDiDx, betatx, \mo vugaLiMda yA kabibxNada paTiTxgaLiMda geraseyaMte heNedu, maNuNXtuMbi, hiMde koVTe \mo vugaLanunx kaTuTxvalilx yA Iga kAmATi kelasagaLalilx baLasuva toMbe. 
\emng
\eentry

\bentry
\word{gabionade}
\pron{geVbianeVDf}
\gl{\nA}
\bmng
 (maNuNx tuMbida) toMbe(gaLa) sAlu. 
\emng
\eentry

\bentry
\word{gable}
\pron{geVbflf}
\gl{\nA}
\bmng
\bnum
\num{1} caMdAya; geVbalu; ipApxru cAvaNiya koneyalilxruva tirxkoVnAkArada BAga. 
\num{2} (kiTaki yA bAgila meVlina) geVbalf (AkArada) -- capapxra, cejajx. 
\num{3} = \hyperlink{gable-end}{gable-end}. 
\enum
\emng
\eentry

\bentry
\word{gabled}
\pron{geVbflfDx}
\gl{\gu}
\bmng
 geVbalulx -- hAkida, hodisida; geVbalilxniMda alaMkarisida. 
\emng
\eentry

\bentry
\word{gable-end}
\pron{geVbflfeMDf}
\gl{\nA}
\bmng
 geVbalulx goVDe; tudiyalilx geVbalilxruva goVDe. 
\emng
\eentry

\bentry
\word{gablet}
\pron{geVbilxTf}
\gl{\nA}
\bmng
 (\sA\ alaMkArakAkxgi mADida) cikakx geVbalulx . 
\emng
\eentry

\bentry
\word{gaby}
\pron{geVbi}
\gl{\nA}
\expl{}
\bmng
(\pArxparx\ yA \pArxM) tiLigeVDi; bepapx; heDaDx; gAMpa. 
\emng
\eentry

\bentry
\word[gad(1)]{gad}
\pron{gAYXDf}
\gl{\BAavayx}
\expl{}
\bmng
 (yA \eng{by gad !}) 
\emng
(AshacxyaR sUcisuvAga, aNe iTuTx heVLuvAga mAtarx baLasuva pada) deVvareV ! deVvarANe! \eentry

\bentry
\word[gad(2)]{gad}
\pron{gAYXDf}
\gl{\BAavayx}
\gl{\akirx}
\expl{(\BU\ matutx \BUkaq\ \eng{gadded}, \vakaq\ \eng{gadding}).}
\bmng
\bnum
\num{1} aledADu; aMDale; manemane tirugu; tirugADu; kelasavilalxde sutAtxDu; tiridADu. 
\num{2} (\sA\ \vakaq dalilx \parx) (sasayxgaLa \vi) (neVra hoVgade) atitxtatx haraDu; AkaDe IkaDe habubx. 
\enum
\emng
\eentry

\bentry
\word[gad(3)]{gad}
\pron{gAYXDf}
\gl{\BAavayx\ \nA}
\bmng
 (dana aTaTxlu baLasuva) cUpAda koVlu. 
\emng

\noindent
\gl{\pagu}
\bmng
\eng{(up) on the gad} sutAtxTadalilx; aledATadalilx; sututxtatx; aleyutatx. 
\emng
\eentry

\bentry
\word[gadabout(1)]{gadabout}
\pron{gAYXDabwTf}
\gl{nA}
\bmng
\bnum
\num{1} alemAri; aMDalega; hArugAli; aledADuvavanu, 
\num{2} alasa vilAsi; soVmAriyAgidudxkoMDu BoVgajiVvana naDesuvava. 
\enum
\emng
\eentry

\bentry
\word[gadabout(2)]{gadabout}
\pron{gAYXDabwTf}
\gl{\gu}
\expl{}
\bmng
 alemAriyAda; aMDaleyuva; sadA -- aledADutitxruva, tirugutitxruva. 
\emng
\eentry

\bentry
\word{Gadarene}
\pron{gAYXDariVnf}
\gl{\gu}
\bmng
\bnum
\num{1} (pArxciVna pAYXlaseTxYnina) gaDArA paTaTXNada. 
\num{2} sikAkxpaTeTx nugAgxTada; vipariVta janajaMguLi iruva. 
\enum
\emng
\eentry

\bentry
\word{gaddi}
\pron{gaDi}
\gl{\nA}
\bmng
\bnum
\num{1}(\ca)(BAratiVya rAjara) siMhAsana; gadudxge; gAdi. 
\num{2} doretana; rAjatavx; ALivxke. 
\enum
\emng
\eentry

\bentry
\word{gadfly}
\pron{gAYXDfphelxY}
\gl{\nA}
\expl{(\bava\ \eng{gadflies}).}
\bmng
\bnum
\num{1} toNaci; danagaLanunx kacucxva noNa. 
\num{2} piVDaka; ragaLemAri; kitApati; kATagAra; upadarxvakAri; reVgisuvavanu yA piVDisuvavanu. 
\num{3} uderxVka; atAyxveVsha; keraLike; hucucxhucucx; yAvudeV kAyaRkekx haThAtAtxgi keYhacucxva parxvaqtitx. 
\enum
\emng
\eentry

\bentry
\word{gadget}
\pron{gAYXjiTf}
\gl{\nA}
\bmng
(\AmA) 
\bnum
\num{1} yaMtorxVpakaraNa; yaMtarx salakaraNe; yaMtarx \mo vugaLalilx aLavaDisiruva oMdu saNaNx upakaraNa yA salakaraNe. 
\num{2} anuSaMga; anubaMdha; gwNavAda yA pArxsaMgikavAda vasutx yA sAdhana. 
\num{3} yaMtarxka; yaMtarxvisheVSa; \sA\ cikakx, naviVna, citarxvicitarx yAMtirxka vasutx. 
\num{4} (keVvala) thaLakupaLaku vasutx; nirupayukatxvAda, keVvala AlaMkArikavAda hagura piVThoVpakaraNa, oDave, \mo vu. 
\enum
\emng
\eentry

\bentry
\word{gadgeteer}
\pron{gAYXjiTiarf}
\gl{\nA}
\bmng
 yaMtarxkanimARpaka; cikakx, naviVna yaMtarxgaLanunx nimiRsuvava yA baLasuvava. 
\emng
\eentry

\bentry
\word{gadgetry}
\pron{gAYXjiTirx}
\gl{\nA}
\bmng
\bnum
\num{1} saNaNxpuTaTx yaMtarxgaLu; yaMtarxkagaLu. 
\num{2} saNaNxpuTaTx yaMtarxgaLa baLake; yaMtarxka baLake. 
\enum
\emng
\eentry

\bentry
\word{gadgety}
\pron{gAYXjiTi}
\gl{\gu}
\bmng
\bnum
\num{1}yaMtarxdaMtha; saNaNx yaMtarxda lakaSxNagaLuLaLx. 
\num{2} saNaNxpuTaTx yaMtarx (gaLanunx) joVDisida; yaMtarxkagaLanunx aLavaDisida. 
\enum
\emng
\eentry

\bentry
\word[gadoid(1)]{gadoid}
\pron{geVDAyfDx}
\gl{\gu}
\bmng
(miVnina \vi) `kADf' jAtige seVrida. 
\emng
\eentry

\bentry
\word[gadoid(2)]{gadoid}
\pron{geVDAyfDx}
\gl{\nA}
\bmng
 (miVnina \vi) `kADf' jAtige seVrida miVnu. 
\emng
\eentry

\bentry
\word{gadolinite}
\pron{gAYXDalineYTf}
\gl{\nA}
\bmng
 viraLaBasamxdhAtugaLa silikeVTugaLanunxLaLx kapupx Kanija. 
\emng
\eentry

\bentry
\word{gadolinium}
\pron{gAYXDaliniamf}
\gl{\nA}
\expl{(\ravi)}
\bmng
 paramANusaMKeyx \eng{64}, paramANu tUka \eng{157.25} uLaLx, oMdu tirxveVlenisxVya viraLa BasamxdhAtu, \saMkeV\ \eng{Gd.} 
\emng
\eentry

\bentry
\word{gadroon}
\pron{gaDUrxnf}
\gl{\nA}
\expl{(\sA\ \bava\ matutx visheVSaNavAgi parxyoVga) }
\bmng
alaMkArakAkxgi goVDe \mo vugaLa meVle sAlAgi koredu yA gAre \mo vugaLiMda mADida hora ubibxna tiruvugaLu. 
\emng
\eentry

\bentry
\word{gadwall}
\pron{gAYXDfvAlf}
\gl{\nA}
\bmng
 (amerika hAgU yUroVpina utatxra BAgada) sihiniVrina bAtukoVLi. 
\emng
\eentry

\bentry
\word{gadzooks}
\pron{gAYXDfsUZkfsx}
\gl{\BAavayx}(\pArxparx)
\bmng
 (parxmANa, samathaRne, \mo vanunx mADuvAga baLasuva pada) deVvarANe(yAgi)! 
\emng
\eentry

\bentry
\word{Gael}
\pron{geV(gAYX)lf}
\gl{\nA}
\bmng
geVlf: 
\banum
\alnum{a} sAkxTalxMDina (ailaRMDf matutx mAYXnf divxVpada) kelfTx kuladavanu. 
\alnum{b} (\pArxparx) ailaRMDina kelfTx kuladavanu. 
\eanum
\emng
\eentry

\bentry
\word{Gaeldom}
\pron{geV(gAYX)laDxmf}
\gl{\nA}
\bmng
\bnum
\num{1} geVlf saMsakxqti yA nAgarikate. 
\num{2} geVlf jana; geVlf parxpaMca; geVlf samudAya. 
\enum
\emng
\eentry

\bentry
\word[Gaelic(1)]{Gaelic}
\pron{geV(gAYX)likf}
\gl{\gu}
\bmng
geVlikf; sAkxTalxMDina, ailaRMDina, mAYXnf divxVpada kelfTx janara yA avara BASeya. 
\emng
\eentry

\bentry
\word[Gaelic(2)]{Gaelic}
\pron{geV(gAYX)likf}
\gl{\nA}
\bmng
 sAkxTalxMDina, ailaRMDina, mAYxnf divxVpada kelfTx janara BASe. 
\emng
\eentry

\bentry
\wordnospeech{Gaelic coffee}{Gaelic coffee}
\pron{geVlikf kAphi}
\gl{\nA}
\bmng
 kene matutx airiSf visikx seVrisida kAphi. 
\emng
\eentry

\bentry
\word{Gaeltacht}
\pron{geVlfTaKfTx}
\gl{\nA}
\bmng
 geVlf nADu; airiSf BASeyeV deVshaBASeyAgiruva. ailaRMDina oMdu parxdeVsha. 
\emng
\eentry

\bentry
\word[gaff(1)]{gaff}
\pron{gAYXphf}
\gl{\nA}
\bmng
\bnum
\num{1} kokekx ITi; miVnu hiDiyalu baLasuva, kokekx muLuLxLaLx ITi. 
\num{2} miVnukokekx; doDaDx miVnugaLanunx cucicx daDakekx taruva, kabibxNada kokekx hAkida daDi. \imglink{gaff-2figure}{\raisebox{-0.10cm}[0pt][0pt]{\pdfimage width 0.7cm height 0.4cm {G_Pictures/gaff-2.jpg}}} 
\num{3} gAyxphf; haDagina hiMtudiyiMda muMtudiyavarege cAcida, kUve hagagxda meVle nililxsirada cwkAkArada hAyipaTada meVlABxgavanunx bicicxDalu, cAcalu kaTuTxva daMDa, aDaDxgoVlu. 
\enum
\emng
\eentry

\bentry
\word[gaff(2)]{gaff}
\pron{gAYXphf}
\gl{\sakirx}
\bmng
kokekx ITiyiMda (miVnu) hiDi. 
\emng
\eentry

\bentry
\word[gaff(3)]{gaff}
\pron{gAYXphf}
\gl{\nA}
\bmng
Thakukx; moVsa; vaMcane. 
\emng

\noindent
\gl{\nuga}
\expl{}
\bmng
\hyperdef{G}{gaff(3) nuga(1)}{} 
\bnum
\numi{1}\eng{blow the gaff} (\ashi) 
\banum
\alnum{a} oLasaMcanunx horageDavu; pitUri bayalu mADu. 
\alnum{b} guTuTx raTuTx mADu; rahasayx biTuTxkoDu. 
\eanum
\numie
\num{2} \eng{stand the gaff} (\ame) (\ashi) kaSaTx sahisiko; toMdare, upaTaLa, TiVke, \mo vanunx -- taDeduko. 
\enum
\emng
\eentry

\bentry
\word[gaff(4)]{gaff}
\pron{gAYXphf}
\gl{\nA}
\bmng
(\birx) 
\bnum
\num{1} (\pArxparx) (\ashi)sAvaRjanika vinoVda sathxLa. 
\hypertarget{gaff(4)2}{} 
\num{2} keLadajeRya agagxda nATaka shAle yA saMgiVta maMdira. 
\enum
\emng

\noindent
\gl{\pagu}
\bmng
\eng{penny gaff} = \hyperlink{gaff(4)2}{$^4$gaff(2)}. 
\emng
\eentry

\bentry
\word{gaffe}
\pron{gAYXphf}
\gl{\nA}
\bmng
\bnum
\num{1} tapupx; parxmAda; aviveVka; eDavaTuTx; tapupx hejejx. 
\num{2} aviveVkada kelasa yA viveVcaneyilalxda mAtu. 
\enum
\emng
\eentry

\bentry
\word{gaffer}
\pron{gAYXpharf}
\gl{nA}
\bmng
\bnum
\num{1} hiriya haLiLxga; haLiLxya hiriya. 
\num{2} muduka yA mudihaLiLxga. 
\num{3} (manuSayx, adhikAra, \mo vugaLa hesarugaLa hiMde mayARdeyAgi seVrisuva yA saMboVdhaneyalilx baLasuva padavAgi) yajamAna; gwDa. 
\num{4} (\AmA) (\birx) taMDada nAyaka; muKayx kelasagAra; oMdu guMpina yA taMDada muMdALu, muKaMDa, nAyaka. 
\num{5} (\AmA) gAYXpharf; sinimA yA TeliviSaznf nimARNa taMDadalilxna parxdhAna viduyxtf kelasagAra. 
\enum
\emng
\eentry

\bentry
\word[gag(1)]{gag}
\pron{gAYXgf}
\gl{\nA}
\bmng
\bnum
\num{1} bAyigiDugu; bAyigiDi; mAtanADadaMte yA shasatxrXcikitesxyalilx cikitesxya swlaBayxkekx bAyanunx terediDalu bAyige turukuva padAthaR. 
\num{2} (shAsana saBe) mAtu taDeta; vAkfsatxMBana; caceR mugita; caceR sAkumADabeVkeMba taDe, aDiDx. 
\num{3} mukatx BASaNakekx taDemADuva vasutx yA parisiThxti. 
\num{4} naTana parxkiSxpatxgaLu; nATaka saMBASaNeyalilx naTanu seVrisuva mAtugaLu. 
\num{5} (nATakashAle) (\kanmu\ nATaka, saMgiVta shAle, manaraMjane), \mo vugaLa naDuve modaleV sidadhxpaDisida oDiDxda (seVrisida) nakali dqshayx; hAsayx parxsaMga. 
\num{6} tamASe; geVli. 
\num{7} tamASeya kAyaR yA saMdaBaR. 
\num{8} suLuLx; moVsa; vaMcane. 
\enum
\emng
\eentry

\bentry
\word[gag(2)]{gag}
\pron{gAYXgf}
\gl{\kirx}
\bmng
(\vakaq\ \eng{gagging} \BU\ matutx \BUkaq\ \eng{gagged}.)
\emng

\noindent
\gl{\sakirx}
\bmng
\bnum
\num{1} bAyi giDi; bAyikaTuTx; mAtADadaMte yA bAyi terediDuvaMte yAvudeV padAthaRvanunx bAyige turuku. 
\num{2} bAyi mucicxsu; mAtanADadaMte mADu; vAkf sAvxtaMtarxyxkekx taDeyoDuDx. 
\num{3} (naTana \vi, nATaka saMBASaNeyalilx) madheyx mAtu seVrisu. 
\num{4} kudurege giDugu kaDivANa hAku. 
\num{5} vaMcisu; moVsamADu. 
\num{6} gaMTalu kaTiTxsu. 
\num{7} vAMti mADisu. 
\enum
\emng

\noindent
\gl{\akirx}
\bmng
\bnum
\num{1} (nATaka \mo\ manaraMjaneya naDuve) nakali seVrisu; hAsayx seVrisu; hAsayx parxsaMga joVDisu; vinoVdavanunx giDugu. 
\num{2} vaMcaneyanunx aBAyxsamADu; moVsa mADuvudanunx kalituko. 
\num{3} gaMTalu kaTuTx yA kaTiTxko. 
\num{4} vAMti mADu. 
\enum
\emng
\eentry

\bentry
\word{gaga}
\pron{gAgA}
\gl{\gu}
\bmng
\bnum
\num{1} (\ashi) modadx; maMka; pedadx; bepapx; bepupx moVreya. 
\num{2} araLu maraLina; maMkubudidhxya. 
\num{3} pedudx utAsxhada; hucucxhucAcxgi Asakitx toVrisuva: \eng{had been so gaga about him at first} moTaTxmodalu avana viSayavAgi atiyAda hucucx utAsxha toVrisidadx(Lu). 
\enum
\emng
\eentry

\bentry
\word{gag-bit}
\pron{gAYXgfbiTf}
\gl{\nA}
\bmng
 (kudureyanunx paLagisalu hAkuva balavAda) giDugu kaDivANa. 
\emng
\eentry

\bentry
\word[gage(1)]{gage}
\pron{geVjf}
\gl{\nA}
\bmng
\bnum
\num{1} otetx; Olu; aDavu; giravi; IDu; hoNe. 
\num{2} hoVrATakekx savAlu; kALagakekx AhAvxna. 
\num{3} (savAligAgi eseda keYgavusu \mo) paNa; peYje. 
\enum
\emng
\eentry

\bentry
\word[gage(2)]{gage}
\pron{geVjf}
\gl{\sakirx}
\bmng
(\pArxparx) giraviyiDu; otetx iDu; Olu koDu; IDumADu. 
\emng
\eentry

\bentry
\word[gage(3)]{gage}
\pron{geVjf}
\gl{\nA}
\bmng
(\ame) =  \hyperlink{gauge(1)}{$^1$gauge}. 
\emng
\eentry

\bentry
\word[gage(4)]{gage}
\pron{geVjf}
\gl{\sakirx}
\bmng
(\ame) =  \hyperlink{gauge(1)}{$^1$gauge}. 
\emng
\eentry

\bentry
\word[gage(5)]{gage}
\pron{geVjf}
\gl{\nA}
\bmng
=  \hyperlink{greengage}{greengage}. 
\emng
\eentry

\bentry
\word{gageable}
\pron{geVjabalf}
\gl{\gu}
\bmng
 (\ame)= \hyperlink{gaugeable}{gaugeable}. 
\emng
\eentry

\bentry
\word{gager}
\pron{geVjarf}
\gl{\nA}
\expl{}
\bmng
 (\ame) = \hyperlink{gauger}{gauger}. 
\emng
\eentry

\bentry
\word[gaggle(1)]{gaggle}
\pron{gAYXgflf}
\gl{\nA}
\bmng
\bnum
\num{1} (bAtina) hiMDu; guMpu. 
\num{2} (\AmA) janara guMpu; janagaLa samUha. 
\enum
\emng
\eentry

\bentry
\word[gaggle(2)]{gaggle}
\pron{gAYXgflf}
\gl{\akirx}
\bmng
 (bAtina \vi) kalxkikxsu; kalxkf kalxkf anunx. 
\emng
\eentry

\bentry
\wordnospeech{gag man}{gag man}
\pron{?}
\gl{\nA}
\bmng
 giDugu vinoVdakataR; nATaka \mo vugaLige tamASe parxsaMgagaLanunx bareyuvava. 
\emng
\eentry

\bentry
\word{gag-rein}
\pron{gAYXgfreVnf}
\gl{\nA}
\bmng
 (kaDivANavanunx matatxSuTx balavAgi mADuva) bigi lagAmu. 
\emng
\eentry

\bentry
\word{gagster}
\pron{gAYXgfsaTxrf}
\gl{\nA}
\bmng
\bnum
\num{1} =  \hyperlink{gag man}{gag man}. 
\num{2} nakalishAyxma; joVkaru; hAsayxgAra; joVkugaLanunx saqSiTxsuvava. 
\enum
\emng
\eentry

\bentry
\word{gaiety}
\pron{geVa(i)Ti}
\gl{\nA}
\bmng
\bnum
\num{1} ulAlxsa; haSaR; gelavu; nalivu; nalidATa; KuSi. 
\num{2} (\sA\ \bava\ dalilx) vinoVda keVLi; kiVrxDe; vilAsa. 
\num{3} beDagu; thaLaku. 
\num{4} sogasugArike; shisutx(gArike); SoVki; ThiVkAda uDupu. 
\enum
\emng

\noindent
\gl{\nuga}
\bmng
\eng{gaiety of nations} janasoVtxmada ha{Sho}RVlAlxsa; apAra janateya nalivu: \eng{that stroke of death, which has eclipsed the gaiety of nations} sAvina A parxhAra, janasotxVmada haSoRVlAlxsavanunx mare mADitu. 
\emng
\eentry

\bentry
\word{gaillardia}
\pron{geVlADiRa, geVlADiRya}
\gl{\nA}
\bmng
 geVlADiRya kulada, AlaMkArika puSapx biDuva oMdu sasayx. 
\emng
\eentry

\bentry
\word{gaily}
\pron{geVli}
\gl{\kirxvi}
\bmng
\bnum
\num{1} KuSiyAgi; ulAlxsadiMda; gelaviniMda. 
\num{2} nirAtaMkavAgi; nishicxMteyiMda. 
\num{3} hagura manasisxniMda. 
\num{4} vinoVdavAgi; liVleyaMte; kirxVDeyaMte. 
\numi{5} (uDupu \mo vugaLa \vi) 
\banum
\alnum{a} beDaginiMda; ADaMbaradiMda; SoVkiyAgi; sogasugArikeyiMda. 
\alnum{b} hoLeyutatx; thaLathaLisutatx; ujavxla vaNaRdiMda kUDi. 
\eanum
\numie
\num{6} tuMba; sAkaSuTx hecicxna parxmANadalilx. 
\enum
\emng
\eentry

\bentry
\word[gain(1)]{gain}
\pron{geVnf}
\gl{\nA}
\bmng
\bnum
\num{1} (Asitx \mo vugaLa) aBivaqdidhx; hecacxLa. 
\num{2} lABa; saMpAdane; paDapu; phAyide; naPe. 
\num{3} muMduvareta; muMduvareyuvudu. 
\num{4} utatxmike; utatxma sithxtige baruvudu. 
\num{5} aishavxyaR saMpAdane; sirigaLike; haNa \mo vugaLa gaLike. 
\num{6} (\bava dalilx) saMpAdane; gaLike; saMbaLa sArige; varamAna; AdAya; vAyxpAra \mo vugaLiMda saMpAdisida, jUjinalilx gedadx -- haNa, duDuDx. 
\num{7} (motatxdalilx) vaqdidhx; hecacxLa; Erike 
\numi{8} (ilekATxrXnikfsx) 
\banum
\alnum{a} lABa; hecacxLa; Adhikayx; adhikAMsha; viduyxcaCxkitx \mo vugaLa hecacxLada parxmANa (\sA\ lAgaridamfnalilx vayxkatxpaDisuvudu). 
\alnum{b} idara lAgaridamf. 
\eanum
\numie
\enum
\emng
\eentry

\bentry
\word[gain(2)]{gain}
\pron{geVnf}
\gl{\sakirx}
\bmng
\bnum
\num{1} (apeVkiSxsida yA apeVkaSxNiVya vasutxvanunx) paDe; hoMdu; gaLisu; saMpAdisu; \eng{gain advantage, recognition, one's ends} lABa, mananxNe, tananx guri -- paDe. 
\num{2} lABagaLisu; naPe hoMdu; parxyoVjana paDe. 
\num{3} (yAvudaralAlxdarU) anukUla hoMdu; aBivaqdidhx paDe. 
\num{4} (tAratamayxdiMda, inonxbabxnigiMta) meVlenisiko; meVlAgu; hecacxLike paDe. 
\num{5} Adhikayx hoMdu; adhikavAgu; hecicxnadAgi, seVrikeyAgi -- paDe, hoMdu: \eng{gain momentum, weight} AveVga, tUka -- paDe. 
\num{6} (samudarx, shaturx, \mo vugaLu Akarxmisida jamiVnu, parxdeVsha, \mo vanunx) biDisiko; hiMdakekx paDe. 
\num{7} (hoVrATa, sapxdheR, vivAda, \mo vugaLalilx) jayagaLisu; gelulx. 
\num{8} olisiko; oDaMbaDisu; tananx kaDege, aBipArxyakekx baruvaMte mADiko; opupxvaMte mADu. 
\num{9} (udedxVshisida sathxLavanunx) seVru; muTuTx; talapu. 
\num{10} (samudarxda \vi) nelavanunx Akarxmisu. 
\num{11} (gaDiyArada \vi) (nidiRSaTx kAla) muMdAgu; (nijavAda samayakikxMta) muMde ODu. 
\enum
\emng

\noindent
\gl{\akirx}
\bmng
\bnum
\num{1} lABa -- gaLisu, paDe. 
\num{2} parxyoVjana -- hoMdu, paDe. 
\num{3} (yAvudaralelxV) sudhArisu; muMde baru; muMduvari: \eng{to gain in health after an illness} roVga vAsiyAda meVle AroVgayx sudhArisu. 
\num{4} (gaDiyArada \vi) muMde hoVgu; nijavAda samayakikxMta hecicxna samaya toVrisu: \eng{gain by an hour a day} dinakokxMdu GaMTe muMde hoVgutatxde. 
\num{5}(hoVlikeyiMda yA vayxtAyxsadiMda) meVlAgu; utatxmavAgu; atishayisu. 
\enum
\emng

\noindent
\gl{\nuga}
\bmng
\bnum
\num{1} \eng{gain face} tananx vacaRsusx, kiVtiR, adhikAra, aMtasutx, \mo vanunx -- hecicxsiko, sAthxpisiko: \eng{a petty official trying to gain face by treating his subordinates arrogantly} tananx keYkeLaginavaranunx sokikxniMda kANutAtx tananx aMtasatxnunx hecicxsikoLaLxlu yatinxsutitxruva keLagaNa adhikAri. 
\num{2} \eng{gain} \hyperlink{ground(1) nuga(9)}{$^1$ground}. 
\num{3} \eng{gain ground (up)on}(tAnu benanxTiTxruva vayxkitx yA vasutxvanunx) samiVpisu. 
\num{4} \eng{gain the ear of } sumuKavAgi keVLuvaMte (obabxnanunx) olisiko; heVLidudakekx kivigoDuvaMte mADiko. 
\num{5} \eng{gain the upper hand} meVlugeY paDe, sAdhisu. 
\num{6} \eng{gain time} (nepa heVLutAtx yA nidhAnoVpAyadiMda) kAlataLuLx; kAlakaLe; kAlaharaNa mADu. 
\num{7} \eng{gain upon} obabxna olavu, anugarxha, daye, kaqpe -- saMpAdisu. 
\enum
\emng
\eentry 

\bentry
\word{gainable}
\pron{geVnabflf}
\gl{\gu}
\bmng
\bnum
\num{1} saMpAdisabalalx; paDeyabalalx; gaLisabalalx; sAdhisabalalx: \eng{greatness in art is not a teachable nor gainable thing} kalApwrxDhimeyu kalisabalalx ilalxve gaLisabalalx viSayavalalx. 
\num{2} olisikoLaLxbalalx. 
\num{3} (\pArxparx) (jamiVnina \vi) kaqSi sAdhayx; kaqSi mADalu -- sAdhayxvAda, sulaBavAda. 
\enum
\emng
\eentry

\bentry
\word{gainer}
\pron{geVnarf}
\gl{\nA}
\bmng
lABadAra; lABa paDeyuvavanu; naPedAra; parxyoVjana hoMduvavanu; olavu, anugarxha, \mo vanunx hoMduvavanu. 
\emng
\eentry

\bentry
\word{gainful}
\pron{geVnfphulf}
\gl{\nA}
\bmng
\bnum
\num{1} lABadAyaka; AdAyakara. 
\num{2} lABadolavina; lABadaqSiTxya; parxtiPalAkAMkeSxya; parxyoVjana daqSiTxya; lABa yA parxyoVjanavanunx gaLisuvudaralilx kaTATxseyiruva. 
\num{3} (kelasa, udoyxVgada \vi) veVtanasahitavAda; saMbaLa yA parxtiPaladiMda kUDida. 
\num{4} (\viparx) lABAkAMkeSxya. 
\enum
\emng
\eentry


\bentry
\word{gainfully}
\pron{geVnfphuli}
\gl{\kirxvi}
\bmng
\bnum
\num{1} lABadAyakavAgi. 
\num{2} parxyoVjanakaravAgi. 
\num{3} parxtiPaladiMda kUDi. 
\enum
\emng
\eentry

\bentry
\word{gainfulness}
\pron{geVnfphulfnisf}
\gl{\nA}
\bmng
\bnum
\num{1} lABadAyakate. 
\num{2} parxyoVjanakaravAgiruvike. 
\num{3} parxtiPaladiMda kUDiruvike. 
\enum
\emng
\eentry

\bentry
\word{gainings}
\pron{geVniMgfs'}
\gl{\nA}
\expl{}
\bmng
 (\bava) gaLike; saMpAdane. 
\emng
\eentry

\bentry
\word{gainsay}
\pron{geVnfseV}
\gl{\sakirx}
\expl{(\BU\ matutx \BUkaq\ \eng{gainsaid} \ucAcx\ geVnfseDf.)}
\bmng
(\pArxparx\ yA sAhitayxka). 
\bnum
\num{1} alalxvenunx; ilalxvenunx; nirAkarisu. 
\num{2} virudadhxvAgi heVLu, parxtipAdisu; parxtiheVLu; edurisu. 
\enum
\emng
\eentry

\bentry
\word{gainsayer}
\pron{geVnfseVarf}
\gl{\nA}
\bmng
\bnum
\num{1} viroVdhi; parxtivAdi; virudadhxvAgi heVLuvavanu. 
\num{2} alalxvenunxvava; alalxgaLeka; nirAkataR; ilalxveMdu heVLuvavanu; nirAkarisuvavanu. 
\enum
\emng
\eentry

\bentry
\word{gainst}
\pron{geVnfsfTx}
\gl{\upa}
\expl{}
\bmng
(\kAparx) \eng{against} (eMbudara \saMkiSx). 
\emng
\eentry


\bentry
\word{gait}
\pron{geVTf}
\gl{\nA}
\expl{}
\bmng
\bnum
\num{1} naDage; naDe; gati; gamana; naDeyuva BaMgi, riVti. 
\num{2} munanxDage; ODuvava, kudure, vAhana, \mo vugaLu muMdakekx calisuva riVti. 
\enum
\emng

\noindent
\gl{\nuga}
\bmng
 \eng{go one's} (\engit{or} \eng{one's own)} \eng{gait} taninxSaTx baMda dAri hiDi; tananx dAri tAnu hiDi; tanage toVcidaMte, anisidaMte -- naDeduko, vatiRsu. 
\emng
\eentry

\bentry
\word{gaiter}
\pron{geVTarf}
\gl{\nA}
\expl{}
\bmng
 kaNakAla kApu; kaNakAlige, kAlina haraDige, yaMtarxda BAgakekx sututxva baTeTxya yA togalina paTiTx. 
\emng
\eentry

\bentry
\word{gaitered}
\pron{geVTarfDx}
\gl{\gu}
\expl{}
\bmng
 kaNakAla kApu kaTiTxda yA toTaTx. 
\emng
\eentry

\bentry
\word{Gal.}
\pron{gAYXlf}
\gl{\saMkiSx}
\expl{}
\bmng
 \eng{Galatians} (beYbalina hosa oDaMbaDike). 
\emng
\eentry

\bentry
\word[gal(1)]{gal}
\pron{gAYXlf}
\gl{\nA}
\expl{}
\bmng
(\ashi) huDugi. 
\emng
\eentry

\bentry
\word[gal(2)]{gal}
\pron{gAYXlf}
\gl{\nA}
\bmng
(\Bwvi) gurutavxda PalavAgi uMTAguva veVgoVtakxSaRda mAna, sekeMDfge oMdu seMTimiVTaru. 
\emng
\eentry

\bentry
\word{gal.}
\pron{gAYXlf}
\gl{\saMkiSx}
\expl{}
\bmng
\eng{gallon(s)} 
\emng
\eentry

\bentry
\word{gala}
\pron{gA(geV)la}
\gl{\nA}
\expl{}
\bmng
\bnum
\num{1} (\sA\ visheVSaNavAgi parxyoVga) utasxva; hababx; samAraMBa; veYBava: \eng{gala day} utasxva dina. \eng{gala dress} hababxduDige; sogasina uDupu. 
\num{2} (\birx) kirxVDA samAraMBa; kirxVDoVtasxva; AToVTagaLigAgi, \kanmu\ IjATakAkxgi, seVrida saMtoVSa kUTa. 
\enum
\emng
\eentry

\bentry
\word[galactagogue(1)]{galactagogue}
\pron{galAYXkaTXgAgf}
\gl{\gu}
\expl{}
\bmng
 kiSxVracoVdaka; hAlina harivanunx parxcoVdisuva. 
\emng
\eentry

\bentry
\word[galactagogue(2)]{galactagogue}
\pron{galAYXkaTXgAgf}
\gl{\nA}
\expl{}
\bmng
 kiSxVracoVdaka; hAlina harivanunx parxcoVdisuva vasutx. 
\emng
\eentry

\bentry
\word{galactic}
\pron{galAYXkiTXkf}
\gl{\gu}
\expl{(\Kavi)}
\bmng
\bnum
\num{1} kiSxVrapathada; AkAshagaMgeya. 
\num{2} gAYxlakisxya; gAYXlakisxgaLa yA avugaLige saMbaMdhisida. 
\enum
\emng
\eentry


\bentry
\word{galacto-}
\pron{galAYXkaTX-}
\gl{\sapUpa}
\expl{}
\bmng
(\veYshA \mo vu) keSxYra, hAlu, kiSxVra eMbathaRdalilx baLasuva \sapUpa. 
\emng
\eentry

\bentry
\word{galago}
\pron{galeVgoV}
\gl{\nA}
\expl{(\bava\ \eng{galagos}). }
\bmng
galeVgoV; galeVgoV kulada, doDaDx kaNuNxgaLu matutx kivigaLu hAgU udadx bAlavU iruva, mara hatutxva, \da\ APiZrxkada kADupApa. 
\emng
\eentry

\bentry
\word{galah}
\pron{galA}
\gl{\nA}
\expl{}
\bmng
\bnum
\num{1} (\AseTxrXV) gulAbi baNaNxda edeya, bUdubeninxna, juTuTxgiLi. 
\num{2} (\ashi) daDaDx; gugugx; pedadx. 
\enum
\emng
\eentry

\bentry
\word{Galahad}
\pron{gAYXlahAYXDf}
\gl{\nA}
\expl{}
\bmng
 sajajxna; saBayx; udAtatx guNa, pArxmANikate, saBayxte, \mo\ sherxVSaThx guNagaLuLaLx vayxkitx. 
\emng
\eentry

\bentry
\word{galantine}
\pron{gAYxlaMTiVnf}
\gl{\nA}
\bmng
gAYxlaMTiVnf; mULegaLanunx tegedu, masAle tuMbisi karidu, taNaNxge (mADi) baDisuva biLimAMsa. 
\emng
\eentry

\bentry
\wordRemoveSpace{galanty-show}{galanty show}
\pron{galAYXMTi SoV}
\gl{\nA}
\expl{ }
\bmng
 (\ca) neraLu boMbeyATa; paradeya meVle boMbegaLa neraLu biVLisi toVrisuva mUkABinayada ATa. 
\emng
\eentry

\bentry
\word{galaxy}
\pron{gAYXlakisx}
\gl{\nA}
\expl{ }
\bmng
\bnum
\numi{1} \eng{(Galaxy)} (\Kavi) 
\banum
\alnum{a} gAYXlakisx; AkAshagaMge gAYXlakisx; rAtirxya veVLe AkAshadalilx hAlu celilxruvaMte kANuva AkAshagaMgeyalilxruva koVTayxMtara nakaSxtarxgaLanUnx AkAshadalilx kANisuva itara nakaSxtarxgaLanUnx namamx swravUyxhavanUnx oLagoMDa oTiTxlu. 
\alnum{b} gAYXlakisx; namamx AkAshagaMge gAYXlakisxya horagaDe dUradalilx(ruva), dUradashaRkadalilx aneVkaveVLe niVhArikegaLaMte kANuva AkAshagaMge gAYXlakisxyaMthadeV Ada itara tArA samUhagaLu. 
\eanum
\numie
\num{2} tArAgaNa; teVjoVgaNa; suMdariyaru, parxtiBAvaMtaru, \mo vara ujavxla guMpu. 
\enum
\emng
\eentry

\bentry
\word{galbanum}
\pron{gAYXlabxnamf}
\gl{\nA}
\expl{ }
\bmng
 (paSiRyA deVshada) oMdu jAtiya marada hAlu gugugxla. 
\emng
\eentry

\bentry
\word[gale(1)]{gale}
\pron{geVlf}
\gl{\nA}
\bmng
= \hyperref{kandict_b.pdf}{B}{bog myrtle}{bog myrtle}. 
\emng
\eentry

\bentry
\word[gale(2)]{gale}
\pron{geVlf}
\gl{\nA}
\bmng
\bnum
\num{1} birugALi; caMDamAruta; kaDugALi; \kanmu\ boVphoVTfR mAnadalilx GaMTege \eng{32}riMda \eng{54} meYli veVgadalilx biVsuva gALi. 
\num{2} (\nw) birugALi; caMDamAruta. 
\num{3} (\kAparx) nasugALi; maMdamAruta. 
\num{4} (\kanmu) naguvina soPxVTa; BAri nagu; keVkeya nagu. 
\enum
\emng
\eentry

\bentry
\word[gale(3)]{gale}
\pron{geVlf}
\gl{\nA}
\bmng
 (\birx) (kAlakAlakekx koDabeVkAda) bADige. 
\emng

\noindent
\gl{\pagu}
\bmng
\eng{hanging gale} bADigeya bAki. 
\emng
\eentry

\bentry
\word{galea}
\pron{geVlia}
\gl{\nA}
\expl{(\bava\ \eng{galeae} \ucAcx\ geVliI, yA \eng{galeas}).}
\bmng
(\savi matutx \pArxvi)(AkAradalilx, kelasadalilx, yA sAthxnadalilx) shirasAtxrXNavanunx hoVluva aMgaracane. 
\emng
\eentry

\bentry
\word{galeate}
\pron{geVliETf}
\gl{\gu}
\expl{ }
\bmng
 (\savi matutx \pArxvi) shirasAtxrXNita; shirasAtxrXNadaMtha yA shirasAtxrXNadaMtha AkArada aMgavuLaLx. 
\emng
\eentry

\bentry
\word{galeated}
\pron{geVliETiDf}
\gl{\gu}
\expl{ }
\bmng
  = \hyperlink{galeate}{galeate}. 
\emng
\eentry

\bentry
\word{galeeny}
\pron{geVliVni}
\gl{\nA}
\expl{ }
\bmng
 (\birx) gini koVLi; oMdu bageya niVru koVLi. 
\emng
\eentry

\bentry
\word{Galen}
\pron{geVlanf}
\gl{\nA}
\expl{ }
\bmng
\bnum
\num{1} geVlanf; \kirxsha\ eraDane shatamAnada girxVkf veYdayx matutx veYdayxshAsatxrXjacnx. 
\num{2} (\hA) (yAvaneV) veYdayx. 
\enum
\emng
\eentry

\bentry
\word{galena}
\pron{galiVna}
\gl{\nA}
\expl{ }
\bmng
 nisagaRdalilx doreyuva siVsada salePxYDf saMyoVjaneyuLaLx siVsada adiru. 
\emng
\eentry

\bentry
\word[galenic(1)]{galenic}
\pron{galenikf}
\gl{\gu}
\bmng
\bnum
\num{1} (girxVkf veYdayxshAsatxrXjacnx) geVlanana; geVlanana vidhAnada. 
\num{2} mUlikegaLiMda, sasayxgaLiMda tayArisida. 
\enum
\emng
\eentry

\bentry
\word[galenic(2)]{galenic}
\pron{galenikf}
\gl{\nA}
\bmng
 hasiru auSadhi; rAsAyanikavAgi tayArisade, mUlikegaLiMda, sasayxgaLiMda tayArisida auSadhi. 
\emng
\eentry

\bentry
\word[galenical(1)]{galenical}
\pron{galenikalf}
\gl{\gu}
\expl{}
\bmng
=  \hyperlink{galenic(1)}{$^1$galenic}. 
\emng
\eentry

\bentry
\word[galenical(2)]{galenical}
\pron{galenikalf}
\gl{\nA}
\expl{}
\bmng
=  \hyperlink{galenic(2)}{$^2$galenic}. 
\emng
\eentry

\bentry
\wordf{galere}
\pron{galeVrf}
\gl{\nA}
\expl{\F}
\bmng
kUTa; paTAlaM; (\sA\ aniSaTx vayxkitxgaLa) guMpu. 
\emng
\eentry

\bentry
\word[Galilean(1)]{Galilean}
\pron{gAYxliliVanf}
\gl{\gu}
\expl{}
\bmng
 geliliVya; geliliyanf; KagoVLashAsatxrXjacnx geliliyoVvina yA avanige saMbaMdhisida. 
\emng
\eentry

\bentry
\word[Galilean(2)]{Galilean}
\pron{gAYxliliVanf}
\gl{\gu}
\bmng
\bnum
\num{1} pAYxleseTxYnina gelili parxdeVshada. 
\num{2} kerxYsatx dhamaRda. 
\enum
\emng
\eentry

\bentry
\word[Galilean(3)]{Galilean}
\pron{gAYxliliVanf}
\gl{\nA}
\bmng
\bnum
\num{1} gelili nivAsi; pAYxleseTxYnina gelili parxdeVshadalilx (huTiTxdava(Lu)) yA vAsisuvava(Lu). 
\num{2} kerxYsatx(dhamaRdava). 
\num{3} (\hiV) Esukirxsatx. 
\enum
\emng
\eentry

\bentry
\wordnospeech{Galilean satellite}{Galilean satellite}
\pron{?}
\gl{\nA}
\bmng
 geliliyanf upagarxha; geliliyo modala bArige dUradashaRkadalilx kaMDa, gurugarxhada ati doDaDx nAlukx upagarxhagaLalolxMdu. 
\emng
\eentry

\bentry
\wordnospeech{Galilean telescope}{Galilean telescope}
\pron{?}
\gl{\nA}
\bmng
 geliliyo dUradashaRka; geliliyo motatxmodalu nimiRsida divxpiVna vasutxka (\eng{biconvex objective}) matutx divxnimanx neVtarxka (\eng{biconcave eyepiece})gaLiruva dUradashaRka. 
\emng
\eentry

\bentry
\word{galilee}
\pron{gAYxliliV}
\gl{\nA}
\bmng
caciRna muKamaMTapa; caciRna mahAdAvxrada muMBAgadalilxruva muKamaMTapa yA ArAdhanA maMdira. 
\emng
\eentry

\bentry
\word{galimatias}
\pron{gAYxlimAYxTiasf, gAYxlimeVSasf}
\gl{\nA}
\bmng
athaRvilalxda mAtu; asaMbadadhx parxlApa. 
\emng
\eentry

\bentry
\word{galingale}
\pron{gAYxliMgeVlf}
\gl{\nA}
\bmng
\bnum
\num{1} duMparAsemxV; sugaMdhavAci giDa; kacUcxra; aDigegU, veYdayxdalUlx baLasuva, Alivxniya kulada, maleVSiyA, iMDoneVSiyagaLa oMdu sugaMdha beVru. 
\hypertarget{galingale(2)}{} 
\num{2} muMja; oMdu jAtiya nodehululx. 
\enum
\emng

\noindent
\gl{\pagu}
\bmng
 \eng{English galingale} = \hyperlink{galingale(2)}{galingale (2)}. 
\emng
\eentry

\bentry
\word{galiot}
\pron{gAYxliaTf}
\gl{\nA}
\bmng
 \eng{galliot} padada rUpAMtara. 
\emng
\eentry

\bentry
\word{galipot}
\pron{gAYxlipATf}
\gl{\nA}
\bmng
 gaDeDxkaTiTxda oMdu bageya TapaRMTeYnf yA kapURra teYla. 
\emng
\eentry

\bentry
\word[gall(1)]{gall}
\pron{gAlf}
\gl{\nA}
\bmng
\bnum
\num{1} (pArxNigaLa, \kanmu\ etitxna) pitatx; pitatxrasa. 
\num{2} (rUpa). kaDukahi(yAda) vasutx. 
\num{3} (rUpa) kaDukahi; kaDukaTutavx. 
\num{4} pitatxkoVsha matutx adara oLagiruvaMthadu. 
\num{5} kaTuBAvane; pAruSayx; kAThinayx; niSuThxrate; devxVSa; veYra. 
\num{6} (\ashi) dhASaTxyXR; sokukx; durahaMkAra. 
\enum
\emng

\noindent
\gl{\nuga}
\bmng
\bnum
\num{1} \eng{dip one's pen in gall} viSadalilx adidx bare; kaTuvAgi bare; AkorxVshadiMda bare. 
\numi{2} \eng{gall and wormwood} 
\banum
\alnum{a} kaDukahi (vasutx). 
\alnum{b} obabxna savxBAvakekx, deVhakekx seVrada, olalxda -- vasutx, vicAra; manasisxna kahi; asamAdhAna. 
\eanum
\numie
\enum
\emng
\eentry

\bentry
\word[gall(2)]{gall}
\pron{gAlf}
\gl{\nA}
\bmng
\bnum
\num{1} (\kanmu\ kudureya) kuru; bobebx; hopapxLe; kiVvu tuMbida, noVvu koDuva -- guLeLx, bAvu. 
\num{2} ujujx -- huNuNx, gAya. 
\num{3} beVgudi; manoVveVdane; manasisxna noVvu. 
\num{4} (EnU ilalxdaMte) ujijx hAkida sathxLa. 
\num{5} doVSa; Una; kuMdu; nUyxnate. 
\num{6} (kurucalu kADinalilxya yA holadalilxya) baTaTxbayalu; boVLu parxdeVsha; KAli jAga. 
\enum
\emng
\eentry

\bentry
\word[gall(3)]{gall}
\pron{gAlf}
\gl{\sakirx}
\bmng
\bnum
\num{1} huNANxguvaMte ujujx; gAyavAguvaSuTx tikukx. 
\num{2} reVgisu;piVDisu; kATakoDu; kADu; goVLu huyudxko. 
\num{3} mAna kaLe; avamAnamADu; hiVnAyagoLisu. 
\enum
\emng

\noindent
\gl{\akirx}
\bmng
ujijx ujijx huNANxgu. 
\emng
\eentry

\bentry
\word[gall(4)]{gall}
\pron{gAlf}
\gl{\nA}
\bmng
 gAlf; (\kanmu\ Okf) maradalilx oMdu jAtiya kirxmiyiMdAda, masi tayArike, camaR hadamADuvudu, baNaNxkaTuTxvudu, auSadha tayArikegaLalilx baLasuva gaMTu. 
\emng
\eentry

\bentry
\word[gall(5)]{gall}
\pron{gAlf}
\gl{\gu}
\bmng
 (kirxmigaLa \vi) Okf maradalilx gAlfgaLanunx yA gaMTugaLanunx uMTumADuva. 
\emng
\eentry

\bentry
\wordnospeech{gall.}{gall.}
\pron{?}
\gl{\saMkiSx}
\bmng
 \eng{gallon(s).} 
\emng
\eentry

\bentry
\word[Galla(1)]{Galla}
\pron{gAYxla}
\gl{\gu}
\bmng
 (AphirxkAda) ithiyoVpiya matutx kiVnAyxgaLa alemAri janAMgadavara. 
\emng
\eentry

\bentry
\word[Galla(2)]{Galla}
\pron{gAYxla}
\gl{\nA}
\bmng
gAYxla: 
\banum
\alnum{a} (AphirxkAda) ithiyoVpiya hAgU kiVnAyxgaLa alemAri janAMgadavanu. 
\alnum{b} ivara BASe. 
\eanum
\emng
\eentry

\bentry
\word[gallant(1)]{gallant}
\pron{gAYxlaMTf}
\gl{\gu}
\bmng
\bnum
\num{1} (\pArxparx) beDaguDupina; ADaMbarada, Dwlina, ThiVku -- uDupina. 
\num{2} (haDagu, kudure, \mo vugaLa \vi) veYBavada; GanavAda; mahA; utakxqqSaTxvAda; gaMBiVravAda; ThiVviyiMda kUDida. 
\num{3} dhiVra; viVra; vikarxma; parxtApashAli. 
\num{4} viVradhamiR; dayAviVra; dubaRlara yA sitxrXVyara paravAgi anukaMpadiMda matutx shwyaRdiMda vatiRsuva. 
\num{5} pAliRmeMTinalilx seVneya yA nwkAbalada sadasayxna sAMparxdAyika visheVSaNa, birudu: \eng{the honourable and gallant member} gwravAnivxta hAgU dhiVra sadasayx. 
\num{6} sitxrXV Bakitxya; heMgasara bagegx visheVSa gwrava, olavu toVruva. 
\num{7} sitxrXV parAyaNa; rasika; parxNayashiVla; kAmAsakatx. 
\enum
\emng
\eentry

\bentry
\word[gallant(2)]{gallant}
\pron{gAYxlaMTf, galAYxMTf}
\gl{\nA}
\bmng
\bnum
\num{1} (\pArxparx) niVTugAra; shisutxgAra; sogasugAra; SoVkilAla. 
\num{2} sitxrXVmoVhi; sitxrXVparAyaNa; heMgasaralilx Asakatx. 
\num{3} sitxrXVBakatx; heMgasara bagegx visheVSa gwravAdara uLaLxvanu, toVruvavanu. 
\num{4} nalalx; kAdala. 
\num{5} upapati; viTa; miMDa. 
\num{6} sajajxna; oLeLxya vayxkitx. 
\enum
\emng
\eentry

\bentry
\word[gallant(3)]{gallant}
\pron{galAYxMTf}
\gl{\sakirx}
\bmng
\bnum
\num{1} (heMgasinoDane) parxNayavADu; sarasavADu; parxNayaceVSeTx, parxNayavilAsa -- naDesu. 
\num{2} (mahiLeya) rakaSxkanAgi, beMgAvalAgi -- joteyalilx hoVgu. 
\enum
\emng

\noindent
\gl{\akirx}
\bmng
 celAlxTavADu; heMgasaroDane sarasavADu: \eng{spent his evenings gallanting with ladies} heMgasaroDane celAlxTavADutAtx saMjegaLanunx kaLeda. 
\emng
\eentry

\bentry
\word{gallantly}
\pron{gAYxlaMTfli}
\gl{\kirxvi}
\bmng
\bnum
\num{1} ADaMbaradiMda; ThiVkAgi; beDaginiMda. 
\num{2} veYBavadiMda; GanavAgi; gaMBiVravAgi; ThiVviyiMda kUDi. 
\num{3} sitxrXVBakitxyiMda; sitxrXVgwravadiMda; mahiLA mayARdeyiMda. 
\num{4} rasikanaMte; parxNayashiVlanAgi; kAmAsakatxnaMte; sitxrXVparAyaNanAgi; mahiLAsakatxnaMte. 
\enum
\emng
\eentry

\bentry
\word{gallantry}
\pron{gAYxlaMTirx}
\gl{\nA}
\expl{(\bava\ \eng{gallantries}).}
\bmng
\bnum
\num{1} dheYyaR; shwyaR; kecucx; pwruSa; sAhasa; edegArike; CAti. 
\num{2} shiSATxcAra; saBayx naDavaLike. 
\num{3} sitxrXVBakitx; mahiLA mayARde; sitxrXVyaralilx visheVSa olavu, gwrava. 
\num{4} vinaya vataRne yA vinayavAda mAtu. 
\num{5} parxNaya vataRne. 
\num{6} parxNaya BASaNa; perxVmada mAtu. 
\num{7} kAmuka naDate; parxNaya saMdhAna; parxNaya saMbaMdha. 
\num{8} vayxBicAra. 
\enum
\emng
\eentry

\bentry
\word{gall-bladder}
\pron{gAlfbAlxYxDarf}
\gl{\nA}
\bmng
pitatxkoVsha; yakaqtitxnalilx utapxtitxyAguva pitatxrasavu saMgarxhavAguva koVsha. 
\emng
\eentry

\bentry
\word{galleon}
\pron{gAYxlianf}
\gl{\nA}
\bmng
\bnum
\num{1} (\ca) gAYxliyanf; haDagigiMta kaDime udadxviruva, Adare hecucx etatxraviruva haDagu. 
\num{2} (\sA\ sepxVnf deVshada) yudadhxda haDagu. 
\num{3} (amerikadoDane vAyxpAra naDesuva) sepxVnf deVshada doDaDx haDagu, vAyxpAranwke. 
\enum
\emng
\eentry

\bentry
\word{galleried}
\pron{gAYxlariVDf}
\gl{\gu}
\bmng
 cAvaNi hAdi racisiruva yA cAvaNi hAdiyiMda alaMkarisiruva; pakakxgaLalilx aretereya meVle cAvaNi hAki naDedADalu hAdi mADiruva. 
\emng
\eentry

\bentry
\word[gallery(1)]{gallery}
\pron{gAYxlari}
\gl{\nA}
\expl{(\bava\ \eng{galleries}).}
\bmng
\bnum
\num{1} Cananxpatha; cAvaNihAdi; pakakxgaLalilx are terediruva, meVle cAvaNiyuLaLx, naDedADalu mADida hAdi. 
\num{2} dAvxramaMTapa; muKamaMTapa. 
\num{3} kaMbahAdi; satxMBapatha; kaMbasAluLaLx hAdi. 
\num{4} (kaTaTxDagaLa oLaBAgakekx hoVgalu hecucx maMdada, dapapxda goVDegaLa naDuve mADida yA cAcu UregaLa AdhArada meVle kaTiTxda) udadxneya kiruhAdi yA ONi. 
\num{5} gAYxlari; upapxrigeya keYsAle. 
\num{6} cacuR, saBAmaMdira, \mo vugaLalilx oLagoVDeyiMda saBAMgaNada kaDege cAcikoMDu hecucx saBikarige sathxLa odagisuva yA gAyakaru, varadigAraru, AgaMtukaru, \mo varige mIsalAgiTiTxruva veVdike: \eng{minstrels' gallery} hADu kavigaLa gAYxlari. 
\numi{7} (nATakashAleyalilx) 
\banum
\alnum{a} gAYxlari; upapxrigeya meTiTxlu piVTha; soVpAna piVTha; atayxMta meVlABxgadalilxna keYsAle, veVdike. 
\alnum{b} gAYxlariyalilx kuLitavaru. 
\alnum{c} gAYxlari parxBugaLu; gAYxlari maMdi; gAyxlari jana; pAmara perxVkaSxkaru; asaMsakxqqta perxVkaSxka yA shorxVtaq vagaR; keLadajeRyavaru. 
\eanum
\numie
\num{8} kiriyagalada niDukoVNe. 
\num{9} oLahAdi; naDave; hajAra. 
\num{10} kalA -- citarxshAle, citarxmaMdira; kalAvasutxgaLa parxdashaRna maMdira. 
\num{11} (diVpada meVlina) cimaNiya hiDike. 
\num{12} (seYnayx, gaNigArike) (neVravAgi samataladalilx hoVguva) suraMgamAgaR. 
\num{13} (gAlfphx paMdayx \mo vugaLalilxna) perxVkaSxkaru; perxVkaSxkataMDa. 
\enum
\emng

\noindent
\gl{\nuga}
\bmng
\hyperdef{G}{gallery(1) nuga}{} \eng{play to the gallery} pAmara raMjanemADu; keLadajeRyavaranunx mecicxsalu, olisikoLaLxlu parxyatinxsu. 
\emng
\eentry

\bentry
\word[gallery(2)]{gallery}
\pron{gAYxlari}
\gl{\sakirx}
\bmng
\bnum
\num{1} gAYxlari(gaLanunx) -- odagisu, aLavaDisu; soVpAnapiVTha odagisu. 
\num{2} (seYnayx) suraMgamAgaR racisu; suraMgahAdi mADu. 
\enum
\emng
\eentry

\bentry
\word{galleryful}
\pron{gAYxlariphulf}
\gl{\nA}
\bmng
 gAYxlari BatiR, tuMbuvaSuTx, hiDiyuvaSuTx jana, perxVkaSxkaru, \mo varu. 
\emng
\eentry

\bentry
\wordnospeech{gallery hit}{gallery hit}
\pron{?}
\gl{\nA}
\bmng
\hypertarget{gallery hit(1)}{} 
\bnum
\num{1} (kirxkeTf ATa) gatitxna hoDeta; gAYxlari -- hoDeta, ATa; tiLiyada perxVkaSxkariMda capApxLe giTiTxsalu ADuva Dwlina ATa. 
\num{2} (nATakada \vi)(pAmara) janaraMjaneya aBinaya; keLadajeRya perxVkaSxkaranunx mecicxsalu mADuva aBinaya. 
\enum
\emng
\eentry

\bentry
\word{galleryite}
\pron{gAYxlariaiTf}
\gl{\nA}
\bmng
 gAYxlariga; gAYxlari perxVkaSxka; gAYxlari siVTinalilx kuLitiruvava. 
\emng
\eentry

\bentry
\wordnospeech{gallery shot}{gallery shot}
\pron{?}
\gl{\nA}
\bmng
 = \hyperlink{gallery hit(1)}{gallery hit(1)}. 
\emng
\eentry

\bentry
\wordnospeech{gallery stroke}{gallery stroke}
\pron{?}
\gl{\nA}
\bmng
  = \hyperlink{gallery hit(1)}{gallery hit(1)}. 
\emng
\eentry

\bentry
\wordnospeech{gallery tray}{gallery tray}
\pron{?}
\gl{\nA}
\bmng
 gAYxlari -- taTeTx, TerxV; loVTagaLu \mo vanunx iTuTxkoMDu hoVgalu baLasuva, etatxrada ENuLaLx beLiLxya taTeTx, harivANa. 
\emng
\eentry

\bentry
\word{galley}
\pron{gAYxli}
\gl{\nA}
\expl{(\bava\ \eng{galleys}).}
\bmng
\bnum
\num{1} (\kanmu\ cariterx) (hAyigaLanUnx huTuTxgaLanUnx baLasuva, \sA\ gulAmaroV keYdigaLoV naDesuva) oMdaMtasitxna capapxTeya haDagu. \imglink{galley-1figure}{\raisebox{-0.10cm}[0pt][0pt]{\pdfimage width 0.7cm height 0.5cm {G_Pictures/galley-1.jpg}}} 
\num{2} (\ca) (oMdu yA hecucx aMtasutxgaLalilx huTuTx hAki naDesuva) pArxciVna roVmananxra yA girxVkara yudadhxnwke. 
\num{3} (yudadhxda haDagina kAYxpaTxnf upayoVgisuva tarahada) terapu doVNi; meVlugaDeyalilx mucicxrada doDaDx huTuTx doVNi. 
\num{4} haDagina yA vimAnada aDigemane. 
\num{5} (mudarxNa) gAYxli; joVDisida moLegaLa sAlugaLanunx iDuva, mUru kaDe aMciruva AyatAkArada taTeTx. 
\num{6}  = \hyperlink{galley proof}{galley proof}. 
\enum
\emng

\noindent
\gl{\nuga}
\bmng
 \eng{in this etc. galley} I aniriVkiSxta parisithxtiyalilx. 
\emng
\eentry


\bentry
\wordnospeech{galley proof}{galley proof}
\pron{?}
\gl{\nA}
\bmng
 gAYxli -- karaDacucx, pUrxphu; gotAtxda oMdu AkArada puTagaLalilxyoV, hALegaLalilxyoV tegeyade, oMdeV kAlamimxna udadxda kAgadagaLalilx tegeda karaDacucx parxti. 
\emng
\eentry

\bentry
\word{galley-slave}
\pron{gAYxliselxVvf}
\gl{\nA}
\bmng
\bnum
\num{1} gAYxli gulAma; haDaginalilx huTuTx hAkuva daMDanege guriyAdavanu. 
\num{2} (\rUpa) duDitada katetx; virAmavilalxde duDiyabeVkAdavanu. 
\enum
\emng
\eentry

\bentry
\word{galley-west}
\pron{gAYxlivesfTx}
\gl{\kirxvi}
\bmng
 (\ame) usirADadaMte; solelxtatxdaMte: \eng{he knocked his opponent galley-west} avanu tananx edurALiyanunx solelxtatxdaMte hoDedu malagisida. 
\emng
\eentry

\bentry
\word{galleyworm}
\pron{gAYxlivamfR}
\gl{\nA}
\bmng
 gAYxli huLu; huTuTxgaLaMte aneVka kAlugaLuLaLx, oMdu bageya huLu. 
\emng
\eentry

\bentry
\word{gall-fly}
\pron{gAYxlfphelxY}
\gl{\nA}
\bmng
 gAlf kiVTa; gaMTu kiVTa; Okf maradalilx gaMTuMTumADuva oMdu jAtiya kirxmi. 
\emng
\eentry

\bentry
\word[galliambic(1)]{galliambic}
\pron{gAYxliAYxMbikf}
\gl{\gu}
\bmng
 gAYxliyAMbikf CaMdasisxna; divxayAMbikf gaNada; paMkitxyalilx eraDu ayAMbikf (\eng{U -- }) gaNagaLiruva. 
\emng
\eentry

\bentry
\word[galliambic(2)]{galliambic}
\pron{gAYxliAYxMbikf}
\gl{\nA}
\bmng
 (\sA\ \bava) gAYxliyAMbikf CaMdasisxnalilx, eraDu ayAMbikf gaNagaLiruva paMkitxgaLalilx, racitavAda kavana, padayx. 
\emng
\eentry

\bentry
\word{galliard}
\pron{gAYxlia(A)DfR}
\gl{\nA}
\bmng
\bnum
\num{1} gAYxliyaDfR (naqtayx); veVgavAda, ulAlxsada, ibabxru kuNiyuva tirxpuTagatiya naqtayx. 
\num{2} I naqtayxda saMgiVta. 
\enum
\emng
\eentry

\bentry
\word[gallic(1)]{gallic}
\pron{gAYxlikf}
\gl{\gu}
\bmng
Okf marada meVle Aguva gaMTina yA A gaMTiniMda tayArisida. 
\emng
\eentry

\bentry
\word[Gallic(2)]{Gallic}
\pron{gAYxlikf}
\gl{\gu}
\bmng
\bnum
\num{1} gAlf janara. 
\num{2} gAlf BASeya. 
\num{3} (\hA) pherxMcara; pherxMcf riVtiya. 
\enum
\emng
\eentry

\bentry
\wordnospeech{gallic acid}{gallic acid}
\pron{?}
\gl{\nA}
\bmng
 (\ravi) gAYxlikf Amalx; sasayxparxpaMcadalilx aneVkaveVLe mukatx sithxtiyalilxruva, TAyxninfgaLalilx saMyukatx sithxtiyalilxruva, bareyuva shAyiya tayArikeyalilx baLasuva, \eng{$\bg\rm G\eg_7\bg\rm H\eg_6\bg\rm O\eg_7$} aNusUtarxvuLaLx, biLiya saPxTika rUpada, kAbaRnika saMyukatx. 
\emng
\eentry

\bentry
\word[Gallican(1)]{Gallican}
\pron{gAYxlikanf}
\gl{\gu}
\bmng
\bnum
\num{1} `gAlf' yA phArxnisxna pArxciVna caciRna. 
\num{2} (poVpanige saMbaMdhisidaMte savxlapxmaTiTxge savxyamAdhipatayx sAdhisuva, modalige phArxnisxna) roVmanf kAYxtholikf pakaSxda, paMgaDada. 
\enum
\emng
\eentry

\bentry
\word[Gallican(2)]{Gallican}
\pron{gAYxlikanf}
\gl{\nA}
\bmng
gAYxlikanf pakaSxda anuyAyi. 
\emng
\eentry

\bentry
\word{Gallicanism}
\pron{gAYxlikanisaZmf}
\gl{\nA}
\bmng
\bnum
\num{1} gAYxlikanf sidAdhxMta; hiMde phArxnisxnalilx parxtipAditavAda,poVpaniMda savxlapxmaTiTxge savxyamAdhipatayx sAdhisuva pherxMcf roVmanf kAYxtholikf paMgaDada sidAdhxMta matutx AcaraNegaLa sUtarx. 
\num{2} gAYxlikanf sidAdhxMtada yA adakekx savxyamAdhipatayx beVkeMba samathaRne. 
\enum
\emng
\eentry

\bentry
\word{Gallicanist}
\pron{gAYxlikanisfTx}
\gl{\nA}
\bmng
 gAYxlikanf sidAdhxMti yA samathaRka. 
\emng
\eentry

\bentry
\word{gallice}
\pron{gAYxlisf}
\gl{\kirxvi}
\bmng
pherxMcinalilx; pherxMcf riVtiyalilx (iMgilxSf mAtu \mo vakekx hoMduva pherxMcf mAtu \mo vanunx heVLuvalilx baLake). 
\emng
\eentry

\bentry
\word{Gallicise}
\pron{gAYxliseYsfZ}
\gl{\kirx}
\bmng
=  \hyperlink{Gallicize}{Gallicize}. 
\emng
\eentry

\bentry
\word{Gallicism}
\pron{gAYxlisisaZmf}
\gl{\nA}
\bmng
\bnum
\num{1} pherxMcf nuDigaTuTx; perxMcf BASA -- veYshiSaTxyX, saMparxdAya; (\kanmu\ anayx BASeya BASaNakAra yA leVKaka baLasuvaMtha) pherxMcf vAkasxraNi. 
\num{2} dhArALavAgi pherxMcf BASA saraNiyanunx baLasuvudu. 
\num{3} (naDenuDi, AcAravicAra. \mo vugaLalilx) pherxMcftana; pherxMcara veYshiSaTxyX, veYlakaSxNayx, padadhxti, riVti: \eng{cheek kissing is a gallicism} kenenxge mutitxDuvuduoMdu pherxMcf padadhxti. 
\enum
\emng
\eentry

\bentry
\word{Gallicize}
\pron{gAYxliseYsfZ}
\gl{\sakirx}
\bmng
\bnum
\num{1} pherxMciVkarisu; pherxMcara guNalakaSxNagaLanunx paDeyuvaMte mADu: \eng{Gallicize an English writer} iMgilxSf leVKakananunx pherxMcf leVKakananAnxgi kANuvaMte mADu. 
\num{2} (beVre BASeya pada yA padagucaCxvanunx) pherxMcf kAguNitakekx yA ucAcxraNege yA pherxMcf saMvAdi padakekx yA padagucaCxkekx aLavaDisu: \eng{Walker gallicized his name to Marcheur} vAkarf tananx hesaranunx mAcaRrf eMdu aLavaDisikoMDa. 
\enum
\emng

\noindent
\gl{\akirx}
\bmng
 pherxMcaraMtAgu; pherxMcara naDenuDi, AloVcane, manoVdhamaR, \mo vugaLanunx anusarisu yA anukarisu. 
\emng
\eentry

\bentry
\word{galligaskins}
\pron{gAYxligAYxsikxnfsx}
\gl{\nA}
\bmng
 (\bava) (\hA) SarAyi; calalxNa; ijAra. 
\emng
\eentry

\bentry
\word{gallimaufry}
\pron{gAlimAphirx}
\gl{\nA}
\expl{(\bava\ \eng{gallimaufries}).}
\bmng
vijAtiVya misharxNa; kalaberake; toLasuMbaLasu; kalasumeVloVgara. 
\emng
\eentry

\bentry
\word[gallinacean(1)]{gallinacean}
\pron{gAYxlineVSanf}
\gl{\gu}
\bmng
 kukukxTa jAtiya. 
\emng
\eentry

\bentry
\word[gallinacean(2)]{gallinacean}
\pron{gAYxlineVSanf}
\gl{\nA}
\bmng
 kukukxTa jAtiya pakiSx. 
\emng
\eentry

\bentry
\word{gallinaceous}
\pron{gAYxlineVSasf}
\gl{\gu}
\bmng
 kukukxTa jAtiya. 
\emng
\eentry

\bentry
\word{gallinazo}
\pron{gAYxlinAsoZV}
\gl{\nA}
\expl{(\bava\ \eng{gallinazos}).}
\bmng
\bnum
\num{1} amerikada oMdu jAtiya raNahadudx. 
\num{2} tukiR DeVge; tukiR deVshada oMdu jAtiya DeVge. 
\enum
\emng
\eentry

\bentry
\word{galling}
\pron{gAliMgf}
\gl{\gu}
\bmng
\bnum
\num{1} reVgisuva; piVDisuva; kATakoDuva. 
\num{2} mAna kaLeyuvaMtha; avamAnakara; hiVnAyavenisuvaMtha. 
\enum
\emng
\eentry

\bentry
\word{gallingly}
\pron{gAliMgfli}
\gl{\kirxvi}
\bmng
\bnum
\num{1} reVgisuvaMte; kATa koDuva hAge. 
\num{2} avamAnakaravAgi. 
\enum
\emng
\eentry

\bentry
\word{gallinule}
\pron{gAYxlinUyxlf}
\gl{\nA}
\bmng
\bnum
\num{1} gAYxlinUyxlf kulada saNaNx niVru hakikx, jalapakiSx. 
\num{2} pAphiRrula yA pAphiRriyoV kulada ideV tarahada vividha pakiSxgaLalolxMdu. 
\enum
\emng
\eentry

\bentry
\word{Gallio}
\pron{gAYxliO}
\gl{\nA}
\expl{(\bava\ \eng{Gallios}).}
\bmng
\bnum
\num{1} (\kanmu\ adhikAriya \vi) tananx adhikAravAyxpitxge saMbaMdhisada viSayadalilx keYhAkalolalxdavanu. 
\num{2} uDAPeyavanu; udAsiVna; beVjavAbAdxriyiMdidudx elalxvanUnx laGuvAgi tegedukoLuLxvava. 
\enum
\emng
\eentry

\bentry
\word{galliot}
\pron{gAYxliaTf}
\gl{\nA}
\bmng
\bnum
\num{1} sAmAnu sAgisuva yA mInu hiDiyuva Dacacxra haDagu. 
\num{2} (\sA\ meDiTareVniyaninxna) cikakx gAYxli haDagu. 
\enum
\emng
\eentry

\bentry
\word{gallipot}
\pron{gAYxlipATf}
\gl{\nA}
\bmng
 (mulAmu \mo vanunx iDuva) piMgANi, loVha, \mo vugaLa kuDike. 
\emng
\eentry

\bentry
\word{gallium}
\pron{gAYxliamf}
\gl{\nA}
\bmng
 (\ravi) gAYxliyaM; niVli CAyeya biLi baNaNxda, tirxveVlenisxVya maqdu loVhadhAtu (\saMkeV\ \eng{Ga.}) 
\emng
\eentry

\bentry
\word{gallivant}
\pron{gAYxlivAYxMTf}
\gl{\akirx}
\bmng
 (\AmA) 
\bnum
\num{1} sutAtxDu; aledADu; ThaLAyisu. 
\num{2} celAlxTavADu; parxNayaceVSeTxyalilx toDagu. 
\enum
\emng
\eentry

\bentry
\word{galliwasp}
\pron{gAYxlivAYxsfpx}
\gl{\nA}
\bmng
vesfTx iMDiVsfna oMdu bageya halilx. 
\emng
\eentry

\bentry
\word{gallless}
\pron{gAlflisf}
\gl{\gu}
\bmng
swmayx; maqdusavxBAvada; kaTuBAvane, kAThinayx, niSuThxrate, devxVSa, veYra -- ilalxda. 
\emng
\eentry

\bentry
\word{gall-mite}
\pron{gAlfmeYTf}
\gl{\nA}
\bmng
 gAlf jeVDa; Okf maradalilx gaMTuMTumADuva oMdu bageya jeVDa. 
\emng
\eentry

\bentry
\word{gallnut}
\pron{gAlfnaTf}
\gl{\nA}
\bmng
  = \hyperlink{gall(4)}{$^4$gall}. 
\emng
\eentry

\bentry
\word{Gallo-}
\pron{gAYxloV-}
\gl{\sapUpa}
\bmng
\bnum
\num{1} pherxMcf -- ; pherxMcf matutx -- eMba athaRdalilx baLasuva \sapUpa: \eng{Gallo-Briton} pherxMcf birxTanf. \eng{Gallo-German} pherxMcfjamaRnf. 
\num{2} = \hyperlink{Gaul}{Gaul}: \eng{Gallo-Roman} gAlfroVmanf. 
\enum
\emng
\eentry

\bentry
\word{Gallomania}
\pron{gAYxloVmeVnia}
\gl{\nA}
\bmng
 pherxMcf moVha; pherxMcfgiVLu; phArxnisxna yA pherxMcara viSayavAgi hucucx moVha yA aBimAna. 
\emng
\eentry

\bentry
\word[Gallomaniac(1)]{Gallomaniac}
\pron{gAYxloVmeVniAYxkf}
\gl{\gu}
\bmng
 pherxMcf moVhada; phArxnisxna yA pherxMcara viSayavAgi giVLu, hucucxmoVha, yA aMdhABimAna -- iruva. 
\emng
\eentry

\bentry
\word[Gallomaniac(2)]{Gallomaniac}
\pron{gAYxloVmeVniAYxkf}
\gl{\nA}
\bmng
 pherxMcf aBimAni; pherxMcara yA phArxnisxna viSayavAgi hucucx moVha yA aBimAna iruvavanu. 
\emng
\eentry

\bentry
\word{gallon}
\pron{gAYxlanf}
\gl{\nA}
\bmng
 (pUNaRvAgi \eng{imperial gallon} eMdu \parx) 
\bnum
\numi{1} gAYxlanunx: 
\banum
\alnum{a} darxva, (\birx) kALu, \mo vugaLa oMdu aLate (\eng{8} peYMTfgaLu). 
\hypertarget{gallon 1(b)}{} 
\alnum{b} (\birx) \eng{4546} Gana seMmI. 
\hypertarget{gallon 1(c)}{} 
\alnum{c} (\ame) \eng{3785} Gana seMmI. 
\eanum
\numie
\num{2} (\sA\ \bava dalilx) gAYxlanfgaTaTxle; hecucx parxmANa; BAri motatx. 
\enum
\emng

\noindent
\gl{\pagu}
\bmng
\bnum
\num{1} \eng{imperial gallon} (\birx)  = \hyperlink{gallon 1(b)}{gallon 1(b)}. 
\num{2} \eng{wine gallon} (\birx)  = \hyperlink{gallon 1(c)}{gallon 1(c)}. 
\enum
\emng
\eentry

\bentry
\word{gallonage}
\pron{gAYxlanijf}
\gl{\nA}
\bmng
 gAYxlanfmAna; gAYxlanunxgaLalilx vayxkatxpaDisida darxvada parxmANa. 
\emng
\eentry

\bentry
\word{galloon}
\pron{galUnf}
\gl{\nA}
\bmng
 alaMkArada aMcupaTiTx; uDupina alaMkArakekx hAkuva, jaratAri, reVSemx, neYlAnf yA hatitxya dAragaLiMda otAtxgi heNeda, kiriya aMcupaTiTx. 
\emng
\eentry

\bentry
\word[gallop(1)]{gallop}
\pron{gAYxlapf}
\gl{\nA}
\bmng
\bnum
\num{1} (kudure \mo vugaLa) nAgAloVTa; dwDu; parxti dApinalilxyU nAlukx kAlugaLanunx neladiMda meVlakekxtitx ODuva OTa. 
\num{2} nAgAloVTada savAri. 
\num{3} nAgAloVTa patha; aMtha savArigAgi iruva hAdi yA meYdAna. 
\enum
\emng

\noindent
\gl{\pagu}
\bmng
\bnum
\num{1} \eng{at a gallop} dwDAyisutitxruva; nAgAloVTadalilx ODutitxruva. 
\num{2} \eng{full gallop} dwDoVTa; nAgAloVTa. 
\enum
\emng
\eentry

\bentry
\word[gallop(2)]{gallop}
\pron{gAYxlapf}
\gl{\kirx}
\expl{(\vakaq\ \eng{galloping,} \BU\ matutx \BUkaq\ \eng{galloped}).}
\bmng
\emng

\noindent
\gl{\sakirx}
\bmng
\bnum
\num{1} nAgAloVTadalilx savAri mADu. 
\num{2} (kudureyanunx) nAgAloVTadalilx ODisu; BaradiMda ODisu. 
\enum
\emng

\noindent
\gl{\akirx}
\bmng
\bnum
\num{1} (kudureya, adara savArana yA itara catuSApxdiya \vi) nAgAloVTa ODu; nAgAloVTadalilx hoVgu. 
\num{2} veVgavAgi Odu, paThisu yA mAtanADu. 
\num{3} veVgavAgi hoVgu; nAgAloVTadalilx sAgu; dwDu hoDi; OTa kiVLu. 
\num{4} nAgAloVTadalilx hoVgu; BaradiMda muMduvari; tiVvarxvAgu; hecAcxgu; parxbalvAgu: \eng{galloping inflation} nAgAloVTadalilx Erutitxruva haNadubabxra. \eng{galloping consumption} kiSxparxvAgi tiVvarxvAgutitxruva kaSxyaroVga. 
\enum
\emng
\eentry

\bentry
\word{gallopade}
\pron{gAYxlapeVDf}
\gl{\nA}
\bmng
 (hiMde haMgariyananxra) oMdu terada ulAlxsanataRna, geluvina kuNita. 
\emng
\eentry

\bentry
\word{galloper}
\pron{gAyxlaparf}
\gl{\nA}
\bmng
\bnum
\num{1} nAgAloVTa ODuva vayxkitx yA kudure. 
\num{2} dwDu savAra; nAgAloVTadalilx savAri mADuvavanu. 
\num{3} veVgavAgi hoVguvavanu(Lu); BaradiMda sAguvavanu(Lu). 
\num{4} (\kanmu\ seYnayx) seVnApati \mo vara -- sahAyaka, aMgarakaSxka, meYgAvalugAra. 
\num{5} (veVgavAgi oyayxbahudAda) hagura PiraMgi. 
\enum
\emng
\eentry

\bentry
\word[Gallophil(1)]{Gallophil}
\pron{gAYxlaphilf}
\gl{\nA}
\bmng
 pherxMcf aBimAni; phArxnisxna mitarx. 
\emng
\eentry

\bentry
\word[Gallophil(2)]{Gallophil}
\pron{gAYxlaphilf}
\gl{\gu}
\bmng
pherxMcf aBimAniyAda; phArxnisxna mitarxnAda. 
\emng
\eentry

\bentry
\word[Gallophile(1)]{Gallophile}
\pron{gAYxlapheYlf}
\gl{\nA}
\bmng
  = \hyperlink{Gallophil(1)}{$^1$Gallophil}. 
\emng
\eentry

\bentry
\word[Gallophile(2)]{Gallophile}
\pron{gAYxlapheYlf}
\gl{\gu}
\bmng
 = \hyperlink{Gallophil(2)}{$^2$Gallophil}. 
\emng
\eentry

\bentry
\word[Gallophobe(1)]{Gallophobe}
\pron{gAYxlaphoVbf}
\gl{\nA}
\bmng
 pherxMcf BiVta; pherxMcara yA pherxMcf saMbaMdhada viSayavAgi hucucx Baya, ati BiVti uLaLxvanu. 
\emng
\eentry

\bentry
\word[Gallophobe(2)]{Gallophobe}
\pron{gAYxlaphoVbf}
\gl{\gu}
\bmng
 pherxMcf BiVtiya. 
\emng
\eentry

\bentry
\word{Gallophobia}
\pron{gAYxlphoVbia}
\gl{\nA}
\bmng
 pherxMcf BiVti; pherxMcara yA pherxMcf saMbaMdhada viSayavAgi hucucx Baya, ati BiVti. 
\emng
\eentry

\bentry
\word[Gallo-Roman(1)]{Gallo-Roman}
\pron{gAyxloVroVmanf}
\gl{\nA}
\bmng
 gAYxloVroVmanf; roVmanf ALivxkeya aMgavAgidAdxgina kAlada gAlf pArxMtakekx seVridavanu yA alilxya BASe. 
\emng
\eentry

\bentry
\word[Gallo-Roman(2)]{Gallo-Roman}
\pron{gAYxloVroVmanf}
\gl{\gu}
\bmng
 gAYxloVroVmanf; roVmanf ALivxkeya aMgavAgidAdxgina kAlada gAlf pArxMtakekx seVrida. 
\emng
\eentry

\bentry
\word[Gallovidian(1)]{Gallovidian}
\pron{gAYxlaviDianf}
\gl{\gu}
\bmng
 sAkxTalxMDina gAYxloVveV pArxMtakekx seVrida. 
\emng
\eentry

\bentry
\word[Gallovidian(2)]{Gallovidian}
\pron{gAYxlaviDianf}
\gl{\nA}
\bmng
 sAkxTalxMDina gAYxloVveV pArxMtadavanu. 
\emng
\eentry

\bentry
\word{Galloway}
\pron{gAYxlaveV}
\gl{\nA}
\bmng
\bnum
\num{1} neYQutayx sAkxTalxMDina gAyxloVveV pArxMtada balavAda giDaDx kudure. 
\num{2} cikakx kudure. 
\num{3} gAYxloVveV taLiya dana. 
\enum
\emng
\eentry

\bentry
\word{gallows}
\pron{gAYxloVsf}
\gl{\nA}
\bmng
 (\sA\ \Eva veMdu parigaNisalAgide). 
\bnum
\num{1} galulx; galulxmara; neVNugaMba. 
\num{2} galulxshikeSx; PAsi saja; neVNu shikeSx; maraNa daMDane. 
\num{3} (aDige, aMgasAdhane, \mo vugaLige upayoVgisuva) galulx maradaMtha racane. 
\enum
\emng

\noindent
\gl{\nuga}
\bmng
\bnum
\num{1} \eng{a gallows look} GAtuka -- daqSiTx, cahare, muKaBAva. 
\num{2} \eng{have the gallows in one's face} muKada meVle GAtuka cahareyuLaLx, muKaBAvavuLaLx. 
\enum
\emng
\eentry

\bentry
\word{gallows-bird}
\pron{gAYxloVsfZbaDfR}
\gl{\nA}
\bmng
 PAsi yoVgayx; galilxge hAkalu takakx aparAdhi. 
\emng
\eentry

\bentry
\wordnospeech{gallows humour}{gallows humour}
\pron{?}
\gl{\nA}
\bmng
 PAsi hAsayx; nidaRyavAda, vayxMgayxvAda hAsayx. 
\emng
\eentry

\bentry
\word{gallows-ripe}
\pron{gAYxloVsfreYpf}
\gl{\gu}
\bmng
 galilxge (hAkalu) -- sidadhxvAgiruva, takakxvanAgiruva. 
\emng
\eentry

\bentry
\word{gallows-tree}
\pron{gAYxloVsfZTirxV}
\gl{\nA}
\bmng
 galulxmara; neVNugaMba; neVNumara. 
\emng
\eentry

\bentry
\word{gallstone}
\pron{gAlfsoTxVnf}
\gl{\nA}
\bmng
 pitAtxshamxri; pitatxgalulx; pitatxkoVshadalilx kelavuveVLe uMTAguva kalulxharaLinaMtha kAya. 
\emng
\eentry

\bentry
\wordRemoveSpace{Gallup-poll}{Gallup poll}
\pron{gAYxlapf poVlf}
\gl{\nA}
\bmng
 gAyxlapf mata saMgarxhaNe; sAvaRjanika aBipArxya saMgarxhaNe; yAvudeV oMdu viSayadalilx sAvaRjanikara aBipArxya garxhisalu pariVkASxthaRvAgi naDesuva matasaMgarxhaNe. 
\emng
\eentry

\bentry
\word{galluses}
\pron{gAYxlasisfZ}
\gl{\nA}
\bmng
 (\bava) ( \pArxM\ matutx \ame ) SarAyipaTiTx; SarAyineVlu; SarAyi keLage biVLadaMte Bujada meVliMda tUgahAkuva paTiTxgaLu. 
\emng
\eentry

\bentry
\word{gall-wasp}
\pron{gAlfvAsfpx}
\gl{\nA}
\bmng
 gaMTukaNaja; Okf maradalilx gaMTuMTumADuva oMdu bageya kaNaja. 
\emng
\eentry

\bentry
\word{galoot}
\pron{galUTf}
\gl{\nA}
\bmng
 (\AmA) oDaDx (manuSayx); oraTa. 
\emng
\eentry

\bentry
\word[galop(1)]{galop}
\pron{gAyxlapf}
\gl{\nA}
\bmng
 gAYxlapf; oMdu bageya gelavina kuNita yA adara saMgiVta. 
\emng
\eentry

\bentry
\word[galop(2)]{galop}
\pron{gAyxlapf}
\gl{\akirx}
\expl{(\vakaq\ \eng{galoping}, \BU\ matutx \BUkaq\ \eng{galoped}).}
\bmng
gAYxlapf kuNita kuNi; gAYxlapf nataRnamADu. 
\emng
\eentry

\bentry
\word[galore(1)]{galore}
\pron{galoVrf}
\gl{\nA}
\bmng
 (\gaparx) samaqdidhx; puSakxLate; heVraLa; daMDi; vipulate; yatheVSaTxte. 
\emng
\eentry

\bentry
\word[galore(2)]{galore}
\pron{galoVrf}
\gl{\kirxvi}
\bmng
 samaqdidhxyAgi; puSakxLavAgi; daMDiyAgi; heVraLavAgi: \eng{with beef and ale galore} mAMsamadayxgaLu samaqdidhxyAgi. \eng{galore of alcohol} madayx koVDi hariyuvaMte. 
\emng
\eentry

\bentry
\word{galosh}
\pron{galASf}
\gl{\nA}
\bmng
 (\eng{golosh} eMdU \parx) 
\bnum
\num{1} (\sA\ \bava dalilx) (pAdarakeSx maNANxgadaMte, neneyadaMte adara meVle toDuva, \sA\ rababxrina) meVlojxVDu. 
\num{2} pAdarakeSx kavaca; pAdarakeSx hodike; bUTisxna keLaBAgavanunx yA pAdarakeSxya meVlABxgavanunx mucucxva camaR \mo vugaLa cUru, paTiTx. 
\enum
\emng
\eentry

\bentry
\word{galoshed}
\pron{galASfTx}
\gl{\gu}
\bmng
(pAdarakeSxya \vi) mucucxpaTiTxyiMda mucicxda; mucucxpaTiTx hAkida; kavaca hAkida; hodike hodisida. 
\emng
\eentry

\bentry
\word{galumph}
\pron{galaMphf}
\gl{\akirx}
\bmng
(\AmA) 
\bnum
\num{1} gelaviniMda, utAsxhadiMda negeyutatx, kuNiyutatx hoVgu. 
\num{2} oDoDxDADxgi yA shabadxmADutatx hoVgu, sAgu, calisu. 
\enum
\emng
\eentry

\bentry
\word{galvanic}
\pron{gAYxlAvxYxnikf}
\gl{\gu}
\bmng
\bnum
\num{1} (\ca) viduyxtitxna yA viduyxtitxnaMtha; viduyxtitxniMda utapxtitxyAda; rAsAyanika kirxyeyiMda viduyxtapxrXvAhavanunx utapxtitx mADuva. 
\num{2} (\rUpa) viduyxtasxMcAradaMtha; thaTaTxneya matutx asAdhAraNavAda: \eng{had a galvanic effect} thaTaTxneya matutx asAdhAraNavAda pariNAma biVritu. 
\num{3} huriduMbisuva; utetxVjakavAda; udidxVpisuva; parxcoVdisuva. 
\num{4} shakitxpUNaRvAda; BajaRri: \eng{a galvanic performance} BajaRri parxdashaRna yA nivaRhaNe. 
\enum
\emng
\eentry

\bentry
\word{galvanically}
\pron{gAYxlAvxYxnikali}
\gl{\kirxvi}
\bmng
\bnum
\num{1} thaTaTxne; haThAtAtxgi. 
\num{2} utetxVjakavAgi; huriduMbisuvaMte. 
\enum
\emng
\eentry

\bentry
\wordnospeech{galvanic battery}{galvanic battery}
\pron{?}
\gl{\nA}
\bmng
 gAYxlavxnika bAyxTari; rAsAyanika kirxyeyiMda viduyxtatxnunx utapxtitx mADuva bAyxTari. 
\emng
\eentry

\bentry
\wordnospeech{galvanic electricity}{galvanic electricity}
\pron{?}
\gl{\nA}
\bmng
 gAYxlavxnika viduyxtutx; pArxthamika bAyxTariyalilx heVgoV hAge rAsAyanika kirxyeyiMda huTuTxva neVra viduyxtapxrXvAha. 
\emng
\eentry

\bentry
\wordnospeech{galvanic pile}{galvanic pile}
\pron{?}
\gl{\nA}
\bmng
 gAYxlavxnika peVrike; Binanx loVhada tagaDugaLu matutx viduyxdivxceCxVdayxdalilx adidxda raTuTx muMtAduvanunx oMdara meVloMdaraMte peVrisi adariMda viduyxtapxrXvAha huTuTxvaMte EpaRDisiruva peVrike. 
\emng
\eentry

\bentry
\word{galvanisation}
\pron{gAYxlavxneYseZVSanf}
\gl{\nA}
\bmng
  = \hyperlink{galvanization}{galvanization}. 
\emng
\eentry

\bentry
\word{galvanise}
\pron{gAYxlavxneYsfZ}
\gl{\sakirx}
\bmng
  = \hyperlink{galvanize}{galvanize}. 
\emng
\eentry

\bentry
\word{galvaniser}
\pron{gAYxlavxneYsaZrf}
\gl{\nA}
\bmng
  = \hyperlink{galvanizer}{galvanizer}. 
\emng
\eentry

\bentry
\word{galvanism}
\pron{gAYxlavxnisaZmf}
\gl{\nA}
\bmng
\bnum
\num{1} (\ca) (\kanmu\ rAsAyanika kirxyeyiMda huTiTxda) neVra viduyxtapxrXvAha. 
\num{2} neVra viduyxtf parxyoVga; veYdayxkiVyadalilx neVra viduyxtapxrXvAhada baLake. 
\enum
\emng
\eentry

\bentry
\word{galvanist}
\pron{gAYxlavxnisfTx}
\gl{\nA}
\bmng
 neVra viduyxtf parxyoVgakAri; veYdayxkiVyadalilx neVra viduyxtapxrXvAhavanunx baLasuvavanu. 
\emng
\eentry

\bentry
\word{galvanization}
\pron{gAYxlavxneYseZVSanf}
\gl{\nA}
\bmng
 gAyxlavxniVkaraNa; \kanmu\ veYdayxkiVya udedxVshagaLige neVra viduyxtapxrXvAhavanunx upayoVgisuvudu. 
\emng
\eentry

\bentry
\word{galvanize}
\pron{gAYxlavxneYsfZ}
\gl{\sakirx}
\bmng
gAyxlavxniVkarisu: 
\banum
\alnum{a} neVra viduyxtapxrXvAhadiMda yA viduyxtitxniMdaloV eMbaMte parxcoVdisu (\rUpa\ saha): \eng{galvanize into life} AGAta yA uderxVkadiMda huriduMbisu. 
\alnum{b} viduyxdivxceCxVdaneyiMda loVhavanunx leVpisu. 
\alnum{c} kabibxNavu tukukx hiDiyadaMte (viduyxtatxnunx baLasade) adara meVle satuvanunx leVpisu. 
\eanum
\emng
\eentry

\bentry
\word{galvanizer}
\pron{gAYxlvxneYsaZrf}
\gl{\nA}
\bmng
 gAyxlavxniVkAri; gAyxlavxniVkarisuva vayxkitx yA sAdhana. 
\emng
\eentry

\bentry
\word{galvano-}
\pron{gAYxlavxnoV(na)-}
\gl{\sapUpa}
\bmng
 gAyxlavxnika viduyxtitxna, adanunxpayoVgisuva, adariMda utapxtitxyAguva eMbathaRgaLalilx baLasuva \sapUpa. 
\emng
\eentry

\bentry
\word{galvanograph}
\pron{gAYxlavxnoVgArxphf}
\gl{\nA}
\bmng
 gAyxlavxnoVleVKa; viduyxninxkeSxVpaNadiMda tayArisida tAmarxda paDiyacucx Palaka. 
\emng
\eentry

\bentry
\word{galvanography}
\pron{gAYxlavxnAgarxphi}
\gl{\nA}
\bmng
 gAyxlavxnoVleVKana; viduyxninxkeSxVpaNadiMda tAmarxda paDiyacucx PalakagaLanunx tayArisuvudu. 
\emng
\eentry

\bentry
\word{galvanometer}
\pron{gAYxlavxnAmiTarf}
\gl{\nA}
\bmng
 gAyxlavxnoVmApaka; calisuva kAMtasUci yA suruLiya sahAyadiMda alapx viduyxtapxrXvAhavanunx gurutisuva yA aLeyuva upakaraNa. 
\emng
\eentry

\bentry
\word{galvanometric}
\pron{gAYxlavxnAmeTirxkf}
\gl{\gu}
\bmng
 gAyxlavxnoV mApakada yA adariMda aLeda. 
\emng
\eentry

\bentry
\word{galvanoplasty}
\pron{gAYxlavxnoVpAlxsiTx}
\gl{\nA}
\bmng
 (\gaparx) gAyxlavxnoVrUpaNe; viduyxninxkeSxVpaNa; viduyxlelxVpana vidhAnadiMda nidiRSaTx AkaqtigaLanunx rUpisuva videyx. 
\emng
\eentry

\bentry
\word{galvanoscope}
\pron{gAyxlavxnasokxVpf}
\gl{\nA}
\bmng
 gAyxlavxnoV dashaRka; calisuva kAMtasUciya sahAyadiMda sUkaSxmX viduyxtapxrXvAhada iruvikeyanUnx adara dikakxnUnx gurutisuva upakaraNa. 
\emng
\eentry

\bentry
\word{galvo}
\pron{gAYxlovxV}
\gl{\nA}
\expl{(\bava\ \eng{galvos}).}
\bmng
(\AmA)  = \hyperlink{galvanometer}{galvanometer}. 
\emng
\eentry

\bentry
\word{gamba}
\pron{gAYxMba}
\gl{\nA}
\bmng
  = \hyperlink{gamba stop}{gamba stop}. 
\emng
\eentry

\bentry
\word{gambade}
\pron{gAYxMbeV(bA)Df}
\gl{\nA}
\bmng
 (\eng{gambado} eMdU \parx) (\bava\ \eng{gambades}). 
\bnum
\num{1} kudureya negeta yA kupapxLike. 
\num{2} vilakaSxNa calane. 
\num{3} hucAcxTa; vicitarx vataRne; coMgATa. 
\enum
\emng
\eentry

\bentry
\word{gambado}
\pron{gAYxMbeV(bA)DoV}
\gl{\nA}
\expl{(\bava\ \eng{gambados} yA \eng{gambadoes}).}
\bmng
 = \hyperlink{gambade}{gambade}. 
\emng
\eentry

\bentry
\wordnospeech{gamba stop}{gamba stop}
\pron{?}
\gl{\nA}
\bmng
 gAyxMbA koLalu; piTiVlu koLalu; piTiVlinaMte dhavxnigoDuva oMdu bageya tidikoLalu. 
\emng
\eentry

\bentry
\word{gambier}
\pron{gAYxMbiarf}
\gl{\nA}
\bmng
 biLi kAcu; gAYxMbiyarf; \kanmu\ malayadalilx beLeyuva saNaNx kAcu giDadiMda paDeda, tAMbUladoDane upayoVgisuvudakUkx camaR hada mADuvudakUkx baruva, oMdu bageya padAthaR. 
\emng
\eentry

\bentry
\word{gambit}
\pron{gAYxMbiTf}
\gl{\nA}
\bmng
\bnum
\num{1} (caduraMgada) bali naDe; sAthxnabala paDeyalu peVde yA itara kAyanunx udedxVshapUvaRkavAgi balikoTuTx AraMBisuva ATada oMdu naDe. 
\num{2} (\rUpa) (yAvudeV kAyARcaraNeyalilx) modala hejejx; moTaTx modalaneya naDe, calana. 
\num{3} (saMBASaNe, caceR, \mo vanunx pArxraMBisuva) modala mAtu; piVThikeya mAtu; AraMBada mAtu; pArxsAtxvika aMsha. 
\num{4} taMtarx; hUTa; upAya. 
\enum
\emng
\eentry

\bentry
\word[gamble(1)]{gamble}
\pron{gAYxMbflf}
\gl{\sakirx}
\bmng
\bnum
\num{1} paNa oDuDx. 
\num{2} paNavAgi oDiDx (haNa) kaLeduko. 
\enum
\emng

\noindent
\gl{\akirx}
\bmng
\bnum
\num{1} duDiDxTuTx (\kanmu\ hecAcxgi haNa oDiDx) jUjADu; jugArADu; dUyxtavADu. 
\num{2} naSaTxkara kelasakekx keYhAku; apAyakara sAhasakekx keYhAku; yudadhx, haNakAsina vayxvahAra, \mo vugaLalilx mahataPxla sAdhisalu apAyada parxyatanxgaLanunx keYgoLuLx. 
\enum
\emng

\noindent
\gl{\pagu}
\bmng
 \eng{gamble away} jUjADi kaLe; paNa oDiDx kaLeduko. 
\emng

\noindent
\gl{\nuga}
\bmng
 \eng{gamble on} Ase yA Baravaseya meVle karxma keYgoLuLx, naDe, muMduvari: \eng{gamble on weather being fine tomorrow} nALe have sariyAgirutatxdeMba Baravaseya meVle muMduvari. 
\emng
\eentry

\bentry
\word[gamble(2)]{gamble}
\pron{gAYxMbflf}
\gl{\nA}
\bmng
\bnum
\num{1} jUju; jugAru; dUyxta: \eng{on the gamble} jUjigoDiDxda; paNavAgiTaTx. 
\num{2} naSaTxkara sAhasa; apAyada udayxma; naSaTxkekx, apAyakekx -- sikakxbahudAda udayxma, parxyatanx. 
\enum
\emng
\eentry

\bentry
\word{gambler}
\pron{gAYxMbalxrf}
\gl{\nA}
\bmng
 jUjugAra; jUjukoVra; jUjALi. 
\emng
\eentry

\bentry
\word{gamblesome}
\pron{gAYxMbflfsamf}
\gl{\gu}
\bmng
 jUjina; jUjADuva (cALiyuLaLx). 
\emng
\eentry

\bentry
\word{gamboge}
\pron{gAYxMbU(boV)SfZ}
\gl{\nA}
\bmng
 kAMBoVjada aMTurALa; haLadiya baNaNx mADalu matutx vireVcakavAgi baLasuva, pUvaR ESAyxda halavu bageya maragaLa aMTurALa. 
\emng
\eentry

\bentry
\word[gambol(1)]{gambol}
\pron{gAYxMbalf}
\gl{\akirx}
\expl{(\vakaq\ \eng{gambolling}, \BU\ matutx \BUkaq\ \eng{gambolled}; (\ame) \vakaq\ \eng{gamboling}, \BU\ matutx \BUkaq\ \eng{gamboled}).}
\bmng
 kuNi; hAru; kupapxLisu; negedADu; jigidADu. 
\emng
\eentry

\bentry
\word[gambol(2)]{gambol}
\pron{gAYxMbalf}
\gl{\nA}
\bmng
 kuNidATa; negedATa; kupapxLike. 
\emng
\eentry

\bentry
\word{gambrel}
\pron{gAYxMbarxlf}
\gl{\nA}
\bmng
  = \hyperlink{gambrel roof}{gambrel roof}. 
\emng
\eentry

\bentry
\wordnospeech{gambrel roof}{gambrel roof}
\pron{?}
\gl{\nA}
\bmng
 ipApxru cAvaNi; cAvaNiya eraDU kaDe eraDu iLijArugaLidudx keLaBAgadudx meVlinadakikxMta hecucx iLijArAgiruva cAvaNi.  \imglink{gambrel rooffigure}{\raisebox{-0.10cm}[0pt][0pt]{\pdfimage width 0.6cm height 0.6cm {G_Pictures/gambrel roof.jpg}}} 
\emng
\eentry

\bentry
\word[game(1)]{game}
\pron{geVmf}
\gl{\nA}
\bmng
\bnum
\num{1} celAlxTa; hAsayx; vinoVda; tamASe; geVli: \eng{was only playing a game with you} ninonxDane bari celAlxTavADutitxdadx. 
\num{2} kirxVDe; keVLi; ATa: \eng{a game of ball} ceMDATa. 
\num{3} vinoVdada, kushAlina -- saMgati, samAcAra, viSaya: \eng{what a game!} eMtha vinoVda! Enu kushAlu! 
\num{4} (jANatana, shakitx, adaqSaTxgaLiMda gelalxbahudAda, niyamAnusAra ADuva) ATa; ATada paMdayx. 
\numi{5} (\bava dalilx) 
\banum
\alnum{a} (girxVkf, roVmanf, pArxkatxnashAsatxrX) (vAyxyAma, nATaka, saMgiVta, \mo vugaLa) paMdayxgaLu; shayaRtutxgaLu. 
\alnum{b} katitxkALaga, malalxyudadhx, \mo\ parxdashaRnagaLu. 
\alnum{c} kirxVDAsapxdheRgaLu; vAyxyAma sapxdheRgaLu: \eng{Olympic games} olaMpikf sapxdheRgaLu. 
\alnum{d} kirxVDAkUTagaLu; shAle, kAleVju, \mo vugaLalilx EpaRDisuva kirxVDegaLu yA vAyxyAma parxdashaRnagaLu. 
\eanum
\numie
\num{6} (ATadaMteyeV) AsakitxyiMda naDesuva, udayxma; kelasa; yoVjane; kAyaRkarxma: \eng{a winning game} gelulxva ATa; yashasivxyAguva udayxma; yashasivxyAguva BaravaseyuLaLx parxyatanx, yoVjane, taMtarx, \mo vu. \eng{a losing game} soVluva ATa; yashasivxyAguva Baravase ilalxda udayxma, hUDike, taMtarx, \mo vu. 
\num{7} hUTa; haMcike; oLasaMcu; pitUri: \eng{was playing a deep game} ALavAda oLasaMcu naDesutitxdadx. \eng{a double game} ibabxMdi ATa; moVsada hUTa. \eng{spoilt my game} nananx haMcike hALumADida. 
\num{8} (\bava dalilx) moVsa; upAya; yukitxgaLu; kuTila taMtarxgaLu; hUTagaLu: \eng{none of your games!} ninanx ATavanenxlAlx nananx eduru toVrisabeVDa; ninanx ATavanunx nananx hatitxra kaTiTxDu; ninanx beVLe ililx beVyuvudilalx. 
\num{9} (birxDfjx, isipxVTATa, Tenisf, \mo\ paMdayxgaLalilx) oMdu ATa; geVmu; iDiV ATada oMdu BAga: \eng{game all} ibabxradU oMdoMdu geVmu; parxtiyoMdu kaDeyadU oMdoMdu ATa. 
\num{10} ATada sAmAnu; kirxVDoVpakaraNagaLu. 
\num{11} (ATadalilx) sokxVrina saMKeyx; gaLisida aMka; gelalxMka. 
\num{12} ATada -- sokxVru, pAyiMTu: \eng{the game is four all} Iga eraDU pakaSxgaLu nAlukx nAlukx ATagaLanunx, pAyiMTugaLanunx gedidxve. 
\num{13} beVTeya pArxNi; aTiTxkoMDu hoVda beVTe. 
\num{14} (\rUpa) (sAdhisabeVkAda) udedxVsha; guri. 
\numi{15} 
\banum
\alnum{a} (samudAyavAcaka parxyoVgadalilx) (kirxVDegAgi, AhArakAkxgi) beVTe; manuSayx beVTeyADuva kADumaqga, pakiSx, mInu, \mo vu. 
\alnum{b} ivugaLa mAMsa. 
\eanum
\numie
\num{16} (vinoVdakAkxgi) sAkida (haMsagaLa) hiMDu. 
\enum
\emng

\noindent
\gl{\pagu}
\bmng
\bnum
\numi{1} \eng{be off one's game} 
\banum
\alnum{a} hurupilalxdiru; utAsxha kaLeda sithxtiyalilxru. 
\alnum{b} sariyAgi ATa ADadiru. 
\eanum
\numie
\numi{2} \eng{be on one's game} 
\banum
\alnum{a} hurupiniMdiru; utAsxha tuMbi oLeLxya sithxtiyalilxru. 
\alnum{b} cenAnxgi ATa ADutitxru. 
\eanum
\numie
\num{3} \eng{big game} (siMha, Ane, \mo) BAri maqgagaLu; BAri beVTe (pArxNigaLu). 
\numi{4} \eng{fair game} 
\banum
\alnum{a} kAnUnubadadhx beVTe (pArxNi); kAnUnina samamxtiya parxkAra beVTeyADabahudAda pArxNi. 
\alnum{b} (\rUpa) nAyxyasamamxtavAgiyeV TiVkisabahudAda yA KaMDisabahudAda vayxkitx yA saMsethx; TiVkAhaR, KaMDanAhaR, AkarxmaNAhaR -- vayxkitx, saMsethx, \mo vu. 
\eanum
\numie
\numi{5} \eng{forbidden game} 
\banum
\alnum{a} niSidadhx beVTe; beVTeyADakUDadeMdu kAnUnu niSeVdhisiruva pArxNi, maqga. 
\alnum{b} TiVkAtiVta, KaMDanAtiVta vayxkitx, saMsethx, \mo vu; TiVke, AkarxmaNa, \mo vanunx niSeVdhisiruva vayxkitx, saMsethx, \mo vu. 
\eanum
\numie
\num{6} \eng{game all} parxti pakaSxvU oMdoMdu ATa gedidxde. 
\num{7} \eng{game and (game and set} enunxvudara harxsavxrUpa) (Tenisf) ATa gedudx seTf mugiyitu. 
\numi{8} \eng{play a good game} 
\banum
\alnum{a} oLeLxya ATavADu. 
\alnum{b} cAtuyaRdiMda vatiRsu. 
\eanum
\numie
\numi{9} \eng{play a poor game} 
\banum
\alnum{a} kaLape ATavADu. 
\alnum{b} (\rUpa) cAtuyaRvilalxde vatiRsu. 
\eanum
\numie
\numi{10} \eng{play the game} 
\banum
\alnum{a} niyamagaLanunx pAlisu; niyamagaLige anusAravAgi ADu. 
\alnum{b} (\rUpa)saBayx riVtiyalilx vatiRsu; dhamaRdiMda, nAyxyavAgi naDeduko; gwravAhaRvAgi naDeduko. 
\eanum
\numie
\numi{11} \eng{the game} (\ashi) 
\banum
\alnum{a} sULegArike; vayxBicAra. 
\alnum{b} kaLaLxtana: \eng{on the game} 
\alnum{a} vayxBicAradalilx toDagi. 
\alnum{b} kadiyutatx. 
\eanum
\numie
\enum
\emng

\noindent
\gl{\nuga}
\bmng
\bnum
\num{1} \eng{beat person at his own game} avana paTaTxnenxV hAki, avanadeV taMtarx baLasi avananunx -- mIrisu, soVlisu, keDavu. 
\num{2} \eng{game that two can play} (\sA\ keTaTx vayxvahArakekx parxtiVkAra mADabahudeMba bedarike hAkuvAga) I ATa eraDu kaDeyavarU ADabahudu; itararU adeV riVti mADabahudu, adeV taMtarx baLasabahudu. 
\num{3} \eng{give the game away} udedxVsha bayalu mADu; iMgita biTuTxkoDu. 
\num{4} \eng{have the game in one's hands} gelulxva necicxke hoMdiru; ATavanunx anukUlakekx takakxMte tirugisabalalxvanAgiru; ATada meVle hatoVTi paDediru. 
\num{5} \eng{make game of} hAsayx, tamASe -- mADu. 
\num{6} \eng{not in the game} gelulxva -- saMBavavilalxde, Baravaseyilalxde. 
\num{7} \eng{on the game} (\birx) (\ashi) sULegArikeyalilx yA kaLaLxtanadalilx sikikxkoMDiru, toDagiru. 
\num{8} \eng{play person's game} beVrobabxna yoVjaneyanunx anudedxVshitavAgi muMduvarisu. 
\num{9} \eng{the game is up} Iga ATa mugiyitu, keTiTxtu; geluvu sAdhayxvilalx. 
\enum
\emng
\eentry

\bentry
\word[game(2)]{game}
\pron{geVmf}
\gl{\gu}
\bmng
\bnum
\num{1} kALagada huMjadaMtha. 
\num{2} joVrAda; AveVshada. 
\num{3} utAsxhaBarita; hurupuLaLx; mADalu humamxsusxLaLx. 
\num{4} dheYyaRvuLaLx; dheYyaRdiMda sidadhxvAgiruva. 
\enum
\emng

\noindent
\gl{\nuga}
\bmng
 \eng{as game as Ned Kelly} (AseTxrXVliya, \AmA) tuMba dheYyaRvuLaLx; bahaLa dhiVranAda; kececxdeyuLaLx. 
\emng
\eentry

\bentry
\word[game(3)]{game}
\pron{geVmf}
\gl{\sakirx}
\bmng
 jUjADi -- kaLe, hALumADu. 
\emng

\noindent
\gl{\akirx}
\bmng
 duDuDx kaTiTx jUjADu. 
\emng
\eentry

\bentry
\word[game(4)]{game}
\pron{geVmf}
\gl{\gu}
\bmng
\bnum
\num{1} (keY \mo vugaLa \vi) UnavAda; vikalavAda. 
\num{2} (kAlina \vi)kuMTa; kuMTAda. 
\enum
\emng
\eentry

\bentry
\wordnospeech{game act}{game act}
\pron{?}
\gl{\nA}
\bmng
 (\sA\ \bava dalilx) beVTe kAnUnu; SikAri kAyide; maqgabeVTe, saMrakaSxNe, \mo vugaLigAgi mADida kAnUnu. 
\emng
\eentry

\bentry
\word{gamebag}
\pron{geVmfbAYxgf}
\gl{\nA}
\bmng
 SikAri ciVla; beVTeyADida pArxNigaLanunx hAkuva ciVla. 
\emng
\eentry

\bentry
\wordnospeech{game ball}{game ball}
\pron{?}
\gl{\nA}
\bmng
 geVmf bAlu; gelulxva bAlu; Tenisf \mo\ ATagaLalilx gelalxlu koneya oMdeV pAyiMTu uLidiruva saMdaBaR. 
\emng
\eentry

\bentry
\word{gamebook}
\pron{geVmfbukf}
\gl{\nA}
\bmng
beVTepusatxka; SikAripusatxka; beVTedAKale; maqgavanunx beVTeyADida,koMda samaya, sathxLa, \mo vanunx beVTegAranu dAKalumADabahudAda pusatxka. 
\emng
\eentry

\bentry
\word{gamechicken}
\pron{geVmfcikinf}
\gl{\nA}
\bmng
 kALagada mari; koVLi kALagakAkxgi taLi mADida, koVLi \mo vugaLa mari. 
\emng
\eentry

\bentry
\wordnospeech{game chips}{game chips}
\pron{?}
\gl{\nA}
\bmng
 beVTe upepxVri; kADupArxNigaLa mAMsadoDane baDisuva, teLuvAda AlUgaDeDx upepxVri. 
\emng
\eentry

\bentry
\word{gamecock}
\pron{geVmfkAkf}
\gl{\nA}
\bmng
 kALagada huMja; koVLi kALagakAkxgi taLimADida huMja. 
\emng
\eentry

\bentry
\wordnospeech{game egg}{game egg}
\pron{?}
\gl{\nA}
\bmng
 kALagada moTeTx; kALagakAkxgi taLi mADida koVLi \mo vugaLa moTeTx. 
\emng
\eentry

\bentry
\word{gamefowl}
\pron{geVmfphwlf}
\gl{\nA}
\bmng
 kALagada hakikx yA koVLi; koVLi kALagakAkxgi taLimADida hakikx yA koVLi. 
\emng
\eentry

\bentry
\word{gamekeeper}
\pron{geVmfkiVparf}
\gl{\nA}
\bmng
 beVTe kAvalugAra; beVTepAlaka; beVTe pArxNigaLanunx aBivaqdidhxpaDisuvudakUkx, beVTe kaLuvAgadaMte noVDikoLuLxvudakUkx neVmakavAdavanu. 
\emng
\eentry

\bentry
\word{gamelan}
\pron{gAYxmalAYxnf}
\gl{\nA}
\bmng
\bnum
\num{1} AgenxVya ESAyxda, \kanmu\ iMDoneVSayxda, parxdhAnavAgi tALavAdayxgaLanonxLagoMDa vAdayxmeVLa. 
\num{2} I meVLadalilx baLasuva, oMdu bageya seZYlaphoZVnf vAdayx.  \imglink{gamelan-2figure}{\raisebox{-0.20cm}[0pt][0pt]{\pdfimage width 0.7cm height 0.5cm {G_Pictures/gamelan-2.jpg}}} 
\enum
\emng
\eentry

\bentry
\wordnospeech{game law}{game law}
\pron{?}
\gl{\nA}
\bmng
  = \hyperlink{game act}{game act}. 
\emng
\eentry

\bentry
\wordnospeech{game licence}{game licence}
\pron{?}
\gl{\nA}
\bmng
 beVTeya -- apapxNe ciVTi, anumati patarx, paravAnagi; kADumaqgagaLanunx beVTeyADalu, beVTeyalilx kolalxlu koDuva paravAnagi. 
\emng
\eentry

\bentry
\word{gamely}
\pron{geVmfli}
\gl{\kirxvi}
\bmng
\bnum
\num{1} utAsxhaBaritavAgi; hurupiniMda; humamxsisxniMda. 
\num{2} bahu dheYyaRdiMda; kececxdeyiMda. 
\enum
\emng
\eentry

\bentry
\word{game-master}
\pron{geVmfmAsaTxrf}
\gl{\nA}
\bmng
 deYhika shikaSxka; kirxVDA shikaSxka; ATa, vAyxyAma, \mo vanunx kalisuva shikaSxka. 
\emng
\eentry

\bentry
\word{gameness}
\pron{geVmfnisf}
\gl{\nA}
\bmng
\bnum
\num{1} tALemx; sahiSuNxte; sahisuva shakitx. 
\num{2} dheYyaRvaMtike; sAhasa. 
\enum
\emng
\eentry

\bentry
\wordnospeech{game point}{game point}
\pron{?}
\gl{\nA}
\bmng
 geVmf pAyiMTu; ATada gelalxMka; ATavanunx, geVmanunx gelalxlu inonxMdu pAyiMTu beVkAda sithxti. 
\emng
\eentry

\bentry
\word{game-preserver}
\pron{geVmfpirxsaZvaRrf}
\gl{\nA}
\bmng
 beVTepAlaka; beVTe poVSaka; beVTe pArxNiyanunx beLesuva matutx beVTe kAnUnanunx kaTuTxniTATxgi jArigoLisuva jamInAdxra \mo varu. 
\emng
\eentry

\bentry
\word{gamesman}
\pron{geVmfs'manf}
\gl{\nA}
\expl{(\bava\ \eng{gamesmen}).}
\bmng
vijayakalegAra; vijayataMtirx; ATagaLu athavA itara sapxdheRgaLalilx kalegiMta hecAcxgi akarxmaveMdu kareyalAgada mAnasika pariNAmagaLanunxMTumADuva vidhAnagaLiMda gelulxva taMtarxda parxtipAdaka. 
\emng
\eentry

\bentry
\word{gamesmanship}
\pron{geVmfs'manfSipf}
\gl{\nA}
\bmng
 vijayataMtarx; vijayakale; ATagaLu athavA itara sapxdheRgaLalilx kalegiMta hecAcxgi, akarxmaveMdu kareyalAgada vidhAnagaLiMda gelulxva (manoVveYjAcnxnika yA mAnasika pariNAmagaLanunxMTumADuva) kale. 
\emng
\eentry

\bentry
\word{gamesome}
\pron{geVmfsamf}
\gl{\gu}
\bmng
 kirxVDAvinoVdi; ATaguLi; celAlxTada; negedADuva; ulAlxsada. 
\emng
\eentry

\bentry
\word{gamesomely}
\pron{geVmfsamfli}
\gl{\kirxvi}
\bmng
 celAlxTadiMda; negedADutatx; ulAlxsadiMda. 
\emng
\eentry

\bentry
\word{gamesomeness}
\pron{geVmfsamfnisf}
\gl{\nA}
\bmng
 celAlxTa(tana); negedATa; ulAlxsa. 
\emng
\eentry

\bentry
\word{gamester}
\pron{geVmfsaTxrf}
\gl{\nA}
\bmng
 jUjukoVra; jUjugAra; jUjALi. 
\emng
\eentry

\bentry
\wordnospeech{game(s) theory}{game(s) theory}
\pron{?}
\gl{\nA}
\bmng
 kirxVDA sidAdhxMta: 
\banum
\alnum{a} yudadhx, AthiRka vayxvasethx, kushala kirxVDegaLalilxna saMGaSiRgaLa gaNitashAsitxrXVya vishelxVSaNe. 
\alnum{b} nidiRSaTx parimitigaLige oLapaTaTx parisithxtigaLalilx, lABagaLanunx adhikagoLisi naSaTxgaLanunx kaniSaThxgoLisuva niNaRya tegedukoLaLxbeVkAdaMtha samaseyxgaLanunx bageharisuvalilx vAyxpakavAgi baLasalAgiruva, taMtarxgaLanunx kurita gaNitashAsitxrXVya sidAdhxMta. 
\eanum
\emng
\eentry

\bentry
\word{gametangium}
\pron{gAYxmiTAYxMjiamf}
\gl{\nA}
\expl{(\bava\ \eng{gametangia} \ucAcx\ gAYxmiTAYxMjia).}
\bmng
(\savi) gamITugaLanunx utapxtitxmADuva aMga. 
\emng
\eentry

\bentry
\word{gamete}
\pron{ga(gAYx)mITf}
\gl{\nA}
\bmng
 (\jiVvi) gamITu; jaMpati; saMtAnoVtapxtitxge heNuNx odagisuva aMDANu matutx gaMDu odagisuva reVtArxNugaLalilx oMdu. 
\emng
\eentry

\bentry
\word{game-tenant}
\pron{geVmfTenaMTf}
\gl{\nA}
\bmng
 beVTe yA mInu gutitxgedAra; oMdu parxdeVshadalilx beVTeyADalu yA mInu hiDiyalu gutitxge paDedavanu. 
\emng
\eentry

\bentry
\word{gametic}
\pron{gameTikf}
\gl{\gu}
\bmng
 (\jiVvi) gamITugaLa yA gamITugaLige saMbaMdhisida. 
\emng
\eentry

\bentry
\word{gameto-}
\pron{gamIToV-}
\gl{\sapUpa}
\bmng
 (\jiVvi) gamIToV-: gamITina eMbathaRda \sapUpa. 
\emng
\eentry

\bentry
\word{gametophyte}
\pron{gamIToVpheYTf}
\gl{\nA}
\bmng
 gamIToV sasayx; oMdu piVLigeyalilx leYMgika avayavagaLidudx muMdina piVLigeyalilx avu ilalxdiruvaMtha sasayx. 
\emng
\eentry

\bentry
\word{gametophytic}
\pron{gamIToVpheYTikf}
\gl{\gu}
\bmng
 gamIToV sasayxda. 
\emng
\eentry

\bentry
\word{game-warden}
\pron{geVmfvADaRnf}
\gl{\nA}
\bmng
 beVTepAla(ka); beVTe kAnUnanunx pAlisuvaMte noVDikoLuLxvava. 
\emng
\eentry

\bentry
\word{gamily}
\pron{geVmili}
\gl{\kirxvi}
\bmng
\bnum
\num{1} utAsxhadiMda; hurupiniMda; humamxsisxniMda. 
\num{2} kececxdeyiMda; dheYyaRdiMda. 
\num{3} (\ame) nAcikegeVDAguvaMte. 
\enum
\emng
\eentry

\bentry
\word{gamin}
\pron{gAYxmi(mAyx)nf}
\gl{\nA}
\bmng
\bnum
\num{1} biVdi huDuga; biVdi aleyuva huDuga; tirukuLi. 
\num{2} taleharaTe huDuga; sokikxna huDuga. 
\enum
\emng
\eentry

\bentry
\word{gamine}
\pron{gAYxmInf}
\gl{\nA}
\bmng
\bnum
\num{1} biVdi huDugi; biVdi biVdi aleyuva huDugi. 
\num{2} mudAdxda, puTaTx tuMTa huDugi. 
\enum
\emng
\eentry

\bentry
\word{gaminess}
\pron{geVminisf}
\gl{\nA}
\bmng
\bnum
\num{1} beVTe nibiDate; beVTeya pArxNi pakiSxgaLiMda tuMbiruvike. 
\num{2} utAsxha; hurupu. 
\num{3} kecucx; dheYyaR. 
\num{4} (\ame) nAcikegeVDu. 
\enum
\emng
\eentry

\bentry
\word{gaming-house}
\pron{geVmiMgfhwsf}
\gl{\nA}
\bmng
 jUju kaTeTx; jUjina mane. 
\emng
\eentry

\bentry
\word{gaming-table}
\pron{geVmiMgfTeVbalf}
\gl{\nA}
\bmng
 jUjATada meVju. 
\emng
\eentry

\bentry
\word{gamma}
\pron{gAYxma}
\gl{\nA}
\bmng
\bnum
\num{1} gAYxma; (gaNanAnukarxmadalilx kelavu veVLe \eng{3} matutx \eng{$\subset$} gaLige badalAgi baLasuva) girxVkf vaNaRmAleya mUraneya akaSxra (\eng{$\Gamma, \gamma$}). 
\num{2} (\Kavi) nakaSxtarxpuMjadalilx mUraneya ujavxla tAre. 
\num{3} pariVkeSxyalilx mUraneya dajeRya gurutu: \eng{gamma minus} mUraneya dajeRgiMta savxlapx kiVLu. \eng{gamma plus} mUraneV dajeRgiMta savxlapx utatxma. 
\enum
\emng
\eentry

\bentry
\word{gammadion}
\pron{gameVDianf}
\gl{\nA}
\bmng
 nAlukx girxVkf gAYxma \eng{$(\Gamma)$} doDaDx akaSxragaLanunx seVrisi, savxsitxka AkAradalolxV girxVkf shilubeyAkAradalolxV mADida AlaMkArika citarx. 
\emng
\eentry

\bentry
\wordnospeech{gamma radiation}{gamma radiation}
\pron{?}
\gl{\nA}
\bmng
  = \hyperlink{gamma rays}{gamma rays}. 
\emng
\eentry

\bentry
\wordnospeech{gamma rays}{gamma rays}
\pron{?}
\gl{\nA}
\bmng
 gAYxma kiraNgaLu; vikiraNapaTu dhAtugaLu horasUsuva, ekfsx kiraNagaLa aleyudadxkikxMta kaDime aleyudadxda viduyxtAkxMta alegaLu. 
\emng
\eentry

\bentry
\word{gammer}
\pron{gAYxmarf}
\gl{\nA}
\bmng
 (\pArxparx) (haLiLxya mAtiyalilx) ajijx; muduki. 
\emng
\eentry

\bentry
\word[gammon(1)]{gammon}
\pron{gAYxmanf}
\gl{\nA}
\bmng
\bnum
\num{1} pakakxda hAgU keLatoDeya BAgada haMdimAMsa: \eng{gammon of bacon} haMdimAMsa KaMDa. 
\num{2} upUpxrisida yA hogeyalilxTuTx saMsakxrisida haMdimAMsa. 
\enum
\emng
\eentry

\bentry
\word[gammon(2)]{gammon}
\pron{gAYxmanf}
\gl{\sakirx}
\bmng
 (haMdimAMsavanunx) upupxhAki pariSakxrisu, saMsakxrisu. 
\emng
\eentry

\bentry
\word[gammon(3)]{gammon}
\pron{gAYxmanf}
\gl{\nA}
\bmng
 bAYxkfgAYxmanf eMba ATadalilx (eraDu ATa gedudx) paDeda pUtiRjaya. 
\emng
\eentry

\bentry
\word[gammon(4)]{gammon}
\pron{gAYxmanf}
\gl{\sakirx}
\bmng
 (bAYxkfgAYxmanf ATadalilx eraDu ATagaLanunx gedudx edurALiyanunx) soVlisu. 
\emng
\eentry

\bentry
\word[gammon(5)]{gammon}
\pron{gAYxmanf}
\gl{\nA}
\bmng
 Thakukx; moVsa; vaMcane. 
\emng
\eentry

\bentry
\word[gammon(6)]{gammon}
\pron{gAYxmanf}
\gl{\sakirx}
\bmng
\bnum
\num{1} naMbike toVruvaMte mAtanADu, naTisu; naMbisuvaMte mAtanADu. 
\num{2} moVsamADu; vaMcisu. 
\enum
\emng

\noindent
\gl{\akirx}
\bmng
 naTanemADu; soVguhAku. 
\emng
\eentry

\bentry
\word[gammon(7)]{gammon}
\pron{gAYxmanf}
\gl{\sakirx}
\bmng
 (\nw) haDagina mUtiya toleyanunx haDagina muMBAgakekx hagagxgaLiMda bigi. 
\emng
\eentry

\bentry
\word[gammon(8)]{gammon}
\pron{gAYxmanf}
\gl{\nA}
\bmng
\bnum
\num{1} haDagina mUtiya toleyanunx haDagina muMgoVTige hagagxgaLiMda bigiyuvudu. 
\num{2} hAge bigida hagagx. 
\enum
\emng
\eentry

\bentry
\word[gammon(9)]{gammon}
\pron{gAYxmanf}
\gl{\BAavayx}
\bmng
 aviveVka! asaMbadadhx! 
\emng
\eentry

\bentry
\word{gammoning}
\pron{gAYxmaniMgf}
\gl{\nA}
\bmng
  = \hyperlink{gammon(8)}{$^8$gammon}. 
\emng
\eentry

\bentry
\word{gammy}
\pron{gAYxmi}
\gl{\gu}
\expl{(\tara\ \eng{gammier,} \tama\ \eng{gammiest}).}
\bmng
 (\birx) (\ashi)(\kanmu\ kAlina \vi) kuMTAda; shAshavxtavAgi UnavAda. 
\emng
\eentry

\bentry
\word{gamo-}
\pron{gAYxma-(\mo)-}
\gl{\sapUpa}
\bmng
 saMyukatx, milita, saMyoVgagoMDa, seVrikoMDiruva eMba athaRgaLalilx baLasuva \sapUpa. 
\emng
\eentry

\bentry
\word{gamogenesis}
\pron{gAYxmajenisisf}
\gl{\nA}
\bmng
 (\jiVvi) leMgika saMtAnoVtapxtitx; heNuNx matutx gaMDugaLu odagisuva gamITugaLa milanadiMda saMtAnoVtapxtitxyAguva vidhAna. 
\emng
\eentry

\bentry
\word{gamopetalous}
\pron{gAyxmapeTalasf}
\gl{\gu}
\bmng
 (\savi) (hUvina \vi) saMyukatx daLada; daLagaLu parasapxra seVrikoMDiruva.  \imglink{gamopetalousfigure}{\raisebox{-0.15cm}[0pt][0pt]{\pdfimage width 0.7cm height 0.7cm {G_Pictures/gamopetalous.jpg}}} 
\emng
\eentry

\bentry
\word{gamp}
\pron{gAYxMpf}
\gl{\nA}
\bmng
 (\birx) (\AmA) koDe (\kanmu\ oDoDxDADxgiruva doDaDx) Catirx. 
\emng
\eentry

\bentry
\word{gamut}
\pron{gAYxmaTf}
\gl{\nA}
\bmng
\bnum
\num{1} (\saM) madhayxyugada savxrasherxVNiyalilx modala savxra. 
\num{2} (\saM) (aMgiVkaqtavAgiruva elalx savxragaLa) pUNaR savxrasherxVNi; savxragArxma.
\num{3} (\saM) variVya savxrASaTxka sherxVNi. 
\num{4} (oMdu kAlada yA janara) aMgiVkaqta savxragArxma; purasakxqqta savxrasherxVNi. 
\num{5} (shAriVrada yA vAdayxda) dhavxnivAyxpitx; nAdada haravu. 
\num{6} (yAvudeV viSayada, vasutxvina) pUNaRvAyxpitx; parxsara; haravu: \eng{the whole gamut of crime} sakala vidhavAda pAtakakaqtayxgaLu. \eng{run up and down the gamut} amUlAgarxvAgi noVDibiDu; buDadiMda tudiyavaregU noVDibiDu. 
\enum
\emng
\eentry

\bentry
\word{gamy}
\pron{geVmi}
\gl{\gu}
\bmng
\bnum
\num{1} beVTeya pArxNipakiSxgaLiMda tuMbida. 
\num{2} (\viparx) = \hyperlink{game(2)}{$^2$game}. 
\num{3} (beVTeya pArxNiya mAMsada \vi) haLasalu vAsaneya yA ruciya; haLasalAguvavaregU beVyisade hAgeyeV iTaTx beVTe mAMsada vAsane yA ruci hiDida, hatitxda. 
\num{4} (\ame) nAcikegeVDina; gulilxna. 
\num{5} (\ame) acacxriya; AshacxyaRkaravAda; mahadAshacxyaR huTiTxsuva. 
\enum
\emng
\eentry

\bentry
\word[gander(1)]{gander}
\pron{gAYxMDarf}
\gl{\nA}
\bmng
\bnum
\num{1} gaMDubAtu; gaMDuvaraTe. 
\num{2} daDaDx; heDaDx; pedadx; ajacnx; gAMpa. 
\num{3} (\ashi) daqSiTx; noVTa. 
\enum
\emng

\noindent
\gl{\nuga}
\bmng
\hypertarget{gander(1) nuga}{} \eng{sauce for the goose is sause for the gander} ibabxrigU oMdeV nAyxya; oMdakekx yA obabxnige anavxyisuva sUtarx, tatatxvX, \mo vu inonxMdakUkx yA inonxbabxnigU anavxyisabeVku. 
\emng
\eentry

\bentry
\word[gander(2)]{gander}
\pron{gAYxMDarf}
\gl{\akirx}
\bmng
 (\ashi) noVDu; kaNuNxhAku; daqSiTx hAyisu. 
\emng
\eentry

\bentry
\word[gang(1)]{gang}
\pron{gAYxMgf}
\gl{\nA}
\bmng
\bnum
\num{1} (kelasagArara, gulAmara yA keYdigaLa) gAYxMgu; taMDa; guMpu; kUTa. 
\num{2} gAYxMgu; paTAlaM; kaLaLxtana, daroVDe, \mo\ akaqtayxgaLigAgi oMdugUDuva janara taMDa. 
\num{3} EkakAladalilx kelasa mADuvaMte aLavaDisida hatAyxra, upakaraNa, \mo vugaLa taMDa. 
\enum
\emng
\eentry

\bentry
\word[gang(2)]{gang}
\pron{gAYxMgf}
\gl{\sakirx}
\bmng
oTuTxseVri, oTATxgi hoMdikoMDu -- kelasamADu, mADuvaMte (upakaraNa \mo vugaLanunx) aLavaDisu. 
\emng

\noindent
\gl{\akirx}
\bmng
\bnum
\num{1} seVri, oTiTxge, oMdugUDi -- kelasa mADu. 
\hypertarget{gang(2)akirx}{} 
\num{2} (obabxnoDane yA itararoDane) oTATxgu; oTuTxgUDu; oMdugUDu; jote seVru. 
\enum
\emng

\noindent
\gl{\pagu}
\bmng
\bnum
\num{1} \eng{gang up on} (\AmA) (obabxna virudadhxvAgi) guMpugUDu; oMdugUDu; oTiTxge seVriko; oTATxgu. 
\num{2} \eng{gang up with} = \hyperlink{gang(2)akirx}{$^2$gang ?akirx? \((2)\)}. 
\enum
\emng
\eentry

\bentry
\word[gang(3)]{gang}
\pron{gAYxMgf}
\gl{\akirx}
\bmng
 (sAkxTalxMDina \parx) hoVgu; naDe. 
\emng

\noindent
\gl{\nuga}
\bmng
 \eng{gang agley} (yoVjane, udedxVsha, \mo vu) dAritapupx; eNisidaMte naDeyade hoVgu; tapApxgu. 
\emng
\eentry

\bentry
\word{gang-bang}
\pron{gAYxMgfbAYxMgf}
\gl{\nA}
\bmng
 (\ashi) sAmUhika saMBoVga; sAmUhika atAyxcAra; guMpu meYthuna; oMdu heMgasanunx aneVka gaMDasaru (obabxrAda meVle obabxraMte) saMBoVgisuvudu. 
\emng
\eentry

\bentry
\word{gangboard}
\pron{gAYxMgfboVDfR}
\gl{\nA}
\bmng
 (doVNiyoLakekx hoVgalu yA adariMda horakekx baralu, jAradaMte paTiTxgaLanunx hoDedu mADida) hAdi halage, naDehalage. 
\emng
\eentry

\bentry
\word{gange}
\pron{gAYxMjf}
\gl{\sakirx}
\bmng
 (gALada kokekxyanunx, adara dArada BAgavanunx) sUkaSxmX taMtiyiMda BadarxpaDisu. 
\emng
\eentry

\bentry
\word{ganger}
\pron{gAYxMgarf}
\gl{\nA}
\bmng
 (\birx) muKayx kelasagAra; hiriya kelasagAra; kelasagArara taMDada nAyaka, muKayxsathx. 
\emng
\eentry

\bentry
\word{Gangetic}
\pron{gAYxMjeTikf}
\gl{\gu}
\bmng
 gaMgAnadiya. 
\emng
\eentry

\bentry
\word{gang-gang}
\pron{gAYxMgfgAYxMgf}
\gl{\nA}
\bmng
 (\AseTxrXV) bUdu baNaNxda puTaTx juTuTxgiLi (idara gaMDu giLiya juTuTx keMpagirutatxde). 
\emng
\eentry

\bentry
\word{ganging}
\pron{gAYxMgiMgf}
\gl{\nA}
\bmng
 (gALada kokekxyanunx, adara dArada BAgavanunx) sUkaSxmX taMtiyiMda BadarxpaDisuvudu. 
\emng
\eentry

\bentry
\word{gangland}
\pron{gAMgflAYxMDf}
\gl{\nA}
\bmng
\bnum
\num{1} kaLaLxkAkara rAjayx; kaLaLxkAkaru, daroVDegAraru, \mo vara parxdeVsha. 
\num{2} kaLaLxkAkara guMpu. 
\enum
\emng
\eentry

\bentry
\word{gangle}
\pron{gAYxMgflf}
\gl{\akirx}
\bmng
 aSATxvakarxvAgi, avalakaSxNavAgi calisu. 
\emng
\eentry

\bentry
\word{ganglia}
\pron{gAYxMgilxa}
\gl{\nA}
\bmng
 \eng{ganglion} padada \bava\ \rUpa 
\emng
\eentry

\bentry
\word{gangliated}
\pron{gAYxMgilxETiDf}
\gl{\gu}
\bmng
 gAyxMgilxyanfyukatx; gAyxMgilxyanfgaLiruva. 
\emng
\eentry

\bentry
\word{gangliform}
\pron{gAYxMgilxphAmfR}
\gl{\gu}
\bmng
 gAyxMgilxyanf -- rupada, Akaqtiya. 
\emng
\eentry

\bentry
\word{gangling}
\pron{gAYxMgilxMgf}
\gl{\gu}
\bmng
 (vayxkitxya \vi) narapeVtala; aMdacaMdavilalxda, asatxvayxsatx meYkaTiTxna, teLaLxneya deVha matutx udadxneya kAlugaLuLaLx. 
\emng
\eentry

\bentry
\word{ganglion}
\pron{gAYxMgilxanf}
\gl{\nA}
\expl{(\bava\ \eng{ganglions} yA \eng{ganglia}).}
\bmng
\bnum
\num{1} gAYxMgilxyanf; naragarxMthi; naragaMTu; sutatxlU narataMtugaLu horaDuvaMtha, narada meVlina gaMTu. 
\num{2} gAYxMgilxyanf; benunxhuriyalilx yA miduLinalilx iruva bUdu padAthaR. 
\num{3} (\roVshA) sAnxyurajujxvina hodikeya meVlina koVsha, koVSaThx, ciVla. 
\num{4} (\rUpa) (shakitx, caTuvaTike, Asakitx, \mo vugaLa) keVMdarx. 
\enum
\emng
\eentry

\bentry
\word{ganglionate}
\pron{gAYxMgilxaneVTf}
\gl{\gu}
\bmng
  = \hyperlink{gangliated}{gangliated}. 
\emng
\eentry

\bentry
\word{ganglionated}
\pron{gAYxMgilxaneVTiDf}
\gl{\gu}
\bmng
  = \hyperlink{gangliated}{gangliated}. 
\emng
\eentry

\bentry
\word{ganglionic}
\pron{gAYxMgilxanikf}
\gl{\gu}
\bmng
 gAyxMgilxyaninxna; gAYxMgilxyanunxLaLx; gAyxMgilxyanf meVle pariNAma biVruva. 
\emng
\eentry

\bentry
\word{gangly}
\pron{gAYxMgilx}
\gl{\gu}
\bmng
  = \hyperlink{gangling}{gangling}. 
\emng
\eentry

\bentry
\word{gangplank}
\pron{gAYxMgfpAlxYxMkf}
\gl{\nA}
\bmng
  = \hyperlink{gangboard}{gangboard}. 
\emng
\eentry

\bentry
\wordnospeech{gang rape}{gang rape}
\pron{?}
\gl{\nA}
\bmng
 sAmUhika atAyxcAra; heMgasina yA heMgasara meVle aneVka gaMDasaru obabxrAda meVlobabxru balavaMtavAgi saMBoVga mADuvudu. 
\emng
\eentry

\bentry
\word[gangrene(1)]{gangrene}
\pron{gAYxMgirxVnf}
\gl{\nA}
\bmng
\bnum
\num{1} gAyxMgirxVnu; koVtha; deVhada yAvudeV BAgakekx rakatxda sarabarAju niMta kAraNa Utaka nijiVRvavAgi \sA\ koLetu hoVguvudu. 
\num{2} (\rUpa)niVtiBarxSaTxte. 
\enum
\emng
\eentry

\bentry
\word[gangrene(2)]{gangrene}
\pron{gAYxMgirxVnf}
\gl{\sakirx}
\bmng
 koVtha uMTumADu; gAYxMgirxVnf uMTumADu. 
\emng

\noindent
\gl{\akirx}
\bmng
 koVtha uMTAgu; gAYxMgirxVnf Agu. 
\emng
\eentry

\bentry
\word{gangrenous}
\pron{gAYxMgirxnasf}
\gl{\gu}
\bmng
\bnum
\num{1} koVthadaMtha; gAyxMgirxVnfnaMtha. 
\num{2} koVtha, gAYxMgirxVnf Ada. 
\enum
\emng
\eentry

\bentry
\word{gangster}
\pron{gAYxMgfsaTxrf}
\gl{\nA}
\bmng
 daroVDekoVra; shasatxrXsajijxta daroVDekoVraralilx obabx, \kanmu\ GoVra aparAdhigaLa taMDadavanu. 
\emng
\eentry

\bentry
\word{gangsterism}
\pron{gAYxMgfsaTxrisaZmf}
\gl{\nA}
\bmng
 daroVDekoVratana; kaLaLxkAkara, daroVDekoVra taMDakekx seVridavana duSakxqqtayxgaLu yA vidhAnagaLu. 
\emng
\eentry

\bentry
\word{gangue}
\pron{gAYxMgf}
\gl{\nA}
\bmng
 (\BUvi) aduru kasa; BUmiyalilx doreyuva belebALuva Kanija, loVha,\mo vanAnxvarisiruva kalulx, maNuNx, \mo\ nirupayukatx padAthaR. 
\emng
\eentry

\bentry
\word{gang-up}
\pron{gAYxMgfapf}
\gl{\nA}
\bmng
 (\AmA) (obabxna virudadhxvAgi) guMpugUDike; oTiTxge seVrikoLuLxvudu. 
\emng
\eentry

\bentry
\wordnospeech{gang war}{gang war}
\pron{?}
\gl{\nA}
\bmng
 gAYxMgf kadana; taMDaGaSaRNe; parasapxra veYSamayxviruva kaLaLxru, daroVDekoVraru, ugarxkamiRgaLu, \mo vara eraDu taMDagaLa naDuvaNa hoVrATa, kAdATa. 
\emng
\eentry

\bentry
\word[gangway(1)]{gangway}
\pron{gAYxMgfveV}
\gl{\nA}
\bmng
 gAYxMgfveV; naDuhAdi: 
\banum
\alnum{a} (\kanmu\ piVThapaMkitxgaLa madheyx tirugADalu biTiTxruva) naDudAri; ONi. 
\alnum{b} birxTiSf pAliRmeMTinalilx parxtiyoMdu pakaSxda parxBAviV sadasayxranUnx hiMbeMcigaranUnx parxteyxVkisuvaMte piVThapaMkitxgaLa naDuve aDaDxlAgi hAduhoVguva dAri. 
\alnum{c} haDagina munanxTaTxkUkx meVlaTaTxkUkx naDuve iruva veVdikeyaMtha etatxrada hAdi. 
\alnum{d} haDagina kaTakaTegaLa naDuveyiMda haDaganunx hatatxlu yA adariMda iLidu baralu anukUlavAguvaMte hAkida halagehAdi, naDuseVtuve, \mo vu. 
\alnum{e} seVtuve; haDaginiMda daMDege hoVgalu kaTaTxDa nimARNavAgutitxruva jAga \mo vugaLalilx ODADalu mADiruva seVtuveyaMtha hAdi. 
\eanum
\emng

\noindent
\gl{\pagu}
\bmng
\bnum
\num{1} \eng{above gangway} (sadasayxra \vi) pakaSxda adhikaqta dhoVraNege nikaTavAgiruva, hatitxrada. 
\num{2} \eng{below gangway} (sadasayxra \vi) pakaSxda adhikaqta dhoVraNege nikaTavAgirada, dUraviruva. 
\enum
\emng
\eentry

\bentry
\word[gangway(2)]{gangway}
\pron{gAYxMgfveV}
\gl{\BAavayx}
\bmng
 (dayaviTuTx) dAribiDi! 
\emng
\eentry

\bentry
\word{ganister}
\pron{gAYxnisaTxrf}
\gl{\nA}
\bmng
 (\BUvi) gAYxnisaTxrf; kulumegaLa asatxrigAgi baLasuva, neYsagiRkavAgi doreyuva jeVDimishirxta kalulx. 
\emng
\eentry

\bentry
\word{ganja}
\pron{gAYxMja}
\gl{\nA}
\bmng
 gAMjA (giDa yA sopupx). 
\emng
\eentry

\bentry
\word{gannet}
\pron{gAYxniTf}
\gl{\nA}
\bmng
\bnum
\num{1} (\jiVvi) gAyxniTf; `sUla' vaMshakekx seVrida, jAlapAdada,oMdu doDaDx kaDala hakikx. 
\num{2} (\ashi) durAsheyava. 
\enum
\emng
\eentry

\bentry
\word{gannetry}
\pron{gAYxniTirx}
\gl{\nA}
\bmng
 gAYxniTf gUDu; gAYxniTf hakikxgaLu vAsisuva matutx saMtAnABivaqdidhx mADuva sathxLa. 
\emng
\eentry

\bentry
\word[ganoid(1)]{ganoid}
\pron{gAYxnAyfDx}
\gl{\gu}
\bmng
\bnum
\num{1} (mInina shalakxda \vi) enAmalf yA gAjuleVpa mADidaMtiruva;nuNupAgi, gaDusAgi -- hoLeyuva. 
\num{2} gAjuleVpadaMtha shalakxgaLiruva. 
\enum
\emng
\eentry

\bentry
\word[ganoid(2)]{ganoid}
\pron{gAYxnAyfDx}
\gl{\nA}
\bmng
 gAYxnAyfDx; ganAyiDeV upavaMshakekx seVrida, gaDusAgiyU nuNupAgiyU hoLeyutatxlU iruva shalakxgaLuLaLx mInu. 
\emng
\eentry

\bentry
\word{gantlet}
\pron{gAYxMTilxTf}
\gl{\nA}
\bmng
 (\ame)  = \hyperlink{gauntlet(2)}{$^2$gauntlet}. 
\emng
\eentry

\bentry
\word{gantry}
\pron{gAYxMTirx}
\gl{\nA}
\expl{(\bava\ \eng{gantries}).}
\bmng
\bnum
\num{1} piVpAyi piVTha; piVpAyi iDuva marada nAgAlu piVTha, kAlamxNe. 
\num{2} (saMcAri kerxVnf yaMtarx, reYlevx siganxlulx, rAkeTf uDAvaNeya sAdhana, \mo vugaLa) AdhAra piVTha. 
\enum
\emng
\eentry

\bentry
\word{Ganymede}
\pron{gAYxnimIDf}
\gl{\nA}
\bmng
\bnum
\num{1} (\hA) pAnaparicAraka; pAnadaMgaDiya mANi; madayxdaMgaDiyalilx pAniVyagaLanunx taMdodagisuva huDuga. 
\num{2} (\Kavi) gAYxnimIDf; baqhasapxti garxhada atayxMta doDaDx upagarxha. 
\num{3} (\girxVpu) gAYxnimIDf; deVvategaLu oliMpasf pavaRtakekx kadodxyadx, alilx tamamx paricArakananAnxgi mADikoMDa obabx TorxVjanf taruNa. 
\enum
\emng
\eentry

\bentry
\word[gaol(1)]{gaol}
\pron{jeVlf}
\gl{\nA}
\bmng
% (\birx) 
\bnum
\num{1} seremane; baMdiVKAne; jeYlu; taraMga; kArAgaqha. 
\num{2} sere; baMdhana; jeYluvAsa; kArAgaqhavAsa. 
\enum
\emng
\eentry

\bentry
\word[gaol(2)]{gaol}
\pron{jeVlf}
\gl{\sakirx}
\bmng
 (\birx) serehAku; baMdhanadalilxDu; jeYlige kaLuhisu; seremanege seVrisu. 
\emng
\eentry

\bentry
\word{gaolbird}
\pron{jeVlfbaDfR}
\gl{\nA}
\bmng
\bnum
\num{1} jeYluhakikx; keYdi; jeYlucALi; cALibidadx aparAdhi, tapipxtasathx, takisxVrudAra. 
\num{2} moVsagAra; Thakakx; dagalAbxji. 
\enum
\emng
\eentry

\bentry
\word{gaolbreak}
\pron{jeVlfberxVkf}
\gl{\nA}
\bmng
 jeYliniMda tapipxsikoLuLxvudu, palAyana. 
\emng
\eentry

\bentry
\wordnospeech{gaol delivery}{gaol delivery}
\pron{?}
\gl{\nA}
\bmng
 jeYlu KulAse; (\kanmu\ nAyxyadashiRya samakaSxma vicAraNeyalilx) vicAraNeyAgabeVkAda keYdigaLanenxlAlx vicAraNe mADi, tapipxtasathxranunxLidu itara elalxranUnx KulAse mADuvudu. 
\emng
\eentry

\bentry
\word{gaoler}
\pron{jeVlarf}
\gl{\nA}
\bmng
 jeYlaru; seremane adhikAri; seremane yA adaralilxya sereyALugaLa javAbAdxriyuLaLx adhikAri. 
\emng
\eentry

\bentry
\wordnospeech{gaol fever}{gaol fever}
\pron{?}
\gl{\nA}
\bmng
 jeYlujavxra; hiMde jeYlugaLalilx tapapxde barutitxdadx, oMdu bageya tiVvarx TeYphAyfDx javxra. 
\emng
\eentry

\bentry
\word{gap}
\pron{gAYxpf}
\gl{\nA}
\bmng
\bnum
\num{1} (goVDe yA beVliyalilxna) kaMDi; terapu; saMdu. 
\num{2} kaNive; girikaMdara; GATi. 
\num{3} terapu; KAli; anukarxmadalilx naDuve biTuTxhoVda jAga, aMtara, yA saMBavisida BaMga. 
\num{4} (aBipArxya, sahAnuBUti, \mo vugaLa naDuvaNa, \sA\ ahitakaravAda) vayxtAyxsa; aMtara: \eng{the gap between ideals and actions} AdashaRgaLigU AcaraNegaLigU naDuvaNa aMtara, vayxtAyxsa. 
\enum
\emng

\noindent
\gl{\pagu}
\bmng
 \eng{bridge, close, fill, stop, supply a gap} korateyanunx tuMbu, BatiRmADu. 
\emng
\eentry

\bentry
\word[gape(1)]{gape}
\pron{geVpf}
\gl{\akirx}
\bmng
\bnum
\num{1} agalavAgi bAyibiDu, tere. 
\num{2} (mutitxna cipupx, gAya, doDaDx biruku, \mo vugaLa \vi) agalavAgi -- tere, terediru, biTiTxru. 
\num{3} siVLu; biri. 
\num{4} bAyi biTuTxkoMDu noVDu. 
\num{5} diTiTxsi noVDu; eveyikakxde noVDu; kutUhaladiMda repepx miTukisade noVDu. 
\num{6} AkaLisu. 
\enum
\emng
\eentry

\bentry
\word[gape(2)]{gape}
\pron{geVpf}
\gl{\nA}
\bmng
\bnum
\num{1} AkaLike; jaqMBaNa. 
\num{2} bAyi biTuTxkoMDu noVDuvudu. 
\num{3} tereda bAyiya yA kokikxna visAtxra. 
\num{4} terediruva kokikxna BAga. 
\num{5} harakugaMDi; biruku; terapu; Cidarx. 
\enum
\emng

\noindent
\gl{\pagu}
\bmng
 \eng{the gapes} hakikxgaLa AkaLike roVga; (hakikxgaLu) bAyi teredukoMDiruva cihenxyuLaLx oMdu roVga. 
\emng
\eentry

\bentry
\word{gaper}
\pron{geVparf}
\gl{\nA}
\bmng
\bnum
\num{1} agalavAgi bAyi biTuTx noVDuvava. 
\num{2} kutUhaladiMda diTiTxsi noVDuvava. 
\num{3} AkaLisuvava. 
\num{4} (\jiVvi) agalavAgi bAyibiDuva kelavu bageya pakiSxgaLu. 
\num{5} meYyAsiDeV vaMshakekx seVrida cipupx miVnu. 
\num{6} eraDU kaDe cipupx teredukoMDiruva mAyx kulada oMdu maqdavxMgi. 
\num{7} (\AmA) (kirxkeTf) sulaBavAda kAYxcu. 
\enum
\emng
\eentry

\bentry
\word{gape-seed}
\pron{geVpfsiVDf}
\gl{\nA}
\bmng
 (\hA) 
\bnum
\num{1} diTiTxsi noVDuvudu. 
\num{2} diTiTxsi noVDuvaMtha saMdaBaR. 
\num{3} daqSiTx neTaTx vasutx. 
\enum
\emng
\eentry

\bentry
\word{gapeworm}
\pron{geVpfvamfR}
\gl{\nA}
\bmng
 AkaLike huLu; koVLi \mo vugaLige AkaLike roVga barisuva huLu. 
\emng
\eentry

\bentry
\word{gapped}
\pron{gAyxpfTx}
\gl{\gu}
\bmng
 terapuLaLx; saMdugaLuLaLx; alalxlilx biDuvugaLuLaLx; niraMtaravAgirada; seVrikoMDirada. 
\emng
\eentry

\bentry
\word{gappy}
\pron{gAYxpi}
\gl{\gu}
\bmng
  = \hyperlink{gapped}{gapped}. 
\emng
\eentry

\bentry
\word{gap-toothed}
\pron{gAYxpfTUtfTx}
\gl{\gu}
\bmng
 saMduhalilxna; halulxsaMduLaLx; halulxgaLa naDuve aMtaraviruva. 
\emng
\eentry

\bentry
\word{gar}
\pron{gArf}
\gl{\nA}
\bmng
  = \hyperlink{garfish}{garfish}. 
\emng
\eentry

\bentry
\word[garage(1)]{garage}
\pron{gAYxrA(ri)jf}
\gl{\nA}
\bmng
 gAYxreVju: 
\banum
\alnum{a} moVTAru -- KAne, mane; (\kanmu\ moVTAru) vAhanagaLanunx biTiTxruva, ripeVri mADuva yA mAruva -- kaTaTxDa, sathxLa, SeDuDx. 
\alnum{b} peTorxVlf \mo vanunx mAruva aMgaDi. 
\eanum
\emng
\eentry

\bentry
\word[garage(2)]{garage}
\pron{gAYxrA(ri)jf}
\gl{\sakirx}
\bmng
 (moVTAru baMDiyanunx) gAyxreVjinalilx biDu, nililxsu. 
\emng
\eentry

\bentry
\word[garb(1)]{garb}
\pron{gAbfR}
\gl{\nA}
\bmng
\bnum
\num{1} (\kanmu\ visheVSa riVtiya) uDupu; uDige. 
\num{2} (obabxna) veVSa; uDigeya riVti. 
\enum
\emng
\eentry

\bentry
\word[garb(2)]{garb}
\pron{gAbfR}
\gl{\sakirx}
\bmng
\bnum
\num{1} (\sA\ \BUkaq dalilx) uDupu toDisu. 
\num{2} (\AtAmx) uDupu -- dharisu, hAkiko, toTuTxko. 
\num{3} (obabxnige) (\kanmu\ veYshiSaTxyXda) uDupu hAku, toDisu. 
\enum
\emng
\eentry

\bentry
\word{garbage}
\pron{gAbiRjf}
\gl{\nA}
\bmng
\bnum
\num{1} nirupayukatx AhAra padAthaR; tinanxlu yoVgayxvalalxveMdu bisuTa pArxNiya (karuLu \mo) aMgagaLu matutx tarakAriya BAgagaLu. 
\num{2} kasa; koLe; kacaDa. 
\num{3} holasu garxMtha; bUsAkaqti; kelasakekx bArada kaqti, pusatxka, \mo vu: \eng{the garbage that passes for art} kaleya hesaralilx sAguva holasu kaqtigaLu. 
\enum
\emng
\eentry

\bentry
\wordnospeech{garbage can}{garbage can}
\pron{?}
\gl{\nA}
\bmng
 (\kanmu\ aDigemaneya) kasada -- toTiTx yA buTiTx. 
\emng
\eentry

\bentry
\word{garble}
\pron{gAbfRlf}
\gl{\sakirx}
\bmng
\bnum
\num{1} (\viparx) atuyxtatxmavAdudanunx -- Arisu, Ayu, cunAyisu; elalxkikxMta oLeLxyadanunx tegeduko. 
\num{2} (viSayagaLu, heVLikegaLu, \mo vugaLiMda tanage beVkAdudanunx mAtarx, \sA\ durudedxVshadiMda) Ayekx mADu; etitx heVLu. 
\num{3} tapupx aBipArxya koDalu (heVLikeyalilx) kelavu BAga biTuTx -- udAharisu, ulelxVKisu. 
\num{4} (EnoMdU durudedxVshavilalxde heVLike, vAsatxvAMsha, \mo vanunx) tirucu; tiruci heVLu; ABAsa mADi heVLu; asatxvayxsatxgoLisu; goMdalagoLisu; goMdalavAguvaMte nirUpisu. 
\enum
\emng
\eentry

\bentry
\word{garboard}
\pron{gAbaRDfR}
\gl{\nA}
\bmng
\bnum
\num{1} kUDuhalage; haDagina taLada meVle hAsiruva halagegaLa sAlinalilx modalaneyadu. 
\num{2} kabibxNada haDaginalilx hAkiruva iMthaveV tagaDugaLu, PalakagaLu. 
\enum
\emng
\eentry

\bentry
\wordnospeech{garboard strake}{garboard strake}
\pron{?}
\gl{\nA}
\bmng
  = \hyperlink{garboard}{garboard}. 
\emng
\eentry

\bentry
\wordf{garcon}
\pron{gArfsAknf}
\gl{\nA}
\expl{\F\ }
\bmng
pherxMcf hoVTelina mANi, paricAraka. 
\emng
\eentry

\bentry
\word[garden(1)]{garden}
\pron{gADaRnf}
\gl{\nA}
\bmng
\bnum
\num{1} (hUvu, haNuNx, kAyipalayxgaLu beLeyuva) toVTa. 
\num{2} (sAvaRjanika vihArakAkxgi mADiruva) vana; upavana; udAyxna. 
\num{3} visheVSa PalavatAtxda parxdeVsha; pArxMta: \eng{the garden of England} iMgelxMDina keMTf pArxMta, evashAMna kaNive \mo\ parxdeVshagaLu. 
\num{4} (\sA\ \bava dalilx hesarina muMde) (biVdi, Uru, \mo vugaLa hesarige \saupa vAgi baMdAga) biVdi, cwka, \mo vu; toVTaveMdu hesariTaTx manegaLa taMDa, sherxVNi: \eng{Onslow Gardens, Spring Gardens.} 
\num{5} AhAra pAniVyagaLanunx odagisuva vayxvasethx iruva upavana: \eng{beer garden} biyarf udAyxna. \eng{tea garden} TiV udAyxna. 
\num{6} (\ame) doDaDx sAvaRjanika saBAMgaNa. 
\enum
\emng

\noindent
\gl{\pagu}
\bmng
\bnum
\num{1} \eng{common or garden} (\AmA) sAdhAraNavAda; sAmAnayxvAda. 
\num{2} \eng{the Garden} epikUyxriyanf tatatxvX; epikUyxrasf eMba \kirxpU\ \eng{3}neV shatamAnada girxVkf tatatxvXjAcnxni (udAyxnadalilx boVdhe mADutitxdadx kAraNadiMda) boVdhisida mata, sidAdhxMta. 
\enum
\emng

\noindent
\gl{\nuga}
\bmng
\bnum
\num{1} \eng{everything in the garden is lovely} (\AmA) elalxvU cenAnxgide, sariyAgide: \eng{But not everything in the garden was lovely and the public, in particular, was not sure what the higher education was for} Adare, elalxvU sariyAgiralilalx; muKayxvAgi sAvaRjanikarige ucacx shikaSxNadiMda Enu parxyoVjana eMbudu KAtirxyAgiralilalx. 
\num{2} \eng{lead up the garden (path)} (\AmA) dAri tapipxsu; tapupxdArige eLe; keTaTx hAdige seLe. 
\enum
\emng
\eentry

\bentry
\word[garden(2)]{garden}
\pron{gADaRnf}
\gl{\gu}
\bmng
\bnum
\num{1} kaqSimADida; neYsagiRkavAgi beLedirada: \eng{garden plants} udAyxna sasayxgaLu; toVTadalilx (beLesida) giDagaLu. 
\num{2} (saveRV) sAmAnayxvAda; sAdhAraNavAda. 
\num{3} toVTavAsi; udAyxnavAsi; toVTadalilx vAsisuva, jiVvisuva: \eng{garden-spider} toVTada jeVDa. 
\enum
\emng
\eentry

\bentry
\word[garden(3)]{garden}
\pron{gADaRnf}
\gl{\akirx}
\bmng
 toVTa mADu; toVTada yA toVTadalilx kelasa mADu; toVTagAranAgi kelasa mADu. 
\emng
\eentry

\bentry
\wordnospeech{garden centre}{garden centre}
\pron{?}
\gl{\nA}
\bmng
 udAyxna keVMdarx; toVTadaMgaDi; toVTa mADalu beVkAda upakaraNagaLu, giDagaLu, \mo vanunx mArATa mADuva sathxLa. 
\emng
\eentry

\bentry
\wordnospeech{garden chair}{garden chair}
\pron{?}
\gl{\nA}
\bmng
toVTada kuciR; udAyxna kuciR; toVTadalilx kuLitukoLaLxlu baLasuva kuciR. 
\emng
\eentry

\bentry
\wordnospeech{garden city}{garden city}
\pron{?}
\gl{\nA}
\bmng
 udAyxnanagara; hululx meYdAnagaLu, upavanagaLu. \mo vanonxLagoMDu AkaSaRkavAgi, vayxvasithxtavAgi, visAtxravAgi kaTiTxsida (keYgArikeya yA itara) paTaTxNa. 
\emng
\eentry

\bentry
\word{gardened}
\pron{gADaRnfDx}
\gl{\gu}
\bmng
\bnum
\num{1} udAyxniVkaqta; toVTadaMte, toVTavAgi -- mADida. 
\num{2} udAyxnayukatx; toVTavuLaLx; toVTagaLuLaLx. 
\enum
\emng
\eentry

\bentry
\word{gardener}
\pron{gADaRnarf}
\gl{\nA}
\bmng
\bnum
\num{1} toVTagAra; mAli; toVTiga; toVTa mADuvavanu. 
\num{2} udAyxnapAla(ka); toVTa noVDikoLuLxvavanu; toVTada ALu. 
\hypertarget{gardener(3)}{} 
\num{3} virAma sikAkxga toVTada kelasa mADuvava. 
\enum
\emng

\noindent
\gl{\pagu}
\bmng
 \eng{jobbing gardener} = \hyperlink{gardener(3)}{gardener (3)}. 
\emng
\eentry

\bentry
\word{gardener-bird}
\pron{gADaRnarfbaDfR}
\gl{\nA}
\bmng
 mANi hakikx; toVTagAra hakikx; latAkuMjada muMde pAciya toVTa nimiRsuva oMdu bageya kuMjavakikx. 
\emng
\eentry

\bentry
\wordnospeech{gardener's garters}{gardener's garters}
\pron{?}
\gl{\nA}
\bmng
= \hyperref{kandict_r.pdf}{R}{ribbon-grass}{ribbon-grass}. 
\emng
\eentry

\bentry
\word{gardenesque}
\pron{gADaRnesfkx}
\gl{\gu}
\bmng
 udAyxna rUpada, riVtiya; toVTadaMte kANuva; toVTada rUpada; toVTada riVtiyalilxruva: \eng{to give a gardenesque appearance to a slope} iLijArige toVTada rUpaviDu. 
\emng
\eentry

\bentry
\word{garden-frame} 
\pron{gADaRnfpherxVmf} 
\gl{\nA}
\expl{}
\bmng
 toVTada peTiTxge; sasayxda peTiTxge; saNaNx sasayxgaLanUnx bitatxneyanUnx oMdu nidiRSaTx uSANxMshadalilxTuTx beLeyuvaMte mADuva gAjina peTiTxge, cwkaTuTx. 
\emng
\eentry

\bentry
\word{garden-glass}
\pron{gADaRnfgAlxsf}
\gl{\nA}
\bmng
 gAjina mucicxke; sasayxda meVle mucucxva gaMTeyAkArada gAjina pAterx. 
\emng
\eentry

\bentry
\word{gardenia}
\pron{gADiRVnia}
\gl{\nA}
\bmng
 gADiRVniya: 
\banum
\alnum{a} suvAsaneyuLaLx biLiya yA haLadiya doDaDx hUgaLanunx biDuva, marada yA podarina oMdu jAti. 
\alnum{b} I jAtiya marada yA podarina hUvu. 
\eanum
\emng
\eentry

\bentry
\word{gardening}
\pron{gADaRniMgf}
\gl{\nA}
\bmng
 toVTagArike; toVTa mADuva yA toVTadalilxna kelasa. 
\emng
\eentry

\bentry
\word{garden-party}
\pron{gADaRnfpATiR}
\gl{\nA}
\bmng
 gADaRnf pATiR; udAyxnakUTa; hululxhAsina meVle yA udAyxnadalilx (tiMDi tiVthaRgaLoMdige) naDeyuva goVSiThx. 
\emng
\eentry

\bentry
\word{garden-plot}
\pron{gADaRnfpAlxTf}
\gl{\nA}
\bmng
 bAgAyata jamInu; toVTada jamInu; toVTa mADiruva parxdeVsha. 
\emng
\eentry

\bentry
\wordnospeech{garden roller}{garden roller}
\pron{?}
\gl{\nA}
\bmng
 toVTada -- roVlarf, uruLugalulx; toVTagArikeyalilx baLasuva uruLugalulx. 
\emng
\eentry

\bentry
\wordnospeech{garden seat}{garden seat}
\pron{?}
\gl{\nA}
\bmng
 udAyxnapiVTha; toVTapiVTha; udAyxnagaLalilx kuLitukoLaLxlu hAkiruva kuciR, beMcu, \mo\ Asana. 
\emng
\eentry

\bentry
\wordnospeech{Garden State}{Garden State}
\pron{?}
\gl{\nA}
\bmng
 (\ame) amerikada saMyukatx saMsAthxnagaLalilxna nUyx jasiR saMsAthxna. 
\emng
\eentry

\bentry
\word{garden-stuff}
\pron{gADaRnfsaTxphf}
\gl{\nA}
\bmng
 toVTada beLe; tarakAri, kAyipalayx, haNuNxhaMpalu, \mo vu. 
\emng
\eentry

\bentry
\wordnospeech{garden suburb}{garden suburb}
\pron{?}
\gl{\nA}
\bmng
 (\birx) udAyxna upanagara; udAyxna nagarada lakaSxNaviruvaMte vayxvasethx mADida upanagara. 
\emng
\eentry

\bentry
\wordnospeech{garden village}{garden village}
\pron{?}
\gl{\nA}
\bmng
 (\birx) udAyxna gArxma; udAyxna nagarada lakaSxNaviruvaMte vayxvasethx mADida haLiLx. 
\emng
\eentry

\bentry
\wordnospeech{garden warbler}{garden warbler}
\pron{?}
\gl{\nA}
\bmng
 oMdu bageya (silivxya boVrinf) kADuhakikx. 
\emng
\eentry

\bentry
\word{garfish}
\pron{gArfphiSf}
\gl{\nA}
\bmng
 (\bava\ adeV). udadxvAda, ITiyaMtha mUtiyuLaLx, oMdu bageya mInu.  \imglink{garfishfigure}{\raisebox{-0.15cm}[0pt][0pt]{\pdfimage width 0.8cm height 0.6cm {G_Pictures/garfish.jpg}}} 
\emng
\eentry

\bentry
\word{garganey}
\pron{gAgaRni}
\gl{\nA}
\expl{(\bava\ \eng{garganeys}).}
\bmng
gAgaRni (bAtu); sihiniVrinalilx vAsisuva oMdu terana bAtukoVLi. 
\emng
\eentry

\bentry
\word{gargantuan}
\pron{gAgAYxRMTuyxanf}
\gl{\gu}
\bmng
 vipariVta; doDaDx; baqhadAkArada; rAkaSxsAkArada; deYtAyxkArada. 
\emng
\eentry

\bentry
\word{garget}
\pron{gAgiRTf}
\gl{\nA}
\bmng
\bnum
\num{1} (hasuvina yA kuriya) kecacxla Uta. 
\num{2} (\ame) = \hyperref{kandict_p.pdf}{P}{pokeweed}{pokeweed}. 
\enum
\emng
\eentry

\bentry
\word[gargle(1)]{gargle}
\pron{gAgaR(gfR)lf}
\gl{\sakirx}
\bmng
 darxva mukakxLisi (bAyi matutx gaMTalanunx) toLe. 
\emng

\noindent
\gl{\akirx}
\bmng
 bAyi mukakxLisu. 
\emng
\eentry

\bentry
\word[gargle(2)]{gargle}
\pron{gAgaR(gfR)lf}
\gl{\nA}
\bmng
\bnum
\num{1} gaMDUSa; bAyi mukakxLisuva darxva. 
\num{2} (\ashi) madayx. 
\enum
\emng
\eentry

\bentry
\word{gargoyle}
\pron{gAgARyflx}
\gl{\nA}
\bmng
\bnum
\num{1} niVrumUti; jalamuKa; jalasUtarx; mALigeyiMda suriyuva niVru goVDege taguladaMte keLakekx biVLalu mADiruva (\kanmu\ haLeya gAthikf shilapxda kaTaTxDagaLalilx manuSayxna yA pArxNiya muKa, bAyi, deVha, \mo vugaLa rUpina) vikaTAkaqtiya mUti, doVNi.  \imglink{gargoyle-1figure}{\raisebox{-0.15cm}[0pt][0pt]{\pdfimage width 0.7cm height 0.5cm {G_Pictures/gargoyle-1.jpg}}} 
\num{2} (yAvudeV) vikaTashilapx; vikaTAkaqti: \eng{strange Ethiopian gargoyle carved on the ebony footposts of his bed} avana eboni marada maMcada kAlina meVle koreda vicitarx itiyoVpiyanf AkaqtigaLu. 
\num{3} vikaTamuKi; vikaTavAda muKa, mUtiyuLaLx vayxkitx: \eng{what you need is a woman, older of course... but not a gorgon or a gargoyle} ninage beVkAgiruvudu oMdu heNuNx, .... nijavAgi savxlapx vayasAsxgiruva heNuNx, Adare karALamuKiyoV vikaTamuKiyoV alalx. 
\enum
\emng
\eentry

\bentry
\word{garibaldi}
\pron{gAYxribAliDx}
\gl{\nA}
\expl{(\bava\ \eng{garibaldis}).}
\bmng
gAYxribAliDx: 
\banum
\alnum{a} heMgasina yA maguvina, (modalige ujavxla keMpu baNanxdAdxgirutitxdadx) kupapxsa, ravike. 
\alnum{b} (\birx) oNa dArxkiSxya padaraviruva bisakxtutx. 
\alnum{c} (\ame) kAyxliphoVniRyada saNaNx mInu. 
\eanum
\emng
\eentry

\bentry
\word{garish}
\pron{geVriSf}
\gl{\gu}
\bmng
\bnum
\num{1} kaNuNx koVreYsuva; atuyxjajxvXla; ati parxkAshada; rAvu baDiyuva. 
\num{2} ADaMbarada; ati DaMBada; bahaLa beDagina. 
\num{3} vipariVta alaMkArada; ramArami(yAda). 
\enum
\emng
\eentry

\bentry
\word{garishly}
\pron{geVriSfli}
\gl{\kirxvi}
\bmng
\bnum
\num{1} kaNuNx koVreYsuvaMte; atuyxjajxvXlavAgi; ati parxkAshamAnavAgi; rAvu baDiyuvaMte. 
\num{2} ADaMbaravAgi; ati DaMBadiMda; bahaLa beDaginiMda. 
\num{3} vipariVta alaMkAradiMda; ramAramiyAgi. 
\enum
\emng
\eentry

\bentry
\word{garishness}
\pron{geVriSfnisf}
\gl{\nA}
\bmng
\bnum
\num{1} atuyxjajxvXlate; kaNuNx koVreYsuvaMtiruvudu; ati parxkAshamAnate; rAvu baDiyuvaMtiruvike. 
\num{2} ADaMbarate; ati beDagu. 
\num{3} atayxlaMkAradiMda kUDiruvudu. 
\enum
\emng
\eentry

\bentry
\word[garland(1)]{garland}
\pron{gAlaRMDf}
\gl{\nA}
\bmng
\bnum
\num{1} (alaMkarakAkxgi taleya sutatxlU dharisuva yA yAvudakAkxdarU iLiyabiDuva, ele, hUvu, \mo vugaLa) daMDe; sara; mAle; hAra. 
\num{2} (jaya \mo vugaLigAgi koDuva) (turAyi, hAra, padaka, \mo) gwrava; parxshasitx; birudu; bahumAna. 
\num{3} padayxsaMgarxha; kAvayxmAle; kavanasaMkalana; padayx, lAvaNi, \mo\ cikakx sAhitayx kaqtigaLa saMkalana. 
\num{4} (loVha \mo vugaLiMda mADida)hAra; sara; mAle. 
\enum
\emng
\eentry

\bentry
\word[garland(2)]{garland}
\pron{gAlaRMDf}
\gl{\sakirx}
\bmng
\bnum
\num{1} daMDe muDisu; hAra toDisu, hAku. 
\num{2} mAlegaLiMda siMgarisu. 
\num{3} (alaMkArada) hAravAgiru; mAleyAgu. 
\enum
\emng
\eentry

\bentry
\word{garlic}
\pron{gAliRkf}
\gl{\nA}
\bmng
 beLuLxLiLx; lashuna. 
\emng
\eentry

\bentry
\word{garlicky}
\pron{gAliRki}
\gl{\gu}
\bmng
 beLuLxLiLx vAsaneya. 
\emng
\eentry

\bentry
\word[garment(1)]{garment}
\pron{gAmaRMTf}
\gl{\nA}
\bmng
\bnum
\num{1} uDupu; uDige. 
\num{2} (\bava dalilx) uDupu; baTeTxbare; vasatxrX; uDige toDige. 
\num{3} (yAvudAdarU vasutxvina) hodike; horakekx kANuva, goVcaravAguva hora hodike. 
\enum
\emng
\eentry

\bentry
\word[garment(2)]{garment}
\pron{gAmaRMTf}
\gl{\sakirx}
\bmng
 (\sA\ \BUkaq dalilx \parx) (\kAparx) baTeTxhAku; uDupu toDisu. 
\emng
\eentry

\bentry
\word[garner(1)]{garner}
\pron{gAnaRrf}
\gl{\nA}
\bmng
 (\kAparx, \alaMshA) (kALu tuMbuva) kaNaja; hageVvu; davasada ugArxNa; dhAnAyxgAra (\rUpa\ saha). 
\emng
\eentry

\bentry
\word[garner(2)]{garner}
\pron{gAnaRrf}
\gl{\sakirx}
\bmng
\bnum
\num{1} kUDahAku; sheVKarisiDu. 
\num{2} saMcayisu; saMgarxhisu. 
\enum
\emng
\eentry

\bentry
\word{garnet}
\pron{gAniRTf}
\gl{\nA}
\bmng
 rakatxmaNi; padamxrAga; gAneRTuTx; pAradashaRkavAda, daTaTx keMpu jAtiyadanunx ratanxvanAnxgi baLasuva, gAjinaMtha oMdu bageya Kanija. 
\emng
\eentry

\bentry
\word[garnish(1)]{garnish}
\pron{gAniRSf}
\gl{\sakirx}
\bmng
\bnum
\num{1} (\kanmu\ tiMDi tiVthaRgaLanunx UTada meVjina meVle) AkaSaRkavAgi alaMkarisu; aMdavAgi aNimADu. 
\hypertarget{garnish(1)2}{} 
\num{2} (\nAyxshA) sAlagArxhige yA parxtivAdige seVrida haNavanunx kAnUnu riVtAyx jaPitx mADalu yA vashapaDisikoLaLxlu (adanunx iTuTxkoMDiruva vayxkitxge) noVTiVsanunx jArimADu, (barahadalilx) tiLuvaLike koDu. 
\num{3} IgAgaleV beVreyavara naDuve naDeyutitxruva dAvege (obabxnanunx) kakiSxyanAnxgi koVTuR mUlaka -- karesu, baramADu. 
\enum
\emng
\eentry

\bentry
\word[garnish(2)]{garnish}
\pron{gAniRSf}
\gl{\nA}
\bmng
\bnum
\num{1} BoVjAyxlaMkAra; KAdAyxlaMkAra; tiMDi tiVthaRgaLu cenAnxgi kANuvaMte yA rucikaravAguvaMte mADalu avakekx seVrisuva alaMkArada yA rucikAraka vasutxgaLu, padAthaRgaLu. 
\num{2} (\rUpa) kAvAyxlaMkAragaLu; sAhitayxkaqti suMdaravAgi kANisuvaMtAgalu baLasuva alaMkAra riVtigaLu. 
\enum
\emng
\eentry

\bentry
\word[garnishee(1)]{garnishee}
\pron{gAniRSiV}
\gl{\nA}
\bmng
 haNada jaPitxge guriyAda sAlagArxhiya yA parxtivAdiya haNavanunx iTuTxkoMDiruvavanu. 
\emng
\eentry

\bentry
\word[garnishee(2)]{garnishee}
\pron{gAniRSiV}
\gl{\sakirx}
\bmng
\bnum
\num{1}  = \hyperlink{garnish(1)2}{$^1$garnish \((2 \& 3)\)}. 
\num{2} gAniRSfmeMTf parxkAra sAlagArxhiya haNavanunx jaPitxmADu, vashapaDisiko. 
\enum
\emng
\eentry

\bentry
\word{garnisher}
\pron{gAniRSarf}
\gl{\nA}
\bmng
\bnum
\num{1} alaMkAraka; aMdagoLika; aNimADuvava; alaMkarisiDuvava. 
\num{2} gAniRSaru; inonxbabxna keYyalilxruva sAlagArxhiya haNavanunx jaPitx mADuvava; jaPitxdAra. 
\enum
\emng
\eentry

\bentry
\word{garnishing}
\pron{gAniRSiMgf}
\gl{\nA}
\bmng
  = \hyperlink{garnish(2)}{$^2$garnish}. 
\emng
\eentry

\bentry
\word{garnishment}
\pron{gAniRSfmaMTf}
\gl{\nA}
\bmng
\bnum
\num{1} alaMkaraNa; aMdagoLike; aMdavAgi aNimADuvudu. 
\numi{2} (\nAyxshA) gAniRSfmeMTu; gAniRSi noVTiVsu, patarx: 
\banum
\alnum{a} haNada jaPitxge guriyAda sAlagArxhiya yA parxtivAdiya haNavanunx iTuTxkoMDiruva avana utatxrAdhikAriyiMda yA itara vayxkitxyiMda A haNavanunx kAnUnuriVtAyx jaPitx mADalu, vashapaDisikoLaLxlu horaDisuva noVTiVsu, AjAcnxpatarx, yA tiLivaLike patarx. 
\alnum{b} haNada jaPitxge guriyAda sAlagArxhiya haNavanunx iTuTxkoMDiruvavanige A haNavanunx jaPitx mADalAgutatxdeyeMdu tiLisuva koVTiRna noVTiVsu, tiLivaLike patarx. 
\eanum
\numie
\enum
\emng
\eentry

\bentry
\word{garniture}
\pron{gAniRcarf}
\gl{\nA}
\bmng
\bnum
\num{1} sAdhana sAmagirx; salakaraNe; upakaraNagaLu; parikaragaLu. 
\num{2} alaMkaraNa; aNigoLike; aMdagoLisuvudu; alaMkarisuvudu; (\kanmu\ tiMDitinisugaLanunx) aNimADuvudu. 
\num{3} aNi; alaMkAra. 
\num{4} uDige toDige; poVSAku; veVSAlaMkAra. 
\enum
\emng
\eentry

\bentry
\word[garotte(1)]{garotte}
\pron{garATf}
\gl{\nA}
\bmng
  = \hyperlink{garrotte(1)}{$^1$garrotte}. 
\emng
\eentry

\bentry
\word[garotte(2)]{garotte}
\pron{garATf}
\gl{\sakirx}
\bmng
  = \hyperlink{garrotte(2)}{$^2$garrotte}. 
\emng
\eentry

\bentry
\word{garpike}
\pron{gApeYRkf}
\gl{\nA}
\bmng
  = \hyperlink{garfish}{garfish}. 
\emng
\eentry

\bentry
\word[garret(1)]{garret}
\pron{gAYxriTf}
\gl{\nA}
\bmng
\bnum
\num{1} (\kanmu\ holasAda) aTaTxda koVNe; (meVlaTaTxda) koLaku koThaDi; maneya meVlAcxvaNige hodidxkoMDiruva koVNe. 
\num{2} (\ashi) tale. 
\enum
\emng

\noindent
\gl{\nuga}
\bmng
\bnum
\num{1} \eng{be wrong in the garret} (\ashi) tale keTiTxru. 
\num{2} \eng{have one's garret unfurnished} (\ashi) tale KAliyAgiru, baridAgiru; daDaDxnAgiru. 
\enum
\emng
\eentry

\bentry
\word[garret(2)]{garret}
\pron{gAYxriTf}
\gl{\sakirx}
\bmng
 (\vAshi) kalulx cakekx seVrisu; oraTu kalulx kaTaTxDadalilxya saMdugaLige kalulx cakekx iDu. 
\emng
\eentry

\bentry
\word{garreteer}
\pron{gAYxriTiarf}
\gl{\nA}
\bmng
 aTaTxnivAsi; aTaTx jiVvi; meVlaTaTxdalilx vAsisuvava (\kanmu\ baDa kUli sAhiti). 
\emng
\eentry

\bentry
\word[garrison(1)]{garrison}
\pron{gAYxrisanf}
\gl{\nA}
\bmng
\bnum
\num{1} (dugaR) rakaSxka seYnayx; kAvalu -- paDe, daMDu; koVTe, dugaR, kilelx, paTaTxNa, \mo vugaLalilx (avugaLa) rakaSxNegAgi pALeyaviruva seYnayx. 
\num{2} (koVTe, paTaTxNa, \mo vugaLa) kAvalupaDe keVMdarx; rakaSxkaseYnayx biDAra hUDiruva kaTaTxDa. 
\enum
\emng
\eentry

\bentry
\word[garrison(2)]{garrison}
\pron{gAYxrisanf}
\gl{\sakirx}
\bmng
\bnum
\num{1} rakaSxkaseYnayxvirisu. 
\num{2} rakaSxkaseYnayxvAgi nelasu; ThANayx hUDu; ThANayxviru. 
\num{3} seYnikaranunx kAvalige, rakaSxNege -- iDu, hAku, niyamisu. 
\enum
\emng
\eentry

\bentry
\wordnospeech{garrison town}{garrison town}
\pron{?}
\gl{\nA}
\bmng
 daMDina paTaTxNa; rakaSxkaseYnayx pALeyaviruva paTaTxNa. 
\emng
\eentry

\bentry
\word{garron}
\pron{gAYxranf}
\gl{\nA}
\bmng
 (sAkxTalxMDf matutx ailaRMDugaLalilx beLesida) kiVLetxrada taTuTx, cikakx kudure. 
\emng
\eentry

\bentry
\word{garrot}
\pron{gAYxraTf}
\gl{\nA}
\bmng
 oMdu jAtiya kaDala bAtu. 
\emng
\eentry

\bentry
\word[garrote(1)]{garrote}
\pron{garATf}
\gl{\nA}
\bmng
 (\ame)  = \hyperlink{garrotte(1)}{$^1$garrotte}. 
\emng
\eentry

\bentry
\word[garrote(2)]{garrote}
\pron{garATf}
\gl{\sakirx}
\bmng
 (\ame)  = \hyperlink{garrotte(2)}{$^2$garrotte}. 
\emng
\eentry

\bentry
\word[garrotte(1)]{garrotte}
\pron{garATf}
\gl{\nA}
\bmng
\bnum
\num{1} katutx hisuki kolulxva sepxVnf deVshada galulxshikeSx, maraNadaMDane. 
\num{2} I daMDanegAgi baLasuva sAdhana, salakaraNe. 
\num{3} (sikikx bidadxvananunx) katutx hisuki, usiru kaTiTxsi mADuva dArigaLaLxtana, daroVDe. 
\enum
\emng
\eentry

\bentry
\word[garrotte(2)]{garrotte}
\pron{garATf}
\gl{\sakirx}
\bmng
\bnum
\num{1} (daMDanegAgi) katutx hisuki kululx. 
\num{2} (sulige mADalu) katutx kivicu, hisuku; katutx hisuki usiru kaTiTxsu. 
\enum
\emng
\eentry

\bentry
\word{garrotter}
\pron{garATarf}
\gl{\nA}
\bmng
\bnum
\num{1} (katutx kivici kolulxva) suligegAra; daroVDekoVra. 
\num{2} (maraNadaMDane jArigoLisalu) katutx kivuci kolulxvava. 
\enum
\emng
\eentry

\bentry
\word{garrulity}
\pron{garUliTi}
\gl{\nA}
\bmng
 gaLapuvike; haraTuvike; ati mAtugArike; vAcALatana; balu mAtanADuvudu. 
\emng
\eentry

\bentry
\word{garrulous}
\pron{gAYxru(ra)lasf}
\gl{\gu}
\bmng
\bnum
\num{1} amuKayx viSayagaLa, saMgatigaLa bagegx -- mAtu beLesuva, atiyAgi mAtanADuva. 
\num{2} vAcAla; gaLahuva; haraTuva; balu yA ati mAtanADuva. 
\num{3} shabadxbAhuLayxda; shabAdxDaMbarada; atimAtina; bariya shabadxgaLiMda tuMbida: \eng{a garrulous speech} bari shabadxgaLiMda tuMbida BASaNa. 
\num{4} (hakikx, hoLe, \mo vugaLa \vi) cilipiliguTuTxva yA juLujuLuguTuTxva. 
\enum
\emng
\eentry

\bentry
\word{garrulously}
\pron{gAYxru(ra)lasfli}
\gl{\kirxvi}
\bmng
\bnum
\num{1} gaLahutAtx; haraTutAtx; vAcAlateyiMda; bahumAtanADutAtx. 
\num{2} (hakikx, hoLe, \mo vugaLa \vi) cilipiliguTuTxtAtx; kalaravadiMda yA juLujuLuguTuTxtatx. 
\enum
\emng
\eentry

\bentry
\word{garrulousness}
\pron{gAYxru(ra)lasfnisf}
\gl{\nA}
\bmng
  = \hyperlink{garrulity}{garrulity}. 
\emng
\eentry

\bentry
\wordRemoveSpace{Garry-oak}{Garry oak}
\pron{gAYxri Okf}
\gl{\nA}
\bmng
 utatxra amerikada oMdu bageya Okfmara. 
\emng
\eentry

\bentry
\word{garrya}
\pron{gAYxria}
\gl{\nA}
\bmng
 gAyxriya; AlaMkArika goMcaluhUvu biDuva, nitayxharidavxNaRda oMdu hodaru. 
\emng
\eentry

\bentry
\word[garter(1)]{garter}
\pron{gATaRrf}
\gl{\nA}
\bmng
\bnum
\num{1} kAlicxVla paTiTx; kAlicxVlavanunx meVletitxruvaMte kaTaTxlu maMDiya keLage yA meVle kaTuTxva paTiTx. 
\num{2} (\birx) (\eng{Garter}) iMgilxSf neYTf padaviyalilx atayxMta ucacx sherxVNiya lAMCana. 
\num{3} (\ame)kAlicxVlada tUgupaTiTx; kAlicxVlavanunx meVletitxruvaMte hiDidiDalu baLasuva tUgupaTiTx, iLipaTiTx. 
\enum
\emng

\noindent
\gl{\pagu}
\bmng
 \eng{the Garter} (\birx) 
\banum
\alnum{a} iMgilxSf neYTf padaviyalilx atayxMta ucacx sherxVNi yA adara padaka. 
\alnum{b} I padaviya sadasayxtavx. 
\eanum
\emng
\eentry

\bentry
\word[garter(2)]{garter}
\pron{gATaRrf}
\gl{\sakirx}
\bmng
\bnum
\num{1} kAlicxVlavanunx paTiTxyiMda bigi. 
\num{2} kAlicxVlapaTiTxyiMda (kAlanunx) sututx, bigi. 
\enum
\emng
\eentry

\bentry
\word{garter-belt}
\pron{gATaRrfbelfTx}
\gl{\nA}
\bmng
 (\ame) kAlicxVlagaLanunx meVletitx hiDidiDuva tUgupaTiTxgaLuLaLx, heMgasara oLauDupu. 
\emng
\eentry

\bentry
\word{garter-snake}
\pron{gATaRrfsenxVkf}
\gl{\nA}
\bmng
 paTeTx hAvu; utatxra amerikada yA dakiSxNa Aphirxkada ududxdadxvAda paTeTxgaLuLaLx hAvu. 
\emng
\eentry

\bentry
\wordnospeech{garter stitch}{garter stitch}
\pron{?}
\gl{\nA}
\bmng
 ENu holige; sAlu biTuTx sAlinalilx ENugaLiruvaMte hAkida heNigeya holige.  \imglink{garter stichfigure}{\raisebox{-0.15cm}[0pt][0pt]{\pdfimage width 0.7cm height 0.5cm {G_Pictures/garter stich.jpg}}} 
\emng
\eentry

\bentry
\word{garth}
\pron{gAtfR}
\gl{\nA}
\bmng
 (\birx) 
\bnum
\num{1} (\pArxparx\ matutx \pArxM) bayalu; meYdAna. 
\num{2} aMgaLa. 
\num{3} toVTa. 
\num{4} kudureya bayalu. 
\num{5} (cacuR, maTha, \mo vugaLalilxya) pArxMgaNa. 
\num{6} mInu hiDiyalu mADikoMDa oDuDx, aNekaTuTx, \mo vu. 
\enum
\emng
\eentry

\bentry
\word{garuda}
\pron{gAruDA}
\gl{\nA}
\bmng
 garuDa: 
\banum
\alnum{a} (BAratiVya purANadalilxna) viSuNxvina vAhanavAda manuSAyxkaqtiya pakiSx. 
\alnum{b} iMDoVneVSayxda rASiTxrXVya lAMCana. 
\eanum
\emng
\eentry

\bentry
\word[gas(1)]{gas}
\pron{gAYxsf}
\gl{\nA}
\expl{(\bava\ \eng{gases} \ucAcx\ gAYxsisf).}
\bmng
% 
\bnum
\numi{1} anila: 
\banum
\alnum{a} \kanmu\ sAmAnayx tApadalilx darxva yA GanavAgadiruva padAthaR. 
\alnum{b} iMdhanAnila; \sA\ vividha heYDorxkAbaRnunxgaLanonxLagoMDiruva, kalilxdadxlu aniladaMte iMdhanavAgi upayoVgisuva yAvudeV mishArxnila. 
\alnum{c} (\gaNi) mItheVnf matutx vAyuvina soPxVTaka misharxNa. 
\alnum{d} balUnugaLanunx tuMbalu baLasuva heYDorxVjanf, hiVliyaM, \mo vu. 
\alnum{e} parxjecnx tapipxsalu baLasuva neYTarxsf AkesxYDu yA itara anila. 
\hypertarget{gas(1)1f}{} 
\hyperdef{G}{gas(lf)(1)}{} 
\alnum{f} yudadhxdalilx shaturxgaLa usiru kaTiTxsalu baLasuva yAvudeV viSAnila. 
\alnum{g} diVpagaLige baLasuva aniladhAre. 
\eanum
\numie
\num{2} (\ame) (\AmA) peTorxVlu; gAyxsoliVnu. 
\num{3} (\AmA) goDuDx haraTe; kelasakekx bArada mAtu. 
\num{4} (\AmA) goDuDxharaTe; kelasakekx bArada baDAyi; bogaLe; buruDe; poLuLxmAtu. 
\num{5} (\ashi) tuMba AkaSaRkavAda yA parxBAvakAriyAda vayxkitx, vasutx. 
\enum
\emng

\noindent
\gl{\pagu}
\bmng
\bnum
\num{1} \eng{natural gas} neYsagiRkAnila; BUmiyaDi doreyuva, kalilxdadxlu \mo vugaLa anila. 
\num{2} \eng{poison gas} = \hyperlink{gas(1)1f}{$^1$gas \((1f)\)}. 
\enum
\emng

\noindent
\gl{\nuga}
\bmng
\hyperdef{G}{gas(1) nuga}{} \eng{step on the gas} 
\banum
\alnum{a} kAliniMda veVgavadhaRkada peDalanunx otitx moVTAru vAhanada veVga hecicxsu. 
\alnum{b} (\rUpa) avasara mADu; avasarapaDisu; tavxremADu. 
\eanum
\emng
\eentry

\bentry
\word[gas(2)]{gas}
\pron{gAYxsf}
\gl{\sakirx}
\bmng
\bnum
\num{1} anilakekx -- oDuDx, oLapaDisu. 
\num{2} (koThaDige, reYlevx boVgige) anila odagisu. 
\num{3} (shaturxvina yA oMdu sathxLada meVle) viSAnila parxyoVgisu. 
\num{4} (dAradiMda yA kasUti paTiTxyiMda edudxkoMDa biDidAragaLanunx hoVgalADisalu) anila jAvxleyalilx ADisu. 
\hypertarget{gas(2)sakirx5}{} 
\num{5} (\ame) (\AmA) (moVTAru vAhanada TAYxMkige) peTorxVlf tuMbu, BatiRmADu. 
\enum
\emng

\noindent
\gl{\akirx}
\bmng
 (\AmA) buruDe hoDi; bogaLe biDu; baDAyi kocucx; gaphA hoDi. 
\emng

\noindent
\gl{\pagu}
\bmng
 \eng{gas up} (\ame)  = \hyperlink{gas(2)sakirx5}{$^2$gas ?sakirx? \((5)\)}. 
\emng
\eentry

\bentry
\word{gasbag}
\pron{gAYxsfbAYxgf}
\gl{\nA}
\bmng
\bnum
\num{1} anilaciVla; anila tuMbalu baLasuva ciVla. 
\num{2} (\hiV) buruDe; baDAyi; bogaLe; poLuLx mAtADuvava. 
\num{3} (vAyuja{ha}jinalilxruva) aniladhAraka. 
\num{4} (vAyuja{ha}jinalilxruva) aniladhAraka. 
\num{5} vAyuja{ha}ju yA balUnu. 
\enum
\emng
\eentry

\bentry
\wordnospeech{gas bracket}{gas bracket}
\pron{?}
\gl{\nA}
\bmng
 anilacAcu; goVDeyiMda cAcikoMDiruva anila jAvxlakada, diVpada koLave. 
\emng
\eentry

\bentry
\wordnospeech{gas chamber}{gas chamber}
\pron{?}
\gl{\nA}
\bmng
 anila koVNe; anila koVSaThx; viSAniladiMda pArxNigaLanonxV keYdigaLanonxV kolulxva koThaDi. 
\emng
\eentry

\bentry
\wordnospeech{gas chromatography}{gas chromatography}
\pron{?}
\gl{\nA}
\bmng
 anila vaNaR reVKana; anila korxVmaToVgarxphi; anilavanunx nidhAnavAgi adhicUSaka padAthaRda mUlaka athavA adara meVle hAyisuva mUlaka padAthaRgaLanunx parxteyxVkisuva vidhAna. 
\emng
\eentry

\bentry
\word{gas-coal}
\pron{gAYxsfkoVlf}
\gl{\nA}
\bmng
 anila kalilxdadxlu; anila tayArikege upayukatxvAda biTUyxminasf kalilxdadxlu. 
\emng
\eentry

\bentry
\word{gas-coke}
\pron{gAYxsfkoVkf}
\gl{\nA}
\bmng
 anila koVku; kalilxdadxliniMda anila tayArisidAga uLiyuva kiTaTx, kariku. 
\emng
\eentry

\bentry
\word{Gascon}
\pron{gAYxsfkanf}
\gl{\nA}
\bmng
\bnum
\num{1} phArxnisxna gAyxsakxni parxdeVshadavanu. 
\num{2} (\eng{gascon}) baDAyikoVra; jaMbakocucxvava. 
\enum
\emng
\eentry

\bentry
\word[gasconade(1)]{gasconade}
\pron{gAYxsakxneVDf}
\gl{\nA}
\bmng
 baDAyi; jaMba kocucxvudu. 
\emng
\eentry

\bentry
\word[gasconade(2)]{gasconade}
\pron{gAYxsakxneVDf}
\gl{\akirx}
\bmng
 baDAyi kocucx; jaMbakocucx. 
\emng
\eentry

\bentry
\wordnospeech{gas constant}{gas constant}
\pron{?}
\gl{\nA}
\bmng
 (\Bwvi) anila sithxra; anilagaLa tApa, otatxDa, matutx gAtarxgaLigiruva saMbaMdhavanunx nirUpisuva `sithxti samiVkaraNa'dalilx (\eng{equation of state of gases}) kaMDubaruva AdashaR anilavanunx parigaNanege tegedukoMDAga oMdu moVlf (\eng{mole}) anilada gAtarx matutx otatxDagaLa guNalabadhxvanunx nirapeVkaSx tApadiMda BAgisidAga dorakuva saMKeyxge samavAgiruva, oMdu sithxra saMKeyx (= oMdu Digirxge \eng{$8.314\times 10^7$} agfRgaLu). 
\emng
\eentry

\bentry
\wordnospeech{gas cooker}{gas cooker}
\pron{?}
\gl{\nA}
\bmng
 gAyxsf swTxvu; anila swTxvu; gAYxsole; aniladiMda uriyuva swTx. 
\emng
\eentry

\bentry
\word{gas-cooled}
\pron{gAyxsfkUlfDx}
\gl{\gu}
\bmng
 anila taMpita; (beYjika riyAkaTxru, eMjinu, \mo vugaLa \vi) anilada hariviniMda taMpAgisida. 
\emng
\eentry

\bentry
\word{gaseity}
\pron{gasiViTi}
\gl{\nA}
\bmng
\bnum
\num{1} aniliVyate; anila sithxtiyalilxruvike. 
\num{2} (\rUpa) asapxSaTxte; saMdigadhxte. 
\enum
\emng
\eentry

\bentry
\word{gaselier}
\pron{gAYxsaliarf}
\gl{\nA}
\bmng
 (\sA\ cAvaNiya oLameYyiMda tUgabiDuva, goMcalugaLalilx uribatitxgaLiruva) aniladiVpada goMcalu. 
\emng
\eentry

\bentry
\wordnospeech{gas engine}{gas engine}
\pron{?}
\gl{\nA}
\bmng
 anila eMjinunx; gAyxsf eMjinunx; peTorxVlf badalu anila matutx vAyu ivugaLa misharxNavanunx upayoVgisuva aMtadaRhana eMjinunx. 
\emng
\eentry

\bentry
\word{gaseous}
\pron{geV(gAYx)si(Si)asf, geV(gAYx)sayxsf}
\gl{\gu}
\bmng
\bnum
\num{1} aniliVya; anilarUpada; anila sithxtiyalilxruva. 
\num{2} (\rUpa) asapxSaTx; saMdigadhx; Kacitavalalxda. 
\enum
\emng
\eentry

\bentry
\wordnospeech{gas fire}{gas fire}
\pron{?}
\gl{\nA}
\bmng
 gAYxsf hiVTaru; anilatApaka; aniladiMda uriyuva swTxvu yA hiVTaru. 
\emng
\eentry

\bentry
\word{gas-fired}
\pron{gAYxsfpheYaDfR}
\gl{\gu}
\bmng
 anilatapatx; anilada dahanadiMda kAyisida, bisimADida. 
\emng
\eentry

\bentry
\word{gas-fitter}
\pron{gAYxsfphiTarf}
\gl{\nA}
\bmng
 anilagAra; aniladiVpa, anilajAvxlike, \mo\ salakaraNegaLanunx manege odagisuva aMgaDiyava yA kelasagAra. 
\emng
\eentry

\bentry
\word{gas-fittings}
\pron{gAYxsfphiTiMgfsx}
\gl{\nA}
\bmng
 gAyxsf upakaraNgaLu; anila salakaraNegaLu; anilavanunx baLasi kAyisuvudeV \mo vanunx mADabahudAda salakaraNegaLu. 
\emng
\eentry

\bentry
\wordnospeech{gas gangrene}{gas gangrene}
\pron{?}
\gl{\nA}
\bmng
 (\roVshA) anila gAyxMgirxVnu; gAyxsfkoVtha; ALavAda gAyagaLalilx (bAyxkiTxVriyagaLiMda soVMku uMTAgi) anila horaDuva, beVgane haraDuva, oMdu bageya gAYxMgirxVnu, koVtha. 
\emng
\eentry

\bentry
\word[gash(1)]{gash}
\pron{gAYxSf}
\gl{\nA}
\bmng
\bnum
\num{1} siVLugAya; (udadxvU ALavU Ada) katatxrisida gAya. 
\num{2} katatxrisidadxriMdAguva -- siVLu, biruku; (katitx \mo vugaLiMda) kaDidumADida siVLike. 
\num{3} siVLuvike; katitx \mo vugaLiMda kaDiyuvike. 
\enum
\emng
\eentry

\bentry
\word[gash(2)]{gash}
\pron{gAYxSf}
\gl{\sakirx}
\bmng
\bnum
\num{1} siVLugAya mADu. 
\num{2} (katitx \mo vugaLiMda) kaDi; katatxrisu; kocucx. 
\enum
\emng
\eentry

\bentry
\word[gash(3)]{gash}
\pron{gAYxSf}
\gl{\gu}
\bmng
 (\birx) (nAvikara \ashi) sAkaSATxgi mikakx; beVkAdaSaTxkikxMta hecAcxda, migilAda. 
\emng
\eentry

\bentry
\wordnospeech{gas helmet}{gas helmet}
\pron{?}
\gl{\nA}
\bmng
  = \hyperlink{gas mask}{gas mask}. 
\emng
\eentry

\bentry
\word{gasholder}
\pron{gAYxsfhoVlaDxrf}
\gl{\nA}
\bmng
 aniladhAraka; iMdhana anilada sheVKaraNege matutx haMcikege baLasuva doDaDx TAyxMku, bAni. 
\emng
\eentry

\bentry
\word{gasifiable}
\pron{gAYxsipheYabflf}
\gl{\gu}
\bmng
 aniliVkaraNiVya; aniliVkaraNasAdhayx; aniliVkarisabahudAda. 
\emng
\eentry

\bentry
\word{gasification}
\pron{gAYxsiphikeVSanf}
\gl{\nA}
\bmng
 aniliVkaraNa; shAKada sahAyadiMda yA rAsAyanika vidhAnadiMda anilavAgi parivatiRsuvudu yA (nisagaRdalelxV) parivataRne hoMduvudu. 
\emng
\eentry

\bentry
\word{gasiform}
\pron{gAYxsiphAmfR}
\gl{\gu}
\bmng
  = \hyperlink{gaseous}{gaseous}. 
\emng
\eentry

\bentry
\word{gasify}
\pron{gAYxsipheY}
\gl{\sakirx}
\bmng
 aniliVkarisu; anilavAgisu; kAyisuvudariMda yA rAsAyanika vidhAnadiMda darxva yA Gana padAthaRvanunx anilavAgi parivatiRsu. 
\emng

\noindent
\gl{\akirx}
\bmng
 anilavAgu; aniliVkaqtavAgu; shAKadiMda yA rAsAyanika vidhAnadiMda (darxva yA Ganavu) anilavAgi parivataRne hoMdu. 
\emng
\eentry

\bentry
\word{gasket}
\pron{gAYxsikxTf}
\gl{\nA}
\bmng
\bnum
\num{1} haDagina sutitxda hAyiyanunx kaMbakekx bigiyuva saNaNx huri. 
\num{2} [ADubeNeya (pisaTxninxna) sutatxlU sututxva athavA haDagina halagegaLa naDuve saMdu mucucxvaMte turukuva] kaLapeya nAru cUru. 
\num{3} gAYxsekxTuTx; loVhada meVlemxYgaLu seVruva jAgavanunx moharu mADalu baLasuva, rababxru, AYxsepxsATxsu, \mo vugaLiMda mADida, maTaTxsavAda hALe yA uMgura. 
\enum
\emng

\noindent
\gl{\nuga}
\bmng
 \eng{blow a gasket} (\ashi) shAMti kaLeduko; koVpagoLuLx; siTATxgu. 
\emng
\eentry

\bentry
\word{gaskin}
\pron{gAYxsikxnf}
\gl{\nA}
\bmng
 kudureya toDeya hiMBAga. 
\emng
\eentry

\bentry
\word{gasless}
\pron{gAYxsflisf}
\gl{\gu}
\bmng
 anilavilalxda; anila rahita; niranila. 
\emng
\eentry

\bentry
\word{gaslight}
\pron{gAYxsfleYTf}
\gl{\nA}
\bmng
\bnum
\num{1} anila parxkAsha; aniladiMda, \kanmu\ kalilxdadxlu aniladiMda,odaguva beLaku. 
\num{2} gAYxsfleYTu; aniladiVpa; uriyutitxruva aniladhAre. 
\enum
\emng
\eentry

\bentry
\wordnospeech{gas lighter}{gas lighter}
\pron{?}
\gl{\nA}
\bmng
 gAYxsfleYTaru; anila diVpaka, hacucxga: 
\banum
\alnum{a} anilavanunx hotitxsalu baLasuva sAdhana. 
\alnum{b} sigareVTu \mo vanunx hacacxlu baLasuva anila iMdhanavuLaLx sAdhana. 
\eanum
\emng
\eentry

\bentry
\wordnospeech{gaslight paper}{gaslight paper}
\pron{?}
\gl{\nA}
\bmng
 kaqtaka nasubeLakinalilx suPxTiVkarisabahudAda phoVToV kAgada. 
\emng
\eentry

\bentry
\wordnospeech{gaslight plate}{gaslight plate}
\pron{?}
\gl{\nA}
\bmng
 kaqtaka nasubeLakinalilx suPxTiVkarisabahudAda phoVToV Palaka. 
\emng
\eentry

\bentry
\word{gaslit}
\pron{gAYxsfliTf}
\gl{\gu}
\bmng
 aniladiVpatx; uriyuva gAYxsiniMda beLaku paDeda. 
\emng
\eentry

\bentry
\wordnospeech{gas main}{gas main}
\pron{?}
\gl{\nA}
\bmng
 anilada (sarabarAju mADuva) parxdhAna, muKayx -- koLave. 
\emng
\eentry

\bentry
\word{gasman}
\pron{gAYxsfmanf}
\gl{\nA}
\bmng
 aniladava; gAYxsinava: 
\banum
\alnum{a} anila tayAraka; anilavanunx tayArisuvava. 
\alnum{b} anila sarabarAjina bAbatu, rusumu vasUlu mADuvava. 
\alnum{c} manege aniladiVpa \mo\ aniloVpakaraNagaLanunx odagisuvava. 
\eanum
\emng
\eentry

\bentry
\wordnospeech{gas mask}{gas mask}
\pron{?}
\gl{\nA}
\bmng
 anila mogavADa; gAYxsf muKavADa; viSAnilagaLiMda rakiSxsikoLaLxlu dharisuva muKavADa, muKakApu.  \imglink{gas maskfigure}{\raisebox{-0.20cm}[0pt][0pt]{\pdfimage width 0.6cm height 0.6cm {G_Pictures/gas mask.jpg}}} 
\emng
\eentry

\bentry
\word{gas-meter}
\pron{gAYxsfmITarf}
\gl{\nA}
\bmng
anilamApaka; gAYxsf mITaru; baLasiruva anilada motatxvanunx aLedu dAKalu mADuva upakaraNa. 
\emng
\eentry

\bentry
\wordnospeech{gas motor}{gas motor}
\pron{?}
\gl{\nA}
\bmng
  = \hyperlink{gas engine}{gas engine}. 
\emng
\eentry

\bentry
\word{gasogene}
\pron{gAYxsajiVnf}
\gl{\nA}
\bmng
 gAYxsojiVnf; anilagUDisida pAniVyagaLanunx tayArisuva upakaraNa. 
\emng
\eentry

\bentry
\word{gasohol}
\pron{gAYxsahAlf}
\gl{\nA}
\bmng
 (\ame) gAYxsohAlf; iMdhanavAgi baLasuva, peTorxVlf matutx AlokxhAlugaLa misharxNa. 
\emng
\eentry

\bentry
\wordnospeech{gas oil}{gas oil}
\pron{?}
\gl{\nA}
\bmng
 anila eNeNx; anila teYla; gAYxseNeNx; peTorxVliyamfna AMshika Asavanadalilx kerosinf Ada naMtara baruva, iMdhanavAgi baLasuva eNeNx. 
\emng
\eentry

\bentry
\word{gasolene}
\pron{gAYxsaliVnf}
\gl{\nA}
\bmng
 (\ame)  = \hyperlink{gasoline}{gasoline}. 
\emng
\eentry

\bentry
\word{gasoline}
\pron{gAyxsaliVnf}
\gl{\nA}
\bmng
\bnum
\num{1} gAYxsoliVnu; peTorxVliyamimxna AMshika AsavanadiMda tayArisuva, diVpagaLigU kAyisuvudakUkx baLasuva, heYDorxkAbaRnf misharxNa. 
\num{2} (\ame) peTorxVlu. 
\enum
\emng
\eentry

\bentry
\word{gasometer}
\pron{gAYxsAmiTarf}
\gl{\nA}
\bmng
\bnum
\num{1} anilamApaka; anilagaLanunx sheVKarisalu matutx aLeyalu anukUlavAguvaMte anilavanunx oLakUkx horakUkx biDalu taDebiraDegaLiruva, aLate siVse yA koLave. 
\num{2}  = \hyperlink{gasholder}{gasholder}. 
\enum
\emng
\eentry

\bentry
\wordnospeech{gas oven}{gas oven}
\pron{?}
\gl{\nA}
\bmng
\bnum
\num{1}  = \hyperlink{gas cooker}{gas cooker}. 
\num{2}  = \hyperlink{gas chamber}{gas chamber}. 
\enum
\emng
\eentry

\bentry
\word[gasp(1)]{gasp}
\pron{gAsfpx}
\gl{\sakirx}
\bmng
 EdutAtx mAtanADu; meVlusireLeyutAtx ucacxrisu. 
\emng

\noindent
\gl{\akirx}
\bmng
 (baLalikeyiMda yA atAyxshacxyaRdiMda AdaMte) meVlusiru eLe; Edu; ugarisu; usirigAgi bAyibAyi biDu; usirADalu kaSaTxvAgu, kaSaTxpaDu. 
\emng

\noindent
\gl{\pagu}
\bmng
\bnum
\num{1} \eng{gasp life etc. away} (\engit{or} \eng{out)} sAyu; pArxNa biDu. 
\num{2} \eng{gasp out} = \hyperlink{gasp(1)}{$^1$gasp ?sakirx?}. 
\enum
\emng
\eentry

\bentry
\word[gasp(2)]{gasp}
\pron{gAsfpx}
\gl{\nA}
\bmng
\bnum
\num{1} Edu; meVlusiru. 
\num{2} Eduvike; meVlusirugareyuvudu. 
\enum
\emng

\noindent
\gl{\nuga}
\bmng
 \eng{at one's last gasp} 
\banum
\alnum{a} kaTaTxkaDeV usirinalilx; sAyuva sithxtiyalilx; koneya usiru eLeyutAtx; maraNamuKadalilx. 
\alnum{b} (\rUpa) susAtxgiruva; pUtiR baLalihoVda. 
\eanum
\emng
\eentry

\bentry
\word{gasper}
\pron{gAsapxrf}
\gl{\nA}
\bmng
\bnum
\num{1} (\birx) (\ashi) (agagxda) sigareVTu. 
\num{2} Eduga; melusirugareyuvavanu. 
\enum
\emng
\eentry

\bentry
\word{gaspereau}
\pron{gAYxsapxroV}
\gl{\nA}
\expl{(\bava\ \eng{gaspereaus} yA \eng{gaspereaux} \ucAcx\ gAYxsapxroVsf).}
\bmng
 (kenaDA) oMdu bageya mInu. 
\emng
\eentry

\bentry
\word{gaspingly}
\pron{gAsipxMgfli}
\gl{\kirxvi}
\bmng
 EdutAtx; meVlusireLeyutAtx; usirADalu kaSaTxpaDutAtx. 
\emng
\eentry

\bentry
\wordnospeech{gas plant}{gas plant}
\pron{?}
\gl{\nA}
\bmng
 anila sAthxvara; peTorxVliyamimxna darxva GaTakagaLanunx adhika tApadalilx viBajisi, anileVMdhanavAgi upayoVgisabahudAda keLa heYDorxVkAbaRnunxgaLa misharxNavanunx tayArisuva yaMtarx sAdhana. 
\emng
\eentry

\bentry
\wordnospeech{gas poker}{gas poker}
\pron{?}
\gl{\nA}
\bmng
 anila salAki; anila diVpaka salAki; salAkiyalilxruva tUtugaLiMda anilavanunx harisi, hacicx, adara mUlaka ididxlu, ole, \mo vanunx hotitxsabahudAda salakaraNe. 
\emng
\eentry

\bentry
\word{gas-proof}
\pron{gAYxsfphUrxphf}
\gl{\gu}
\bmng
 anilataDe; anila roVdhaka: 
\banum
\alnum{a} anila oLakekx hoVgada. 
\alnum{b} anilada hAni, apAya tagaladaMtha. 
\eanum
\emng
\eentry

\bentry
\wordnospeech{gas range}{gas range}
\pron{?}
\gl{\nA}
\bmng
 (\ame)  = \hyperlink{gas cooker}{gas cooker}. 
\emng
\eentry

\bentry
\wordnospeech{gas ring}{gas ring}
\pron{?}
\gl{\nA}
\bmng
 anila baLe; saNaNx raMdharxgaLiruva baLeyAkArada koLaveya mUlaka anilavanunx harisi aDige \mo vugaLige anukUlavAguvaMte hotitxsalu mADiruva jAvxlaka.  \imglink{gas ringfigure}{\raisebox{-0.15cm}[0pt][0pt]{\pdfimage width 0.6cm height 0.6cm {G_Pictures/gas ring.jpg}}} 
\emng
\eentry

\bentry
\word{gassed}
\pron{gAYxsfTx}
\gl{\gu}
\bmng
\bnum
\num{1} anilaseVvaneyiMda kolalxlapxTaTx yA dubaRlavAda. 
\num{2} (\ashi) amaleVruvaMte kuDida; matatx. 
\enum
\emng
\eentry

\bentry
\word{gasser}
\pron{gAYxsarf}
\gl{\nA}
\bmng
\bnum
\num{1} anila odagisuva vayxkitx yA vasutx. 
\num{2} anila bAvi; anilavanunx koDuva (teYla) bAvi. 
\num{3} haraTemalalx; buruDekoVra; baDAyigAra. 
\num{4} (\ashi) bahaLa AkaSaRkavAda yA parxBAvakAriyAda vayxkitx, vasutx. 
\enum
\emng
\eentry

\bentry
\word{gassiness}
\pron{gAYxsinisf}
\gl{\nA}
\bmng
\bnum
\num{1} anilapUNaRte; aniladiMda tuMbiruvike. 
\num{2} anilate; anilada lakaSxNagaLiruvike. 
\num{3} anilasadaqshate; aniladaMtiruvike. 
\num{4} (mAtu \mo vugaLa \vi) poLuLxtana; shabAdxDaMbara. 
\enum
\emng
\eentry

\bentry
\wordnospeech{gas station}{gas station}
\pron{?}
\gl{\nA}
\bmng
 (\ame) peTorxVlf baMku; peTorxVlf, DiVsalf, \mo vanunx mAruva sathxLa. 
\emng
\eentry

\bentry
\wordnospeech{gas stove}{gas stove}
\pron{?}
\gl{\nA}
\bmng
 gAYxsf swTxvu; anila swTxvu; gAYxsole; aniladiMda uriyuva swTxvu. 
\emng
\eentry

\bentry
\word{gassy}
\pron{gAYxsi}
\gl{\gu}
\bmng
\bnum
\num{1} anilada. 
\num{2} anila tuMbida. 
\num{3} aniladaMtiruva. 
\num{4} (mAtu \mo vugaLa \vi) ToLALxda; poLALxda; bariya shabadxmayavAda; shabAdxDaMbarada. 
\enum
\emng
\eentry

\bentry
\wordnospeech{gas tar}{gas tar}
\pron{?}
\gl{\nA}
\bmng
 anil DAMbaru; kalilxdadxliniMda anilavanunx tayArisuvAga utapxtitxyAguva DAMbaru. 
\emng
\eentry

\bentry
\word{gasteropod}
\pron{gAYxsaTxrapADf}
\gl{\nA}
\bmng
 \eng{gastropod} padada rUpAMtara. 
\emng
\eentry

\bentry
\wordf{gasthaus}
\pron{gAsfTxhwsf}
\gl{\nA}
\expl{\G\ }
\bmng
jamaRniya cikakx hoVTelu. 
\emng
\eentry

\bentry
\word{gas-tight}
\pron{gAYxsfTeYTf}
\gl{\gu}
\bmng
 anilaBadarx; anilarudadhx; anila hogada, tUrada. 
\emng
\eentry

\bentry
\word{gastraea}
\pron{gAYxsiTxrXVa}
\gl{\nA}
\bmng
 gAYxsiTxrxVya; pArxNirAjayxkekxlalx mUlapArxNi eMdu kelavaru BAvisiruva, eraDu jiVvakoVsha padarugaLiruva, ciVladaMtha pArxNi. 
\emng
\eentry

\bentry
\word{gastrectomy}
\pron{gAYxseTxrXkaTxmi}
\gl{\nA}
\bmng
 jaTharaCeVdana; hoTeTxyanunx pUtiRyAgi yA BAgashaH katatxrisi tegeduhAkuvudu. 
\emng
\eentry

\bentry
\word{gastric}
\pron{gAYxsiTxrXkf}
\gl{\gu}
\bmng
 jaTharada; jaTharakekx saMbaMdhisida. 
\emng
\eentry

\bentry
\wordnospeech{gastric fever}{gastric fever}
\pron{?}
\gl{\nA}
\bmng
 (\roVshA) aMtarxjavxra; viSamashiVtajavxra. 
\emng
\eentry

\bentry
\wordnospeech{gastric influenza}{gastric influenza}
\pron{?}
\gl{\nA}
\bmng
jaTharada phUlx; oMdu bageya karuLina beVne. 
\emng
\eentry

\bentry
\wordnospeech{gastric juice}{gastric juice}
\pron{?}
\gl{\nA}
\bmng
 (\jiVra) jaThararasa; jaTharada loVLeporeyalilxna garxMthigaLu sarxvisuva, jiVraka eMjeYmugaLiruva tiLiyAda AmIlxya darxva. 
\emng
\eentry

\bentry
\word{gastrin}
\pron{gAYxsiTxrXnf}
\gl{\nA}
\bmng
 (\jiVra) gAYxsiTxrXnf; jaThararasa utapxtitxyAgalu utetxVjana niVDuva oMdu hAmoVRnu. 
\emng
\eentry

\bentry
\word{gastritis}
\pron{gAYxseTxrXYTisf}
\gl{\nA}
\bmng
 (\roVshA) jaTharadurita; jaTharada, adaralUlx muKayxvAgi adaroLagina loVLeporeya, uriyUta. 
\emng
\eentry

\bentry
\word{gastro-}
\pron{gAYxsoTxrXV-}
\gl{\sapUpa}
\bmng
 jaThara, udara eMbathaRdalilx baLasuva \sapUpa. 
\emng
\eentry

\bentry
\word{gastro-enteric}
\pron{gAYxsoTxrXVeMTarikf}
\gl{\gu}
\bmng
 jaTharagaruLina; jaThara matutx karuLugaLa. 
\emng
\eentry

\bentry
\word{gastro-enteritis}
\pron{gAYxsoTxrXVeMTareYTisf}
\gl{\nA}
\bmng
 jaTharagaruLina uriyUta; jaThara matutx karuLugaLa uriyUta. 
\emng
\eentry

\bentry
\word{gastrologer}
\pron{gAYxsATxrXlajarf}
\gl{\nA}
\bmng
 pAkashAsatxrXjacnx; aDugeya videyx balalxvanu. 
\emng
\eentry

\bentry
\word{gastrologist}
\pron{gAYxsATxrXlajisfTx}
\gl{\nA}
\bmng
  = \hyperlink{gastrologer}{gastrologer}. 
\emng
\eentry

\bentry
\word{gastrology}
\pron{gAYxsATxrXlaji}
\gl{\nA}
\bmng
 pAkashAsatxrX; sUpashAsatxrX; aDugeya videyx. 
\emng
\eentry

\bentry
\word{gastronome}
\pron{gAYxsaTxrXnoVmf}
\gl{\nA}
\bmng
 pAkavimashaRka; pAkarasika; savigAra; aDugeya ruciyanunx niNaRyisabalalxvanu. 
\emng
\eentry

\bentry
\word{gastronomer}
\pron{gAYxsATxrXnamarf}
\gl{\nA}
\bmng
 BoVjana rasika; ruciveVtatx; BoVjana kalAvida; BoVjanashAsatxrXjacnx; oLeLxya UTa uNuNxva kale matutx shAsatxrX balalxvanu. 
\emng
\eentry

\bentry
\word{gastronomic}
\pron{gAYxsaTxrXnAmikf}
\gl{\gu}
\bmng
 BoVjana kaleya hAgU shAsatxrXda. 
\emng
\eentry

\bentry
\word{gastronomical}
\pron{gAYxsaTxrXnAmikalf}
\gl{\gu}
\bmng
  = \hyperlink{gastronomic}{gastronomic}. 
\emng
\eentry

\bentry
\word{gastronomically}
\pron{gAYxsaTxrXnAmikali}
\gl{\kirxvi}
\bmng
 BoVjanarasikateyiMda. 
\emng
\eentry

\bentry
\word{gastronomist}
\pron{gAYxsATxrXnamisfTx}
\gl{\nA}
\bmng
 BoVjana rasika hAgU shAsatxrXjacnx. 
\emng
\eentry

\bentry
\word{gastronomy}
\pron{gAYxsATxrXnami}
\gl{\nA}
\bmng
 BoVjana kale matutx shAsatxrX; oLeLxya UTa uNuNxva kale matutx shAsatxrX. 
\emng
\eentry

\bentry
\word{gastropod}
\pron{gAYxsaTxrXpADf}
\gl{\nA}
\bmng
 gAYxsATxrXpaDa vagaRkekx seVrida, basavana huLuveV modalAda maqdavxMgi. 
\emng
\eentry

\bentry
\word{gastropodous}
\pron{gAYxsATxrXpaDasf}
\gl{\gu}
\bmng
 gAYxsATxrXpaDa vagaRkekx seVrida. 
\emng
\eentry

\bentry
\word{gastroscope}
\pron{gAYxsaTxrXsokxVpf}
\gl{\nA}
\bmng
 jaTharadashaRka; bAyiya mUlaka jaTharakekx tUrisi adara oLaBAgavanunx pariVkiSxsalu baLasuva upakaraNa. 
\emng
\eentry

\bentry
\word{gastrula}
\pron{gAYxsuTxrXla}
\gl{\nA}
\expl{(\bava\ \eng{gastrulae} \ucAcx\ gAYxsuTxrXliV).}
\bmng
 (\pArxvi) gAYxsuTxrXla; BUrxNada oMdu pArxraMBAvasethx. 
\emng
\eentry

\bentry
\wordnospeech{gas turbine}{gas turbine}
\pron{?}
\gl{\nA}
\bmng
 anila gAli; gAYxsf TabeYRnu; dahana saMpuTadiMda baruva adhika otatxDada bisi anilagaLiMda ODuva TabeYRnu. 
\emng
\eentry

\bentry
\word{gasworks}
\pron{gAYxsfvakfsxR}
\gl{\nA}
\bmng
 anila kAKARne; gAyxsf kAKARne; anilavanunx tayArisuva matutx saMsakxrisuva kAKARne. 
\emng
\eentry

\bentry
\word[gat(1)]{gat}
\pron{gAYxTf}
\gl{\nA}
\bmng
 (\ashi) (\ame) 
\bnum
\num{1} baMdUku; tupAki. 
\num{2} rivAlavxrf. 
\enum
\emng
\eentry

\bentry
\word[gat(2)]{gat}
\pron{gAYxTf}
\gl{\kirx}
\bmng
 \eng{get} dhAtuvina pArxciVna \BU\ rUpa. 
\emng
\eentry

\bentry
\word[gate(1)]{gate}
\pron{geVTf}
\gl{\nA}
\bmng
\bnum
\num{1} (Urina pArxkAra, koVTe goVDe, kaTaTxDa, \mo vugaLa) horabAgilu; mahAdAvxra; hebAbxgilu. 
\num{2} (\beY) nagarada nAyxyasaBe seVruva sathxLa; Ura cAvaDi. 
\num{3} kaNivemAgaR; beTaTxda kaNive. 
\num{4} bAgilu; dAvxra; parxveVsha yA nigaRmana patha; oLakekx baruva, horakekx hoVguva yAvudeV hAdi, mAgaR. 
\num{5} geVTu; kada; pArxkArada bAgilina, kAMpwMDina kada. 
\num{6} tUbu; niVru bAgilu; tUbina kada. 
\numi{7} 
\banum
\alnum{a} perxVkaSxkara saMKeyx;(phuTfbAlf paMdayx \mo vanunx noVDuvudakAkxgi bAgilalilx duDuDx koTuTx)oLakekx baruvavara saMKeyx. 
\hypertarget{gate(1)7b}{} 
\alnum{b} geVTubAbutx; geVTuvasUli; bAgiluvasUli;hiVge vasUlAda haNa. 
\eanum
\numie
\num{8} (vimAna nilAdxNadalilx vimAnaveVralu hoVgabeVkAda)naMbaru bAgilu; dAvxrasaMKeyx; saMKeyx hAkida parxveVsha dAvxra. 
\num{9} (\ashi) bAyi. 
\num{10} (\ame) (\ashi) vajA mADuvudu; horage, Acege kaLuhisuvudu. 
\numi{11} geVTu: 
\banum
\alnum{a} goVDe, rasetx yA ONiya teravanunx mucucxva aDaDxtaDe, taDegaTuTx. 
\alnum{b} kiVlugaLa meVle kUrisida yA tirugANiya meVle tiruguva, yA Ace Ice jArisabahudAda, marada yA kabibxNada cwkaTuTx, jAlari. 
\alnum{c} moVTAru vAhanada geVrfpeTiTxgeya sanenxyanunx calisi vividha geVrugaLanunx hAkabahudAdaMte, \eng{H} akaSxrada AkAradalilx mADiruva siVLugaMDigaLa vayxvasethx. 
\alnum{d} calanacitarx kAyxmarAda yA porxjakaTxrina lenisxna hiMBAgadalilx philamxnunx kaSxNakAla hiDidiDuva sAdhana. 
\alnum{e} itara viduyxtf saMjecnxgaLige kAraNavAguva yA avugaLanunx niyaMtirxsuva viduyxtfsaMjecnx. 
\alnum{f} halavu BukatxsaMjecnxgaLa saMyoVgadiMda nidhARritavAguva oMdeV oMdu pArxpatxsaMjecnxyuLaLx viduyxnamxMDala. 
\eanum
\numie
\enum
\emng

\noindent
\gl{\pagu}
\bmng
\hyperdef{G}{gate(1) pagu(1)}{} 
\bnum
\num{1} \eng{the gate of horn} (\girxVpu) satayx savxpanxdAvxra; nijada kanasu baruva hAdi, dAvxra. 
\num{2} \eng{the gate of ivory} (\girxVpu) mithAyx savxpanxdAvxra; suLuLx kanasu baruva hAdi, dAvxra. 
\enum
\emng

\noindent
\gl{\nuga}
\bmng
 \eng{get the gate} (\ashi) 
\banum
\alnum{a} nirAkaqtanAgu; tirasakxqqtanAgu. 
\alnum{b} (kelasadiMda) vajA Agu; ciVTi paDe. 
\eanum
\emng
\eentry

\bentry
\word[gate(2)]{gate}
\pron{geVTf}
\gl{\sakirx}
\bmng
 (\birx) pUtiRyAgi yA niyata kAlavAda taruvAya (vidAyxthiRyanunx) sUkxlinalilx yA kAleVjinalilx -- taDediDu, nibaRMdhisiDu. 
\emng
\eentry

\bentry
\word[gate(3)]{gate}
\pron{geVTf}
\gl{\nA}
\bmng
 (yAvudAdarU hesaru pUvaRpadavAgiruvaMte saMyukatxvAgi \parx)biVdi; rasetx: \eng{North gate} utatxrabAgilina biVdi. 
\emng
\eentry

\bentry
\word{gateau}
\pron{gAYxToV}
\gl{\nA}
\expl{(\bava\ \eng{gateaus}, yA \eng{gateaux} \ucAcx\ gAyxToVsfZ).}
\bmng
 cenAnxgi lavAjame hAkida, doDaDx kene(yuLaLx) keVku. 
\emng
\eentry

\bentry
\word{gate-bill}
\pron{geVTfbilf}
\gl{\nA}
\bmng
 (AkfsxphaDfR matutx keVMbirxjf kAleVjugaLalilx) 
\banum
\alnum{a} viLaMba dAKale; sAnxtakapUvaR vidAyxthiRyu kAleVjige veVLe mIri baMda dinagaLa dAKale. 
\alnum{b} viLaMbadaMDa; idakAkxgi vidhisida daMDa, julAmxne. 
\eanum
\emng
\eentry

\bentry
\word{gatecrash}
\pron{geVTfkArxYxSf}
\gl{\sakirx}
\bmng
 (saMtoVSakUTa \mo vakekx) kareyade nugugx (\akirx\ saha). 
\emng
\eentry

\bentry
\word{gatecrasher}
\pron{geVTfkArxYxSarf}
\gl{\nA}
\bmng
(\ashi) nugAgxLi; anAhUta vayxkitx; saMtoVSakUTa \mo vakekx kareyade nugugxvavanu. 
\emng
\eentry

\bentry
\word{gatefold}
\pron{geVTfphoVlfDx}
\gl{\nA}
\bmng
 maDicu hALe; niyatakAlika \mo vugaLalilx puTada aLategiMta agalavAda BAgavanunx maDiciTiTxruva hALe. 
\emng
\eentry

\bentry
\word{gatehouse}
\pron{geVTfhwsf}
\gl{\nA}
\bmng
\bnum
\num{1} (udAyxna \mo vugaLa) bAgilumane; dAvxragaqha; bAgilinalilxruva kAvalugArana mane. 
\num{2} (\ca) mahAdAvxragaqha; Urina horabAgilina meVliruva, halavomemx seremaneyAgi baLasuva kAvalumane. 
\num{3} dugaRda parxveVshagaqha; dugaRda oLakekx parxveVsha kalipxsuva kaTaTxDa. 
\enum
\emng
\eentry

\bentry
\word{gatekeeper}
\pron{geVTfkiVparf}
\gl{\nA}
\bmng
\hypertarget{gatekeeper(1)}{} 
\bnum
\num{1} dAvxrapAlaka; bAgilALu; parxtiVhAri; geVTu, bAgilu kAyuvavanu. 
\num{2} (\birx) kaMdubaNaNxda, doDaDx ciTeTx. 
\enum
\emng
\eentry

\bentry
\word{gateleg}
\pron{geVTflegf}
\gl{\gu}
\bmng
  = \hyperlink{gatelegged}{gatelegged}. 
\emng
\eentry

\bentry
\word{gatelegged}
\pron{geVTflegfDx}
\gl{\gu}
\bmng
 (meVjina \vi) geVTu kAlina; geVTinaMtaha cwkaTiTxna meVle joVDisida matutx maDisi iLibiDuvaMtha meVlu halageyuLaLx. 
\emng
\eentry

\bentry
\wordnospeech{gatelegged table}{gatelegged table}
\pron{?}
\gl{\nA}
\bmng
geVTukAlina meVju.  \imglink{gatelegged tablefigure}{\raisebox{-0.20cm}[0pt][0pt]{\pdfimage width 0.6cm height 0.6cm {G_Pictures/gatelegged table.jpg}}} 
\emng
\eentry

\bentry
\word{gateman}
\pron{geVTfmanf}
\gl{\nA}
\bmng
  = \hyperlink{gatekeeper(1)}{gatekeeper (1)}. 
\emng
\eentry

\bentry
\word{gate-meeting}
\pron{geVTfmITiMgf}
\gl{\nA}
\bmng
 geVTu vasUli saBe; geVTu baLi duDuDx koTuTx parxveVshisabeVkAda saBe. 
\emng
\eentry

\bentry
\word{gate-money}
\pron{geVTf mani}
\gl{\nA}
\bmng
  = \hyperlink{gate(1)7b}{$^1$gate (7b)}. 
\emng
\eentry

\bentry
\word{gatepost}
\pron{geVTfpoVsfTx}
\gl{\nA}
\bmng
 geVTukaMba; horabAgila kaMba; mahAdAvxra satxMBa; geVTige AdhAravAgiruva kaMba. 
\emng

\noindent
\gl{\nuga}
\bmng
 \eng{between you and me and the gatepost} namamxnamamxlelxV; kaTeTxVkAMtadalilx; guTeTxMdare guTiTxnalilx; bahaLa AMtayaRdalilx. 
\emng
\eentry

\bentry
\word{gateway}
\pron{geVTfveV}
\gl{\nA}
\bmng
\bnum
\num{1}  = \hyperlink{gate(1)}{$^1$gate(1)}. 
\num{2} bAgila cwkaTuTx; geVTina cwkaTuTx. 
\num{3} bAgila geVTina meVle kaTiTxda goVpura \mo\ kaTaTxDa. 
\num{4} bAgilu; dAvxra; parxveVshada, nigaRmanada dAvxra (\rUpa\ saha). 
\enum
\emng
\eentry

\bentry
\word{gather}
\pron{gAYxdarf}
\gl{\sakirx}
\bmng
\bnum
\num{1} oTuTxgUDisu; oTeTxYsu; oTiTxge -- taru, seVrisu, kUDisu. 
\num{2} kUDahAku; saMgarxhisu; saMcayisu; saMkalisu; sheVKarisu. 
\num{3} (hUvanunx) saMgarxhisu; saMcayamADu; hekukx; tiri; Ayu; biDisu; koyuyx; kiVLu. 
\num{4} (kuyilina dhAnayxvanunx) oTuTxgUDisu; saMgarxhisu; sheVKarisu. 
\num{5} (shakitx, tArxNa) taMduko; kUDisiko: \eng{invalid is gathering strength} roVgiyu shakitxyanunx kUDisikoLuLxtitxdAdxne. 
\num{6} (usiru) taMduko; kUDisiko: \eng{gathering my breath I resumed the climb} usirugUDisikoMDu punaH EratoDagide. 
\num{7} Uhisu; garxhisu; anumAnisu; takiRsu; yAvudeV AdhAradiMda niNaRyisu: \eng{I gather that he is the real leader} avaneV nijavAda muMdALu eMdu nAnu UhisutetxVne, takiRsutetxVne. 
\num{8} (vasatxrX, uDupu,\mo vanunx) maDikemaDikeyAgi seVrisu; teretereyAgi joVDisu; nirigenirigeyAgi kaTuTx. 
\num{9} (hububx) gaMTuhAkiko. 
\num{10} neladiMda etitxko, Ayudxko. 
\num{11} (keYkAlu, deVha, \mo vanunx) maDiciko; muduru; saMkucisu. 
\num{12} (asatxvayxsatxvAgiruva saMgatigaLanunx, satAyxMshagaLanunx) oTuTxgUDisu; saMgarxhisu; korxVDiVkarisu. 
\num{13} (yAvudeV kAyaR, parxyatanxkAkxgi AloVcane, \mo vanunx) kUDisiko: \eng{He rose, gathering up his thoughts} tananx AloVcanegaLanenxlalx kUDisikoMDu avanu meVlakekxdadx. 
\num{14} hecAcxgi paDe; adhikavAgi kUDisiko; hecucxvariyAgi hoMdu: \eng{books gathering dust} dhULu tuMbutitxruva pusatxkagaLu. \eng{train gathered speed} reYlu veVgavAyitu, veVga hecicxsikoMDitu. 
\num{15} (pusatxkada raTuTxkaTuTxva kelasadalilx) mudirxta hALegaLanunx anukarxmadalilx joVDisu. 
\enum
\emng

\noindent
\gl{\akirx}
\bmng
\bnum
\num{1} oTuTxgUDu; oTiTxge seVru; oTATxgu. 
\num{2} guMpugUDu; guMpAgi seVru; sAmUhikavAgi nere. 
\num{3} rAshiyAgu; samUhavAgu; saMcayavAgu; seVrikoLuLxtAtx hoVgu; seVrikoMDu beLe; aMshagaLu rAshigUDutAtx, beLeyutAtx hoVgu: \eng{the tale gathered like a snowball} A kateyu himada uMDeyaMte (hosa hosa aMshagaLanunx kUDisikoLuLxtAtx) beLeyitu. 
\num{4} (gAya \mo vugaLa \vi) pakavxvAgu; haNANxgu; tuMbikoMDu pakavxsithxtige baru; kiVvugaTiTx UdutAtx hoVgi varxNa oDeyuvaMtAgu: \eng{a redness arose in the skin, swelled, gathered and burst} camaR keMpAgi, ubibxkoMDu huNANxgi konege oDeduhoVyitu. 
\num{5} (\rUpa) (\sA\ viSamasithxtiya \vi) paripakavxvAgu; hadakekx baru; soPxVTaka sithxtige baru. 
\enum
\emng

\noindent
\gl{\pagu}
\bmng
 \eng{gather way} (haDagina \vi) calisalAraMBisu; calisatoDagu: \eng{the ship soon gathered way} haDagu beVga calisatoDagitu. 
\emng

\noindent
\gl{\nuga}
\bmng
\bnum
\num{1} \eng{be gathered to one's fathers} pitaqloVka seVru; sAyu. 
\numi{2} \eng{gather head} 
\banum
\alnum{a} shakitx gaLisu; balagUDisiko: \eng{I will go to a health resort and gather head} AroVgayxdhAmakekx hoVgi nAnu balagUDisikoLuLxtetxVne. 
\alnum{b} vadhiRsutAtx hoVgu; beLeyutAtx hoVgu: \eng{sin gathering head shall lead to death} pApa beLeyutAtx hoVgi sAvige oyuyxtatxde. 
\eanum
\numie
\num{3} \eng{rolling stone gathers no} \hyperref{kandict_m.pdf}{M}{moss pagu(1)}{$^1$moss}. 
\enum
\emng
\eentry

\bentry
\wordnospeech{gathered skirt}{gathered skirt}
\pron{?}
\gl{\nA}
\bmng
 naDuvina yA soMTada hatitxra niri hiDida tuMDulaMga, sakxTuR. 
\emng
\eentry

\bentry
\word{gatherer}
\pron{gAYxdararf}
\gl{\nA}
\bmng
\bnum
\num{1} saMgArxhaka; saMgarxhakAra; saMcayi; saMkalanakAra; saMgarxhisuva vayxkitx: \eng{a gatherer of moral anecdotes} niVtikathegaLa saMgarxhakAra. 
\num{2} (rusumu, terige, \mo vugaLa) vasUligAra; avugaLa haNavanunx saMgarxhisuvava. 
\enum
\emng
\eentry

\bentry
\word{gathering}
\pron{gAYxdariMgf}
\gl{\nA}
\bmng
\bnum
\num{1} saMgarxhaNa; saMcayana; saMkalana; oTuTxgUDisuvudu; oTeTxYsuvudu; oTiTxge -- taruvudu, seVrisuvudu, kUDisuvudu. 
\num{2} sheVKaraNe; kUDihAkuvudu; saMgarxhisuvudu; sheVKarisuvudu. 
\num{3} oTuTxgUDuvudu; oTATxguvudu; oMdAguvudu. 
\num{4} oMdAgi beLeyuvudu, rAshiyAguvudu. 
\num{5} (hUgaLa \vi) saMcayana; Ayuvudu; biDisuvudu; kiVLuvudu. 
\num{6} nelada meVle bididxruva vasutxvanunx etitxkoLuLxvudu, bAcikoLuLxvudu. 
\num{7} (vasatxrX, uDupu, \mo vanunx) nirigaTiTxsuvudu; maDikemaDikeyAgi maDicuvudu. 
\num{8} (hububx) gaMTumADikoLuLxvudu; gaMTikikxkoLuLxvudu, 
\num{9} (\kanmu) oDeyuva, biriyuva sithxti muTiTxruva Uta; kiVvu tuMbida kuru. 
\num{10} nere; neravi; kUTa; saBe; samUha. 
\num{11} (raTuTx kaTuTxvikeyalilx) oTiTxge tegedukoMDa anukarxmavAda mudirxta hALegaLa taMDa, guMpu. 
\enum
\emng
\eentry

\bentry
\word{gathering-coal}
\pron{gAYxdariMgfkoVlf}
\gl{\nA}
\bmng
 (rAtirx beMki Aradiralu yA gUDina beMbUdiyalilx) hUtiTaTx doDaDx kalilxdadxla gaTiTx. 
\emng
\eentry

\bentry
\word{gathers}
\pron{gAYxdarfs'}
\gl{\nA}
\bmng
 (\bava) nirige; nirige mADida, nirige hiDida uDupina BAga. 
\emng
\eentry

\bentry
\wordRemoveSpace{Gatling-gun}{Gatling gun}
\pron{gAYxTilxMgf ganf}
\gl{\nA}
\bmng
goMcalukoVvi(gaLiMda kUDida) PiraMgi. 
\emng
\eentry

\bentry
\word{gator}
\pron{geVTarf}
\gl{\nA}
\bmng
(\ame, \AmA) = \hyperref{kandict_a.pdf}{A}{alligator}{alligator}. (\saMkiSx) 
\emng
\eentry

\bentry
\word{GATT}
\pron{gAYxTf}
\gl{\saMkiSx}
\bmng
 \eng{ General Agreement on Tariffs and Trade.} 
\emng
\eentry

\bentry
\word{gauche}
\pron{goVSf}
\gl{\gu}
\bmng
\bnum
\num{1} samayajAcnxnavilalxda; shiSATxcAra, samayAcAra ariyada; aucitayx jAcnxnavilalxda; samAjada riVtirivAjugaLu gotitxlalxda. 
\num{2} nayanAjUkariyada; naDenuDiyalilx nayavilalxda. 
\num{3} oDuDx; oraTu; goDuDx; janaroDane sariyAgi bereyuva, vayxvaharisuva jANemx ilalxda. 
\enum
\emng
\eentry

\bentry
\word{gaucheness}
\pron{goVSfnisf}
\gl{\nA}
\bmng
\bnum
\num{1} samayAcArada, shiSATxcArada aBAva; aucitayxjAcnxnavilalxdiruvike; samAjada riVtiniVti, naDenuDigaLa arivilalxdiruvike. 
\num{2} nayanAjUkilalxdiruvike; naDenuDigaLa arivilalxdiruvike. 
\num{3} oDuDxtana; oraTutana; janaroDane sariyAgi bereyuva jANemx ilalxdiruvike. 
\enum
\emng
\eentry

\bentry
\word{gaucherie}
\pron{goVSariV}
\gl{\nA}
\bmng
\bnum
\num{1} oraTu naDate; oDuDx naDate; shiSATxcAravalalxda naDate. 
\num{2} oraTu kelasa; oDoDxDADxda kAyaR; nayavilalxda naDevaLike yA vataRne. 
\enum
\emng
\eentry

\bentry
\word{gaucho}
\pron{gA(gw)coV}
\gl{\nA}
\expl{(\bava\ \eng{gauchos}).}
\bmng
 yUroVpiyanf matutx amerikada iMDiyanf seVri huTiTxda misharx buDakaTiTxna, dakiSxNa amerikadalilx kudureyeVri dana kAyuva guMpinavanu. 
\emng
\eentry

\bentry
\word{gaud}
\pron{gADf}
\gl{\nA}
\bmng
\bnum
\num{1} thaLakupaLakina vasutx. 
\num{2} beDagina, ADaMbarada oDave. 
\num{3} (\bava\ dalilx) (ADaMbarada) utasxvagaLu; samAraMBagaLu; parxdashaRnagaLu; vinoVdagaLu. 
\enum
\emng
\eentry

\bentry
\word{gaudily}
\pron{gADili}
\gl{\kirxvi}
\bmng
 atAyxDaMbaradiMda; ati beDaginiMda; thaLakupaLakAgi; alaMkArAtireVkadiMda. 
\emng
\eentry

\bentry
\word{gaudiness}
\pron{gADinisf}
\gl{\nA}
\bmng
 atibeDagu; atAYxDaMbara; thaLakupaLaku mereta; atireVkada alaMkaraNa; oLeLxya aBiruci, aucitayxjAcnxnagaLilalxde aMdagoLisuvike, parxdashiRsuvike. 
\emng
\eentry

\bentry
\word[gaudy(1)]{gaudy}
\pron{gADi}
\gl{\nA}
\bmng
 (\birx) (\kanmu\ kAleVjinalilx haLeya vidAyxthiRgaLu \mo varigAgi koDuva) vASiRka BoVjanakUTa; doDaDx saMtoVSakUTa. 
\emng
\eentry

\bentry
\word[gaudy(2)]{gaudy}
\pron{gADi}
\gl{\gu}
\bmng
\bnum
\num{1} beDagina yA aucitayx jAcnxnavilalxda; ADaMbarada; thaLukupaLakina; oLeLxya aBiruciyilalxde yA aucitayx jAcnxnavilalxde aMdagoLisida, alaMkarisida. 
\num{2} (uDupu, alaMkaraNa, sAhitayxsheYli, \mo vugaLalilx) atayxlaMkArada; alaMkArAtireVkada; atibeDagina. 
\enum
\emng
\eentry

\bentry
\word{gaudy-day}
\pron{gADiDeV}
\gl{\nA}
\bmng
\bnum
\num{1} utasxvadina; saMtoVSakUTada dina. 
\num{2} kAleVjina vASiRkoVtasxvada dina. 
\enum
\emng
\eentry

\bentry
\word[gauge(1)]{gauge}
\pron{geVjf}
\gl{\nA}
\bmng
\bnum
\num{1} geVju; vasutxgaLanunx tayArisuvalilx anusarisaleVbeVkAda, \kanmu\ piVpAyiya oLagaNa Gana aLateya, tupAki guMDina vAyxsada, nUlineLeya navurina, kabibxNada tagaDina, dapapxda parxmANada aLate. 
\num{2} vAyxpanashakitx; eSuTx dUra haraDabalulxdoV aSuTx shakitx. 
\num{3} vAyxpitx; visAtxra; haravu. 
\num{4} (reYlina) geVju; parxmANaka (aLate); reYlu mAgaRdalilx kaMbigaLa yA edurubadurina cakarxgaLa naDuvaNa agala: \eng{broad gauge} bArxDf geVju; parxmANaka aLategiMta hecucx geVjina. \eng{narrow gauge} parxmANaka aLategiMta kaDime geVjina. 
\num{5} (\nw) (gALi biVsuva dikikxgU matotxMdu haDagigU saMbaMdhisidaMte) haDagina sAthxna; haDagoMdu iruva jAga. 
\numi{6} mApaka: 
\banum
\alnum{a} maLeya motatx, hoLeya ubabxraviLita, samudarxda EriLita, gALiya bala, \mo vanunxaLeyuva, aLategurutugaLuLaLx upakaraNa, sAdhana. 
\alnum{b} pAterxya oLage iruva darxva \mo vugaLa etatxravanunx toVrisalu pAterxge tagulisiruva aLeyuva sAdhana. 
\alnum{c} taMti, upakaraNgaLu, \mo vugaLa aLategaLanunx pariVkiSxsi, sarinoVDuva sAdhana. 
\eanum
\numie
\num{7} (samAnAMtara reVKegaLanenxLeyuvudakAkxgi baLasuva, beVkAdaMte hoMdisikoLaLxbalalx) baDagiya reVKakasAdhana. 
\num{8} (mudarxNa) (puTada aMcina agala \mo vanunx takakxMte hoMdisikoLaLxlu baLasuva) aLateya paTiTx. 
\num{9} aMdAju; aMdAju mADalu AdhAravAda sAdhana, vidhAna, upAya. 
\num{10} manadaMDa; aLategoVlu; oregalulx; nikaSa; oMdu vasutx yA viSayada mwlayxvanunx gotutxhacacxlu beVkAda mApaka. 
\enum
\emng

\noindent
\gl{\pagu}
\bmng
\bnum
\numi{1} \eng{have the weather gauge of} 
\banum
\alnum{a} (haDagina \vi)(matotxMdu haDagiruva sAthxnakekx saMbaMdhisidaMte) gALiya dikikxge iru; gALi biVsuva kaDegiru. 
\alnum{b} (\rUpa) anukUla sithxtiyalilxru; meVlugeYyAgiruva sAthxnadalilxru; edurALiyanunx soVlisuva sithxtiyalilxru. 
\eanum
\numie
\num{2} \eng{take the gauge of} aLe; aMdAju mADu. 
\enum
\emng
\eentry

\bentry
\word[gauge(2)]{gauge}
\pron{geVjf}
\gl{\sakirx}
\bmng
\bnum
\num{1} (\kanmu\ taMti, boVluTx, \mo\ parxmANaka gAtarxda vasutxgaLanunx, motatxdalolxV baladalolxV EriLitagaLuLaLx maLe, gALi, \mo vanunx yAvudeV dhArakadalilxruva darxvada ALavanunx) mApana mADu; niSakxqqSaTxvAgi aLe. 
\num{2} (piVpAyi \mo vugaLa) oLagaNa Gana aLateyanunx (gaNitadiMdaloV aLeyuvudariMdaloV) kaMDuhiDi. 
\num{3} (vayxkitx, shiVla, \mo vanunx) aMdAjumADu; aLe. 
\numi{4} 
\banum
\alnum{a} parxmANaka gAtarxkekx, aLatege, yA AkArakekx taru. 
\alnum{b} hiVge EkarUpagoLisu. 
\eanum
\numie
\enum
\emng
\eentry

\bentry
\word{gaugeable}
\pron{geVjabflf}
\gl{\gu}
\bmng
mApayx: 
\banum
\alnum{a} aLeyalu sAdhayxvAda. 
\alnum{b} suMkada vasUligAgi, aLateya taniKege oLapaDuva. 
\eanum
\emng
\eentry

\bentry
\wordnospeech{gauge pressure}{gauge pressure}
\pron{?}
\gl{\nA}
\bmng
 geVjf otatxDa; vAtAvaraNada otatxDakikxMta hecicxge otatxDa iruvAgina hecucxvari otatxDa. 
\emng
\eentry

\bentry
\word{gauger}
\pron{geVjarf}
\gl{\nA}
\bmng
\bnum
\numi{1}mApaka: 
\banum
\alnum{a} aLategAra; aLate mADuvava. 
\alnum{b} aLate mADuva upakaraNa yA yaMtarx. 
\eanum
\numie
\num{2} seVMdimApaka; seVMdiya meVle tagaluva suMkavasUligAgi piVpAyi \mo vugaLalilxruva seVMdiyanunx aLedu pariVkiSxsuva adhikAri. 
\num{3} oLasuMkada adhikAri; madayx, taMbAku, \mo vugaLa meVlina oLanADina suMkavanunx vasUlu mADuva adhikAri. 
\num{4} guNamaTaTx mApaka; yaMtarxgaLiMda tayArisida vasutxgaLa aLate yA guNagaLanunx pariVkiSxsuvavanu. 
\enum
\emng
\eentry

\bentry
\word{Gaul}
\pron{gAlf}
\gl{\nA}
\bmng
\bnum
\num{1} pArxciVna kAladalilx, phArxnisxna gAlf parxdeVshadavanu. 
\num{2} (\hA) phArxnfsx deVshadavanu. 
\enum
\emng
\eentry

\bentry
\word{gauleiter}
\pron{gwleYTarf}
\gl{\nA}
\bmng
\bnum
\num{1} (nAji pakaSxda) jilAlxnAyaka; DisiTxrXkiTxna rAjakiVya muMdALu. 
\num{2} (\rUpa) pALeyagAra; sathxLiVya yA cikakx dabAbxLikegAra. 
\enum
\emng
\eentry

\bentry
\word[Gaulish(1)]{Gaulish}
\pron{gAliSf}
\gl{\gu}
\bmng
\bnum
\num{1} pArxciVna gAlf parxdeVshada -- nivAsiya, BASeya. 
\num{2} (\hA) pherxMcina; phArxnisxnavana. 
\enum
\emng
\eentry

\bentry
\word[Gaulish(2)]{Gaulish}
\pron{gAliSf}
\gl{\nA}
\bmng
\bnum
\num{1} pArxciVna gAlf parxdeVshada keliTxkf BASe. 
\num{2} (\hA) phArxsinxnavanu. 
\enum
\emng
\eentry

\bentry
\word{Gaullism}
\pron{goVlisaZmf}
\gl{\nA}
\bmng
goVlisaM: 
\banum
\alnum{a} pherxMcf rAjakiVya hAgU miliTari nAyakanAda cAlfs'R Da gAlana (maraNa \eng{1970}) tatatxvXgaLu matutx niVtigaLu. 
\alnum{b} I tatatxvXgaLanunx matutx niVtigaLanunx beMbalisuvudu yA pAlisuvudu. 
\eanum
\emng
\eentry

\bentry
\word{Gaullist}
\pron{goV(gA)lisfTx}
\gl{\nA}
\bmng
 gAlisuTx; cAlfs'R Da gAlana tatatxvXgaLanUnx niVtigaLanUnx beMbalisuvava. 
\emng
\eentry

\bentry
\word{gault}
\pron{gAlfTx}
\gl{\nA}
\bmng
 (\BUvi) gAlfTx: 
\banum
\alnum{a} dakiSxNa iMgelxMDina hasiru maraLugalilxna padaragaLa naDuve iruva jeVDi matutx suNaNxkalulx jeVDimaNuNxgaLa padaragaLu. 
\alnum{b} I padaragaLiMda paDeda jeVDimaNuNx. 
\eanum
\emng
\eentry

\bentry
\word{gaultheria}
\pron{gAlitxaria}
\gl{\nA}
\bmng
gAlitxVriya kulada oMdu bageya nitayxhasiru vaNaRda suvAsanA sasayx (ivugaLalilx kelavu bageyavu auSadhige baLasuva eNeNxyanunx koDutatxve). 
\emng
\eentry

\bentry
\word{gaunt}
\pron{gAMTf}
\gl{\gu}
\bmng
\bnum
\num{1} kaqsha; sapUra; narapeVtala; saNakalAda; baDakalAgiruva; (hasivu, anAroVgayx, kAyile, kaSaTx kApaRNayx, \mo vugaLiMda) niVLavAgi, teLaLxge, mULe biTuTxkoMDiruva. 
\num{2} bikoV enunxvaMtha; hALu suriyuvaMtiruva: \eng{the deserted house stood before us, gaunt and forlorn} maneyavarelalx biTuTxhoVgidadx A maneyu bikoV eMdu hALu suriyutAtx namemxdurinalilx niMtitutx. 
\enum
\emng
\eentry

\bentry
\word[gauntlet(1)]{gauntlet}
\pron{gAMTfliTf}
\gl{\nA}
\bmng
\bnum
\num{1} (\ca) keYpoDe; ukikxna keYkavaca; hasatxtArxNa; ukikxna keYgavusu; togalina meVle ukikxna tagaDu hodisida, maNikaTiTxgiMtalU meVlakekx haraDi muMgeY mucucxva, ukikxna keYciVla.  \imglink{gaunlet-1figure}{\raisebox{-0.20cm}[0pt][0pt]{\pdfimage width 0.5cm height 0.5cm {G_Pictures/gaunlet-1.jpg}}} 
\num{2} (moVTAru vAhana naDesuvudu, kudure savAri mADuvudu, katitxvarase mADuvudu, kirxkeTfnalilx vikeTf kAyuvudu, \mo vugaLigAgi baLasuva, maNikaTiTxgiMtalU meVlakekx vAyxpisuva, aLaLxkavAda) keY -- ciVla, gavusu, kApu. \imglink{gauntlet-2figure}{\raisebox{-0.15cm}[0pt][0pt]{\pdfimage width 0.5cm height 0.5cm {G_Pictures/gauntlet-2.jpg}}} 
\num{3} keYgavusina (agalavAda) maNikaTiTxna BAga. 
\enum
\emng

\noindent
\gl{\nuga}
\bmng
\hyperdef{G}{gauntlet(1) nuga(1)}{} 
\bnum
\numi{1} \eng{fling} (\engit{or} \eng{throw) down the gauntlet} 
\banum
\alnum{a} kALagakekx, seNasATakekx kare; peYje hUDu; samarakAkxgi savAlu hAku. 
\alnum{b} (idara saMkeVtavAgi) ukikxna keYkavaca ese; muDigeyikukx. 
\hyperdef{G}{gauntlet(1) nuga(2)}{} 
\eanum
\numie
\numi{2} \eng{pick} (\engit{or} \eng{take) up the gauntlet} 
\banum
\alnum{a} kALagakekxMdu koTaTx kareyanonxpipxko; samarakAkxgi koTaTx savAlu sivxVkarisu, savAlopupx. 
\alnum{b} (idara saMkeVtavAgi edurALiya keLageseda) keYkavacavanenxtitxko. 
\eanum
\numie
\enum
\emng
\eentry

\bentry
\word[gauntlet(2)]{gauntlet}
\pron{gAMTfliTf}
\gl{\nA}
\bmng
\bnum
\num{1} tapipxtasathx vidAyxthiRge shAleyalilx, tapipxtasathx seYnikanige BU yA nwkA seVneyalilx vidhisuva, eduru baduru sAlinalilx doNeNx hiDidu niMtiruvavara ETugaLige guriyAgi, naDuve hAduhoVguva oMdu shikeSx. 
\num{2} viSama pariVkeSx; tiVvarx parisithxti. 
\enum
\emng

\noindent
\gl{\pagu}
\bmng
 \eng{run the gauntlet} 
\banum
\alnum{a} ETina hAdiyalilx sAgu; daMDanapathavanunx hAyu; seVneya, nwkAseVneya, vidAyxshAleya shikeSx anuBavisalu ikekxlagaLiMdalU doNeNxgaLu, hagagxgaLu, \mo vugaLiMda baDiyuva janagaLa eraDu sAlugaLa naDuve -- naDe, sAgu, ODu. 
\alnum{b} (\rUpa) ikukxLakekx sikukx; eraDU kaDeya ETige, hoDetakekx sikukx; uBayatarx TiVkege, KaMDanege, apAyakekx sikikxko; uBayAGAtakekx guriyAgu. 
\eanum
\emng
\eentry

\bentry
\word{gauntleted}
\pron{gAMTfliTiDf}
\gl{\gu}
\bmng
 ukikxna keYkApu dharisida; ukikxna keYkApiniMda mucicxda keYyuLaLx. 
\emng
\eentry

\bentry
\word{gauntness}
\pron{gAMTfnisf}
\gl{\nA}
\bmng
\bnum
\num{1} kaqshate; kAshayxR; kaqshavAgiruvike; saNakalutana; baDakalutana. 
\num{2} bikoV enunxvaMtiruvike; hALu suriyuvaMtiruvike. 
\enum
\emng
\eentry

\bentry
\word{gauntry}
\pron{gAMTirx}
\gl{\nA}
\bmng
  = \hyperlink{gantry}{gantry (1)}. 
\emng
\eentry

\bentry
\word{gaur}
\pron{gwarf}
\gl{\nA}
\bmng
kATi; \pU ESayxda -- kADetutx, vanavaqSaBa. 
\emng
\eentry

\bentry
\word{gauss}
\pron{gwsf}
\gl{\nA}
\expl{(\bava\ adeV yA \eng{gaussed}).}
\bmng
(\Bwvi) gwsf; kAMtiVya perxVraNeya viduyxtAkxMtamAna. 
\emng
\eentry

\bentry
\word{gauze}
\pron{gAsfZ}
\gl{\nA}
\bmng
\bnum
\num{1} navirujAlari; reVSemx, hatitxnUlu, taMti, \mo vugaLiMda mADida teLuvAda, pAradashaRkavAda heNige, baTeTx, vasatxrX. 
\num{2} (teLuvAda) himadhUma; maMjina tere, musuku. 
\enum
\emng
\eentry

\bentry
\word{gauzily}
\pron{gAsiZli}
\gl{\kirxvi}
\bmng
 jAlariyaMte; navirAgiyU pAradashaRkavAgiyU iruvaMte. 
\emng
\eentry

\bentry
\word{gauziness}
\pron{gAsiZnisf}
\gl{\nA}
\bmng
jAlariyaMtiruvike; navirAgi pAradashaRkavAgiruvike. 
\emng
\eentry

\bentry
\word{gauzy}
\pron{gAsiZ}
\gl{\gu}
\bmng
\bnum
\num{1} jAlariyaMtiruva; navirAgi pAradashaRkavAgiruva. 
\num{2} (jAlarijAlariyAda) uDupinaMte yA hodikeyaMte iruva. 
\enum
\emng
\eentry

\bentry
\word{gave}
\pron{geVvf}
\gl{\kirx}
\bmng
 \eng{give} dhAtuvina BUta rUpa. 
\emng
\eentry

\bentry
\word[gavel(1)]{gavel}
\pron{gAYxvalf}
\gl{\nA}
\bmng
 saNaNxsutitxge; cikakx koDati; gamana seLeyalu, harAjugAra, saBAdhayxkaSx yA nAyxyAdhipati baLasuva keYsutitxge, cikakx koDati. \imglink{gavelfigure}{\raisebox{-0.15cm}[0pt][0pt]{\pdfimage width 0.8cm height 0.5cm {G_Pictures/gavel.jpg}}} 
\emng
\eentry

\bentry
\word[gavel(2)]{gavel}
\pron{gAYxvalf}
\gl{\akirx}
\expl{(\BU\ matutx \BUkaq\ \eng{gavelled}, \vakaq\ \eng{gavelling}).}
\bmng
\bnum
\num{1} (harAjugAra, saBAdhayxkaSx yA nAyxyAdhiVshana \vi gamana seLeyalu yA sadadxDagisalu, tamamx muMdiruva meVju \mo vanunx) keYsutitxgeyiMda, koDatiyiMda -- kuTuTx. 
\num{2} keYsutitxgeyiMdaloV eMbaMte kuTuTx, taTuTx. 
\enum
\emng
\eentry

\bentry
\word{gavelkind}
\pron{gAYxvalfkeYMDf}
\gl{\nA}
\bmng
 (\kanmu\ iMgelxMDina keMTfnalilx rUDhiyalilxdadx) samapAlina hiDuvaLi padadhxti; uyilu bareyade satatxvana Asitxyanunx avana elalx gaMDumakakxLigU samaBAgavAguvaMte haMcuva jamInu hiDuvaLi padadhxti. 
\emng
\eentry


\bentry
\word{gavial}
\pron{geVvialf}
\gl{\nA}
\bmng
udadxmUtiya ESAyxda mosaLe.  \imglink{gavialfigure}{\raisebox{-0.10cm}[0pt][0pt]{\pdfimage width 0.8cm height 0.5cm{G_Pictures/gavial.jpg}}} 
\emng
\eentry

\bentry
\word{gavotte}
\pron{gavATf}
\gl{\nA}
\bmng
\bnum
\num{1} (\eng{18}neya shatamAnada) gavATf naqtayx; oMdu bageya maMdagatiya (jAnapada) naqtayx. 
\num{2} gavATf saMgiVta; gavATf naqtayxkAkxgi racisida saMgiVta. 
\num{3} kaqtiya parxtiyoMdu savxrapuMjavU sAthxyiV reVKAvaLiya taqtiVya tALadiMda pArxraMBisuva, sAmAnayx layada yA madhayxmakAlada saMgiVta kaqti. 
\enum
\emng
\eentry

\bentry
\word{Gawd}
\pron{gADf}
\gl{\nA}
\bmng
 deVvaru (\eng{God} padada asaMsakxqqta yA ashiSaTx \ucAcx). 
\emng
\eentry

\bentry
\word[gawk(1)]{gawk}
\pron{gAkf}
\gl{\nA}
\bmng
 shudadhx daDaDx; maDiDx; maDeya; muThAThxLa; maMka; modadx. 
\emng
\eentry

\bentry
\word[gawk(2)]{gawk}
\pron{gAkf}
\gl{\akirx}
\bmng
 (\AmA) daDaDxtanadiMda noVDu; pedudxpedAdxgi noVDu. 
\emng
\eentry

\bentry
\word{gawkily}
\pron{gAkili}
\gl{\kirxvi}
\bmng
 oDoDxDADxgi; vakarxvakarxvAgi; citarxvicitarxvAgi. 
\emng
\eentry

\bentry
\word{gawkiness}
\pron{gAkinisf}
\gl{\nA}
\bmng
 vakarxvakarxvAgiruvike; oDoDxDADxgiruvike. 
\emng
\eentry

\bentry
\word[gawky(1)]{gawky}
\pron{gAki}
\gl{\gu}
\bmng
\bnum
\num{1} vakarxvakarxvAda; oDoDxDADxda. 
\num{2} muKaheVDiyAda. 
\enum
\emng
\eentry

\bentry
\word[gawky(2)]{gawky}
\pron{gAki}
\gl{\nA}
\bmng
\bnum
\num{1} aSATxvakarx; vakarxvakarxvAda vayxkitx. 
\num{2} muKaheVDi. 
\enum
\emng
\eentry

\bentry
\word{gawp}
\pron{gApf}
\gl{\akirx}
\bmng
  = \hyperlink{gawk(2)}{$^2$gawk}. 
\emng
\eentry

\bentry
\word[gay(1)]{gay}
\pron{geV}
\gl{\gu}
\expl{(\tara\ \eng{gayer,} \tama\ \eng{gayest}).}
\bmng
\bnum
\num{1} gelavina; nalivina; gelavanunx sUcisuva. 
\num{2} gelavu tuMbida; ulAlxsaBarita; ulAlxsapUNaR. 
\num{3} galavina savxBAvada; ulAlxsashiVla; ulAlxsa -- parxvaqtitxya, parxkaqtiya. 
\num{4} nirAloVcaneya; ArAmavAda; nishicxMteya; nirAtaMkavAda; hAyAgiruva savxBAvada; hagura haqdayada; yAvudanUnx manasisxge hacicxkoLaLxda. 
\num{5} (\sw) niVtigeTaTx; niVtiBarxSaTx; vilAsiyAda; viSayalaMpaTanAda; laMDanAda. 
\num{6} (heMgasina \vi) vilAsiniyAda; vayxBicAradiMda badukuva. 
\num{7} beDagina; thaLaku paLakina; ADaMbarada. 
\num{8} hoLapina; ujajxvXla; ujajxvXlavaNaRda; thaLathaLisuva baNaNxda. 
\num{9} SoVki; sogasAda uDupu dharisida; cenAnxgi siMgarisikoMDa: \eng{the King wanted to know this gay gentleman} I SoVkilAla mahaniVya yAreMdu arasa ariyabayasida. 
\num{10} (\ashi) saliMgakAmiyAda. 
\num{11} (\ashi)(sathxLada \vi) saliMgakAmigaLu padeV padeV BeVTimADuva; saliMgakAmigaLigAgi iruva yA saliMgakAmigaLu baLasuva: \eng{a gay bar} saliMgakAmigaLa bAru, madayxdaMgaDi. 
\enum
\emng
\eentry

\bentry
\word[gay(2)]{gay}
\pron{geV}
\gl{\nA}
\bmng
(\ashi) saliMgakAmi (\kanmu\ gaMDasu). 
\emng
\eentry

\bentry
\word{gayety}
\pron{geVaTi}
\gl{\nA}
\bmng
 (\ame)  = \hyperlink{gaiety}{gaiety}. 
\emng
\eentry

\bentry
\word{gayness}
\pron{geVnisf}
\gl{\nA}
\bmng
  = \hyperlink{gaiety}{gaiety}. 
\emng
\eentry

\bentry
\word{gazabo}
\pron{gaseZVboV}
\gl{\nA}
\expl{(\bava\ \eng{gazabos}).}
\bmng
(\ame) (\ashi, aneVka veVLe \hiV) AsAmi; kuLa; vayxkitx. 
\emng
\eentry

\bentry
\word{gazania}
\pron{gaseZVnia}
\gl{\nA}
\bmng
 AlaMkArikavAda haLadi yA kitatxLe baNaNxda hUgaLuLaLx, dakiSxNa Aphirxkada oMdu mUlike(yA kula). 
\emng
\eentry

\bentry
\word[gaze(1)]{gaze}
\pron{geVsfZ}
\gl{\akirx}
\bmng
 diTiTxsi noVDu; eveyikakxde noVDu. 
\emng
\eentry

\bentry
\word[gaze(2)]{gaze}
\pron{geVsfZ}
\gl{\nA}
\bmng
 neTaTxnoVTa; diTiTxsida noVTa; eveyikakxda noVTa. 
\emng

\noindent
\gl{\pagu}
\bmng
 \eng{at gaze} AshacxyaR \mo vugaLiMda diTiTxsi noVDutatx, eveyikakxde noVDutatx. 
\emng
\eentry

\bentry
\word[gazebo(1)]{gazebo}
\pron{gasiZVboV}
\gl{\nA}
\expl{(\bava\ \eng{gazebos} yA \eng{gazeboes}).}
\bmng
goVpura; dUrada noVTa kANuvaMte etatxradalilx kaTiTxruva kaTaTxDa; (upapxrigeya) shiKara, goVpura, mogasAle, \mo vu. 
\emng
\eentry

\bentry
\word[gazebo(2)]{gazebo}
\pron{gasiZVboV}
\gl{\nA}
\bmng
  = \hyperlink{gazabo}{gazabo}. 
\emng
\eentry

\bentry
\word{gazelle}
\pron{gaseZlf}
\gl{\nA}
\bmng
 (cikakxdU koVmalavU melunoVTavuLaLxdUdx Ada) oMdu jAtiya jiMke.  \imglink{gazellefigure}{\raisebox{-0.15cm}[0pt][0pt]{\pdfimage width 0.7cm height 0.6cm{G_Pictures/gazelle.jpg}}} 
\emng
\eentry

\bentry
\word{gazer}
\pron{geVsaZrf}
\gl{\nA}
\bmng
 diTiTxsi, eveyikakxde -- noVDuvavanu. 
\emng
\eentry

\bentry
\word[gazette(1)]{gazette}
\pron{gaseZTf}
\gl{\nA}
\bmng
\bnum
\num{1} (\ca) sudidxya hALe. 
\num{2} (\ca) (parxcalita saMgatigaLanunx tiLisuva) niyatakAlika (patirxke). 
\numi{3} geseZTuTx; sakARrada parxkaTana patirxke: 
\banum
\alnum{a} sakARriV neVmakagaLu, divALi Adavaru matutx itara sAvaRjanika parxkaTaNegaLanonxLagoMDa, vArakekxraDu bAri horaDisuva iMgelxMDina sakARrada mUru adhikaqta patirxkegaLalilx oMdu: \eng{London Gazette, Edinburgh Gazette, Belfast Gazette} laMDanf, eDinfbaroV, belfphAsfTx geseZTuTx. 
\alnum{b} kAnUnubadadhx tiLivaLike patarxgaLu, adhikArigaLa neVmaka, baDitx, \mo\ sudidxgaLanonxLagoMDa sakARrada niyatakAlika patirxke. 
\eanum
\numie
\num{4} (kelavu vaqtatxpatirxkegaLa aMkitadalilx baruvAga) vaqtatxpatirxke: \eng{Bermingham Gazette} bamiRMgfhAyxmf geseZTuTx. 
\enum
\emng
\eentry

\bentry
\word[gazette(2)]{gazette}
\pron{gaseZTf}
\gl{\sakirx}
\bmng
 (\birx) (\kanmu\ \kaparx dalilx) sakARriV geseZTiTxnalilx parxkaTisu. 
\emng
\eentry

\bentry
\word{gazetteer}
\pron{gAyxsiZTiarf}
\gl{\nA}
\bmng
 geseZTiyaru; BUvivara niGaMTu; deVshavicArakoVsha; deVsha viSayaka koVsha. 
\emng
\eentry

\bentry
\word{gazogene}
\pron{gAyxsaZjiVnf}
\gl{\nA}
\bmng
 gAyxsoZjiVnf; anilapUraka; anilayukatx pAniVyagaLanunx tayArisuva yaMtarx. 
\emng
\eentry

\bentry
\word{gazpacho}
\pron{gAYxsfpAcoV}
\gl{\nA}
\expl{(\bava\ \eng{gazpachos}).}
\bmng
 taMpu sAMbAru; swtekAyi, IruLiLx, berxDuDx, \mo vanunx hAki beLuLxLiLxya ogagxraNe koTuTx tayArisida, sepxVnina taNaNxneya sAru, sAMbAru. 
\emng
\eentry

\bentry
\word{gazump}
\pron{gasaZMpf}
\gl{\sakirx}
\bmng
 (\ashi) 
\bnum
\num{1} moVsamADu; vaMcisu; ToVpi hAku. 
\num{2} (koLaLxbayasuvavaniMda) bele opipxgeyAda naMtara mane \mo vugaLa bele Erisu, hecicxsu (\akirx\ saha). 
\enum
\emng
\eentry

\bentry
\wordnospeech{GB}{GB}
\pron{?}
\gl{\saMkiSx}
\bmng
 \eng{Great Britain.} 
\emng
\eentry

\bentry
\wordnospeech{GBE}{GBE}
\pron{?}
\gl{\saMkiSx}
\bmng
 (\birx) \eng{Knight} (\engit{or} \eng{Dame) Grand Cross (of the Order) of the British Empire.} 
\emng
\eentry

\bentry
\wordnospeech{GC}{GC}
\pron{?}
\gl{\saMkiSx}
\bmng
 (\birx) \eng{George Cross.} 
\emng
\eentry

\bentry
\wordnospeech{GCB}{GCB}
\pron{?}
\gl{\saMkiSx}
\bmng
 (\birx) \eng{Knight (or Dame) Grand Cross (of the Order of the Bath).} 
\emng
\eentry

\bentry
\wordnospeech{GCE}{GCE}
\pron{?}
\gl{\saMkiSx}
\bmng
 (\birx) \eng{General Certificate of Education.} 
\emng
\eentry

\bentry
\wordnospeech{GCIE}{GCIE}
\pron{?}
\gl{\saMkiSx}
\bmng
 (\birx) \eng{Knight Grand Commander (of the Order) of the Indian Empire.} 
\emng
\eentry

\bentry
\wordnospeech{GCMG}{GCMG}
\pron{?}
\gl{\saMkiSx}
\bmng
 (\birx) \eng{Knight} (\engit{or} \eng{Dame) Grand Cross (of the Order) of St. Michael \& St. George.} 
\emng
\eentry

\bentry
\wordnospeech{GCSI}{GCSI}
\pron{?}
\gl{\saMkiSx}
\bmng
 \eng{Knight Grand Commander (of the Order) of the Star of India.} 
\emng
\eentry

\bentry
\wordnospeech{GCVO}{GCVO}
\pron{?}
\gl{\saMkiSx}
\bmng
 \eng{Knight (or Dame) Grand Cross of the Royal Victorian Order.} 
\emng
\eentry

\bentry
\wordnospeech{Gd}{Gd}
\pron{?}
\gl{\saMkeV}
\bmng
 \eng{gadolinium.} 
\emng
\eentry

\bentry
\wordnospeech{GDn.}{GDn.}
\pron{?}
\gl{\saMkiSx}
\bmng
 \eng{Garden.} 
\emng
\eentry

\bentry
\wordnospeech{Gdns.}{Gdns.}
\pron{?}
\gl{\saMkiSx}
\bmng
 \eng{Gardens.} 
\emng
\eentry

\bentry
\wordnospeech{GDP}{GDP}
\pron{?}
\gl{\saMkiSx}
\bmng
 \eng{gross domestic product.} 
\emng
\eentry

\bentry
\wordnospeech{GDR}{GDR}
\pron{?}
\gl{\saMkiSx}
\bmng
 \eng{German Democratic Republic.} 
\emng
\eentry

\bentry
\wordnospeech{Ge}{Ge}
\pron{?}
\gl{\saMkeV}
\bmng
 \eng{germanium.} 
\emng
\eentry

\bentry
\word{gean}
\pron{jiVnf}
\gl{\nA}
\bmng
 (\birx) (kADu) ceri -- giDa, haNuNx. 
\emng
\eentry

\bentry
\word[gear(1)]{gear}
\pron{giarf}
\gl{\nA}
\bmng
\bnum
\num{1} sajujx sAmagirx. 
\num{2} (\kanmu\ \AmA) yuvakara uDupu, uDigetoDige. 
\num{3} (BAra eLeyuva pArxNigaLa) sajujx saraMjAmu; jatitxge; jotitxge; yoVkatxrX. 
\num{4} yaMtarx -- sAdhana, salakaraNe, : \eng{aircraft's landing gear} vimAnada iLiyuva salakaraNegaLu. 
\num{5} hagagx, kokekx, \mo vugaLanonxLagoMDa sAdhana, parikara. 
\num{6} upakaraNagaLu; hatAyxra; salakaraNegaLu; sAdhanasAmagirx. 
\num{7} (\sA\ vishiSaTx udedxVshakAkxgi) cakarxgaLu, rATegaLu, sanenxgaLu, \mo vugaLa -- vayxvasethx, joVDaNe: \eng{winding gear} sutitxDuva vayxvasethx. 
\num{8} geVru; giyaru; cAlaka; moVTAranunx yA eMjinanunx adara kAyaRkAriyAda aMgagaLige kUDisuva aLavaDike. 
\num{9} geVru; halulxgaLu \mo vugaLa mUlaka oMdakokxMdu taguli kelasa mADuva cakarxgaLu.
\num{10} (haDagina) kUve hagagxgaLu; kUvegaMbakekx bigiyuva hagagxgaLu. 
\num{11} saraku sAmAnugaLu. 
\num{12} maneya pAterx padAthaRgaLu; pAterxpagaDi. 
\enum
\emng

\noindent
\gl{\pagu}
\bmng
\hyperdef{G}{gear(1) pagu(1)}{} 
\hypertarget{gear1 pagu1}{} 
\bnum
\num{1} \eng{bottom gear} (\birx) kaniSaThx geVru; taLa giyaru; beYsikalf, moVTArf kAru, \mo vugaLalilx cAlita BAgagaLu cAlaka BAgagaLigiMta kaniSaThx veVgadalilx sututxtitxruvudu. 
\num{2} \eng{first gear} = \hyperlink{gear1 pagu1}{?pagu? \((1)\)}. 
\num{3} \eng{high gear} ErugeVru; beYsikalf, moVTArf kAru, \mo vugaLalilx cAlita BAgagaLu cAlaka BAgagaLigiMta hecucx veVgadalilx sututxtitxruvudu. 
\num{4} \eng{in gear} moVTArige kUDi yA kUDisi; moVTAru kelasa mADutitxruva sithxtiyalilx. 
\num{5} \eng{low gear} iLigeVru; beYsikalf, moVTArf kAru, \mo vugaLalilx cAlita BAgagaLu cAlaka BAgagaLigiMta kaDime veVgadalilx sututxtitxruvudu. 
\numi{6} \eng{out of gear} 
\banum
\alnum{a} moVTAroDane kUDade yA kUDisade; moVTAru niSikxrXya sithxtiyalilx. 
\alnum{b} (\birx) (\rUpa) karxmageTuTx; sariyAgilalxde. 
\hyperdef{G}{gear(1) pagu(7)}{} 
\eanum
\numie
\num{7} \eng{top gear} gariSaTx geVru; meVlugiyaru; tudigeVru; beYsikalf, moVTArf kAru, \mo vugaLalilx cAlita BAgagaLu cAlaka BAgagaLigiMta hecucx veVgadalilx sututxtitxruvudu. 
\enum
\emng
\eentry

\bentry
\word[gear(2)]{gear}
\pron{giarf}
\gl{\sakirx}
\bmng
\bnum
\num{1} (BAra eLeyuva pArxNige) jatitxge hAku; sajujx saraMjAmu toDisu, aLavaDisu. 
\num{2} (yaMtarxkekx) cAlakadoDane kUDisu, saMbaMdha uMTumADu. 
\num{3} (yaMtarxkekx) cAlaka havaNisu; geVru joVDisu. 
\num{4} (oMdu keYgArikeyanonxV kAKARneyanonxV inonxMdu keYgArikege yA kAKARnege yA kAyaRniVtige) adhiVnagoLisu; anubaMdhiyanAnxgi mADu. 
\num{5} (halulxcakarxda \vi\ yaMtarxBAgavanunx) halulxcakarxda meVle sariyAgi kUruvaMte mADu, joVDisu. 
\num{6} (parisithxti \mo vugaLige) sari hoMdisu: \eng{gear the economy to wartime requirements} yudadhxkAlada agatayxgaLige anuguNavAgi athaRvayxvasethxyanunx sarihoMdisu. 
\enum
\emng

\noindent
\gl{\akirx}
\bmng
\bnum
\num{1} (yaMtarxBAga) halulxcakarxda meVle sariyAgi kUru. 
\num{2} sugamavAgi kelasa mADu; hoMdikoMDu kelasamADu. 
\enum
\emng

\noindent
\gl{\pagu}
\bmng
\bnum
\num{1} \eng{gear down} veVgavanunx tagigxsu. 
\num{2} \eng{gear up} veVga hecicxsu. 
\enum
\emng
\eentry

\bentry
\word{gear-box}
\pron{giarfbAkfsx}
\gl{\nA}
\bmng
 geVrupeTiTxge; beYsikalf \mo vugaLa cAlaka BAgagaLaninxTiTxruva peTiTxge. 
\emng
\eentry

\bentry
\word{gear-case}
\pron{giarfkeVsf}
\gl{\nA}
\bmng
 = \hyperlink{gear-box}{gear-box}. 
\emng
\eentry

\bentry
\word{gearing}
\pron{giariMgf}
\gl{\nA}
\bmng
\bnum
\num{1} cAlaka salakaraNe. 
\num{2} cAlaka salakaraNeyanunx joVDisuvudu. 
\enum
\emng
\eentry

\bentry
\word{gearless}
\pron{giarflisf}
\gl{\gu}
\bmng
\bnum
\num{1} geVrilalxda. 
\num{2} savxyaM cali; savxtashacxli; tAneV calisuva. 
\enum
\emng
\eentry

\bentry
\word{gear-lever}
\pron{giarflivarf}
\gl{\nA}
\bmng
 geVrusanenx; geVru hAkalu yA badalAyisalu baLasuva sanenx. 
\emng
\eentry

\bentry
\word{gear-shift}
\pron{giarfSiphfTx}
\gl{\nA}
\bmng
 (\ame)  = \hyperlink{gear-lever}{gear-lever}. 
\emng
\eentry

\bentry
\word{gear-wheel}
\pron{giarfviVlf}
\gl{\nA}
\bmng
 geVrucakarx; (\kanmu\ beYsikalilxnalilx kAlotitxniMdAda calanavanunx acicxge oyuyxva) halulxcakarx. 
\emng
\eentry

\bentry
\word{gecko}
\pron{gekoV}
\gl{\nA}
\expl{(\bava\ \eng{geckos} yA \eng{geckoes}).}
\bmng
(uSaNxvalayada) maneya halilx; gaqha gwLi. 
\emng
\eentry

\bentry
\word[gee(1)]{gee}
\pron{jiV}
\gl{\nA}
\bmng
 (\AmA) kudure. 
\emng
\eentry

\bentry
\word[gee(2)]{gee}
\pron{jiV}
\gl{\BAavayx}
\bmng
 (kudure \mo vugaLige heVLuva Ajecnxya mAtu.) 
\banum
\alnum{a} naDe muMde. 
\alnum{b} beVga, beVga! jalidx! 
\alnum{c} (omomxmemx) balakekx (tirugu!). 
\eanum
\emng
\eentry

\bentry
\word[gee(3)]{gee}
\pron{jiV}
\gl{\BAavayx}
\bmng
 (\ame) parxmANa, parxtijecnx, huDukutitxdadx vasutx sikikxdudx, \mo vugaLanunx sUcisuva udAgxra: jiV! (pArxyashaH jiVsasf eMba udAgxrada saMkiSxpatx). 
\emng
\eentry

\bentry
\word[gee(4)]{gee}
\pron{jiV}
\gl{\nA}
\bmng
 (\ame) (\ashi) AsAmi; isamu. 
\emng
\eentry

\bentry
\word{gee-gee}
\pron{jiVjiV}
\gl{\nA}
\bmng
 (\birx) (\AmA)  = \hyperlink{gee(1)}{$^1$gee}. 
\emng
\eentry

\bentry
\word{gee-ho}
\pron{jiVhoV}
\gl{\BAavayx}
\bmng
  = \hyperlink{gee(2)}{$^2$gee}. 
\emng
\eentry

\bentry
\word{geese}
\pron{giVsf}
\gl{\nA}
\bmng
 \eng{goose} padada \bava. 
\emng
\eentry

\bentry
\word{gee-string}
\pron{jiVsiTxrXMgf}
\gl{\nA}
\bmng
  = \hyperlink{G-string(2)}{G-string(2)}. 
\emng
\eentry

\bentry
\wordRemoveSpace{gee-whillikins}{gee whillikins}
\pron{jiV vilikinfs'}
\gl{\BAavayx}
\bmng
 = \hyperlink{gee(3)}{$^3$gee}. 
\emng
\eentry

\bentry
\wordRemoveSpace{gee-whiz}{gee whiz}
\pron{jiV visfZ}
\gl{\BAavayx}
\bmng
  = \hyperlink{gee(3)}{$^3$gee}. 
\emng
\eentry

\bentry
\word{geezer}
\pron{giVsaZrf}
\gl{\nA}
\bmng
\bnum
\num{1} (\ashi) muduka; mudigoDuDx. 
\num{2} AsAmi; isamu. 
\enum
\emng
\eentry

\bentry
\word{Gehenna}
\pron{gihena}
\gl{\nA}
\bmng
\bnum
\num{1} naraka; dahisuva, citarxhiMsege guripaDisuva, ati saMkaTakokxLapaDisuva loVka. 
\num{2} (\rUpa) naraka; citarxhiMse koDuva sathxLa; kArAgaqha; baMdiVKAne. 
\enum
\emng
\eentry

\bentry
\wordRemoveSpace{Geiger-counter}{Geiger counter}
\pron{geYgarf kwMTarf}
\gl{\nA}
\bmng
 geYgarf gaNaka; ayAniVkAraka kaNagaLu alapxkAlika viduyxtapxrXvAhavanunx utApxdisuva tatatxvXvanunx upayoVgisikoMDu vishavxkiraNagaLanUnx vikiraNapaTu padAthaRgaLanUnx gurutisalu baLasuva oMdu upakaraNa. 
\emng
\eentry

\bentry
\word{geisha}
\pron{geVSa}
\gl{\nA}
\expl{(\bava\ \eng{geishas} yA adeV).}
\bmng
\bnum
\num{1} geVSa; (japAnina) nataRki; kuNidu, hADi gaMDasara manasasxMtoVSa paDisuvavaLu. 
\num{2} japAniV veVsheyx. 
\enum
\emng
\eentry

\bentry
\wordRemoveSpace{Geissler-tube}{Geissler tube}
\pron{geYsalxrf TUyxbf}
\gl{\nA}
\bmng
geYsalxrf naLike; kaDime otatxDadalilx anila tuMbi moharu mADiruva, viduyxtapxrXvAhavanunx kaLisidAga javxlisuva gAjina naLike. 
\emng
\eentry

\bentry
\wordf{geist}
\pron{geYsfTx}
\gl{\nA}
\expl{\G}
\bmng
\bnum
\num{1} bwdidhxkate; parxjAcnxvaMtike; budidhxvaMtike. 
\num{2} Atamx; jiVva; ceVtana. 
\enum
\emng
\eentry

\bentry
\word[gel(1)]{gel}
\pron{jelf}
\gl{\nA}
\bmng
 (\ravi) jelf; are GanarUpada kalila dArxvaNa. 
\emng
\eentry

\bentry
\word[gel(2)]{gel}
\pron{jelf}
\gl{\akirx}
\expl{(\BU\ matutx \BUkaq\ \eng{gelled}, \vakaq\ \eng{gelling}).}
\bmng
 (\ravi)jelf Agu. 
\emng
\eentry

\bentry
\word{gelatin}
\pron{jelaTinf}
\gl{\nA}
\bmng
 jelaTinf; pArxNigaLa camaR, mULe, asithxrajujx, \mo vanunx niVrinalilx kudisi tayArisuva, nasu haLadi baNaNxda, oNagidAga BiduravAgiyU pAradashaRkavAgiyU iruva jelilxgaLanunx tayArisalu upayoVgisuva, ruciyilalxda porxVTiVnf padAthaR. 
\emng

\noindent
\gl{\pagu}
\bmng
\bnum
\num{1} \eng{blasting gelatin} soPxVTaka jelaTinf; neYTorx gilxsarinf matutx neYTorx neluyxloVsfgaLiMda tayArisida, rababxrinaMtiruva apAradashaRka soPxVTaka padAthaR. 
\num{2} \eng{vegetable gelatin} sasayx jelaTinf; sasayxmUladiMda tayArisida jelaTinfnaMtha padAthaR. 
\enum
\emng
\eentry

\bentry
\word{gelatine}
\pron{jelaTiVnf}
\gl{\nA}
\bmng
 (\birx)  = \hyperlink{gelatin}{gelatin}. 
\emng
\eentry

\bentry
\word{gelatinize}
\pron{jilAYxTineYsfZ}
\gl{\sakirx}
\bmng
 jelaTiniVkarisu: 
\banum
\alnum{a} jelaTinfnaMte mADu. 
\alnum{b} jelaTinfyukatxvanAnxgi mADu. 
\alnum{c} jelaTinf leVpisu. 
\eanum
\emng

\noindent
\gl{\akirx}
\bmng
 jelaTinfnaMte Agu; jelaTinAnxgu. 
\emng
\eentry

\bentry
\word[gelatinoid(1)]{gelatinoid}
\pron{jelAYxTinAyfDx}
\gl{\gu}
\bmng
  = \hyperlink{gelatinous(1)}{gelatinous (1)}. 
\emng
\eentry

\bentry
\word[gelatinoid(2)]{gelatinoid}
\pron{jelAYxTinAyfDx}
\gl{\nA}
\bmng
 jelaTinABa; jelaTininxnaMtha padAthaR. 
\emng
\eentry

\bentry
\word{gelatinous}
\pron{jilAYxTinasf}
\gl{\gu}
\bmng
\hypertarget{gelatinous(1)}{} 
\bnum
\num{1} jelaTinfnaMtha; maMdateyalilx jelilxyanunx hoVluva. 
\num{2} jelaTinfyukatx; jelaTinfpUrita. 
\enum
\emng
\eentry

\bentry
\word[gelation(1)]{gelation}
\pron{jileVSanf}
\gl{\nA}
\bmng
 shiVta GaniVkaraNa; shiVtadiMda gaDeDxkaTuTxvike. 
\emng
\eentry

\bentry
\word[gelation(2)]{gelation}
\pron{jeleVSanf}
\gl{\nA}
\bmng
 jeliVkaraNa; jelf rUpugoLuLxvike. 
\emng
\eentry

\bentry
\wordnospeech{gelation paper}{gelation paper}
\pron{?}
\gl{\nA}
\bmng
 jelaTinf kAgada; CAyAcitarx garxhaNadalilx baLasuva, duyxti sUkaSxmXgoLisida jelaTinanxnunx leVpisida kAgada. 
\emng
\eentry

\bentry
\word{geld}
\pron{gelfDx}
\gl{\sakirx}
\bmng
\bnum
\num{1} (\sA\ gaMDupArxNiya) janakashakitx kaLe; biVjavoDe; hiDamADu; (gaMDu pArxNiya) biVjagaLanunx shasatxrXcikitesxyiMda tegedubiDu. 
\num{2} (heNuNxpArxNiya) jananashakitx kaLe; aMDAshayavanunx shasatxrXcikitesxyiMda tegeduhAku. 
\enum
\emng
\eentry

\bentry
\word{gelder}
\pron{gelaDxrf}
\gl{\nA}
\bmng
\bnum
\num{1} (\sA\ gaMDupArxNiya) jananashakitx -- nAshaka, kaLeyuvavanu; biVja oDeyuvavanu; hiDamADuvavanu. 
\num{2} (heNuNx pArxNiya) jananashakitx -- nAshaka, kaLeyuvavanu; aMDAshayavanunx shasatxrXcikitesxyiMda tegeyuvavanu. 
\enum
\emng
\eentry

\bentry
\word{gelding}
\pron{geliDxMgf}
\gl{\nA}
\bmng
 biVjavoDeda, hiDa mADida pArxNi (\kanmu\ gaMDukudure). 
\emng
\eentry

\bentry
\word{gelid}
\pron{jiliDf}
\gl{\gu}
\bmng
\bnum
\num{1} himasheYtayxda; himashiVtala; himagalilxnaSuTx shiVtavAda. 
\num{2} koreyuva; tiVra taNaNxgiruva; ati shiVtala. 
\enum
\emng
\eentry

\bentry
\word{gelignite}
\pron{jeligenxYTf}
\gl{\nA}
\bmng
 jeligenxYTf; neYTorxgilxsarinf, neYTorx seluyxloVsf, poTAyxsiyaM neYTerxVTf matutx maradapuDi -- ivugaLanunx seVrisi tayArisida soPxVTaka. 
\emng
\eentry

\bentry
\word{gelly}
\pron{jeli}
\gl{\nA}
\bmng
 (\birx) (\ashi)  = \hyperlink{gelignite}{gelignite}. 
\emng
\eentry

\bentry
\word[gem(1)]{gem}
\pron{jemf}
\gl{\nA}
\bmng
\bnum
\num{1} ratanxmaNi; haraLu; (\kanmu\ kaDedu hoLapugoTaTx) parxshasatx shile. 
\num{2} ratanx; bahu celuvAda yA belebALuva vasutx. 
\num{3} (amUlayxveMdu BAvisida vasutxvina) utakxqqSaTx BAga; atuyxtatxma BAga. 
\num{4} nakAse ketitxda, citAtxra ketitxda -- parxshasatx shile. 
\enum
\emng
\eentry

\bentry
\word[gem(2)]{gem}
\pron{jemf}
\gl{\sakirx}
\expl{(\BU\ matutx \BUkaq\ \eng{gemmed}, \vakaq\ \eng{gemming}).}
\bmng
 (ratanxgaLiMda yA ratanxgaLiMdaloV eMbaMte) alaMkarisu; ratAnxlaMkAra mADu. 
\emng
\eentry

\bentry
\word{Gemara}
\pron{gi(ge)mAra}
\gl{\nA}
\bmng
 (yehUdayx dhamaR garxMthavAda) 
\banum
\alnum{a} TAYxlfmaDf garxMthada utatxra BAga. 
\alnum{b} TAYxlfmaDfna `miSAnx' eMba divxtiVya sakxMdhada vAyxKAyxna. 
\eanum
\emng
\eentry

\bentry
\word[geminate(1)]{geminate}
\pron{jeminaTf}
\gl{\gu}
\bmng
 (\jiVvi) joVDijoVDiyAgi kUDikoMDiruva; joVDigaLalilxruva; eraDeraDAda; jotejoteyAda; joVDi; jote; yugamx; yugaLa; eNe. 
\emng
\eentry

\bentry
\word[geminate(2)]{geminate}
\pron{jemineVTf}
\gl{\sakirx}
\bmng
 (\jiVvi) eraDeraDu mADu; joVDi joVDiyAgi joVDisu; yugamxgoLisu; yugaLa mADu. 
\emng
\eentry

\bentry
\word{gemination}
\pron{jemineVSanf}
\gl{\nA}
\bmng
 eraDeraDAgi yA joVDijoVDiyAgi -- joVDisuvike; yugimxVkaraNa; yugaLiVkaraNa. 
\emng
\eentry

\bentry
\word[Geminean(1)]{Geminean}
\pron{jeminianf}
\gl{\gu}
\bmng
 (\joyxV) mithunarAshija; mithunarAshiyalilx huTiTxda. 
\emng
\eentry

\bentry
\word[Geminean(2)]{Geminean}
\pron{jeminianf}
\gl{\nA}
\bmng
 (\joyxV) mithunarAshija; mithunarAshiyalilx huTiTxda vayxkitx. 
\emng
\eentry

\bentry
\word{Gemini}
\pron{jemineY(niV)}
\gl{\nA}
\bmng
 (\joyxV) mithuna; rAshi cakarxdalilx mUraneya rAshi. 
\emng
\eentry

\bentry
\word{gemlike}
\pron{jemfleYkf}
\gl{\gu}
\bmng
 ratanxdaMtha; parxshasatx shileyaMtha. 
\emng
\eentry

\bentry
\word{gemma}
\pron{jema}
\gl{\nA}
\expl{(\bava\ \eng{gemmae}).}
\bmng
(\jiVvi) 
\banum
\alnum{a} jemamx; aMkura; mogugx; elemogugx; muMde elegaLAgi rUpugoLuLxva mogugx. 
\alnum{b} (baMDepAci \mo vugaLalilx) tAyigiDadiMda beVpaRTuTx savxtaMtarx giDavAgi beLeyabalalx jiVvakoVsha yA jiVvakoVshagaLa samucacxya. 
\alnum{c} keLamaTaTxda pArxNigaLalilx mogigxnaMte kANisikoMDu, AnaMtara beVpaRTuTx savxtaMtarxvAgi beLeyabalalx BAga. 
\eanum
\emng
\eentry

\bentry
\word[gemmate(1)]{gemmate}
\pron{jemeVTf}
\gl{\gu}
\bmng
\bnum
\num{1} (\jiVvi) jemamxyukatx; sAMkura; aMkurayukatx; mogugxgaLiruva; jemamxgaLiMda kUDida. 
\hypertarget{gemmate(1)2}{} 
\num{2} (\jiVvi) aMkuraja; aMkurajanayx; mogugxtaLiya; jemimxVkaraNa vidhAnadiMda huTuTxva, vaqdidhxyAguva. 
\enum
\emng
\eentry

\bentry
\word[gemmate(2)]{gemmate}
\pron{jemeVTf}
\gl{\akirx}
\bmng
 (\jiVvi) jemImxkarisu: 
\banum
\alnum{a} aMkurisu; jemamx, aMkura, mogugx -- taLe, biDu. 
\alnum{b} aMkurajavAgu; aMkurajanayxvAgu; mogugxtaLiyAgu; jemImxkaraNa vidhAnadiMda -- huTuTx, saMtAnoVtapxtitxyAgu, vaqdidhxyAgu. 
\eanum
\emng
\eentry

\bentry
\word{gemmation}
\pron{jemeVSanf}
\gl{\nA}
\bmng
 (\jiVvi) jemImxkaraNa; aMkurajate; aMkurajanayxte; mogugx taLi; jemamxgaLa, aMkuragaLa mUlaka Aguva vaqdidhx, saMtAnoVtapxtitx. 
\emng
\eentry

\bentry
\word{gemmiferous}
\pron{jemipharasf}
\gl{\gu}
\bmng
\bnum
\num{1} ratanxBarita; ratanxgaBaR; ratanxgaLanunx, parxshasatx shilegaLanunx oLagoMDiruva, niVDuva. 
\num{2} aMkurita; mogugx hotitxruva, biTiTxruva. 
\enum
\emng
\eentry

\bentry
\word{gemmiparous}
\pron{jemiparasf}
\gl{\gu}
\bmng
\bnum
\num{1} (\jiVvi) jemimxVkaraNada; aMkurajanayxteya; mogugx taLiya. 
\num{2}  = \hyperlink{gemmate(1)2}{$^1$gemmate \((2)\)}. 
\enum
\emng
\eentry

\bentry
\word{gemmologist}
\pron{jemAlajisfTx}
\gl{\nA}
\bmng
 ratanxshAsatxrXjacnx; ratanxvijAcnxni; parxshasatx shilegaLa adhayxyana mADidavanu. 
\emng
\eentry

\bentry
\word{gemmology}
\pron{jemAlaji}
\gl{\nA}
\bmng
 ratanxshAsatxrX; ratanxvideyx; ratanxvijAcnxna; parxshasatxshilegaLa adhayxyana. 
\emng
\eentry

\bentry
\word{gemmule}
\pron{jemUyxlf}
\gl{\nA}
\bmng
\bnum
\num{1} (\jiVvi) kiru jemamx; kirumogugx; alApxMkura. 
\num{2} jemUyxlf; DAviRnanxna pAyxMjenesisf vAdada parxkAra deVhada elalx jiVvakoVshagaLiMdalU punarutApxdaka koVshakekx AnuvaMshika aMshagaLanunx taMdodagisuva GaTakagaLalolxMdu. 
\enum
\emng
\eentry

\bentry
\word{gemmy}
\pron{jemi}
\gl{\gu}
\bmng
\bnum
\num{1} ratanxBarita; ratanxmaya; maNimaya. 
\num{2} ratanxKacita; maNiKacita. 
\num{3} ratanxgaLaMtaha vasutxgaLiMda kaTiTxda, alaMkarisida. 
\num{4} ratanxdaMtha. 
\num{5} (ratanxdaMte) ujavxla; parxkAshamAna; thaLathaLisuva. 
\enum
\emng
\eentry

\bentry
\word{gemsbok}
\pron{jemfsfZbAkf}
\gl{\nA}
\bmng
 udadxvU teLaLxneyU neTaTxneyU Ada koMbugaLuLaLx, dakiSxNa Aphirxkada doDaDx jiMke.  \imglink{gemsbokfigure}{\raisebox{-0.15cm}[0pt][0pt]{\pdfimage width 0.7cm height 0.7cm {G_Pictures/gemsbok.jpg}}} 
\emng
\eentry

\bentry
\wordf{gemutlich}
\pron{gamU(mUyx)Tilxkf}
\gl{\gu}
\expl{\G}
\bmng
geluvina; ulAlxsada; nemamxdiya; saMtoVSada. 
\emng
\eentry

\bentry
\wordf{gemutlichkeit}
\pron{gamU(mUyx)TilxkeYTf}
\gl{\nA}
\expl{\G}
\bmng
geluvu; nemamxdi; ulAlxsa parxkaqti; saMtoVSa parxkaqti. 
\emng
\eentry

\bentry
\word[gen(1)]{gen}
\pron{jenf}
\gl{\nA}
\bmng
 (\ashi) mAhiti; savaRsAmAnayx viSayavivaragaLu; sAvaRjanika viSayAMsha gaLu. 
\emng
\eentry

\bentry
\word[gen(2)]{gen}
\pron{jenf}
\gl{\sakirx}
\expl{(\BU\ matutx \BUkaq\ \eng{genned}, \vakaq\ \eng{genning}).}
\bmng
(\ashi) mAhiti odagisu (\akirx\ saha). 
\emng

\noindent
\gl{\pagu}
\bmng
 \eng{gen up} = \hyperlink{gen(2)}{$^2$gen} \eng{: He wanted information; I had it, I was in a position to gen him up} avanigoMdu mAhiti beVkitutx; adu nananx hatitxra itutx, nAnu A mAhitiyanunx avanige odagisabalalxvanAgidedx. 
\emng
\eentry

\bentry
\wordwithhyphen{hyp-gen}{-gen}
\pron{-janf}
\gl{\uparx}
\bmng
\bnum
\num{1} `janaka' eMbathaRdalilx: \eng{oxygen, hydrogen.} 
\num{2} (\savi) `beLavaNige' yA `vikAsa' eMbathaRdalilx: \udA\ \eng{endogen exogen.} 
\enum
\emng
\eentry

\bentry
\wordnospeech{Gen.}{Gen.}
\pron{?}
\gl{\saMkiSx}
\bmng
\bnum
\num{1} \eng{General.} 
\num{2} \eng{Genesis} (beYbalina haLe oDaMbaDike). 
\enum
\emng
\eentry

\bentry
\word{gena}
\pron{jiV(je)na}
\gl{\nA}
\bmng
 (\pArxvi, \aMrashA) jiVna: 
\banum
\alnum{a} kenenx; kapoVla. 
\alnum{b} taleya pAshavxR; shiraHpAshavxR. 
\eanum
\emng
\eentry

\bentry
\word{genal}
\pron{jiVnalf}
\gl{\gu}
\bmng
 (\pArxvi, \aMrashA) jiVnada: 
\banum
\alnum{a} kenenxya; kapoVlada. 
\alnum{b} taleya pAshavxRda; shiraHpAshavxRda. 
\eanum
\emng
\eentry

\bentry
\word{genappe}
\pron{janAYxpf}
\gl{\nA}
\bmng
bahaLa nayavU nuNupU Ada uNeNxya nUlu; kasUti uNeNxya dAra. 
\emng
\eentry

\bentry
\word{gendarme}
\pron{SAZMDAmfR}
\gl{\nA}
\bmng
\bnum
\num{1} (\kanmu\ phArxnisxnalilx poliVsu kelasadalilx niyukatxnAgiruva)rAvuta yA seYnika; poliVsu seYnika. 
\num{2} aDaDxkoVDugalulx; beTaTx hatutxvudaralilx, beTaTxda kaDidAda ENugaLalilx dArige aDaDx niMtiruva koVDugalulx. 
\enum
\emng
\eentry

\bentry
\word{gendarmerie}
\pron{SAMDAmaRriV}
\gl{\nA}
\bmng
\bnum
\num{1} poliVsu seYnika daLa. 
\num{2} poliVsf seYnika -- ThANe, keVMdarx. 
\enum
\emng
\eentry

\bentry
\word[gender(1)]{gender}
\pron{jeMDarf}
\gl{\nA}
\bmng
\bnum
\num{1} (\vAyx) liMga: \eng{masculine, feminine, neuter gender} pulilxMga, sitxrXVliMga, napuMsakaliMga. \eng{common gender} puMsitxrXVliMga yA uBayaliMga. 
\num{2} (nAmavAcakagaLa, savaRnAmagaLa \vi) puM, sitxrXV yA napuMsaka -- vAcaka. 
\num{3} (guNavAcakagaLa \vi) visheVSAyxnuguNa rUpa; puM-, sitxrXV -- , yA napuMsaka vAcakagaLAda nAmavAcakagaLige yA savaRnAmagaLige anurUpavAda visheVSaNa rUpa. 
\num{4} (\AmA) (sitxrXV yA puruSa) jAti; vagaR; liMga: \eng{She has a spirit more masculine than the first gender} Akege puruSa jAtige mIrida puMsatxvX ide. 
\enum
\emng
\eentry

\bentry
\word[gender(2)]{gender}
\pron{jeMDarf}
\gl{\sakirx}
\bmng
 (\kAparx) = \hyperref{kandict_e.pdf}{E}{engender}{engender}. 
\emng
\eentry

\bentry
\word{genderless}
\pron{jeMDarflisf}
\gl{\gu}
\bmng
 (\vAyx) liMgavilalxda; niliRMga; liMgarahita; yAva liMgakUkx seVrada: \eng{literarians are still in search of a genderless pronoun of the third person singular} sAhitayx vidAvxMsaru parxthama puruSa Ekavacanada liMgarahita savaRnAma (rUpa)kAkxgi inUnx huDukutitxdAdxre. 
\emng
\eentry

\bentry
\word{gene}
\pron{jiVnf}
\gl{\nA}
\bmng
 (\jiVvi) jiVnu; korxVmosoVmina oMdu GaTakavAgiruva, taMde tAyigaLiMda yAvudeV AnuvaMshika aMshavanunx muMdina saMtatige vagARyisuva EkamAna, aMsha,dhAtu. 
\emng
\eentry

\bentry
\word{genealogical}
\pron{jiV(je)nialAjikalf}
\gl{\gu}
\bmng
\bnum
\num{1} vaMshaparaMpareya; vaMshAvaLiya; vaMshada; sAlina; piVLigeya; saMtatiya. 
\num{2} vaMshAnukarxmada; vaMshAnugata; vaMshapAraMpayaRvAda. 
\enum
\emng
\eentry

\bentry
\word{genealogically}
\pron{jiV(je)nialAjikali}
\gl{\kirxvi}
\bmng
\bnum
\num{1} vaMshaparaMpareyAgi; sAliniMda; piVLige piVLigeyAgi. 
\num{2} vaMshAnukarxmavAgi; vaMshAnugatavAgi; vaMshapAraMpayaRvAgi. 
\enum
\emng
\eentry

\bentry
\wordnospeech{genealogical tree}{genealogical tree}
\pron{?}
\gl{\nA}
\bmng
 vaMshavaqkaSx; oMdu kuTuMbada yA pArxNijAtiya paraMpareyanunx shAKegaLiMda kUDida vaqkASxkArada mUlaka toVrisuva citarx, paTiTx. 
\emng
\eentry

\bentry
\word{genealogise}
\pron{jiV(je)niAYxlajeYsfZ}
\gl{\sakirx}
\bmng
  = \hyperlink{genealogize}{genealogize}. 
\emng
\eentry

\bentry
\word{genealogist}
\pron{jiV(je)niAYxlajisfTx}
\gl{\nA}
\bmng
\bnum
\num{1} vaMshaparaMparAshAsatxrXjacnx. 
\num{2} vaMshAnevxVSaka; vaMshada mUla matutx saMtatiyanunx huDuki barediDuvavanu. 
\enum
\emng
\eentry

\bentry
\word{genealogize}
\pron{jiV(je)niAYxlajeYsfZ}
\gl{\sakirx}
\bmng
\bnum
\num{1} vaMshamUla matutx paraMpareyanunx huDuku, gurutisu; vaMshAnevxVSaNa mADu. 
\num{2} vaMshavaqkaSxgaLanunx racisu, bare. 
\enum
\emng
\eentry

\bentry
\word{genealogy}
\pron{jiV(je)niAYxlaji}
\gl{\nA}
\bmng
\bnum
\num{1} vaMshAvaLiV kathana; parxvara; vaMshAnukarxma heVLuvudu; kulada kathe; mUlapuruSaniMda toDagi pitaqvagaRvanenxlalx hesarisi naDuve baruvavara sAthxna sUcisuvudu. 
\num{2} vaMshAnevxVSaNa; piVLigeyanunx huDuki kaMDuhiDiyuvudu; vaMshavaqkaSx gotutxhacucxvudu. 
\num{3} (sasayx yA pArxNi AdirUpadiMda mApaRTuTx vividha rUpagaLanunx taLeda) vikasanasaraNiya nirUpaNa; vikAsavaNaRne. 
\enum
\emng
\eentry

\bentry
\word{genera}
\pron{jenara}
\gl{\nA}
\bmng
 \eng{genus} padada \bava. 
\emng
\eentry

\bentry
\word[general(1)]{general}
\pron{jenaralf}
\gl{\gu}
\bmng
\bnum
\num{1} (saMpUNaRvAgi yA sarisumArAgi) savaRsAmAnayxvAda; sAdhAraNavAda; sAvaRtirxkavAda; pArxyikavAda. 
\numi{2} sAvaR; sAvaRlwkika: 
\banum
\alnum{a} elalxranUnx, elalxvanUnx yA hecucx kaDame elalxranUnx, elalxvanUnx -- oLagoMDiruva. 
\alnum{b} elalx vayxkitxgaLa, viSayagaLa yA hecucx kaDame elalx vayxkitxgaLa, viSayagaLa meVle pariNAma uMTumADuva yA parxBAva biVruva. 
\eanum
\numie
\num{3} EkadeVshavalalxda; AMshikavalalxda. 
\num{4} visheVSavalalxda; avishiSaTx; vayxkitxsiVmitavalalxda. 
\num{5} sathxLiVyavalalxda; pArxdeVshikavalalxda; sAvaRdeVshika; sAvaRdeVshiVya. 
\num{6} vagiRVyavalalxda; pAMthikavalalxda: \eng{general confession} (kerxYsatxralilx) sAmUhika pApaniveVdana; oMdu caciRge seVridavarelalx oTATxgi pAdirxya edurinalilx, caciRnalilx tamamx pApagaLanunx opipxkoLuLxvudu. 
\num{7} parxcalita; rUDhiyalilxruva; baLakeyalilxruva. 
\num{8} elelxlUlx haraDiruva; sAvaRtirxkavAda. 
\num{9} vADikeyAda; padadhxtiganuguNavAda: \eng{in a general way} (loVka)rUDhiyaMte; vADikeya parxkAra; sAmAnayx padadhxtiyaMte. 
\num{10} savARnavxyi(ka); sAdhAraNa; oMdu parimita keSxVtarxkekx mAtarx anavxyisada; vasutxgaLu, saMdaBaRgaLu, \mo vugaLa iDiV vagaRkekx saMbaMdhisida; savARnavxyavAda yA hecucx kaDime elalxkUkx anavxyisuva: \eng{as a general rule} sAmAnayxvAgi; sAdhAraNavAgi; sAmAnayxtaH; bahu saMdaBaRgaLalilx. 
\num{11} sAmAnayx; sAdhAraNa; oMdu vagaRda vayxkitxgaLa, vasutxgaLa parasapxra BeVdagaLanunx lekikxsade sAmAnAyxMshagaLanonxLagoMDa: \eng{general word} sAmAnayx pada. \eng{general notion} sAmAnayx BAvane. 
\num{12} sAmAnayx; oMdu shAKege yA viSayakekx mIsalalalxda; vishiSaTx bageyadalalxda: \eng{general dealer} sAmAnayx saraku vAyxpAri. 
\num{13} sUthxlavAgi hoMduva; vivaragaLiMda kUDirada; hecucx kaDame samapaRkavAda; sAmAnayx vayxvahAragaLige sAkAguva: \eng{general resemblance} sAmAnayx hoVlike; hecucx kaDame hoVluvudu. 
\num{14} asapxSaTx; moguM; anidiRSaTx; aniSakxqqSaTx; Kacitavalalxda; sUthxla: \eng{spoke only in general terms} sUthxlavAgi mAtanADida; moguM Agi mAtanADida. 
\num{15} (seYnAyxdhikArigaLa \vi) janaralf; kanaRlilxge meVlapxTaTx adhikArada matutx aMtasitxna. 
\num{16} muKayx; parxdhAna; muKayxsathxnAda; adhikAra matutx kAyaRkeSxVtarxgaLalilx parimiti ilalxda, aMke ilalxda (aneVka veVLe \eng{Adjutant-General, Attorney-General, Secretary-General, Solicitor-General} \mo\ hudedxgaLa hesarugaLalilx \parx): \eng{general manager} muKayx nivARhaka, \eng{Secretary general} muKayx kAyaRdashiR; mahAkAyaRdashiR. 
\num{17} (\hA) (\nA gaLoDane) elalxrigU saMbaMdhisida; savaRsaMbaMdhi: \eng{lover general} savaRsitxrXVyaralUlx parxNaya taLeyuva, toVruva; savaRsitxrXVperxVmi; savaRparxNayi. 
\enum
\emng

\noindent
\gl{\pagu}
\bmng
\bnum
\numi{1} \eng{General Certificate of Education} sAmAnayx shikaSxNa saTiRphikeVTu: 
\banum
\alnum{a} iMgelxMDu matutx veVlfsxgaLalilx, \kanmu\ mAdhayxmika shAlA vidAyxthiRgaLige EpaRDisuva pariVkeSx. 
\alnum{b} I pariVkeSxyalilx teVgaRDeyAdare koDuva parxmANapatarx. 
\eanum
\numie
\num{2} \eng{in general} sAmAnayxvAgi; sAdhAraNavAgi; elalx nidashaRnagaLigU sAmAnayxvAgi anavxyisuvaMte; kelavu visheVSa nidashaRnagaLanunx biTuTx; bahuvAgi; bahushaH; mukAkxlupAlu elalxkUkx anavxyisuvaMte. 
\enum
\emng
\eentry

\bentry
\word[general(2)]{general}
\pron{jenaralf}
\gl{\nA}
\bmng
\bnum
\num{1} (\pArxparx) (\eng{the} oDane) janasAmAnayx; janate. 
\num{2} (\bava\ dalilx) (Iga \viparx) sAmAnayx tatatxvXgaLu, BAvanegaLu yA niyamagaLu. 
\num{3} dhAmiRka muKaMDa; oMdu matiVya paMthada muMdALu; nAyaka, \udA\ jesUTf, DAminikanf, \mo\ kerxYsatx paMthagaLa muKayxsathx. 
\num{4} (seYnayx) janaralf; phiVlfDx mASaRlf hudedxgiMta oMdu aMtasutx kaDime aMtasitxnavanu yA lephiTxnaMTf janaralfgiMta meVlinavanu. 
\num{5} seYnAyxdhipati; seVnApati. 
\num{6} (\ame) seVneya yA vAyuseVneya janaralf, atuyxcacx adhikAri. 
\num{7} kAyaRtaMtarxkushala; kAyaRniVtijacnx; upAyajacnx. 
\num{8} visheVSa pariNati iruvavanu; tajacnx; pariNata: \eng{a good general} utatxma taMtarxkushala. 
\num{9} (\AmA) sAmAnayx kelasagAti; manegelasadavaLu. 
\num{10} (\AmA) janaralf poVsfTx AphiVsu. 
\enum
\emng
\eentry

\bentry
\wordnospeech{General American}{General American}
\pron{?}
\gl{\nA}
\bmng
 sAmAnayx amerikanf (BASe); pArxdeVshika yA pArxMtiVya lakaSxNagaLilalxda amerikada ADuBASe. 
\emng
\eentry

\bentry
\wordnospeech{general delivery}{general delivery}
\pron{?}
\gl{\nA}
\bmng
 (\ame) sAmAnayx baTavADe: 
\banum
\alnum{a} maneya viLAsavilalxdavaru yA maneya viLAsakekx aMceya baTavADe apeVkiSxsadiruvavaru tAveV KudAdxgi aMcekaCeVriya kwMTarinalilx baTavADe paDeyabahudAda aMcevayxvasethx. 
\alnum{b} I riVtiya aMcevayxvasethxyiruva aMcekaCeVri. 
\eanum
\emng
\eentry

\bentry
\wordnospeech{general election}{general election}
\pron{?}
\gl{\nA}
\bmng
sAvaRtirxka cunAvaNe; shAsanasaBegAgi iDiV deVshadalilx naDesuva cunAvaNe. 
\emng
\eentry

\bentry
\wordnospeech{general headquarters}{general headquarters}
\pron{?}
\gl{\nA}
\bmng
(muKayx seVnAdhipatiya) muKayx, parxdhAna -- nivAsa (sAthxna) 
\emng
\eentry

\bentry
\word{generalisation}
\pron{jenaraleYseZVSanf}
\gl{\nA}
\bmng
  = \hyperlink{generalization}{generalization}. 
\emng
\eentry

\bentry
\word{generalise}
\pron{jenaraleYsfZ}
\gl{\kirx}
\bmng
  = \hyperlink{generalize}{generalize}. 
\emng
\eentry

\bentry
\word{generalissimo}
\pron{jenaralisimoV}
\gl{\nA}
\expl{(\bava\ \eng{generalissimos}).}
\bmng
(\BU\ nwkA matutx vAyuseVnegaLelalxdara yA halavu seVnegaLa) savaRseVnAni; mahAseVnApati. 
\emng
\eentry

\bentry
\word{generalist}
\pron{jenaralisfTx}
\gl{\nA}
\bmng
 sAmAnayxjacnx; halavAru keSxVtarxgaLalilx, viSayagaLalilx tiLuvaLike iruvavanu (virudadhxpada \eng{specialist}). 
\emng
\eentry

\bentry
\word{generality}
\pron{jenarAYxliTi}
\gl{\nA}
\bmng
\bnum
\num{1} sAmAnayxte; sAdhAraNate; sAdhAraNayx; elalx nidashaRnagaLanUnx oLagoMDa iDiV vagaRkekx anavxyisuvike. 
\num{2} asapxSaTxte; aniSakxqqSaTxte; vivaragaLilalxdiruvike. 
\num{3} savaRsAdhAraNa, savaRsAmAnayx -- aMsha, tatatxvX, shAsana, heVLike. 
\num{4} parxdhAna BAga; bahavxMsha; adhika BAga; bahupAlu. 
\enum
\emng
\eentry

\bentry
\word{generalization}
\pron{jenaraleYseZVSanf}
\gl{\nA}
\bmng
\bnum
\numi{1} sAmAniyxVkaraNa: 
\banum
\alnum{a} sAmAnayx niyamagaLa, BAvaneya, tatatxvXda rUpakekx taruvudu. 
\alnum{b} (takaR) anugamana takaRdiMda niNaRyisida sAmAnayx BAvane, parxmeVya yA parxtijecnx. 
\eanum
\numie
\num{2} (bahu veVLe hiVnAthaRdalilx baLasuva mAtAgi) sAmAnayx heVLike yA sAmAniyxVkaraNa: \eng{hasty generalization} (ati kaDime AdhAragaLiMda niNaRyisuva) duDukina, avasarada -- sAmAnayx nirUpaNe yA sAmAniyxVkaraNa. 
\enum
\emng
\eentry

\bentry
\word{generalize}
\pron{jenaraleYsfZ}
\gl{\sakirx}
\bmng
\bnum
\numi{1} sAmAniyxVkarisu: 
\banum
\alnum{a} sAmAnayx niyamagaLa, BAvaneya, tatatxvXda rUpakekx taru, iLisu. 
\alnum{b} sAmAnayx hesariniMda kare. 
\alnum{c} (\ga\ matutx \tashA) sAmAnayxrUpakekx taru. 
\alnum{d} (\ga\ matutx \tashA) (satAyxMshagaLa) anavxyavanunx yA vAyxpitxyanunx hecicxsu; vAyxpitx visatxrisu. 
\alnum{e} (satAyxMsha, vasutxsaMgati, \mo vugaLa AdhArada meVle) sAmAnayx heVLikeyanunx, nirUpaNeyanunx -- sAthxpisu, kalipxsu. 
\alnum{f} (biDi saMgatigaLu \mo vugaLa guNalakaSxNagaLanunx beVpaRDisuva mUlaka) sAmAnayx BAvanegaLanunx, kalapxnegaLanunx racisu. 
\eanum
\numie
\num{2} anugamana takaRdiMda (sAmAnayx niyamavanunx, niNaRyavanunx) -- Uhisu, anumAnisu. 
\num{3} (citarx) vishiSaTx lakaSxNagaLanunx mAtarx nirUpisu, rUpisu. 
\num{4} asapxSaTxgoLisu; vivaragaLanunx niVDade niSakxqqSaTxvalalxdaMte, KacitavAgiradaMte mADu. 
\num{5} asapxSaTxvAgi, vivaragaLilalxde, KacitavAgirada, niSakxqqSaTxvalalxda riVtiyalilx -- mAtanADu. 
\num{6} sAmAnayx rUDhige taru. 
\enum
\emng

\noindent
\gl{\akirx}
\bmng
\bnum
\num{1} (vasutxgaLa yA viSayagaLa sAmAnayx lakaSxNagaLanunx parxteyxVkavAgi BAvisikoMDu) sAmAnayx BAvanegaLanunx kalipxsu, nirUpisu. 
\num{2} sAmAnayx heVLikegaLanunx nirUpaNegaLanunx baLasu. 
\num{3} asapxSaTxvAgi mAtanADu. 
\enum
\emng
\eentry

\bentry
\word{generalizer}
\pron{jenaraleYsaZrf}
\gl{\nA}
\bmng
 sAmAniyxVkarisuvavanu. 
\emng
\eentry

\bentry
\word{generally}
\pron{jenarali}
\gl{\kirxvi}
\bmng
\bnum
\num{1} bahumaTiTxge; bahuvAgi; vAyxpakavAgi: \eng{the new plan was generally welcomed} hosa yoVjanege bahumaTiTxna sAvxgata doreyitu. 
\num{2} sAmAnAyxthaRdalilx; sUthxlavAgi; vishiSaTx aMshagaLanunx lekikxsade; visheVSataH gamanisade: \eng{generally speaking} sAmAnayxtaH, sAmAnayxvAgi -- heVLuvudAdare; sUthxlavAgi heVLidare. 
\enum
\emng
\eentry

\bentry
\word{generalship}
\pron{jenaralfSipf}
\gl{\nA}
\bmng
\bnum
\num{1} janaralfna adhikAra; seVnApatiya adhikAra. 
\num{2} yudadhxtaMtarx; samarakwshala. 
\num{3} vayxvahAra kwshala, jANemx; nivaRhaNA cAtuyaR. 
\enum
\emng
\eentry

\bentry
\word{generate}
\pron{jenareVTf}
\gl{\sakirx}
\bmng
\bnum
\num{1} (shAKa, bala, beLaku, GaSaRNe, viduyxtutx, PalitAMsha, vidayxmAna, manaHsithxti, \mo vanunx) huTiTxsu; utApxdisu; asitxtavxkekx taru; vikAsagoLisu. 
\num{2} (\ga) (calisutitxruvudeMdu parigaNitavAda biMdu, reVKe yA talada \vi reVKe, tala yA Gana Akaqtiyanunx) racisu; nimiRsu. 
\enum
\emng
\eentry

\bentry
\word{generation}
\pron{jenareVSanf}
\gl{\nA}
\bmng
\bnum
\num{1} saMtAna; saMtAnoVtapxtitx; saMtAnABivaqdidhx. 
\num{2} (neYsagiRka yA kaqtaka) utapxtitx; utApxdane; \kanmu\ viduyxdutApxdane. 
\num{3} saMtati; piVLige; talemore; talemAru; vaMshavaqkaSxdalilx oMdu haMta: \eng{have known them for three generations} mUru talemArugaLiMda avaranunx balelx. \eng{his descendant in the tenth generation} avara hatatxneV piVLigeyava. 
\numi{4} saMtati; piVLige; talemAru: 
\banum
\alnum{a} hecucx kaDime oMdeV kAladalilx huTiTxdavarelalxru: \eng{my generation} nananx talemArinavaru. \eng{the postwar generation} yudodhxVtatxra piVLigeyavaru. \eng{the rising generation} beLeyutitxruva, parxvadhaRmAnakekx barutitxruva piVLigeyavaru. 
\alnum{b} taMdetAyigaLa sathxLavanunx makakxLu Akarxmisalu beVkAguvudeMdu gaNisalAgiruva sarAsari kAla (sumAru \eng{30} vaSaR): \eng{a generation ago} oMdu talemArina hiMde. 
\alnum{c} oMdu kAladalilx oMdeV vagaRkekx, samAnavagaRkekx seVridavaru: \eng{the generation of silent -- screen stars} mUka calanacitarxda naTanaTiyara piVLige. 
\eanum
\numie
\num{5} piVLige; vikAsada, beLavaNigeya haMta: \eng{third-generation computers} mUraneV piVLigeya kaMpUyxTarugaLu. 
\enum
\emng

\noindent
\gl{\pagu}
\bmng
\hypertarget{generation pagu1}{} 
\bnum
\num{1} \eng{equivocal generation} savxyaM janana; savxyaMsaMtAna; nijiRVva vasutxgaLiMda jiVvigaLu tamage tAveV udaBxvisuvudu. 
\num{2} \eng{spontaneous generation} = \hyperlink{generation pagu1}{?pagu? \((1)\)}. 
\enum
\emng
\eentry

\bentry
\word{generational}
\pron{jenareVSanalf}
\gl{\gu}
\bmng
\bnum
\num{1} saMtAnoVtapxtitxya yA adakekx saMbaMdhisida. 
\num{2} saMtatiya; talemArina; piVLigeya. 
\num{3} piVLige saMbaMdhada; talemArugaLa naDuvaNa saMbaMdhagaLige saMbaMdhisida. 
\enum
\emng
\eentry

\bentry
\wordnospeech{generation gap}{generation gap}
\pron{?}
\gl{\nA}
\bmng
 talemArina vayxtAyxsa; piVLige aMtara; beVre beVre talemArinavarigiruva aBipArxyaBeVdagaLu. 
\emng
\eentry

\bentry
\word{generative}
\pron{jenareV(ra)Tivf}
\gl{\gu}
\bmng
\bnum
\num{1} saMtAnoVtapxtitxya; saMtAnABivaqdidhxya. 
\num{2} utApxdaka; utApxdisuva; utApxdisabalalx. 
\enum
\emng
\eentry

\bentry
\wordnospeech{generative grammar}{generative grammar}
\pron{?}
\gl{\nA}
\bmng
 utApxdaka vAyxkaraNa; BASeyoMdara GaTakAMshagaLiMda aMgiVkArAhaR vAkayxgaLanunx, vAkayxrUpagaLanunx racisalu neravAguva BASA niyamAvaLi, vAyxkaraNasUtarxgaLu. 
\emng
\eentry

\bentry
\word{generator}
\pron{jenareVTarf}
\gl{\nA}
\bmng
\bnum
\num{1} janaka; yAvudeV oMdara huTiTxge kAraNanAdava(Lu). 
\num{2} utApxdaka; anila, habe, \mo vugaLanunx utapxtitx mADuva upakaraNa. 
\num{3} janareVTaru; yAMtirxka shakitxyanunx viduyxcaCxkitxyanAnxgi parivatiRsuva yaMtarx, \udA\ DeYnamo. 
\enum
\emng
\eentry

\bentry
\word{generic}
\pron{ji(je)nerikf}
\gl{\gu}
\bmng
\bnum
\num{1} jAtivishiSaTxvAda; jAtiveYsheVSayxda; oMdu jAti yA vagaRkekx vishiSaTxvAda. 
\num{2} jAtayxnavxyada; jAtivAcakavAda; doDaDxguMpige seVrida; yAvudakekx beVkAdarU anavxyisuvaMtha. 
\num{3} sAmAnayx; sAvaRtirxka; vishiSaTxvalalxda; elalxkUkx anavxyisuva. 
\num{4} (\jiVvi) jAtiya; kulada; kulakekx saMbaMdhisida yA kulada dajeRyuLaLx. 
\enum
\emng
\eentry

\bentry
\word{generically}
\pron{ji(je)nerikali}
\gl{\kirxvi}
\bmng
\bnum
\num{1} jAtivishiSaTxvAgi; oMdu jAti yA vagaRkekx vishiSaTxvAgi. 
\num{2} jAtayxnavxyavAgi; jAtivAcakavAgi; doDaDx guMpige seVrida yAvudakekx beVkAdarU anavxyisuvaMte. 
\num{3} sAmAnayxvAgi; sAvaRtirxkavAgi. 
\enum
\emng
\eentry

\bentry
\word{generosity}
\pron{jenarAsiTi}
\gl{\nA}
\bmng
\bnum
\num{1} udArate; audAyaR; udAra parxkaqti, savxBAva. 
\num{2} udAramanasakxte; udArabudidhx; alapxtana yA pUvaRgarxhagaLilalxdiruvike; saMkucita budidhx yA rAgadevxVSagaLu ilalxdiruvike. 
\num{3} doDaDxmanasusx(LaLxvanAgiruvudu); vishAla haqdayate. 
\num{4} koDugeYtana; dhArALate; dAnashiVlate. 
\enum
\emng
\eentry

\bentry
\word{generous}
\pron{jenarasf}
\gl{\gu}
\bmng
\bnum
\num{1} udAra; udAra savxBAvada. 
\num{2} doDaDx manasisxna; vishAla haqdayada. 
\num{3} udArabudidhxya; alapxbudidhxyalalxda yA pUvaRgarxhagaLilalxda; saMkucita manoVvaqtitx yA rAgadevxVSagaLu ilalxda. 
\num{4} koDugeYya; dhArALada; dAnashiVla. 
\num{5} (nelada \vi) PalavatAtxda; hulusAda. 
\num{6} heVraLa; puSakxLa; yatheVSaTx; samaqdadhx. 
\num{7} (veYnina \vi) tiVkaSxNXvAda matutx tuMbu ruciya. 
\enum
\emng
\eentry

\bentry
\word{generously}
\pron{jenarasfli}
\gl{\kirxvi}
\bmng
\bnum
\num{1} udAravAgi; audAyaRdiMda; udAtatx savxBAvadiMda. 
\num{2} doDaDx manasisxniMda; vishAla haqdayadiMda. 
\num{3} udArabudidhxyiMda; udAramanasisxniMda; alapxbudidhx yA pUvaRgarxhagaLu ilalxde; saMkucita manoVvaqtitx yA rAgadevxVSagaLu ilalxde. 
\num{4} koDugeYyAgi; dhArALavAgi; dAnashiVlanAgi; dAnashiVlateyiMda. 
\enum
\emng
\eentry

\bentry
\word{genesis}
\pron{jenisisf}
\gl{\nA}
\bmng
\bnum
\num{1} (\eng{Genesis}). beYbalina haLeya oDaMbaDikeyalilx vishavxsaqSiTxyanunx vivarisuva modala pavaR; saqSiTxpavaR. 
\num{2} huTuTx; mUla; ugama. 
\num{3} utapxtitx -- karxma yA vidhAna; utApxdanA karxma yA viBAga. 
\enum
\emng
\eentry

\bentry
\wordwithhyphen{hyp-genesis}{-genesis}
\pron{-jenisisf}
\gl{\saupa}
\bmng
 janana, janayxte, utapxtitx eMbathaRgaLa \saupa: \eng{abiogenesis, parthenogenesis.} 
\emng
\eentry

\bentry
\word{genet}
\pron{jeneTf}
\gl{\nA}
\bmng
\bnum
\num{1} oMdu bageya punugina bekukx.  \imglink{genetfigure}{\raisebox{-0.15cm}[0pt][0pt]{\pdfimage width 0.5cm height 0.6cm {G_Pictures/genet.jpg}}} 
\num{2} idara tupupxLudogalu, tupupxLu camaR. 
\enum
\emng
\eentry

\bentry
\word{genetically}
\pron{je(ji)neTikali}
\gl{\kirxvi}
\bmng
\bnum
\num{1} huTiTxna, ugamada, mUlada -- daqSiTxyiMda. 
\num{2} taLivijAcnxnadaMte; taLivijAcnxnada parxkAra. 
\enum
\emng
\eentry

\bentry
\wordnospeech{genetic code}{genetic code}
\pron{?}
\gl{\nA}
\bmng
 AnuvaMshika saMkeVtaBASe; jiVvakoVshadoLage nUyxkelxyikf Amalxvu porxVTiVnf aNuvina saMshelxVSaNeyanunx nideRVshisuva saMdaBaRdalilx nUyxkelxyikf Amalxdalilxna nUyxkilxyoVTeYDfgaLa anukarxmakUkx A nUyxkelxyikf Amalxda nideRVshanadalilx saMshelxVSaNegoLuLxva porxVTiVninalilxna amInoV AmalxgaLa anukarxmakUkx iruva saMbaMdhavanunx sUcisuva ``saMkeVta''gaLa vayxvasethx. 
\emng
\eentry

\bentry
\wordnospeech{genetic engineering}{genetic engineering}
\pron{?}
\gl{\nA}
\bmng
 taLi eMjiniyariMgf; jiVviyoMdara AnuvaMshika lakaSxNagaLanunx nidhaRrisuva jiVnfgaLa samudAyakekx udedxVshapUvaRkavAgi apeVkiSxta jiVnfgaLanunx seVrisuva mUlaka athavA inAnxvudeV riVtiyalilx jiVnfgaLanunx badalAyisuva mUlaka jiVviya AnuvaMshikateyanunx badalAyisuvudu. 
\emng
\eentry

\bentry
\wordnospeech{genetic fingerprinting}{genetic fingerprinting}
\pron{?}
\gl{\nA}
\bmng
 jeneTikf beraLa gurutu; jiVviyoMdara jiVvakoVshadoLagiruva jiVnf samudAyada neraviniMda jiVviyanunx gurutisuva karxma. 
\emng
\eentry

\bentry
\word{geneticist}
\pron{je(ji)neTisisfTx}
\gl{\nA}
\bmng
 taLivijAcnxni; taLivijAcnxnadalilx pariNata. 
\emng
\eentry

\bentry
\word{genette}
\pron{jeniTf}
\gl{\nA}
\bmng
  = \hyperlink{genet}{genet}. 
\emng
\eentry

\bentry
\word{geneva}
\pron{jiniVva}
\gl{\nA}
\bmng
 jiniVva; hAlaMDfsx jinunx (madayx); dhAnayxdiMda Asavisi `jUniparf' haNuNxgaLiMda rucigoLisida oMdu bageya madayx. 
\emng
\eentry

\bentry
\word{Geneva}
\pron{jiniVva}
\gl{\gu}
\bmng
 (visheVSaka \parx) (sivxTasxralxMDina) jiniVva nagarada yA jiniVva nagaradiMda baMda. 
\emng
\eentry

\bentry
\wordnospeech{Geneva bands}{Geneva bands}
\pron{?}
\gl{\nA}
\bmng
 jiniVva kAlarugaLu; sivxsf kAyxlivxnisfTx paMthadavara koraLupaTiTxgaLaMtaha pAdirxgaLa koraLupaTiTxgaLu.  \imglink{Geneva-bandsfigure}{\raisebox{-0.15cm}[0pt][0pt]{\pdfimage width 0.6cm height 0.6cm {G_Pictures/Geneva-bands.jpg}}} 
\emng
\eentry

\bentry
\wordnospeech{Geneva convention}{Geneva convention}
\pron{?}
\gl{\nA}
\bmng
 (yudadhxdalilx gAyagoMDavaru, AsapxterxgaLu, saMcAriV cikitAsxlayagaLu, \mo vanunx taTasathx sAdhanagaLeMdu, aMdare hoVrATada pakaSxgaLige seVridadxlalxveMdu vidhisida \eng{1864--65}ra) jiniVvA opapxMda. 
\emng
\eentry

\bentry
\wordnospeech{Geneva Cross}{Geneva Cross}
\pron{?}
\gl{\nA}
\bmng
 jiniVvA shilube; yudadhxdalilx saMcAriV Asapxterx \mo vugaLanunx gurutisalu yoVjisida, biLiya hinenxleya meVle citirxsida keMpu shilube. 
\emng
\eentry

\bentry
\wordnospeech{Geneva gown}{Geneva gown}
\pron{?}
\gl{\nA}
\bmng
 jiniVva gwnu; upadeVsha veVdikeyalilx kAyxlivxnisfTx paMtha matutx loV cacfR paMthadavaru toDuva kariya gwnu. 
\emng
\eentry

\bentry
\wordnospeech{Geneva Protocol}{Geneva Protocol}
\pron{?}
\gl{\nA}
\bmng
 jiniVva opapxMda; yudadhxdalilx viSAnilagaLanUnx bAyxkiTxVriyagaLanUnx baLasabAradeMba opapxMda. 
\emng
\eentry

\bentry
\word[Genevese(1)]{Genevese}
\pron{jeniviVsfZ}
\gl{\gu}
\bmng
 (sivxTasxrelxMDinalilxna) jiniVvada. 
\emng
\eentry

\bentry
\word[Genevese(2)]{Genevese}
\pron{jeniviVsfZ}
\gl{\nA}
\bmng
 (sivxTasxrelxMDinalilxna) jiniVvadava(Lu). 
\emng
\eentry

\bentry
\word[genial(1)]{genial}
\pron{jiVnialf, jiVnayxlf}
\gl{\gu}
\bmng
\bnum
\num{1} (\viparx) maduveya; vivAhada; veYvAhika; niSeVkada. 
\num{2} (\viparx) janaka; saMtAnoVtApxdaka; saMBoVgada: \eng{genial bed} saMBoVgada sejejx. 
\num{3} (havA, vAyuguNa, \mo vugaLa \vi) beLavaNigege anukUlavAda; hitakaravAda; becacxgiruva. 
\num{4} geluvu uMTumADuva; ulAlxsakara. 
\num{5} saMtoVSa parxkaqtiya; sarasiyAda. 
\num{6} dayALuvAda; karuNe tuMbida. 
\num{7} senxVhapara; saKayxshiVla. 
\num{8} (\viparx) parxtiBeya; parxtiBeyanunx uLaLx yA toVrisuva. 
\enum
\emng
\eentry

\bentry
\word[genial(2)]{genial}
\pron{jiniValf}
\gl{\gu}
\bmng
 (\aMrashA) gadadxda; cibukada. 
\emng
\eentry

\bentry
\word{genialise}
\pron{jiVnialeYsfZ}
\gl{\sakirx}
\bmng
=  \hyperlink{genialize}{genialize}. 
\emng
\eentry

\bentry
\word{geniality}
\pron{jiVniARYxliTi}
\gl{\nA}
\bmng
\bnum
\num{1} saMtoVSada parxkaqti; geluvu tuMbiruvudu. 
\num{2} dayALutana; karuNe tuMbiruvike. 
\num{3} senxVhaparate; saKayxshiVlate. 
\enum
\emng
\eentry

\bentry
\word{genialize}
\pron{jiVnialeYsfZ}
\gl{\sakirx}
\bmng
\bnum
\num{1} saMtoVSa parxkaqtiyuLaLxvananAnxgi mADu; sarasiyanAnxgi mADu. 
\num{2} dayALuvanAnxgi mADu. 
\num{3} senxVhaparananAnxgi, saKayxshiVlananAnxgi mADu. 
\enum
\emng
\eentry

\bentry
\word{genic}
\pron{jiVnikf}
\gl{\gu}
\bmng
 (\jiVvi) jiVnina; jiVnige saMbaMdhisida. 
\emng
\eentry

\bentry
\wordwithhyphen{hyp-genic}{-genic}
\pron{-jenikf}
\gl{\saupa}
\bmng
 I athaRgaLa guNavAcakagaLanunx racisuva \saupa: 
\bnum
\num{1} janaka; utApxdaka; utapxtitx mADuva: \eng{pathogenic} roVgoVtApxdaka. 
\num{2} lAyakAkxda; -- kekx hoMduva; takukxdAda; yoVgayxvAda; ucitavAda: \eng{photogenic} phoVToVyoVgayx. 
\num{3} -- iMda uMTAda, utapxtitxyAda; -- janayx: \eng{iatrogenic} cikitAsxjanayx. 
\num{4} (\jiVvi) jeYnika; jiVnige yA jiVnugaLige saMbaMdhisida. 
\enum
\emng
\eentry

\bentry
\wordwithhyphen{hyp-genically}{-genically}
\pron{-jenikali}
\gl{\saupa}
\bmng
 I athaRgaLa \kirxvi gaLanunx racisuvalilx baLasuva \saupa: 
\bnum
\num{1} -- janakavAgi; utApxdakavAgi; utapxtitx mADuvaMte. 
\num{2} takukxdAgi; yoVgayxvAgi; lAyakAkxgiruvaMte. 
\num{3} -- janayxvAgi; -- iMda utapxtitxyAda riVtiyalilx. 
\enum
\emng
\eentry

\bentry
\word{geniculate}
\pron{jenikuyxleVTf}
\gl{\gu}
\bmng
 (\jiVvi) 
\bnum
\num{1} moNakAlinaMtha kiVlugaLuLaLx; jAnusaMdhiya. 
\num{2} kiVlinalilx moNakAlinaMte bAgiruva. 
\enum
\emng
\eentry

\bentry
\word{geniculated}
\pron{jenikuyxleVTiDf}
\gl{\gu}
\bmng
 (\jiVvi)  = \hyperlink{geniculate}{geniculate}. 
\emng
\eentry

\bentry
\word{genie}
\pron{jiVni}
\gl{\nA}
\expl{(\bava\ \sA\ \eng{genii} \ucAcx\ jiVniai).}
\bmng
(areVbiyanf kathegaLa) BUtagaLu; atimAnuSa jiVvigaLu; beVtALagaLu. 
\emng
\eentry

\bentry
\word{genii}
\pron{jiVniai}
\gl{\nA}
\bmng
 \eng{genius} hAgU \eng{genie} padagaLa \bava. 
\emng
\eentry

\bentry
\word{genista}
\pron{jenisaTx}
\gl{\nA}
\bmng
(kelavaru vagiRVkarisiruvaMte haMciVkaDiDxya jAtiyU seVrida) haLadi hUbiDuva bagebageya kurucalu podegaLu. 
\emng
\eentry

\bentry
\word{genital}
\pron{jeniTalf}
\gl{\gu}
\bmng
\bnum
\num{1} (pArxNigaLa) saMtAnoVtapxtitxya. 
\num{2} jananAMgagaLa. 
\enum
\emng
\eentry

\bentry
\word{genitalia}
\pron{jeniTeVlia}
\gl{\nA}
\expl{}
\bmng
(\bava) = \hyperlink{genitals}{genitals}. 
\emng
\eentry

\bentry
\word{genitals}
\pron{jeniTalfsx}
\gl{\nA}
\bmng
 jananAMgagaLu. 
\emng
\eentry

\bentry
\word{genitival}
\pron{jeniTeYvalf}
\gl{\gu}
\bmng
  = \hyperlink{genitive(1)}{$^1$genitive}. 
\emng
\eentry

\bentry
\word{genitivally}
\pron{jeniTeYvali}
\gl{\kirxvi}
\bmng
 SaSiThxV viBakitxyalilx; SaSiThxV viBakitxyAgi. 
\emng
\eentry

\bentry
\word[genitive(1)]{genitive}
\pron{jeniTivf}
\gl{\gu}
\bmng
 (\vAyx) SaSiThxV viBakitxya. 
\emng
\eentry

\bentry
\word[genitive(2)]{genitive}
\pron{jeniTivf}
\gl{\nA}
\bmng
 (\vAyx) SaSiThxV viBakitx; saMbaMdhasUcaka viBakitx. 
\emng

\noindent
\gl{\pagu}
\bmng
\bnum
\num{1} \eng{genitive absolute} lAYxTinf BASeya savxtaMtarx paMcamI viBakitxyanunx hoVluva girxVkf rUpa. 
\num{2} \eng{genitive case} = \hyperlink{genitive(2)}{$^2$genitive}. 
\enum
\emng
\eentry

\bentry
\word{genito-}
\pron{jeniToV-}
\gl{\sapUpa}
\bmng
 `saMtAnoVtapxtitxya matutx' eMbathaRda \sapUpa. 
\emng
\eentry

\bentry
\word{genito-urinary}
\pron{jeniToVyuarinari}
\gl{\gu}
\bmng
 (\aMrashA) saMtAnoVtapxtitx matutx mUtarxda aMgagaLige saMbaMdhisida. 
\emng
\eentry

\bentry
\word{genius}
\pron{jiVniasf, jiVnayxsf}
\gl{\nA}
\expl{(\bava\ \eng{geniuses}, yA \eng{genii} \ucAcx\ jiVniai).}
\bmng
\bnum
\num{1} rakaSxka deVvate; aBimAni deVvate. 
\num{2} sathxLadeVvate; gArxmadeVvate. 
\num{3} kuladeVvate; gaqhadeVvate. 
\num{4} adhideVvate: \eng{good genius} BAgayxdeVvate; vayxkitxyu moVkaSx yA sherxVyasusx sAdhisalu anukUlavAda deVvate. \eng{evil genius} dwBARgayx deVvate; dudeRVvate; vayxkitxyu narakakekx yA dugaRtige guriyAgalu kAraNavAda deVvate. 
\num{5} (\sA\ \bava dalilx \eng{genii})  = \hyperlink{genie}{genie}. 
\num{6} (rASaTxrX, yuga, \mo vugaLa parxcalita) manoVdhamaR; BAvane; daqSiTx; aBiruci. 
\num{7} (BASe, kAnUnu, \mo vugaLa) savxBAva; lakaSxNa; savxrUpa; dhamaR; jiVvALa; jAyamAna; mayARde; majiR; mamaR; riVti; parxvaqtitx; vidhAna; saMparxdAya. 
\num{8} (oMdu sathxLadoDane beretuhoVgiruva matutx adara samxraNeyoDaneyeV baruva) samxqqtigaLu yA sUPxtiRgaLu. 
\num{9} sahaja -- shakitx, sAmathayxR yA parxvaqtitx. 
\num{10} vishiSaTxvAda budidhxshakitx; visheVSa lakaSxNagaLiMda kUDida budidhxya -- guNa, sAmathayxR. 
\num{11} unanxta, asAdhAraNa -- bwdidhxka shakitx. 
\num{12} (viBAvane, kalAsaqSiTx matutx vasutxsaqSiTxgaLalilx sahajasidadhxvAda) asAdhAraNa shakitx; parxtiBe. 
\num{13} (\bava\ dalilx, \eng{geniuses}) parxtiBAshAligaLu; asAdhAraNa vayxkitxgaLu. 
\enum
\emng
\eentry 

\bentry
\wordf{genius loci}
\pron{jiVniasf loVseY}
\gl{\nA}
\expl{\eng{Latin} (\bava\ \engit{genii lici}). }
\bmng
\bnum
\num{1} adhideVvate; sathxLABimAni deVvate. 
\num{2} sathxLa -- mahime, mAhAtamxyX; sathxLadoDane kUDikoMDiruva BAvanegaLu, samxqqtigaLu, \mo vu. 
\enum
\emng
\eentry

\bentry
\word{genizah}
\pron{jeVnisaZ}
\gl{\nA}
\bmng
 hALAda, tirasakxqqtavAda, yA saMparxdAyabAhiravAda pusatxkagaLanUnx pavitarx avasheVSagaLanUnx iDuva, yehUdayxra ArAdhana maMdirakekx seVrida koVNe. 
\emng
\eentry

\bentry
\word{Genoa}
\pron{jenoVa}
\gl{\nA}
\bmng
\bnum
\num{1} iTaliya oMdu paTaTxNa. 
\num{2}  = \hyperlink{Genoa jib}{Genoa jib}. 
\enum
\emng
\eentry

\bentry
\wordnospeech{Genoa cake}{Genoa cake}
\pron{?}
\gl{\nA}
\bmng
 bAdAmi keVku; meVlABxgadalilx dapapxnAgi bAdAmi seVrisida, oNadArxkiSx \mo vanunx hAkida keVku. 
\emng
\eentry

\bentry
\wordnospeech{Genoa jib}{Genoa jib}
\pron{?}
\gl{\nA}
\bmng
 paMdayxda doVNiya doDaDx paTa. 
\emng
\eentry

\bentry
\word{genocidal}
\pron{jenaseYDalf}
\gl{\gu}
\bmng
 janahateyxya yA janahateyxge saMbaMdhisida. 
\emng
\eentry

\bentry
\word{genocide}
\pron{jenaseYDf}
\gl{\nA}
\bmng
 janahateyx; yAvudeV vagaR, jAti, saMsakxqqti, muMtAdavakekx seVrida janara hateyx. 
\emng
\eentry

\bentry
\word[Genoese(1)]{Genoese}
\pron{jenoVIsfZ}
\gl{\gu}
\bmng
 (iTaliya) jenoVvada. 
\emng
\eentry

\bentry
\word[Genoese(2)]{Genoese}
\pron{jenoVIsfZ}
\gl{\nA}
\bmng
 jenoVvadavanu. 
\emng
\eentry

\bentry
\word{genotype}
\pron{jiV(je)naTeYpf}
\gl{\nA}
\bmng
 (\jiVvi) jiVnf namUne; jiVnoVTeYpf; (pArxNi yA sasayxda) jiVvaMta vayxkitxyoMdara jiVnf saMyoVjane. 
\emng
\eentry

\bentry
\word{genotypic}
\pron{jiV(je)naTipikf}
\gl{\gu}
\bmng
 jiVnf namUneya; vayxkitxya jiVnf saMyoVjaneya yA adakekx saMbaMdhisida. 
\emng
\eentry

\bentry
\wordwithhyphen{hyp-genous}{-genous}
\pron{-jenasf}
\gl{\saupa}
\bmng
 utapxtitxyAda, -janayx, -janita, -huTiTxda, -jAta eMbathaRda guNavAcakagaLanunx racisalu baLasuva \uparx: \eng{endogenous} aMtajaRnita; aMtajARta. 
\emng
\eentry

\bentry
\word{genre}
\pron{SAZnfrx}
\gl{\nA}
\bmng
\bnum
\num{1} (\kanmu\ kAvayxda yA kaleya) parxkAra; bage; sheYli; rUpa; jAti. 
\hypertarget{genre(2)}{} 
\num{2} (citarxkale) loVkacitarxNa; sAmAnayx janajiVvanada daqshayxgaLanunx citirxsuva sheYli. 
\enum
\emng
\eentry

\bentry
\word{genre-painting}
\pron{SAZnfrxpeVMTiMgf}
\gl{\nA}
\bmng
  = \hyperlink{genre(2)}{genre (2)}. 
\emng
\eentry

\bentry
\word{genro}
\pron{jenorxV}
\gl{\nA}
\bmng
 (\bava) (aneVka veVLe \eng{Genro}). japAnina hiriya rAjataMtarxjacnxru. 
\emng
\eentry

\bentry
\word{gens}
\pron{jenfs'}
\gl{\nA}
\expl{(\bava\ \eng{gentes} \ucAcx\ jenfTiVsfZ).}
\bmng
\bnum
\num{1} (\pArxparx) (girxVkaralilx, roVmanaralilx) buDakaTuTx; baNa; kula; oMdeV mUlada matutx hesaru, dhAmiRka vidhigaLanunx samAnavAgi hoMdiruva kuTuMbagaLa guMpu, vagaR. 
\num{2} (\jiVvi) jiVvibaNa; jiVvikula; parasapxra saMbaMdhavuLaLx jiVvigaLa taMDa. 
\num{3} peYtaqkavaMsha; pitaqvaMsha; taMdeya kaDeyiMda baMda vaMsha, sAlu. 
\enum
\emng
\eentry

\bentry
\word{gent}
\pron{jeMTf}
\gl{\nA}
\bmng
\bnum
\num{1} (\asaM) saBayx; saMBAvita; doDaDx manuSayx. 
\num{2} (\hA) saBayxsoVgi; saMBAvitanaMte naTisuvavanu. 
\num{3} (\bava dalilx) (aMgaDigaLa PalakagaLalilx) gaMDasaru: \eng{gents hairdresser} gaMDasara kwSxrika, keVshAlaMkAri. 
\enum
\emng

\noindent
\gl{\pagu}
\bmng
 \eng{the Gents} (\birx) (\AmA) gaMDasara sAvaRjanika shwcagaqha, shwcAlaya. 
\emng
\eentry

\bentry
\word{genteel}
\pron{jeMTiVlf}
\gl{\gu}
\bmng
\bnum
\num{1} (\pArxparx, vayxMgayx \parx, yA \asaM) kuliVna; meVlavxgaRda; utatxmavagaRda; meVlina aMtasitxnavarige takakx; avara veYshiSaTxyXda, lakaSxNada; avarige -- saMbaMdhisida, ucitavAda, yoVgayxvAda. 
\numi{2} (nijavAgiyU yA toVrikeyAgi) 
\banum
\alnum{a} ThiVviya; ThiVkina. 
\alnum{b} sogasina; niVTutanada. 
\alnum{c} aMdavAda; nAjUkAda. 
\alnum{d} sogasuDipina; aMdavAgi uDupu toTiTxruva. 
\eanum
\numie
\enum
\emng
\eentry

\bentry
\word{genteelism}
\pron{jeMTiVlisaZmf}
\gl{\nA}
\bmng
 shiSaTx -- parxyoVga, pada; yAvudeV oMdu shabadxkikxMta hecucx saBayxveMdu heVLalAda inonxMdu shabadx. \udA\ \eng{bitch} (heNuNxnAyi) enunxvudara badalu \eng{lady-dog} (sitxrXVshAvxna), \eng{sweat} (bevaru) badalu \eng{perspire} (sevxVda) muMtAda parxyoVgagaLu. 
\emng
\eentry

\bentry
\word{genteelly}
\pron{jeMTiVlfli}
\gl{\kirxvi}
\bmng
\bnum
\num{1} saBayxvAgi; shiSaTxriVtiyalilx. 
\num{2} nAjUkAgi. 
\num{3} ThiVviyiMda; ThiVkAgi. 
\num{4} niVTugArikeyiMda; sogasina riVtiyalilx. 
\num{5} aMdavAgi uDupu toTuTx. 
\enum
\emng
\eentry

\bentry
\word{gentes}
\pron{jeMTiVsfZ}
\gl{\nA}
\bmng
 \eng{gens} padada \bava. 
\emng
\eentry

\bentry
\word{gentian}
\pron{jenfSa(Sf)nf, jeniSxanf}
\gl{\nA}
\bmng
\bnum
\num{1} cirAyata; kiriyAtu giDa; \sA\ beTaTxguDaDxgaLalilx beLeyuva, niVli hUbiDuva oMdu sasayxjAti. 
\num{2}  = \hyperlink{gentian bitter}{gentian bitter}. 
\enum
\emng
\eentry

\bentry
\wordnospeech{gentian bitter}{gentian bitter}
\pron{?}
\gl{\nA}
\bmng
 cirAyata madayx; cirAyata giDada beVriniMda iLisida, deVhadADhayxR koDuva, utetxVjakavAda kahi madayx. 
\emng
\eentry

\bentry
\wordnospeech{gentian violet}{gentian violet}
\pron{?}
\gl{\nA}
\bmng
 cirAyata raMgu, vaNaR; suTaTx gAyagaLanunx vAsimADalu upayoVgisuva, cirAyatadiMda mADida. UdA baNaNxda, pUtinAshaka vaNaRdarxvayx. 
\emng
\eentry

\bentry
\word[gentile(1)]{gentile}
\pron{jeMTeYlf}
\gl{\gu}
\bmng
\bnum
\num{1} (yehUdayxra baLakeyalilx) yehUdeyxVtara. 
\num{2} (mAmaRnara baLakeyalilx) mAmaRneVtara; mAmaRnf guMpige seVrirada. 
\num{3} (\vAyx) deVshavAcaka yA janAMgavAcaka; oMdeV deVsha yA janAMgavanunx sUcisuva: \eng{British, German, Irish, etc. are gentile adjectives} birxTiSf, jamaRnf, airiSf, \mo vugaLu janAMgasUcaka guNavAcakagaLu. 
\num{4} `peVganf'; kerxYsatx, yehUdayx, yA musilxM alalxda. 
\num{5} yAvudeV janAMga yA baNavanunx sUcisuva. 
\enum
\emng
\eentry

\bentry
\word[gentile(2)]{gentile}
\pron{jeMTeYlf}
\gl{\nA}
\bmng
\bnum
\num{1} (yehUdayxra baLakeyalilx) yehUdeyxVtara (vayxkitx). 
\num{2} (mAmaRnara baLakeyalilx) mAmaRneVtara (vayxkitx); mAmaRnf matiVyanalalxdavanu. 
\num{3} (\vAyx) deVshavAcaka yA janAMgavAcaka (pada): \eng{the words Italian, American are gentiles} iTAYxliyanf, amerikanf eMba padagaLu deVshavAcakagaLu, janAMgavAcaka padagaLu. 
\num{4} `peVganf'; kerxYsatx, yehUdayx yA musilxM alalxdavanu. 
\enum
\emng
\eentry

\bentry
\word{gentiledom}
\pron{jeMTeYlfDamf}
\gl{\nA}
\bmng
\bnum
\num{1} yehUdeyxVtara loVka, parxpaMca; yehUdeyxVtara naMbike, AcAragaLa vAyxpitxgoLapaTaTx parxdeVsha. 
\num{2} yehUdeyxVtaranAgiruvike; yehUdeyxVtarara naMbike, AcAragaLanunx hoMdiruvudu. 
\enum
\emng
\eentry

\bentry
\word{gentilitial}
\pron{jeMTiliSa(Sf)lf}
\gl{\gu}
\bmng
\bnum
\num{1} oMdu janAMgada, kulada, buDakaTiTxna, baNada, manetanada: \eng{gentilitial insignia} oMdu kulada lAMCana, gurutu. 
\num{2} (samAjada) divxtiVya vagaR yA sherxVNiyavara; huTuTx kuliVnara taruvAyada keLagina vagaRkekx seVrida. 
\enum
\emng
\eentry

\bentry
\word{gentility}
\pron{jeMTiliTi}
\gl{\nA}
\bmng
\bnum
\numi{1} (\pArxparx) 
\banum
\alnum{a} sadavxMsha janana; kuliVnate; sadavxMshate. 
\alnum{b} kuliVna vagaR. 
\alnum{c} kuliVnaru. 
\eanum
\numie
\num{2} (sAmAjika) doDaDxsitxke; shirxVmaMtike; kuliVnavagaRkekx seVriruvike. 
\num{3} saMBAvitate; saBayxte; saBayx naDavaLike. 
\num{4} swjanayx; vinaya. 
\num{5} kuliVnate; utatxmavagaRdavara shiVla savxBAvagaLu; kuliVna naDenuDigaLu. 
\enum
\emng

\noindent
\gl{\pagu}
\bmng
 \eng{shabby gentility} doDaDxsitxkeya soVgu. 
\emng
\eentry

\bentry
\word[gentle(1)]{gentle}
\pron{jeMTflf}
\gl{\gu}
\bmng
\bnum
\num{1} kuliVna vaMshada; utatxma kulada; vaMshalAMCana dharisuva hakukxLaLx. 
\num{2} (\eng{gentleman} \mo\ samasatx padagaLalilx) aBijAta; sadavxMshiVya; satukxlaparxsUta; utatxma kuladalilx huTiTxda. 
\num{3} (huTuTx, rakatx, vaMsha, udayxma, \mo vugaLa \vi) mAnayx; gwravAhaR; shiSaTx; shiSoTxVcita. 
\num{4} (\pArxparx) (IgalU vinoVdavAgi garxMthakataRnu Odugaranunx saMboVdhisuvAga): udAra; viniVta; saMBAvita: \eng{gentle reader!} udAra pAThaka mahAshaya! 
\num{5} sAdhu; swmayx; shAMta. 
\num{6} sAdhu; sulaBavAgi nivaRhisalAguva; hatoVTiyalilxTuTxkoLaLxlAguva: \eng{a gentle horse} sAdhuvAda kudure. 
\num{7} maqdu; metatxneya; swmayx; birusu, oraTutana, kAThiNayx, ugarxtegaLilalxda: \eng{gentle breeze} maqdu mAruta; maMdAnila. 
\num{8} (auSadhi) swmayx; tiVkaSxNXvalalxda. 
\num{9} (niyama, vidhi, \mo vu) swmayx; kaThiNavalalxda; hitavAda; mitavAda. 
\num{10} nasu; maMda; sAdhAraNa; hALatavAda: \eng{a gentle heat} nasu kAvu; maMdoVSaNx. 
\num{11} anukarxmavAda; nidhAnavAgi, karxmakarxmavAgi, karxmeVNa -- Aguva: \eng{a gentle slope} nasuvoVre; savxlapxsavxlapxvAgi Eruva yA iLiyuva Ore, iLukalu. 
\num{12} dayApara; maqdu haqdayada; koVmala; sukumAra: \eng{the gentle touch of her hand} avaLa keYya maqdu sapxshaR. 
\enum
\emng

\noindent
\gl{\pagu}
\bmng
\hyperdef{G}{gentle(1) pagu(1)}{} 
\bnum
\numi{1} \eng{the gentle art} (\engit{or} \eng{craft)} 
\banum
\alnum{a} gALagArike; gALadiMda mInu hiDiyuvudu. 
\alnum{b} nayagArike; tALemx athavA (vayxMgayxvAgi heVLuvAga) otAtxya beVkAguva yAvudeV kAyaR, caTuvaTike. 
\eanum
\numie
\num{2} \eng{the gentle sex} koVmaleyaru; sitxrXVyaru. 
\enum
\emng
\eentry

\bentry
\word[gentle(2)]{gentle}
\pron{jeMTflf}
\gl{\nA}
\bmng
 gALada huLu; mInina gALadalilx upayoVgisuva mAMsada noNa \mo vugaLa huLumari. 
\emng
\eentry

\bentry
\word[gentle(3)]{gentle}
\pron{jeMTflf}
\gl{\sakirx}
\bmng
\bnum
\num{1} kudureyanunx -- paLagisu, sAdhumADu. 
\num{2} kudureyanunx daqDhavAgi, Adare maqduvAgi hiDidu naDesu. 
\num{3} ucACxrXyakekx taru; meVlina padavige etutx, Erisu; shirxVmaMtara jatege seVrisu: \eng{trading class which having enriched itself, sought desperately to gentle itself} beVkAdaSuTx duDuDx mADida vataRkavagaR shirxVmaMtara sAlige Eralu bahu parxyAsa paTiTxtu. 
\num{4} paLagisu; nayagoLisu; maqdugoLisu; naDateya riVtiniVtigaLalilx maqduvAgi mADu: \eng{honoured for gentling the barbarian} anAgarikananunx paLagisidaneMba kiVtiRge pAtarxnAgi. 
\num{5} sAMtavxnagoLisu; samAdhAnapaDisu; saMtayisu; boVLeYsu: \eng{the old man is in a rage and has to be gentled} muduka ati koVpadalilxdAdxne, avananunx samAdhAna mADabeVku. 
\num{6} nidhAnavAgi, maMdagatiyalilx -- sAgu, naDe: \eng{the broad shouldered train gentles its way} mahABujada reYlu maMdagatiyalilx sAgutatxde. 
\enum
\emng
\eentry

\bentry
\word{gentlefolks}
\pron{jeMTflfphoVkfsx}
\gl{\nA}
\bmng
 saBayxru; gaqhasathxru; manetanasathxru; kuliVna vaMshasathxru; utatxma vaMshiVyaru. 
\emng
\eentry

\bentry
\word{gentlehood}
\pron{jeMTflfhuDf}
\gl{\nA}
\bmng
\bnum
\num{1} kuliVnate; saBayxte; sadavxMsha -- lakaSxNa, sAthxna, sithxti. 
\num{2} sajajxnike; swjanayx; maqdu savxBAva. 
\enum
\emng
\eentry

\bentry
\word{gentleman}
\pron{jeMTflfmanf}
\gl{\nA}
\expl{(\bava\ \eng{gentlemen}).}
\bmng
\bnum
\num{1} (shirxVmaMta vagaRdavanalalxdidadxrU vaMshalAMCana dharisuva hakukxLaLx) sadavxMshadavanu; kuliVna. 
\num{2} (\pArxparx) seYnayx, vakiVli, \mo\ kelavu vaqtitxgaLige seVridavanu. 
\num{3} rAjana yA doDaDx padaviyalilxruvavana parivArakekx seVrida sadavxMshiVya. 
\num{4} (udAra parxvaqtitx, utatxma naDenuDi matutx sushikaSxNa {u}LaLx) saMBAvita; saBayx. 
\num{5} samAjadalilx oLeLxya sAthxnavuLaLx vayxkitx. 
\num{6} ArAma shirxVmaMta; aishavxyaR matutx virAma uLaLxvanu; duDiDxdudx jiVvanakAkxgi duDiyabeVkAgilalxde iruvavanu. 
\num{7} (pAliRmeMTu \mo vugaLalilx, mayARdeya mAtAgi) manuSayx. 
\num{8} (\bava dalilx) (saBeyalilxna puruSavagaRvanunx saMboVdhisuva mAtAgi) mahaniVyare! 
\num{9} (patarxvayxvahAradalilx saMboVdhane mADuvAga) mahAshaya! 
\num{10} (\ca) haNakAkxgi ADada kirxkeTf ATagAra; vaqtitxparanalalxda kirxkeTf ATagAra. 
\num{11} (\sw) kaLaLxsAgANikegAra. 
\enum
\emng

\noindent
\gl{\pagu}
\bmng
\bnum
\num{1} \eng{gentleman at} \hyperref{kandict_l.pdf}{L}{large(2) pagu(2)}{$^2$large}. 
\num{2} \eng{gentleman in waiting} (aramane yA shirxVmaMtara maneyalilxna) hAjaruBASi; apapxNe kAyuva sadavxMshiVya, doDaDx manuSayx. 
\num{3} \eng{gentleman's gentleman} doDaDx manuSayxna anucara, seVvaka. 
\num{4} \eng{my gentleman} nAnu heVLutitxdadx vayxkitx. 
\num{5} \eng{old gentleman} (vinoVdavAgi) seYtAna. 
\num{6} \eng{the Gentlemen('s)} (\birx) (\Eva vAgi \parx) puruSara, gaMDasara sAvaRjanika shwcagaqha, shwcAlaya. 
\enum
\emng

\noindent
\gl{\nuga}
\bmng
 \eng{gentleman's} (\engit{or} \eng{gentlemen's) agreement} saBayxra opapxMda; saMBAvitara karAru; niVtibadadhxvAda, Adare kAnUnina meVrege karxma naDesalAgada opapxMda. 
\emng
\eentry

\bentry
\word{gentleman-at-arms}
\pron{jeMTflfmaneTAmfs'R}
\gl{\nA}
\bmng
 rAjana aMgarakaSxka. 
\emng
\eentry

\bentry
\wordnospeech{gentleman commoner}{gentleman commoner}
\pron{?}
\gl{\nA}
\bmng
 (AkfsxphaDfR matutx keVMbirxjfgaLalilx kelavu gotAtxda riyAyitigaLuLaLx) sAnxtakapUvaR vidAyxthiR. 
\emng
\eentry

\bentry
\wordnospeech{gentleman farmer}{gentleman farmer}
\pron{?}
\gl{\nA}
\bmng
 kuliVna kaqSikAra; saBayx beVsAyagAra; aishavxyaRvU manetanada doDaDxsitxkeyU idudx lABAkAMkeSxyilalxde, keVvala saMtoVSakAkxgi tananx jamiVnanunx beVsAya mADuva doDaDx manuSayx. 
\emng
\eentry

\bentry
\word{gentlemanhood}
\pron{jeMTflfmanfhuDf}
\gl{\nA}
\bmng
 shirxVmaMtike; doDaDxsitxke; shirxVmaMtana lakaSxNa, aMtasutx. 
\emng
\eentry

\bentry
\word{gentlemanlike}
\pron{jeMTflfmanfleYkf}
\gl{\gu}
\bmng
\bnum
\num{1} doDaDx manuSayxnige takakx; saBayxgaqhasothxVcita. 
\num{2} saBayx gaqhasathxnaMtha; doDaDx manuSayxnanunx hoVluva. 
\enum
\emng
\eentry

\bentry
\word{gentlemanliness}
\pron{jeMTflfmanilxnisf}
\gl{\nA}
\bmng
 saBayxte; shiSaTxte; sadagxqqhasathxna naDevaLike, riVti niVti. 
\emng
\eentry

\bentry
\word{gentlemanly}
\pron{jeMTflfmanilx}
\gl{\gu}
\bmng
\bnum
\num{1} saMBAvita vataRneya, naDevaLikeya. 
\num{2} saBayxtanada; doDaDx manuSayxnige takakx. 
\enum
\emng
\eentry

\bentry
\wordnospeech{gentleman ranker}{gentleman ranker}
\pron{?}
\gl{\nA}
\bmng
 (sAmAnayx sipAyiyAgidudx) baDitxyiMda sananxdu paDeda adhikAri. 
\emng
\eentry

\bentry
\word{gentlemanship}
\pron{jeMTflfmanfSipf}
\gl{\nA}
\bmng
\bnum
\num{1}  = \hyperlink{gentlemanhood}{gentlemanhood}. 
\num{2} shirxVmaMtana naDevaLike. 
\enum
\emng
\eentry

\bentry
\wordnospeech{gentleman usher}{gentleman usher}
\pron{?}
\gl{\nA}
\bmng
 shirxVmaMtana -- harikAra, dwvArika, dAvxrapAlaka. 
\emng
\eentry

\bentry
\word{gentleness}
\pron{jeMTflfnisf}
\gl{\nA}
\bmng
\bnum
\num{1} dayAparate. 
\num{2} maqdutavx; maqdusavxBAva; koVmalate; melupu. 
\num{3} sAdhu guNa; swmayxte. 
\num{4} vinaya; swjanayx. 
\num{5} hitavAgiruvike. 
\num{6} kaDidAgilalxdiruvudu; hecucx iLukalu ilalxdiruvike. 
\num{7} karxmeVNa nidhAnavAgi Aguvike, saMBavisuvike. 
\enum
\emng
\eentry

\bentry
\word{gentlepeople}
\pron{jeMTflfpiVpflf}
\gl{\nA}
\bmng
  = \hyperlink{gentlefolks}{gentlefolks}. 
\emng
\eentry

\bentry
\word{gentlewoman}
\pron{jeMTflfvumanf}
\gl{\nA}
\expl{(\bava\ \eng{gentlewomen} \ucAcx\ jeMTflfviminf).}
\bmng
(\pArxparx) 
\bnum
\num{1} kuliVne; satukxlaparxsUte; saMBAvite; saBeyx; garati. 
\num{2} mahiLe; sitxrXV. 
\enum
\emng
\eentry

\bentry
\word[gently(1)]{gently}
\pron{jeMTfli}
\gl{\kirxvi}
\bmng
\bnum
\num{1} sadidxlalxde; melalxne. 
\num{2} metatxge; koVmalavAgi; maqduvAgi. 
\num{3} swmayxvAgi. 
\num{4} nidhAnavAgi; sAvadhAnavAgi. 
\num{5} nayavAgi; hitavAgi. 
\num{6} pirxVtiyiMda; dayApUritavAgi. 
\enum
\emng

\noindent
\gl{\pagu}
\bmng
 \eng{gently born} suparxsUta; sadavxMshada; kuliVnavaMshada. 
\emng
\eentry

\bentry
\word[gently(2)]{gently}
\pron{jeMTfli}
\gl{\BAavayx}
\bmng
 (tiLivaLike, ecacxrike koDuvAga) huSAru! nidhAnavAgi! sAvadhAnavAgi! aSuTx veVga beVDa! 
\emng
\eentry

\bentry
\word{gentoo}
\pron{jeMTU}
\gl{\nA}
\bmng
 oMdu bageya peMgivxnf hakikx. 
\emng
\eentry

\bentry
\word{gentry}
\pron{jeMTirx}
\gl{\nA}
\bmng
 (\bava) 
\bnum
\num{1} divxtiVya sherxVNiyavaru, vagaRdavaru; kuliVnaru; saBayxru; sAthxnadalilxyU huTiTxnalilxyU shirxVmaMta dajeRya taruvAyada keLagina vagaRdavaru. 
\num{2} (\tu) maMdi; jana: \eng{these gentry} I jana; I maMdi. \eng{they do a lot of damage, these gentry with their open diplomacy} bahiraMga rAjataMtarx naDesuva I maMdi beVkAdaSuTx hAni mADutAtxre. 
\enum
\emng
\eentry

\bentry
\word{genual}
\pron{jenUyxalf}
\gl{\gu}
\bmng
 moNakAlina. 
\emng
\eentry

\bentry
\word{genuflect}
\pron{jenUyx(nuyx)phelxkfTx}
\gl{\akirx}
\bmng
\bnum
\num{1} (\kanmu\ gwravAthaRvAgi yA pUje mADuvAga) moNakAlUru; maMDiyUru. 
\num{2} dAsayxBAva toVrisu, parxkaTisu. 
\enum
\emng
\eentry

\bentry
\word{genuflection}
\pron{jenUyx(nuyx)phelxkfSanf}
\gl{\nA}
\bmng
  = \hyperlink{genuflexion}{genuflexion}. 
\emng
\eentry

\bentry
\word{genuflector}
\pron{jenUyx(nuyx)phelxkaTxrf}
\gl{\nA}
\bmng
jAnu namasAkxri; nelakekx maMDiyUri upAsane mADuvavanu. 
\emng
\eentry

\bentry
\word{genuflectory}
\pron{jenUyx(nuyx)phelxkaTxri}
\gl{\gu}
\bmng
\bnum
\num{1} (\kanmu\ pUjeyalilx) maMDiyUruvikege, jAnuparxNAmakekx saMbaMdhisida. 
\num{2} (\shaveY) (balavaMtadiMda) maMDi maDisuvudakekx saMbaMdhisida. 
\enum
\emng
\eentry

\bentry
\word{genuflexion}
\pron{jenUyx(nuyx)phelxkfSanf}
\gl{\nA}
\bmng
\bnum
\num{1} jAnu parxNAma; maMDiyUri mADuva upAsane. 
\num{2} atidiVnate; namarxte; vidheVyate; deYnayx. 
\enum
\emng
\eentry

\bentry
\word{genuine}
\pron{jenuyxinf}
\gl{\gu}
\bmng
\bnum
\num{1} AditaLiya; shudadhx jAtiya; apapxTa, acacx -- taLiya: \eng{the genuine breed of mastiff} shudadhxtaLiya mAyxsiTxphf nAyigaLu. 
\num{2} nijavAda; yathAthaR; neYja; mUlasAthxnadiMda, neYja kataqRviniMda nijavAgi baMda: \eng{a genuine text} yathAthaR garxMthapATha. 
\num{3} nijavAda; niSakxpaTavAda; akaqtirxmavAda; sAca: \eng{questions that are asked are genuine questions} keVLida parxshenxgaLu sAca parxshenxgaLu. 
\num{4} nijavAda; pArxmANikavAda; Kare; satayxvAda; kaqtakavalalxda: \eng{a genuine signature} satayxvAda ruju, dasakxtutx. 
\num{5} hesarige takakxMtiruva; yathAnAma; yathAMkita; nijavAda; yathAthaRvAgiruva: \eng{a genuine conservative} nijavAda saMparxdAyavAdi. \eng{a genuine idealist} nijavAda AdashaRvAdi. 
\enum
\emng
\eentry

\bentry
\word{genuinely}
\pron{jenuyxinfli}
\gl{\kirxvi}
\bmng
 nijavAgi; satayxvAgi; pArxmANikavAgi; akaqtirxmavAgi: \eng{policies genuinely serving the national interests} nijavAgi deVshada hitAsakitxgaLanunx kApADuva niVtigaLu. 
\emng
\eentry

\bentry
\word{genuineness}
\pron{jenuyxinfnisf}
\gl{\nA}
\bmng
 nijasithxti; yathAthaRte; satayxte; pArxmANikate; sAcAtana: \eng{mistrusting the genuineness of the invitation} AhAvxnada satayxteyanunx shaMkisutAtx. 
\emng
\eentry

\bentry
\word{genus}
\pron{jiV(je)nasf}
\gl{\nA}
\expl{(\bava\ \eng{genera} \ucAcx\ jenara).}
\bmng
\bnum
\num{1} (takaR) (upapaMgaDagaLU, vaMshagaLU uLaLx) jAti. 
\num{2} (\jiVvi) kula; jiVvigaLa vagiRVkaraNadalilx vaMshakikxMta (\eng{family}) saMkucitavAda matutx jAti (\eng{species}) giMta visAtxravAda guMpu: \eng{the species of oak collectively form the genus quercus} Okf marada jAtigaLelalx oTuTxgUDi kevxkaRsf kulavAgutatxde. 
\num{3} (\saDi) vagaR; taragati; paMgaDa; buDakaTuTx; baNa; kula; kuTuMba; guMpu. 
\enum
\emng

\noindent
\gl{\pagu}
\bmng
\bnum
\num{1} \eng{highest genus} (matotxMdara upapaMgaDavalalxda) parxdhAna jAti. 
\num{2} \eng{subaltern genus} upajAti. 
\enum
\emng
\eentry

\bentry
\wordwithhyphen{hyp-geny}{-geny}
\pron{-jini}
\gl{\uparx}
\bmng
 utapxtitx, janana, yA vikAsa eMbathaRgaLanunx sUcisuva \nA gaLanunx racisuvAga baLasuva \uparx: \eng{anthropogeny} mAnavana ugamada adhayxyana, cariterx. 
\emng
\eentry

\bentry
\wordnospeech{Geo.}{Geo.}
\pron{?}
\gl{\saMkiSx}
\bmng
 \eng{George.} 
\emng
\eentry

\bentry
\word{geo-}
\pron{jiVO-}
\gl{\sapUpa}
\bmng
 BU, BUmiya, BUmige saMbaMdhisida eMbathaRgaLalilx baLasuva \sapUpa: \eng{geo-selenic} BUcaMdarxra. 
\emng
\eentry

\bentry
\word{geobotony}
\pron{jiVObATani}
\gl{\nA}
\bmng
 BwgoVLika sasayxvijAcnxna; sasayxgaLa BwgoVLika haMcikeya adhayxyana; BUparxdeVshagaLalilx sasayxgaLu haMcikoMDiruva riVtiyanunx kurita adhayxyana. 
\emng
\eentry

\bentry
\word{geocentric}
\pron{jiVOseMTirxkf}
\gl{\gu}
\bmng
 BUkeVMdirxVya: 
\banum
\alnum{a} BUmiya keVMdarxdiMda noVDutitxruvaMte BAvisuva. 
\alnum{b} BUmiyeV vishavxda keVMdarxveMdu parigaNisuva. 
\eanum
\emng
\eentry

\bentry
\wordnospeech{geocentric latitude}{geocentric latitude}
\pron{?}
\gl{\nA}
\bmng
 (garxhada) BUkeVMdirxVya akASxMsha; BUmiya keVMdarxdalilxruva viVkaSxkanige garxhavu yAva akASxMshadalilx kANisuvudoV A akASxMsha. 
\emng
\eentry

\bentry
\word{geochemical}
\pron{jiVOkemikalf}
\gl{\gu}
\bmng
 BUrAsAyanika; BUrasAyana vijAcnxnada; BUrasAyana vijAcnxnakekx saMbaMdhisida. 
\emng
\eentry

\bentry
\word{geochemist}
\pron{jiVOkemisfTx}
\gl{\nA}
\bmng
 BUrasAyana vijAcnxni; BUmiya cipipxna rAsAyanika saMyoVjane matutx alilx naDeyuva rAsAyanika badalAvaNegaLa adhayxyana mADuvava yA A keSxVtarxdalilx tajacnx. 
\emng
\eentry

\bentry
\word{geochemistry}
\pron{jiVOkemisiTxrX}
\gl{\nA}
\bmng
 BUrasAyana vijAcnxna; BUmiya cipipxna rAsAyanika saMyoVjane matutx alilx Aguva rAsAyanika badalAvaNegaLa adhayxyana. 
\emng
\eentry

\bentry
\word{geochronological}
\pron{jiVOkArxnalAjikalf}
\gl{\gu}
\bmng
 BUkAlagaNaneya yA adakekx saMbaMdhisida. 
\emng
\eentry

\bentry
\word{geochronologist}
\pron{jiVOkarxnAlajisfTx}
\gl{\nA}
\bmng
 BUkAlagaNaka; BUkAlagaNana tajacnx; BUveYjAcnxnika mAhitiya AdhArada meVle gatakAlada GaTanegaLa kAlavanunx nidhaRrisuvava(Lu). 
\emng
\eentry

\bentry
\word{geochronology}
\pron{jiVOkarxnAlaji}
\gl{\nA}
\bmng
 BUkAlagaNane: 
\banum
\alnum{a} BUveYjAcnxnika mAhitiya neraviniMda gatakAlada GaTanegaLa, avasheVSagaLa kAlagaLanunx nidhaRrisuva shAsatxrX, viBAga. 
\alnum{b} hAge nidhaRrisida kAlika anukarxma. 
\eanum
\emng
\eentry

\bentry
\word{geochrony}
\pron{jiVAkarxni}
\gl{\nA}
\bmng
 BUveYjAcnxnika kAlAnukarxmaNike; BUvijAcnxnadalilx baLasuva kAlaviBajaneya vayxvasethx. 
\emng
\eentry

\bentry
\word{geocyclic}
\pron{jiVOseYkilxkf}
\gl{\gu}
\bmng
\bnum
\num{1} BUpariBArxmaka; BUmiyanunx sututxhAkuva. 
\num{2} BUpariBarxmaNada; BUmiyu sututxhAkuva yA adakekx saMbaMdhisida. 
\enum
\emng
\eentry

\bentry
\word{geode}
\pron{jiVODf}
\gl{\nA}
\bmng
 (\BUvi) jiyoVDu: 
\banum
\alnum{a} haraLugaLu yA Kanija padAthaRgaLa asatxriyuLaLx poTare iruva shile. 
\alnum{b} aMtaha poTare. 
\eanum
\emng
\eentry

\bentry
\word[geodesic(1)]{geodesic}
\pron{jiVODiV(De)siZkf}
\gl{\nA}
\bmng
  = \hyperlink{geodesic line}{geodesic line}. 
\emng
\eentry

\bentry
\word[geodesic(2)]{geodesic}
\pron{jiVODiV(De)siZkf}
\gl{\gu}
\bmng
\bnum
\num{1} BUmitiVya; BUmitiya yA adakekx saMbaMdhisida. 
\num{2} vakarxtalajAyxmitiya; samatalajAyxmitiyalilxna saraLareVKegaLa sAthxnavanunx jiyoVDesikf reVKegaLu tegedukoLuLxva vakarxtalagaLa jAyxmiti. 
\enum
\emng
\eentry

\bentry
\wordnospeech{geodesic dome}{geodesic dome}
\pron{?}
\gl{\nA}
\bmng
 jiyoVDesikf gumamxTa; haguravAda matutx neVravAda racanAvasutxgaLiMda nimiRsalAda gumamxTa. \imglink{geodesic-domefigure}{\raisebox{-0.15cm}[0pt][0pt]{\pdfimage width 0.8cm height 0.5cm{G_Pictures/geodesic-dome.jpg}}} 
\emng
\eentry

\bentry
\wordnospeech{geodesic line}{geodesic line}
\pron{?}
\gl{\nA}
\bmng
 jiyoVDesikf reVKe; vakarxtalada meVle eraDu biMdugaLa naDuvaNa atayxMta moTakAda reVKe. 
\emng
\eentry

\bentry
\word{geodesist}
\pron{jiVADisisfTx}
\gl{\nA}
\bmng
 BUmitishAsatxrXjacnx. 
\emng
\eentry

\bentry
\word{geodesy}
\pron{jiVADisi}
\gl{\nA}
\bmng
\bnum
\num{1} BUmiti; BUgaNita; BUmiya yA adara vishAla BAgagaLa Akaqti matutx visitxVNaRgaLanunx kurita gaNitashAsatxrXda shAKe. 
\num{2} BUmitiVya moVjaNi; BUtalada vakarxteyanunx lekakxkekx tegedukoLuLxva moVjaNi. 
\enum
\emng
\eentry

\bentry
\word{geodetic}
\pron{jiVaDiV(De)Tikf}
\gl{\gu}
\bmng
  = \hyperlink{geodesic(2)}{$^2$geodesic}. 
\emng
\eentry

\bentry
\word{geodetically}
\pron{jiVaDiTikali}
\gl{\kirxvi}
\bmng
 BUmitiVyavAgi; BUmitiya riVtAyx. 
\emng
\eentry

\bentry
\wordnospeech{geodetic dome}{geodetic dome}
\pron{?}
\gl{\nA}
\bmng
  = \hyperlink{geodesic dome}{geodesic dome}. 
\emng
\eentry

\bentry
\wordnospeech{geodetic line}{geodetic line}
\pron{?}
\gl{\nA}
\bmng
 = \hyperlink{geodesic line}{geodesic line}. 
\emng
\eentry

\bentry
\word{geodic}
\pron{jiVaDikf}
\gl{\gu}
\bmng
\bnum
\num{1} jiyoVDina yA adakekx saMbaMdhisida. 
\num{2} jiyoVDinaMtha. 
\enum
\emng
\eentry

\bentry
\word{geognostic}
\pron{jiVagAnxsiTxkf}
\gl{\gu}
\bmng
 BUveYjAcnxnika. 
\emng
\eentry

\bentry
\word{geognostical}
\pron{jiVagAnxsiTxkalf}
\gl{\gu}
\bmng
  = \hyperlink{geognostic}{geognostic}. 
\emng
\eentry

\bentry
\word{geognosy}
\pron{jiVAganxsi}
\gl{\nA}
\bmng
\bnum
\num{1}  = \hyperlink{geology}{geology}. 
\num{2} yAvudeV pArxMtada BUveYjAcnxnika vaqtAtxMta. 
\num{3} yAvudeV shilegaLa Kanija savxrUpa yA avugaLa haMcike. 
\enum
\emng
\eentry

\bentry
\word{geogony}
\pron{jiVagani}
\gl{\nA}
\bmng
 BUjananashAsatxrX; BUmiya ugamavanunx kurita shAsatxrX yA sidAdhxMta. 
\emng
\eentry

\bentry
\word{geographer}
\pron{jiAgarxpharf}
\gl{\nA}
\bmng
 BUgoVLashAsatxrXjacnx. 
\emng
\eentry

\bentry
\word{geographic}
\pron{jiagArxYxphikf}
\gl{\gu}
\bmng
 BwgoVLika; BUgoVLashAsatxrXda yA adakekx saMbaMdhisida. 
\emng
\eentry

\bentry
\word{geographical}
\pron{jiVagArxYxphikalf}
\gl{\gu}
\bmng
  = \hyperlink{geographic}{geographic}. 
\emng
\eentry

\bentry
\wordnospeech{geographical latitude}{geographical latitude}
\pron{?}
\gl{\nA}
\bmng
 BwgoVLika akASxMsha; BUtalada meVlina yAvudeV biMduvinalilx eLeda laMbareVKeyu viSuvadavxqqtatxda samataladoDane mADuva koVna. 
\emng
\eentry

\bentry
\word{geographically}
\pron{jiVagArxYxphikali}
\gl{\kirxvi}
\bmng
 BwgoVLikavAgi. 
\emng
\eentry

\bentry
\wordnospeech{geographical mile}{geographical mile}
\pron{?}
\gl{\nA}
\bmng
 BwgoVLika meYli; viSudavxqqtatxda meVle reVKAMshada oMdu miniTuTx (sumAru \eng{1850} mITarugaLu). 
\emng
\eentry

\bentry
\wordnospeech{geographic latitude}{geographic latitude}
\pron{?}
\gl{\nA}
\bmng
  = \hyperlink{geographical latitude}{geographical latitude}. 
\emng
\eentry

\bentry
\word{geography}
\pron{jiAgarxphi}
\gl{\nA}
\bmng
\bnum
\num{1} BUgoVLashAsatxrX; BUvivaraNe; BUmiya meVlemxY, rUpa, pArxkaqtika lakaSxNagaLu, neYsagiRka matutx rAjakiVya viBAgagaLu, havAguNa, utapxnanxgaLu, janasaMKeyx, \mo vanunx kurita shAsatxrX. 
\num{2} BwgoVLika viSaya, vasutx; BUgoVLashAsatxrXda viSaya, vasutx. 
\num{3} pArxdeVshika BUgoVLa; BUvivaraNe; oMdu parxdeVshada BUlakaSxNagaLu, BUvinAyxsa. 
\num{4} BUgoVLashAsatxrX (garxMtha). 
\num{5} (\AmA) manevivara; maneyalilx kakakxsu yA itara koVNegaLu iruva sathxLa, sAthxna. 
\enum
\emng

\noindent
\gl{\pagu}
\bmng
\bnum
\num{1} \eng{mathematical geography} gaNita BUgoVLashAsatxrX; gaNitada neraviniMda adhayxyana mADatakakx BUgoVLashAsatxrXda viSayagaLu. 
\num{2} \eng{physical geography} pArxkaqtika BUgoVLashAsatxrX; BUmiya meVlemxYya pArxkaqtika lakaSxNagaLige siVmitavAda BUgoVLashAsatxrX. 
\num{3} \eng{political geography} rAjakiVya BUgoVLashAsatxrX; rAjakiVyakekx saMbaMdhisida BUgoVLashAsatxrXda viSayagaLu. 
\enum
\emng
\eentry

\bentry
\word{geoid}
\pron{jiVAyfDx}
\gl{\nA}
\bmng
 jiVyAyfDx: 
\banum
\alnum{a} sAgaragaLa sarAsari samudarxmaTaTxkekx hoMdikoMDidudx, gurutavxda dikikxge elelxDeyalilxyU laMbavAgiruva BUtala. 
\alnum{b} BUrUpa; BUmiya rUpa, Akaqti. 
\eanum
\emng
\eentry

\bentry
\word{geologic}
\pron{jiValAjikf}
\gl{\gu}
\bmng
 BUveYjAcnxnika; BUvijAcnxnada, adakekx saMbaMdhisida yA adara parxkArada. 
\emng
\eentry

\bentry
\word{geological}
\pron{jiValAjikalf}
\gl{\gu}
\bmng
 = \hyperlink{geologic}{geologic}. 
\emng
\eentry

\bentry
\word{geologically}
\pron{jiValAjikali}
\gl{\kirxvi}
\bmng
 BUveYjAcnxnikavAgi. 
\emng
\eentry

\bentry
\word{geologise}
\pron{jiAlajeYsfZ}
\gl{\kirx}
\bmng
  = \hyperlink{geologize}{geologize}. 
\emng
\eentry

\bentry
\word{geologist}
\pron{jiAlajisfTx}
\gl{\nA}
\bmng
 BUvijAcnxni; BUvijAcnxnada adhayxyana mADuvavanu. 
\emng
\eentry

\bentry
\word{geologize}
\pron{jiAlajeYsfZ}
\gl{\sakirx}
\bmng
 (yAvudanenxV) BUveYjAcnxnikavAgi adhayxyana mADu. 
\emng

\noindent
\gl{\akirx}
\bmng
 BUveYjAcnxnika adhayxyana, saMshoVdhane, parishiVlane -- naDesu. 
\emng
\eentry

\bentry
\word{geology}
\pron{jiAlaji}
\gl{\nA}
\bmng
\bnum
\num{1} BUvijAcnxna; BUmiya cipupx, adara satxragaLu, avugaLigiruva parasapxra saMbaMdha, avugaLalAlxguva badalAvaNegaLu, \mo vanunx kurita vijAcnxna viBAga. 
\num{2} (yAvudeV parxdeVshada) BUveYjAcnxnika lakaSxNagaLu, vivaragaLu. 
\num{3} (caMdarx \mo vugaLa) BUveYjAcnxnika adhayxyana. 
\enum
\emng
\eentry

\bentry
\word{geomagnetic}
\pron{jiVOmAYxgenxTikf}
\gl{\gu}
\bmng
\bnum
\numi{1} BUkAMtiVya: 
\banum
\alnum{a} BUmiya kAMtateyanunx kurita, adariMda udaBxvisida yA adakekx saMbaMdhisida. 
\alnum{b} BUkAMtakeSxVtarxda yA adakekx saMbaMdhisida. 
\eanum
\numie
\enum
\emng
\eentry

\bentry
\word{geomagnetism}
\pron{jiVOmAYxginxTisaZmf}
\gl{\nA}
\bmng
 BUkAMtiVyate; BUmiya kAMtiVya guNagaLa adhayxyana. 
\emng
\eentry

\bentry
\word{geomancer}
\pron{jiVOmAYxnasxrf}
\gl{\nA}
\bmng
\bnum
\num{1} BUkaNigAra, BaviSayxvAdi. 
\num{2} Akaqti -- kaNigAra, BaviSayxvAdi. 
\enum
\emng
\eentry

\bentry
\word{geomancy}
\pron{jiVOmAYxnisx}
\gl{\nA}
\bmng
\bnum
\num{1} BUkaNi; BUBaviSayx; oMdu hiDi maNaNxnunx nelada meVle eracuvudariMdAda AkaqtiyiMda BaviSayx nuDiyuvudu. 
\num{2} Akaqti kaNi, BaviSayx; kAgadada meVle manabaMdaMte hAkida cikekxgaLiMdAda reVKe yA AkAradiMda BaviSayx heVLuvudu. 
\enum
\emng
\eentry

\bentry
\word{geomantic}
\pron{jiVOmAYxMTikf}
\gl{\gu}
\bmng
\bnum
\num{1} BU BaviSayxda. 
\num{2} Akaqti BaviSayxda. 
\enum
\emng
\eentry

\bentry
\word{geometer}
\pron{jiAmiTarf}
\gl{\nA}
\bmng
\hypertarget{geometer(1)}{} 
\bnum
\num{1} jAYxmitijacnx; reVKAgaNitajacnx; jAyxmiti yA reVKAgaNitavanunx adhayxyana mADidava yA adaralilx nipuNate paDedava. 
\num{2} jAyxmitiga; jAyxmiTirxDeV vaMshada pataMga yA adara kaMbaLihuLu. 
\enum
\emng
\eentry

\bentry
\word{geometric}
\pron{jiameTirxkf}
\gl{\gu}
\bmng
 jAyxmitiVya; jAyxmitiya, adakekx saMbaMdhisida yA adara parxkArada. 
\emng
\eentry

\bentry
\word{geometrical}
\pron{jiameTirxkalf}
\gl{\gu}
\bmng
  = \hyperlink{geometric}{geometric}. 
\emng
\eentry

\bentry
\wordnospeech{geometrical architecture}{geometrical architecture}
\pron{?}
\gl{\nA}
\bmng
 jAyxmitiVya vAsutxshilapx; vatuRlagaLu, mUrelegaLu, \mo\ jAyxmitiVya AkAragaLuLaLx vAsutxsheYli. 
\emng
\eentry

\bentry
\word{geometrically}
\pron{jiameTirxkali}
\gl{\kirxvi}
\bmng
 jAyxmitiVyavAgi. 
\emng
\eentry

\bentry
\wordnospeech{geometrical mean}{geometrical mean}
\pron{?}
\gl{\nA}
\bmng
 jAyxmitiVya madhayxma; (eraDu saMKeyxgaLa \vi) guNalabadhxda vagaRmUla. 
\emng
\eentry

\bentry
\wordnospeech{geometrical optics}{geometrical optics}
\pron{?}
\gl{\nA}
\bmng
 jAyxmitiVya daqgivxjAcnxna; saraLavAda anuBavasidadhx niyamagaLa AdhArada meVle parxtiPalana matutx vakirxVBavana vidayxmAnagaLanunx gaNitariVtAyx nigamana mADuva daqgivxjAcnxna viBAga. 
\emng
\eentry

\bentry
\wordnospeech{geometrical progression}{geometrical progression}
\pron{?}
\gl{\nA}
\bmng
 jAyxmitiVya sherxVNi; oMdu motatxkUkx adara muMdina motatxkUkx sithxra dAmASA iruvaMtha sherxVNi: \eng{1, 3, 9, 27, 81, 243, 729......} 
\emng
\eentry

\bentry
\wordnospeech{geometrical proportion}{geometrical proportion}
\pron{?}
\gl{\nA}
\bmng
 jAyxmitiVya anupAta; eraDu BAgagaLalilxyU oMdeV dAmASA iruva anupAta -- \eng{1:3::4:12.} 
\emng
\eentry

\bentry
\word{geometrician}
\pron{jiVamiTirxSanf}
\gl{\nA}
\bmng
  = \hyperlink{geometer(1)}{geometer (1)}. 
\emng
\eentry

\bentry
\wordnospeech{geometric isomerism}{geometric isomerism}
\pron{?}
\gl{\nA}
\bmng
 (\ravi) jAyxmitiVya samAMgate; aNuvina anibaRMdhita BarxmaNa asAdhayxvAgiruvudara PalavAgi paramANugaLu yA paramANupuMjagaLu nidiRSaTx dikukxgaLige tiruguvudariMda udaBxvisuva sisfTArxnfsx samAMgate (\eng{cis-trans isomerism}). 
\emng
\eentry

\bentry
\wordnospeech{geometric mean}{geometric mean}
\pron{?}
\gl{\nA}
\bmng
  = \hyperlink{geometrical mean}{geometrical mean}. 
\emng
\eentry

\bentry
\wordnospeech{Geometric pottery}{Geometric pottery}
\pron{?}
\gl{\nA}
\bmng
 jAyxmitiVya kuMbArike; alaMkaraNakekx jAyxmitiVya AkAragaLanunx baLasutitxdadx, pArxciVna girxVsina kuMbAra kale. 
\emng
\eentry

\bentry
\wordnospeech{geometric progression}{geometric progression}
\pron{?}
\gl{\nA}
\bmng
  = \hyperlink{geometrical progression}{geometrical progression}. 
\emng
\eentry

\bentry
\wordnospeech{geometric proportion}{geometric proportion}
\pron{?}
\gl{\nA}
\bmng
  = \hyperlink{geometrical proportion}{geometrical proportion}. 
\emng
\eentry

\bentry
\wordnospeech{geometric spider}{geometric spider}
\pron{?}
\gl{\nA}
\bmng
 jAyxmitiVya jeVDa; jAyxmitiVya Akaqtiyalilx bale nimiRsuva jeVDa. 
\emng
\eentry

\bentry
\wordnospeech{geometric tracery}{geometric tracery}
\pron{?}
\gl{\nA}
\bmng
 jAyxmitiVya reVKana; vaqtatx, tirxdaLa, \mo\ jAyxmitiVya Akaqtiya terapugaLiruva citarxreVKana. 
\emng
\eentry

\bentry
\word{geometrise}
\pron{jiAmiTerxYsfZ}
\gl{\sakirx}
\bmng
  = \hyperlink{geometrize}{geometrize}. 
\emng
\eentry

\bentry
\word{geometrize}
\pron{jiAmiTerxYsfZ}
\gl{\sakirx}
\bmng
 jAyxmitiVyagoLisu: 
\banum
\alnum{a} (yAvudanenxV) jAyxmitiVyavAgi nirUpisu. 
\alnum{b} (yAvudakekxV) jAyxmitiVya tatatxvXgaLanunx matutx niyamagaLanunx anavxyisu. 
\eanum
\emng

\noindent
\gl{\akirx}
\bmng
 jAyxmitiVyavAgi adhayxyana mADu; jAyxmitiVya vidhAnadiMda parishiVlane naDesu. 
\emng
\eentry

\bentry
\word{geometry}
\pron{jiAmiTirx}
\gl{\nA}
\bmng
 jAyxmiti: 
\banum
\alnum{a} biMdugaLu, reVKegaLu, talagaLu, koVnagaLu matutx GanagaLa aLate, avugaLa guNalakaSxNagaLu matutx parasapxra saMbaMdhagaLanunx kurita shAsatxrX. 
\alnum{b} vasutxgaLa yA BAgagaLa sApeVkaSx vinAyxsa, joVDane. 
\eanum
\emng
\eentry

\bentry
\word{geomorphological}
\pron{jiVOmAphaRlAjikalf}
\gl{\gu}
\bmng
 BUrUpashAsatxrXda yA adakekx saMbaMdhisida. 
\emng
\eentry

\bentry
\word{geomorphologist}
\pron{jiVOmAphARlajisfTx}
\gl{\nA}
\bmng
 BUrUpashAsatxrXjacnx; BUrUpashAsatxrXdalilx pariNata yA BUrUpashAsatxrXda vidAyxthiR. 
\emng
\eentry

\bentry
\word{geomorphology}
\pron{jiVOmAphARlaji}
\gl{\nA}
\bmng
 BUrUpashAsatxrX; BUmiya cipipxna BwtalakaSxNagaLu matutx adara BUveYjAcnxnika racaneyanunx kurita shAsatxrX. 
\emng
\eentry

\bentry
\word{geophagist}
\pron{jiAphajisfTx}
\gl{\nA}
\bmng
 maqdaBxkaSxka; maNuNx tinunxga; maNuNx tinunxvavanu. 
\emng
\eentry

\bentry
\word{geophagy}
\pron{jiAphaji}
\gl{\nA}
\bmng
 maqdaBxkaSxNa; maNuNx tinunxvudu; jeVDi, siVmesuNaNx, \mo vanunx tinunxva aBAyxsa. 
\emng
\eentry

\bentry
\word{geophone}
\pron{jiVOphoVnf}
\gl{\nA}
\bmng
 BUkaMpanamApaka; BUdhavxnimApaka; BUmiyalilxna dhavxnitaraMga yA AGAtataraMgagaLaMtaha kaMpanagaLanunx patetx hacucxva sAdhana. 
\emng
\eentry

\bentry
\word{geophysical}
\pron{jiVOphisikalf}
\gl{\gu}
\bmng
BUBwta; BUBwtavijAcnxnada, adakekx saMbaMdhisida yA adara parxkAra kaMDubaruva. 
\emng
\eentry

\bentry
\word{geophysicist}
\pron{jiVOphisisisfTx}
\gl{\nA}
\bmng
 BUBwtavijAcnxni; BUBwtavijAcnxnavanunx adhayxyana mADuvava. 
\emng
\eentry

\bentry
\word{geophysics}
\pron{jiVOphisikfsx}
\gl{\nA}
\bmng
 BUBwtavijAcnxna; BwtavijAcnxnada daqSiTxyiMda naDesuva BUmiya adhayxyana. 
\emng
\eentry

\bentry
\word{geopolitical}
\pron{jiVOpaliTikalf}
\gl{\gu}
\bmng
 BUrAjayxshasatxrXda yA adakekx saMbaMdhisida. 
\emng
\eentry

\bentry
\word{geopolitics}
\pron{jiVOpAliTikfsx}
\gl{\nA}
\bmng
 BUrAjayxshAsatxrX; oMdu deVshada BwgoVLika lakaSxNagaLiMda nidhARravAguva rAjayxshAsatxrX yA rAjakiVya saMgatigaLa adhayxyana. 
\emng
\eentry

\bentry
\word{geoponic}
\pron{jiVOpAnikf}
\gl{\gu}
\bmng
 (ADaMbarada \parx\ yA \hA) vayxvasAyada; beVsAyada; kaqSiya; okakxlutanada; sheVtakiya. 
\emng
\eentry

\bentry
\word{Geordie}
\pron{jADiR}
\gl{\nA}
\bmng
 (\birx) 
\bnum
\num{1} (\AmA) (iMgelxMDina nAdaRMbalaRMDina) TeYnfseYDf parxdeVshadavanu. 
\num{2} ipapxtotxMdu SiliMgugaLu. 
\num{3} (utatxra iMgelxMDina) TeYnf nadiya tiVrada oMdu kalilxdadxlu gaNi. 
\num{4} haDagu. 
\num{5} (sAkxTalxMDina \parx) (kalilxdadxlu gaNiyalilx baLasuva) surakiSxta diVpa. 
\enum
\emng
\eentry

\bentry
\word{George}
\pron{jAjfR}
\gl{\nA}
\bmng
\bnum
\num{1} saMta jAjfR; mUraneya eDavxDfR doreya kAladiMda iMgelxMDina rakaSxkaneMdu BAvisiruva saMta. 
\num{2} `gATaRrf' eMba birudu lAMCanadalilxruva ABaraNa. 
\num{3} (\birx) (\ashi) vimAnadalilxya savxyaMcAlita cAlaka. 
\enum
\emng

\noindent
\gl{\pagu}
\bmng
\bnum
\num{1} \eng{Brown George} kaMdumaNiNxna pAterx. 
\num{2} \eng{by george} (ANe iTuTxkoLuLxvAga, yA AshacxyaRvanonxV samamxtavanonxV sUcisuva udAgxra) saMta jAjfRna ANe! 
\num{3} \eng{George Cross, Medal} (\saMkiSx\ \eng{GC., GM.}) vishiSaTx sAhasa seVvegAgi (\kanmu\ aseYnikarige yA nAgarikarige) koDuva, \eng{1940}ralilx \eng{6}neV jAjfR doreyu iMgelxMDinalilx sAthxpisida padakagaLu. 
\num{4} \eng{St. George's Cross} saMta jAjaRna shilube; paTeTxgaLu keVMdarxdalilx aDaDxhAyuva keMpu baNaNxda shilube. 
\num{5} \eng{St. George's Day} saMta jAjaRna dina; Epirxlf \eng{23}neya tAriVKu. 
\enum
\emng
\eentry

\bentry
\word{georgette}
\pron{jAjeRTf}
\gl{\nA}
\bmng
 jAjeRTuTx; nayavAda teLureVSemxya yA jAlariyaMtha uDupu baTeTx. 
\emng
\eentry

\bentry
\word[Georgian(1)]{Georgian}
\pron{jAjiRanf, jAjayxRnf}
\gl{\gu}
\bmng
 jAjiRVya; jAjiRyanf: 
\banum
\alnum{a} iMgelxMDina modala nAlukx jAjfR doregaLa kAlada \eng{(1714--1830)}. 
\alnum{b} aidaneya hAgU Araneya jAjaRra kAlada \eng{(1910--52)} (\kanmu\ \eng{1910--20}ra)varegina sAhitayxda. 
\eanum
\emng
\eentry

\bentry
\word[Georgian(2)]{Georgian}
\pron{jAjiRanf, jAjayxRnf}
\gl{\gu}
\bmng
 jAjiRyanf: 
\banum
\alnum{a} (raSAyxda kAkasasisxna) jAjiRya pArxMtada. 
\alnum{b} (\ame da) jAjiRya saMsAthxnada. 
\eanum
\emng
\eentry

\bentry
\word[Georgian(3)]{Georgian}
\pron{jAjiRanf, jAjayxRnf}
\gl{\nA}
\bmng
jAjiRyanf: 
\banum
\alnum{a} raSAyxda kAkasasisxna jAjiRya pArxMtada BASe. 
\alnum{b} (raSAyxda) jAjiRyadavanu; jAjiRya pArxMtadavanu. 
\alnum{c} (\ame) jAjiRya saMsAthxnadavanu. 
\eanum
\emng
\eentry

\bentry
\word{geosphere}
\pron{jiVasiphxarf}
\gl{\nA}
\bmng
\bnum
\num{1} BUKaMDa; vAyumaMDala matutx jalaBAgagaLanunx biTuTx BUmiya Gana BAga. 
\num{2} BUvalaya; BUmi matutx adara vAyumaMDalada samAna keVMdarxkekx sututxpaTiTxyaMtiruva yAvudeV parxdeVsha. 
\enum
\emng
\eentry

\bentry
\word{geostationary}
\pron{jiVOseTxVSanari}
\gl{\gu}
\bmng
 BUsAthxyiV; (BUmiya kaqtaka upagarxhada \vi) BUmiyanunx sututxhAkutitxdadxrU BUmiya meVlina nidiRSaTx biMduvoMdara netitxya meVle sadA iruvaMte kANisuva. 
\emng
\eentry

\bentry
\word{geostrophic}
\pron{jiVOsATxrXphikf}
\gl{\gu}
\bmng
 (\pashA) BUBarxmaNaka; BUBarxmaNavanunx avalaMbisiruva yA adakekx saMbaMdhisida. 
\emng
\eentry

\bentry
\word{geosynchronous}
\pron{jiVOsiMkarxnasf}
\gl{\gu}
\bmng
 BU samakarxmika; BUmeVLayaka; BUsAthxyiV: 
\banum
\alnum{a} (kaqtaka upagarxhada \vi) BUmiyu tananx akaSxda meVle oMdu sututx tiruguvaSuTx kAladalilx, sariyAgi BUmiyanunx oMdu sututx hAkuva. 
\alnum{b} (kaqtaka upagarxhagaLa kakeSxya \vi) BUsAthxyiV upagarxhada. 
\eanum
\emng
\eentry

\bentry
\word{geothermal}
\pron{jiVOtamaRlf}
\gl{\gu}
\bmng
 BUshAKada; BUmiya oLagaNa shAKakekx saMbaMdhisida. 
\emng
\eentry

\bentry
\word{geotropic}
\pron{jiVOTArxpikf}
\gl{\gu}
\bmng
 jAyxvataRka: 
\banum
\alnum{a} BUmiya gurutAvxkaSaRNege parxtikirxyeyAgi nidiRSaTx dikikxge tiruguva. 
\alnum{b} BUkeVMdarxkekx aBimuKavAgi beLeyuva. 
\eanum
\emng
\eentry

\bentry
\word{geotropically}
\pron{jiVOTArxpikali}
\gl{\kirxvi}
\bmng
 jAyxvataRkavAgi: 
\banum
\alnum{a} BUmiya gurutAvxkaSaRNege parxtikirxyeyAgi nidiRSaTx dikikxge tiruguvaMte. 
\alnum{b} BUkeVMdarxkekx aBimuKavAgi. 
\eanum
\emng
\eentry

\bentry
\word{geotropism}
\pron{jiVOTarxpisaZmf}
\gl{\nA}
\bmng
jAyxvataRne: 
\banum
\alnum{a} (\sA\ sasayxgaLa \vi, matutx kelavu veVLe pArxNigaLa \vi) BUmiya gurutAvxkaSaRNege parxtikirxyeyAgi nidiRSaTx dikikxge tiruguvike. 
\alnum{b} BUkeVMdarxkekx aBimuKavAgi beLeyuvudu yA calisuvudu. 
\eanum
\emng

\noindent
\gl{\pagu}
\bmng
\hyperdef{G}{geotropism pagu}{} \eng{negative geotropism} jAyxpavataRne; giDagaLa kAMDagaLu \mo vu BUkeVMdarxkekx vimuKavAgi beLeyuvudu yA calisuvudu. 
\bnum
\num{2} \eng{positive geotropism} jAyxBivataRne; giDagaLa beVrugaLu \mo vu BUkeVMdarxkekx aBimuKavAgi beLeyuvudu yA calisuvudu. 
\enum
\emng
\eentry

\bentry
\wordnospeech{Ger.}{Ger.}
\pron{?}
\gl{\saMkiSx}
\bmng
 \eng{German.} 
\emng
\eentry

\bentry
\word{geranium}
\pron{jareVniamf}
\gl{\nA}
\bmng
 jareVniyaM: 
\banum
\alnum{a} kokakxreya kokikxnaMtha haNuNx biDuva kADu sasayxgaLa oMdu kula. 
\alnum{b} jereVniyaMge hatitxra saMbaMdhada pelagoRVniyamf kulada yAvudeV sasayx. 
\alnum{c} jareVniyaM hUgaLiMda tayArisida hoLapina keMpu baNaNx. 
\alnum{d} (janapirxyavAgi) beLesida pelagoRVniyamf. 
\eanum
\emng
\eentry

\bentry
\word{gerbera}
\pron{ga(ja)baRra}
\gl{\nA}
\bmng
 jabaRra; Aphirxkada yA ESayxda oMdu mUlike(yA kula). 
\emng
\eentry

\bentry
\word{gerbil}
\pron{jabiRlf}
\gl{\nA}
\bmng
 jabiRlf; udadxvAda hiMgAlugaLuLaLx, iliyaMtha, maraLugADina oMdu daMshaka. 
\emng
\eentry

\bentry
\word{gerenuk}
\pron{jeranUkf}
\gl{\nA}
\bmng
 udadxkatitxna pUvaR Aphirxkada eraLe. 
\emng
\eentry

\bentry
\word{gerfalcon}
\pron{jarfphAlakxnf}
\gl{\nA}
\bmng
 \eng{gyrfalcon} padada rUpAMtara. 
\emng
\eentry

\bentry
\word{geriatric}
\pron{jeriAYxTirxkf}
\gl{\gu}
\bmng
 mupupxshAsatxrXda yA adakekx saMbaMdhisida; jarAshAsatxrXda yA A shAsatxrXkekx saMbaMdhisida; vaqdAdhxpayxshAsatxrXda yA A shAsatxrXkekx saMbaMdhisida. 
\emng
\eentry

\bentry
\word{geriatrician}
\pron{jeriAYxTirxSanf}
\gl{\nA}
\bmng
 mupupxshAsatxrXjacnx; jarAshAsatxrXjacnx; vaqdAdhxpayxshAsatxrXjacnx; jarAshAsatxrXdalilx tajacnx. 
\emng
\eentry

\bentry
\word{geriatrics}
\pron{jeriAYxTirxkfsx}
\gl{\nA}
\bmng
 (\bava) mupupx veYdayxshAsatxrX; jarAroVgayxshAsatxrX; vaqdAdhxroVgayxshAsatxrX; vaqdadhxyoVgakeSxVma; mupupx matutx mupipxna kAyilegaLanUnx mudukara AroVgayx matutx yoVgakeSxVmavanUnx kurita veYdayxvijAcnxnada yA samAjashasatxrXda viBAga. 
\emng
\eentry

\bentry
\word{geriatrist}
\pron{jeriAYxTirxsfTx}
\gl{\nA}
\bmng
  = \hyperlink{geriatrician}{geriatrician}. 
\emng
\eentry

\bentry
\word{geriatry}
\pron{jeriAYxTirx}
\gl{\nA}
\bmng
  = \hyperlink{geriatrics}{geriatrics}. 
\emng
\eentry

\bentry
\word[germ(1)]{germ}
\pron{jamfR}
\gl{\nA}
\bmng
\bnum
\num{1} jiVvAMkura; hosa jiVvivayxkitxyAgi rUpugoLaLxbalalx jiVviya BAga (biVja, mogugx, koMbe, \mo vu). 
\num{2} aMkura; moLake; age; jiVviya mumomxdala rUpa. 
\num{3} sUkaSxmXjiVvi; sUkaSxmXjiVvANu, \kanmu\ roVgANu. 
\num{4} (\rUpa) mUla; mUla kAraNa; yAvudeV oMdu huTiTxge kAraNavAguvaMthadu. 
\enum
\emng

\noindent
\gl{\pagu}
\bmng
 \eng{in germ} inUnx -- araLilalxda, vikasisada; inUnx moLakeyalilxruva. 
\emng
\eentry

\bentry
\word[germ(2)]{germ}
\pron{jamfR}
\gl{\akirx}
\bmng
 (\rUpa) moLakeyAgu; aMkurisu; moLe; kuDiyiDu. 
\emng
\eentry

\bentry
\word{german}
\pron{jamaRnf}
\gl{\gu}
\bmng
\bnum
\num{1} savxMta; oMdeV tAyitaMdegaLige huTiTxda (mala aNaNx \mo vu alalxda) (\eng{brother german, sister german, cousin german} \mo vu). 
\num{2} (\pArxparx)  = \hyperlink{germane}{germane}. 
\enum
\emng
\eentry

\bentry
\word[German(1)]{German}
\pron{jamaRnf}
\gl{\gu}
\bmng
 jamaRnf: 
\banum
\alnum{a} jamaRniya. 
\alnum{b} jamaRnf -- BASeya, deVshada, janara yA avugaLa guNalakaSxNada. 
\eanum
\emng
\eentry

\bentry
\word[German(2)]{German}
\pron{jamaRnf}
\gl{\nA}
\bmng
jamaRnf: 
\banum
\alnum{a} jamaRniya deVshadavanu. 
\alnum{b} jamaRniya BASe. 
\eanum
\emng

\noindent
\gl{\pagu}
\bmng
\bnum
\num{1} \eng{High German} `heY jamaRnf' BASe; hiMde dakiSxNa pArxMtadalilx ADuBASeyAgidudx, Iga jamaRniyalelxlalx sAhitayxka hAgU shiSaTx baLakeyalilxruva BASe. 
\numi{2} \eng{Low German} 
\banum
\alnum{a} `loV jamaRnf' BASe; `heYjamaRnf' alalxda itara jamaRnf pArxMta BASegaLu. 
\alnum{b} heYjamaRnf BASeyanunxLidu iMgilxSf matutx Dacf BASegaLanonxLagoMDa `vesfTx jamaRnf' BASAvagaR. 
\eanum
\numie
\enum
\emng
\eentry

\bentry
\wordnospeech{German band}{German band}
\pron{?}
\gl{\nA}
\bmng
 jamaRnf -- bAyxMDu, meVLa; hAdigAyakara goVSiThx. 
\emng
\eentry

\bentry
\word{germander}
\pron{jamAYxRMDarf}
\gl{\nA}
\bmng
 (\savi) jamAyxRMDarf; sahadeVvi giDadaMte TuyxkirxyaM kulakekx seVrida giDa. 
\emng
\eentry

\bentry
\wordnospeech{germander speedwell}{germander speedwell}
\pron{?}
\gl{\nA}
\bmng
 jamAYxRMDarf sipxVDfvelf; jamAYxRMDarf giDadaMtha elegaLuLaLx, veronika kameVDirxsf kulakekx seVrida, niVli hUbiDuva baLiLx. 
\emng
\eentry

\bentry
\word{germane}
\pron{jameRVnf}
\gl{\gu}
\bmng
 (viSayakekx) saMgata; parxsakatx; parxkaqta; saMbadadhx; saMbaMdhisida: \eng{the illustration was hardly germane to the case} daqSATxMta A parxsaMgakekx savxlapxvU saMbaMdhisiralilalx. 
\emng
\eentry

\bentry
\word{germanely}
\pron{jameRVnfli}
\gl{\kirxvi}
\bmng
 (viSayakekx) parxsakatxvAgi; saMbaMdhisidaMte. 
\emng
\eentry

\bentry
\word[Germanic(1)]{Germanic}
\pron{jamAYxRnikf}
\gl{\gu}
\bmng
\bnum
\num{1} (\ca) jamaRnara: \eng{Germanic Confederation} jamaRnara okUkxTa. \eng{Germanic Empire} jamaRnara cakArxdhipatayx. 
\num{2} sAkxyXMDineVviyananxra, AYxMgolxVsAyxkasxnara yA jamaRnara. 
\num{3} sAkxYxMDineVviyanara, AYxMgolxVsAYxkasxnara, yA jamaRnara purAtana BASeya. 
\num{4} jamaRnf lakaSxNagaLuLaLx. 
\enum
\emng

\noindent
\gl{\pagu}
\bmng
\bnum
\num{1} \eng{East Germanic} gAthikf matutx Iga hecucx kaDime aLiduhoVgiruva bagaRMDiyanf, vAyxMDalf, \mo\ BASegaLu. 
\num{2} \eng{North Germanic} sAkxYxMDineVviyanf BASe. 
\num{3} \eng{West Germanic} heY matutx loV jamaRnf, iMgilxSf, phirxsiZyanf, Dacf, \mo\ BASAvagaR. 
\enum
\emng
\eentry

\bentry
\word[Germanic(2)]{Germanic}
\pron{jamAYxRnikf}
\gl{\nA}
\bmng
 jamAYxRnikf BASegaLu; iMgilxSf, jamaRnf, Dacf, Aphirxkanfsx, phelxmiSf, phirxsiyanf, gAthikf, sAkxYxMDineVviyanf, \mo\ purAtana BASegaLu. 
\emng
\eentry

\bentry
\word{Germanicism}
\pron{jamARYxnisisaZmf}
\gl{\nA}
\bmng
 sAkxyXMDineVviyanf, AYxMgolxVsAYxkasxnf, yA jamaRnara BASeya -- veYshiSaTxyX, saMparxdAya. 
\emng
\eentry

\bentry
\word{Germanish}
\pron{jamaRniSf}
\gl{\gu}
\bmng
\bnum
\num{1} jamaRnf BASeya veYshiSaTxyXda lakaSxNagaLuLaLx. 
\num{2} jamaRniya pakaSxpAtada yA olavina. 
\enum
\emng
\eentry

\bentry
\word{Germanism}
\pron{jamaRnisaZmf}
\gl{\nA}
\bmng
\bnum
\num{1} (\kanmu\ jamaRneVtara BASe yA upaBASegaLalilx parxyoVgisuva) jamaRnf nuDigaTuTx; jamaRnf BASA mayARde; jamaRnf BASeya vishiSaTx lakaSxNa. 
\num{2} jamaRnf pirxVti, vAyxmoVha, mamate. 
\num{3} jamaRnara riVti niVtigaLa, BAvanegaLa yA saMsethxgaLa olavu yA avugaLa anukaraNa. 
\num{4} jamaRni deVsha yA janarige vishiSaTxvAda dhoVraNegaLu, riVtiniVtigaLu, gotutxgurigaLu. 
\enum
\emng
\eentry

\bentry
\word{Germanist}
\pron{jamaRnisfTx}
\gl{\nA}
\bmng
 jamaRnf tajacnx: 
\banum
\alnum{a} jamaRni matutx adara BASeyanunx aritiruvavanu. 
\alnum{b} jamaRnf yA TUyxTAnikf BASA paMDita. 
\alnum{c} jamaRnf BAvanegaLa parxBAvakokxLagAdavanu. 
\eanum
\emng
\eentry

\bentry
\word{germanium}
\pron{jameRVniamf}
\gl{\nA}
\bmng
 (\ravi) jameRVniyamf; kAbaRnf matutx silikAnfgaLanunx hoVluva, paramANu saMKeyx \eng{32} matutx paramANu tUka \eng{72.59} uLaLx, oMdu areloVha (loVha, aloVhagaLa madhayx savxBAvada dhAtu). 
\emng
\eentry

\bentry
\word{Germanization}
\pron{jamaRneYseZVSanf}
\gl{\nA}
\bmng
 jamaRniVkaraNa: 
\banum
\alnum{a} jamaRnf riVtiniVtigaLa matutx parxvaqtitxgaLa aLavaDike, sivxVkAra. 
\alnum{b} jamaRnf parxBAvakokxLagAguvudu; (sheYli, aBiruci, padadhxti, BAva, sahAnuBUti, \mo vugaLalilx) jamaRnanaMtAguvike. 
\eanum
\emng
\eentry

\bentry
\word{Germanize}
\pron{jamaRneYsfZ}
\gl{\sakirx}
\bmng
\bnum
\num{1} (\pArxparx) jamaRnf BASege parivatiRsu. 
\num{2} jamaRniVkarisu; jamaRnanaMtAgisu; jamaRnf riVti niVtigaLanunx, manoVvaqtitxgaLanunx aLavaDisu: \eng{cities of the Polish corridor were Germanized when the Nazis took over} nATisxgaLu AkarxmisikoMDAga poVlaMDina parxdeVshadalilxya nagaragaLu jamaRniVkaqtavAdavu. 
\enum
\emng

\noindent
\gl{\akirx}
\bmng
 jamaRnAnxgu; jamaRniVyavAgu; jamaRnf saMparxdAya, riVtiniVtigaLiMda parxBAvitanAgu; sheYli, aBiruci, padadhxtigaLu, BAva, sahAnuBUti, \mo vugaLalilx jamaRnanaMtAgu. 
\emng
\eentry

\bentry
\word{Germanizer}
\pron{jamaRneYsaZrf}
\gl{\nA}
\bmng
\bnum
\num{1} jamaRniVkarisuvavanu; jamaRnf riVtiniVtigaLanunx, manaHparxvaqtitxgaLanunx aLavaDisikoLuLxvavanu. 
\num{2} jamaRnAnxguvavanu; jamaRnanaMtAguvavanu; jamaRnf saMparxdAya, riVtiniVtigaLiMda parxBAvitanAguvavanu. 
\enum
\emng
\eentry

\bentry
\wordnospeech{German measles}{German measles}
\pron{?}
\gl{\nA}
\bmng
 jamaRnf daDAra; oMdu bageya daDAra, gobabxra. 
\emng
\eentry

\bentry
\word{Germano-}
\pron{jamAYxRnoV-}
\gl{\sapUpa}
\bmng
\bnum
\num{1} jamaRniya, adakekx saMbaMdhisida eMbathaRgaLalilx baLasuva \sapUpa: \eng{Germanophobia} jamaRniya BiVti. 
\num{2} jamaRniya matutx eMbathaRdalilx baLasuva \sapUpa: \eng{Germano Russion} jamaRni matutx raSayxda. 
\enum
\emng
\eentry

\bentry
\wordnospeech{German Ocean}{German Ocean}
\pron{?}
\gl{\nA}
\bmng
 (\pArxparx) utatxra samudarx; nAtfR siV. 
\emng
\eentry

\bentry
\word{Germanomania}
\pron{jamAYxRnameVnia}
\gl{\nA}
\bmng
 jamaRni -- giVLu, hucucx, vAyxmoVha; jamaRnige saMbaMdhisidavugaLalilx hucucx pirxVti. 
\emng
\eentry

\bentry
\word[Germanophil(1)]{Germanophil}
\pron{jamAYxRnaphilf}
\gl{\nA}
\bmng
 jamaRnf perxVmi; jamaRnaralilx, avara saMsethxgaLalilx, padadhxtigaLalilx mecucxgeyuLaLxvanu. 
\emng
\eentry

\bentry
\word[Germanophil(2)]{Germanophil}
\pron{jamAYxRnaphilf}
\gl{\gu}
\bmng
 jamaRni perxVmada; jamaRnaralilx, avara saMsethxgaLalilx, padadhxtigaLalilx mecucxge iruva. 
\emng
\eentry

\bentry
\word[Germanophile(1)]{Germanophile}
\pron{jamAYxRnapheYlf}
\gl{\nA}
\bmng
  = \hyperlink{Germanophil(1)}{$^1$Germanophil}. 
\emng
\eentry

\bentry
\word[Germanophile(2)]{Germanophile}
\pron{jamAYxRnapheYlf}
\gl{\gu}
\bmng
  = \hyperlink{Germanophil(2)}{$^2$Germanophil}. 
\emng
\eentry

\bentry
\word[Germanophobe(1)]{Germanophobe}
\pron{jamAYxRnaphoVbf}
\gl{\gu}
\bmng
 jamaRnf BiVta; jamaRni Bayada; jamaRniya bagege asahayx yA Baya paDuva. 
\emng
\eentry

\bentry
\word[Germanophobe(2)]{Germanophobe}
\pron{jamAYxRnaphoVbf}
\gl{\nA}
\bmng
 jamaRnfBiVta; jamaRniya viSayadalilx BayapaDuvavanu. 
\emng
\eentry

\bentry
\word{Germanophobia}
\pron{jamAYxRnaphoVbia}
\gl{\nA}
\bmng
 jamaRnfBiVti; jamaRniya, adara sakARrada, adara kAyaRkalApagaLa, adara veYlakaSxNayxgaLa viSayadalilx atayxMta asahayx, BiVti. 
\emng
\eentry

\bentry
\word{germanous}
\pron{jameRVnasf}
\gl{\gu}
\bmng
 (\ravi) jameRVnasf; divxveVlanisxVya sithxtiya jameRVniyamamxnunx hoMdiruva. 
\emng
\eentry

\bentry
\wordnospeech{German sausage}{German sausage}
\pron{?}
\gl{\nA}
\bmng
 jamaRnf sAseVju; masAle hAki arebeVyisida mAMsada hUraNavuLaLx oMdu BakaSxyX. 
\emng
\eentry

\bentry
\wordnospeech{German shepherd}{German shepherd}
\pron{?}
\gl{\nA}
\bmng
 = \hyperref{kandict_a.pdf}{A}{Alsatian(2)}{$^2$Alsatian}. 
\emng
\eentry

\bentry
\wordnospeech{German silver}{German silver}
\pron{?}
\gl{\nA}
\bmng
 jamaRnf beLiLx; nikakxlf, satu matutx tAmarx seVrida biLiya misharx loVha. 
\emng
\eentry

\bentry
\wordnospeech{German text}{German text}
\pron{?}
\gl{\nA}
\bmng
 = \hyperref{kandict_b.pdf}{B}{black letter}{black letter}. 
\emng
\eentry

\bentry
\wordnospeech{germ carrier}{germ carrier}
\pron{?}
\gl{\nA}
\bmng
roVgANuvAhaka; roVgakekx guriyAgadeV roVgANuvanunx vayxkitxyiMda vayxkitxge haraDuva jiVvi. 
\emng
\eentry

\bentry
\word{germ-cell}
\pron{jamfRself}
\gl{\nA}
\bmng
=  \hyperlink{gamete}{gamete}. 
\emng
\eentry

\bentry
\word{germen}
\pron{jameRnf}
\gl{\nA}
\bmng
 (\savi) jameRnf; sasayxpAtarxda -- kuDi, aMkura, AdimarUpa. 
\emng
\eentry

\bentry
\word{germicidal}
\pron{jamiRseYDalf}
\gl{\gu}
\bmng
 roVgANuhAraka; roVgANuhAriya yA adakekx saMbaMdhisida. 
\emng
\eentry

\bentry
\word{germicide}
\pron{jamiRseYDf}
\gl{\nA}
\bmng
 roVgANuhAri; roVgANuhara; roVgANugaLanunx nAsha mADabalalx padAthaR. 
\emng
\eentry

\bentry
\word{germinal}
\pron{jamiRnalf}
\gl{\gu}
\bmng
\bnum
\num{1} roVgANugaLa. 
\num{2} roVgANu savxBAvada. 
\num{3} moLakeyalilxruva; kuDiyetutxtitxruva; aMkurAvasethxyalilxruva: \eng{the germinal philosophical ideas underlying western culture} pAshAcxtayx saMsakxqqtiya aDi aMkurAvasethxyalilxruva tAtitxvXka vicAragaLu. 
\num{4} saqjanashiVla; hosa vicAragaLanunx, parxBAvagaLanunx parxcoVdisuva. 
\enum
\emng
\eentry

\bentry
\word{germinally}
\pron{jamiRnali}
\gl{\kirxvi}
\bmng
 aMkurAvasethxyalilx; moLakeya rUpavAgi; kuDiyAgi. 
\emng
\eentry

\bentry
\word{germinant}
\pron{jamiRnaMTf}
\gl{\gu}
\bmng
 (\sA\ \rUpa) moLeyuva; vaqdidhxhoMduva; beLeyuva shakitxyuLaLx; vadhiRSuNx. 
\emng
\eentry

\bentry
\word{germinate}
\pron{jamiRneVTf}
\gl{\sakirx}
\bmng
\bnum
\num{1} cigurisu; kuDiyiDisu. 
\num{2} araLisu; beLesu; vadhiRsu. 
\num{3} utapxtitx mADu. 
\enum
\emng

\noindent
\gl{\akirx}
\bmng
 aMkurisu; moLe; beLe; ciguru; kuDiyoDe; mogugxbiDu (\rUpa\ saha). 
\emng
\eentry

\bentry
\word{germination}
\pron{jamiRneVSanf}
\gl{\nA}
\bmng
 aMkuraNa; ciguruvudu; moLake oDeyuvudu; kuDiyoDeyuvudu (\rUpa\ saha). 
\emng
\eentry

\bentry
\word{germinative}
\pron{jamiRneV(na)Tivf}
\gl{\gu}
\bmng
\bnum
\num{1} moLetakekx saMbaMdhisida; moLetada; aMkuraNda. 
\num{2} aMkuraNa, moLeyuva, beLeyuva -- shakitxyuLaLx. 
\enum
\emng
\eentry

\bentry
\word{germinator}
\pron{jamiRneVTarf}
\gl{\nA}
\bmng
\bnum
\num{1} vadhaRka; beLeyisuvaMthadu. 
\num{2} aMkurashakitxmApaka; biVjagaLa moLeyuva shakitxyananxLeyuva salakaraNe. 
\enum
\emng
\eentry

\bentry
\word{germon}
\pron{jamaRnf}
\gl{\nA}
\bmng
 = \hyperref{kandict_a.pdf}{A}{albacore}{albacore}. 
\emng
\eentry

\bentry
\word{germ-plasm}
\pron{jamfRpAlxyXsfZmx}
\gl{\nA}
\bmng
 jananadarxvayx; AnuvaMshika darxvayxvanunx hotatx gamITf matutx avugaLa pUvaRgAmigaLu. 
\emng
\eentry

\bentry
\wordnospeech{germ warfare}{germ warfare}
\pron{?}
\gl{\nA}
\bmng
 roVgANu yudadhx; yudadhxdalilx roVgakAraka sUkaSxmXjiVvigaLanunx asatxrXvAgi baLasuvudu. 
\emng
\eentry

\bentry
\word{gerontocracy}
\pron{jerAMTAkarxsi}
\gl{\nA}
\bmng
\bnum
\num{1} vaqdadhx(ra) parxButavx; vaqdadhxra, mudukara -- ADaLita, dabARru, sakARra. 
\num{2} mudukara, vaqdadhxra ADaLitamaMDaLi. 
\enum
\emng
\eentry

\bentry
\word{gerontocratic}
\pron{jerAMTakArxYxTikf}
\gl{\gu}
\bmng
 mudukarALivxkeya; vaqdadhxra parxButavxda; vaqdadhxra parxButavxkekx -- saMbaMdhisida yA vishiSaTxvAda. 
\emng
\eentry

\bentry
\word{gerontological}
\pron{jerAMTalAjikalf}
\gl{\gu}
\bmng
 jarAshAsatxrXda yA adakekx saMbaMdhisida. 
\emng
\eentry

\bentry
\word{gerontologist}
\pron{jerAMTAlajisfTx}
\gl{\nA}
\bmng
 mupupx tajacnx; jarAshAsatxrXjacnx; vaqdAdhxpayxtajacnx. 
\emng
\eentry

\bentry
\word{gerontology}
\pron{jerAMTAlaji}
\gl{\nA}
\bmng
 mupupxshAsatxrX; vaqdAdhxpayxshAsatxrX; jarAshAsatxrX; mupupx, adara parxkirxye, hAgU vayasAsxdavara sAmAjika samaseyxgaLanunx kurita adhayxyana. 
\emng
\eentry

\bentry
\word[gerontophil(1)]{gerontophil}
\pron{jerAMTaphilf}
\gl{\nA}
\bmng
 vaqdadhxkAmi; vayasAsxdavaroDane leYMgika saMbaMdha apeVkiSxsuva vayxkitx. 
\emng
\eentry

\bentry
\word[gerontophil(2)]{gerontophil}
\pron{jerAMTaphilf}
\gl{\gu}
\bmng
 vaqdadhxkAmiyAda; vayasAsxdavaroDane leYMgika saMbaMdha apeVkiSxsuva. 
\emng
\eentry

\bentry
\word[gerontophile(1)]{gerontophile}
\pron{jerAMTapheYlf}
\gl{\nA}
\bmng
  = \hyperlink{gerontophil(1)}{$^1$gerontophil}. 
\emng
\eentry

\bentry
\word[gerontophile(2)]{gerontophile}
\pron{jerAMTapheYlf}
\gl{\gu}
\bmng
  = \hyperlink{gerontophil(2)}{$^2$gerontophil}. 
\emng
\eentry

\bentry
\word{gerontophilia}
\pron{jerAMTaphilia}
\gl{\nA}
\bmng
 vaqdadhxkAma; vayasAsxdavaroDane saMBoVga apeVkiSxsuvudu. 
\emng
\eentry

\bentry
\word{gerontophilic}
\pron{jerAMTaphilikf}
\gl{\gu}
\bmng
  = \hyperlink{gerontophil(2)}{$^2$gerontophil}. 
\emng
\eentry

\bentry
\word{gerontophily}
\pron{jerAMTaphili}
\gl{\nA}
\bmng
  = \hyperlink{gerontophilia}{gerontophilia}. 
\emng
\eentry

\bentry
\wordwithhyphen{hyp-gerous}{-gerous}
\pron{-jarasf}
\gl{\saupa}
\bmng
\bnum
\num{1} hoMdiruva, uLaLx, paDediruva eMbathaRda \gu gaLanunx racisuvalilx: \eng{dentigerous} halulxLaLx. 
\num{2} kAraka, -kAri, utApxdaka, utapxtitx mADuva eMbathaRda \gu gaLanunx racisuvalilx: \eng{crystalligerous} saPxTikakAraka; haraLugaLanunx utapxtitx mADuva. 
\enum
\emng
\eentry

\bentry
\word[gerrymander(1)]{gerrymander}
\pron{jerimAYxMDarf}
\gl{\sakirx}
\bmng
(cunAvaNe keSxVtarx \mo vugaLalilx) keYvADa naDesu; hasatxkeSxVpa naDesu; (oMdu paMgaDakAkxgali, pakaSxkAkxgali vipariVta pArxbalayx doreyuvaMte akarxmavAgi) cunAvaNeya keSxVtarxda elelxgaLu \mo vanunx kuTila riVtiyalilx vayxtAyxsa mADu, badalAyisu. 
\emng
\eentry

\bentry
\word[gerrymander(2)]{gerrymander}
\pron{jerimAYxMDarf}
\gl{\nA}
\bmng
(cunAvaNe keSxVtarx \mo vugaLalilx) moVsada keYvADa; kuTila badalAvaNe, vayxtAyxsa. 
\emng
\eentry

\bentry
\word{gerrymanderer}
\pron{jerimAYxMDararf}
\gl{\nA}
\bmng
 (cunAvane keSxVtarx \mo vugaLalilx) kuTiloVpAyi; keYvADa naDesuvavanu; hasatxkeSxVpa naDesuvavanu; kuTila kelasa naDesuvavanu. 
\emng
\eentry

\bentry
\word{gertcha}
\pron{gecaR}
\gl{\BAavayx}
\bmng
 (\asaM) tolagi hoVgu! (apanaMbikeyanunx sUcisuva \parx). 
\emng
\eentry

\bentry
\word{gerund}
\pron{jeraMDf}
\gl{\nA}
\bmng
 kaqnAnxma; kaqdaMta BAvanAma; kirxyApadakekx \eng{-ing} seVrisidare baruva, kirxyeyanunx sUcisuva BAvanAma. \udA\ \eng{doing}. 
\emng
\eentry

\bentry
\word{gerund-grinder}
\pron{jeraMDfgerxYMDarf}
\gl{\nA}
\bmng
\bnum
\num{1} (\pArxparx) lAyxTinf shikaSxka; lAyxTinf vAyxkaraNavanunx aredu kuDisuvavanu. 
\num{2} shuSakx shikaSxNa; oNa pAMDitayxda riVtiya boVdhane. 
\enum
\emng
\eentry

\bentry
\word{gerundial}
\pron{jeraMDialf}
\gl{\gu}
\bmng
 kaqnAnxmada; kaqnAnxmasUcaka: \eng{a gerundial suffix} kaqnAnxma sUcaka \uparx. 
\emng
\eentry

\bentry
\word{gerundival}
\pron{jeraMDeYvalf}
\gl{\gu}
\bmng
kaqdaMta guNavAcakada. 
\emng
\eentry

\bentry
\word{gerundivally}
\pron{jeraMDeYvali}
\gl{\kirxvi}
\bmng
 kaqdaMta guNavAcakavAgi; kaqdaMta guNavAcakada athaRdalilx. 
\emng
\eentry

\bentry
\word[gerundive(1)]{gerundive}
\pron{jeraMDivf}
\gl{\gu}
\bmng
 kaqnAnxmada yA kaqnAnxmadaMtha. 
\emng
\eentry

\bentry
\word[gerundive(2)]{gerundive}
\pron{jeraMDivf}
\gl{\nA}
\bmng
 (lAyxTinf \vAyx) `karaNiVya' eMbathaRda kaqdaMta guNavAcaka. 
\emng
\eentry

\bentry
\word{gesso}
\pron{jesoV}
\gl{\nA}
\expl{(\bava\ \eng{gessoes}).}
\bmng
 pAlxsaTxrf Aphf pAyxrisf; citarxkale hAgU shilapxkaleyalilx upayoVgisalu tayArisida `jipasxM' yA `pAlxsaTxrf Aphf pAyxrisf' eMba vasutx. 
\emng
\eentry

\bentry
\word{gestalt}
\pron{gasATxlfTx}
\gl{\nA}
\bmng
% (\mashA) gasATxluTx; samaSiTx: 
\banum
\alnum{a} samagarxsamaSiTx; tananx avayavagaLa, aMgaBAgagaLa yA GaTakAMshagaLa bari saMkalana yA motatxkikxMtalU hecicxnadAgiruvudAgi-- \udA\ tananx GaTakagaLAda biDi savxragaLigiMtalU BinanxvAgi toVruva rAgadaMte garxhisalapxDuva suvayxvasithxta pUNARkaqti, pUNaRrUpa; tananx GaTakAMshagaLa keVvala motatxdiMda paDeyalAgada vishiSaTx guNalakaSxNavanunx hoMdiruva Akaqti, vinAyxsa yA vayxvasithxta racane. 
\alnum{b} aMtha samagarx samaSiTxya oMdu nidashaRna. 
\eanum
\emng
\eentry

\bentry
\wordnospeech{gestalt psychology}{gestalt psychology}
\pron{?}
\gl{\nA}
\bmng
 (\mashA) gasATxlfTx manoVvijAcnxna; smaSiTx manoVvijAcnxna; saMveVdanegaLu, parxtikirxyegaLu, \mo vu samagarx samaSiTxgaLu, pUNARkaqtigaLu eMba manoVveYjAcnxnika sidAdhxMta; deYhika yA mAnasika parxtikirxyegaLu biDi GaTakAMshagaLAda parxtikirxyegaLa matutx saMveVdanegaLa keVvala motatxdiMda saMBavisade, avugaLigiMta BinanxvAgi yA avugaLige AMtarikavAgi saMbaMdhisiruvaMte kAyaR mADuva samagarx samaSiTxgaLa mUlaka odagutatxve eMba manoVveYjAcnxnika sidAdhxMta. 
\emng
\eentry

\bentry
\word{Gestapo}
\pron{gesATxpoV}
\gl{\nA}
\bmng
 gesATxpoV: 
\banum
\alnum{a} rahasayx jamaRnf nAji poliVsu daLa. 
\alnum{b} (\hiV) idanunx hoVluva yAvudeV saMsethx. 
\eanum
\emng
\eentry

\bentry
\word{gestate}
\pron{jeseTxVTf}
\gl{\sakirx}
\bmng
 gaBaRdalilx (yA A riVtiyalilx) dharisu yA dharisiru, horu yA hotitxru. 
\emng
\eentry

\bentry
\word{gestation}
\pron{jeseTxVSanf}
\gl{\nA}
\bmng
\bnum
\num{1} gaBARvasethx; gaBaRdhAraNe; gaBaRvAsa; gaBoRVtapxtitx kAladiMda parxsavavAguvavaregU gaBaRdalilx iruvudu yA dharisiruvudu. 
\num{2} gaBARvadhi; gaBaRvAsada yA gaBaRdharisiruva kAla. 
\num{3} (\rUpa) savxMtavadhaRne; yoVjane \mo vugaLanunx manasisxnalilx kalipxsikoMDu savxMtavAgi beLesuvudu. 
\enum
\emng
\eentry

\bentry
\wordRemoveSpace{gestatorial-chair}{gestatorial chair}
\pron{jesaTxToVrialf ceVrf}
\gl{\nA}
\bmng
 poVpana palalxkikx, kuciR; kelavu veVLe poVpananunx hegala meVle oyuyxva kuciR. 
\emng
\eentry

\bentry
\word{gesticulate}
\pron{jesiTxkuyxleVTf}
\gl{\sakirx}
\bmng
 aBinayada mUlaka sUcisu; aBinayisi toVrisu. 
\emng

\noindent
\gl{\akirx}
\bmng
 (mAtilalxde) BAvABinaya mADu; aMgABinaya mADu; avayavagaLanunx yA deVhavanunx BAvasUcakavAgi ADisu. 
\emng
\eentry

\bentry
\word{gesticulation}
\pron{jesiTxkuyxleVSanf}
\gl{\nA}
\bmng
 BAvABinaya; aMgABinaya; AMgikABinaya; aBinayada calanavalanagaLu. 
\emng
\eentry

\bentry
\word{gesticulative}
\pron{jesiTxkuyxleV(la)Tivf}
\gl{\gu}
\bmng
 BAvABinayasUcakavAda; aMgABinayadiMda kUDida; aBinayada. 
\emng
\eentry

\bentry
\word{gesticulator}
\pron{jesiTxkuyxleVTarf}
\gl{\nA}
\bmng
 naTa yA naTi; BAvABinaya, aMgABinaya mADuvavanu(Lu). 
\emng
\eentry

\bentry
\word{gesticulatory}
\pron{jesiTxkuyxleVTari}
\gl{\gu}
\bmng
  = \hyperlink{gesticulative}{gesticulative}. 
\emng
\eentry

\bentry
\word{gestural}
\pron{jesacxralf}
\gl{\gu}
\bmng
 aBinayada; aMgada yA deVhada BAvapUrita calaneya yA adakekx saMBaMdhisida. 
\emng
\eentry

\bentry
\word[gesture(1)]{gesture}
\pron{jesacxrf}
\gl{\nA}
\bmng
\bnum
\num{1} aBinaya; aMgada yA deVhada BAvasUcaka, athaRsUcaka calane. 
\num{2} (BAvasUcanegAgi yA BASaNada pariNAmakAkxgi) iMtha calanavalanagaLa parxyoVga; aMgasUcane; aMgasanenx; BAvABinaya. 
\num{3} sUcayxvataRne; iMgitasUcane; matotxbabxnalilx parxtivataRneyanunxMTumADalu yA (\kanmu\ senxVhapUvaRkavAda) tananx iMgitavanunx tiLisalu udedxVshisida naDe(vaLi), vataRne, vayxvahAra: \eng{a political gesture to draw popular support} keVvala sAvaRjanikara beMbalavanunx giTiTxsuva udedxVshada rAjakiVya vataRne, iMgitasUcane. 
\enum
\emng
\eentry

\bentry
\word[gesture(2)]{gesture}
\pron{jesacxrf}
\gl{\kirx}
\bmng
  = \hyperlink{gesticulate}{gesticulate}. 
\emng
\eentry

\bentry
\word[get(1)]{get}
\pron{geTf}
\gl{\kirx}
\expl{[\BU\ \eng{got}, (\pArxparx) \eng{gat}; \BUkaq\ \eng{got};}
samAsadalilx, \pArxparx dalilx matutx \ame gaLalilx mAtarx\eng{gotten}; \udA\  \eng{illgotten}].

\noindent
\gl{\sakirx}
\bmng
\bnum
\num{1} (parxyatanxdiMda, upAyadiMda, yaMtorxVpakaraNadiMda) paDe; tege; hoMdu: \eng{get coal from mine} gaNiyiMda kalilxdadxlanunx (agedu) tege, paDe. 
\num{2} gaLisu; dorakisiko; saMpAdisu; ajiRsu: \eng{cannot get a living} jiVvanoVpAya dorakisikoLuLxvudu kaSaTx. 
\num{3} lABa -- paDe, gaLisu: \eng{got little by it} adariMda EnU lABa sikakxlilalx. 
\num{4} (heVgAdarU mADi) jayagaLisu; (EnanAnxdarU) gelulx, paDe: \eng{get fame, credit or glory} oLeLxyA hesaru, hecacxLike yA kiVtiR gaLisu, paDe. \eng{get runs} (kirxkeTiTxnalilx)ranunxgaLanunx gaLisu, hoDe. \eng{get wickets} (kirxkeTiTxnalilx) vikeTuTxgaLanunx paDe, uruLisu. 
\num{5} kaMThapATha mADu; bAyipATha mADu; gaTiTxmADu: \eng{to get the part without the book} pusatxkavilalxde pAtarxda mAtugaLanunx gaTiTxmADu. 
\num{6} lekAkxcArada PalavAgi paDe; lekAkxcAradiMda paDe: \eng{we get 9.5 as the average} sarAsari \eng{9.5}nunx paDeyutetxVve. 
\num{7} (kUli, saMbaLa, uDugore, \mo vanunx) paDe; hoMdu; gaLisu. 
\num{8} (kADi, beVDi, vicArisi) paDe; sigu; giTiTxsu; saMpAdisu: \eng{could not get leave} avanige raja sikakxlilalx. \eng{got his father's consent to use the car} moVTAru kAranunx baLasalu taMdeyanunx kADi samamxti giTiTxsida. 
\num{9} (iSaTxvAda vasutx \mo vanunx) paDe; paDeyuvaMtAgu: \eng{get rest} vishArxMti paDe. \eng{get one's way} tanage beVkAdudanunx sAdhisiko. \eng{get speech of (some one)} (obabxriMda) mAtu horaDisu; (obabxnu) mAtanADuvaMte mADu. \eng{get a sight of} dashaRnalABa paDe. \eng{get possession of} vashapaDisiko; sAvxdhiVna paDe. \eng{get religion} dhamaRvanunx sivxVkarisu; matAMtara hoMdu. 
\num{10} (BAvane \mo vugaLanunx) hacicxsiko; hacicxko; aMTisiko: \eng{get measles} daDAra aMTu; taTaTxmamx ELu. 
\num{11} (obabxna meVle) horisu; vidhisu. 
\num{12} (pAlige baMda ahita \mo vanunx) anuBavisu: \eng{get a fall} biVLu. \eng{get a cold} negaDi hatutx; negaDiyiMda toMdare paDu. \eng{get a blow} ETu tinunx. \eng{get six months in prison} Aru tiMgaLu saja vidhisalapxDu. 
\num{13} koDisu; tarisikoDu; siguvaMte mADu: \eng{got him a place} avanigoMdu kelasa koDisida. \eng{we can get it for you} nAvu adanunx ninage tarisikoDabalelxvu. 
\num{14} (mInu, beVTe, pArxNi, \mo vanunx) hiDi: \eng{got several trout before breakfast} beLagina upAhArakekx muMce halavu TwrxTf mInugaLanunx hiDida. 
\num{15} (udedxVshapUvaRkavAgi) kolulx yA gAyagoLisu. 
\num{16} (peYru, utapxtitx, \mo vanunx) beLe; Pasalu tege, paDe: \eng{got a good crop of wheat} oMdu sogasAda goVdi Pasalu tegeda. 
\num{17} (parxsAra mADida saMkeVtavanunx) garxhisu. 
\num{18} (vayxkitx yA sathxLavanunx) dUravANiya mUlaka saMpakiRsu. 
\num{19} peVcige sikikxsu; kakAkxbikikxyanunxMTumADu; digaBxrXme hiDisu; vAdadalilx soVlisu, hiDidu hAku: \eng{this problem really gets me} nijavAgiyU I samaseyx nanage digaBxrXme hiDisutatxde. 
\num{20} (\AmA) (vayxkitx yA vasutxvanunx) athaRmADiko; hiDi; garxhisu; tiLiduko: \eng{sorry, I didn't get your name} kaSxmisi, nimamx hesaru nanage gotAtxgalilalx (keVLisalilalx). \eng{the audience readily got the speaker's point} shorxVtaqgaLu BASaNakArana mAtina athaRvanunx (iMgitavanunx) kUDaleV hiDidaru. \eng{don't get me wrong} nananxnunx tapApxgi BAvisabeVDa. 
\num{21} (\AmA) reVgisu; keraLisu; kirikirigoLisu: \eng{It got me, her talking that way} avaLu hAge mAtADidudx nananxnunx keraLisitu. 
\num{22} (\AmA)AkaSiRsu; manasusx seLe; giVLu huTiTxsu; hucucxhiDisu: \eng{It's the rhythm that gets you} laya ninanxnunx AkaSiRsuvudu. 
\num{23} (\AmA) (UTa, upAhAra, \mo vanunx) seVvisu; tegeduko: \eng{come and get your tea with us} baninx, namamx jote TiV tegedukoLiLx. 
\num{24} (Iga pArxNigaLa \vi mAtarx \eng{beget}) huTiTxsu; paDe; saMtAnoVtapxtitxmADu. 
\num{25} (taruvudu, sAgisuvudu, iDuvudu, oLakekx baramADikoLuLxvudu, horakekx kaLuhisuvudu, \mo vanunx) mADu; mADi mugisu: \eng{got it through the door} bAgilina mUlaka sAgisidAdxyitu. \eng{got it into the room} koThaDiya oLakekx taMdadAdxyitu. (\rUpa) \eng{flattery will get you nowhere} muKasutxti ninanxnunx elilxgU koMDoyuyxvudilalx, ninage EnU taruvudilalx. 
\num{26} (yAvudoV) oMdu sithxtige taru; oMdu sithxtiyanunxMTumADu: \eng{get wet} toyudxhoVgu; odedxyAgu. \eng{get them ready} (avanunx) tayArisiDu; sidadhxmADu; (avaranunx) sidadhxrAgiruvaMte mADu. \eng{get your hair cut} ninanx tale kwSxramADisiko. \eng{get it done} adanunx mADi mugisu. \eng{get oneself elected} cunAyitanAgu; cunAyisalapxDu. \eng{get him arrested} avananunx dasatxgiri mADu. 
\num{27} (obabx vayxkitx oMdu kelasavanunx mADuvaMte) manavolisu; perxVrisu; opipxsu; oDaMbaDisu; pusalAyisu: \eng{we'll get him to go with us} namamx jote baruvaMte avananunx opipxsutetxVve. \eng{got the publisher to bring out a new deluxe edition} garxMthada hosa shirxVmaMta Avaqtitxyanunx parxkaTisuvaMte parxkAshakana manavolisida. 
\num{28} (UTa \mo vanunx) tayArisu; sidadhxpaDisu: \eng{get dinner} UTa sidadhxmADu. 
\num{29} serehiDi; vashapaDisiko. 
\num{30} seVDu tiVrisiko; parxtiVkAra mADu. 
\num{31} (BAvuka) pariNAma biVru: \eng{her tears got me} avaLa kaNiNxVru nananx meVle parxBAva uMTumADitu, nananx manasisxna meVle pariNAma biVritu. 
\num{32} AgamADisu; yAranenxV yA yAvudanenxV kelasa mADuvaMte mADu: \eng{get the old car going again} A haLe kAru matetx calisuvaMte mADu. 
\num{33} (\AmA) hoMdiru; paDediru: \eng{have not got a penny} nananxlilx oMdu cikAkxsU ilalx. 
\num{34} (yAvudAdarU) oMdu aMga, avayava -- Una mADiko, GAsi mADiko: \eng{got my wrist dislocated} nananx maNikaTuTx tirucihoVyitu, kiVlu tapipxtu. 
\enum
\emng

\noindent
\gl{\akirx}
\bmng
\bnum
\num{1} (hoVgi yA baMdu) talapu; seVru; muTuTx: \eng{when do we get there?} yAvAga alilx taluputetxVve? \eng{where has it got to!} adu elilxge muTiTxtu? 
\num{2} (\ashi) horaDu; horaTuhoVgu; tolagu; nikalAyisu. 
\num{3} (dhAtavxthaRvAciyoDane) aBAyxsavAgu; baLakeyAgu; rUDhiyAgu: \eng{one soon gets to like it} adu beVga aBAyxsavAgutatxde, rUDhiyAgutatxde. 
\num{4} (EnanAnxdarU ADalu yA mADalu) toDagu; shuru mADu; pArxraMBisu; ADa hatutx: \eng{they got talking} avaru mAtanADatoDagidaru. \eng{get going} horaDalu shurumADu. 
\num{5} Agu; uMTAgu: \eng{get hot} bisiyAgu. \eng{get tired} susAtxgu. \eng{get married}maduveyAgu. \eng{get shelved} mUlege biVLu; mUlege hAkisu. \eng{get used to it} (adanunx) aBAyxsa, rUDhi, baLake -- mADiko. \eng{get caught in the rain} maLeyalilx sikikxbiVLu; sikikxhAkiko. \eng{get in a panic} gAbariyAgu; goMdalakekx biVLu. 
\enum
\emng

\noindent
\gl{\pagu}
\bmng
\bnum
\numi{1} \eng{get about} 
\banum
\alnum{a} eDeyiMdeDege hoVgu, aDADxDu. 
\alnum{b} (roVga guNavAgi) edudx -- ODADu, tirugADu. 
\hypertarget{get nuga1c}{} 
\alnum{c} (sudidxya \vi\ \kanmu\ kaNARkaNiRkeyAgi, bAyiMda bAyige) haraDu; parxcAravAgu. 
\eanum
\numie
\num{2} \eng{get abroad =} \hyperlink{get nuga1c}{?nuga? \((1c)\).} 
\numi{3} \eng{get along} 
\banum
\alnum{a} muMduvari. 
\alnum{b} (cenAnxgiyoV, alalxdeyoV, heVgoV) iru; bALu. 
\alnum{c} (AvashayxkavAdadudx ilalxdidadxrU) heVgoV -- nivaRhisu, sAvarisikoMDu hoVgu, anusarisikoMDu naDe. 
\alnum{d} (-oDane) sAmarasayxdiMdiru; kUDikoMDu, hoMdikoMDu -- hoVgu. 
\alnum{e} (rUpaka) jayagaLisu; yashasivxyAgu. 
\eanum
\numie
\numi{4} \eng{get along with you!} 
\banum
\alnum{a} naDe! tolagu! 
\alnum{b} (ninanx mAtu, vataRne) shudadhx asaMbadadhx! 
\eanum
\numie
\numi{5} \eng{get away} 
\banum
\alnum{a} tapipxsikoMDu hoVgu; tapipxsiko. 
\alnum{b} horaDu. 
\alnum{c} (vidhiyAgi) horaTu hoVgu! tolagu! naDe! 
\eanum
\numie
\numi{6} \eng{get back} 
\banum
\alnum{a} (manege) hiMdirugu. 
\alnum{b} kaLeduhoVdadadxnunx hiMdakekx paDe. 
\alnum{c} (AjecnxyAgi) hiMdakekx hoVgu! hiMde idadx jAgakekx hoVgu! 
\eanum
\numie
\numi{7} \eng{get down} 
\banum
\alnum{a} (vAhanadiMda) iLi. 
\alnum{b} nuMgu: \eng{the pill was so large that he couldn't get it down} guLige nuMgalAradaSuTx dapapxgitutx. 
\alnum{c} baravaNigeyalilx iDu, dAKalu mADu. 
\alnum{d} (\AmA) nirutAsxhagoLisu; kugigxsu; daNisu; AyAsapaDisu: \eng{nothing gets me down so much as cold} negaDiyaSuTx nananxnunx inAnxva roVgavU daNisuvudilalx. 
\eanum
\numie
\num{8} \eng{get down to} (yAvudaradeV bagegx) kelasa pArxraMBisu; kAyARraMBamADu; kAyaRparxvaqtatxnAgu. 
\numi{9} \eng{get in} 
\banum
\alnum{a} (shAsanasaBe \mo vugaLige)cunAyitanAgu. 
\alnum{b} (gADi \mo vanunx) Eru; hatutx; hoVgu. 
\alnum{c} (beLeyanunx) (manege) taMduko; tuMbiko; sheVKarisu: \eng{get in the crops} Pasalanunx manege taMdu tuMbiko. 
\alnum{d} (sAla \mo vanunx) vasUlu mADu. 
\alnum{e} (kelasa \mo vanunx) gotAtxda kAlAvadhige hoMdisu; nigadiyAda kAlAvadhiyoLage mADi mugisu yA mugisuvaMte EpARDu mADu: \eng{we are not bound to get in by a certain date} nidiRSaTx tAriVKinoLage mADi mugisabeVkeMba nibaRMdha namagilalx. 
\alnum{f} (ETanunx) Aya noVDi hoDe; (pariNAmakAriyAguvaMte) hAku: \eng{the youngster got in a nasty blow on his opponent's face} edurALiya moVreya meVle A huDuga balavAda ETanunx hAkida. 
\alnum{g} (barabeVkAda haNa, sAla, \mo vanunx) vasUlu mADu. 
\alnum{h} (kAleVju \mo vugaLige) seVru. 
\alnum{i} senxVha gaLisu. 
\alnum{j} (oMdu sathxLa) talupu; seVru. 
\eanum
\numie
\numi{10} \eng{get into} (\AmA) 
\banum
\alnum{a} (pAdarakeSx, baTeTx) hAkiko; sikikxsiko. 
\alnum{b} (yAvudeV oMdaralilx)Asakitx huTuTx; muLugu: \eng{`What are you reading?' `logic'. `I couldn't get into it'} `niVnu Enu OdutitxdedxVye? `takaRshAsatxrX'. `nanage adaralilx Asakitxyilalx.' 
\alnum{c} (madayxda \vi) (\sA\ \eng{get into one's head}) talegeVru; talekeDisu; budidhxBarxmaNe uMTumADu. 
\alnum{d} hiDi; vashapaDisiko; sAvxdhiVnapaDisiko: \eng{what's got into you tonight} I rAtirx ninage Enu hiDidide, ninageVnAgide? 
\eanum
\numie
\numi{11} \eng{get off} 
\banum
\alnum{a} (vAhanadiMda) iLi; keLakikxLi. 
\alnum{b} (yAvudoV oMdu jAgadiMda) Ace hoVgu; horage nilulx: \eng{get off the grass} (hulilxna meVle naDeyabeVDa eMbathaRdalilx) hulilxniMdAce nililx. 
\alnum{c} (opapxMda, mAtu, \mo vugaLiMda) biDugaDe paDe; vimoVcane hoMdu; vimukatxnAgu. 
\alnum{d} (yAvudariMdaleV) pArAgu; tapipxsiko. 
\alnum{e} (parxyANa, kelasa, \mo vanunx) pArxraMBisu; shurumADu. 
\alnum{f} nidedx -- mADahatutx, mADalu pArxraMBisu. 
\alnum{g} nidedx hoVguvaMte, pArxraMBisuvaMte mADu. 
\alnum{h} (yAvudeV shikeSxyiMda) pArAgu; biDugaDe hoMdu. 
\alnum{i} (yAvudeV aparAdhakAkxgi) shikeSxyiMda saMpUNaRvAgi yA alapx shikeSxyoDane biDugaDe hoMdu: \eng{he had got off very well with a reprimand} avanu keVvala CiVmAriyoDane pUNaRvAgi biDugaDe paDeda. 
\alnum{j} (shikeSxyiMda) tapipxsu; pArumADu; biDugaDe dorakisu. 
\alnum{k} (kAgada \mo vanunx) ravAnisu. 
\hypertarget{get pagu12}{} 
\eanum
\numie
\numi{12} \eng{get on} 
\banum
\alnum{a} (kudure, seYkalulx, \mo vanunx) hatutx; Eru. 
\alnum{b} (uDupu \mo vanunx) hAkiko; dharisu; Erisu. 
\alnum{c} (veVgavanunx) paDe; hecicxsu. 
\alnum{d} muMduvari; muMdakekx calisu; hejejxhAku. 
\alnum{e} muMduvari; parxgatipaDe. 
\alnum{f} ELigehoMdu; aBivaqdidhx paDe: \eng{get on in the world} aishavxyaR, adhikAra gaLisu. 
\alnum{g} (yAvudAdarU riVtiyalilx) iru; Agu: \eng{tell me how you got on up there} alilx niVnu heVge idedx enunxvudanunx nanage heVLu. 
\alnum{h} (gADi, reYlu, basusx, \mo vanunx) hatutx; Eru; hoVgu. 
\hypertarget{get pagu13}{} 
\eanum
\numie
\num{13} \eng{go on one's feet to speak (in public)} (saBeyalilx mAtanADalu) edudx nilulx; meVleVLu. 
\num{14} \eng{get on one's legs to speak in public} = \hyperlink{get pagu13}{?pagu? \((13)\)}. 
\num{15} \eng{get on something etc. about (person)} (vayxkitxyobabxna meVle tapupx aparAdha horisuva) yAvudanAnxdarU kaMDuhiDi, kaMDuko. 
\num{16} \eng{get on or get out} kelasamADu, ilalxdidadxre naDe, horaTuhoVgu. 
\numi{17} \eng{get on with} 
\banum
\alnum{a} oDane kUDikoMDu, hoMdikoMDu -- hoVgu. 
\eanum
\numie
\num{18} \eng{get on without (something)} (AvashayxkavAdadudx ilalxdidadxrU) sAvarisikoMDu hoVgu. 
\numi{19} \eng{get out} 
\banum
\alnum{a} (\AmA) (vidhi rUpa) horaDu; tolagu. 
\alnum{b} (havAmAna \mo vu) Agu; uMTAgu; saMBavisu: \eng{the afternoon got out very well} madhAyxhanx havA tuMba utatxmavAyitu. 
\alnum{c} (obabxriMda samAcAra) horaDisu; horatege. 
\alnum{d} parxyatanxdiMda ucacxrisu. 
\alnum{e} parxkaTisu. 
\alnum{f} (kirxkeTf) (obabxnanunx)auTf mADu yA (obabxnu) auTAgu. 
\alnum{g} (samaseyx) biDisu. 
\alnum{h} mane biDu; maneyiMda horaDu, nigaRmisu. 
\alnum{i} vAhanadiMda iLi. 
\eanum
\numie
\numi{20} \eng{get out of hand} 
\banum
\alnum{a} keYmIri hoVgu; hatoVTi tapipxhoVgu. 
\alnum{b} (kelasa \mo vanunx) pUreYsibiDu; mADi mugisibiDu. 
\eanum
\numie
\num{21} \eng{get out of sight} kaNamxreyAgu; kANade hoVgu. 
\numi{22} \eng{get over} 
\banum
\alnum{a} (kaSaTxda kelasavanunx) konegANisu; kone muTiTxsu. 
\alnum{b} (kaSaTx \mo vanunx) dATu; kaSaTxdiMda pArAgu. 
\alnum{c} (rujuvAtu, vAda, \mo vanunx) tapepxMdu toVrisu; asamaMjasaveMdu, opipxgeyAgadeMdu -- toVrisu. 
\alnum{d} (roVgadiMda) pArAgu; guNahoMdu. 
\alnum{e} (AshacxyaRdiMda, digaBxrXmeya sithxtiyiMda) ececxtutxko. 
\alnum{f} (dAriyanunx) naDedu mugisu. 
\alnum{g} (\ashi) sikikxsiko; upAyadiMda balege keDavu. 
\alnum{h} (\ashi) upAyadiMda -- tapipxsiko, nuNuciko. 
\eanum
\numie
\numi{23} \eng{get round} 
\banum
\alnum{a} opipxsu; oDaMbaDisu. 
\alnum{b} (upAya mADi) tapipxsiko; keYyiMda nuNuciko. 
\alnum{c} (sudidxya \vi) = \hyperlink{get nuga1c}{?pagu? \((1c)\)}. 
\eanum
\numie
\numi{24} \eng{get through} 
\banum
\alnum{a} mugisu; samApitxgoLisu; kone muTiTxsu. 
\alnum{b} (masUde \mo vu shAsanasaBeyalilx) aMgiVkaqtavAgu; opipxge, aMgiVkAra -- paDe. 
\alnum{c} (haNavanunx) KacuRmADi mugisu. 
\alnum{d} (kAla \mo vanunx) kaLe; vayxya mADu. 
\alnum{e} (pariVkeSxyalilx) teVgaRDeyAgu. 
\alnum{f} (reVDiyoV yA dUravANiya mUlaka) saMpakiRsu. 
\eanum
\numie
\num{25} \eng{get through with} mADi mugisu; niBAyisu. 
\numi{26} \eng{get to} 
\banum
\alnum{a} (kelasa \mo vanunx) AraMBisu. 
\alnum{b} (kelasa \mo vanunx) mADuvudaralilx yashasivxyAgu. 
\alnum{c} talupu; seVru. 
\eanum
\numie
\numi{27} \eng{get together} 
\banum
\alnum{a} (vAda, yoVjaneya kAyARcaraNe, \mo vugaLalilx) kUDu; oTATxgu; oTuTxgUDu; oTuTx seVru. 
\alnum{b} (vAda, yoVjaneya kAyARcaraNe, \mo vugaLalilx) kUDisu; oTuTx seVrisu; oTuTxgUDisu. 
\alnum{c} (\ashi) vayxvasethxgoLisu; karxmadalilx saMyoVjisu. 
\eanum
\numie
\numi{28} \eng{get up} 
\banum
\alnum{a} (\kanmu\ hAsigeyiMda) ELu; ededxVLu. 
\alnum{b} (\kanmu\ kudureya meVle) Eru; hatutx. 
\alnum{c} (beMki, gALi, kaDalu) joVrAgu; birusAgu; raBasagoLuLx. 
\alnum{d} (beVTeya pArxNi) marebiTuTx horaDu; mareyiMdAce ODu. 
\alnum{e} (kirxkeTf ceMDu) birusiniMda puTaveVLu. 
\alnum{f} aNimADu; sajujxgoLisu; vayxvasethxgoLisu. 
\alnum{g} sidadhxmADu; EpARTumADu. 
\alnum{h} (baTeTx) toDalu sidadhxgoLisu. 
\alnum{i} ceMdagANisu. 
\alnum{j} (kUdalanunx, vayxkitxyanunx, nATakadalilx veVSavanunx, pusatxka mudarxNa, raTuTx, \mo vanunx) aMdagoLisu; cenAnxgi rUpisu: \eng{she was got up like a filmstar} avaLanunx sinimAtAreyaMte siMgarisalAgitutx. 
\alnum{k} meVlakekxtutx; ELisu; ebibxsu; edudx nilulxvaMte mADu. 
\alnum{l} uMTumADu: \eng{get up speed} veVgagoLisu; veVga hecicxsu. 
\alnum{m} (kaqtakavAda BAva, uderxVka) ebibxsu; parxcoVdisu. 
\alnum{n} (pariVkeSxya viSaya \mo vakekx) tayArAgu; sidadhxvAgu; AvashayxkavAda jAcnxna saMpAdisu; kaSaTxpaTuTx Odu; adhayxyana mADu: \eng{what subjects have you to get up for the exam?} pariVkeSxge niVnu yAva viSayagaLanunx OdabeVku? yAva viSayagaLige tayArAgabeVku? 
\eanum
\numie
\num{29} \eng{get upon =} \hyperlink{get pagu12}{?pagu? \((12)\)}. 
\num{30} \eng{get up to} (kiVTale \mo vugaLalilx) toDagu. 
\numi{31} \eng{get well} 
\banum
\alnum{a} aBivaqdidhxge baru. 
\alnum{b} AroVgayx sudhArisu; guNamuKanAgu. 
\eanum
\numie
\num{32} \eng{has got} hoMdiru; paDediru: \eng{have not got a penny} oMdu peninxyU ilalx. 
\num{33} \eng{has got to} = \hyperref{kandict_m.pdf}{M}{must(5)}{must}: \eng{It has got to be done} adanunx mADiyeV tiVrabeVku. 
\enum
\emng

\noindent
\gl{\nuga}
\bmng
% 
\hypertarget{getting nuga1}{} 
\bnum
\num{1} \eng{be getting on for} (\engit{or} \eng{to)} (oMdu vayasusx, saMKeyx, \mo vanunx) muTuTx; samIpisu: \eng{an over crowded population getting on to 90 million} toMbatutx miliyananxnunx muTuTxtitxruva daTaTx janasaMKeyx. 
\num{2} \eng{be getting on towards} = \hyperlink{getting nuga1}{?nuga? \((1)\)}. 
\hypertarget{get nuga3}{} 
\hyperdef{G}{get(1) nuga(3)}{} 
\numi{3} \eng{get across} 
\banum
\alnum{a} (\AmA) pariNAmakAriyAgiru; parxBAvabiVru. 
\alnum{b} pariNAmakAriyanAnxgi mADu; parxBAvakAriyAgisu. 
\alnum{c} samamxtavAgu; opipxgeyAgu. 
\alnum{d} opupxvaMtAgisu; smamxtavAguvaMte mADu. 
\alnum{e} (\ashi) reVgisu; kirikiriyuMTumADu. 
\alnum{f} athaRvAgu yA athaRvAgisu: \eng{I spoke slowly, but my meaning did not get across} nAnu nidhAnavAgi mAtADide, AdarU nananx athaR shorxVtaqgaLige tiLiyalilalx, nananx udedxVsha athaRvAgalilalx. 
\eanum
\numie
\numi{4} \eng{get around} (\ame) 
\banum
\alnum{a} UrUru aleyutitxru; oMdu kaDeyiMda inonxMdu kaDege sututxtitxru. 
\alnum{b} gotAtxgu; parxcalitavAgu; tiLi: \eng{sooner or later everybody's business gets around} ivatatxlalx nALe parxtiyobabxra vayxvahAravU gotAtxgutatxde. 
\alnum{c} (sudidxya \vi) (\kanmu\ bAyiMda bAyige) haraDu; parxcAravAgu. 
\eanum
\numie
\numi{5} \eng{get at} 
\banum
\alnum{a} muTuTx; seVru; talupu. 
\alnum{b} eTukisiko (rUpa saha). 
\alnum{c} (vicArisi) kaMDuhiDi. 
\alnum{d} (\ashi) hasatxkeSxVpa naDesu; nAyxyavirudadhxvAgi parxvatiRsu. 
\alnum{e} (\ashi) laMcakoDu. 
\alnum{f} (\ashi) KaMDisu: \eng{the author's burning anxiety to get at capital} baMDavALashAhiyanunx KaMDisabeVkeMba garxMthakataRna utakxTeVceCx. 
\alnum{g} (\ashi) kucoVdayx mADu; geVlimADu: \eng{who are you getting at?} yAranunx kucoVdayx mADutitxdidxVye? (aneVka veVLe) yAranunx vaMcisutitxdidxVye? (hAge vaMcisalu sAdhayxvilalx eMba athaRdalilx). 
\alnum{h} (\AmA) iMgitakoDu; aBipArxya sUcisu: \eng{what are you getting at?} ninanx iMgitavAdarU Enu? 
\eanum
\numie
\numi{6} \eng{get away with it} (\AmA) 
\banum
\alnum{a} (kelasada \vi) (sikikxbiVLade yA kaSaTxnaSaTxgaLige oLagAgade) mADuva parxyatanxdalilx jayashiVlanAgu, yashasivxyAgu. 
\alnum{b} (kole \mo vugaLa \vi) parxtiVkAra, daMDa, shikeSx -- tapipxsiko. 
\alnum{c} hedarade, heVsade, dhASaTxyXRdiMda vatiRsu, naDeduko, niBAyisu, kelasa mADu, kAyaRnivaRhisu. 
\eanum
\numie
\numi{7} \eng{get back at} 
\banum
\alnum{a} (\AmA) (obabxna meVle) (mADidadxkekx, Agidadxkekx) parxtimADu; muyiyx tiVrisu. 
\alnum{b} (\AmA) (obabxna meVle) parxtAyxroVpa, parxtAyxpAdane horisu. 
\eanum
\numie
\num{8} \eng{get back some of one's own} (\ashi) seVDu, muyiyx tiVrisiko; parxtiVkAra mADu; takakx shAsitx mADu. 
\hypertarget{get nuga9}{} 
\num{9} \eng{get better} AroVgayx sudhArisu; kAyileyiMda guNamuKanAgu. 
\numi{10} \eng{get by} (\AmA) 
\banum
\alnum{a} (sariyAgi) sAkAguvaSiTxru. 
\alnum{b} (yAvudeV ilalxde hoVdarU, mADabeVkAdadxnunx) nivaRhisu; niBAyisu. 
\hypertarget{get nuga11}{} 
\eanum
\numie
\num{11} \eng{get by heart} kaMThapATha mADu; bAyipATha mADu; kali; gaTiTxmADu. 
\num{12} \eng{get by rote} = \hyperlink{get nuga11}{?nuga? \((11)\)}. 
\num{13} \eng{get his} (\engit{or} \eng{theirs)} (\ashi) koleyAgu; kolegiVDAgu: \eng{He'll get his one of these days} avanu inunx kelaveV divasagaLalilx kolegiVDAgutAtxne. 
\num{14} \eng{get it} beYguLa tinunx; shikeSxge, AkeSxVpaNege guriyAgu. 
\num{15} \eng{get it into one's head} (manasisxnalilx) aMduko; aMdukoMDubiDu; BAvisu. 
\hypertarget{get nuga16}{} 
\num{16} \eng{get knowledge of} (oMdu viSayada bagegx) sudidx keVLu; suLivupaDe; gALi vataRmAna keVLu. 
\num{17} \eng{get off with} (\AmA) virudadhxliMgada vayxkitxyoDane senxVhadiMda yA parxNayadiMda vatiRsu. 
\num{18} \eng{get on one's nerves} taleciTuTx hiDisu; (obabxnanunx) siDimiDigoLisu. 
\numi{19} \eng{get on to} 
\banum
\alnum{a} (\AmA) athaRmADiko; tiLiduko; athaRvanunx hiDi, garxhisu: \eng{he soon got on to the racket they were working} avaru naDesutitxdadx haMcikeyanunx avanu beVga tiLidukoMDa. 
\alnum{b} (\AmA) (vayxkitxyanunx \udA\ TeliphoVnf mUlaka) saMpakiRsu: \eng{if you are not satisfied with us, get on to the manager} nAvu nimage hiDisadidadxre, nimage taqpitx niVDadidadxre mAyxneVjaranunx saMpakiRsi. 
\eanum
\numie
\num{20} \eng{get (one) with child} basirumADu. 
\num{21} \eng{get one's hand in} (oMdu kelasadalilx) keY kudurisiko; paLagu: \eng{you won't find it difficult, once you get your hand in it} keYkuduridare tiVritu, ninageVnU kaSaTxvAguvudilalx. 
\numi{22} \eng{get out of} 
\banum
\alnum{a} (aBAyxsa \mo vanunx) karxmeVNa biTuTxbiDu. 
\alnum{b} (mADade) tapipxsiko. 
\alnum{c} (obabxniMda) haNa, duDuDx -- kiVLu, vasUlimADu. 
\eanum
\numie
\num{23} \eng{get out of one's depth} tananx ALa mIri hoVgu; tananx keYlAgada kelasakekx keYhAku. 
\num{24} \eng{get over} = \hyperlink{get nuga3}{?nuga? \((3)\)}. 
\num{25} \eng{get person or thing on the brain} oMdeV vayxkitx yA viSayavanunx kuritu yAvAgalU ciMtisu. 
\num{26} \eng{get quit of} = \hyperlink{get nuga27}{?nuga? \((27)\)}. 
\hypertarget{get nuga27}{} 
\num{27} \eng{get rid of} tapipxsiko; tolagisiko; pArAgu. 
\num{28} \eng{get round to} (yAvudanenxV mADalu) kAla, shakitx, yA olavanunx paDeduko. 
\hypertarget{get nuga29}{} 
\num{29} \eng{get the advantage of a person} (obabxnanunx) mIrisu; (obabxnige) meVlugeY Agu. 
\num{30} \eng{get the best of it} gelulx; jayashiVlanAgu. 
\num{31} \eng{get the better of a person} = \hyperlink{get nuga29}{?nuga? \((29)\)}. 
\num{32} \eng{get the} \hyperref{kandict_b.pdf}{B}{boot(1) nuga(4)}{$^1$boot}. 
\num{33} \eng{get the mitten} kelasadiMda tegeduhAkalapxDu. 
\numi{34} \eng{get there} (\ashi) 
\banum
\alnum{a} yashasivxyAgu; jayagaLisu. 
\alnum{b} athaR, sUcane -- garxhisu; iMgita tiLiduko. 
\eanum
\numie
\num{35} \eng{get the sack} kelasa kaLeduko; kelasadiMda vajA Agu. 
\num{36} \eng{get the start of a person} = \hyperlink{get nuga29}{?nuga? \((29)\)}. 
\num{37} \eng{get the sun of a person} = \hyperlink{get nuga29}{?nuga? \((29)\)}. 
\num{38} \eng{get the upper hand of a person} = \hyperlink{get nuga29}{?nuga? \((29)\)}. 
\num{39} \eng{get the wind of a person} = \hyperlink{get nuga29}{?nuga? \((29)\)}. 
\num{40} \eng{get the wind up} (\ashi) BayapaDu; hedaru. 
\num{41} \eng{get to do} (\ame) mADi mugisu; mADuvudaralilx yashasivxyAgu. 
\numi{42} \eng{get together} 
\banum
\alnum{a} saMgarxhisu; saMgarxhavAgu. 
\alnum{b} caceRyalilx, yoVjane \mo vanunx IDeVrisuvalilx -- oMdAgu. 
\alnum{c} (\ashi) karxmavAgiDu; suvayxvasethxgoLisu. 
\eanum
\numie
\num{43} \eng{get under} (beMki) Arisu; hatoVTige taru; aDagisu. 
\numi{44} \eng{get under way} 
\banum
\alnum{a} horaDuvaMte, calisuvaMte mADu: \eng{get ship under way} haDaganunx cAlane mADu. 
\alnum{b} pArxraMBavAgu. 
\eanum
\numie
\num{45} \eng{get used to it} adakekx hoMdiko; adanunx rUDhisiko. 
\num{46} \eng{get well} = \hyperlink{get nuga9}{?nuga? \((9)\)}. 
\num{47} \eng{get wind} (samAcAra \mo vu) haraDu; parxsAravAgu; elalxrigU gotAtxgu. 
\num{48} \eng{get wind of} = \hyperlink{get nuga16}{?nuga? \((16)\)}. 
\num{49} \eng{get wise to} arivAgu; manavarikeyAgu. 
\num{50} \eng{got out of} \hyperref{kandict_b.pdf}{B}{bed(2) nuga(3)}{$^2$bed on wrong side.} 
\num{51} \eng{have got it bad} (\engit{or} \eng{badly)} (\ashi) moVhaparavashanAgiru; moVhakekx bididxru; giVLu hatitxru; hucucx hiDidiru. 
\numi{52} \eng{to get another's back up} 
\banum
\alnum{a} reVgisu; reVguvaMte mADu. 
\alnum{b} haTa, muSakxra hiDiyuvaMte mADu. 
\eanum
\numie
\numi{53} \eng{to get one's back up} 
\banum
\alnum{a} reVgi biVLu. 
\alnum{b} haTa, muSakxra -- hiDi. 
\eanum
\numie
\enum
\emng
\eentry

\bentry
\word[get(2)]{get}
\pron{geTf}
\gl{\nA}
\bmng
\bnum
\numi{1} (\kanmu\ beVTeya BASeyalilx, pArxNigaLa \vi) 
\banum
\alnum{a} Inuvudu; saMtAnoVtapxtitx. 
\alnum{b} saMtAna; saMtati. 
\eanum
\numie
\num{2} (birxTanf) (\ashi) saMpAdane; AdAya; varamAna; saMbaLa, lABa, \mo vu: \eng{what is your week's get?} vAradalilx ninanx saMpAdane eSuTx? 
\num{3} (\ashi) daDaDx; muTAThxLa. 
\enum
\emng
\eentry

\bentry
\word{geta}
\pron{geVTa}
\gl{\nA}
\bmng
 (\bava) geVTa; kAlina hebebxraLiMda beVre beraLugaLige paTiTx iruva, japAnina, marada pAdarakeSx. 
\emng
\eentry

\bentry
\word{get-at-able}
\pron{geTfAYxTabflf}
\gl{\gu}
\bmng
 sAdhayx; gArxhayx; sikakxbalalx; siguva; paDeyabahudAda. 
\emng
\eentry

\bentry
\word{get-away}
\pron{geTfaveV}
\gl{\nA}
\bmng
 (\kanmu\ aparAdha mADida naMtara) tapipxsikoLuLxvudu; palAyana. 
\emng
\eentry

\bentry
\word{get-out}
\pron{geTfauTf}
\gl{\nA}
\bmng
 (\AmA) tapipxsikoLuLxva dAri; palAyana mAgaR. 
\emng

\noindent
\gl{\nuga}
\bmng
 \eng{for} (\engit{or} \eng{like) all get-out} (\ashi) atayxMta birusAgi; bahaLa joVrAgi. 
\emng
\eentry

\bentry
\word{get-rich-quick}
\pron{geTfricfkivxkf}
\gl{\gu}
\bmng
diDhiVrf shirxVmaMtike parxyatanxda; beVga duDuDx saMpAdisalu parxyatanx mADuva. 
\emng
\eentry

\bentry
\word{gettable}
\pron{geTabflf}
\gl{\gu}
\bmng
 paDeyalu sAdhayxvAda; sikukxvaMtha; sAdhisabahudAda; sAdhisuvaMtha. 
\emng
\eentry

\bentry
\word[getter(1)]{getter}
\pron{geTarf}
\gl{\nA}
\bmng
\bnum
\num{1} gArxhaka; parxtigarxhi; paDevAta; (yAvudanenxV) paDeyuva vayxkitx. 
\num{2} paDeyuva vasutx. 
\num{3} (\Bwvi) geTarf; KAli mADida nALa \mo vugaLalilx uLidirabahudAda anilavanunx tegeduhAkalu baLasuva, rAsAyanikavAgi sakirxyavAda vasutx (\udA\ lwhika beVriyaM). 
\enum
\emng
\eentry

\bentry
\word[getter(2)]{getter}
\pron{geTarf}
\gl{\sakirx}
\bmng
 geTaranunx baLasi anilavanunx tege yA nALavanunx KAli mADu. 
\emng
\eentry

\bentry
\word{get-together}
\pron{geTfTugedarf}
\gl{\nA}
\bmng
 (\AmA) saMtoVSakUTa; senxVha samemxVLana. 
\emng
\eentry

\bentry
\word{get-up}
\pron{geTfapf}
\gl{\nA}
\bmng
\bnum
\num{1} (uDupu, sajujx, \mo vugaLa) aMda; ceMda; riVti. 
\num{2} (pusatxka \mo vugaLa) tayArikeya riVti; hora rUpa, meY. 
\enum
\emng
\eentry

\bentry
\word{geum}
\pron{jiVamf}
\gl{\nA}
\bmng
 jiVyaM kulada giDa. 
\emng
\eentry

\bentry
\wordnospeech{GeV}{GeV}
\pron{?}
\gl{\saMkiSx}
\bmng
 \eng{gigaelectronvolt ($10^9$ electron volts)}. 
\emng
\eentry

\bentry
\word{gewgaw}
\pron{gUyxgA}
\gl{\nA}
\bmng
\bnum
\num{1} thaLakina ATada sAmAnu yA oDave; jujubi ATada sAmAnu. 
\num{2} kelasakekx bArada, bariya Dwlina alapxvasutx. 
\enum
\emng
\eentry

\bentry
\word[gey(1)]{gey}
\pron{geV}
\gl{\gu}
\bmng
(sAkxTelxMDf \parx) sAkaSuTx; takakxSuTx; takakxmaTiTxna. 
\emng

\noindent
\gl{\pagu}
\bmng
 \eng{gey and} bahaLa: \eng{gay and pretty} bahaLa suMdara. 
\emng
\eentry

\bentry
\word[gey(2)]{gey}
\pron{geV}
\gl{\kirxvi}
\bmng
 (sAkxTelxMDf \parx) hecAcxgi; bahaLa; bahumaTiTxge; visheVSavAgi. 
\emng
\eentry

\bentry
\word{geyser}
\pron{geV(geY)sarf, giVsaZrf}
\gl{\nA}
\bmng
\bnum
\num{1} (bisi niVrina dhAreyanunx AgAga meVle cimumxva) bisi niVrina -- bugegx, cilume, UTe. 
\num{2} (\ucAcx\ giVsaZrf) giVsaZrf; sAnxnakekx, keYkAlu muKa toLedukoLaLxlu beVga niVru kAyisuva salakaraNe. 
\enum
\emng
\eentry

\bentry
\wordnospeech{GG}{GG}
\pron{?}
\gl{\saMkiSx}
\bmng
 \eng{Governor-General.} 
\emng
\eentry

\bentry
\word[Ghanaian(1)]{Ghanaian}
\pron{gAneV(niV)anf}
\gl{\nA}
\bmng
 (pashicxma Aphirxkada) GAnA deVshiVya; GAnA deVshadalilx huTiTxdavanu yA vAsisuvavanu. 
\emng
\eentry

\bentry
\word[Ghanaian(2)]{Ghanaian}
\pron{gAneV(niV)anf}
\gl{\gu}
\bmng
 GAnAdeVshada. 
\emng
\eentry

\bentry
\word{gharial}
\pron{geVrialf}
\gl{\nA}
\bmng
 \eng{gavial}na rUpAMtara. 
\emng
\eentry

\bentry
\word{gharry}
\pron{gAYxri}
\gl{\nA}
\bmng
(bADigeya) kuduregADi; jaTakA. 
\emng
\eentry

\bentry
\word{ghastlily}
\pron{gAsfTxlili}
\gl{\kirxvi}
\bmng
\bnum
\num{1} BayaMkaravAgi; GoVravAgi; BiVkaravAgi; vikAravAgi; rwdarxvAgi. 
\num{2} heNada hAge; heNadaMte. 
\num{3} maMku kavidu; nirutAsxhakaravAgi; kaLeguMdi. 
\enum
\emng
\eentry

\bentry
\word{ghastliness}
\pron{gAsfTxlinisf}
\gl{\nA}
\bmng
\bnum
\num{1} BayaMkarate; rwdarxte; karALatavx; BiVkarate; vikArate. 
\num{2} vivaNaRvAgiruvike; biLicikoMDiruvudu; heNada hAgiruvike. 
\num{3} viSaNaNxte; kaLeguMdiruvike; maMku kavidiruvike. 
\enum
\emng
\eentry

\bentry
\word[ghastly(1)]{ghastly}
\pron{gAsfTxli}
\gl{\gu}
\bmng
\bnum
\num{1} BayaMkara; rwdarx; karALa; BiVkara; GoVra; vikAra. 
\num{2} (\AmA) AkeSxVpaNiVya. 
\num{3} (\AmA) ahitavAda. 
\num{4} heNamoVreya; heNadaMtha; perxVtakaLeya. 
\num{5} vivaNaR; biLicikoMDa; kaLe ilalxda. 
\num{6} GoVra vaNaRda. 
\num{7} (muguLanxge \mo vu) saMkaTada; balAtAkxrada; balavaMtada; balavaMtadiMda horaDisida. 
\enum
\emng
\eentry

\bentry
\word[ghastly(2)]{ghastly}
\pron{gAsfTxli}
\gl{\kirxvi}
\bmng
\bnum
\num{1} BayaMkaravAgi: \eng{ghastly pale} BayaMkaravAgi vivaNaR hoMdida. 
\num{2} heNadaMte; perxVtakaLeyiMda. 
\enum
\emng
\eentry

\bentry
\word{ghat}
\pron{gATf}
\gl{\nA}
\bmng
\bnum
\num{1} (BArata) GaTaTx: \eng{Eastern or Western ghats} pUvaR yA pashicxma GaTaTxgaLu; dakiSxNa BAratada pUvaR matutx pashicxmadalilx udadxkUkx habibxruva eraDu beTaTxda sAlugaLu. 
\num{2} GaTaTx; GATi; kaNive. 
\num{3} sAnxnaGaTaTx; (nadige iLiyuva) soVpAna paMkitx; nadiyiMda meVlakekx hatutxva sathxLa, GaTaTx. 
\enum
\emng

\noindent
\gl{\pagu}
\bmng
 \eng{burning-ghat} (nadiya tiVrada) shamxshAnaGaTaTx; (hoLeya daDada) suDugADu GaTaTx. 
\emng
\eentry

\bentry
\word{ghaut}
\pron{gATf}
\gl{\nA}
\bmng
  = \hyperlink{ghat}{ghat}. 
\emng
\eentry

\bentry
\word{ghazi}
\pron{gAsiZ}
\gl{\nA}
\expl{(\bava\ \eng{ghazis}).}
\bmng
\bnum
\num{1} mahamamxdiVya dhamARMdha. 
\num{2} (\kanmu\ isAlxminalilx naMbikeyilalxdavara virudadhx hoVrADuva) musilxM yoVdha, seYnika. 
\num{3} (tukiRyalilx) gAji; mayARdeya oMdu birudu. 
\enum
\emng
\eentry

\bentry
\word{ghee}
\pron{giV}
\gl{\nA}
\bmng
 (ememxya yA hasuvina) tupapx. 
\emng
\eentry

\bentry
\word{gherao}
\pron{gerw}
\gl{\nA}
\expl{(\bava\ \eng{gheraos}).}
\bmng
 (BArata matutx pAkisAtxna) GeVrAvf; kelasagAraru tamamx beVDikegaLu IDeVruvavarege tamamx dhaNiyanunx yA mAyxneVjaranunx kAKARneyiMda horage biDade kADuvudu. 
\emng
\eentry

\bentry
\word{gherkin}
\pron{gakiRnf}
\gl{\nA}
\bmng
 gakiRnf; (upipxnakAyige baLasuva) eLeya yA saNaNx jAtiya swteyakAyi; mekekxya kAyi.  \imglink{gherkinfigure}{\raisebox{-0.15cm}[0pt][0pt]{\pdfimage width 0.5cm height 0.7cm{G_Pictures/gherkin.jpg}}} 
\emng
\eentry

\bentry
\word[ghetto(1)]{ghetto}
\pron{geToV}
\gl{\nA}
\expl{(\bava\ \eng{ghettos}).}
\bmng
(\ca) 
\bnum
\num{1} geToTxV; yehUdayx keVri, haTiTx; nagaradalilx yehUdayxra vasati, vAsasathxLa. 
\num{2} keVri; alapxsaMKAyxtaru vAsisuva nagarada BAga, \kanmu\ koLace parxdeVsha. 
\num{3} parxteyxVkita vagaR yA parxdeVsha; parxteyxVkisida, beVreyAgiTaTx -- guMpu yA parxdeVsha. 
\enum
\emng
\eentry

\bentry
\word[ghetto(2)]{ghetto}
\pron{geToV}
\gl{\sakirx}
\bmng
 (janaranunx) koLace parxdeVshadalilx yA parxteyxVkavAda parxdeVshadalilx -- iDu, irisu, nelegoLisu. 
\emng
\eentry

\bentry
\wordnospeech{ghi}{ghi}
\pron{?}
\gl{\nA}
\bmng
  = \hyperlink{ghee}{ghee}. 
\emng
\eentry

\bentry
\word[Ghibelline(1)]{Ghibelline}
\pron{gibaleYnf}
\gl{\gu}
\bmng
 gibaleYnf pakaSxda; (gevxlfphx pakaSxvanunx viroVdhisuva) madhayxyugada iTaliya sAmArxjayxshAhi pakaSxkekx seVrida. 
\emng
\eentry

\bentry
\word[Ghibelline(2)]{Ghibelline}
\pron{gibaleYnf}
\gl{\nA}
\bmng
(iTaliya) gibaleYnf eMba sAmArxjayxshAhi pakaSxkekx seVridavanu. 
\emng
\eentry

\bentry
\word{Ghibellinism}
\pron{gibalinisaZmf}
\gl{\nA}
\bmng
sAmArxjayxshAhi tatatxvX; gibaleYnf -- vAda, dhoVraNe, tatatxvX, sUtarx. 
\emng
\eentry

\bentry
\word{ghillie}
\pron{gili}
\gl{\nA}
\bmng
 \eng{gillie}eMba padada rUpAMtara. 
\emng
\eentry

\bentry
\word[ghost(1)]{ghost}
\pron{goVsfTx}
\gl{\nA}
\bmng
\bnum
\num{1} jiVva (tatatxvX); pArxNa; Atamx; pArxNavAyu. 
\hypertarget{ghost(1)2}{} 
\num{2} pavitArxtamx; kerxYsatxra tirxmUtiR kalapxneyalilx mUraneyadu. 
\num{3} (heVDiVsf eMba maqtaloVka \mo vugaLalilxruva) satatxvana -- Atamx, perxVta. 
\num{4} perxVta; BUta; devavx; badukiruvavarige kANisuvudenanxlAda satatxvana rUpa, AkAra. 
\num{5} heNadatatx; jiVvacaCxva; narapeVta(la). 
\num{6} oMdiSuTx; cUru; lavaleVsha: \eng{not the ghost of a chance} (Aguva) saMBava oMdiSUTx ilalx; (Aguva) lakaSxNaveV ilalx. 
\num{7} (\daqvi) perxVta; masUrada doVSadiMdAgi dUradashaRkada daqshayxdalilx kANabaruva parxkAshamAnavAda boTuTx yA divxtiVyaka biMba. 
\hypertarget{ghost(1)8}{} 
\num{8} perxVtaleVKaka; BUtabarahagAra; tananx yajamAnana KAyxtigAgi duDiyuva barahagAra, sAhiti. 
\num{9} CAye; neraLu; perxVta; asapxSaTx Akaqti yA hoVlike: \eng{he is the ghost of his former self} Iga avanu tananx pUvaRrUpada perxVta mAtarx. 
\num{10} (\bava dalilx) padavanunx pUNaRgoLisadaMte, obobxbabxnU oMdoMdu akaSxravanunx seVrisi padavanunx racisuva oMdu terana ATa. 
\enum
\emng

\noindent
\gl{\pagu}
\bmng
 \eng{the Holy Ghost} = \hyperlink{ghost(1)2}{$^1$ghost (2)}. 
\emng

\noindent
\gl{\nuga}
\bmng
\hypertarget{ghost nuga1}{} 
\bnum
\numi{1} \eng{give up the ghost} 
\banum
\alnum{a} sAyu. 
\alnum{b} (yAvudaradeV bagegx) Aseyanunx, naMbikeyanunx biDu. 
\eanum
\numie
\num{2} \eng{lay ghost} perxVtavanunx aDagisu; perxVta kANadaMte mADu. 
\num{3} \eng{raise ghost} perxVtavanunx ebibxsu; perxVta kANuvaMte mADu. 
\num{4} \eng{the ghost walks} (nATaka) (\ashi) saMbaLa koDutAtxre, sadayxdalilx koDalAgutatxde. 
\hyperdef{G}{ghost(1) nuga(5)}{} 
\num{5} \eng{yield up the ghost} (\pArxparx) = \hyperlink{ghost nuga1}{?nuga? \((1)\)}. 
\enum
\emng
\eentry

\bentry
\word[ghost(2)]{ghost}
\pron{goVsfTx}
\gl{\sakirx}
\bmng
\bnum
\num{1} (oMdu sathxLakekx perxVtadaMte) padeV padeV hoVgutitxru. 
\num{2} perxVtadaMte -- kADu, benunxhatutx. 
\enum
\emng

\noindent
\gl{\akirx}
\bmng
\bnum
\num{1} perxVtadaMte suLidADu. 
\num{2} perxVtaleVKanadalilx toDagu; (obabxnigAgi) perxtaleVKaka Agu. 
\enum
\emng
\eentry

\bentry
\word{ghosthood}
\pron{goVsfTxhuDf}
\gl{\nA}
\bmng
 perxVtatavx; devavxvAgiruvike; perxVta sithxti. 
\emng
\eentry

\bentry
\word{ghostlike}
\pron{goVsfTxleYkf}
\gl{\gu}
\bmng
 devavxdaMtha; perxVtarUpada. 
\emng
\eentry

\bentry
\word{gostliness}
\pron{goVsfTxlinisf}
\gl{\nA}
\bmng
\bnum
\num{1} perxVtatavx; perxVtasithxti. 
\num{2} perxVtalakaSxNa; perxVtakaLe; perxVtadaMtiruvike. 
\num{3} perxVtagaLiMda tuMbiruvike. 
\enum
\emng
\eentry

\bentry
\word{ghostly}
\pron{goVsfTxli}
\gl{\gu}
\bmng
\bnum
\num{1} (\pArxparx) AdhAyxtimxka. 
\num{2} adeYhika; amUtaR. 
\num{3} kerxYsatxmatiVya; kerxYsatxmaThiVya; kerxYsatxmataviSayaka. 
\num{4} devavxdaMtha; pishAcadaMtha; perxVtada hAgiruva; gALirUpada. 
\num{5} neraLinaMtha; CAyeyaMtha; asapxSATxkaqtiya. 
\enum
\emng

\noindent
\gl{\pagu}
\bmng
\hypertarget{ghostly pagu1}{} 
\bnum
\num{1} \eng{ghostly adviser} pApa niveVdane keVLuva pAdirx. 
\hypertarget{ghostly pagu2}{} 
\num{2} \eng{ghostly comfort} (pApa niveVdane keVLuva) pAdirxya -- samAdhAna vacana, hitavacana. 
\num{3} \eng{ghostly counsel} = \hyperlink{ghostly pagu2}{?pagu? \((2)\)}. 
\num{4} \eng{ghostly director} = \hyperlink{ghostly pagu1}{?pagu? \((1)\)}. 
\num{5} \eng{ghostly father} = \hyperlink{ghostly pagu1}{?pagu? \((1)\)}. 
\numi{6} \eng{ghostly weapons} (kerxYsatx) 
\banum
\alnum{a} dhAmiRka, matatatavxda vAdagaLu. 
\alnum{b} caciRna daMDanegaLu. 
\eanum
\numie
\num{7} \eng{our ghostly enemy} seYtAna. 
\enum
\emng
\eentry

\bentry
\wordnospeech{ghost moth}{ghost moth}
\pron{?}
\gl{\nA}
\bmng
 perxVtapataMga; BUtaciTeTx; rAtirx hotutx hArADuva biLiciTeTx. 
\emng
\eentry

\bentry
\wordnospeech{ghost town}{ghost town}
\pron{?}
\gl{\nA}
\bmng
perxVtanagara; perxVtapaTaTxNa; devavxdUru; atayxlapx maMdi vAsavAgiruva yA yArU vAsavAgilalxda Uru. 
\emng
\eentry

\bentry
\word{ghost-word}
\pron{goVsfTxvaDfR}
\gl{\nA}
\bmng
 BArxMtipada; UhApada; perxVtapada; BASeyalilxlalxde Adare mudarxNakArana tapipxniMda yA ucAcxraNeya tapipxniMda rUDhige baMdiruva pada. \udA \eng{celt, flyfot, tweed.} 
\emng
\eentry

\bentry
\wordnospeech{ghost writer}{ghost writer}
\pron{?}
\gl{\nA}
\bmng
  = \hyperlink{ghost(1)8}{$^1$ghost (8)}. 
\emng
\eentry

\bentry
\word{ghoul}
\pron{gU(sw)lf}
\gl{\nA}
\bmng
\bnum
\num{1} (musilxM mUDhanaMbikeyalilx) shavaBakaSxka pishAci; koMti; heNadini; heNa tinunxva pishAci; piritini; maruLu. 
\num{2} (\rUpa) pishAci; sAvu \mo vugaLalilx atiyAda, asahajavAda AsakitxyuLaLx vayxkitx. 
\enum
\emng
\eentry

\bentry
\word{ghoulish}
\pron{gU(gw)liSf}
\gl{\gu}
\bmng
 perxVtaBakaSxka pishAciyaMtha; heNa tinunxva pishAciya rUpada. 
\emng
\eentry

\bentry
\word{ghoulishly}
\pron{gU(gw)liSfli}
\gl{\kirxvi}
\bmng
 perxVtaBakaSxka pishAciyaMte; heNa tinunxva pishAciya hAge; piritiniyaMte. 
\emng
\eentry

\bentry
\wordnospeech{GHQ}{GHQ}
\pron{?}
\gl{\saMkiSx}
\bmng
 \eng{General Headquarters.} 
\emng
\eentry

\bentry
\word{ghyll}
\pron{gilf}
\gl{\nA}
\bmng
  = \hyperlink{gill(3)}{$^3$gill}. 
\emng
\eentry

\bentry
\word[GI(1)]{GI}
\pron{jiVai}
\gl{\gu}
\bmng
 amerikada sashasatxrX seYnikarigAgi iruva yA avarige saMbaMdhisida. 
\emng
\eentry

\bentry
\word[GI(2)]{GI}
\pron{jiVai}
\gl{\nA}
\bmng
 amerikada sashasatxrX daLada seYnika. 
\emng
\eentry

\bentry
\wordf{giallo antico}
\pron{jAloV AMTiVkoV}
\gl{\nA}
\bmng
 \It\ haLadi hAlugalulx; haLadi amaqtashile; iTaliya jiVNARvasheVSagaLalilx doreyuva, alaMkArada, kaDu haLadi baNaNxda amaqtashile. 
\emng
\eentry

\bentry
\word[giant(1)]{giant}
\pron{jeYaMTf}
\gl{\nA}
\bmng
\bnum
\num{1} deYtayx; rAkaSxsa; asura; amAnuSa deVhaparxmANavuLaLx, purANa parxsidadhx manuSayxrUpi vayxkitx. 
\num{2} (\girxVpu) asura; deYtayx; BUmayxdhideVvate, `geVya' yA savxgARdhideVvate `yureVnesf' yA narakAdhideVvate TATaRrasfgU huTiTxda makakxLalilx deVvategaLoDane yudadhx mADidavanu. 
\num{3} mahAbali; mahAshakitxyuLaLxdudx. 
\num{4} deYtayx; BAri; baqhadedxVhi; asAdhAraNa etatxrada manuSayx, pArxNi yA sasayx. 
\num{5} rAkaSxsa; deYtayx; (mahA) parxcaMDa; asAmAnayx sAmathayxR, dheYyaR, shakitx, \mo vugaLuLaLx vayxkitx: \eng{there were giants in those days} A dinagaLalilx namagiMta mahAparxcaMDaridadxru; namamx pUviRkaru namagiMta bahaLa utatxmarAgidadxru. 
\num{6} deYtayx; rAkaSxsa; asAdhAraNa shakitxyuLaLxdudx. 
\enum
\emng
\eentry

\bentry
\word[giant(2)]{giant}
\pron{jeYaMTf}
\gl{\gu}
\bmng
\bnum
\num{1} BUtAkArada; vipariVta doDaDx gAtarxda. 
\num{2} rAkaSxsiV; asAdhAraNa balada; deYtayxshakitxya. 
\num{3} (sasayxda \vi) deYtayx; asAdhAraNa gAtarxda; vipariVta doDaDx. 
\enum
\emng
\eentry

\bentry
\wordnospeech{giant book}{giant book}
\pron{?}
\gl{\nA}
\bmng
 daDipusatxka; jAhiVrAtigAgi baLasuva, deYtAyxkArada raTiTxniMda mADida pusatxkada AkAra. 
\emng
\eentry

\bentry
\wordnospeech{giant cement}{giant cement}
\pron{?}
\gl{\nA}
\bmng
 deYtayx simeMTu; balu gaTiTxyAda simeMTu. 
\emng
\eentry

\bentry
\word{giantess}
\pron{jeYaMTe(Ti)sf}
\gl{\nA}
\bmng
 rAkaSxsi; deYtAyxkArada heMgasu. 
\emng
\eentry

\bentry
\word{giantism}
\pron{jeYaMTisaZmf}
\gl{\nA}
\bmng
\bnum
\num{1} rAkaSxsatana; rAkaSxsa parxkaqti; rAkaSxsiV parxvaqtitx; deYtAyxMsha. 
\num{2}  = \hyperlink{gigantism(1)}{gigantism (1)}. 
\enum
\emng
\eentry

\bentry
\word{giant-killer}
\pron{jeYaMTfkilarf}
\gl{\nA}
\bmng
 deYtayxvijayi; deYtayxBaMjaka; noVDuvudakekx tanagiMta hecucx parxbalavAgiruvavananunx soVlisuvava. 
\emng
\eentry

\bentry
\word{giant-like}
\pron{jeYaMTfleYkf}
\gl{\gu}
\bmng
 deYtayx; rAkaSxsa; rAkaSxsana yA rAkaSxsanaMtha; rAkaSxsAkArada. 
\emng
\eentry

\bentry
\word{giant-powder}
\pron{jeYaMTfpwDarf}
\gl{\nA}
\bmng
 rAkaSxsa siDimadudx; oMdu bageya DeYnameYTu. 
\emng
\eentry

\bentry
\word{giant's-stride}
\pron{jeYaMTfsx seTxrXYDf}
\gl{\nA}
\bmng
  = \hyperlink{giant-stride}{giant-stride}. 
\emng
\eentry

\bentry
\word{giant-stride}
\pron{jeYaMTfseTxrXYDf}
\gl{\nA}
\bmng
 dApugaNe; deYtayx naDeya gaNe; dApugAlu gaNe; gaNeya tirugutaleyiMda iLibiTaTx hagagxgaLa sahAyadiMda doDaDxdAgi dATugAlu hAki A gaNeya sutatx tirugabahudAda vAyxyAma sAdhaneya gaNe. 
\emng
\eentry

\bentry
\wordf{giaour}
\pron{jwarf}
\gl{\nA}
\bmng
 \Per\ (musalAmxnaralalxdavaranunx, \kanmu\ kerxYsatxranunx, kuritu tukiRyavaru tucaCxvAgi baLasuva pada) pASaMDa; dhamaRbAhira. 
\emng
\eentry

\bentry
\word{gib}
\pron{gi(ji)bf}
\gl{\nA}
\bmng
\bnum
\num{1} gibubx; (yaMtarxBAga \mo vanunx adara sathxLadalilxruvaMte bigisuva) loVhada yA marada beNe. 
\num{2} moLe; gUTa; ANi. 
\enum
\emng
\eentry

\bentry
\word{Gib.}
\pron{jibf}
\gl{\nA}
\bmng
 (\AmA) \eng{Gibralter} (enunxvudara \saMkiSx). 
\emng
\eentry

\bentry
\word[gibber(1)]{gibber}
\pron{jibarf}
\gl{\akirx}
\bmng
\bnum
\num{1} veVgavAgi, asapxSaTxvAgi mAtanADu. 
\num{2} koVtiyaMte kiruguTuTx. 
\enum
\emng
\eentry

\bentry
\word[gibber(2)]{gibber}
\pron{jibarf}
\gl{\nA}
\bmng
\bnum
\num{1} veVgavAda, asapxSaTx -- nuDi, mAtu. 
\numi{2} 
\banum
\alnum{a} koVtiyaMte kiruguTuTxvudu. 
\alnum{b} aMtha mAtu, dhavxni. 
\eanum
\numie
\enum
\emng
\eentry

\bentry
\word[gibber(3)]{gibber}
\pron{gibarf}
\gl{\nA}
\bmng
 (\AseTxrXV) buruDugalulx; baMDegalulx; dapapxkalulx; bwlaDxru. 
\emng
\eentry

\bentry
\word{gibberellin}
\pron{jibarelinf}
\gl{\nA}
\bmng
 jibarelinf; ele matutx cigurugaLa beLavaNigeyanunx parxcoVdisuva sasayx hAmoRVnugaLa oMdu guMpu. 
\emng
\eentry

\bentry
\word{gibberish}
\pron{jibariSf}
\gl{\nA}
\bmng
\bnum
\num{1} athaRvAgada mAtu, padagaLu. 
\num{2} gojagoja; athaR rahita, athaRvilalxda -- dhavxnigaLu. 
\num{3} ashudadhx nuDi yA leVKana; tapupxtapApxda yA vAyxkaraNabadadhxvalalxda mAtu yA bareha. 
\enum
\emng
\eentry

\bentry
\word[gibbet(1)]{gibbet}
\pron{jibiTf}
\gl{\nA}
\bmng
\bnum
\num{1} (\ca) galulxmara. 
\num{2} galulxgaMba; neVNugaMba; galilxge hAkida aparAdhigaLa deVhavanunx sarapaNiyiMda tUguhAkutitxdadx toVLugaMba. 
\num{3} galulxshikeSx; maraNadaMDane; neVNuhAki sAyisuvudu. 
\enum
\emng
\eentry

\bentry
\word[gibbet(2)]{gibbet}
\pron{jibiTf}
\gl{\sakirx}
\bmng
\bnum
\num{1} galilxge hAki kolulx; galilxge hAku; galilxgeVrisu. 
\num{2} galulxmarada meVle (aparAdhiya deVhavanunx) tUgakaTuTx. 
\num{3} galulxmarada meVlinaMte tUgakaTuTx. 
\num{4} apakiVtiRge IDu mADu; avaheVLanakekx guripaDisu. 
\enum
\emng
\eentry

\bentry
\word{gibbon}
\pron{gibanf}
\gl{\nA}
\bmng
 gibanf (koVti); (\kanmu\ AgenxVya iMDiyA divxVpa samudAyagaLalilx doreyuva) kelavu jAtiya niDudoVLina vAnara.  \imglink{gibbonfigure}{\raisebox{-0.25cm}[0pt][0pt]{\pdfimage width 0.6cm height 0.8cm{G_Pictures/gibbon.jpg}}} 
\emng
\eentry

\bentry
\word{gibbosity}
\pron{gibAsiTi}
\gl{\nA}
\bmng
\bnum
\num{1} utatxlate; piVnate; udavxkarxte; hora ubibxke; hora ububx; muMdakekx ubibxruvudu. 
\num{2} (\Kavi) gUnu; piVnate; (caMdarx yA garxhada \vi) parxkAshita BAga adhaRvaqtatxkikxMta hecucx, pUNaRvaqtatxkikxMta kaDime iruvudu. 
\num{3} DububxLaLxdAdxgiruvudu; gUnu (benAnxgiruvudu). 
\enum
\emng
\eentry

\bentry
\word{gibbous}
\pron{gibasf}
\gl{\gu}
\bmng
\bnum
\num{1} utatxla; piVna; udavxkarx; hora ubibxna; muMdakekx ubibxda. 
\num{2} (\Kavi) gUnu; piVna; gUnAda; (caMdarx yA garxhada \vi) parxkAshita BAga adhaRvaqtatxkikxMtalU hecAcxgiruva matutx pUNaR vaqtatxkikxMta kaDime iruva. 
\num{3} DububxLaLx; gUnubeninxna. 
\enum
\emng
\eentry

\bentry
\word{gibbously}
\pron{gibasfli}
\gl{\kirxvi}
\bmng
\bnum
\num{1} hora ubibx; utatxlavAgi; piVnavAgi; udavxkarxvAgi; muMdakekx ubibx. 
\num{2} DubAbxgi; gUnubenAnxgi. 
\enum
\emng
\eentry

\bentry
\word[gibe(1)]{gibe}
\pron{jeYbf}
\gl{\sakirx}
\bmng
 aNakisu; avaheVLana mADu. 
\emng

\noindent
\gl{\akirx}
\bmng
 mUdalisu; aNakisi mAtanADu; avaheVLanada mAtanADu; apahAsayxmADu. 
\emng
\eentry

\bentry
\word[gibe(2)]{gibe}
\pron{jeYbf}
\gl{\nA}
\bmng
 mUdalike; apahAsayx; avaheVLana mADuvudu. 
\emng
\eentry

\bentry
\word{giber}
\pron{jeYbarf}
\gl{\nA}
\bmng
 mUdalisuvavanu; aNakisuvavanu; avaheVLana, apahAsayx mADuvavanu. 
\emng
\eentry

\bentry
\word{gibingly}
\pron{jeYbiMgfli}
\gl{\kirxvi}
\bmng
 mUdalisuvaMte; aNakisutAtx; avaheVLana mADi; apahAsayx mADi. 
\emng
\eentry

\bentry
\word{giblets}
\pron{jibilxTfsx}
\gl{\nA}
\bmng
 (\bava) beVyisaleMdu katatxrisi tegeda, hakikxya (pitatxkoVsha, jaThara, rekekx, pAda, \mo) tinanxbahudAda BAgagaLu: \eng{giblets soup} I BAgagaLiMda mADida sAru, sUpu. 
\emng
\eentry

\bentry
\wordnospeech{GI bride}{GI bride}
\pron{?}
\gl{\nA}
\bmng
 jiai vadhu; amerikada sashasatxrX daLada seYnikanu kelasada meVle videVshakekx hoVdAga alilx maduve mADikoMDa videVshiV heMgasu. 
\emng
\eentry

\bentry
\word{gibus}
\pron{jeYbasf}
\gl{\nA}
\bmng
 apera hAyxTu; saMgiVta nATakada pAtarxdhAri dharisuva oMdu terana otutxhAyxTu. 
\emng
\eentry

\bentry
\word{giddap}
\pron{giDAYxpf}
\gl{\BAavayx}
\bmng
 (\AmA) (kudurege niVDuva AdeVsha) ODu! caloV! beVga! jalidx! 
\emng
\eentry

\bentry
\word{giddily}
\pron{giDili}
\gl{\kirxvi}
\bmng
\bnum
\num{1} girarxne tirugi; sututxtAtx. 
\num{2} tale tirugidaMte; hucucxhucAcxgi: \eng{giddily hatted} hucucxhucAcxgi hAyxTu dharisida. 
\enum
\emng
\eentry

\bentry
\word{giddiness}
\pron{giDinisf}
\gl{\nA}
\bmng
\bnum
\num{1} tiLigeVDitana; budidhxgeVDitana; viveVcanAshUnayxte; tale tirukatana: \eng{the giddiness of youth} ywvanasahaja tiLigeVDitana. 
\num{2} taletirugu; talesututx; tale tirugi biVLuvaMtAguvudu. 
\num{3} veVgavAda calane. 
\num{4} caMcalate; asithxrate. 
\enum
\emng
\eentry

\bentry
\word[giddy(1)]{giddy}
\pron{giDi}
\gl{\gu}
\bmng
\bnum
\num{1} (kAyile, vijayoVtAsxha, \mo vugaLiMda) tale tiruguva; tale sututxva; talesutitx -- biVLuvaMtAguva yA biVLuva; giraki hoDeyuva. 
\num{2} tale tirugisuvaMte mADuva; tale giraki hoDesuva: \eng{a giddy precipice} tale tiruguvaMte mADuva parxpAta. 
\num{3} girarxne tiruguva; ativeVgadalilx sututxva, tiruguva: \eng{giddy round of Fortune's wheel} girarxne tiruguva adaqSaTx deVvateya cakarxda sututx, tirugu. 
\numi{4} (vayxkitxgaLa \vi) (adhikAra \mo\ kAraNadiMda) 
\banum
\alnum{a} madisida; tale tirugida; tale tiruka. 
\alnum{b} budidhxgeTaTx; viveVcanegeTaTx; ydAvxtadAvx naDedukoLuLxva; tiLigeVDitana toVrisuva. 
\eanum
\numie
\num{5} caMcala; sulaBavAgi utAsxhaBarakekx oLagAguva. 
\num{6} tikakxlina; huDugATada; ailupeYlina. 
\num{7} (\kanmu\ vayxMgayxvAgi) Adhikayx, tiVvarxte sUcisuva \parx. 
\enum
\emng

\noindent
\gl{\nuga}
\bmng
\bnum
\num{1} \eng{my giddy aunt!} (AshacxyaRsUcaka udAgxra) ayoyxV nananx cikakxmamx! 
\hypertarget{giddy(1) nuga(2)}{} 
\num{2} \eng{play the giddy goat} (\engit{or} \eng{ox)} shudadhx kuriyaMte yA tiLigeVDiyaMte vatiRsu; hucucxhucAcxgi vatiRsu. 
\enum
\emng
\eentry

\bentry
\word[giddy(2)]{giddy}
\pron{giDi}
\gl{\sakirx}
\bmng
 tale tiruguvaMte mADu; tale sutitx biVLuvaMte mADu; tale giraki hoDesu: \eng{the sight giddied his sense} A daqshayx avananunx tale sutitx biVLuvaMte mADitu. 
\emng

\noindent
\gl{\akirx}
\bmng
 tale -- tirugu, sututx; tale sutitx biVLuvaMtAgu; talesutitx biVLu; tale giraki hoDe. 
\emng
\eentry

\bentry
\word{giddy-go-round}
\pron{giDigoVrwMDf}
\gl{\nA}
\bmng
 (\birx) raMkarATe; marada kuduregaLu, toTiTxlugaLu, \mo vugaLuLaLx tiruguyaMtarx. 
\emng
\eentry

\bentry
\word{Gideon}
\pron{giDianf}
\gl{\nA}
\bmng
 giDiyanf; hoVTelina rUmugaLu \mo\ kaDe beYbalalxnunx iDuva, \eng{1899} ralilx sAthxpitavAda kerxYsatxsaMGada sadasayx. 
\emng
\eentry



\bentry
\word[gift(1)]{gift}
\pron{giphfTx}
\gl{\nA}
\bmng
\bnum
\num{1} koDuge; dAna; koVDu; koTaTxdudx; koDuvudu; dAnamADuvudu: \eng{came to me by free gift} adu nanage pukakxTe dAnavAgi baMtu. (\pArxparx) \eng{would not have it at a gift} koDugeyAgi baMdarU beVDa; pukakxTe koTaTxrU beVDa. 
\num{2} (\nAyxshA) dAna; parxtiPalApeVkeSxyilalxde manasAra vagARyisikoTaTx Asitx. 
\num{3} dAna koTaTx vasutx; koDuge; upAyana; dAna; datitx. 
\hypertarget{gift(1)4}{} 
\num{4} vara; deVvara varaparxsAdaveMdu parigaNisuva guNa, shakitx. 
\num{5} sAvxBAvika yA neYsagiRka shakitx yA guNa; huTuTx -- shakitx, kwshala; sahaja parxtiBe. 
\num{6} sulaBavAda kelasa \mo vu. 
\enum
\emng

\noindent
\gl{\pagu}
\bmng
 \eng{gift of tongues} 
\banum
\alnum{a} (\kanmu\ Adayx kerxYsatxrige deYvavaravAgi baMda) aparicita BASegaLalilx mAtanADuva shakitx. 
\alnum{b} bahuBASABijacnxte; bahuBASikate. 
\eanum
\emng

\noindent
\gl{\nuga}
\bmng
\bnum
\num{1} \eng{gift of the} \hyperlink{gab nuga(1)}{gab}. 
\num{2} \eng{is in the gift of} koDuge (obabxna) vashadalilxde: \eng{the living is in the gitf of the Bishop} A vaqtitxya koDuge biSapapxna kaYyalilxde, vashadalilxde. 
\enum
\emng
\eentry

\bentry
\word[gift(2)]{gift}
\pron{giphfTx}
\gl{\sakirx}
\bmng
\bnum
\num{1} datitx yA koDuge biDu. 
\num{2} dAna mADu; bahumAnavAgi koDu. 
\num{3} varavAgi niVDu, koDu. 
\enum
\emng
\eentry

\bentry
\word{gift-book}
\pron{giphfTxbukf}
\gl{\nA}
\bmng
 bahumAna pusatxka; bahumAnavAgi baMda, bahumAnavAgi koDuvaMtha -- pusatxka. 
\emng
\eentry

\bentry
\wordnospeech{gift coupon}{gift coupon}
\pron{?}
\gl{\nA}
\bmng
 inAmuciVTi; oMdu gotAtxda saMKeyxya ciVTigaLanunx hiMdakekx koTATxga bahumAna doreyuva, kelavu sarakugaLanunx koMDAga koDuva dAKale ciVTi. 
\emng
\eentry

\bentry
\word{gifted}
\pron{giphiTxDf}
\gl{\gu}
\bmng
 sahaja parxtiBeyuLaLx; parxtiBAshAliyAda; sAvxBAvika yA neYsagiRka kwshalavuLaLx; huTuTxshakitxyiMda kUDida. 
\emng
\eentry

\bentry
\word{gift-horse}
\pron{giphfTxhAsfR}
\gl{\nA}
\bmng
 (\rUpa) dAnavAgi baMda kudure; alapxbeleya koDuge. 
\emng

\noindent
\gl{\nuga}
\eng{do not look the gift-horse in the mouth} dAnada kudureya halulxhiDidu noVDabeVDa; dAnada ememxya halelxNisabeVDa.\eentry

\bentry
\word{giftie}
\pron{giphiTx}
\gl{\nA}
\bmng
 (sAkxTalxMDf \parx) = \hyperlink{gift(1)4}{$^1$gift (4)}: \eng{the giftie gie us to see oursels as others see us} itararu namamxnunx heVge kANutAtxroV namamxnunx nAveV hAge kaMDukoLuLxva varavanunx koDu. 
\emng
\eentry

\bentry
\wordnospeech{gift shop}{gift shop}
\pron{?}
\gl{\nA}
\bmng
 bahumAnadaMgaDi; muyayxMgaDi; bahumAnada vasutxgaLanunx mAruva aMgaDi. 
\emng
\eentry

\bentry
\wordnospeech{gift token}{gift token}
\pron{?}
\gl{\nA}
\bmng
 muyiyxciVTi; bahumAna ciVTi; bahumAnada vasutxvoMdanunx koLaLxlu duDuDx koTuTx paDeyuva ciVTi. 
\emng
\eentry

\bentry
\wordnospeech{gift voucher}{gift voucher}
\pron{?}
\gl{\nA}
\bmng
  = \hyperlink{gift token}{gift token}. 
\emng
\eentry

\bentry
\word{gift-wrap}
\pron{giphfTxrAYxpf}
\gl{\sakirx}
\bmng
 (bahumAnavAgi koDuva yAvudeV vasutxvanunx alaMkArada kAgada hAki) suMdaravAgi sututx, kaTuTx; AkaSaRka kavaca -- hAku, hodesu. 
\emng
\eentry

\bentry
\word[gig(1)]{gig}
\pron{gigf}
\gl{\nA}
\bmng
\bnum
\num{1} (haguravAda, eraDu cakarxda) oMTikudureya gADi.  \imglink{gig-1figure}{\raisebox{-0.15cm}[0pt][0pt]{\pdfimage width 0.8cm height 0.5cm{G_Pictures/gig-1.jpg}}} 
\num{2} (haDaginalilx koMDoyuyxva, huTuTxhAkiyoV hAyikaTiTxyoV naDesabahudAda) oMdu bageya hagura kirudoVNi. 
\num{3} (huTuTxhAki naDesuva) paMdayxda doVNi. 
\enum
\emng
\eentry

\bentry
\word[gig(2)]{gig}
\pron{gigf}
\gl{\nA}
\bmng
 gigf; mInu tiviyuva oMdu bageya ITi. 
\emng
\eentry

\bentry
\word[gig(3)]{gig}
\pron{gigf}
\gl{\nA}
\bmng
 (\AmA) 
\bnum
\num{1} gigf; jAsfZ saMgiVta \mo vanunx naDesalu saMgiVtagAraranunx, \kanmu\ oMdu rAtirxge, neVmisikoLuLxvudu. 
\num{2} aMtha kAyaRkarxmada sathxLa. 
\enum
\emng
\eentry

\bentry
\word[gig(4)]{gig}
\pron{?}
\gl{\akirx}
\bmng
 `gigf' kAyaRkarxma mADu. 
\emng
\eentry

\bentry
\word{giga-}
\pron{jeY(geY)ga-, ji(gi)ga-}
\gl{\sapUpa}
\bmng
 \eng{$10^9$} raSuTx eMbudanunx sUcisuva \sapUpa: \eng{gigawatt} jigavAyxTf; \eng{$10^9$} vAyxTugaLu. 
\emng
\eentry

\bentry
\word{gigantesque}
\pron{jeYgAYxMTesfkx}
\gl{\gu}
\bmng
\bnum
\num{1} rAkaSxsa lakaSxNagaLuLaLx; deYtayxguNada. 
\num{2} rAkaSxsiVya; rAkaSxsanige takakx. 
\enum
\emng
\eentry

\bentry
\word{gigantic}
\pron{jeYgAYxMTikf}
\gl{\gu}
\bmng
\bnum
\num{1} (AkAra, etatxra, \mo vugaLalilx) rAkaSxsa parxmANada; deYtAyxkArada. 
\num{2} bahudoDaDx; bahaLa BAri; baqhadAkArada; baqhatapxrXmANada; aduBxtAkArada; vipariVta doDaDx. 
\enum
\emng
\eentry

\bentry
\word{gigantically}
\pron{jeYgAYxMTikali}
\gl{\kirxvi}
\bmng
\bnum
\num{1} deYtayxparxmANadalilx. 
\num{2} bahaLa BAriyAgi; vipariVta doDaDxdAgi: \eng{yawned gigantically} vipariVta doDaDxdAgi AkaLisida. 
\enum
\emng
\eentry

\bentry
\word{gigantism}
\pron{jeYgAYxMTisaZmf}
\gl{\nA}
\bmng
\hypertarget{gigantism(1)}{} 
\bnum
\num{1} (\roVshA) deYtayxte; piTUyxTari garxMthiya atiyAda caTuvaTikeyiMda uMTAguva bahaLa BAri beLavaNige hAgU adananxnusarisi baruva sAnxyugaLa dubaRlate matutx napuMsakatavx. 
\num{2} (sasigaLalilx) atiyAda beLavaNige. 
\enum
\emng
\eentry

\bentry
\word[giggle(1)]{giggle}
\pron{gigflf}
\gl{\akirx}
\bmng
\bnum
\num{1} (oyAyxrada, ashikiSxta, asaBayx huDugiyaMte) kilakilane nagu, gulugu. 
\num{2} musumusu, kisikisi -- nagu. 
\enum
\emng

\noindent
\gl{\sakirx}
\bmng
 musumusu nagutAtx heVLu. 
\emng
\eentry

\bentry
\word[giggle(2)]{giggle}
\pron{gigflf}
\gl{\nA}
\bmng
\bnum
\num{1} (\sA\ \bava dalilx \parx) nagu; ciricu; (oyAyxrada, ashikiSxta, asaBayx huDugiya) kilakila nage: \eng{(fit of) the giggles} naguvina keraLu. 
\num{2} kisikisi naguvudu. 
\num{3} (\AmA) vicitarx vasutx yA vayxkitx. 
\num{4} tamASe; joVku: \eng{did it for a giggle} tamASegAgi mADida. 
\enum
\emng
\eentry

\bentry
\word{giggly}
\pron{gigilx}
\gl{\gu}
\bmng
 kilakilane naguva savxBAvada; musinageya. 
\emng
\eentry

\bentry
\wordnospeech{gig lamps}{gig lamps}
\pron{?}
\gl{\nA}
\bmng
 (\ashi) kananxDaka. 
\emng
\eentry

\bentry
\word{giglet}
\pron{gigfliTf}
\gl{\nA}
\bmng
 kilakilane naguva huDugi; halulx kisiyuvavaLu. 
\emng
\eentry

\bentry
\word{giglot}
\pron{gigflaTf}
\gl{\nA}
\bmng
  = \hyperlink{giglet}{giglet}. 
\emng
\eentry

\bentry
\word{gigman}
\pron{gigfmanf}
\gl{\nA}
\bmng
\bnum
\num{1} oMTi kudureya gADi iTiTxruvavanu. 
\num{2} vAhanavaMta; vasutx vAhanAdigaLige pArxdhAnayx koTuTx mereyuva vagaRkekx seVridavanu. 
\enum
\emng
\eentry

\bentry
\word{gigmanity}
\pron{gigfmAYxniTi}
\gl{\nA}
\bmng
 vAhanavaMtaru; saMsakxqqti veYrigaLu; lwkika vayxvahArigaLu; shuSakx haqdayigaLu; arasika vaqMda; unanxta guNagaLilalxde, vasutx vAhanagaLige pArxdhAnayx koTuTx mereyuva madhayxmavagaRda janaru; doDaDxvara madheyx ODADuvudeV jiVvana dheyxVyaveMdu tiLidu naDeyuvavaru. 
\emng
\eentry

\bentry
\word{gig-mill}
\pron{gigfmilf}
\gl{\nA}
\bmng
 gigfmilulx; baTeTxya meVle juMgebibxsuva yaMtarx yA yaMtarxshAle. 
\emng
\eentry

\bentry
\word{gigolo}
\pron{ji(SiZ)galoV}
\gl{\nA}
\expl{(\bava\ \eng{gigolos}).}
\bmng
\bnum
\num{1} sahanataRka; inonxbabxra jote natiRsuvavanu. 
\num{2} puruSasaMgAti; jotegAranAgiralu matutx leYMgika taqpitxgAgi vayasAsxda heMgasu duDuDxkAsu koTuTx poVSisikoMDiruva taruNa. 
\enum
\emng
\eentry

\bentry
\wordRemoveSpace{gigot-sleeve}{gigot sleeve}
\pron{jigaTf silxVvf}
\gl{\nA}
\bmng
 (aMgiya) kurikAlu toVLu; kurikAlina mAMsada AkArada toVLu. 
\emng
\eentry

\bentry
\word{gigue}
\pron{SiZVgf}
\gl{\nA}
\bmng
\bnum
\num{1} jiVgf (naqtayx); oMdu bageya geluvina naqtayx. 
\num{2} jiVgf naqtayxkekx hoMdisida saMgiVta. 
\num{3} (\saM) eraDu BAgagaLidudx avugaLanunx punarAvatiRsuva, oMdu bageya geluvina naqtayx. 
\enum
\emng
\eentry

\bentry
\wordnospeech{GI joe}{GI joe}
\pron{?}
\gl{\nA}
\bmng
 (\AmA) amerikada (sAmAnayx dajeRya) seYnika. 
\emng
\eentry

\bentry
\wordRemoveSpace{Gila-monster}{Gila monster}
\pron{hiVla mAnfsaTxrf}
\gl{\nA}
\bmng
 rAkaSxsa halilx; deYtayxgwLi; (arijoVna, mekisxko, \mo\ kaDegaLalilx vAsisuva) viSapUrita doDaDxhalilx.  \imglink{gila-monsterfigure}{\raisebox{-0.15cm}[0pt][0pt]{\pdfimage width 0.7cm height 0.5cm {G_Pictures/gila-monster.jpg}}} 
\emng
\eentry

\bentry
\word{Gilbertian}
\pron{gilabxTiRanf, gilabxSaRnf}
\gl{\gu}
\bmng
 vicitarx; asaMbadadhx; vipayARsa; talekeLagAda; tirugumurugina; iMgelxMDina gilfbaTfR matutx salivanf eMbuvara giVtanATakagaLalilxnaMte hAsayxBarita vipayARsada aMshagaLa: \eng{a gilbertian situation} gilfbaTiRyanf saninxveVsha; vipayARsada sithxti. 
\emng
\eentry

\bentry
\word[gild(1)]{gild}
\pron{gilfDx}
\gl{\sakirx}
\expl{(\BUkaq\ \eng{gilded, gilt}).}
\bmng
\bnum
\num{1} cinanxda mulAmu mADu; cinanxda leVpa hAku; cinanxda reVku hacucx. 
\num{2} cinanxda baNaNx hAku; cinanxda hAge kANuvaMte mADu; giliVTu mADu. 
\num{3} (oMdu sithxti \mo vanunx) sahisalAguvaMte, mayARdAhaRvAguvaMte haNa koTuTx EpaRDisu. 
\num{4} hoMbaNaNxda, hoMbeLagina -- CAye koDu, alaMkarisu. 
\num{5} (mAtiniMda) giliVTu mADu; aMdavAda mAtugaLiMda horakekx kaLe kaTiTxsu, thaLaku mADu. 
\enum
\emng

\noindent
\gl{\nuga}
\bmng
\bnum
\num{1} \eng{gild the lily} kamalakekx baNaNx hacucx; sahajavAgiyeV suMdaravAgiruva vasutxvige anagatayx alaMkAra mADu, baNaNx -- baLi, hacucx. 
\hyperdef{G}{gild(1) nuga(2)}{} 
\num{2} \eng{gild the pill} guLige hitavAgisu; anivAyaR ahitavanunx sahisalAguvaMte maqdu mADu. 
\enum
\emng
\eentry

\bentry
\word[gild(2)]{gild}
\pron{gilfDx}
\gl{\nA}
\bmng
  = \hyperlink{guild}{guild}. 
\emng
\eentry

\bentry
\wordnospeech{gilded cage}{gilded cage}
\pron{?}
\gl{\nA}
\bmng
 honanxpaMjara; cinanxda boVnu; savxNaRpaMjara; tuMba shirxVmaMtavAda, Adare nibaRMdhagaLiruva vAtAvaraNa. 
\emng
\eentry

\bentry
\wordnospeech{Gilded Chamber}{Gilded Chamber}
\pron{?}
\gl{\nA}
\bmng
 (birxTaninxna) lADfsxR saBe. 
\emng
\eentry

\bentry
\wordnospeech{gilded spurs}{gilded spurs}
\pron{?}
\gl{\nA}
\bmng
 (\pArxparx) neYTf padaviya lAMCanagaLu. 
\emng
\eentry

\bentry
\wordnospeech{gilded youth}{gilded youth}
\pron{?}
\gl{\nA}
\bmng
 shirxVmaMta SoVkilAla; nayanAjUkina sogasugAra; phAyxSaninxna shirxVmaMta taruNaru. 
\emng
\eentry

\bentry
\word{gilder}
\pron{gilaDxrf}
\gl{\nA}
\bmng
 giliVTugAra; cinanxda mulAmu hAkuvavanu. 
\emng
\eentry

\bentry
\word{Gilderoy}
\pron{gilaDxrAyf}
\gl{\nA}
\bmng
 sAkxTalxMDina obabx daroVDegArana hesaru. 
\emng

\noindent
\gl{\nuga}
\bmng
 \eng{higher than Gilderoy's kite} tuMba etatxrada; kaNiNxge kANisada: \eng{the colt threw him higher than Gilderoy's kite} kudure mariyu avananunx kaNiNxge kANisadaSuTx etatxrakekx eseyitu. 
\emng
\eentry

\bentry
\word{gilding}
\pron{giliDxMgf}
\gl{\nA}
\bmng
 giliVTu mADuvudu; cinanxda mulAmu hacucxvudu. 
\emng
\eentry

\bentry
\word{gilgai}
\pron{gilfgeY}
\gl{\nA}
\bmng
 (\AseTxrXV) maLe -- kere, kaTeTx; maLeya niVru sheVKaraNeyAguva, sAsarf AkArada neYsagiRka kere, kaTeTx. 
\emng
\eentry

\bentry
\word[gill(1)]{gill}
\pron{gilf}
\gl{\nA}
\bmng
 (\sA\ \bava dalilx) 
\bnum
\num{1} kiviru; ToLeLx; mInu matitxtara jalashAvxsa pArxNigaLa shAvxsAMga. 
\num{2} (koVLiya) toMgudovalu. 
\num{3} kiviru; nAyikoDeya keLamuKadalilx keVMdarxdiMda aMcina kaDege horaDuva paTiTxgaLalolxMdu. 
\num{4} galalxda sutatxmutatxla mAMsa. 
\enum
\emng

\noindent
\gl{\nuga}
\bmng
\bnum
\num{1} \eng{green about the gills} roVgada cihenx iruva. 
\num{2} \eng{rosy about the gills} AroVgayxda cihenx iruva. 
\enum
\emng
\eentry

\bentry
\word[gill(2)]{gill}
\pron{gilf}
\gl{\sakirx}
\bmng
\bnum
\num{1} (mInina) karuLu tege. 
\num{2} nAyikoDeya kivirugaLanunx katatxrisi hAku. 
\num{3} kivirubaleyalilx mInu hiDi. 
\enum
\emng
\eentry

\bentry
\word[gill(3)]{gill}
\pron{gilf}
\gl{\nA}
\bmng
 (\birx) 
\bnum
\num{1} (\sA\ kADu beLeda) ALavAda kamari. 
\num{2} beTaTxda kiru Jari, tore. 
\enum
\emng
\eentry

\bentry
\word[gill(4)]{gill}
\pron{jilf}
\gl{\nA}
\bmng
jilf: 
\banum
\alnum{a} kAlu peYMTina darxvada aLate. 
\alnum{b} (\pArxM) adhaR peYMTu. 
\eanum
\emng
\eentry

\bentry
\word[gill(5)]{gill}
\pron{jilf}
\gl{\nA}
\bmng
% 
\bnum
\num{1} (\hiV) huDugi; taruNi. 
\num{2} (\AmA) heNuNx phereTf bekukx. 
\enum
\emng

\noindent
\gl{\pagu}
\bmng
% \eng{Jack and Gill} jAkf matutx jilf; huDuga huDugi; poVra poVri. 
\emng
\eentry

\bentry
\wordnospeech{gill arch}{gill arch}
\pron{?}
\gl{\nA}
\bmng
% kiviru bAgu; mInu matutx uBayacaragaLalilx gaMTala kuharada baLi eraDU kaDe kivirugaLige AdhAravAgiruva bAgu. 
\emng
\eentry

\bentry
\word{gillaroo}
\pron{gilarU}
\gl{\nA}
\bmng
% ailaRMDina oMdu terana mInu. 
\emng
\eentry

\bentry
\wordnospeech{gill cavity}{gill cavity}
\pron{?}
\gl{\nA}
\bmng
% kiviru koVSaThx; kivirubAgigU kiviru mucacxLakUkx madheyx, kiviru taMtugaLu cAcikoMDiruva saMpuTa. 
\emng
\eentry

\bentry
\wordnospeech{gill chamber}{gill chamber}
\pron{?}
\gl{\nA}
\bmng
 = \hyperlink{gill cavity}{gill cavity}. 
\emng
\eentry

\bentry
\word{gill-cover}
\pron{gilfkavarf}
\gl{\nA}
\bmng
kiviru mucacxLa; mInina kivirugaLanunx mucicxruva mucacxLa. 
\emng
\eentry

\bentry
\word{gilled}
\pron{gilfDx}
\gl{\gu}
\bmng
kiviruLaLx; kivirugaLiruva. 
\emng
\eentry

\bentry
\word{gillie}
\pron{gili}
\gl{\nA}
\bmng
(\ca) 
\bnum
\num{1} (sAkxTelxMDina) heYlaMDf nAyakana, hegagxDeya -- sahacara, seVvaka. 
\num{2} (sAkxTelxMDinalilx) beVTegArana yA mInu hiDiyuvavana -- sahacara, seVvaka, huDuga. 
\enum
\emng
\eentry

\bentry
\word{gill-net}
\pron{gilfneTf}
\gl{\nA}
\bmng
% kivirubale; mInugaLa kivirugaLu toDarikoMDu avu sikikxbiVLuvaMte EpaRDisiruva bale. 
\emng
\eentry

\bentry
\wordRemoveSpace{gilly-flower}{gilly flower}
\pron{jiliphwlxarf}
\gl{\nA}
\bmng
jilihUvu; lavaMgada vAsaneyuLaLx kelavu bageya hUvugaLu. 
\emng

\noindent
\gl{\pagu}
\bmng
\eng{clove gillyflower} = \hyperlink{gilly flower}{gilly flower}. 
\emng
\eentry

\bentry
\word[gilt(1)]{gilt}
\pron{gilfTx}
\gl{\gu}
\bmng
\bnum
\num{1} cinanxda mulAmu hAkisida, leVpisida. 
\num{2} hoMbaNaNxda; savxNaRda. 
\enum
\emng
\eentry

\bentry
\word[gilt(2)]{gilt}
\pron{gilfTx}
\gl{\nA}
\bmng
\bnum
\num{1} cinanxda mulAmu; savxNaRleVpa; giliVTu. 
\num{2} hecucx KAtariya, Baravaseya, necicxkeyuLaLx -- sAlapatarxgaLu, (sakARrada) AdhArapatarxgaLu. 
\enum
\emng

\noindent
\gl{\nuga}
\bmng
\hypertarget{gilt(2) nuga}{} \eng{take the gilt off the gingerbread} (oMdu vasutxvina) mithAyxkaSaRNegaLanunx kitutx hAku; bAhAyxlaMkAragaLanunx kaLacihAku. 
\emng
\eentry

\bentry
\word[gilt(3)]{gilt}
\pron{gilfTx}
\gl{\nA}
\bmng
 heNuNx haMdimari. 
\emng
\eentry

\bentry
\word{gilt-edged}
\pron{gilfTxejfDx}
\gl{\gu}
\bmng
 (sAlapatarxgaLu, sATxkugaLu, \mo vugaLa \vi) honanxMcina; hecucx -- KAtariya, Baravaseya, necicxkeya: \eng{gilt-edged securities} honanxMcina sAlapatarxgaLu; (sakARrada) AdhArapatarxgaLu. 
\emng
\eentry

\bentry
\word{gilt-wood}
\pron{gilfTxvuDf}
\gl{\gu}
\bmng
 giliVTu marada; maradiMda mADi giliVTumADida. 
\emng
\eentry

\bentry
\word{gimbals}
\pron{jiMbalfs'}
\gl{\nA}
\bmng
 (\bava) jiMbalfsx; (\kanmu\ dikUsxciyU kAlamApakayaMtarxvU) haDagina cilaneyiMda bAdhitavAgade yAvAgalU samamaTaTxdalilxyeV ADuvaMtAgisuva, (\sA\ baLegaLiMdalU tirugugUTagaLiMdalU racisida) salakaraNe.  \imglink{gimbalsfigure}{\raisebox{-0.15cm}[0pt][0pt]{\pdfimage width 0.6cm height 0.5cm{G_Pictures/gimbals.jpg}}} 
\emng
\eentry

\bentry
\word{gimbals-ring}
\pron{jiM(giM)balfs'riMgf}
\gl{\nA}
\bmng
 jiMbalfsx(nalilxruva) baLe. 
\emng
\eentry

\bentry
\word[gimcrack(1)]{gimcrack}
\pron{jimfkArxYxkf}
\gl{\nA}
\bmng
 bari Dwlina, kelasakekx bArada, thaLakupaLakina -- oDave, \mo\ vasutx. 
\emng
\eentry

\bentry
\word[gimcrack(2)]{gimcrack}
\pron{jimfkArxYxkf}
\gl{\gu}
\bmng
 kelasakekx bArada; bari Dwlina; giliVTina. 
\emng
\eentry

\bentry
\word{gimcrackery}
\pron{jimfkArxYxkari}
\gl{\nA}
\bmng
 kelasakekx bArada, Dwlina oDave \mo\ vasutxgaLu. 
\emng
\eentry

\bentry
\word{gimcracky}
\pron{jimfkArxYxki}
\gl{\gu}
\bmng
 kelasakekx bArada; Dwlina; giliVTina. 
\emng
\eentry

\bentry
\word{gimlet}
\pron{gimilxTf}
\gl{\nA}
\bmng
\bnum
\num{1} oMdu terana niDusuruLeya beYrige.  \imglink{gimlet-1figure}{\raisebox{-0.15cm}[0pt][0pt]{\pdfimage width 0.4cm height 0.6cm {G_Pictures/gimlet-1.jpg}}} 
\num{2} `jinf' matutx niMberasada misharxNa. 
\enum
\emng
\eentry

\bentry
\wordnospeech{gimlet eye}{gimlet eye}
\pron{?}
\gl{\nA}
\bmng
tiVkaSxNXdaqSiTx; tiVkaSxNXvAda yA cucucxva noVTavuLaLx kaNuNx. 
\emng
\eentry

\bentry
\word{gimme}
\pron{gimi}
\gl{\sakirx}
\bmng
 (\AmA)  = \hyperlink{giveme}{give me} (eMbudara saMkiSxpatx). 
\emng
\eentry

\bentry
\word{gimmer}
\pron{jimarf}
\gl{\nA}
\bmng
 (\birx) 
\bnum
\num{1} heNuNx kurimari. 
\num{2} (\hiV) heNuNx; heMgasu. 
\enum
\emng
\eentry

\bentry
\word{gimmick}
\pron{gimikf}
\gl{\nA}
\bmng
 (\AmA) (beVreyavara gamana seLeyalu yA parxcArada udedxVshadiMda hUDuva, mADuva) taMtarx; yukitx; upAya; caLaka; cAtuyaR; kwshala; camatAkxra. 
\emng
\eentry

\bentry
\word{gimmickry}
\pron{gimikirx}
\gl{\nA}
\bmng
 (beVreyavara gamana seLeyalu yA parxcArakAkxgi mADuva) taMtarx; upAya. 
\emng
\eentry

\bentry
\word{gimmicky}
\pron{gimiki}
\gl{\gu}
\bmng
 (beVreyavara gamana seLeyalu yA parxcArakAkxgi mADuva) taMtarxda; upAyada. 
\emng
\eentry

\bentry
\word[gimp(1)]{gimp}
\pron{giMpf}
\gl{\nA}
\bmng
\bnum
\num{1} alaMkArada kucucx, paTiTx; oLagiruva taMtiya yA huriya meVle hoseda reVSemxya, uNeNxya, yA hatitxya nUliniMda mADida alaMkArada -- kare, kucucx, paTiTx.
\num{2} (reVSemx \mo vugaLiMda mADi taMtiyiMda balapaDisida) mInu hiDiyuva huri. 
\num{3} (leVsf tayArikeyalilx) hinenxleyiMda oMdu citArxkaqti edudx toVruvaMte mADuva dapapx nUleLe. 
\enum
\emng
\eentry

\bentry
\word[gimp(2)]{gimp}
\pron{giMpf}
\gl{\nA}
\bmng
 (\ashi) dheYyaR; kececxde; edegArike. 
\emng
\eentry

\bentry
\word[gin(1)]{gin}
\pron{jinf}
\gl{\nA}
\bmng
\bnum
\num{1} bale; jAla; boVnu. 
\numi{2} jinunx: 
\banum
\alnum{a} hagagx sututxva rATe matutx kerxVnugaLanonxLagoMDa, BAra eLeyuva yA etutxva yaMtarx. 
\alnum{b} araLerATe; hatitxyiMda biVjavanunx beVpaRDisuva yaMtarx. 
\eanum
\numie
\enum
\emng
\eentry

\bentry
\word[gin(2)]{gin}
\pron{jinf}
\gl{\sakirx}
\expl{(\BU\ matutx \BUkaq\ \eng{ginned,} \vakaq\ \eng{ginning}).}
\bmng
 (araLerATeyiMda) hatitx biVja -- tege, beVpaRDisu. 
\emng
\eentry

\bentry
\word[gin(3)]{gin}
\pron{jinf}
\gl{\nA}
\bmng
 jinf (madayx); (moLeta) dhAnayxdiMda baTiTxyiLisida madayx. 
\emng

\noindent
\gl{\pagu}
\bmng
\bnum
\num{1} \eng{gin and it} `jinf' matutx `iTAYxliyanf varfmUtf' eMba madayxgaLiMda tayArisida pAniVya. 
\num{2} \eng{pink gin} pATala (baNaNxda) jinf; aMgasUTxra togaTeyiMda rucikaTiTxda jinunx. 
\enum
\emng
\eentry

\bentry
\word[gin(4)]{gin}
\pron{jinf}
\gl{\saMavayx}
\bmng
(sAkxTalxMDf matutx \pArxM) -- re; pakaSxdalilx. 
\emng
\eentry

\bentry
\word[gin(5)]{gin}
\pron{jinf}
\gl{\nA}
\bmng
(\AseTxrXV) AdivAsi heMgasu. 
\emng
\eentry

\bentry
\word{gingall}
\pron{jiMgAlf}
\gl{\nA}
\bmng
 jiMgAlf; jiMjAlu; (ciVna matutx BAratada) 
\banum
\alnum{a} hagura tirugu baMdUka. 
\alnum{b} AdhArada meVliMda guMDu hArisuva BArada baMdUka. 
\eanum
\emng
\eentry

\bentry
\word[ginger(1)]{ginger}
\pron{jiMjarf}
\gl{\nA}
\bmng
\bnum
\num{1} shuMThi; suMTi; alalx. 
\num{2} shuMThi giDa. 
\num{3} satatxvX; kecucx; humamxsusx. 
\num{4} parxcoVdane; utetxVjana. 
\num{5} nasukeMpu CAyeya haLadi baNaNx. 
\enum
\emng

\noindent
\gl{\pagu}
\bmng
\bnum
\num{1} \eng{black ginger} (sipepx tegeyada) kapupx shuMThi; karishuMThi. 
\num{2} \eng{white ginger} (sipepx tegeda) biLi(yA) shuMThi. 
\enum
\emng

\noindent
\gl{\nuga}
\bmng
 \eng{ginger shall be hot in the mouth} BoVgApeVkeSxge, suKeVceCxge koneyeV ilalx; kAmada vAsane sulaBavAgi aLiyuvudilalx. 
\emng
\eentry

\bentry
\word[ginger(2)]{ginger}
\pron{jiMjarf}
\gl{\gu}
\bmng
 shuMThibaNaNxda; nasukeMpu CAyeya haLadi baNaNxda: \eng{ginger hair} shuMThi baNaNxda kUdalu. 
\emng
\eentry

\bentry
\word[ginger(3)]{ginger}
\pron{jiMjarf}
\gl{\sakirx}
\bmng
\bnum
\num{1} shuMThiya rucikaTuTx; shuMThi berasu. 
\num{2} (kudureyanunx) curuku mADalu gudakekx shuMThihacucx. 
\num{3} (\rUpa) (obabxnanunx) curuku mADu; parxcoVdisu; huriduMbisu. 
\enum
\emng
\eentry

\bentry
\word{gingerade}
\pron{jiMjareVDf}
\gl{\nA}
\bmng
 (\birx) shuMThi biyarf; shuMThi vAsane kaTiTxda biyaru. 
\emng
\eentry

\bentry
\word{ginger-ale}
\pron{jiMjarfElf}
\gl{\nA}
\bmng
 shuMThiElf; shuMThiya ruci yA vAsane kaTiTxda, AlokxhAlf ilalxda oMdu pAniVya. 
\emng
\eentry

\bentry
\wordnospeech{ginger beer}{ginger beer}
\pron{?}
\gl{\nA}
\bmng
 shuMThi biyaru; shuMThiElfnaMtaha, Adare hecucx shuMThi parimaLavuLaLx biyaru. 
\emng
\eentry

\bentry
\wordnospeech{ginger brandy}{ginger brandy}
\pron{?}
\gl{\nA}
\bmng
 shuMThi bArxMdi; shuMThi ruciya bArxMdi. 
\emng
\eentry

\bentry
\word[gingerbread(1)]{gingerbread}
\pron{jiMjarfberxDf}
\gl{\nA}
\bmng
 shuMThi berxDuDx; shuMThi hAkida sakakxre pAkadiMda mADida oMdu terana keVku. 
\emng

\noindent
\gl{\nuga}
\bmng
 \eng{take the} \hyperlink{gilt(2) nuga}{$^2$gilt} \eng{off the gingerbread}. 
\emng
\eentry

\bentry
\word[gingerbread(2)]{gingerbread}
\pron{jiMjarfberxDf}
\gl{\gu}
\bmng
atayxMkArada; thaLukina; beDagina. 
\emng
\eentry


\bentry
\wordnospeech{gingerbread nut}{gingerbread nut}
\pron{?}
\gl{\nA}
\bmng
 shuMThiberxDf bilelx; shuMThi keVkina guMDiyAkArada saNaNx bilelx. 
\emng
\eentry

\bentry
\wordnospeech{ginger group}{ginger group}
\pron{?}
\gl{\nA}
\bmng
 (\birx) shuMThitaMDa; shuMThi baNa; pakaSxvanunx yA caLavaLiyanunx hecucx nidhARraka kAyaRkekx otAtxyapaDisuva sadasayxra guMpu, taMDa. 
\emng
\eentry

\bentry
\word[gingerly(1)]{gingerly}
\pron{jiMjarfli}
\gl{\kirxvi}
\bmng
 (tanagAgali, tAnu muTiTxdudakAkxgali, meTiTxdudakAkxgali, apAyavAgadaMte, savxlapxvU shabadxvAgadaMte) atayxMta joVkeyiMda; bahaLa joVpAnavAgi; bahu ecacxrikeyiMda. 
\emng
\eentry

\bentry
\word[gingerly(2)]{gingerly}
\pron{jiMjarfli}
\gl{\gu}
\bmng
 atayxMta joVkeya; bahaLa joVpAnavAda; bahu ecacxrikeya. 
\emng
\eentry

\bentry
\word{ginger-nut}
\pron{jiMjarfnaTf}
\gl{\nA}
\bmng
  = \hyperlink{gingerbread nut}{gingerbread nut}. 
\emng
\eentry

\bentry
\word{ginger-pop}
\pron{jiMjarfpApf}
\gl{\nA}
\bmng
 (\AmA)  = \hyperlink{ginger-ale}{ginger-ale}. 
\emng
\eentry

\bentry
\word{ginger-race}
\pron{jiMjarfreVsf}
\gl{\nA}
\bmng
 shuMThi-kone, beVru. 
\emng
\eentry

\bentry
\word{ginger-snap}
\pron{jiMjarfsAnxYxpf}
\gl{\nA}
\bmng
 shuMThiya ruci kaTiTxda, sulaBavAgi oDeyuva, oMdu bageya bisakxtutx. 
\emng
\eentry

\bentry
\wordnospeech{ginger wine}{ginger wine}
\pron{?}
\gl{\nA}
\bmng
 shuMThi madayx; sakakxrege niVru hAki, jajijxda shuMThi berasi, huLisi mADida, oMdu tarahada birxTiSf madayx. 
\emng
\eentry

\bentry
\word{gingery}
\pron{jiMjari}
\gl{\gu}
\bmng
\bnum
\num{1} shuMThiya guNavuLaLx. 
\num{2} shuMThi baNaNxda. 
\num{3} satavxshAli; kecucxLaLx; hurupina; humamxsisxna; viVyaRvatAtxda: \eng{in gingery good health} oLeLxya viVrayxvatAtxda AroVgayxdalilx. 
\enum
\emng
\eentry

\bentry
\word{gingham}
\pron{giMgamf}
\gl{\nA}
\bmng
\bnum
\num{1} giMgamf (baTeTx); baNaNx hAkida hatitxya dArada yA nArina (linanf) dArada paTeTxgaLoV papapxLigaLoV uLaLx, oMdu tarada baTeTx. 
\num{2} (\AmA) koDe; Catirx; Atapatarx. 
\enum
\emng
\eentry

\bentry
\word{gingili}
\pron{jiMjili}
\gl{\nA}
\bmng
\bnum
\num{1} eLuLxgiDa. 
\num{2} eLuLx (kALu). 
\num{3} eLeLxNeNx. 
\enum
\emng
\eentry

\bentry
\word{gingival}
\pron{jiMjeYvalf}
\gl{\gu}
\bmng
 osaDina; osaDugaLige saMbaMdhisida. 
\emng
\eentry

\bentry
\word{gingivitis}
\pron{jiMjiveYTisf}
\gl{\nA}
\bmng
 osaDina uriyUta. 
\emng
\eentry

\bentry
\word{gingko}
\pron{jiMkoV}
\gl{\nA}
\bmng
  = \hyperlink{ginkgo}{ginkgo}. 
\emng
\eentry

\bentry
\word{ginglymus}
\pron{giM(jiM)gilxmasf}
\gl{\nA}
\expl{(\bava\ \eng{ginglymi} \ucAcx\ giM(jiM)gilxmeY).}
\bmng
(\aMrashA) jiMgilxmasf; Ekatalacali kiVlu; (moNakeY kiVlinaMte) oMdu samataladalilx mAtarx calane sAdhayxviruva yAvudeV kiVlu. 
\emng
\eentry

\bentry
\word{gink}
\pron{giMkf}
\gl{\nA}
\bmng
 (\ashi) (sAmAnayxvAgi \hiV) isamu; manuSayx; AsAmi. 
\emng
\eentry

\bentry
\word{ginkgo}
\pron{giM(jiM)koV}
\gl{\nA}
\expl{(\bava\ \eng{ginkgos} yA \eng{ginkgoes}).}
\bmng
 jiMkoV; biVsaNigeyAkArada elegaLuLaLx, ciVnA japAnugaLalilxna oMdu jAtiya mara. 
\emng
\eentry

\bentry
\word{gin-mill}
\pron{jinfmilf}
\gl{\nA}
\bmng
 (\ashi) kuDitada mane; paDaKAne; bAru. 
\emng
\eentry

\bentry
\word{gin-palace}
\pron{jinfpAYxli(la)sf}
\gl{\nA}
\bmng
 atayxlaMkArada, beDagina -- madayxshAle, madayxda aMgaDi. 
\emng
\eentry

\bentry
\wordnospeech{gin rummy}{gin rummy}
\pron{?}
\gl{\nA}
\bmng
 oMdu bageya ramimx (isipxVTATa). 
\emng
\eentry

\bentry
\word{ginseng}
\pron{jinfseMgf}
\gl{\nA}
\bmng
 (ciVnA, neVpALa, kenaDa, utatxra amerika parxdeVshagaLalilx beLeyuva) oMdu auSadha sasayxda beVru. 
\emng
\eentry

\bentry
\wordnospeech{gin sling}{gin sling}
\pron{?}
\gl{\nA}
\bmng
 ruci vAsane kaTiTx, sihi beresida, (amerikada) oMdu bageya taNaNxneya `jinf' madayx. 
\emng
\eentry

\bentry
\word{Gioconda}
\pron{joV(jA)kAMDa}
\gl{\gu}
\bmng
 (nagu \mo vugaLa \vi) nigUDhavAda; rahasAyxthaRgaBiRtavAda. 
\emng
\eentry

\bentry
\word{gippo}
\pron{jipoV}
\gl{\nA}
\bmng
 (\birx) (seYnayx \ashi) mAMsada -- sAru, rasa. 
\emng
\eentry

\bentry
\word{gippy}
\pron{jipi}
\gl{\nA}
\bmng
\bnum
\num{1} (seYnayxa \ashi) IjipiTxna seYnika. 
\num{2} IjipiTxna sigareVTu. 
\enum
\emng
\eentry

\bentry
\wordnospeech{gippy tummy}{gippy tummy}
\pron{?}
\gl{}
\bmng
 jipipxBeVdi; parxvAsi BeVdhi; uSaNxdeVshagaLige parxvAsigaru hoVdAga avarige baruva atisAra, BeVdi. 
\emng
\eentry

\bentry
\word{gipsy}
\pron{jipisx}
\gl{\nA}
\bmng
 \eng{gypsy} enunxvudara rUpAMtara. 
\emng
\eentry

\bentry
\word{giraffe}
\pron{ji(ja)rAYxphf}
\gl{\nA}
\bmng
 jirAphe; Aphirxkada macecx camaRda roVmaMtha catuSApxdi. 
\emng
\eentry

\bentry
\word{girandole}
\pron{jiraMDoVlf}
\gl{\nA}
\bmng
\bnum
\num{1} cakarxbANa; sututxva cakarxdiMda hArisuva AkAshabANa. 
\num{2} cakarxkAraMji; sututxtitxruva -- kAraMji, bugegx, jaladhAre. 
\num{3} kavaloDeda moVMbatitx kaMba; kavaloDediruva moVMbatitxya gujujx baMka yA kaMba. 
\num{4} tATaMka; madhayxdalilx doDaDx haraLiTuTx sutatxlU cikakx haraLugaLanunx hAkiruva oMdu namUneya Ole, hatatxkaDaku, yA loVlAku. 
\enum
\emng
\eentry

\bentry
\word{girasol}
\pron{jirasAlf}
\gl{\nA}
\bmng
 jirasAlf; sUyaRkAMta(maNi); keMpu hoLapanunx biVruva oMdu tarada kiSxVra saPxTikamaNi. 
\emng
\eentry

\bentry
\word{girasole}
\pron{jirasoVlf}
\gl{\nA}
\bmng
  = \hyperlink{girasol}{girasol}. 
\emng
\eentry

\bentry
\word[gird(1)]{gird}
\pron{gaDfR}
\gl{\sakirx}
\expl{(\BU\ matutx \BUkaq\ \eng{girded} yA \eng{girt}).}
\bmng
(sAhitayxka) 
\bnum
\num{1} (soMTavanunx) paTiTxniMda -- kaTuTx, sututxvari. 
\num{2} (naDupaTiTxyiMda \kanmu\ uDupanunx baMdhisalu, vayxkitxya) soMTakekx kaTuTx, bigi; naDuvige kaTuTx. 
\num{3} (\pArxparx) shakitx niVDu, odagisu; adhikAra koDu: \eng{thou hast girded me with strength into the battle} yudadhx mADalu niVnu nanage shakitx odagisididxVye. 
\num{4} (yArobabxnige) katitxyuLaLx naDupaTiTx toDisu. 
\num{5} (katitxyanunx) naDupaTiTxge hAku, seVrisu. 
\num{6} (naDupaTiTx, DAbu, \mo vugaLiMda) sututxvari. 
\num{7} (naDukaTuTx, samudarx, beVli, \mo vugaLa \vi) sututxgaTuTx; sututxvari; baLasu. 
\num{8} (paTaTxNa \mo vanunx) AkarxmaNakArariMda yA koVTekotatxlagaLiMda sututxvari, sututxgaTuTx. 
\enum
\emng

\noindent
\gl{\nuga}
\bmng
\hypertarget{grid nuga1}{} 
\bnum
\num{1} \eng{gird oneself (up)} soMTakaTiTx nilulx; kAyaRkekx sidadhxnAgu. 
\num{2} \eng{gird one's loins} = \hyperlink{grid nuga1}{?nuga? \((1)\)}. 
\num{3} \eng{gird up} = \hyperlink{gird(1)}{$^1$gird \((1, 2)\)}. 
\enum
\emng
\eentry

\bentry
\word[gird(2)]{gird}
\pron{gaDfR}
\gl{\akirx}
\bmng
 hAsayx, nakali, geVli, parihAsa -- mADu; aNakisu: \eng{girding at the wrong-headedness of the officials} adhikArigaLa vakarxtanavanunx geVlimADutatx. 
\emng
\eentry

\bentry
\word[gird(3)]{gird}
\pron{gaDfR}
\gl{\nA}
\bmng
 hAsayx(da mAtu); nakali; geVli; parihAsa. 
\emng
\eentry

\bentry
\word{girder}
\pron{gaDaRrf}
\gl{\nA}
\bmng
\bnum
\num{1} (mALigeya jaMti yA tarAyigaLige AdhAravAgiruva) tole; dUla; sara. 
\num{2} (ideV upayoVgada) kabibxNada yA ukikxna tole, sara. 
\num{3} (cAvaNi seVtuveya dATu yA kaNuNxgaLige hAkuva, ukukx \mo vugaLiMda mADida, jAlAMtarada yA itara saMyukatx racaneya) sarakaTuTx. 
\enum
\emng
\eentry

\bentry
\word[girdle(1)]{girdle}
\pron{gaDfRlf}
\gl{\nA}
\bmng
\bnum
\num{1} naDukaTuTx; naDupaTiTx; uDidAra; DAbu; meVKale; oDAyxNa; kaTibaMdha; kaTisUtarx. 
\num{2} oLauDupu; oLakavaca; paqSaThx, hoTeTxgaLige AkArakoTuTx saNaNxdAgi kANuvaMte mADalu heMgasaru dharisuva, soMTadavarege baruva, haguravAda, elAsiTxkf hAkida oLa uDupu. 
\num{3} maMDala; veVSaTxna; naDupaTiTxyaMte sututxvaridiruvudu. 
\num{4} (sANe hiDida, cinanxda kaTaTxDa sututxvarida) ratanxda naDu aMcu, BAga. 
\num{5} (\aMrashA) asithxcakarx; toVLu matutx kAlugaLige AdhAravAgiruva, hecucx kaDime saMpUNaR cakArxkaqtiya mULe: \eng{shoulder girdle} BujAsithx cakarx. 
\num{6} sututxpaTeTx; marada sutatxlU togaTe tegedu mADiruva paTiTx, paTeTx. 
\enum
\emng
\eentry

\bentry
\word[girdle(2)]{girdle}
\pron{gaDfRlf}
\gl{\sakirx}
\bmng
\bnum
\num{1} (naDukaTiTxniMda, sututxvalayadiMda) sututxgaTuTx; sututxvari; baLasu; naDukaTuTx, DAbu -- hAku. 
\num{2} sututx paTiTxyanunx ketitx maravanunx hecucx PalabiDuvaMte maDu. 
\enum
\emng
\eentry

\bentry
\word[girdle(3)]{girdle}
\pron{gaDfRlf}
\gl{\nA}
\bmng
 (sAkxTelxMDf \parx) (roTiTx, doVse, \mo vanunx suDuva) guMDukAvali; duMDuheMcu. 
\emng
\eentry

\bentry
\word{girdle-cake}
\pron{gaDaRlfkeVkf}
\gl{\nA}
\bmng
 (duMDu) kAvali -- doVse, roTiTx. 
\emng
\eentry

\bentry
\word{girdler}
\pron{gaDalxRrf}
\gl{\nA}
\bmng
 (\pArxparx) DAbugAra; DAbu, naDupaTiTx -- tayArisuvava. 
\emng
\eentry

\bentry
\word{girl}
\pron{galfR}
\gl{\nA}
\bmng
\bnum
\num{1} heNuNx (magu); huDugi. 
\num{2} inUnx maduveyilalxda heMgasu; koDagUsu; kaneyx; bAlike; bAle. 
\num{3} manegelasadavaLu; kelasagititx. 
\num{4} (obabxna) perxVyasi; nalelx. 
\num{5} heNuNx; sitxrXV: \eng{girl friend} (huDugana yA gaMDasina) heMgeLeti; saMgAti; jategAtiR; senxVhite. 
\num{6} (kaCeVri, aMgaDi, kAKARne, \mo vugaLalilx) kelasamADuva heMgasu. 
\num{7} kAyaRdashiRNi yA sahAyaki. 
\enum
\emng

\noindent
\gl{\pagu}
\bmng
\bnum
\num{1} \eng{best girl} (obabxna) perxVyasi; nalelx. 
\num{2} \eng{girl} \hyperref{kandict_f.pdf}{F}{friday(1)pagu2}{$^1$friday}. 
\num{3} \eng{les girls} (\ucAcx\ --leV galfsxR). bAleyaru; huDugiyaru; meVLabAleyaru; \kanmu\ samUhagAnada yA samUha naqtayxda huDugiyaru. 
\num{4} \eng{old girl} mudi huDugi (heMgasu, heNuNxkudure, \mo vugaLa \vi\ pirxVti yA agwrava sUcakavAda mAtu, yA saMboVdhaneyAgi \parx). 
\numi{5} \eng{the girls} 
\banum
\alnum{a} (oMdu kuTuMbada beLeda) heNuNx makakxLu. 
\alnum{b} (\kanmu\ samAnAsakitxyuLaLx) heMgasaru; sitxrXVyaru. 
\eanum
\numie
\enum
\emng
\eentry

\bentry
\word{girldom}
\pron{galfRDamf}
\gl{\nA}
\bmng
\bnum
\num{1} sitxrXVloVka; huDugiyara parxpaMca, jagatutx. 
\num{2} sitxrXVyaru; huDugiyaru. 
\enum
\emng
\eentry

\bentry
\wordnospeech{girl guide}{girl guide}
\pron{?}
\gl{\nA}
\bmng
 galfRgeYDf; bAlikAcamU; bAyfswkxTfsx eMba bAlacamU saMsethxge samanAda, bAlikAcamU saMsethxya sadaseyx. 
\emng
\eentry

\bentry
\word{girlhood}
\pron{galfRhuDf}
\gl{\nA}
\bmng
\bnum
\num{1} kanenxtana; kanAyxtavx; kwmAyaR; kanAyxvasethx. 
\num{2} (sAmUhikavAgi) huDugiyaru; bAlikeyaru. 
\enum
\emng
\eentry

\bentry
\word[girlie(1)]{girlie}
\pron{galiR}
\gl{\nA}
\bmng
\bnum
\num{1} (hecucx salige, pirxVti toVrisuvAga) puTaTx huDugi. 
\num{2} (\ashi) veVsheyx; sULe. 
\enum
\emng
\eentry

\bentry
\word[girlie(2)]{girlie}
\pron{galiR}
\gl{\gu}
\bmng
 (patirxke \mo vugaLa \vi) huDugiyara; taruNiyara; atayxMta kaDime baTeTx hAkiruva yA baTeTxyeV ilalxda taruNiyara kAmaparxcoVdaka citarxgaLiruva. 
\emng
\eentry

\bentry
\word{girlish}
\pron{galiRSf}
\gl{\gu}
\bmng
\bnum
\num{1} huDugiya. 
\num{2} huDugitanada. 
\num{3} huDugiyaMtha yA huDugitanadaMtha; kanAyxsahaja. 
\enum
\emng
\eentry

\bentry
\word{girlishly}
\pron{galiRSfli}
\gl{\kirxvi}
\bmng
\bnum
\num{1} huDugiyaMte. 
\num{2} kanAyxsahajavAgi. 
\enum
\emng
\eentry

\bentry
\word{girlishness}
\pron{galiRSfnisf}
\gl{\nA}
\bmng
 huDugitana. 
\emng
\eentry

\bentry
\wordnospeech{girl scout}{girl scout}
\pron{?}
\gl{\nA}
\bmng
 (\ame)  = \hyperlink{girl guide}{girl guide}. 
\emng
\eentry

\bentry
\word{girly-girly}
\pron{galiRgaliR}
\gl{\gu}
\bmng
 ati huDugitanada; ati heNiNxgatanada; atiyAgi yA kaqtakavAgi huDugiyaMte ADuva. 
\emng
\eentry

\bentry
\word{giro}
\pron{jeYroV}
\gl{\nA}
\expl{(\bava\ \eng{giros}).}
\bmng
jeYroV; bAyxMkugaLu, aMcekaceVrigaLu, \mo vugaLu parasapxra haNavagARvaNe mADuva oMdu vidhAna. 
\emng
\eentry

\bentry
\word[Girondist(1)]{Girondist}
\pron{jirAMDisfTx}
\gl{\gu}
\bmng
\bnum
\num{1} jirAMDisfTx pakaSxda; (\eng{1791--93}ra kAlada) pherxMcf shAsana saeyalilxdadx parxjAdhipatayx sakARrada maMdagAmi pakaSxda. 
\num{2} jirAMDisfTx pakaSxda aBipArxyagaLuLaLx. 
\enum
\emng
\eentry

\bentry
\word[Girondist(2)]{Girondist}
\pron{jirAMDisfTx}
\gl{\nA}
\bmng
jirAMDisfTx: 
\banum
\alnum{a} (\eng{1791--93}ra naDuvaNa) pherxMcf parxjAdhipatayx sakARrada maMdagAmi pakaSxdavanu. 
\alnum{b} jirAMDisfTx pakaSxda aBipArxyagaLuLaLxvanu. 
\eanum
\emng
\eentry

\bentry
\word[girt(1)]{girt}
\pron{gaTfR}
\gl{\nA}
\bmng
  = \hyperlink{girth(1)}{$^1$girth}. 
\emng
\eentry

\bentry
\word[girt(2)]{girt}
\pron{gaTfR}
\gl{\sakirx}
\bmng
  = \hyperlink{girth(2)}{$^2$girth}. 
\emng
\eentry

\bentry
\word[girt(3)]{girt}
\pron{gaTfR}
\gl{\sakirx}
\bmng
 \eng{gird} dhAtuvina \BU\ matutx \BUkaq\ rUpa. 
\emng
\eentry

\bentry
\word[girth(1)]{girth}
\pron{gatfR}
\gl{\nA}
\bmng
\bnum
\num{1} TaMgu; jiVnupaTiTx; (jiVnu \mo vanunx BadarxpaDisalu kudure \mo vugaLa) oDala sutatx bigiyuva togalina yA baTeTxya paTiTx. 
\num{2} (hecucx kaDime satxMBAkaqtiyAgiruva yAvudeV vasutxvina) sutatxLate; ububxtagugxgaLanunx seVrisikoMDu mADida, capapxTeyalalxda keSxVtarxda aDaDxLate yA sutatxLate. 
\enum
\emng
\eentry

\bentry
\word[girth(2)]{girth}
\pron{gatfR}
\gl{\sakirx}
\bmng
\bnum
\num{1} sututx; sututxvari; sututxkaTuTx. 
\num{2} (kudure \mo vugaLige) sututxpaTiTx hAki (jiVnu \mo vugaLige) bigi, kaTuTx. 
\enum
\emng

\noindent
\gl{\akirx}
\bmng
 (iSuTx) sutatxLateyAgu. 
\emng
\eentry

\bentry
\word{girth-web}
\pron{gatfRvebf}
\gl{\nA}
\bmng
 (sututxpaTiTxgAgi heNedu mADida) heNigeya paTiTx. 
\emng
\eentry

\bentry
\word{gismo}
\pron{gisoZmxV}
\gl{\nA}
\expl{(\bava\ \eng{gismos}).}
\bmng
(\ashi) 
\bnum
\num{1} salakaraNe. 
\num{2} taMtarx; hUTa. 
\enum
\emng
\eentry

\bentry
\word{gist}
\pron{jisfTx}
\gl{\nA}
\bmng
\bnum
\num{1} muKAyxMsha; oMdu viSayada sArAMsha, tiruLu. 
\num{2} (\nAyxshA) kAnUnu karxma \mo vugaLige nijavAda, vAsatxvikavAda AdhAra. 
\enum
\emng
\eentry

\bentry
\word{git}
\pron{giTf}
\gl{\nA}
\bmng
 (\birx) (\hiV) niSapxrXyoVjaka; aparxyoVjaka; kelasakekx bArada vayxkitx. 
\emng
\eentry

\bentry
\word{gittern}
\pron{giTanfR}
\gl{\nA}
\bmng
 giTanfR; karuLiniMda tayArisida taMti kaTiTxda, pUvaRkAlada oMdu giTArf vAdayx. 
\emng
\eentry

\bentry
\word[give(1)]{give}
\pron{givf}
\gl{\kirx}
\expl{(\BU\ \eng{gave} \ucAcx\ geVvf; \BUkaq\ \eng{given}).}


\noindent
\gl{\sakirx}
\bmng
\bnum
\num{1} koDu; niVDu; Iyu. 
\num{2} ucitavAgi koDu; bahumAnavAgi -- niVDu, dayapAlisu. 
\num{3} (vasutx \mo vanunx vAsatxvavAgi koTuTx yA koDadeyeV) sAvxmayx koDu; vashapaDisu; sAvxdhiVna mADu. 
\num{4} (pirxVti, senxVha, gwrava, \mo vanunx) niVDu; dayapAlisu; anugarxhisu; karuNisu; parxdAna mADu: \eng{give me the liberty} nanage sAvxtaMtarxyXvanunx dayapAlisu. 
\num{5} (tananx haqdaya, pirxVti, vishAvxsa, naMbike) samapiRsu; koDu; niVDu; odagisu: \eng{he does not readily give a stranger his confidence} avanu aparicitaralilx kUDale vishAvxsa iDuvudilalx. 
\num{6} (deVvaru \mo vara \vi) (shakitx, sAmathayxR, \mo vanunx) karuNisu; anugarxhisu; salilxsu; koDu. 
\num{7} Asitx biDu; Asitx -- baredukoDu, vahisikoDu: \eng{I give and devise my polyglot Bible} nAnu nananx bahuBASegaLa beYbalf garxMthavanunx AsitxyAgi biDutetxVne. 
\num{8} (\sA\ \eng{in marriage} jote \parx) (magaLu \mo varanunx) maduvemADikoDu: \eng{he would not give his daughter in marriage to a stranger} aparicitanige magaLanunx maduve mADikoDalu avanu opapxlilalx. 
\num{9} (sAvxmayxkekx saMbaMdhisiradaMte) vahisikoDu. 
\num{10} (sudidx, shuBAshaya, \mo vugaLanunx) koDu; talupisu; muTiTxsu. 
\num{11} (AhAra \mo vanunx) baDisu; niVDu. 
\num{12} (auSadhi) koDu; niVDu. 
\num{13} opipxsu; vahisu; apiRsu; suPadiRge koDu; vashakekx koDu: \eng{give into custody} poliVsana vashakekx koDu. \eng{give in charge} vashakekx vahisu. 
\num{14} BASe koDu; mAtukoDu; ANe iDu; parxmANamADu: \eng{give one's word, honour, etc.} tananx mAtu koDu; tananx gwravada meVle ANe iDu. \eng{I gave them the word of a sailor} nAnu avarige nAvikana vacana koTeTx. 
\num{15} (yAvudeV padAthaR yA haNavanunx) parxtiyAgi koDu; badalige koDu; vinimayavAgi koDu: \eng{what will you give for my car} nananx kArige parxtiyAgi niVnu Enanunx yA eSuTx haNavanunx koDutitxVye? 
\num{16} haNakoDu; pAvatimADu; baTavADemADu; salilxsu; saMdAyamADu. 
\num{17} (belege) mAru; mArATamADu. 
\num{18} mIsaliDu; muDupiDu; opipxsibiDu; apiRsu; samapiRsu; niveVdisu; viniyoVgisu: \eng{gave his life to the nation} rASaTxrXkAkxgi tananx bALanunx muDupiTaTx, samapiRsida. \eng{much given to these pursuits} I havAyxsagaLige bahaLa niratavAgi. 
\num{19} koDu; (\kanmu\ inonxbabxna meVle) pariNAma uMTumADalu kAyaR yA parxyatanx mADu: \eng{give him a kick} avanige oMdu lAta, ode koDu. \eng{give orders} Ajecnx koDu. \eng{give person one's blessing} (vayxkitxyanunx) AshiVvaRdisu; harasu. \eng{give you joy} (deVvaru) ninage saMtoVSa koDali; ninage saMtoVSavAgali. 
\num{20} (\AmA) (\kanmu\ opipxgeyAgadiruva yAvudanenxV) heVLu; koDu. 
\num{21} (tiVpuR \mo vanunx adhikArapUvaRkavAgi) koDu; niVDu; heVLu; tiLisu: \eng{give batsman out or not out} (kirxkeTiTxna aMpeYrina \vi) bAyxTugArananunx auTAdaneMdu yA auTAgalilalxveMdu heVLu. \eng{give the case for or against person} vayxkitxya paravAgi yA virudadhxvAgi tiVpuRkoDu. 
\num{22} (\BUkaq dalilx) (dAKale patarxda \vi) tAriVKu hAku, koDu. 
\num{23} AtitheVyanAgi (naqtayxkUTa, saMtoVSakUTa, BoVjanagaLanunx) EpaRDisu: \eng{intended giving the young ladies a ball} taruNiyarige naqtayxkUTavanunx EpaRDisalu udedxVshisidadxnu. 
\num{24} koDu; niVDu; oDuDx: \eng{give person one's hand} keYniVDu; hasatxlAGava koDu. 
\num{25} horageDahu; parxkaTisu; parxdashiRsu; kANisu; toVrisu: \eng{Prajavani gives the facts} parxjAvANi patirxke naDeda saMgatigaLanunx horageDahutatxde, parxkaTisutatxde. \eng{gives no sign of life} badukiruva cihenxyanenxV toVruvudilalx. \eng{thermometer gives} \eng{$80^\circ$} \eng{in the shade} tApamApakavu neraLinalilx \eng{$80^\circ$} toVrisutatxde. 
\num{26} (yAvudeV paThayxBAga, nATaka, \mo vanunx) Odu, vAcanamADu, hADu, aBinayisu, naTisu, ADu yA parxdashiRsu: \eng{he promised to give another chapter out of his book} avanu tananx pusatxkadiMda inonxMdu adhAyxyavanunx OduvudAgi mAtukoTaTxnu. \eng{who will give us a song?} namamx muMde yAru hADu heVLutAtxre? \eng{the opera was given again} giVtanATakavanunx matotxmemx parxdashiRsalAyitu. 
\num{27} pAlogxLuLxvaMte mADu; BAgiyAgisu; haMcu: \eng{give me his sore throat} avana gaMTala noVvanunx nanage koTaTx. 
\num{28} mUlavAgu; kAraNavAgu; mUla odagisu: \eng{gave its name to the battle} yudadhxkekx adara hesaru baMditu (mUlavAyitu). 
\num{29} (pAleMdu, pAlige, vashakekx) koDu: \eng{give him the best room available} iruvudaralilx atuyxtatxmavAda koThaDiyanunx avanige koDu. \eng{he was given the contract} avanige gutitxge koDalAyitu. 
\num{30} kare; hesaru koDu; nAmakaraNa mADu: \eng{he gave the child the name John} avanu maguvanunx jAnf eMdu kareda. 
\num{31} AroVpisu; seVridudeMdu -- heVLu, tiLisu, eNisu: \eng{a good argument for giving the painting to Rembrandt} vaNaRcitarxvanunx reMbArxMTanadeMdu heVLalu oLeLxya vAda. 
\num{32} (nijaveMdu) iTuTxko; eNisu; BAvisu; tiLiduko: \eng{given good health, the work is possible} AroVgayx cenAnxgiruvudeMdu iTuTxkoDare I kelasa sAdhayx. 
\num{33} (PalavAgi, utapxnanxvAgi) niVDu; koDu: \eng{analysis gives the following figures} vishelxVSaNeyu I muMdina aMki saMKeyxgaLanunx koDutatxde; vishelxVSaNeyiMda I muMdina aMkisaMKeyxgaLu horaDutatxve. \eng{lamp gives a dim light} diVpa maMda beLakanunx niVDutatxde. 
\num{34} uMTumADu; eDegoDu; Asapxda -- koDu, kalipxsu; kAraNavAgu; avakAshamADikoDu: \eng{solitude gives it its only charm} adara sobagige EkAMtate oMdeV kAraNa. \eng{gave me much pain} nanage bahaLa noVvuMTumADitu. \eng{this gives him a right to complain} idu avanige dUru koDuva avakAsha niVDutatxde. \eng{gave myself an hour to go there} alilxge hoVgalu oMdu gaMTe (avakAsha) koTuTxkoMDe. \eng{give him one minute start} avanige oMdu nimiSa muMce hoVguva avakAsha koDu. 
\enum
\emng

\noindent
\gl{\akirx}
\bmng
\bnum
\num{1} dAnakoDu; dhamaRmADu; BikeSx yA koDugeyanunx niVDu. 
\num{2} tanage tiLidiruvudanunx heVLu, tiLisu. 
\num{3} kusi; kusidu biVLu; otatxDakekx bagugx: \eng{the haystack is giving} hululxmede kusiyutitxde. 
\num{4} bigi kaLeduko; saDilavAgu: \eng{rendered useless by his knee giving} avana maMDi saDilavAgi parxyoVjanakekx barada hAgAyitu. 
\num{5} sathxLa koDu; avakAsha niVDu; Asapxda koDu; eDegoDu. 
\num{6} kugugx; muduriko; suruTiko; sukAkxgu: \eng{like seasoned timber, never gives} hadagoLisida cwbiVneyaMte eMdigU sukAkxguvudilalx. 
\num{7} (kiTaki \mo vugaLa \vi) daqshayx kANu, kANisu, tere: \eng{no window giving on to the street} biVdiya kaDe kANisuva kiTakiyilalx. 
\num{8} (hAdi, ONi, \mo vugaLa \vi) koMDoyuyx; seVru: \eng{the road which gave on to the highway} rAjamAgaRkekx seVruva dAri. 
\num{9} (\AmA) Agu; Agutitxru; uMTAgu; saMBavisu: \eng{what gives?} EnAgutitxde? 
\enum
\emng

\noindent
\gl{\pagu}
\bmng
\bnum
\numi{1} \eng{give away} 
\banum
\alnum{a} (dAnavAgi) koTuTxbiDu: \eng{he gave away most of his income} tananx varamAnada bahuBagavanunx avanu dAnavAgi koTuTxbiTaTx. 
\alnum{b} kanAyxdAna mADu; maduve mADikoDu: \eng{I gave her away} nAnu avaLanunx maduvemADikoTeTx. 
\alnum{c} bayalu mADu; biTuTx koDu; bahiraMgagoLisu; parxkaTapaDisu: \eng{give away the show} koMdukorategaLanunx toVragoDu; poLuLxtanavanunx bayalu mADu. 
\alnum{d} nagegiVDu mADu; apahAsayxkekx gurimADu. 
\alnum{e} patetx sikukxvaMte mADu. 
\alnum{f} (bahumAna) haMcu; vitaraNe mADu; viniyoVgamADu. 
\alnum{g} tAyxga mADu; inonxbabxnigAgi biTuTxkoDu: \eng{he gave away a good chance of winning the match} avanu paMdayx gelulxva oLeLxya avakAshavanunx biTuTxbiTaTx, biTuTxkoTaTx. 
\eanum
\numie
\num{2} \eng{give back} hiMdakekx -- koDu, opipxsu; sAvxdhiVnakekx hiMdirugisu. 
\numi{3} \eng{give birth to} 
\banum
\alnum{a} heru; haDe; Iyu; parxsavisu; janamx koDu. 
\alnum{b} (\rUpa) pariNAmavAgu; Palisu; uMTumADu: \eng{her hobby gave birth to a successful business} avaLa havAyxsa yashasivxyAda oMdu udoyxVgadalilx Palisitu. 
\eanum
\numie
\num{4} \eng{give chase} benanxTiTx horaDu; aTiTxkoMDu hoVgu. 
\numi{5} \eng{give down} 
\banum
\alnum{a} (hasuvina \vi) soru biDu; hAlanunx kecacxliniMda biDu. 
\alnum{b} (hasuvina hAlina \vi) kecacxliniMda hari, suri. 
\eanum
\numie
\num{6} \eng{give ear} keVLu; kivigoDu. 
\numi{7} \eng{give forth} 
\banum
\alnum{a} horasUsu; horabiDu; horacelulx: \eng{the fields give forth an odour of spring} holagaLu vasaMtada kaMpanunx horasUsutatxve. 
\alnum{b} parxkaTisu; sAru parxsidadhxpaDisu; varadimADu: \eng{the king gave forth a proclamation} dore GoVSaNeyanunx parxkaTisidanu. 
\eanum
\numie
\num{8} \eng{give ground} himemxTuTx; hiMdakekx -- sari, hoVgu: \eng{the enemy was beginning to give ground} shaturxvu himemxTaTxlu pArxraMBisidadxnu. 
\numi{9} \eng{give in} 
\banum
\alnum{a} (hoVrATa yA vAdadalilx) biTuTxkoDu; nililxsi soVlopipxko; maNi: \eng{they tire and give in} avaru AyAsagoMDu hoVrATa nililxsutAtxre. \eng{she did not give in on that point} A aMshada viSayadalilx avaLu vAda biTuTx koDalilalx. 
\alnum{b} (dasetxYvajanunx) sUkatx adhikArige talupisu, opipxsu, koDu: \eng{please give in your examination papers} dayaviTuTx nimamx pariVkASx patirxkegaLanunx koDi. 
\eanum
\numie
\num{10} \eng{give off} (Avi \mo vanunx) horasUsu; horabiDu; horage kaLuhisu. 
\num{11} \eng{give one best} (kamaRpadadoDane) (\AmA) obabxna hirimeyanunx opipxko. 
\num{12} \eng{give oneself trouble} toMdare -- vahisu, tegeduko. 
\num{13} \eng{give or take} (\AmA) (aMdAju niSakxqqSaTxvAdudakekx hatitxravAdudu eMdu BAvisabeVku enunxvalilx) kAla, parxmANa, haNada motatx, \mo vanunx seVrisu yA tege, kUDu, yA kaLe. 
\numi{14} \eng{give out} 
\banum
\alnum{a} tiLiyapaDisu; sAru; parxkaTisu; \eng{it was given out that the minister would be the chief speaker} maMtirxgaLeV parxdhAna BASaNakArareMdu parxkaTisalAgitutx. 
\alnum{b} horasUsu; horabiDu: \eng{the gold gave out its red glow} baMgAra tananx keMpu parxkAshavanunx horasUsitu. 
\alnum{c} haMcu; viniyoVgamADu: \eng{the king gave out the arms to them} doreyu avarige shasatxrXgaLanunx haMcidanu. 
\alnum{d} (vayxkitxgaLa \vi) tayxjisu; tore; dUraviDu; biTuTxbiDu; nililxsibiDu: \eng{he is willing rather to play small play, than to give out} nepamAtarxda ATavAdarU adanunx ADiyAnu, saMpUNaRvAgi biTuTxbiDalu avanige iSaTxvilalx. 
\alnum{e} (yaMtarx, avayava, \mo vugaLa \vi baLalike \mo vugaLiMda) keTuTx hoVgu; muridubiVLu; niMtuhoVgu; kusi: \eng{his eyes have given out} avana kaNuNxgaLu keTuTxhoVgive. 
\alnum{f} sAlade hoVgu; kaDimeyAgiru; sAkAgade hoVgu; mugidu hoVgu: \eng{before spring, their finances may give out} vasaMtakekx modaleV avara haNa mugiduhoVgabahudu. 
\eanum
\numie
\numi{15} \eng{give over} 
\banum
\alnum{a} mADade biTuTxbiDu; nililxsu: \eng{plese give over crying} dayaviTuTx aLu nililxsu. 
\alnum{b} (aBAyxsa \mo vanunx) keYbiDu; biDu; tore; tayxjisu; dUravAgu: \eng{they gave over the contest} avaru sapxdheRyanunx keYbiTaTxru. 
\alnum{c} koDu; opipxsu; vashamADu; sAvxdhiVnagoLisu: \eng{she gave over all her property to her son} Ake tananx elalx AsitxyanUnx maganige koTuTxbiTaTxLu. 
\alnum{d} toDagu; maganxnAgu; anurakatxnAgu; niratanAgu: \eng{give over to sports} ATapATagaLalilx maganxnAgi. 
\alnum{e} vashanAgu; adhiVnanAgu; biVLu: \eng{given over to despair or evil courses} nirAshege vashavAgi yA keTaTx cALigaLige bidudx. 
\hyperdef{G}{give(1) pagu(16)}{} 
\eanum
\numie
\num{16} \eng{give person to understand}, \eng{know, etc.} tiLisu; tiLiyaheVLu; BaravasekoDu; necicxke koDu: \eng{he gave me to understand that his intentions were honourable} avana udedxVshagaLu gwravayutaveMdu avanu nanage BaravasekoTaTxnu. 
\numi{17} \eng{give place to} 
\banum
\alnum{a} sathxLa biTuTxkoDu; jAgabiDu; avakAshakoDu; eDekoDu. 
\alnum{b} muMdakekx biDu; hiMdakekx sari; meVlina yA muMdina sAthxna biTuTxkoDu. 
\alnum{c} taLiLxhAkalapxDu; keLagina vayxkitx yA vasutx meVleVruvaMtAgu; taleya meVle kUrisalapxDu. 
\eanum
\numie
\num{18} \eng{give rise to} kAraNavAgu; AguvaMte mADu; uMTumADu: \eng{the industrial revolution gave rise to urbanization} audayxmika kArxMti nagariVkaraNakekx kAraNavAyitu. 
\num{19} \eng{give the time of day} aBinaMdisu; beLagegx, saMje, \mo vugaLa namasAkxra heVLu. 
\numi{20} \eng{give tongue} 
\banum
\alnum{a} (beVTenAyigaLa \vi) (\kanmu\ beVTeya vAsane baMdAga) bogaLu. 
\alnum{b} gaTiTxyAgi, elalxrigU keVLuvaMte -- heVLu, nuDi. 
\eanum
\numie
\numi{21} \eng{give up} 
\banum
\alnum{a} tore; tayxjisu; vajiRsu; biTuTxbiDu. 
\alnum{b} vashapaDisu; opipxsu; koTuTxbiDu; koDu; niVDu: \eng{give up a fortress} koVTeyanunx opipxsu. \eng{give up a seat in a crowded train} nUkunugagxlina reYlinalilx tananx jAgavanunx (inonxbabxnige) koDu. 
\alnum{c} visajiRsu; tayxjisu; beVpaRDu: \eng{give up ghost} sAyu; pArxNabiDu; tayxjisu. 
\alnum{d} (tapipxsikoMDu ODibaMdavanu \mo varanunx avana benanxTiTx baruvavaru \mo varige) opipxsu; vashapaDisu: \eng{give up the thief to the police} kaLaLxnanunx poliVsarige opipxsu. \eng{the escaped criminal gave himself up} tapipxsikoMDidadx aparAdhi tananxnunx tAnu opipxsikoMDu, tAneV vashanAda. 
\alnum{e} (BAva, uderxVka, \mo vugaLige) vashanAgu; adhiVnanAgu. 
\alnum{f} saMbaMdha -- tore, kaDiduko. 
\alnum{g} (yAvudeV parxyatanx yA kAyaRvanunx) biTuTxbiDu; keYbiDu; nililxsu: \eng{give up smoking} dhUmapAna biDu. 
\alnum{h} (AtAmxthaRka yA \BUkaq dalilx) opipxsiko; samapiRsiko; maganxnAgu; niratanAgu: \eng{give oneself up to studies} vAyxsaMgakekx apiRsiko; Odinalilx maganxnAgu. 
\alnum{i} (BAgigaLAdavara hesaru \mo vugaLanunx) tiLisu; bayalumADu; horageDahu. 
\alnum{j} aparihAyaRveMdu heVLu; bagehariyadedxMdu heVLu; parihAravAgadedxMdu tiLisu; samaseyxyanunx biDisalAgadedxMdu yA roVgavanunx guNapaDisalAgadedxMdu keYbiDu: \eng{the teachers gave up the child as incorrigible} magu tidadxlAgadedxMdu upAdhAyxyaru adara keYbiTaTxru. \eng{the doctors have given him up} avananunx guNapaDisalu AgadeMdu veYdayxru keYbiTiTxdAdxre. 
\alnum{k} (oMdara) AsebiDu; nirAshanAgu. 
\alnum{l} (\AmA) inunx (muMde) niriVkiSxsadiru; eduru noVDadiru: \eng{she was so late that we had given her up} avaLu eSuTx taDavAgidadxLeMdare, nAvu avaLanunx niriVkiSxsuvudanenxV biTuTxbiTiTxdedxvu. 
\eanum
\numie
\enum
\emng

\noindent
\gl{\nuga}
\bmng
\bnum
\num{1} \eng{give a} \hyperref{kandict_b.pdf}{B}{back(1) nuga(15)}{$^1$back}. 
\num{2} \eng{give a cry} kUgu. 
\num{3} \eng{give a good} \hyperref{kandict_a.pdf}{A}{account(2) nuga(7)}{$^2$account of oneself.} 
\num{4} \eng{give a jump} hAru; nege. 
\num{5} \eng{give a performance} (saMgiVta, nATaka, \mo vugaLa) parxdashaRna niVDu. 
\num{6} \eng{give a piece of one's} \hyperref{kandict_m.pdf}{M}{mind(1) pagu(16)}{$^1$mind}. 
\num{7} \eng{give a reading} vAcana mADu. 
\num{8} \eng{give a Roland for an Oliver} takakx javAbu koDu; pariNAmakAriyAda parxtijavAbu koDu; muyiyxge muyiyx tiVrisu; takakx parxtiVkAra mADu. 
\num{9} \eng{give as good as one gets} mAtige mAtu -- ADu, tirugisu; ETige ETu -- koDu, tirugisu; (muyiyxge) muyiyx tiVrisu. 
\num{10} \eng{give a song} hADu. 
\num{11} \eng{give child etc. something to cry for} niSAkxraNavAgi aLuva magu \mo vanunx -- hoDe, shikiSxsu, daMDisu; aLuvudakekx kAraNa odagisu. 
\num{12} \eng{give dinner} autaNakUTa, BoVjana samAraMBa -- EpaRDisu. 
\numi{13} \eng{give ground} 
\banum
\alnum{a} himemxTuTx; hiMjari. 
\alnum{b} eDegoDu; avakAshakoDu. 
\eanum
\numie
\num{14} \eng{give it him etc. hot} avanige bisi toVrisu; avanige shikeSx koDu; avananunx daMDisu. 
\hypertarget{giveme}{} 
\num{15} \eng{give me} (vidhirUpadalilx) nanage iSaTx, sari cenanx, mecicxke: \eng{give me the good old times} nanage haLeya kAlaveV iSaTx, cenanx; nAnu hiMdina A dinagaLanunx bayasutetxVne yA mecucxtetxVne. 
\num{16} \eng{give one his due} obabxnalilxrabahudAda guNagaLanunx opipxko, guNakekx purasAkxra niVDu. 
\num{17} \eng{give one his head} (kudureya \vi idadxMte) lagAmu biDu; sevxVceCxyAgi biDu; savxtaMtarxvAgi hoVgalu biDu. 
\num{18} \eng{give oneself} (heMgasina \vi) tananxnunx (gaMDisige) opipxsiko; saMBoVgakekx opipxsiko. 
\num{19} \eng{give oneself airs} doDaDxsitxke toVrisu; doDaDxtanada veVSahAku; soVgu hAku; parxtiSeThx toVrisu. 
\num{20} \eng{give person one's blessings} obabxnanunx harasu; obabxnige AshiVvARda mADu. 
\numi{21} \eng{give person one's hand} 
\banum
\alnum{a} (AsaregAgi) obabxnige keY niVDu. 
\alnum{b} hasatxlAGavakoDu. 
\alnum{c} obabxna keY hiDi; obabxnige sahAya mADu. 
\alnum{d} pANigarxhaNa mADu; maduveyAgu; keYhiDi. 
\eanum
\numie
\numi{22} \eng{give person what for} (\AmA) 
\banum
\alnum{a} daMDisu; shikiSxsu. 
\alnum{b} bayuyx; tegaLu. 
\eanum
\numie
\num{23} \eng{give the} \hyperref{kandict_b.pdf}{B}{boot(1) nuga(5)}{$^1$boot}. 
\num{24} \eng{give the} \hyperref{kandict_l.pdf}{L}{lie(3) pagu(3)}{$^3$lie to} 
\numi{25} \eng{give the mitten} (\ashi) 
\banum
\alnum{a} parxNayiyanunx tirasakxrisu; nirAkarisi kaLuhisi biDu. 
\hypertarget{give nuga25b}{} 
\alnum{b} (kelasadiMda) vajAmADu; tegeduhAku. 
\eanum
\numie
\num{26} \eng{give the sack} = \hyperlink{give nuga25b}{?nuga? \((25b)\)}. 
\num{27} \eng{give to the world} parxkaTisu; parxkAshamADu; parxsidadhxpaDisu: \eng{the results of these enquiries have been given to the world} I vicAraNegaLa PalitAMshagaLu parxkaTisalapxTiTxve. 
\num{28} \eng{give up the ghost} pArxNabiDu; maraNa hoMdu; sAyu. 
\numi{29} \eng{give way} 
\banum
\alnum{a} himemxTuTx; hiMdege; hoVrADadiru. 
\alnum{b} dArikoDu; sAthxna mADikoDu; jAga biTuTxkoDu; vayxkitxge,vasutxvige -- sAthxna biTuTxkoDuvaMtAgu: \eng{faith has given way to doubt} sharxdedhx saMshayakekx dArimADikoTiTxde. 
\alnum{c} (beVre vasutxviniMda) sAthxna biDuvaMtAgu; sAthxnacuyxtanAgu. 
\alnum{d} muridubiVLu; bidudxhoVgu; kusi; kitutxhoVgu: \eng{the rope gave way} hagagx kitutxhoVyitu. 
\alnum{e} (vayxkitxgaLa \vi) maNi; soVlopupx: \eng{when he gives way, he does it with so bad a grace} avanu soVlopupxvAga GanateyiMda soVlopupxvudilalx. 
\alnum{f} (duHKa \mo vakekx) vashavAgu; oLagAgu: \eng{he never gave way either to anger or alarm} avanu koVpakAkxgali BayakAkxgali eMdU vashavAgalilalx. 
\alnum{g} bele -- biVLu, iLi, kaDimeyAgu, kusi. 
\alnum{h} huTuTxhAkalu pArxraMBisu yA joVrAgi huTuTxhAku: \eng{the steersman should encourage the rowers to give way} taMDeVlanu doVNi naDesuvavaranunx joVrAgi huTuTx hAkuvaMte porxVtAsxhisabeVku. 
\eanum
\numie
\num{30} \eng{would give one's ears} = \hyperlink{give nuga31}{?nuga? \((31)\)}. 
\hypertarget{give nuga31}{} 
\num{31} \eng{would give the world} (oMdu vasutxvanunx paDeyalu) bayasida vasutxvigAgi, yA adu nijavAdare Enu beVkAdarU koTeTxVnu; Enu tAyxgavanAnxdarU mADiyeVnu; savaRsavxvanUnx dhAreyeredeVnu: \eng{many a girl would give the world to have such a complexion} eSoTxV huDugiyaru aMtha muKakAMtigoVsakxra Enu beVkAdarU koTATxru. 
\enum
\emng
\eentry

\bentry
\word[give(2)]{give}
\pron{givf}
\gl{\nA}
\bmng
 sithxtisAthxpakate; otatxDakekx sagugxva guNa: \eng{there is no give in a stone floor} kalilxna neladalilx sithxtisAthxpakate ilalx. 
\emng
\eentry

\bentry
\word{giveable}
\pron{givabflf}
\gl{\gu}
\bmng
 sagugxva; bagugxva; namayx. 
\emng
\eentry

\bentry
\fiveargs give and take(1){give and take}
\pron{?}
\gl{\akirx}
\bmng
 (mAtugaLu, ETugaLu, yA riyAyitigaLanunx) koTuTx tegeduko; koDukoLe mADu; parasapxra vinimaya mADu: \eng{willingness to give and take is important for success in social life} sAmAjika jiVvanadalilx yashasivxyAgalu koDukoLuLxva budidhx muKayx. 
\emng
\eentry

\bentry
\fiveargs give and take(2){give and take}
\pron{?}
\gl{\nA}
\bmng
\bnum
\num{1} (saMBASaNeyalilx) sarasasalAlxpa; sahaqdaya saMvAda. 
\num{2} (BAvane \mo vugaLalilx) vinimaya; koDukoLe. 
\num{3} koDukoLe; oDaMbaDike; rAji; sahakAra; parasapxra riyAyiti; eraDU kaDegaLavaralilx sagugxva manoVBAva: \eng{is not marriage a give and take affair?} maduve oMdu koDukoLeya, parasapxra sahakArada saMgatiyalalxve? 
\enum
\emng
\eentry

\bentry
\word{give-away}
\pron{givfaveV}
\gl{\nA}
\bmng
 (\AmA) 
\bnum
\num{1} dAna; koTuTxbiDuvudu. 
\num{2} dAna; pukakxTe koDuvudu. 
\num{3} agagx; sasAtx; kaDime bele; alapx bele. 
\num{4} beVhuSAriniMda -- bayalugoLisuvudu, bahiraMgagoLisuvudu, guTuTx biTuTxkoDuvudu. 
\enum
\emng
\eentry

\bentry
\word[given(1)]{given}
\pron{giva(vf)nf}
\gl{\gu}
\bmng
\bnum
\num{1} datatx; dAnavAgi koTaTx, niVDida. 
\num{2} (cALi \mo vakekx) vashavAgiruva; sikikxbididxruva: \eng{given to drinking} kuDitada cALige bididxruva. 
\num{3} (gaNita, vAda, \mo vugaLalilx) datatx; samamxta; opipxkoMDa; aMgiVkaqta. 
\num{4} nidiRSaTx; gotAtxda: \eng{at a given time} nidiRSaTx kAladalilx. 
\enum
\emng

\noindent
\gl{\nuga}
\bmng
\bnum
\num{1} \eng{given over} (duSaTxmAgaR \mo vakekx) bidadx; vashavAda. 
\num{2} \eng{given to} parxvaqtitxyiruva; olavuLaLx; havAyxsaviruva: \eng{given to reading} Oduva havAyxsaviruva. 
\enum
\emng
\eentry


\bentry
\word[given(2)]{given}
\pron{giva(vf)nf}
\gl{\nA}
\bmng
jAcnxta; tiLida -- viSaya yA parisithxti. 
\emng
\eentry

\bentry
\wordnospeech{given name}{given name}
\pron{?}
\gl{\nA}
\bmng
(\ame) (kerxYsatxralilx shishuvige nAmakaraNadalilx iTaTx) modala hesaru; nAmakaraNada hesaru. 
\emng
\eentry

\bentry
\word{giver}
\pron{givarf}
\gl{\nA}
\bmng
koDuga; dAta; dAni; koDuvava; niVDuvava. 
\emng

\noindent
\gl{\pagu}
\bmng
\bnum
\num{1} \eng{alms giver} BikeSxdAta; BikeSxniVDuvava. 
\num{2} \eng{law giver} shAsanadAra; kAnUnu mADuvavanu. 
\enum
\emng
\eentry

\bentry
\word{gizmo}
\pron{gisoZmxV}
\gl{\nA}
\bmng
 (\ame)  = \hyperlink{gismo}{gismo}. 
\emng
\eentry

\bentry
\word{gizzard}
\pron{gisaZDfR}
\gl{\nA}
\bmng
\bnum
\num{1} (modalaneya jaTharadalilx jiVNaRrasadoDane misharxvAda AhAravanunx areyalu iruva) hakikxya eraDaneya jaThara. 
\num{2} (kelavu bageya mInu, huLu, cipupxpArxNi, \mo vugaLa) sAnxyujaThara; hecucx balavuLaLx sAnxyugaLiMda kUDida jaThara. 
\enum
\emng

\noindent
\gl{\nuga}
\bmng
\bnum
\num{1} \eng{fret one's gizzard} ciMtepaDu; toMdarepaDu; kelxVshapaTuTxko; jaMjATapaDu; manasisxge hacicxko. 
\hyperdef{G}{gizzard nuga(2)}{} 
\num{2} \eng{stick in one's gizzard} rucisadiru; seVradiru; hitavenisadiru; sahayxvAgadiru: \eng{it sticks in my gizzard} adu nanage rucisadu. 
\enum
\emng
\eentry

\bentry
\word{glabella}
\pron{galxbela}
\gl{\nA}
\expl{(\bava\ \eng{glabellae}).}
\bmng
 (\aMrashA) kUcaR; hububxgaLa naDuvaNa nuNupAda ubibxda BAga. 
\emng
\eentry

\bentry
\word{glabrous}
\pron{gelxVbarxsf}
\gl{\gu}
\bmng
 (\aMrashA matutx \savi) 
\bnum
\num{1} kUdalilalxda; roVmarahita; tupupxLilalxda. 
\num{2} nuNupu camaRda. 
\enum
\emng
\eentry

\bentry
\word{glace}
\pron{gAlxYxseV}
\gl{\gu}
\bmng
\bnum
\num{1} (baTeTx, togalu, \mo vugaLa \vi) nuNupAda; nayavAda; meragu hAkida yA hoLeyuva meYyuLaLx. 
\num{2} (haNuNxgaLa \vi) sakakxre hAkida; sakakxre pAka baLida. 
\enum
\emng
\eentry

\bentry
\word{glacial}
\pron{gelxVsialf, gelxVSalf}
\gl{\gu}
\bmng
\bnum
\num{1} himagaDeDxya; niVgaRlilxna; himashileya. 
\num{2} himagaDeDxyaMtaha; himashileyaMtha; niVgaRlilxnaMtaha. 
\num{3} (himagaDeDxyaMte) koreyuva; vipariVta sheYtayxda; atishiVtalavAda. 
\num{4} (\ravi) himavatf; GaniVkarisidAga himagalilxnaMtiruva: \eng{glacial phosphoric acid} himavatApxsAphxrikAmalx. 
\numi{5} (\BUvi) 
\banum
\alnum{a} himaviruva; himashilA vishiSaTx; niVgaRlulxLaLx. 
\alnum{b} himashilAkaqta; himagalilxna pariNAmadiMdAda. 
\eanum
\numie
\num{6} (\rUpa) niBARvada; BAvarahita; BAvashUnayx: \eng{a glacial stare} BAvashUnayx birunoVTa. 
\enum
\emng
\eentry

\bentry
\wordnospeech{glacial epoch}{glacial epoch}
\pron{?}
\gl{\nA}
\bmng
 himashilAyuga; niVgaRlalxyuga; utatxragoVLAdhaRda bahuBAga himashileyiMda AvarisalapxTiTxdadx kAla. 
\emng
\eentry

\bentry
\wordnospeech{glacial erosion}{glacial erosion}
\pron{?}
\gl{\nA}
\bmng
 niVgaRlalx saveta; himashileyiMdAda saveta. 
\emng
\eentry

\bentry
\word{glacially}
\pron{gelxVsiali, gelxVSali}
\gl{\kirxvi}
\bmng
 himashileyaMte; niVgaRlilxnaMte; himashileya riVtiyalilx. 
\emng
\eentry

\bentry
\wordnospeech{glacial period}{glacial period}
\pron{?}
\gl{\nA}
\bmng
  = \hyperlink{glacial epoch}{glacial epoch}. 
\emng
\eentry

\bentry
\word{glaciated}
\pron{gAlxYx(gelxV) siETiDf}
\gl{\gu}
\bmng
\bnum
\num{1} himashilA pariNAmada lakaSxNavuLaLx; hima(shilA) kirxyeya cihenxgaLiruva, lakaSxNagaLiruva; niVgaRlalx pariNAmadiMdAda gurutugaLuLaLx. 
\num{2} hima (shilA) kirxyeyiMda -- nuNupAda, nuNaNxneya meVlemxYyuLaLx, AkAra tALida. 
\num{3} himAcACxdita; niVgaRlulx kavida; himagalulx hALegaLiMda mucicxda. 
\enum
\emng
\eentry

\bentry
\word{glaciation}
\pron{gelxVsiESanf}
\gl{\nA}
\bmng
\bnum
\numi{1} himIkaraNa: 
\banum
\alnum{a} himashilAracane; himagalulx hALegaLu rUpugoLuLxvudu. 
\alnum{b} himanadigaLiMda yA himagalulx hALegaLiMda mucicxda sithxti. 
\eanum
\numie
\num{2} himakirxye; himagalilxna kirxye yA adara pariNAma. 
\enum
\emng
\eentry

\bentry
\word{glacier}
\pron{gAlxYxsiarf}
\gl{\nA}
\bmng
 himanadi; unanxta parxdeVshadalilx himarAshi sheVKarisi alilxMda nidhAnavAgi jAri hariyuva niVgaRlalx nadi. 
\emng
\eentry

\bentry
\word{glaciered}
\pron{gAlxYxsiaDfR}
\gl{\gu}
\bmng
 himanadigaLiMda tuMbiruva. 
\emng
\eentry

\bentry
\word{glaciological}
\pron{gelxVsialAjikalf}
\gl{\gu}
\bmng
 himanadi vijAcnxnada yA himanadi vijAcnxnakekx saMbaMdhisida. 
\emng
\eentry

\bentry
\word{glaciologist}
\pron{gelxVsiAlajisfTx}
\gl{\nA}
\bmng
 himanadi vijAcnxni; himanadigaLa veYjAcnxnika adhayxyana naDesidava. 
\emng
\eentry

\bentry
\word{glaciology}
\pron{gelxVsiAlaji}
\gl{\nA}
\bmng
 himanadivijAcnxna; himanadigaLa savxrUpa, kirxyegaLanUnx matutx avugaLiMda BUmiya meVle Aguva pariNAmagaLanUnx adhayxyana mADuva BUvijAcnxnada shAKe. 
\emng
\eentry

\bentry
\word{glacis}
\pron{gAlxYxsisf, gAlxYxsiV}
\gl{\nA}
\expl{(\bava\ adeV; \ucAcx\ gAlxyXsisf yA gAlxyXsiVsfZ).}
\bmng
 (koVTeya) iLivAra; iLukalu; iLimeVDu; (mutitxge hAkiruva shaturxseVneyanunx guMDina surimaLege gurimADuva) koVTe goVDe yA iLijAru daMDe, aMcu. 
\emng
\eentry

\bentry
\word[glad(1)]{glad}
\pron{gAlxYxDf}
\gl{\gu}
\expl{(\tara\ \eng{gladder,} \tama\ \eng{gladdest}). }
\bmng
\bnum
\num{1} (\AKAyx) saMtoVSagoMDa; haSaRgoMDa; saMtuSaTx; haSiRta: \eng{I am glad of it} adariMda nAnu haSiRtanAde. 
\num{2} (muKaBAva, manoVBAva, \mo vugaLa \vi) geluvina; nalivina; saMtoVSasUcaka; saMtoVSada kaLeyuLaLx; geluvu tuMbida; saMtoVSaBarita; haSaR -- Barita, pUrita. 
\num{3} (sudidx, saMgati, \mo vugaLa \vi) geluvina; haSaRda; saMtoVSada; geluvuMTumADuva; haSaRdAyaka; saMtoVSakara. 
\numi{4} (parxkaqti \mo vugaLa \vi) 
\banum
\alnum{a} kAMtiyuta; parxkAshamAna. 
\alnum{b} celuvAda; suMdara; ramaNiVya; manoVhara. 
\eanum
\numie
\enum
\emng

\noindent
\gl{\nuga}
\bmng
\hyperdef{G}{glad(1) nuga}{} \eng{the glad eye} 
\banum
\alnum{a} (\ashi) beVTada noVTa; kAmuka daqSiTx. 
\alnum{b} ulAlxsada noVTa; haSaRdaqSiTx. 
\eanum
\emng
\eentry

\bentry
\word[glad(2)]{glad}
\pron{gAlxYxDf}
\gl{\sakirx}
\expl{(\BU\ matutx \BUkaq\ \eng{gladded,} \vakaq\ \eng{gladding}).}
\bmng
(\pArxparx) geluvuMTumADu; ulAlxsagoLisu; saMtoVSapaDisu; haSaRgoLisu. 
\emng
\eentry

\bentry
\word[glad(3)]{glad}
\pron{gAlxYxDf}
\gl{\nA}
\bmng
 (\AmA, \sA\ \bava dalilx) \eng{gladiolus.} 
\emng
\eentry

\bentry
\word{gladden}
\pron{gAlxYxDfnf}
\gl{\sakirx}
\bmng
 saMtoVSapaDisu; haSaRgoLisu; geluvuMTumADu; ulAlxsagoLisu. 
\emng
\eentry

\bentry
\word{glade}
\pron{gelxVDf}
\gl{\nA}
\bmng
 kADina maragaLa naDuve maragiDagaLilalxde savxcaCxvAgiruva parxdeVsha yA hAdi. 
\emng
\eentry

\bentry
\wordspecial{glad hand}{1}{1}{\hyperlink{glad-hand(1)}{\quad\textcolor{superscript}{$^2$}\eng{glad-hand}}}
\pron{?}
\gl{\nA}
\bmng
 sAvxgatahasatx; hasatxlAGava; keYkuluku. 
\emng
\eentry

\bentry
\wordspecial{glad-hand}{1}{2}{\hyperlink{glad hand(1)}{\quad\textcolor{superscript}{$^1$}\eng{glad hand}}}
\pron{gAlxYxDfhAyxMDf}
\gl{\sakirx}
\bmng
 AdarapUvaRkavAgi sAvxgatisu, aBinaMdisu. 
\emng
\eentry

\bentry
\word{gladiator}
\pron{gAlxYxDiETarf}
\gl{\nA}
\bmng
\bnum
\num{1} gAlxYxDiyeVTaru; katitxmalalx; pArxciVna roVmfna sAvaRjanika parxdashaRnagaLalilx katitxyiMda yA beVre AyudhadiMda hoVrADalu tarabeVtAdavanu. 
\num{2} muSiTxmalalx; kusitxmalalx. 
\num{3} (rAjakiVya \mo vugaLalilx) vAdacatura; vAdamalalx; vAdaviVra. 
\enum
\emng
\eentry

\bentry
\word{gladiatorial}
\pron{gAlxYxDiaToVrialf}
\gl{\gu}
\bmng
\bnum
\num{1} katitxmalalxna; (pArxciVna roVmfna sAvaRjanika parxdashaRnagaLalilx katitxyiMdaloV beVre AyudhadiMdaloV hoVrADalu tarabeVtAda) yoVdhana yA avanige saMbaMdhisida. 
\num{2} (\rUpa) (vAda yA caceRya \vi) keVvala -- kadana rUpada, jagaLagaMTitanada; vivAdAtamxka. 
\enum
\emng
\eentry

\bentry
\word{gladiolus}
\pron{gAlxYxDiOlasf}
\gl{\nA}
\expl{(\bava\ \eng{gladioli} \ucAcx\ gAlxYxDiOleY, yA}
\eng{gladioluses}).\bmng
katitxgiDa; katitxyAkArada elegaLanUnx ujavxla vaNaRda puSapxvaqMtagaLanUnx uLaLx, oMdu bageya sasayx. \imglink{gladiolusfigure}{\raisebox{-0.20cm}[0pt][0pt]{\pdfimage width 0.5cm height 0.6cm {G_Pictures/gladiolus.jpg}}} 
\emng
\eentry

\bentry
\word{gladly}
\pron{gAlxYxDfli}
\gl{\kirxvi}
\bmng
 saMtoVSadiMda; haSaRdiMda; geluviniMda; ulAlxsadiMda. 
\emng
\eentry

\bentry
\word{gladness}
\pron{gAlxYxDfnisf}
\gl{\nA}
\bmng
 saMtoVSa; haSaR; geluvu; nalivu; ulAlxsa. 
\emng
\eentry

\bentry
\word{gladsome}
\pron{gAlxYxDfsamf}
\gl{\gu}
\bmng
 (\kAparx) gelavuMTumADuva; ulAlxsakara; saMtoVSadAyaka; haSaRdAyaka. 
\emng
\eentry

\bentry
\word{gladsomely}
\pron{gAlxYxDfsamfli}
\gl{\kirxvi}
\bmng
 (\kAparx) geluvuMTumADuvaMte; ulAlxsakaravAgi; saMtoVSadAyakavAgi; haSaRdAyakavAgi. 
\emng
\eentry

\bentry
\wordRemoveSpace{Gladstone-bag}{Gladstone bag}
\pron{gAlxYxDfsaTxnf bAYxgf}
\gl{\nA}
\bmng
 keYpeTiTxge; eraDu samAna viBAgagaLuLaLx, oMdu bageya haguravAda camaRda peTiTxge.  \imglink{gladstone-bagfigure}{\raisebox{-0.15cm}[0pt][0pt]{\pdfimage width 0.7cm height 0.5cm {G_Pictures/gladstone-bag.jpg}}} 
\emng
\eentry

\bentry
\word{Glagolitic}
\pron{gAlxYxgaliTikf}
\gl{\gu}
\bmng
 gAlxYxgaliTikf; sumAru \kirxsha\ \eng{865}ralilx saMta sirilf eMbAta saqSiTxsida, kelavu sAlxvf BASegaLanunx bareyalu baLasuva lipiya. 
\emng
\eentry

\bentry
\word[glair(1)]{glair}
\pron{gelxVrf}
\gl{\nA}
\bmng
\bnum
\num{1} moTeTxya biLiloVLe. 
\numi{2} loVLe aMTu: 
\banum
\alnum{a} moTeTxya biLiloVLeyiMda mADida oMdu bageya aMTu. 
\alnum{b} aMtha yAvudeV bageya aMTu padAthaR. 
\eanum
\numie
\enum
\emng
\eentry

\bentry
\word[glair(2)]{glair}
\pron{gelxVrf}
\gl{\sakirx}
\bmng
 moTeTxya biLiloVLe -- savaru, baLi, leVpisu. 
\emng
\eentry

\bentry
\word[glaire(1)]{glaire}
\pron{gelxVrf}
\gl{\nA}
\bmng
  = \hyperlink{glair(1)}{$^1$glair}. 
\emng
\eentry

\bentry
\word[glaire(2)]{glaire}
\pron{gelxVrf}
\gl{\sakirx}
\bmng
  = \hyperlink{glair(2)}{$^2$glair}. 
\emng
\eentry

\bentry
\word{glaireous}
\pron{gelxVriasf}
\gl{\gu}
\bmng
 (moTeTxya) biLi loVLeyaMte kANuva yA iruva. 
\emng
\eentry

\bentry
\word{glairy}
\pron{gelxVri}
\gl{\gu}
\bmng
\bnum
\num{1} moTeTxya biLiloVLeyaMtaha yA A lakaSxNada. 
\num{2} aMTuva; aMTaMTAda. 
\enum
\emng
\eentry

\bentry
\word{glaive}
\pron{gelxVvf}
\gl{\nA}
\bmng
 (\pArxparx\ yA \kAparx) 
\bnum
\num{1} agalavAda alagina katitx. 
\num{2} katitx; KaDagx. 
\enum
\emng
\eentry

\bentry
\word[glam(1)]{glam}
\pron{gAlxYxmf}
\gl{\nA}
\bmng
 (\AmA)  = \hyperlink{glamour(1)}{$^1$glamour} (eMbudara \saMkiSx). 
\emng
\eentry

\bentry
\word[glam(2)]{glam}
\pron{gAlxYxmf}
\gl{\gu}
\bmng
 (\AmA)  = \hyperlink{glamorous}{glamorous} (eMbudara \saMkiSx). 
\emng
\eentry

\bentry
\word[glam(3)]{glam}
\pron{gAlxYxmf}
\gl{\sakirx}
\expl{(\BU\ matutx \BUkaq\ \eng{glammed,} \vakaq\ \eng{glamming}).}
\bmng
(\AmA)  = \hyperlink{glamorize}{glamorize} (eMbudara \saMkiSx). 
\emng
\eentry

\bentry
\word{Glam.}
\pron{gAlxYxmf}
\gl{\saMkiSx}
\bmng
 (birxTaninxna) \eng{Glamorgan(shire).} 
\emng
\eentry

\bentry
\word[glamor(1)]{glamor}
\pron{gAlxYxmarf}
\gl{\nA}
\bmng
(\ame)  = \hyperlink{glamour(1)}{$^1$glamour}. 
\emng
\eentry

\bentry
\word[glamor(2)]{glamor}
\pron{gAlxYxmarf}
\gl{\sakirx}
\bmng
 (\ame)  = \hyperlink{glamour(2)}{$^2$glamour}. 
\emng
\eentry

\bentry
\word{glamorisation}
\pron{gAlxYxmareYseZVSanf}
\gl{\nA}
\bmng
  = \hyperlink{glamorization}{glamorization}. 
\emng
\eentry

\bentry
\word{glamorise}
\pron{gAlxYxmareYsfZ}
\gl{\sakirx}
\bmng
 (\ame)  = \hyperlink{glamorize}{glamorize}. 
\emng
\eentry

\bentry
\word{glamorization}
\pron{gAlxYxmareYseZVSanf}
\gl{\nA}
\bmng
\bnum
\num{1} manamoVhakagoLisuvudu; citAtxkaSaRkavAgi mADuvudu. 
\num{2} maruLugoLisuvudu; maMtarxmugadhxvAgi mADuvike. 
\enum
\emng
\eentry

\bentry
\word{glamorize}
\pron{gAlxYxmareYsfZ}
\gl{\sakirx}
\bmng
\bnum
\num{1} manamoVhakagoLisu; citAtxkaSaRkavAgi mADu. 
\num{2} maruLugoLisu; maMtarxmugadhxgoLisu. 
\enum
\emng
\eentry

\bentry
\word{glamorous}
\pron{gAlxYxmarasf}
\gl{\gu}
\bmng
\bnum
\num{1} manamoVhaka; citAtxkaSaRka; moVhakavAda; celuvuLaLx. 
\num{2} maruLugoLisuva; maMtarxmugadhxvAgisuva. 
\enum
\emng
\eentry

\bentry
\word[glamour(1)]{glamour}
\pron{gAlxYxmarf}
\gl{\nA}
\bmng
\bnum
\num{1} maMtarxshakitx; mAyAjAla; kaNakxTuTx; iMdarxjAla; mATa; moVhana videyx yA shakitx: \eng{this species of witchcraft is well known in Scotland as the glamour} I bageya mATagArikege sAkxTalxMDinalilx mAyAjAlaveMdu hesaru. 
\num{2} moVhaka lAvaNayx; BarxmegoLisuva yA manamoVhakavAda swMdayaR matutx AkaSaRNe. 
\num{3} (\kanmu\ heMgasina) deYhika swMdayaR; sobagina meYmATa; shariVra lAvaNayx. 
\enum
\emng

\noindent
\gl{\pagu}
\bmng
 \eng{cast a glamour over} saMmoVhanagoLisu; mAyAjAla biVsu; maruLumADu; maMtarxshakitx biVru; maMtarxmugadhxgoLisu; moVhaka lAvaNayxdiMda manasusx seLe. 
\emng
\eentry

\bentry
\word[glamour(2)]{glamour}
\pron{gAlxYxmarf}
\gl{\sakirx}
\bmng
\bnum
\num{1} mAyAjAla biVsu; iMdarxjAlakokxLapaDisu; maMtarxmugadhxnanAnxgisu; mATakekx vashamADiko; moVDi hAku. 
\num{2} moVhagoLisu; moVhakekx -- sikikxsu, biVLisu, vashapaDisu. 
\num{3} (\AmA) manamoVhakagoLisu. 
\enum
\emng
\eentry

\bentry
\wordnospeech{glamour boy}{glamour boy}
\pron{?}
\gl{\nA}
\bmng
 beDagina huDuga; moVhaka bAlaka; manaseLeyuva deVhaswMdayaRvuLaLx huDuga, yuvaka. 
\emng
\eentry

\bentry
\wordnospeech{glamour girl}{glamour girl}
\pron{?}
\gl{\nA}
\bmng
 beDagina -- huDugi, bAle; moVhini; manaseLeyuva deVhaswMdayaRvuLaLx huDugi, yuvati. 
\emng
\eentry

\bentry
\word[glance(1)]{glance}
\pron{gAlxnfsx}
\gl{\sakirx}
\bmng
\bnum
\num{1} (vasutx \mo vugaLa kaDege yA meVle) kaNaNxnunx hAyisu; daqSiTxyanunx biVru. 
\num{2} (kirxkeTf) (ceMDanunx) OrehoDetadiMda hoDe, tirugisu. 
\enum
\emng

\noindent
\gl{\akirx}
\bmng
\bnum
\num{1} (Ayudhada \vi) (guriyanunx neVravAgi baDiyade adara meVliMda) jAribiDu; savarikoMDu hoVgu. 
\num{2} (hoLeyuva vasutxvina yA beLakina \vi) miMcu; thaTaTxne hoLeyu; minugu; suPxrisu; parxBebiVru. 
\num{3} (kaNiNxna \vi) kaNoNxVDisu; daqSiTxhAyisu; kaSxNakAla noVTa hAyisu; minugunoVTa biVru. 
\num{4} (mAtina, mAtADuvavana \vi) jArisu; teVlisu; viSaya biTuTx yA viSayadiMda viSayakekx beVga -- sari, hoVgu, badalAyisu. 
\num{5} (kirxkeTf) OrehoDeta hoDe; ceMDanunx OrehoDetadiMda tirugisu. 
\enum
\emng

\noindent
\gl{\pagu}
\bmng
\bnum
\numi{1} \eng{glance at} 
\banum
\alnum{a} pArxsaMgikavAgi (matutx \sA\ vayxMgayxvAgi) -- parxsAtxpisu, sUcisu. 
\alnum{b} kaSxNanoVTa biVru; kaSxNamAtarx noVDu. 
\eanum
\numie
\num{2} \eng{glance down, up, etc.} (meVliMda keLakekx, keLagiMda meVlakekx, itAyxdi) daqSiTx hAyisu; kaNANxDisu. 
\num{3} \eng{glance one's eye} (oMdara meVle) kaNuNxhAyisu; daqSiTx biVru. 
\num{4} \eng{glance over} sUthxlavAgi Odu; meVle meVle Odu; kaNuNx hAyisutAtx Odu. 
\enum
\emng
\eentry

\bentry
\word[glance(2)]{glance}
\pron{gAlxnfsx}
\gl{\nA}
\bmng
\bnum
\num{1} veVgavAda matutx OreyAda calana yA peTuTx. 
\hypertarget{glance(2)2}{} 
\num{2} (kirxkeTf) jAru hoDeta; OrehoDeta; bAYxTu ceMDige OreyAgi tagaluvaMte adanunx Olisi ceMDige hoDeda hoDeta. 
\num{3} (thaTaTxneya calanadiMdAguva) miMcu; minugu; hoLapu; suPxraNa. 
\num{4} kaSxNanoVTa; kaSxNadaqSiTx; minugunoVTa; nasunoVTa; kuDinoVTa. 
\enum
\emng

\noindent
\gl{\nuga}
\bmng
 \eng{at a glance} noVDidoDane; noVDida kUDaleV. 
\emng
\eentry

\bentry
\word[glance(3)]{glance}
\pron{gAlxnfsx}
\gl{\nA}
\bmng
 (\ravi) gAlxnusx; kaMdu baNaNxda matutx loVhada hoLapuLaLx halavAru Kanija salePxYDugaLalolxMdu: \eng{copper glance} tAmarxda gAlxnusx. 
\emng
\eentry

\bentry
\word[gland(1)]{gland}
\pron{gAlxYxMDf}
\gl{\nA}
\bmng
 (\shavi) garxMthi: 
\banum
\alnum{a} rakatxdiMda nidiRSaTx padAthaRgaLanunx Ayudx, deVhada upayoVgakAkxgi yA deVhadiMda horahAkuvudakAkxgi adanunx sUkatx riVti mApaRDisuva vishiSaTx jiVvakoVshagaLiMdAda aMga. 
\alnum{b} (\savi) darxvavisheVSagaLanunx sarxvisuva, sasayxda meYmeVlina koVsha yA koVshasamUha. 
\eanum
\emng
\eentry

\bentry
\word[gland(2)]{gland}
\pron{gAlxYxMDf}
\gl{\nA}
\bmng
 (\yaMshA) gAlxYxMDu; pisaTxnf baLe; pisaTxnanxnunx siliMDarinoLakekx parxveVsha mADuva sathxLadalilx anila yA ugi soVrihoVgadaMte pisaTxninxge toDisiruva baLe. 
\emng
\eentry

\bentry
\word{glandered}
\pron{gAlxYxMDarfDx}
\gl{\gu}
\bmng
 gAlxYxMDasfR roVga taguliruva. 
\emng
\eentry

\bentry
\word{glanderous}
\pron{gAlxYxMDarasf}
\gl{\gu}
\bmng
\bnum
\num{1}  = \hyperlink{glandered}{glandered}. 
\num{2} gAlxYxMDasfR roVgadiMdAda. 
\num{3} gAlxYxMDasfR lakaSxNagaLuLaLx. 
\enum
\emng
\eentry

\bentry
\word{glanders}
\pron{gAlxYxMDasfR}
\gl{\nA}
\bmng
 (kelavu veVLe \Eva vAgiyU \parx) gAlxYxMDasfR: 
\banum
\alnum{a} kudure siMbaLa roVga; davaDeya keLaBAga UdikoMDu, mUginiMda siMbaLa suriyuva, kudureya oMdu aMTu roVga. 
\alnum{b} manuSayxnige yA itara pArxNigaLige aMTikoLuLxva I roVga. 
\eanum
\emng
\eentry

\bentry
\word{glandiferous}
\pron{gAlxYxMDipherasf}
\gl{\gu}
\bmng
 akAnfR yA aMthadeV haNuNxgaLanunx biDuva. 
\emng
\eentry

\bentry
\word{glandiform}
\pron{gAlxYxMDiphAmfR}
\gl{\gu}
\bmng
\bnum
\num{1} akAnfR AkArada. 
\num{2} (\aMrashA) garxMthiyaMtha. 
\enum
\emng
\eentry

\bentry
\word{glandless}
\pron{gAlxYxMDflisf}
\gl{\gu}
\bmng
 nigarxRMthika; garxMthigaLilalxda; garxMthirahita. 
\emng
\eentry

\bentry
\word{glandular}
\pron{gAlxYxMDuyxlarf}
\gl{\gu}
\bmng
 garxMthiya; garxMthige saMbaMdhisida. 
\bnum
\num{2} garxMthigaLiruva; sagarxMthika; garxMthivishiSaTx. 
\enum
\emng
\eentry

\bentry
\wordnospeech{glandular fever}{glandular fever}
\pron{?}
\gl{\nA}
\bmng
 gaLalejavxra; garxMthijavxra; dugadhxgarxMthigaLu UdikoLuLxva oMdu bageya soVMkuroVga. 
\emng
\eentry

\bentry
\word{glandule}
\pron{gAlxYxMDUyxlf}
\gl{\nA}
\bmng
 (\aMrashA) kirugarxMthi. 
\emng
\eentry

\bentry
\word{glanduliferous}
\pron{gAlxYxMDUyxlipherasf}
\gl{\gu}
\bmng
 kirugarxMthiBarita; cikakx garxMthigaLiMda tuMbida. 
\emng
\eentry

\bentry
\word{glandulose}
\pron{gAlxYxMDuyxloVsf}
\gl{\gu}
\bmng
  = \hyperlink{glanduliferous}{glanduliferous}. 
\emng
\eentry

\bentry
\word{glandulous}
\pron{gAlxYxMDuyxlasf}
\gl{\gu}
\bmng
  = \hyperlink{glandular}{glandular}. 
\emng
\eentry

\bentry
\word{glans}
\pron{gAlxYxnfs'}
\gl{\nA}
\expl{(\bava\ \eng{glandes} \ucAcx\ gAlxYxnfDiVsfZ).}
\bmng
(\aMrashA) shishAnxgarx yA BagAMkurAgarx; shishanxda yA BagAMkurada guMDAda tudi. 
\emng
\eentry

\bentry
\word[glare(1)]{glare}
\pron{gelxVrf}
\gl{\sakirx}
\bmng
 (devxVSavanunx, parxtiBaTaneyanunx) kaNaNxlilx sUsu; daqSiTxyiMda toVrisu. 
\emng

\noindent
\gl{\akirx}
\bmng
\bnum
\num{1} kaNuNx koVreYsu; JaLa hoDe; kaNuNx koVreYsuvaMte yA kaNiNxge ahitavAguvaMte -- hoLe, javxlisu, parxkAshisu. 
\num{2} atiyAgi gamana seLeyuvaMtiru; kaNiNxge -- rAcuvaMtiru, hoDeyuvaMtiru; kaNuNx kukukxvaMtiru. 
\num{3} duradurane yA ugarxvAgi noVDu. 
\enum
\emng
\eentry

\bentry
\word[glare(2)]{glare}
\pron{gelxVrf}
\gl{\nA}
\bmng
\bnum
\num{1} JaLapu; cucucxbeLaku; tiVkaSxNXparxkAsha; parxKaraparxBe; parxbalavAda, parxjavxlisuva beLaku. 
\num{2} JaLa; uri bisilu; suDubisilu; uriyuva, suDuva -- bisilu; sahisalAgada, eDebiDada -- bisilu; dhagadhagisuva sUyaRtApa. 
\num{3} thaLaku hoLapu; kapaTakAMti; horabeDagu. 
\num{4} neTaTxnoVTa; diTiTxsinoVDuva noVTa. 
\num{5} duruduru noVTa; ugarxnoVTa; urinoVTa; duruguTiTx yA ugarxvAgi noVDuva noVTa. 
\enum
\emng
\eentry

\bentry
\word[glare(3)]{glare}
\pron{gelxVrf}
\gl{\gu}
\bmng
 (\ame) (\kanmu\ maMjugaDeDxya \vi) nayavAda matutx hoLeyuva; nuNupAda hoLapumeYyina. 
\emng
\eentry

\bentry
\word{glaring}
\pron{gelxVriMgf}
\gl{\gu}
\bmng
\bnum
\num{1} thaLathaLa hoLeyuva; kaNuNx koVreYsuva. 
\num{2} atAyxDaMbarada; thaLakupaLakina. 
\num{3} kaNuNxkukukxva; edudx kANuva: \eng{several glaring defects} kelavu edudx kANuva doVSagaLu. 
\num{4} ugarxdaqSiTxya; urigaNiNxna. 
\num{5} duruduru noVTada. 
\enum
\emng
\eentry

\bentry
\word{glaringly}
\pron{gelxVriMgfli}
\gl{\kirxvi}
\bmng
\bnum
\num{1} kaNuNx koVreYsuvaMte. 
\num{2} kaNuNx kukukxvaMte; kaNiNxge baDiyuvaMte; edudx kANuvaMte: \eng{glaringly absurd} kaNuNx kukukxvaSuTx athaRshUnayx. 
\enum
\emng
\eentry

\bentry
\word{glaringness}
\pron{gelxVriMgfnisf}
\gl{\nA}
\bmng
\bnum
\num{1} kaNuNx koVreYsuvaMtiruvike. 
\num{2} kaNiNxge baDiyuvaMtiruvike; edudx kANuvaMtiruvike. 
\enum
\emng
\eentry

\bentry
\word{glary}
\pron{gelxVri}
\gl{\gu}
\bmng
\bnum
\num{1} (beLakina \vi) kaNuNx koVreYsuva. 
\num{2} kaNuNx kukukxva; edudxkANuva. 
\num{3} dhage yA JaLa biVruva. 
\enum
\emng
\eentry

\bentry
\word[glass(1)]{glass}
\pron{gAlxsf}
\gl{\nA}
\bmng
\bnum
\num{1} gAju; kAca; maraLanunx soVDA yA poTAYxSf yA averaDaroDane matutx itara kelavu padAthaRgaLoDane beresi, shAKadiMda karagisi tayArisida, \sA, pAradashaRkavAgi, hoLapuLaLxdAdxgi, kaThinavAgi matutx BiduravAgi iruva padAthaR. 
\num{2} gAju; saMyoVjanadalilx yA guNagaLalilx gAjinaMtaha vasutx: \eng{glass of antimony} AyxMTimani gAju. 
\num{3} gAjina -- pAterxgaLu, oDavegaLu, kiTakigaLu, sasayxrakaSxka gaqha(gaLu). 
\num{4} gAjina loVTa yA baTaTxlu. 
\hypertarget{glass(1)5}{} 
\num{5} gAjina loVTada yA baTaTxlina tuMba iruva pAniVyada (\sA\ madayxda) parxmANa: \eng{a friendly glass} senxVhaloVTa; senxVhada kuDita. \eng{fond of his glass} madayxpirxyanAgidAdxne. 
\num{6} = \hyperref{kandict_s.pdf}{S}{sand-glass}{sand-glass}. 
\num{7} = \hyperref{kandict_h.pdf}{H}{hourglass}{hour-glass}. 
\num{8} (gADi \mo vugaLa) kiTaki. 
\num{9} gAju; citarxpaTada meVlina gAjuhalage, kananxDi. 
\num{10} (sasigaLige kavisuva) gAju peTiTxge. 
\num{11} (gAjina) kananxDi; dapaRNa. 
\num{12} (\kanmu\ \bava\ dalilx) kananxDaka; suloVcana. 
\num{13} lenfsx; yava; masUra. 
\num{14} (gaDiyArada muKakekx hAkiruva) gAju bilelx. 
\num{15} (\kanmu\ \bava dalilx) = \hyperref{kandict_f.pdf}{F}{field-glasses}{field-glasses}. 
\num{16} dUradashaRka. 
\num{17} = \hyperref{kandict_o.pdf}{O}{opera-glasses}{opera-glass}. 
\num{18} sUkaSxmXdashaRka. 
\num{19} vAyuBAramApaka. 
\enum
\emng

\noindent
\gl{\nuga}
\bmng
 \eng{has had a glass too much} savxlapx hecAcxgi (madayx) kuDididAdxne; (madayxda) oMdu loVTa hecucx hAkidAdxne, ErisidAdxne. 
\emng
\eentry

\bentry
\word[glass(2)]{glass}
\pron{gAlxsf}
\gl{\sakirx}
\bmng
\bnum
\num{1} (\kanmu\ \BUkaq dalilx) gAjuhAku; gAju joVDisu. 
\num{2} gAjinaMte -- nuNupumADu, nayamADu. 
\num{3} (kaNaNxnunx) kAMtigeDisu; nisetxVjagoLisu; gAjinaMte maMdagoLisu. 
\num{4} kananxDiyaMte parxtiPalisu, parxtibiMbisu: \eng{trees glass themselves in the lake} saroVvaradalilx maragaLu parxtibiMbisutatxve. 
\num{5} dubiRVnina mUlaka noVDu yA huDuku. 
\enum
\emng
\eentry

\bentry
\word{glass-blower}
\pron{gAlxsfbolxVarf}
\gl{\nA}
\bmng
 gAjUduga; gAjUduvava; karagida gAjanunx (koLaviya koneyalalxdidxkoMDu) Udi AkAra koDuvavanu. 
\emng
\eentry

\bentry
\wordnospeech{glass case}{glass case}
\pron{?}
\gl{\nA}
\bmng
 (vasutxgaLanunx parxdashiRsalu yA joVpAnavAgiDalu mADida) gAjina -- biVru, kapATu. 
\emng
\eentry

\bentry
\wordspecial{glass-cloth}{1}{1}{\hyperlink{glass cloth(1)}{\quad\textcolor{superscript}{$^2$}\eng{glass cloth}}}
\pron{gAlxsfkAlxtf}
\gl{\nA}
\bmng
 gAjubaTeTx: 
\banum
\alnum{a} gAjorasuva baTeTx. 
\alnum{b} gAjupuDi baTeTx; gAjukAgadadaMte gAjina puDi aMTisida baTeTx. 
\eanum
\emng
\eentry

\bentry
\wordspecial{glass cloth}{1}{2}{\hyperlink{glass-cloth(1)}{\quad\textcolor{superscript}{$^1$}\eng{glass-cloth}}}
\pron{?}
\gl{\nA}
\bmng
 gAjubaTeTx; gAjuvasatxrX; sUkaSxmX neyegxya gAjina dArada baTeTx. 
\emng
\eentry

\bentry
\word{glass-culture}
\pron{gAlxsfkalacxrf}
\gl{\nA}
\bmng
 gAjukaqSi; gAjina AvaraNadalilx, rakaSxNeyalilx sasayxgaLanunx kaqSimADuvudu, beLesuvudu. 
\emng
\eentry

\bentry
\word{glass-cutter}
\pron{gAlxsfkaTarf}
\gl{\nA}
\bmng
\bnum
\num{1} gAju koyuyxvavanu, katatxrisuvavanu. 
\num{2} gAju katatxri; gAju koyuyxva upakaraNa, sAdhana. 
\enum
\emng
\eentry

\bentry
\word{glass-dust}
\pron{gAlxsfDasfTx}
\gl{\nA}
\bmng
 (meragu koDalu baLasuva) gAjupuDi. 
\emng
\eentry

\bentry
\wordnospeech{glass eye}{glass eye}
\pron{?}
\gl{\nA}
\bmng
 gAjukaNuNx: 
\banum
\alnum{a} gAjinalilx mADida kaqtaka kaNuNx. 
\alnum{b} kuduregaLa oMdu bageya kuruDu. 
\eanum
\emng
\eentry

\bentry
\wordnospeech{glass fibre}{glass fibre}
\pron{?}
\gl{\nA}
\bmng
gAjunAru: 
\banum
\alnum{a} gAjina taMtu, eLe. 
\alnum{b} gAjina taMtuviniMda tayArisida baTeTx. 
\alnum{c} gAjina taMtuvanunx oLaseVrisida pAlxsiTxkukx. 
\eanum
\emng
\eentry

\bentry
\word{glassful}
\pron{gAlxsfphulf}
\gl{\nA}
\bmng
  = \hyperlink{glass(1)5}{$^1$glass \((5)\)}. 
\emng
\eentry

\bentry
\word{glass-gall}
\pron{gAlxsfgAlf}
\gl{\nA}
\bmng
 = \hyperref{kandict_s.pdf}{S}{sandiver}{sandiver}. 
\emng
\eentry

\bentry
\word{glasshouse}
\pron{gAlxsfhwsf}
\gl{\nA}
\bmng
\bnum
\num{1} gAju kAKARne; gAjanunx tayArisuva mane. 
\num{2} gAjumane; gAjugaqha; sasayxgaLanunx beLesuva, gAjina CAvaNi hAkiruva AvaraNa, kaTaTxDa. 
\num{3} (CAyAcitarxkAkxgi kaTiTxruva) gAju cAvaNiya mane, koThaDi. 
\num{4} (\ashi) seYnika baMdiVKAne. 
\enum
\emng
\eentry

\bentry
\word{glassily}
\pron{gAlxsili}
\gl{\kirxvi}
\bmng
\bnum
\num{1} gAjinaMte. 
\num{2} (meVlemxY \vi) kaThinavAgi matutx hoLeyutatx. 
\numi{3} (kaNiNxna \vi) 
\banum
\alnum{a} acala daqSiTxyiMda; niMta noVTadiMda. 
\alnum{b} nisetxVjavAgi; kAMtirahitavAgi; shUnayxdaqSiTxyiMda: \eng{he stared glassily about him} Ata sutatxmutatxlu shUnayxdaqSiTxyiMda diTiTxsutitxdadx. 
\eanum
\numie
\numi{4} (niVrina \vi) gAjinaMte 
\banum
\alnum{a} hoLeyutatx matutx pAradashaRkavAgi. 
\alnum{b} nuNupAgi, ErupeVrilalxde, samatalavAgi. 
\eanum
\numie
\enum
\emng
\eentry

\bentry
\word{glassine}
\pron{gAlxsiVnf}
\gl{\nA}
\bmng
 gAjukAgada; hoLeyuva, nuNupAda pAradashaRka kAgada. 
\emng
\eentry

\bentry
\word{glassiness}
\pron{gAlxsinisf}
\gl{\nA}
\bmng
\bnum
\num{1} gAjinaMtiruvike; gAjina lakaSxNaviruvike. 
\numi{2} (kaNuNx \mo vugaLa \vi) 
\banum
\alnum{a} kAMtirahitate; nisetxVjate; maMkAgiruvike; hoLapilalxdiruvike. 
\alnum{b} calisadiruvike; noVTavilalxdiruvike. 
\eanum
\numie
\num{3} (meVlemxY \vi) kaThinavAgi matutx hoLeyutatx iruvike. 
\numi{4} (niVrina \vi) 
\banum
\alnum{a} hoLeyutatx matutx pAradashaRkavAgiruvike. 
\alnum{b} nuNupAgi samatalavAgiruvike. 
\eanum
\numie
\enum
\emng
\eentry

\bentry
\word{glassing-jack}
\pron{gAlxsiMgfjAYxkf}
\gl{\nA}
\bmng
 (\birx) camaRvanunx nayagoLisi, hoLeyuvaMte hada mADuvudakekx baLasuva yaMtarx. 
\emng
\eentry

\bentry
\word{glass-making}
\pron{gAlxsfmeVkiMgf}
\gl{\nA}
\bmng
 gAju tayArike; gAjanunx yA gAjina padAthaRgaLanunx tayArisuva kale. 
\emng
\eentry

\bentry
\word{glass-paper}
\pron{gAlxsfpeVparf}
\gl{\nA}
\bmng
 gAjukAgada; haMsapaTiTx; merugu koDalu baLasuva, gAjupuDi aMTisida kAgada. 
\emng
\eentry

\bentry
\wordnospeech{glass snake}{glass snake}
\pron{?}
\gl{\nA}
\bmng
 hAvuhalilx; \da amerikadalilx vAsisuva, peDasu bAlada, hAvanunx hoVluva halilx. 
\emng
\eentry

\bentry
\word{glassware}
\pron{gAlxsfveVrf}
\gl{\nA}
\bmng
 gAjina padAthaRgaLu, sAmAnugaLu. 
\emng
\eentry

\bentry
\wordnospeech{glass wool}{glass wool}
\pron{?}
\gl{\nA}
\bmng
 (poTaTxNagaLalilx BatiRgAgiyU viduyxdapArakavAgiyU baLasuva, navirAda) gAjineLegaLu; gAjutaMtu. 
\emng
\eentry

\bentry
\word{glasswort}
\pron{gAlxsfvaTfR}
\gl{\nA}
\bmng
 (gAju tayArikeyalilx hiMde baLasutitxdadx) sAlikoVniRya yA salosxVla kulada sasayx. 
\emng
\eentry

\bentry
\word{glassy}
\pron{gAlxsi}
\gl{\gu}
\bmng
\bnum
\num{1} gAjina lakaSxNagaLuLaLx. 
\num{2} gAjinaMtha; gAjanunx hoVluva. 
\numi{3} (kaNuNx \mo vugaLa \vi) 
\banum
\alnum{a} hoLapilalxda; kAMti rahita; nisetxVja; maMkAda. 
\alnum{b} calisada; nishacxla; daqSiTx niMtuhoVgiruva. 
\eanum
\numie
\num{4} (niVrina \vi) (gAjinaMte) hoLapuLaLxdUdx pAradashaRkavU samatala shariVravuLaLxdUdx Ada; nishacxlavU parxshAMtavU parxkAshamayavU Ada: \eng{glassy surface} gAjina meVlemxY; gAjinaMte samatalavU, nayavAdudU parxkAshamayavU Ada horameY. 
\enum
\emng
\eentry

\bentry
\word[Glaswegian(1)]{Glaswegian}
\pron{gAlxYx(gAlx)sfviVjianf, gAlxYx(gAlx)sfviVjanf}
\gl{\gu}
\bmng
 (sAkxTelxMDina) gAlxsogxV nagarada. 
\emng
\eentry

\bentry
\word[Glaswegian(2)]{Glaswegian}
\pron{gAlxYx(gAlx)sfviVjianf, gAlxYx(gAlx)sfviVjanf}
\gl{\nA}
\bmng
 gAlxsogxV nagaradalilx huTiTxruva yA vAsisuvava. 
\emng
\eentry

\bentry
\wordRemoveSpace{Glauber's-salts}{Glauber's salts}
\pron{gAlx(gwlx)basfR sAlfTxsX}
\gl{\nA}
\bmng
 (\ravi) gwlxbarf lavaNagaLu; vireVcakavAgi baLasuva, jaliVkaqta soVDiyaM salephxVTu. 
\emng
\eentry

\bentry
\word{glaucoma}
\pron{gAlx(gwlx)koVma}
\gl{\nA}
\bmng
 gAlxkoVma; kaNuNxguDeDxyalilx otatxDa hecAcxgi, karxmeVNa daqSiTx iMgihoVguva oMdu vAyxdhi. 
\emng
\eentry

\bentry
\word{glaucomatous}
\pron{gAlx(gwlx)koVmaTasf}
\gl{\gu}
\bmng
 gAlxkoVmIya; gAlxkoVmada yA adakekx saMbaMdhisida yA A roVga taguliruva. 
\emng
\eentry

\bentry
\word{glaucous}
\pron{gAlxkasf}
\gl{\gu}
\bmng
\bnum
\num{1} (muKayxvAgi \savi) mAsalu bUdu hasurina yA niVliya. 
\num{2} (dArxkiSxya haNiNxna meVlinaMte) puDi savaridaMtiruva. 
\enum
\emng
\eentry

\bentry
\word[glaze(1)]{glaze}
\pron{gelxVsfZ}
\gl{\sakirx}
\bmng
\bnum
\num{1} (kiTaki, citarxpaTa, \mo vugaLige) gAju -- hAku, joVDisu. 
\num{2} (kaTaTxDakekx) gAjina kiTakigaLanunx -- hAku, havaNisu. 
\num{3} (maDake, kuDike, \mo vugaLige gAjAguva padAthaR baLidu karagisi) gAjumeY koDu; gAjina horameY uMTumADu; gAjuleVpa koDu. 
\num{4} (maDake, kuDike, \mo vugaLige I vidhAnadiMda) baNaNx hacucx. 
\num{5} (baTeTx, kAgada, togalu, hiTiTxnalilx mADida BakaSxyXgaLu, \mo vugaLige) gAjumeY koDu; gAju meragu koDu; nayavAda hoLapumeY koDu. 
\num{6} (kaNaNxnunx) (kaNiNxVru \mo vugaLa) pore mucucx. 
\num{7} gAju baLi; gAjuleVpa baLi; (baNaNx baLida meVlemxYyanunx) adara baNaNxda hoLapanunx badalAyisuvaMte beVroMdu pAradashaRka baNaNxvanunx teLuvAgi baLi, savaru. 
\num{8} (ujujxvudu \mo vugaLiMda) hoLapu koDu; meragu koDu; gAjinaMte hoLeyuva meVlemxY koDu. 
\enum
\emng

\noindent
\gl{\akirx}
\bmng
 (\kanmu\ kaNuNxgaLa \vi) gAjinaMtAgu; nisetxVjavAgu; kAMtirahitavAgu. 
\emng

\noindent
\gl{\pagu}
\bmng
 \eng{glaze in} (sutatxlU) gAju hAku; gAjina AvaraNa koDu. 
\emng
\eentry

\bentry
\word[glaze(2)]{glaze}
\pron{gelxVsfZ}
\gl{\nA}
\bmng
\bnum
\num{1} gelxVsuZ; karagisidAga gAjAguva padAthaR. 
\num{2} gAjumeY; iMtaha padAthaRvanunx baLiyuvudariMdAda nuNupu meY. 
\numi{3} (baTeTx, togalu, hiTiTxnalilx mADida BakaSxyXgaLu, \mo vugaLige koTaTx) gAju meragu: 
\banum
\alnum{a} nuNupAda hoLapu leVpa. 
\alnum{b} hiVgAda hoLapu meY. 
\eanum
\numie
\num{4} (kaNiNxVru \mo) pore mucicxda noVTa. 
\num{5} gAju leVpa; (baNaNx baLida meVlemxYya baNaNxda hoLapanunx badalAyisalu) beVroMdu pAradashaRka baNaNxvanunx teLuvAgi baLida leVpa. 
\num{6} merugu; hoLapu; merugu yA hoLapu koTaTxdadxriMda kANuva parxkAsha. 
\num{7} (\ame) (tiMDi tinisugaLige baLiyuva) sakakxreya yA hiTiTxna leVpa. 
\enum
\emng
\eentry

\bentry
\word{glazed}
\pron{gelxVsfZDx}
\gl{\gu}
\bmng
 hoLapu koTiTxruva; hoLapu meYyuLaLx; gAjumeragu koTiTxruva. 
\emng
\eentry

\bentry
\wordnospeech{glazed frost}{glazed frost}
\pron{?}
\gl{\nA}
\bmng
= \hyperref{kandict_s.pdf}{S}{silver thaw}{silver thaw}. 
\emng
\eentry

\bentry
\word{glazer}
\pron{gelxVsaZrf}
\gl{\nA}
\bmng
\bnum
\num{1} hoLapugAra; hoLapu koDuvavanu; meragu mADuvavanu. 
\num{2} hoLapu koDuva, merugu koDuva salakaraNe. 
\enum
\emng
\eentry

\bentry
\word{glazier}
\pron{gelxVsiZarf, gelxVSaZrf}
\gl{\nA}
\bmng
 gAjugAra; kiTaki \mo vugaLige gAju, kananxDi, \mo vanunx hAkuva kasabinavanu. 
\emng
\eentry

\bentry
\word{glaziery}
\pron{gelxVsiZari, gelxVSaZri}
\gl{\nA}
\bmng
 gAjugArike; gAjugArana kelasa. 
\emng
\eentry

\bentry
\word{glazing}
\pron{gelxVsiZMgf}
\gl{\nA}
\bmng
 (kiTaki, citarxpaTa, \mo vugaLige) gAju -- hAkuvudu, joVDisuvudu. 
\bnum
\num{2} (kaTaTxDakekx) gAjina kiTakigaLanunx iDuvudu, havaNisuvudu. 
\num{3} (maDake, kuDike, \mo vugaLige) gAjubaNaNx kacicxsuvudu. 
\num{4} (baTeTx, kAgada, togalu, hiTiTxnalilx mADida BakaSxyXgaLu, \mo vugaLige) gAju meragu koDuvudu; nayavAda hoLapumeY koDuvudu. 
\num{5} (kaNaNxnunx kaNiNxVru \mo vugaLa) pore mucucxvudu. 
\num{6} gAju (baNaNxda) leVpana; baNaNx baLida meVlemxYge adara baNaNxda hoLapu badalAyisuvaMte beVroMdu pAradashaRka baNaNxvanunx teLuvAgi baLiyuvudu. 
\num{7} (ujujxvudu \mo vugaLiMda gAjinaMte) hoLapumeY koDuvudu. 
\num{8} (\kanmu\ kaNuNxgaLa \vi) gAjinaMtAguvudu; gAjinaMte maMda daqSiTxyAguvudu. 
\num{9} kiTaki gAjugaLu. 
\num{10} merugu; merugu yA hoLapu koDalu baLasuva sAmagirx. 
\enum
\emng
\eentry

\bentry
\word{glazy}
\pron{gelxVsiZ}
\gl{\gu}
\bmng
\bnum
\num{1} gAjinaMtiruva; gAjinaMtha. 
\num{2} gAjuhoLapina; gAjumeragina; gAju meYyaMte hoLeyuva. 
\enum
\emng
\eentry

\bentry
\wordnospeech{GLC}{GLC}
\pron{?}
\gl{\saMkiSx}
\bmng
 (\birx) \eng{Greater London Council.} 
\emng
\eentry

\bentry
\word[gleam(1)]{gleam}
\pron{gilxVmf}
\gl{\nA}
\bmng
\bnum
\num{1} minugu; thaLathaLa; kaSxNikaparxBe; hoLapu. 
\num{2} maMdaparxkAsha; nasuhoLapu. 
\num{3} minugu; kiraNa; kaLe; kiSxVNavU, kaSxNikavU, caMcalavU Agi kANisikoLuLxva yAvudeV guNa \mo vu: \eng{an occasional gleam of humour} Agomemx Igomemx minuguva hAsayx. \eng{not a gleam of hope} AshAkiraNavU ilalx. 
\enum
\emng
\eentry

\bentry
\word[gleam(2)]{gleam}
\pron{gilxVmf}
\gl{\akirx}
\bmng
\bnum
\num{1} minugu; minuguTuTx; thaLathaLa hoLe; mirumirugu. 
\num{2} maMda parxkAshadiMda yA caMcala parxkAshadiMda hoLe. 
\enum
\emng
\eentry

\bentry
\word{gleamy}
\pron{gilxVmi}
\gl{\gu}
\bmng
\bnum
\num{1} mirumiruguva; minugutitxruva. 
\num{2} (bisilina \vi) caMcala; Agomemx Igomemx beLaguva. 
\num{3} (beLakina yA baNaNxda \vi) minuguva savxBAvada; mirumiruguva lakaSxNavuLaLx. 
\enum
\emng
\eentry

\bentry
\word{glean}
\pron{gilxVnf}
\gl{\sakirx}
\bmng
\bnum
\num{1} hakakxlAyu; koyulxgAraru biTuTx hoVgiruva dhAnayxda tenegaLanunx sheVKarisiko. 
\num{2} (hola, gadedx, \mo vanunx) koyulxgAraru biTuTxhoVdadadxnenxlalx AyudxkoMDu baridu mADibiDu. 
\num{3} (sudidx, satAyxMshagaLu, \mo vanunx) savxlapx savxlapxvAgi saMgarxhisu, kUDahAku, sheVKarisu. 
\enum
\emng

\noindent
\gl{\akirx}
\bmng
 hakakxlAyu; koyulxgAraru biTuTx hoVgiruva dhAnayxda tenegaLanunx Ayudxko. 
\emng
\eentry

\bentry
\word{gleaner}
\pron{gilxVnarf}
\gl{\nA}
\bmng
\bnum
\num{1} hakakxlu -- Ayuvavanu, AyuvaMthadu. 
\num{2} (viSaya, sudidx, \mo vanunx) saMgarxhisuvavanu; saMgArxhaka. 
\enum
\emng
\eentry

\bentry
\word{gleaning}
\pron{gilxVniMgf}
\gl{\nA}
\bmng
\bnum
\num{1} hakakxlu; hakakxlAyuvudu. 
\num{2} (viSaya, sudidx, \mo vanunx) saMgarxhisuvudu; saMgarxhaNa. 
\enum
\emng
\eentry

\bentry
\word{glebe}
\pron{gilxVbf}
\gl{\nA}
\bmng
\bnum
\num{1} pAdirxya mAnayx; vikArf \mo\ dajeRya pAdirxyu oMdu caciRnalilx adhikAradalilxdAdxga tananx jiVvikeyAgi baLasalu hakukxLaLx, caciRge seVrida datitx jamInu. 
\numi{2} (\kAparx) 
\banum
\alnum{a} BUmi; nela; sasayxsaMtAna niVDuva BUmi tAyi. 
\alnum{b} jamInu; keSxVtarx; hola yA gadedx. 
\eanum
\numie
\enum
\emng
\eentry

\bentry
\word{glede}
\pron{gilxVDf}
\gl{\nA}
\bmng
 (\pArxparx\ matutx \pArxM) hadudx; gaqdharx. 
\emng
\eentry

\bentry
\word{glee}
\pron{gilxV}
\gl{\nA}
\bmng
\bnum
\num{1} cAraNa giVte; pakakxvAdayxvilalxde hADuva samUha giVte; kaqtiya oMdoMdu BAgakUkx obobxbabxraMte mUru yA inUnx hecucx maMdiya (\kanmu\ pArxpatxvayasakx puruSara) shAriVragaLige aLavaDisi, gaMBiVravoV laGuvoV ada padasaraNiyalilx heNedu, halamomemx parasapxra veYdaqshayxvuLaLx layagaLalilx saMcarisuva matutx pakakxvAdayxgaLilalxdeyeV hADalu yoVgayxvAda, saMgiVta kaqti. 
\num{2} geluvu; higugx; haSaR; ulAlxsa tuMbi horahomumxva saMtoVSa. 
\enum
\emng
\eentry

\bentry
\wordnospeech{glee club}{glee club}
\pron{?}
\gl{\nA}
\bmng
 cAraNa giVteya -- kalxbubx, saMGa; pakakxvAdayxvilalxde giVtegaLanunx hADuva kalxbubx. 
\emng
\eentry

\bentry
\word{gleeful}
\pron{gilxVphulf}
\gl{\gu}
\bmng
 ulAlxsaBarita; hirihigugxva; (horahomumxtitxruvaMtha) haSaR tuMbida. 
\emng
\eentry

\bentry
\word{gleefully}
\pron{gilxVphuli}
\gl{\kirxvi}
\bmng
 ulAlxsa tuMbi; hirihigugxtatx; horahomumxtitxruvaMtha haSaRdiMda tuMbi. 
\emng
\eentry

\bentry
\word{gleeman}
\pron{gilxVmanf}
\gl{\nA}
\bmng
 (\ca) caracAraNa; UrUru tiruguva gAyaka. 
\emng
\eentry

\bentry
\word{gleesome}
\pron{gilxVsamf}
\gl{\gu}
\bmng
  = \hyperlink{gleeful}{gleeful}. 
\emng
\eentry

\bentry
\word{gleet}
\pron{gilxVTf}
\gl{\nA}
\bmng
 (\roVshA) (\kanmu\ beVrUrida ganoVriyadalilx) sArxva; soVrike; gAya, huNuNx, \mo vugaLiMda horabaruva roVga darxva, yA mUtarx nALagaLiMda horabaruva meVha sArxva. 
\emng
\eentry

\bentry
\word{gleety}
\pron{gilxVTi}
\gl{\gu}
\bmng
 (\roVshA) (gAya, huNuNx, \mo vugaLa \vi\ darxva) soVruva; osaruva; soVrutitxruva. 
\emng
\eentry

\bentry
\wordf{Gleichschaltung}
\pron{gelxYKfSAYxluTxknuf}
\gl{\nA}
\expl{\G}
\bmng
 savARdhikAri rASaTxrXgaLalilx rAjakiVya, AthiRka, sAmAjika, \mo\ saMsethxgaLanunx EkarUpatege taruvudu. 
\emng
\eentry

\bentry
\word{glen}
\pron{gelxnf}
\gl{\nA}
\bmng
 kaMdara; ikakxTATxda kaNive. 
\emng
\eentry

\bentry
\word{glengarry}
\pron{gelxnfgAYxri}
\gl{\nA}
\bmng
 gelxnfgAYxri; sAkxTelxMDfna, heYlaMDf parxdeVshada, cUpAda muMBAgavuLaLx matutx \sA\ hiMBAgadalilx oMdu jote ribabxnunxgaLanunx kaTiTxruva, oMdu bageya ToVpi. \imglink{glengarryfigure}{\raisebox{-0.30cm}[0pt][0pt]{\pdfimage width 0.5cm height 0.6cm{G_Pictures/glengarry.jpg}}} 
\emng
\eentry

\bentry
\wordRemoveSpace{glenoid-cavity}{glenoid cavity}
\pron{gilxVnAyfDx kAYxviTi}
\gl{\nA}
\bmng
 (\aMrashA) gilxVnAyfDx kuhara, kuLi; hegalelubina meVlABxgadalilxruva, toVLu mULe ADalu Asapxda mADikoDuva, teTeTxyAda kuhara. 
\emng
\eentry

\bentry
\word{gley}
\pron{gelxV}
\gl{\nA}
\bmng
 niVli bUdubaNaNxda, niVru niMta nela. 
\emng
\eentry

\bentry
\word{glia}
\pron{gelxYa}
\gl{\nA}
\bmng
 (\aMrashA) galxya; keVMdarx naravUyxhadalilx nijavAda nara Utakakekx AhAra odagisi adanunx oTATxgi kUDisuva, naragaLiMdAgirada, nArunArAda Utaka. 
\emng
\eentry

\bentry
\word{glial}
\pron{gelxYalf}
\gl{\gu}
\bmng
 galxyakekx saMbaMdhisida yA galxyada. 
\emng
\eentry

\bentry
\word{gliadin}
\pron{gelxYaDinf}
\gl{\nA}
\bmng
 (\jiVra) 
\bnum
\num{1} galxyaDinf; goVdhi, reY matutx aMtaha itara dhAnayxgaLalilx sikukxva porxVlamInf vagaRda oMdu porxVTiVnu. 
\num{2} yAvudeV porxVlamInf. 
\enum
\emng
\eentry

\bentry
\word{glib}
\pron{gilxbf}
\gl{\gu}
\bmng
\bnum
\num{1} (\viparx) (meVlemxY \mo vugaLa \vi) nuNupAda; nayavAda; parxtibaMdha uMTumADada. 
\num{2} (\pArxparx) (calaneya \vi) aDetaDeyilalxda; sarAgavAda; saliVsAda; sugamavAda. 
\numi{3} (mAtu, BASaNa, \mo vugaLa \vi) 
\banum
\alnum{a} niragaRLavAda; taDebaDeyilalxda; parxvAhadaMtiruva. 
\alnum{b} Ashu; sidadhxte beVkilalxda; takaSxNa mADabahudAda. 
\alnum{c} (vicArapUNaRvAgi, pArxmANikavAgi iruvudara badalu) mAtina surimaLeya; vAgaJxriya; keVvala vAgADaMbarada. 
\eanum
\numie
\enum
\emng
\eentry

\bentry
\word{glibly}
\pron{gilxbfli}
\gl{\kirxvi}
\bmng
(mAtu, BASaNa, \mo vugaLa \vi) 
\banum
\alnum{a} sarAgavAgi; saliVsAgi; aDetaDeyilalxde; niragaRLavAgi. 
\alnum{b} vAgADaMbaradiMda; mAtina maLegareyutatx; bariya vAgfJariyiMda kUDi. 
\eanum
\emng
\eentry

\bentry
\word{glibness}
\pron{gilxbfnisf}
\gl{\nA}
\bmng
 (mAtanADuvudu \mo vugaLa \vi) 
\banum
\alnum{a} niragaRLate; parxvAhadaMtiruvike. 
\alnum{b} keVvala vAgADaMbarate. 
\eanum
\emng
\eentry

\bentry
\word[glide(1)]{glide}
\pron{gelxYDf}
\gl{\sakirx}
\bmng
\bnum
\num{1} jArisu; sarAgavAgi muMdUDu: \eng{light airs glided the ship on her course} maMdamAruta haDaganunx sarAgavAgi muMdUDitu. 
\num{2} gelxYDarf vimAnadalilx parxyANa mADu. 
\enum
\emng

\noindent
\gl{\akirx}
\bmng
\bnum
\num{1} (darxva, haDagu, hakikx, gADi, hAvu, sekxVTf mADuvavanu, \mo vara \vi) taDeyilalxde, nilalxde, maMdagatiyiMda -- sari, muMduvari, sathxLadiMda sathxLakekx hoVgu. 
\num{2} (vimAnada \vi) eMjininxna shakitx baLasade, eMjininxlalxde hAru. 
\num{3} jAru; mwnavAgi, sadidxlalxde, yArigU tiLiyadaMte -- jAriko, nusuLu, hoVgu. 
\num{4} (kAla \mo vugaLa \vi) sariduhoVgu; jArihoVgu; melalxge, tiLiyadaMte, ariveV Agade -- kaLeduhoVgu. 
\num{5} (karxmakarxmavAgi, arivAgadeyeV beVroMdaroDane) beretu hoVgu; liVnavAgi hoVgu. 
\enum
\emng
\eentry

\bentry
\word[glide(2)]{glide}
\pron{gelxYDf}
\gl{\nA}
\bmng
\bnum
\num{1} jAruvudu; sarita; jArike; jAri, saridu hoVguvudu. 
\num{2} (\saM) savxrajAru; gAyana yA vAdanavanunx nililxsade, savxrasAthxnavanunx badalAyisuvAga uMTAguva savxravAhini. 
\num{3} (\dhavxni) dhavxnijAru; ucAcxraNeyanunx mADutitxruvaMteyeV ucAcxraNAMgagaLa sAthxnavanunx vayxtAyxsa mADidAga horaDuva, karxmeVNa vayxtAyxsavAguva dhavxni. 
\num{4} (kirxkeTf)  = \hyperlink{glance(2)2}{$^2$glance \((2)\)} 
\num{5} jAru kuNita; jArunataRna yA jAru nataRnada hejejx. 
\num{6} gelxYDarf vimAnadalilx hArATa. 
\enum
\emng
\eentry

\bentry
\wordnospeech{glide path}{glide path}
\pron{?}
\gl{\nA}
\bmng
 jArupatha; avaroVhaNa patha; BUmigiLiyuva vimAnavu iLiyuva reVKe (\kanmu\ nelada meVlina reVDArf salakaraNeyu vimAna cAlakanige toVrisuvaMthadu). 
\emng
\eentry

\bentry
\word{glider}
\pron{gelxYDarf}
\gl{\nA}
\bmng
 jAruga; jAruva vayxkitx yA vasutx. 
\bnum
\num{2} gelxYDaru (vimAna); eMjinf ilalxde hAruva vimAna. 
\num{3} gelxYDaru (vimAna) -- cAlaka, naDesuvavanu. 
\num{4} gelxYDaru (vimAnada cAlaneyalilx) pariNata. 
\enum
\emng
\eentry

\bentry
\word{glidingly}
\pron{gelxYDiMgfli}
\gl{\kirxvi}
\bmng
\bnum
\num{1} sarAgavAgi; saliVsAgi; sugamavAgi; sukaravAgi. 
\num{2} nidhAnavAgi sariyutatx, jArutatx. 
\enum
\emng
\eentry

\bentry
\word{glim}
\pron{gilxmf}
\gl{\nA}
\bmng
\bnum
\num{1} maMda -- beLaku, parxkAsha. 
\numi{2} (\pArxparx) (\ashi) 
\banum
\alnum{a}moVMbatitx. 
\alnum{b} lAMdarx; lATiVnu. 
\alnum{c} kaNuNx. 
\eanum
\numie
\enum
\emng
\eentry

\bentry
\word[glimmer(1)]{glimmer}
\pron{gilxmarf}
\gl{\akirx}
\bmng
 kiSxVNavAgi yA caMcalavAgi minugu; maMkAgi yA biTuTxbiTuTx hoLe. 
\emng
\eentry

\bentry
\word[glimmer(2)]{glimmer}
\pron{gilxmarf}
\gl{\nA}
\bmng
\bnum
\num{1} minugu; kiSxVNavAda yA caMcalavAda beLaku, parxkAsha. 
\num{2} (Ashe, athaRvaMtike, \mo vugaLa) nasu beLaku; minugu; alapxsavxlapx parxkAsha. 
\num{3} nasunoVTa; kaSxNanoVTa; minugunoVTa. 
\num{4} arenoVTa; asapxSaTx toVkeR. 
\enum
\emng
\eentry

\bentry
\word{glimmering}
\pron{gilxmariMgf}
\gl{\nA}
\bmng
\bnum
\num{1} kiSxVNavAgi yA caMcalavAgi minuguvudu; maMkAgi yA Agomemx Igomemx hoLeyuvudu. 
\num{2}  = \hyperlink{glimmer(2)}{$^2$glimmer}. 
\enum
\emng
\eentry

\bentry
\word[glimpse(1)]{glimpse}
\pron{gilxMpfsx}
\gl{\nA}
\bmng
\bnum
\num{1} minugunoVTa; kiSxVNavU kaSxNikavU Ada noVTa, toVkeR. 
\num{2} nasunoVTa; minugunoVTa; kaSxNadashaRna; kaSxNikavU aparipUNaRvU Ada dashaRna. 
\enum
\emng

\noindent
\gl{\pagu}
\bmng
\eng{glimpses of the moon} 
\banum
\alnum{a} rAtirxya kAlada BUmiya daqshayx. 
\alnum{b} (\rUpa) lwkika daqshayxgaLu, vayxvahAragaLu. 
\eanum
\emng
\eentry

\bentry
\word[glimpse(2)]{glimpse}
\pron{gilxMpfsx}
\gl{\sakirx}
\bmng
 nasunoVTa kANu; kiSxVNavAgi aMshataH kANu; kaSxNadashaRnamADu. 
\emng

\noindent
\gl{\akirx}
\bmng
\bnum
\num{1} mabAbxgi hoLe; maMdavAgi parxkAshisu. 
\num{2} biTuTxbiTuTx hoLe. 
\num{3} (\kAparx) mabAbxgi kANisiko; asapxSaTxvAgi goVcarisu. 
\enum
\emng
\eentry

\bentry
\word[glint(1)]{glint}
\pron{gilxMTf}
\gl{\sakirx}
\bmng
\bnum
\num{1} (beLakanunx) miMcisu; thaTaTxne hoLedu mareyAguvaMte mADu. 
\num{2} (beLakanunx) parxtiPalisu. 
\enum
\emng

\noindent
\gl{\akirx}
\bmng
\bnum
\num{1} miMcu; thaTaTxne hoLedu mareyAgu. 
\num{2} mirugu; PaLaPaLa hoLe; thaLathaLisu. 
\enum
\emng
\eentry

\bentry
\word[glint(2)]{glint}
\pron{gilxMTf}
\gl{\nA}
\bmng
\bnum
\num{1} miMcu; thaTaTxne hoLedu mAyavAguvudu. 
\num{2} miruguvike; minugu; PaLaPaLa hoLeyuvike; thaLathaLisuvike. 
\enum
\emng
\eentry

\bentry
\word[glissade(1)]{glissade}
\pron{gilxsA(seV)Df}
\gl{\nA}
\bmng
\bnum
\num{1} (pavaRtAroVhaNa sAhasadalilx) jArATa; \kanmu\ himagaDeDxya yA maMjina rAshiya kaDidAda iLijArinalilx himagudadxli \mo vugaLa AsareyiMda, \sA\ kAlUri niMtu, jAruvudu. 
\num{2} (nataRna) jAruhejejx. 
\enum
\emng
\eentry

\bentry
\word[glissade(2)]{glissade}
\pron{gilxsA(seV)Df}
\gl{\akirx}
\bmng
\bnum
\num{1} (pavaRtAroVhaNadalilx) himagaDeDxya yA maMjina rAshiya kaDidAda iLijArinalilx himagudadxli \mo vugaLa AsareyiMda, \sA\ kAlUri niMtu, jAru. 
\num{2} (nataRna) jAruhejejx hAku. 
\enum
\emng
\eentry

\bentry
\word{glissando}
\pron{gilxsAYxMDoV}
\gl{\nA}
\expl{(\bava\ \eng{glissandi,} \ucAcx\ gilxsAYxMDiV yA \eng{glissandos}).}
\bmng
 (\saM) jAru savxrAvaLi; hADuvudanunx yA vAdanavanunx nililxsade, savxradiMda savxrakekx hoVguvAga janisuva savxragaLa anukarxma. 
\emng
\eentry

\bentry
\wordf{glisse}
\pron{gilxVseV}
\gl{\nA}
\expl{\F}
\bmng
(bAyxle nataRna) aMgAlanunx hecAcxgi upayoVgisuva jAruhejejx. 
\emng

\noindent
\gl{\pagu}
\bmng
 \eng{pas glisse} = \hyperlink{glisse}{\it glisse}. 
\emng
\eentry

\bentry
\word[glisten(1)]{glisten}
\pron{gilxsa(sf)nf}
\gl{\akirx}
\bmng
\bnum
\num{1} biTuTxbiTuTx hoLe, minugu; Agomemx Igomemx hoLe. 
\num{2} thaLathaLisu. 
\enum
\emng
\eentry

\bentry
\word[glisten(2)]{glisten}
\pron{gilxsa(sf)nf}
\gl{\nA}
\bmng
 thaLathaLa; minugu; mirugu; hoLapu. 
\emng
\eentry

\bentry
\word[glister(1)]{glister}
\pron{gilxsaTxrf}
\gl{\akirx}
\bmng
 (\pArxparx) thaLathaLisu; mirugu; PaLaPaLa hoLe. 
\emng
\eentry

\bentry
\word[glister(2)]{glister}
\pron{gilxsaTxrf}
\gl{\nA}
\bmng
 (\pArxparx) mirugu; hoLapu; PaLaPaLa hoLeyuvudu. 
\emng
\eentry

\bentry
\word{glitch}
\pron{gilxcf}
\gl{\nA}
\bmng
 (\AmA) 
\bnum
\num{1} upakaraNa \mo vugaLu idadxkikxdadxMte eDADxdiDiDxyAgi ilalxve tapupxtapApxgi kelasamADuvudu. 
\num{2} (yAMtirxka) aDacaNe; toDaku. 
\enum
\emng
\eentry

\bentry
\word[glitter(1)]{glitter}
\pron{gilxTarf}
\gl{\akirx}
\bmng
\bnum
\num{1} minugu. 
\num{2} hoLe; parxkAshisu; minugu; miMcu; thaLathaLisu. 
\num{3} (vayxkitxya \vi vasAtxrXBaraNagaLiMda) shoVBisu; JugaJagisu. 
\num{4} (ratanxgaLu, oDavegaLa \vi) JagaJagisu; hoLe. 
\num{5} (\rUpa) ujavxlavAgiru; veYBavadiMdiru; beDaginiMda kUDiru: \eng{glittering prospects} ujavxla BaviSayx. \eng{glittering rhetoric} beDagina vAgfJari; veYBavada vAgADaMbara. 
\enum
\emng
\eentry

\bentry
\word[glitter(2)]{glitter}
\pron{gilxTarf}
\gl{\nA}
\bmng
\bnum
\num{1} minugu. 
\num{2} hoLapu; parxkAsha. 
\num{3} thaLathaLike. 
\num{4} (ratanxgaLu, oDavegaLa \vi) JagaJagisuvike. 
\num{5} (\rUpa) veYBava; ujavxlate. 
\enum
\emng
\eentry

\bentry
\word{gloaming}
\pron{golxVmiMgf}
\gl{\nA}
\bmng
 (\kAparx) mucacxMje; saMjeya -- mabubx, beYgu. 
\emng
\eentry

\bentry
\word[gloat(1)]{gloat}
\pron{golxVTf}
\gl{\akirx}
\bmng
 (kAmukate, durAshe, devxVSa, vijayoVtAsxha, \mo vugaLiMda) higugx; ububx; biri; kaNiNxge yA manasisxge hababxmADiko; kaNuNx yA manasusx taqpitxpaDisiko, taNisiko: \eng{he gloats over his rival's bad luck} edurALiya duradaqSaTxvanunx kaMDu avanu higugxtAtxne. 
\emng
\eentry

\bentry
\word[gloat(2)]{gloat}
\pron{golxVTf}
\gl{\nA}
\bmng
\bnum
\num{1} higugxvike; ububxvike; saMtoVSa. 
\num{2} (vijayoVtAsxhada saMtaqpitxyiMda) hirihiri higugxva noVTa, mAtu, yA muKaBAva. 
\enum
\emng
\eentry

\bentry
\word{gloatingly}
\pron{golxVTiMgfli}
\gl{\kirxvi}
\bmng
 (kAmukate, durAshe, devxVSa, \mo vugaLiMda) higugxtatx; ububxtatx; kaNiNxge yA manasisxge hababx mADikoLuLxtatx. 
\emng
\eentry

\bentry
\word{global}
\pron{golxVbalf}
\gl{\gu}
\bmng
\bnum
\num{1} vishavxvAyxpaka. 
\num{2} oTiTxna; samaSiTxya; halavu aMshagaLu, vagaRgaLu, \mo vugaLa saMpUNaR samudAyavanonxLagoMDiruva yA adakekx saMbaMdhisida. 
\num{3} saMpUNaRvAda; samagarxvAda; iDiV. 
\enum
\emng
\eentry

\bentry
\word[globe(1)]{globe}
\pron{golxVbf}
\gl{\nA}
\bmng
\bnum
\num{1} goVLa; maMDala; guMDu; goVLAkArada kAya. 
\num{2} BUgoVLa; BUmaMDala; BUmi. 
\num{3} garxha. 
\num{4} nakaSxtarx. 
\num{5} sUyaR. 
\num{6} goVLa; goVLapaTa; BUpaTavanunx yA aMtarikaSxda nakeSxyanunx barediruva goVLa. 
\num{7} (sAvaRBwmatavxda saMkeVtavAgi baLasuva) savxNaRgoVLa; cinanxda goVLa. 
\num{8} (\aMrashA) kaNuNxguDeDx. 
\num{9} gAjugoVLa; gAjuburuDe; sarisumAru guMDAgiruva, gAjupAterx yA buruDe (\kanmu\ diVpada buruDe yA mInu sAkuva gAjina goVLa yA viduyxdidxVpada buruDe). 
\enum
\emng

\noindent
\gl{\pagu}
\bmng
\bnum
\num{1} \eng{globe} \hyperref{kandict_a.pdf}{A}{artichoke pagu(1)}{artichoke}. 
\num{2} \eng{use of the globes} (\pArxparx) goVLapaTagaLa mUlaka kalike; BUgoVLa matutx KagoVLagaLanunx goVLapaTagaLa mUlaka kalisuvudu. 
\enum
\emng
\eentry

\bentry
\word[globe(2)]{globe}
\pron{golxVbf}
\gl{\sakirx}
\bmng
 (\sA\ \kaparx) goVLiVkarisu; goVLAkAra koDu. 
\emng

\noindent
\gl{\akirx}
\bmng
 goVLavAgu; goVLAkAravAgu. 
\emng
\eentry

\bentry
\word{globe-fish}
\pron{golxVbfphiSf}
\gl{\nA}
\bmng
 budadxli mInu; buruDe mInu; usirUdikoMDu guMDAgabalalx mInu. 
\emng
\eentry

\bentry
\word{golbe-flower}
\pron{golxVbfphwlxarf}
\gl{\nA}
\bmng
 goVLa -- hU, puSapx; guMDAda haLadi baNaNxda hU biDuva, TArxliyasf kulakekx seVrida sasayx. 
\emng
\eentry

\bentry
\wordnospeech{globe lightning}{globe lightning}
\pron{?}
\gl{\nA}
\bmng
 = \hyperref{kandict_b.pdf}{B}{ball lightning}{ball lightning}. 
\emng
\eentry

\bentry
\word{globe-trotter}
\pron{golxVbfTArxTarf}
\gl{\nA}
\bmng
 vishavxsaMcAri; noVTa noVDaleMdu videVshagaLalilx AturAturavAgi saMcarisuvavanu. 
\emng
\eentry

\bentry
\word{globe-trotting}
\pron{golxVbfTArxTiMgf}
\gl{\nA}
\bmng
 vishavx saMcAra; noVTa noVDalikAkxgi videVshagaLalilx AturAturavAgi parxyANa mADuvudu. 
\emng
\eentry

\bentry
\word{globigerina}
\pron{golxVbijareY(riV)na}
\gl{\nA}
\bmng
 golxVbijareYna; samudarxda meVlemxY hatitxra jiVvisuva, suNaNxyukatx cipupxLaLx jiVvigaLa kuLa. 
\emng
\eentry

%%%%%%%%Till here second level correction over(prasaranga)%%%%%%%%%%%%%%%%%%%%%%%2/06/2005%%%%%%%%%%%%%%%%%%%%%
\bentry
\word{globin}
\pron{golxVbinf}
\gl{\nA}
\bmng
(\jiVra) golxVbinf; rakatxda hiVmogolxVbininxna keMpu baNaNxkekx kAraNavAda hiVminf eMba vaNaRdarxvayxvanunx hiVmogolxVbininxniMda parxteyxVkisidare uLiyuva porxVTiVnf BAga. 
\emng
\eentry

\bentry
\word[globoid(1)]{globoid}
\pron{golxVbAyfDx}
\gl{\nA}
\bmng
 goVLABa; hecucx kaDime goVLadaMtiruva vasutx yA Akaqti. 
\emng
\eentry

\bentry
\word[globoid(2)]{globoid}
\pron{golxVbAyfDx}
\gl{\gu}
\bmng
 goVLABa; goVLasadaqsha; goVLadaMtha; hecucx kaDame goVLadaMtiruva. 
\emng
\eentry

\bentry
\word{globose}
\pron{golxVboVsf}
\gl{\gu}
\bmng
 goVLAkArada; saMpUNaRvAgiyoV sarisumArAgiyoV goVLadaMtiruva. 
\emng
\eentry

\bentry
\word{globosity}
\pron{golxVbAsiTi}
\gl{\nA}
\bmng
 goVLate; goVLAkAradalilxruvike. 
\emng
\eentry

\bentry
\word{globular}
\pron{gAlxbuyxlarf}
\gl{\gu}
\bmng
\bnum
\num{1} goVLAkArada; goVLAkaqtiya; guMDagiruva. 
\num{2} goVLa kaNa -- racita, GaTita; goVLakaNagaLiMda GaTitavAda; goVLakaNagaLu seVri racitavAda. 
\enum
\emng
\eentry

\bentry
\word{globularity}
\pron{gAlxbuyxlAYxriTi.}
\gl{\nA}
\bmng
goVLate: 
\banum
\alnum{a} goVLAkAradalilxruvike. 
\alnum{b} goVLakaNa GaTitate; goVLakaNagaLiMda racitavAgiruvudu. 
\eanum
\emng
\eentry

\bentry
\word{globularly}
\pron{gAlxbuyxlarfli}
\gl{\kirxvi}
\bmng
 goVLAkAravAgi. 
\emng
\eentry

\bentry
\word{globule}
\pron{gAlxbUyxlf}
\gl{\nA}
\bmng
\bnum
\num{1} goVLaka; saNaNx goVLa. 
\num{2} goVLakaNa. 
\num{3} hani; boTuTx; toTuTx. 
\num{4} guLige. 
\enum
\emng
\eentry

\bentry
\word{globulin}
\pron{gAlxbuyxlinf}
\gl{\nA}
\bmng
 gAlxbuyxlinf; niVrinalilx karagada, Adare sArarikatx lavaNagaLa dArxvaNagaLalilx karaguva, shAKadiMda garaNegaTuTxva oMdu porxVTiVnf vagaR. 
\emng
\eentry

\bentry
\word{glochidiate}
\pron{golxVkiDiETf}
\gl{\gu}
\bmng
 (\savi) muLuLxtudiya; shUkAgarxda; tudiyalilx koMkida muLiLxruva. 
\emng
\eentry

\bentry
\word{glockenspiel}
\pron{gAlxkanfsipxVlf}
\gl{\nA}
\bmng
 (\saM) gAlxkanfsipxVlf; GaMTA(tALa)vAdayx; karxmabadadhxvAgi joVDisida gaMTegaLa yA loVhapaTiTxgaLa sherxVNiyuLaLx matutx eraDu (saNaNx) sutitxgegaLa tADanadiMda vAdana mADuva oMdu pAshAcxtayx tALa vAdayx. \imglink{glockenspielfigure}{\raisebox{-0.35cm}[0pt][0pt]{\pdfimage width 0.6cm height 0.6cm {G_Pictures/glockenspiel.jpg}}} 
\emng
\eentry

\bentry
\word{glom}
\pron{gAlxmf}
\gl{\kirx}
\expl{(\vakaq\ \eng{glomming,} \BU\ matutx \BUkaq\ \eng{glommed}).}
\bmng
\emng

\noindent
\gl{\sakirx}
\bmng
 (\ame) (\ashi) kadi; lapaTAyisu; egurisu; kasiduko (\akirx\ saha). 
\emng

\noindent
\gl{\pagu}
\bmng
 \eng{glom on to} = \hyperlink{glom}{glom}. 
\emng
\eentry

\bentry
\word{glomerate}
\pron{gAlxmareV(ra)Tf}
\gl{\gu}
\bmng
 (\savi matutx \aMrashA) ceMDugUDida; goMcalAda; gucaCxvAda; otAtxgi guMpugUDiruva. 
\emng
\eentry

\bentry
\word{glomerular}
\pron{galxmeralarf}
\gl{\gu}
\bmng
 (saNaNx jiVvigaLu, UtakagaLu, \mo vugaLa \vi) goMcalAgiruva; goMDeyaMtha yA adakekx saMbaMdhisida. 
\emng
\eentry

\bentry
\word{glomerule}
\pron{galxmarUlf}
\gl{\nA}
\bmng
\bnum
\num{1} hUgoMcalu; puSapxgucaCx; hUgudi. 
\num{2} goMcalu; gucaCx; saNaNx saNaNx jiVvigaLu yA jiVvANugaLu, aMgAMshagaLu, rakatxnALagaLu, \mo vugaLa guMpu. 
\enum
\emng
\eentry

\bentry
\word{glomerulus}
\pron{galxmeralasf}
\gl{\nA}
\expl{(\bava\ \eng{glomeruli} \ucAcx, galxmeraleY).}
\bmng
 (saNaNx saNaNx jiVvigaLu, UtakagaLu, rakatxnALagaLu, \mo vugaLa, mUtarxpiMDagaLalilxna rakatxnALagaLa) gucaCx; goMcalu; goMDe; guMpu. 
\emng
\eentry

\bentry
\word[gloom(1)]{gloom}
\pron{gUlxmf}
\gl{\nA}
\bmng
\bnum
\num{1} katatxle; iruLu; timira; aMdhakAra. 
\num{2} mabubx; masuku; masukAda sithxti. 
\num{3} mAlxnate; gAlxni; maMku; Kinanxte; viSaNaNxte; geluvilalxda, maMkubaDida sithxti. 
\num{4} nirAshe; AshArahitate. 
\num{5} (\kAparx) katatxlu parxdeVsha. 
\enum
\emng
\eentry

\bentry
\word[gloom(2)]{gloom}
\pron{gUlxmf}
\gl{\sakirx}
\bmng
\bnum
\num{1} mabAbxgisu; masukAgisu. 
\num{2} maMkugavisu; viSaNaNxgoLisu; mAlxnagoLisu; gAlxniyuMTumADu. 
\num{3} aMdhakAragoLisu; katatxlu kavidaMte mADu. 
\enum
\emng

\noindent
\gl{\akirx}
\bmng
\bnum
\num{1} hububx gaMTikukx; siDukugoMDaMte kANu; gaMTumoVre hAkiko; muKa siMDarisu. 
\numi{2} (AkAsha \mo vugaLa \vi) 
\banum
\alnum{a} mabubxgavidiru; moVDa mucicxru; maMku kavidiru. 
\alnum{b} (birumaLe baruvaMte) maMku kavidiru; BayaMkaravAgiru. 
\eanum
\numie
\num{3} kapApxgi, masuku masukAgi yA asapxSaTxvAgi kANu. 
\num{4} maMkAgiru; mAlxnavAgiru; KinanxvAgiru; viSAdadiMda kUDiru. 
\enum
\emng
\eentry

\bentry
\word{gloomily}
\pron{gUlxmili}
\gl{\kirxvi}
\bmng
\bnum
\num{1} gelavilalxdaMte; maMkubaDidaMte; mAlxnavAgi; gAlxni tuMbi. 
\num{2} siDukugoMDu; gaMTumoVre hAki; muKa siMDarisikoMDu. 
\num{3} gelavilalxde; ulAlxsarahitavAgi; nirAseyiMda; goVLugareyutatx. 
\enum
\emng
\eentry

\bentry
\word{gloominess}
\pron{gUlxminisf}
\gl{\nA}
\bmng
\bnum
\num{1} mabubx; masuku; katatxlegavida sithxti. 
\num{2} mAlxnate; gAlxni; Kinanxte; viSaNaNxte; geluvilalxdiruvike; maMkubaDidaMtiruvudu; kaLegeTiTxruvudu. 
\num{3} goVLu kareyuvaMtiruvudu; kaLegeTaTx sithxti. 
\enum
\emng
\eentry

\bentry
\word{gloomy}
\pron{gUlxmi}
\gl{\gu}
\bmng
\bnum
\num{1} katatxle kavida; mabubx musukida; iruLu kavida. 
\num{2} gelavilalxda; maMkubaDida; mAlxnate musukiruva; gAlxni tuMbida. 
\num{3} siDukugoMDa; gaMTu moVre hAkida; siMDarisida muKada. 
\num{4} kaLeyilalxda; maMku kavida; goVLukareva: \eng{a gloomy house} goVLugareva mane. 
\num{5} Kinanx goLisuva; ulAlxsageDisuva; nirAshegoLisuva. 
\enum
\emng
\eentry

\bentry
\word{glop}
\pron{gAlxpf}
\gl{\nA}
\bmng
 (\ame) (\ashi) aMTaMTAda yA darxvarUpadalilxruva vasutx, \kanmu\ kaLapeyAda yA ruciyilalxda yA AkaSaRkavalalxda -- AhAra. 
\emng
\eentry

\bentry
\word{gloria}
\pron{golxVria}
\gl{\nA}
\bmng
\bnum
\num{1} (\eng{Gloria}) (kerxYsatxralilx deVvara sotxVtarxgiVtagaLAda \eng{Gloria Patri} \mo vugaLalilxna parxthama pada hAgU A giVtagaLa \saMkiSx) jayavAgali! vaqdidhxyAgali! vadhiRsali! vadhaRtAM! aBivadhaRtAM! 
\num{2} parxBAvalaya; parxBAmaMDala; teVjoVmaMDala. 
\enum
\emng
\eentry

\bentry
\word{glorification}
\pron{golxVriphikeVSanf}
\gl{\nA}
\bmng
\bnum
\num{1} (\kanmu\ Esukirxsatxna \vi) divayxtegeVrisuvudu; deYviVpadavi koDuvudu. 
\num{2} (mAnava AtamxgaLa) savxgaR parxveVsha; savxgaRda AnaMdadalilx BAgigaLanAnxgi mADuvudu. 
\num{3} mahatavxgoLisuvudu; GanategoLisuvudu. 
\numi{4} veYBaviVkaraNa: 
\banum
\alnum{a} BavayxgoLisuvudu. 
\alnum{b} (sAmAnayxvAdudanunx yA kiVLAdudanunx) BavayxvAgi kANuvaMte mADuvudu; BavAyxkAra koDuvudu; DabubxgoLisuvudu: \eng{nothing more than a glorification of a cottage} DabubxgoLisida keVvala oMdu kuTiVra. 
\eanum
\numie
\num{5} (obabx vayxkitxya yA oMdu vasutxvina viSayada) koMDATa; hogaLike; sutxti; sotxVtarx; shAlxGane: \eng{Huxley's glorification of science} hakisxlXV mADiruva vijAcnxnada koMDATa. 
\enum
\emng
\eentry

\bentry
\word{glorify}
\pron{golxVripheY}
\gl{\sakirx}
\bmng
\bnum
\num{1} divayxtegeVrisu; deYviVpadavi koDu: \eng{Jesus shall glorify the good and bring them joy without end} puNAyxtamxranunx yeVsuvu KaMDita divayxtegeVrisi avarige anaMtavAda AnaMda taruvanu. 
\num{2} teVjoVmayagoLisu; parxkAshamAnavanAnxgi mADu: \eng{the beams of the setting sun glorify the mountain peaks} muLugutitxruva sUyaRna kiraNagaLu pavaRta shiKaragaLanunx teVjoVmayagoLisutatxve. 
\num{3} BavayxvAgisu; veYBavagoLisu; BavayxtegoLisu: \eng{Wordsworth knows how to glorify common life} sAmAnayx janara bALanunx BavayxgoLisuvudu heVgeMbudanunx vaDfsxRvatfR kavi balalx. 
\num{4} (sAmAnayxvAda yA kaLape vasutxvanunx iruvudakikxMta hecucx) veYBaviVkarisu; AkaSaRkavAgi yA celuvAgi kANuvaMte mADu yA mADalu parxyatinxsu; ramaNiVyavAda horarUpa koDu yA koDalu sharxmisu: \eng{nothing more than a glorified cottage} AkaSaRkavAgi mADida bari oMdu guDisalu, keVvala oMdu kuTiVra. 
\num{5} (bahuvAgi) koMDADu; hogaLu; shAlxGisu; kiVtiRsu; parxshaMsisu; sotxVtarx mADu: \eng{Caeser was glorified} siVsarfna mahimeyanunx atiyAgi koMDADidaru. 
\num{6} GanategoLisu; mahatavxkekxVrisu; mahimAnivxtavAgi mADu. 
\num{7} (deVvara) mahimeyanunx koMDADu, hogaLu. 
\enum
\emng
\eentry

\bentry
\word{gloriole}
\pron{golxVriOlf}
\gl{\nA}
\bmng
 parxBAvalaya; parxBAmaMDala; divayx teVjoVrAshi. 
\emng
\eentry

\bentry
\word{glorious}
\pron{golxVriasf}
\gl{\gu}
\bmng
\bnum
\num{1} mahimAnivxta; mahatavxvuLaLx; mahimeyuLaLx. 
\num{2} hesaruvetatx; hesarAMta; yashoVvaMtanAda; kiVtiR paDeda; KAyxtivetatx; suparxsidadhx; viKAyxta: \eng{by nothing is England so glorious as by her poetry} iMgelxMDf tananx kAvayxdiMda viKAyxtavAgiruvaSuTx inAnxvudariMdalU Agilalx. 
\num{3} hesaru taruva; kiVtiRdAyaka; KAyxti uMTumADuva. 
\num{4} shoVBAyamAna; Bavayx; veYBavayuta: \eng{a glorious view} Bavayx daqshayx. 
\num{5} divayx; aduBxta; mahadAnaMdakara; atAyxnaMdakara: \eng{a glorious day} atAyxnaMdakara dina. \eng{glorious fun} (\hA) oLeLx divayxvAda tamASe. \eng{a glorious muddle} (vayxMgayxdalilx) aduBxta avayxvasethx. \eng{the glorious uncertainty of cricket} kirxkeTiTxna divayx anishicxtate. 
\num{6} (\AmA) kuDidu AnaMdaparavashanAgiruva; madirAnaMdadalilxruva: \eng{as soon as one man was flogged into sobriety, another became glorious} obabxnanunx baDidu samasithxtige taruvaSaTxralelxV inonxbabx kuDidu AnaMdaparavashanAgibiDutitxdadx. 
\enum
\emng
\eentry

\bentry
\word{gloriously}
\pron{golxVriasfli}
\gl{\kirxvi}
\bmng
\bnum
\num{1} hesaruvetutx; hesarAMtu; yashoVvaMtanAgi; kiVtiR paDedu; KAyxtivetutx. 
\num{2} hesaru taruvaMte; kiVtiRdAyakavAgi; KAyxti baruvaMte. 
\num{3} atAyxnaMdakaravAgi; divayxvAgi; aduBxtavAgi. 
\num{4} (\AmA) kuDidu AnaMdaparavashanAgi: \eng{he was gloriously drunk} avanu kuDidu AnaMdaparavashanAgidadx. 
\enum
\emng
\eentry

\bentry
\word[glory(1)]{glory}
\pron{golxVri}
\gl{\nA}
\bmng
\bnum
\num{1} Ganate; mahime; veYBava. 
\num{2} hirimevetatx hesaru; meVlemx paDeda hesaru; unanxta kiVtiR; gwravayuta KAyxti; nimaRla yashasusx; dhavaLakiVtiR. 
\num{3} BakitxpUNaR koMDATa matutx kaqtajacnxte (samapaRNe). 
\num{4} BUSaNa; hememx; parxtiSeThx; visheVSa -- gwrava, birudu, sanAmxna; hememxpaDuvaMtha viSaya: \eng{this grand old castle is the chief glory of the district} I pArxciVnavAda BavayxswdhaveV I parxdeVshada parxdhAna BUSaNa. 
\num{5} ujajxvXla gAMBiVyaR, swMdayaR, Bavayxte. 
\num{6} divayx teVjoVrAshi. 
\num{7} (kalapxneya) alwkika swMdayaR. 
\num{8} savxgaRda AnaMda matutx veYBava. 
\num{9} veYBava; aBuyxdaya; ELige; ucACxrXyasithxti: \eng{is in his glory} avanu veYBava sithxtiyalilx, bahaLa ELigeyalilx idAdxne. 
\num{10} (deVvateya sutatxlU haraDiruva) parxBAvalaya; parxBAmaMDala; pariveVsha; divayxparxBe; teVjasusx. 
\num{11} hesaru, gwrava taruvaMtha vasutx. 
\num{12} veYshiSaTxyX; veYlakaSxNayx; 
\num{13} = \hyperref{kandict_a.pdf}{A}{anthelion}{anthelion}. 
\enum
\emng

\noindent
\gl{\pagu}
\bmng
 \eng{glory!} yA \eng{glory be!} 
\banum
\alnum{a} (pUjAdigaLalilx apiRsuva sutxti hAgU kaqtajacnxtA samapaRNe) jaya jaya! 
\alnum{b} (\ashi) sAmAnayx janara AshacxyaR yA saMtoVSa sUcaka udAgxra: \eng{is it only you? Oh, glory be!} ayoyxV, niVneV? eMtha AshacxyaR! eSuTx saMtoVSa! 
\eanum
\emng

\noindent
\gl{\nuga}
\bmng
\bnum
\num{1} \eng{go to glory!} (\ashi) sAyu; maqtanAgu; savxgaRsathxnAgu; divaMgatanAgu. 
\num{2} \eng{send to glory} (\hA) kolulx; savxgaRkekx kaLuhisu. 
\enum
\emng
\eentry

\bentry
\word[glory(2)]{glory}
\pron{golxVri}
\gl{\akirx}
\bmng
 (oMdu viSayakAkxgi yA kAyaR mADuvudaralilx) higugx; ububx; hememxpaDu. 
\emng
\eentry

\bentry
\word{glory-box}
\pron{golxVribAkfsx}
\gl{\nA}
\bmng
 (\AseTxrXV\ matutx nUyxsiZVlaMDf) baTeTxbare peTiTxge; maduveya sidadhxtegAgi vadhuvina baTeTxbare \mo vugaLaninxTiTxruva peTiTxge. 
\emng
\eentry

\bentry
\word{glory-hole}
\pron{golxVrihoVlf}
\gl{\nA}
\bmng
\bnum
\num{1} (\ame) gaNiyalilx toVDiruva gaviyaMtha tereda haLaLx, kuLi. 
\num{2} (\ashi) koLaku -- koThaDi, are, seLeKAne, pAterx, \mo vu. 
\enum
\emng
\eentry

\bentry
\word{gloryingly}
\pron{golxVri{i}Mgfli}
\gl{\kirxvi}
\bmng
 higugxtAtx; ububxtAtx; biVgutAtx; hememxpaDutAtx. 
\emng
\eentry

\bentry
\wordRemoveSpace{glory-of-the-snow}{glory-of-the snow}
\pron{golxVriAphfdisonxV}
\gl{\nA}
\bmng
 = \hyperref{kandict_c.pdf}{C}{chionodoxa}{chionodoxa}. 
\emng
\eentry

\bentry
\wordnospeech{Glos.}{Glos.}
\pron{?}
\gl{\saMkiSx}
\bmng
 \eng{Gloucestershire.} 
\emng
\eentry

\bentry
\word[gloss(1)]{gloss}
\pron{gAlxsf}
\gl{\nA}
\bmng
\bnum
\num{1} vivaraNa pada; garxMtha BAgada yAvudeV padavanunx vivarisalu, paMkitxgaLa naDuveyoV aruginalolxV seVrisida pada. 
\num{2} TipapxNi. 
\num{3} vivaraNe; athaR nirUpaNe. 
\num{4} BAvAnuvAda. 
\num{5} (inonxbabxna mAtugaLa) tirucaNe; mithAyxnirUpaNe; apAthaR; tapupx nirUpaNe. 
\num{6}  = \hyperlink{glossary}{glossary}. 
\num{7} sAlugaLa naDuve seVrisiruva BASAMtara, anuvAda yA vivaraNe; paMkatxyXMtara BASAMtara yA vivaraNe. 
\enum
\emng
\eentry

\bentry
\word[gloss(2)]{gloss}
\pron{gAlxsf}
\gl{\sakirx}
\bmng
\bnum
\num{1} (garxMthaBAga \mo vugaLa naDuve yA pada \mo vanunx kuritu) athaR, TipapxNi, \mo vanunx seVrisuva. 
\num{2} vAyxKAyxna, TiVke, TipapxNi, \mo vanunx bare. 
\num{3} tapapxthaR mADu; vipariVtAthaR kalipxsu. 
\num{4} samadhAnavAguvaMte heVLi, saritoVruvaMte atheYRsi teVlisi biDu, beVre baNaNx koDu. 
\enum
\emng

\noindent
\gl{\akirx}
\bmng
\bnum
\num{1} TiVke, TipapxNi, vAyxKAyxna, vivaraNe -- bare, racisu. 
\num{2} (\kanmu\ parxtikUlavAda) TiVkegaLanunx mADu. 
\enum
\emng
\eentry

\bentry
\word[gloss(3)]{gloss}
\pron{gAlxsf}
\gl{\nA}
\bmng
\bnum
\num{1} horahoLapu; bAhayxkAMti; meVle meVle kANuva kAMti; horage toVruva hoLapu. 
\num{2} husitoVkeR; mithAyxrUpa; moVsa hoVgisuva horarUpa. 
\num{3} horaceluvu; bAhayx swMdayaR; thaLaku paLaku. 
\enum
\emng
\eentry

\bentry
\word[gloss(4)]{gloss}
\pron{gAlxsf}
\gl{\sakirx}
\bmng
\bnum
\num{1} hoLeyuvaMte mADu; thaLathaLisuvaMte mADu; shoVBAyamAnavAgisu. 
\num{2} husiceluvu koDu; baNaNx kaTuTx; baNaNx koDu; raMguraMgAgi mADu. 
\num{3} maresu; kANadaMte mADu; mucicx hAku: \eng{gloss over the defects} tapupxgaLanunxmucicx hAku. 
\enum
\emng
\eentry

\bentry
\word{glossal}
\pron{gAlxsalf}
\gl{\gu}
\bmng
 (\aMrashA) nAlageya; jihevxya. 
\emng
\eentry

\bentry
\word{glossarial}
\pron{gAlxseVrialf}
\gl{\gu}
\bmng
 laGushabadx koVshada yA adakekx saMbaMdhisida. 
\emng
\eentry

\bentry
\word{glossarist}
\pron{gAlxsarisfTx}
\gl{\nA}
\bmng
\bnum
\num{1} laGushabadxkoVshakAra; saMgarxha koVshakAra. 
\num{2} vAyxKAyxnakAra; TiVkAkAra. 
\enum
\emng
\eentry

\bentry
\word{glossary}
\pron{gAlxsari}
\gl{\nA}
\bmng
\bnum
\num{1} shabAdxthaRgaLu, TipapxNi, vivaraNegaLu, \mo vugaLa -- saMkalana, saMgarxha. 
\num{2} shabadxsaMgarxha; shabAdxvaLi; laGushabadxkoVsha; saMgarxha padakoVsha; kilxSaTx, rUDhiyaLida, pArxdeVshika matutx pAriBASika padagaLa paTiTx hAgU vivaraNe. 
\enum
\emng
\eentry

\bentry
\word{glossator}
\pron{gAlxseVTarf}
\gl{\nA}
\bmng
\bnum
\num{1} saMgarxha (pada) koVshakAra. 
\num{2} (\kanmu\ madhayxyugadalilx lwkika yA vayxvahAra nAyxyashAsatxrX matutx kerxYsatx dhamaRshAsatxrXgaLa) vAyxKAyxnakAra. 
\enum
\emng
\eentry

\bentry
\word{glosseme}
\pron{gAlxsiVmf}
\gl{\nA}
\bmng
 gAlxsiVmu; padime; shabidxme; tanagiMtalU cikakxdAda athaRvishiSaTx aMshagaLanunx tananxlilx oLagoLaLxda, athaRvAhakavAda BASAMsha; sAthaRka athARMshagaLiMda GaTitavAgirada, athaRvAhaka BASAMsha. 
\emng
\eentry

\bentry
\word{glossily}
\pron{gAlxsili}
\gl{\kirxvi}
\bmng
\bnum
\num{1} hoLeyutatx; thaLathaLisutatx. 
\num{2} thaLakupaLakAgi. 
\enum
\emng
\eentry

\bentry
\word{glossiness}
\pron{gAlxsinisf}
\gl{\nA}
\bmng
\bnum
\num{1} hoLeyuvike; hoLapu; thaLathaLisuvike. 
\num{2} thaLakupaLakAgiruvudu; husi celuvuLaLxdAdxgiruvudu. 
\enum
\emng
\eentry

\bentry
\word{glossitis}
\pron{gAlxseYTisf}
\gl{\nA}
\bmng
 (\roVshA) gAlxseYTisf; nAlageya uriyUta. 
\emng
\eentry

\bentry
\word{glosso-}
\pron{gAlxsoV-}
\gl{\sapUpa}
\bmng
 nAligeya, kiru nAligeya, jihevxya, upajihevxya eMbathaRgaLalilx baLasuva \sapUpa. 
\emng
\eentry

\bentry
\word{glossographer}
\pron{gAlxsAgarxpharf}
\gl{\nA}
\bmng
 vAyxKAyxnakAra; TiVkAkAra; TipapxNigAra. 
\emng
\eentry

\bentry
\word{glossoid}
\pron{gAlxsAyfDx}
\gl{\gu}
\bmng
 jihAvxBa; nAlageyaMtha. 
\emng
\eentry

\bentry
\word{glossolalia}
\pron{gAlxsaleVlia}
\gl{\nA}
\bmng
 aparicita BASAjAcnxna; (\kanmu\ Adayx kerxYsatxrige deVvara varavAgi) baMditeMdu heVLalAda matutx kelavu Adhunika paMthadavaru tAvu paDedidedxVveMdu heVLikoLuLxva, aparicita BASegaLalilx mAtanADuva shakitx. 
\emng
\eentry

\bentry
\word{glosso-laryngeal}
\pron{gAlxsoVlAYxriMjialf}
\gl{\gu}
\bmng
 jihAvxkaMThayx; nAlage matutx dhavxnikuharada. 
\emng
\eentry

\bentry
\word{glossology}
\pron{gAlxsAlaji}
\gl{\nA}
\bmng
\bnum
\num{1} pariBASAshAsatxrX; vividha shAsatxrXgaLa pAriBASika padagaLanunx kurita shAsatxrX. 
\num{2} (\pArxparx) BASAshAsatxrX. 
\enum
\emng
\eentry

\bentry
\wordnospeech{gloss paint}{gloss paint}
\pron{?}
\gl{\nA}
\bmng
 hoLapu peVMTu; mirugu baNaNx; miruguva meVlemxYkoDalu baLasuva, vAniRSf uLaLx peVMTu. 
\emng
\eentry

\bentry
\word{glossy}
\pron{gAlxsi}
\gl{\gu}
\bmng
\bnum
\num{1} thaLakupaLakAgiruva; husi celuvina. 
\num{2} (kAgada \mo vugaLa \vi) hoLapu meYya; nayavAgiyU hoLeyuvaMteyU iruva. 
\num{3} (mAsapatirxke \mo vugaLa \vi) hoLapu kAgadada; nayavAda matutx hoLapAda kAgadada meVle mudirxsida. 
\enum
\emng
\eentry

\bentry
\word{glottal}
\pron{gAlxTalf}
\gl{\gu}
\bmng
 gAlxTalf; kaMThadAvxriVya; kAkalayx; kaMThadAvxrada, adakekx saMbaMdhisida, adaralilx yA adariMda huTiTxda. 
\emng
\eentry

\bentry
\word{glottalize}
\pron{gAlxTaleYsfZ}
\gl{\sakirx}
\bmng
 kAkaliyxVkarisu; kaMThadAvxravanunx pUtiRyAgi yA AMshikavAgi mucicx ucacxrisu. 
\emng
\eentry

\bentry
\wordnospeech{glottal stop}{glottal stop}
\pron{?}
\gl{\nA}
\bmng
 (\dhavxni) kaMThadAvxravanunx thaTaTxne teredu yA mucicx horaDisida dhavxni; kaMThadAvxriVya soPxVTa (dhavxni). 
\emng
\eentry

\bentry
\word{glottic}
\pron{gAlxTikf}
\gl{\gu}
\bmng
\bnum
\num{1}  = \hyperlink{glottal}{glottal}. 
\num{2} (\pArxparx) BASAshAsatxrXda. 
\enum
\emng
\eentry

\bentry
\word{glottis}
\pron{gAlxTisf}
\gl{\nA}
\bmng
 (\aMrashA) gAlxTisf; kaMThadAvxra; kAkala; shAvxsanALadAvxra; shAvxsanALada meVlagxDe matutx dhavxnitaMtugaLa madheyx iruva, saMkoVcana matutx visatxraNagaLiMda dhavxniya EriLitagaLanunx niyaMtirxsabalalx terapu. 
\emng
\eentry

\bentry
\word{Gloucester}
\pron{gAlxsaTxrf}
\gl{\nA}
\bmng
 gAlxsaTxrf; (iMgelxMDina) gAlxsaTxrfSeYrfnalilx tayArisida oMdu bageya gaTiTxyAda ciVsu, giNuNx. 
\emng

\noindent
\gl{\pagu}
\bmng
 \eng{double Gloucester} Dabalf gAlxsaTxrf; hiMde, tuMba puSiTxkaravAgi tayArisutitxdadx ciVsu, giNuNx. 
\emng
\eentry

\bentry
\word[glove(1)]{glove}
\pron{galxvf}
\gl{\nA}
\bmng
\bnum
\num{1} keYgavasu; togalu, hatitx, reVSemx, uNeNx yA hiMdina kAladalilx ukukx ivugaLiMda mADida, \sA\ beraLugaLu viMgaDavAgiruva, keYgaLanunx rakiSxsalu, becacxgiDalu, savxcaCxvAgiDalu, yA taMpAgirisalu baLasuva hodike. 
\num{2} (metetx hAki holida) muSiTxyudadhxda keYgavasu. 
\enum
\emng

\noindent
\gl{\nuga}
\bmng
\bnum
\num{1} \eng{fit like a glove} sariyAgi hoMdiko; gutatxnAgi hoMdiko; karAruvAkAkxgi aLavaDu. 
\num{2} \eng{hand and glove} = \hyperlink{glove nuga3}{?nuga? \((3)\)}. 
\hypertarget{glove nuga3}{} 
\num{3} \eng{hand in glove} anoyxVnayxvAgiru; AtimxVyavAgiru; obabxranonxbabxru biTiTxrade oTiTxge iru. 
\num{4} \eng{take off the gloves} (vAda, caceR, \mo vugaLalilx) birusAgi, nidARkiSxNayxvAgi, nayanAjUku lekikxsade -- vAdamADu. 
\hyperdef{G}{glove(1) nuga(5)}{} 
\num{5} \eng{take up the glove} (kALagakAkxgi, vAdakAkxgi) koTaTx kareyanunx, AhAvxnavanunx, savAlanunx opipxko. 
\num{6} \eng{throw down the glove} (kALagakekx, vAdakekx) kare; AhAvxnisu; savAluhAku. 
\hypertarget{glove nuga7}{} 
\num{7} \eng{without the golves} (vAda, caceR, \mo vugaLalilx) birusAgi; nidARkiSxNayxvAgi; nayanAjUku lekikxsade. 
\num{8} \eng{with the gloves off} = \hyperlink{glove nuga7}{?nuga? \((7)\)}. 
\enum
\emng
\eentry

\bentry
\word[glove(2)]{glove}
\pron{galxvf}
\gl{\sakirx}
\bmng
\bnum
\num{1} keYgavasugaLanunx -- havaNisu, odagisu. 
\num{2} keYgavasugaLiMda yA keYgavasugaLiMdaloV eMbaMte keYgaLanunx mucucx, Avarisu. 
\enum
\emng
\eentry

\bentry
\wordnospeech{glove box}{glove box}
\pron{?}
\gl{\nA}
\bmng
\bnum
\num{1} keYgavasu peTiTxge; keYgavasugaLanunx iDuva peTiTxge. 
\num{2}  = \hyperlink{glove compartment}{glove compartment}. 
\num{3} keYgavasu koVSaThx; oLagina vikiraNa padAthaRgaLu \mo vanunx keYgavasu dharisida keYyiMda muTaTxlu anukUlavAguvaMte keYgavasugaLanunx oLagiTuTx moharu mADida koVSaThx. 
\enum
\emng
\eentry

\bentry
\wordnospeech{glove compartment}{glove compartment}
\pron{?}
\gl{\nA}
\bmng
 DAYxSf boVDfR are gUDu; moVTAru kArina DAYxSf boVDiRnalilx saNaNxpuTaTx vasutxgaLaninxDalu mADiruva are, gUDu. 
\emng
\eentry

\bentry
\word{glove-fight}
\pron{galxvfpheYTf}
\gl{\nA}
\bmng
 keYgavasina muSiTxyudadhx; keYgavasu hAkikoMDu mADuva muSiTxyudadhx. 
\emng
\eentry

\bentry
\word{gloveless}
\pron{galxvflisf}
\gl{\gu}
\bmng
 keYhodikeyilalxda; keYgavasu toTiTxrada. 
\emng
\eentry

\bentry
\wordnospeech{glove puppet}{glove puppet}
\pron{?}
\gl{\nA}
\bmng
 keYgavasu goMbe; keYgavasinaMte keYge hoMdisida, veVSa hAkida, tale matutx keYgaLiruva -- keYgoMbe, sUtarxda boMbe. 
\emng
\eentry

\bentry
\word{glover}
\pron{galxvarf}
\gl{\nA}
\bmng
 keYgavasugAra; keYgavasugaLanunx tayArisuvavanu yA mAruvavanu. 
\emng
\eentry

\bentry
\word{glove-sponge}
\pron{galxvfsapxMjf}
\gl{\nA}
\bmng
 keYgavasu sapxMju; keYgavasina AkArada sapxMju. 
\emng
\eentry

\bentry
\word{glove-stretcher}
\pron{galxvfseTxrXcarf}
\gl{\nA}
\bmng
 keYgavasina beraLugaLanunx higigxsuva sAdhana. 
\emng
\eentry

\bentry
\word[glow(1)]{glow}
\pron{golxV}
\gl{\akirx}
\bmng
\bnum
\num{1} javxlisuvaMtAgu; javxlisuvaSuTx kAveVru; niganigi kAyu; uriyilalxde, jAvxleyilalxde beLakanUnx kAvanUnx sUsu. 
\num{2} (tiVkaSxNXvAgi kAda vasutxvinaMte) hoLe; parxkAshisu; parxjavxlisu. 
\num{3} ujavxla vaNaRdiMda -- hoLe, thaLathaLisu, raMjitavAgiru: \eng{pictures glowing with colour} ujavxla vaNaRdiMda thaLathaLisuva citarxgaLu; ujavxla vaNaRda citarxgaLu. 
\num{4} (meY bisiyiMda yA BAvada kAviniMda) raMgAgu; raMgeVru; tapisu; keMpeVru; ujavxlavAgu: \eng{glowing cheeks} raMgeVrida kenenxgaLu. \eng{gave me a glowing account of the achievement} sAdhaneya ujavxlavAda varadiyanunx koTaTxnu. 
\enum
\emng
\eentry

\bentry
\word[glow(2)]{glow}
\pron{golxV}
\gl{\nA}
\bmng
\bnum
\num{1} javxlana; javxlisuva sithxti; niginigisuva sithxti. 
\num{2} ujavxlate; parxjavxlate; raMguraMgina thaLathaLike; baNaNxgaLa ujavxlate, parxkAsha (\udA\ kenenxgaLa keMpu). 
\num{3} (shariVrada yA BAvada) ujavxlate; parxjavxlate; kAyupx; kaDurAga; BAvAveVsha. 
\enum
\emng

\noindent
\gl{\pagu}
\bmng
\bnum
\num{1} \eng{all of a glow} = \hyperlink{glow pagu2}{?pagu? \((2)\)}. 
\hypertarget{glow pagu2}{} 
\numi{2} \eng{in a glow} 
\banum
\alnum{a} kAveVri; parxjavxlisi. 
\alnum{b} keMpeVri; ujavxlavAgi. 
\eanum
\numie
\enum
\emng
\eentry

\bentry
\word{glow-discharge}
\pron{golxVDisfcAjfR}
\gl{\nA}
\bmng
 (\Bwvi) javxlana visajaRna; kiDigaLu utapxtitxyAgade shAMtavAgi beLaguva viduyxtitxna visajaRne. 
\emng
\eentry

\bentry
\word{glower}
\pron{golxVarf}
\gl{\akirx}
\bmng
\bnum
\num{1} (AshacxyaR \mo vugaLiMda) birugaNiNxMda noVDu; diTiTxsinoVDu. 
\num{2} (koVpa \mo vugaLiMda) durudurane noVDu; duruguTiTxkoMDu noVDu. 
\enum
\emng
\eentry

\bentry
\word{gloweringly}
\pron{golxVariMgfli}
\gl{\kirxvi}
\bmng
 durudurane; duruguTiTxkoMDu. 
\emng
\eentry

\bentry
\word{glowingly}
\pron{golxViMgfli}
\gl{\kirxvi}
\bmng
\bnum
\num{1} javxlisutatx; niginigisutatx. 
\num{2} ujavxlavAgi; ujavxla vaNaRdiMda -- thaLathaLisutatx, raMguraMgAgi; vaNaRraMjitavAgi. 
\num{3} BAvapUritavAgi; utAsxhaBaritavAgi. 
\enum
\emng
\eentry

\bentry
\word{glow-lamp}
\pron{golxVlAYxMpf}
\gl{\nA}
\bmng
 javxlanadiVpa; viduyxtitxniMda javxlisuva kAbaRnf yA inAnxvudeV kAyxthoVDf uLaLx diVpa. 
\emng
\eentry

\bentry
\word{glow-worm}
\pron{golxVvamfR}
\gl{\nA}
\bmng
 miNuku huLu; miMcu huLu; hiMdugaDe tudiyalilx beLaku sUsuva, rekekx ilalxda, jiVruMDe, lAvaR yA heNuNx jiVruMDe. 
\emng
\eentry

\bentry
\word{gloxinia}
\pron{gAlxkisxnia}
\gl{\nA}
\bmng
 gaMTe hUgiDa; GaMTApuSiTx: 
\banum
\alnum{a} gaMTeyAkArada vividha vaNaRgaLa doDaDx doDaDx hUvugaLanunx biDuva, amerikada uSaNxvalaya parxdeVshada sasayxkula. 
\alnum{b} I kulada giDa. 
\eanum
\emng
\eentry

\bentry
\word{gloze}
\pron{golxVsfZ}
\gl{\sakirx}
\bmng
\bnum
\num{1} satayxveninxsuvaMte mAtanADu; beNeNxyaMtha mAtanADu; ApAtaramaNiVyavAgi mAtanADu. 
\num{2} (tapupx, tapipxta, \mo vanunx) laGuveMdu toVrisu; teVlisi vivarisibiDu. 
\enum
\emng

\noindent
\gl{\akirx}
\bmng
 (\pArxparx) 
\bnum
\num{1} vAyxKAyxnisu; TiVke mADu. 
\num{2} satayxvenisuvaMte mAtanADu; beNeNxyaMtha mAtanADu; ApAtaramaNiVyavAgi mAtanADu. 
\num{3} icaCxkavADu; muKasutxtimADu; husiyAgi hogaLu. 
\enum
\emng

\noindent
\gl{\pagu}
\bmng
 \eng{gloze over} = \hyperlink{gloze}{gloze (?sakirx?).} 
\emng
\eentry

\bentry
\word{glozingly}
\pron{golxVsiZMgfli}
\gl{\kirxvi}
\bmng
 husiyAgi hogaLutatx; muKasutxti mADutatx. 
\emng
\eentry

\bentry
\word{glucagon}
\pron{gUlxkaganf}
\gl{\nA}
\bmng
 (\veYshA) gUlxkaganf; meVdoVjiVrakAMgadalilx utapxtitxyAguva oMdu pAlipepeTxYDf hAmoRVnu. 
\emng
\eentry

\bentry
\word{glucose}
\pron{gUlxkoVsf}
\gl{\nA}
\bmng
\bnum
\num{1} (\ravi) gUlxkoVsu; Aru AkisxjaniVkaqta kAbaRnf paramANugaLiruva oMdu AloDxVsf sakakxre, \kanmu\ dArxkiSx haNiNxnalilxruva sATxcfR \mo\ pAlisAyxkareYDugaLa jalaviBajaneyiMda tayArisuva, adara balamuri rUpa, \eng{$\bg\rm C\eg_6\bg\rm H\eg_\bg 12\eg\bg\rm O\eg_6$}. 
\num{2} gUlxkoVsu; sATxcfRna jalaviBajaneyiMda doreyuva gUlxkoVsf, mAloTxVsfgaLiruva pAkadaMtha padAthaR. 
\enum
\emng
\eentry

\bentry
\word{glucosic}
\pron{gUlxkAsikf}
\gl{\gu}
\bmng
 gUlxkoVsina yA adakekx saMbaMdhisida. 
\emng
\eentry

\bentry
\word{glocoside}
\pron{gUlxkaseYDf}
\gl{\nA}
\bmng
 (\ravi) gUlxkoseYDf; jalaviBajanege oLagAdAga gUlxkoVsf hAgU kAboRheYDerXVTalalxda inAnxvudeV saMyukatxvanunx niVDuva, yAvudeV neYsagiRka yA saMshelxVSita saMyukatx. 
\emng
\eentry

\bentry
\word{glucosidic}
\pron{gUlxkasiDikf}
\gl{\gu}
\bmng
 gUlxkoseYDf lakaSxNada. 
\emng
\eentry

\bentry
\word[glue(1)]{glue}
\pron{gUlx}
\gl{\nA}
\bmng
 gUlx; maravajarx; sari; camaR matutx mULegaLanunx niVrinoDane kudisi tayArisi, gaDusAgiyU BiduravAgiyU iruva, bisiyalilx aMTAgi upayoVgisuva, kaMdu baNaNxda jelaTinf. 
\bnum
\num{2} aMTu; goVMdu; itara mUlagaLiMda tayArisida iMtahadeV aMTisuva padAthaR: \eng{fish glue} mIninaMTu; mIniniMda tayArisida aMTu. 
\enum
\emng
\eentry

\bentry
\word[glue(2)]{glue}
\pron{gUlx}
\gl{\sakirx}
\bmng
\bnum
\num{1} maravajarxdiMda yA adariMdaloV eMbaMte -- aMTisu, kUDisu, baMdhisu. 
\num{2} balavAgi -- aMTisu, neDisu, kUDisu, hacicxsu: \eng{eyes glued to the picture} citarxda meVle kaNaNxnunx neTuTx. 
\enum
\emng
\eentry

\bentry
\word{glue-pot}
\pron{gUlxpATf}
\gl{\nA}
\bmng
\bnum
\num{1} maravajarxpAterx; maravajarxvanunx karagisalu baLasuva, horagaDe niVrina AvaraNa uLaLx pAterx. 
\num{2} jigaTu BUmi; aMTunela; teVvadiMdaloV maNiNxruvudariMdaloV aMTaMTAgiruva nelada BAga. 
\enum
\emng
\eentry

\bentry
\word{glue-sniffer}
\pron{gUlxsinxpharf}
\gl{\nA}
\bmng
 mAdaka vasutxvAgi pAlxsiTxkf simeMTina hogeyanunx seVduvava. 
\emng
\eentry

\bentry
\word{gluey}
\pron{gUlxi}
\gl{\gu}
\bmng
 (\tara\ \eng{gluier}, \tama\ \eng{gluiest}). 
\bnum
\num{1} aMTaMTAda; jigaTAda. 
\num{2} aMTinaMtiruva. 
\num{3} aMTu savarida. 
\enum
\emng
\eentry

\bentry
\word{glum}
\pron{galxmf}
\gl{\gu}
\bmng
\bnum
\num{1} bADida moVreya; mAlxna muKada; viSaNaNx muKada. 
\num{2} siDuku moVreya; siDuku musuDiya; uri moVreya. 
\num{3} ataqpatx muKada; samAdhAnavilalxda muKada. 
\enum
\emng
\eentry

\bentry
\word{glumaceous}
\pron{gUlxmeVSasf}
\gl{\gu}
\bmng
\bnum
\num{1} gUlxmf (\eng{glume})naMtha; gUlxmuLaLx. 
\num{2} hoTiTxnaMtha yA hoTiTxruva. 
\enum
\emng
\eentry

\bentry
\word{glume}
\pron{gUlxmf}
\gl{\nA}
\bmng
\bnum
\num{1} (\savi) gUlxmf; hulilxna baLagada matitxtara kelavu sasayxgaLalilx puSapxpAtarxda hoTiTxnaMtha daLagaLu. 
\num{2} (kALina) hoTuTx. 
\enum
\emng
\eentry

\bentry
\word{glumly}
\pron{galxmfli}
\gl{\kirxvi}
\bmng
\bnum
\num{1} bADida moVreyiMda; Kinanx, viSaNaNx muKa mADikoMDu. 
\num{2} siDuku moVreyiMda; urimusuDi hAkikoMDu. 
\num{3} ataqpatx muKadiMda; samAdhAnavilalxde. 
\enum
\emng
\eentry

\bentry
\word{glumness}
\pron{galxmfnisf}
\gl{\nA}
\bmng
\bnum
\num{1} bADida moVre hAkikoMDiruvudu; viSaNaNx muKadiMdiruvudu. 
\num{2} siDuku moVre mADikoMDiruvudu. 
\num{3} ataqpatx muKa; samAdhAnavilalxda muKaBAva. 
\enum
\emng
\eentry

\bentry
\word{glumose}
\pron{gUlxmoVsf}
\gl{\gu}
\bmng
\bnum
\num{1} gUlxmf (\eng{glume})gaLiruva. 
\num{2} hoTiTxruva. 
\enum
\emng
\eentry

\bentry
\word[glut(1)]{glut}
\pron{galxTf}
\gl{\kirx}
\expl{ (\BU\ matutx \BUkaq\ \eng{glutted,} \vakaq\ \eng{glutting}).}


\noindent
\gl{\sakirx}
\bmng
\bnum
\num{1} (vayxkitxge yA hoTeTxge) biriyuvaSuTx tininxsu; kaMThapUtiR tininxsu; hALata mIri tininxsu; kaTaTxreyAguvaSuTx tininxsu. 
\num{2} (hasivanunx, haMbalavanunx) pUtiR tiVrisu; mitimIri taNisu. 
\num{3} AhAravanunx mitimIri, parxmANa mIri tuMbu, turuku (\rUpa\ saha). 
\num{4} kaTaTxreyAgisu; vegaTAgisu; rucihoVgi Okarike baruvaSuTx -- tininxsu yA taNisu. 
\num{5} giDi; giDugu; kaTaTxDacu; tuMbi turuku; mitimIri tuMbu. 
\num{6} (mArukaTeTxyalilx) sarakugaLanunx mitimIri, atiyAgi sheVKarisu. 
\enum
\emng

\noindent
\gl{\akirx}
\bmng
 hoTeTx biriyuvaSuTx, kaMThapUtiR, hALata mIri yA kaTaTxreyAguvaSuTx tinunx: \eng{he glutted all night} kaTaTxreyAguvaSuTx rAtirxyelalx tinunxtatxleV idadx. 
\emng

\noindent
\gl{\pagu}
\bmng
 \eng{glut with} = \hyperlink{glut(1)}{glut (?sakirx?)}. 
\emng
\eentry

\bentry
\word[glut(2)]{glut}
\pron{galxTf}
\gl{\nA}
\bmng
\bnum
\num{1} BoVgAtireVka; mitimIrida BoVga. 
\num{2} kaMThapUtiR (parxmANa); manaHpUtiR; manadaNiyuvaSuTx. 
\num{3} mitimIrike; atireVka parxmANa. 
\num{4} ati pUreYke; mitimIrida sarabarAju; beVDikeyanunx mIrida pUreYke; janagaLa AvashayxkategiMta hecucx sarabarAju: \eng{a glut in the market} mArukaTeTxyalilx ati sarabarAju. 
\enum
\emng
\eentry

\bentry
\word{glutamate}
\pron{gUlxTameVTf}
\gl{\nA}
\bmng
 (\ravi) gUlxTameVTf; gUlxTAYxmikf Amalxda yAvudeV lavaNa yA esaTxru, \kanmu\ AhArakekx ruci kaTaTxlu baLasuva soVDiyaM lavaNa. 
\emng
\eentry

\bentry
\wordRemoveSpace{glutamic-acid}{glutamic acid}
\pron{gUlxTAAmikf AYxsiDf}
\gl{\nA}
\bmng
 (\ravi) gUlxTAyxmikf Amalx; aneVka porxVTiVnugaLa GaTakavAda, neYsagiRkavAgi doreyuva oMdu amInoV Amalx. 
\emng
\eentry

\bentry
\word{gluteal}
\pron{gUlxTialf}
\gl{\gu}
\bmng
 (\aMrashA) pirerx sAnxyugaLa; nitaMbasAnxyugaLa; gUlxTiyasfna; gUlxTiyasfge saMbaMdhisida; pirerxyalilxna sAnxyugaLa yA avakekx saMbaMdhisida. 
\emng
\eentry

\bentry
\word{gluten}
\pron{gUlxTanf}
\gl{\nA}
\bmng
\bnum
\num{1} (\pArxparx) aMTu; sari; goVMdu; jibubx; aMTupadAthaR. 
\num{2} pArxNi -- aMTu, jigaTu; pArxNigaLa UtakadiMda baruva aMTu. 
\num{3} gUlxTanf; hiTiTxna aMTu; goVdi hiTiTxniMda piSaTxvanunx beVpaRDisidare uLiyuva birusAda porxVTiVnu padAthaR. 
\enum
\emng
\eentry

\bentry
\wordnospeech{gluten bread}{gluten bread}
\pron{?}
\gl{\nA}
\bmng
 gUlxTanf berxDuDx; hiTiTxna aMTina aMsha hecAcxgiruva berxDuDx. 
\emng
\eentry

\bentry
\word{gluteus}
\pron{gUlxTiasf}
\gl{\nA}
\expl{(\bava\ \eng{glutei} \ucAcx\ gUlxTiai).}
\bmng
(\aMrashA) gUlxTiyasf; pirerx sAnxyu; nitaMbasAnxyu; pirerxyalilxna mUru sAnxyugaLalilx oMdu. 
\emng
\eentry

\bentry
\word{glutinize}
\pron{gUlxTineYsfZ}
\gl{\sakirx}
\bmng
 aMTumADu; aMTaMTumADu; jigaTAgisu. 
\emng
\eentry

\bentry
\word{glutinous}
\pron{gUlxTinasf}
\gl{\gu}
\bmng
\bnum
\num{1} aMTAda; jigaTAda; aMTaMTAda. 
\num{2} aMTinaMte, goVMdinaMte iruva. 
\enum
\emng
\eentry

\bentry
\word{glutinously}
\pron{gUlxTinasfli}
\gl{\kirxvi}
\bmng
 aMTAgi; aMTaMTAgi. 
\emng
\eentry

\bentry
\word{glutinousness}
\pron{gUlxTinasfnisf}
\gl{\nA}
\bmng
 aMTu; jigaTu; aMTAgiruvike. 
\emng
\eentry

\bentry
\word{glutton}
\pron{galxTanf}
\gl{\nA}
\bmng
\bnum
\num{1} hoTeTxbAka; tiMDipoVta; tiVnALi. 
\num{2} pusatxkapishAci; pusatxka vayxsani; pusatxkagaLanunx Oduvudaralilx tiVvarx AsheyuLaLxvanu. 
\num{3} kAyaRvayxsana; kelasada pishAci; kAyARnurakatx; kelasa mADuvudaralilx taqpitxyeV ilalxdaSuTx AseyuLaLxvanu. 
\hyperdef{G}{glutton(4)}{} 
\num{4} galxTanf; `viVsalf' vaMshakekx seVrida, `guloV guloV' kulada, hoTeTxbAka pArxNi. 
\enum
\emng
\eentry

\bentry
\word{gluttonise}
\pron{galxTaneYsfZ}
\gl{\akirx}
\bmng
  = \hyperlink{gluttonize}{gluttonize}. 
\emng
\eentry

\bentry
\word{gluttonize}
\pron{galxTaneYsfZ}
\gl{\akirx}
\bmng
 hoTeTxbAkanaMte tinunx, mukukx. 
\emng
\eentry

\bentry
\word{gluttonous}
\pron{galxTanasf}
\gl{\gu}
\bmng
\bnum
\num{1} hoTeTxbAkatanada; tiVnALitanada. 
\num{2} (yAvudeV \vi) ati AsakitxyuLaLx. 
\enum
\emng
\eentry

\bentry
\word{gluttonously}
\pron{galxTanasfli}
\gl{\kirxvi}
\bmng
\bnum
\num{1} hoTeTxbAkanaMte; hoTeTxbAkatanadiMda. 
\num{2} (yAvudeV \vi) atAyxsakitxyiMda. 
\enum
\emng
\eentry

\bentry
\word{gluttony}
\pron{galxTani}
\gl{\nA}
\bmng
 tiVnALike; tiMDipoVtatana; hoTeTxbAkatana; atiyAgi seVvisuvike. 
\emng
\eentry

\bentry
\word{glycerate}
\pron{gilxsareVTf}
\gl{\nA}
\bmng
 (\ravi) gilxsareVTf; gilxsarikf Amalxda lavaNa yA esaTxru. 
\emng
\eentry

\bentry
\word{glyceride}
\pron{gilxsareYDf}
\gl{\nA}
\bmng
 (\ravi) gilxsareYDu; neYsagiRka eNeNx matutx kobubxgaLalilxruva, gilxsarAlina meVdAmalxgaLa esaTxru. 
\emng
\eentry

\bentry
\word{glycerin}
\pron{gilxsarinf}
\gl{\nA}
\bmng
 (\ame)  = \hyperlink{glycerine}{glycerine}. 
\emng
\eentry

\bentry
\word{glycerinate}
\pron{gilxsarineVTf}
\gl{\sakirx}
\bmng
 gilxsariniVkarisu; gilxsarininxniMda saMsakxrisu. 
\emng
\eentry

\bentry
\word{glycerine}
\pron{gilxsariVnf}
\gl{\nA}
\bmng
 (\ravi) gilxsariVnu; neYsagiRka eNeNx matutx kobubxgaLalilx esaTxru rUpadalilxruva, sAbUnu yA meVdAmalxgaLa tayArikeyalilx upoVtapxnanxvAgi baruva, auSadha sAmagirxgaLu, kAMtivadhaRkagaLu, \mo vugaLa tayArikeyalilx upayoVgakekx baruva, baNaNxvilalxda, sihiyAda maMdadarxvavAgiruva oMdu tirxheYDirxka AlokxhAlu, \eng{$\bg\rm CH\eg_2\bg\rm OH\eg-\bg\rm CHOH\eg-\bg\rm CH\eg_2\bg\rm OH\eg$}. 
\emng
\eentry

\bentry
\word{glycerol}
\pron{gilxsarAlf}
\gl{\nA}
\bmng
 (\ravi)  = \hyperlink{glycerine}{glycerine}. 
\emng
\eentry

\bentry
\word{glyceryl}
\pron{gilxsarilf}
\gl{\nA}
\bmng
 (\ravi) gilxsarilf; gilxsarAlinalilxruva mUru heYDArxkisxlf guMpugaLanUnx tegeyuvudariMda baruva, kelavu veVLe oMdu heYDArxkisxlanonxV eraDu heYDArxkisxlanonxV tegedare baruva rAyxDikalu. 
\emng
\eentry

\bentry
\wordnospeech{glyceryl trinitrate}{glyceryl trinitrate}
\pron{?}
\gl{\nA}
\bmng
 (\ravi) neYTirxkf matutx salUphxrikf AmalxgaLige gilxsarAlf hAki tayArisuva, DeYnameYTf \mo\ soPxVTaka vasutxgaLa tayArikeyalilx baLasuva, nasu haLadi darxva. 
\emng
\eentry

\bentry
\word{glycine}
\pron{gelxYsiVnf}
\gl{\nA}
\bmng
 (\ravi) gelxYsiVnf; porxVTiVnugaLa jalaviBajaneyiMda doreyuva, ameYno AmalxgaLalilx atayxMta saraLavAda, ameYno asiTikf Amalx. 
\emng
\eentry

\bentry
\word{glyco-}
\pron{gelxYka(koV)-}
\gl{\sapUpa}
\bmng
 (\ravi) sakakxre, shakaRra, gilxsarAlf, gelxYkoVjanf, gelxYkAlf, gelxYsiVnf eMbathaRgaLalilx baLasuva \sapUpa. 
\emng
\eentry

\bentry
\word{glycocoll}
\pron{gelxYkakAlf}
\gl{\nA}
\bmng
  = \hyperlink{glycine}{glycine}. 
\emng
\eentry

\bentry
\word{glycogen}
\pron{gelxYkajanf}
\gl{\nA}
\bmng
(\ravi) gelxYkojanf; sasayxgaLalilx kAboRVheYDerxVTu sATxcfR rUpadalilx sheVKaragoLuLxvaMteyeV pArxNigaLalilx sheVKaragoLuLxva oMdu pAlisAyxkareYDu. 
\emng
\eentry

\bentry
\word{glycogenesis}
\pron{gelxYkajenisisf}
\gl{\nA}
\bmng
 gelxYkojanana: 
\banum
\alnum{a} yakaqtitxnalilx AguvaMte gelxkojaninxniMda gUlxkoVsf utapxtitxyAguvudu. 
\alnum{b} pArxNi deVhadalilx Aguva gelxYkojanf utApxdane. 
\eanum
\emng
\eentry

\bentry
\word{glycogenic}
\pron{gelxYkajanikf}
\gl{\gu}
\bmng
 gelxYkojaninxna yA adakekx saMbaMdhisida. 
\emng
\eentry

\bentry
\word{glycol}
\pron{gelxYkAlf}
\gl{\nA}
\bmng
 (\ravi) gelxYkAlf: 
\banum
\alnum{a} eraDu heYDArxkisxlf guMpugaLiruva yAvudeV AlokxhAlu. 
\alnum{b} aMtaha alokxhAlugaLalilx atayxMta saraLavAda etiliVnf gelxYkAlf, \eng{$\bg\rm CH\eg_2\bg\rm OH\eg-\bg\rm CH\eg_2\bg\rm OH\eg$}. 
\eanum
\emng
\eentry

\bentry
\word{glycolic}
\pron{gelxYkAlikf}
\gl{\gu}
\bmng
 (\ravi) gelxYkAlikf; (AliDxheYDu matutx AmalxgaLa \vi) gelxYkAlina oMdu heYDArxkisxlf guMpanunx utakxSiRsuvudariMda baruva: \eng{glycolic acid} gelxYkAlikf Amalx. 
\emng
\eentry

\bentry
\word{glycollic}
\pron{gelxYkAlikf}
\gl{\gu}
\bmng
  = \hyperlink{glycolic}{glycolic}. 
\emng
\eentry

\bentry
\word{glycolysis}
\pron{gelxYkAlisisf}
\gl{\nA}
\expl{(\bava\ \eng{glycolyses} \ucAcx\ gelxYkAlisiVsfZ).}
\bmng
(\jiVra) gelxYkAlisisf; shakaRra viBajane; jiVvigaLa deVhadalilx Aguva gUlxkoVsf, gelxYkojanf, \mo vugaLa viBajane. 
\emng
\eentry

\bentry
\word{glycoprotein}
\pron{gelxYka(koV)porxVTiVnf}
\gl{\nA}
\bmng
 (\ravi) gelxYkoVporxVTiVnf; jeYvika vayxvasethxgaLalilx kaMDubaruva, kAboRheYDerxVTugaLoMdige saMyoVgagoMDiruva saMkiVNaR porxVTiVnu. 
\emng
\eentry

\bentry
\word{glycoside}
\pron{gelxYkaseYDf}
\gl{\nA}
\bmng
 (\ravi) gelxYkoseYDf; jeYvika vayxvasethxgaLalilx kaMDubaruva kAboRheYDerxVTf matutx heYDArxkisx saMyukatxgaLa saMyoVgadiMda uMTAda saMkiVNaR saMyukatx. 
\emng
\eentry

\bentry
\word{glycosidic}
\pron{gelxYkasiDikf}
\gl{\gu}
\bmng
\bnum
\num{1} gelxYkoseYDf savxrUpada, savxBAvada. 
\num{2} gelxYkoseYDfge saMbaMdhisida. 
\enum
\emng
\eentry

\bentry
\word{glycosuria}
\pron{gelxYkasuyxaria}
\gl{\nA}
\bmng
 (\roVshA) gelxYkoVsUriya; sakakxre mUtarx; madhumUtarx; shakaRramUtarx; mUtarxdalilx sakakxre iruva roVgasithxti. 
\emng
\eentry

\bentry
\word{glycosuric}
\pron{gelxYkasuyxarikf}
\gl{\gu}
\bmng
\bnum
\num{1} gelxYkosUriyadaMtha; sakakxre mUtarxda. 
\num{2} gelxYkasUriyaviruva; sakakxremUtarxviruva. 
\enum
\emng
\eentry

\bentry
\word{glyph}
\pron{gilxphf}
\gl{\nA}
\bmng
\bnum
\num{1} (\vAshi) AlaMkArika -- gADi, toVDu. 
\num{2} (\vAshi) ketitxda ububx citarx; utikxVNaRcitarx. 
\enum
\emng
\eentry

\bentry
\word{glyphic}
\pron{gilxphikf}
\gl{\gu}
\bmng
 (\vAshi) ketitxda ububxcitarxda; utikxVNaR citarxda. 
\emng
\eentry

\bentry
\word[glyphograph(1)]{glyphograph}
\pron{gilxphagArxphf}
\gl{\nA}
\bmng
 gilxphogArxphf; utikxVNaRdacucx; utikxVNaRTeYpu; loVhada tagaDina meVle meVNavanunx baLidu, adara meVle akaSxragaLanunx koredu, A koretadalilx loVhavanunx viduyxlelxVpana mADi tayArisida viduyxdacucx yA elekoTxrX TeYpu. 
\emng
\eentry

\bentry
\word[glyphograph(2)]{glyphograph}
\pron{gilxphagArxphf}
\gl{\sakirx}
\bmng
 (yAvudanenxV) gilxphogArxphf vidhAnadiMda, utikxVNaRdacicxniMda -- mudirxsu. 
\emng
\eentry

\bentry
\word{glyphographer}
\pron{gilxphAgarxpharf}
\gl{\nA}
\bmng
 gilxphogArxphf mudarxka; utikxVNaR(dacucx) mudarxka. 
\emng
\eentry

\bentry
\word{glyphographic}
\pron{gilxphagArxYxphikf}
\gl{\gu}
\bmng
 gilxphogArxphikf; gilxphogArxphf mudarxNakekx, utikxVNaRdacicxge saMbaMdhisida yA A vidhAnadiMda mudirxsida. 
\emng
\eentry

\bentry
\word{glyphography}
\pron{gilxphAgarxphi}
\gl{\nA}
\bmng
 gilxphogArxphf vidhAna; utikxVNaR(dacucx) mudarxNa. 
\emng
\eentry

\bentry
\word{glyptal}
\pron{gilxpATxYxlf}
\gl{\nA}
\bmng
 (\ravi) gilxpATxlf; gilxsarAlf matutx thAyxlikf Amalx athavA A Amalxda anfheYDerxYDfgaLa saMyoVgadiMda doreyuva AlikxDf vagaRda saMshelxVSita rALa, pAlxsiTxkukx. 
\emng
\eentry

\bentry
\word{glyptic}
\pron{gilxpiTxkf}
\gl{\gu}
\bmng
 (\kanmu\ ratanxgaLa meVle) ketatxneya; koreyuva; utikxVNaR mADuva. 
\emng
\eentry

\bentry
\word{glyptodont}
\pron{gilxpaTxDAnfTx}
\gl{\nA}
\bmng
 gilxpaTxDAnfTx; AmaRDiloV eMba pArxNiya jAtige seVrida, dakiSxNa amerikada naSaTxvaMshada catuSApxdi. \imglink{glyptodontfigure}{\raisebox{-0.15cm}[0pt][0pt]{\pdfimage width 0.7cm height 0.5cm{G_Pictures/glyptodont.jpg}}} 
\emng
\eentry

\bentry
\word{glyptography}
\pron{gilxpATxgarxphi}
\gl{\nA}
\bmng
utikxVNaRkale; ratanx koreyuva kale matutx shAsatxrX. 
\emng
\eentry

\bentry
\wordnospeech{GM}{GM}
\pron{?}
\gl{\saMkiSx}
\bmng
\bnum
\num{1} (\birx) \eng{George Medal.} 
\num{2} (amerikada saMyukatx saMsAthxna) \eng{General Motors.} 
\num{3} \eng{general manager.} 
\enum
\emng
\eentry

\bentry
\wordnospeech{gm.}{gm.}
\pron{?}
\gl{\saMkiSx}
\bmng
\eng{gram(s).} 
\emng
\eentry

\bentry
\word{G-man}
\pron{jiVmAYxnf}
\gl{\nA}
\expl{(\bava\ \eng{G-men}).}
\bmng
\bnum
\num{1} (\ashi) amerikada keVMdarx sakARrada patetxVdAri adhikAri; sakARri patetxVdAra. 
\num{2} (airiSf BASe) rAjakiVya patetxVdAra. 
\enum
\emng
\eentry

\bentry
\wordnospeech{GMT}{GMT}
\pron{?}
\gl{\saMkiSx}
\bmng
\eng{Greenwich mean time.} 
\emng
\eentry

\bentry
\wordnospeech{GMWU}{GMWU}
\pron{?}
\gl{\saMkiSx}
\bmng
(\birx) \eng{General \& Municipal Workers' Union.} 
\emng
\eentry

\bentry
\word{gnarled}
\pron{nAlfDxR}
\gl{\gu}
\bmng
 (mara, muKa, avayavagaLu, \mo vugaLa \vi) gaMTugaMTAda; tiricumuricAda; oraToraTAda. 
\emng
\eentry

\bentry
\word{gnarly}
\pron{nAliR}
\gl{\gu}
\bmng
  = \hyperlink{gnarled}{gnarled}. 
\emng
\eentry

\bentry
\word{gnash}
\pron{nAYxSf}
\gl{\sakirx}
\bmng
halulx -- kaDi, mase, kaTakaTisu. 
\emng

\noindent
\gl{\akirx}
\bmng
(halilxna \vi) kaDi; mase; kaTakaTa mADu. 
\emng
\eentry

\bentry
\word{gnat}
\pron{nAYxTf}
\gl{\nA}
\bmng
\bnum
\num{1} guMgare; guMgADu; doVme; heNuNx jAti mAtarx rakatx hiVruva, kUyxlakfsx kulada, eraDu rekekxya, cikakx kiVTa. 
\num{2} (\ame) soLeLx. 
\num{3} (\kanmu) (kuSxlalxka) kirukuLa; kATa. 
\num{4} puTaTx vasutx; saNaNxvasutx. 
\enum
\emng

\noindent
\gl{\nuga}
\bmng
\bnum
\num{1} \eng{strain at a gnat} alapx viSayagaLige ati gamana koDu; kuSxdarx viSayagaLalilx ecacxravahisu. 
\num{2} \eng{strain at a gnat and swallow a camel} saNaNxpuTaTx viSayagaLalilx cwkAsi mADi muKayx viSayagaLalilx sulaBavAgi samamxti koDu. 
\enum
\emng
\eentry

\bentry
\word{gnathic}
\pron{nAYxtikf}
\gl{\gu}
\bmng
 davaDeya; tAluvina; davaDegaLa. 
\emng
\eentry

\bentry
\word{gnaw}
\pron{nA}
\gl{\kirx}
\expl{(\BUkaq\ \eng{gnawed} yA \eng{gnawn}).}


\noindent
\gl{\sakirx}
\bmng
\bnum
\num{1} biDade kaDi; koraku; oMdeV samane kacucx, agi. 
\num{2} (kaDidu) saveyisu; samesu: \eng{the dog was gnawing a bone} nAyiyu elubanunx kaDiyutitxtutx. 
\numi{3} (nAshakAri vasutx, noVvu, \mo vugaLa \vi) 
\banum
\alnum{a} tukukx hiDisu; saveyisu; kaSxyisu; tiMduhAku; jiVNaR mADu: \eng{fear and anxiety gnawing the heart} Baya matutx ciMtegaLu haqdayavanunx tiMduhAkutAtx. 
\alnum{b} hiMse koDu; yAtanepaDisu; kADu; piVDisu: \eng{her brain was gnawed by savage thoughts} avaLa tale duSaTx AloVcanegaLiMda yAtanegoMDitu. 
\eanum
\numie
\enum
\emng

\noindent
\gl{\akirx}
\bmng
 kaDi; kacucx: \eng{he gnawed at his underlip} avanu tananx keLatuTiyanunx kacicxdanu. 
\emng
\eentry

\bentry
\word{gnawing}
\pron{nAiMgf}
\gl{\nA}
\bmng
\bnum
\num{1} (\kanmu\ hasiviniMda hoTeTxyalilx yA karuLinalilx eDebiDade Aguva) saMkaTa; bAdhe; hiMDuvaMtha noVvu. 
\num{2} agita; biDade kaDiyutitxruvudu. 
\enum
\emng
\eentry

\bentry
\word{gnawingly}
\pron{nAiMgfli}
\gl{\kirxvi}
\bmng
(\kanmu\ hasiviniMda hoTeTxyalilx yA karuLinalilx) saMkaTa, bAdhe yA noVvu -- uMTu mADuva riVtiyalilx. 
\emng
\eentry

\bentry
\word{gneiss}
\pron{genxY(neY)sf}
\gl{\nA}
\bmng
(\BUvi) neYsf; kAvxTfsxR, phelfsApxrf matutx aBarxkagaLiruva oMdu bageya padara padaravAda rUpAMtarita shile. 
\emng
\eentry

\bentry
\word{gneissic}
\pron{genxY(neY)sikf}
\gl{\gu}
\bmng
(\BUvi) neYsf savxBAvada, lakaSxNada. 
\emng
\eentry

\bentry
\word{gneissoid}
\pron{genx(neY)sAyfDx}
\gl{\gu}
\bmng
 (\BUvi) neYsfnaMtha. 
\emng
\eentry

\bentry
\word{gneissose}
\pron{genxY(neY)soVsf}
\gl{\gu}
\bmng
(\BUvi) neYsf uLaLx. 
\emng
\eentry

\bentry
\word{gneissy}
\pron{genxY(neY)si}
\gl{\gu}
\bmng
neYsf uLaLx; neYsf guNagaLiruva. 
\emng
\eentry

\bentry
\word{gnocchi}
\pron{nA(nAyx)ki}
\gl{\nA}
\bmng
(\bava) nAki; hiTuTx, goVdhiya tavuDu, AlUgeDeDx, \mo vugaLanunx beVyisi tayArisida kaNakada KAdayx. 
\emng
\eentry

\bentry
\word[gnome(1)]{gnome}
\pron{noVmI, noVmf}
\gl{\nA}
\bmng
sUkitx; sUtarx; sAroVkitx; nANuNxDi; gAde. 
\emng
\eentry

\bentry
\word[gnome(2)]{gnome}
\pron{noVmf}
\gl{\nA}
\bmng
\bnum
\num{1} BUnikeSxVpagaLanunx kAdukoMDiruva adhoVloVkadalilx vAsisuva kubajx pishAci, gujAjxri devavx. 
\num{2} kuLaLx pishAci; gujAjxri devavx. 
\num{3} kuLaLx; kubajx; gujAjxri. 
\num{4} (\kanmu\ toVTadalilx iTiTxruva) pishAciya parxtime yA citarx. 
\numi{5} (\AmA) (\kanmu\ \bava dalilx) pishAci: 
\banum
\alnum{a} (\kanmu\ aMtararASiTxrXVya haNakAsina vayxvahAradalilx) duSaTx parxBAva biVruva vayxkitx. 
\alnum{b} (sivxDasxleRMDina) aMtararASiTxrXVya baMDavALagAra yA bAyxMkaru: \eng{the gnomes of Zurich} (sivxDasxleRMDina) sUZyxrikf (nagarada) baMDavALagAra. 
\eanum
\numie
\enum
\emng
\eentry

\bentry
\word{gnomic}
\pron{noVmikf}
\gl{\gu}
\bmng
\bnum
\num{1} sUtarxda; sUtarxpArxyavAda; sUtArxtamxka; sUkitxrUpada; sUkitxmaya; sAroVkitx baLasuva; sUkitxgaLanunx oLagoMDa. 
\num{2} (\vAyx) (kAlada \vi) anishicxta; anidhARrita; kAlada sUcane ilalxde sAmAnayx satayxvanunx vayxkatxpaDisalu upayoVgisuva. \udA\ \eng{men were deceivers ever} manuSayxru yAvAgalU moVsagArareVye. 
\enum
\emng
\eentry

\bentry
\word{gnomically}
\pron{noVmikali}
\gl{\kirxvi}
\bmng
\bnum
\num{1} sUtarxpArxyavAgi; sUtArxtamxka riVtiyalilx. 
\num{2} (\vAyx) (kAlada \vi) anishicxtavAgi; anidhARritavAgi. 
\enum
\emng
\eentry

\bentry
\word{gnomish}
\pron{noVmiSf}
\gl{\gu}
\bmng
 gujAjxri devavxdaMtaha; kuLaLx BUtadaMtiruva; kubajx pishAciyanunx hoVluva. 
\emng
\eentry

\bentry
\word{gnomon}
\pron{noVmanf}
\gl{\nA}
\bmng
 noVmanf: 
\banum
\alnum{a} (gurutu mADiruva meVlemxY meVle bidadx tananx neraLiniMda kAlavanunx toVrisuva) neraLu gaDiyArada -- kaMba, saraLu, shaMku, sUci, kiVlaka, Palaka. 
\alnum{b} sUyaRna madhAyxhanxreVKeya unanxtiyanunx nidhaRrisalu hiMde baLasutitxdadx, kiSxtijiVyakekx laMbavAgi neTaTx satxMBa \mo vu. 
\alnum{c} (\jAyx) samAMtara catuBuRjada oMdu mUleyalilx adara samarUpa catuBuRjavoMdanunx tegedare uLiyuva Akaqti. \imglink{atod-21figure}{\raisebox{-0.15cm}[0pt][0pt]{\pdfimage width 0.7cm height 0.5cm {G_Pictures/atod-21.jpg}}} 
\eanum
\emng
\eentry

\bentry
\word{gnomonic}
\pron{noVmAnikf}
\gl{\gu}
\bmng
\bnum
\num{1} neraLu gaDiyArada kaMbakekx, sUcige, yA Palakakekx saMbaMdhisida. 
\num{2} (kAla heVLuvudaralilx) neraLu gaDiyArada kaMba, sUci, Palaka baLasuva. 
\num{3} neraLu gaDiyAradaMtiruva. 
\num{4}  = \hyperlink{gnomic}{gnomic}. 
\enum
\emng
\eentry

\bentry
\word{gnosis}
\pron{noVsisf}
\gl{\nA}
\expl{(\bava\ \eng{gnoses} \ucAcx\ noVsiVsfZ).}
\bmng
\bnum
\num{1} adhAyxtamx rahasayx jAcnxna; adhAyxtamx rahasayx videyx. 
\num{2}  = \hyperlink{gnosticism}{gnosticism}. 
\enum
\emng
\eentry

\bentry
\word[gnostic(1)]{gnostic}
\pron{nAsiTxkf}
\gl{\gu}
\bmng
\bnum
\num{1} jAcnxnaviSayaka; jAcnxnasaMbaMdhada; jAcnxnada; arivina; arivige saMbaMdhapaTaTx. 
\num{2} adhAyxtamx rahasayxjAcnxnavuLaLx. 
\numi{3} (\eng{Gnostic}) 
\banum
\alnum{a} (kerxYsatx) nAsiTxkf paMgaDada yA paMgaDadavara. 
\alnum{b} atiVMdirxya; AdhAyxtimxka; nigUDha; gahana. 
\eanum
\numie
\enum
\emng
\eentry

\bentry
\word[gnostic(2)]{gnostic}
\pron{nAsiTxkf}
\gl{\nA}
\bmng
 (\eng{Gnostic}) (\sA\ \bava dalilx) nAsiTxkf paMthi; adhAyxtamx rahasayx jAcnxnavuLaLxvareMdu heVLikoLuLxtitxdadx, \kirxsha \eng{1--3}neV shatamAnada Adi kerxYsatx pASaMDi. 
\emng
\eentry

\bentry
\word{gnosticism}
\pron{nAsiTxsisaZmf}
\gl{\nA}
\bmng
 nAsiTxkfvAda; nAsiTxkf tatatxvX; (kelavu AdikerxYsatx pASaMDigaLu parxtipAdisida) adhAyxtamx rahasayxjAcnxnavAda. 
\emng
\eentry

\bentry
\word{gnosticize}
\pron{nAsiTxseYsfZ}
\gl{\sakirx}
\bmng
 nAsiTxkf savxrUpa koDu; nAsiTxkf tatatxvXgaLanunx AdhAravAgiTuTxkoMDu vAyxKAyxnisu. 
\emng

\noindent
\gl{\akirx}
\bmng
 nAsiTxkf tatatxvXgaLanunx -- aMgiVkarisu, nirUpisu, parxtipAdisu. 
\emng
\eentry

\bentry
\wordnospeech{GNP}{GNP}
\pron{?}
\gl{\saMkiSx}
\bmng
 \eng{gross national product.} 
\emng
\eentry

\bentry
\wordnospeech{Gnr.}{Gnr.}
\pron{?}
\gl{\saMkiSx}
\bmng
 (\birx) \eng{Gunner.} 
\emng
\eentry

\bentry
\wordnospeech{gns.}{gns.}
\pron{?}
\gl{\saMkiSx}
\bmng
 (\birx) (\ca) \eng{guineas.} 
\emng
\eentry

\bentry
\word{gnu}
\pron{nU, nUyx}
\gl{\nA}
\bmng
 nU; kAnakeVTiVsf kulada, etatxnunx hoVluva jiMke, cigare.  \imglink{gnufigure}{\raisebox{-0.15cm}[0pt][0pt]{\pdfimage width 0.8cm height 0.6cm {G_Pictures/gnu.jpg}}} 
\emng
\eentry

\bentry
\word[go(1)]{go}
\pron{goV}
\gl{\kirx}
\bmng
(\BU\ \eng{went}, \BUkaq\ \eng{gone}, \vakaq\ \eng{going}, vataRmAnakAla madhayxmapuruSa \Eva (\pArxparx) \eng{goest}, \ucAcx\ goVisfTx, parxthamapuruSa\Eva \eng{goes}, \ucAcx\ goVsfZ, (\pArxparx) \eng{goeth}, \ucAcx\ goVitf).
\emng

\noindent
\gl{\akirx}
\bmng
\bnum
\num{1} (oMdu gotutxpaDisida yA nidiRSaTx sathxLa, sAthxna, kAla, \mo vugaLiMda) hoVgu; horaDu; teraLu; naDe; calisu; sari; jarugu; kadalu; calisutitxru; poramaDu. 
\hyperdef{G}{go akirx2}{} 
\num{2} hoVgu; hoVgiseVru; parxyANamADu; payaNisu: \eng{go easy} sulaBavAgi hoVgu. \eng{go by air} vimAnadalilx parxyANamADu. \eng{went miles round} meYligaTaTxle sutAtxDida. \eng{go on a journey} parxyANa hoVgu. \eng{go (for) a walk} tirugADalu hoVgu. 
\num{3} muMduvari; munanxDe; muMdakekx hoVgu, sAgu: \eng{go the same way} adeV dAriyalilx muMduvari. 
\num{4} beLe; vadhiRsu; aBivaqdidhx hoMdu; parxgati paDe. 
\num{5} (reVKe \mo vugaLa \vi) nidiRSaTx dikikxnalilxru; nidiRSaTx dikikxge aBimuKavAgiru; nidiRSaTx dikikxnalilx hoVgu: \eng{the boundary goes parallel with the river} gaDireVKe nadige samAMtaravAgi hoVgutatxde. 
\num{6} (oMdara) parxkAra -- naDe, vatiRsu; (oMdanunx) avalaMbisu; anusarisu; (oMdakekx) hoMdikoMDiru; (oMdakekx) anuguNavAgi -- vatiRsu yA tiVmARnisu: \eng{a good rule to go by} AcaraNege, avalaMbisalu -- oLeLxya niyama, sUtarx. \eng{always goes with his party} avanu yAvAgalU tananx pakaSxvanenxV anusarisutAtxne. \eng{promotion goes by favour} baDitx (sikukxvudu) kaqpeyanunx avalaMbisutatxde. 
\num{7} (vADikeganusAravAgi yA savxlapx samaya) oMdu nidiRSaTx sithxtiyalilxru: \eng{go hungry} hasidukoMDiru. \eng{go in rags} ciMdibaTeTxyalilxru. \eng{go in fear of one's life} jiVvaBayadalilxru. \eng{plea went unheeded} koVrikeyanunx nilaRkiSxsalAyitu; koVrike lakiSxtavAgade hoVyitu. 
\num{8} calisutatx, kelasa mADutatx, Agutatx -- iru: \eng{the clock does not go well} gaDiyAra sariyAgi naDeyutitxlalx. 
\num{9} (nideRVshita riVtiyalilx) naDe; calisu: \eng{go like this with your left foot} ninanx eDahejejxyanunx hiVge iTuTx naDe. 
\num{10} (gaMTe, gaMTe hoDeyuva gaDiyAra, baMdUku, tAsu yA shabadxkoDuva yAvudeV mUlada \vi) bArisu; baDi; hoDe; dhavxnisu; shabadx horaDisu. 
\num{11} nidiRSaTx bageya shabadx mADu: \eng{go bang} DhaM enunx; DhaM eMba shabadx mADu. 
\num{12} (kAlada \vi) kaLe; sari; kaLedu hoVgu; gatisu: \eng{one week is already gone} AgaleV oMdu vAra kaLeduhoVgide. \eng{ten days to go before elections} cunAvaNegaLige inunx hatutx divasa ide. 
\num{13} (dUra \mo vugaLa \vi) hoVgabeVkAgiru; sAgabeVkAgiru; parxyANa mADabeVkAgiru: \eng{ten miles to go} hatutx meYli sAgabeVkAgide, parxyANa mADabeVkAgide. 
\num{14} calAvaNeyalilxru; cAlitxyalilxru; naDe; hoVgu: \eng{the sovereign went any where} savaranf nANayxvu elilx beVkAdarU naDeyutitxtutx. \eng{the story goes} katheyeVneMdare; kathe parxcalitavAgiruvudu hiVge. 
\num{15} (hesaru \mo vanunx) hotitxru; paDediru; hoMdiru; hesariniMda kareyalapxDu; parxcalitavAgiru; parxsidadhxvAgiru: \eng{go by a false name} suLuLx hesariTuTxkoMDiru. 
\num{16} sAdhAraNa maTaTxdAdxgiru; sAmAnayxvAda oMdu maTaTxdAdxgiru: \eng{he is a good actor as actors go nowadays} Igina kAlada naTara maTaTxdiMda heVLuvudAdare avanobabx oLeLxya naTaneMdeV heVLabeVku. 
\num{17} (dAKale, kavana, rAga, \mo vugaLa \vi) hoVgu; nidiRSaTx riVtiyalilxru; nidiRSaTx viSaya yA padagaLanunx oLagoMDiru. 
\num{18} (kavana, hADugaLa \vi) tALabadadhxvAgiru; layabadadhxvAgiru; CaMdoVbadadhxvAgiru; rAgakekx aLavaDisalu yA hADalu -- takukxdAgiru: \eng{the verses go easily enough} padayxgaLu sulaBavAgi hADuvaMtive. 
\num{19} (GaTanegaLu \mo vugaLa \vi) Agu; naDe; jarugu; uMTAgu; saMBavisu: \eng{go well} cenAnxgi Agu. \eng{go ill} keTuTxhoVgu. \eng{go hard} kaSaTxvAgu. 
\num{20} (tiVmARna, cunAvaNA PalitAMsha, \mo vugaLu) Agu; hoVgu; pariNamisu: \eng{go for} paravAgi, anukUlavAgi -- Agu. \eng{go against} virudadhxvAgi -- Agu, hoVgu. 
\num{21} (cunAvaNA keSxVtarx, rAjakAriNi, matadArara \vi) nidiRSaTx dAri yA aBipArxya -- hiDi, anusarisu: \eng{Bangalore went Congress} beMgaLUru paTaTxNavu kAMgerxsf pakaSxvanunx hiDiyitu. \eng{America went dry} amerikavu pAna niroVdhavanunx anusarisitu. 
\num{22} (samAraMBa \mo vugaLa \vi) cenAnxgi naDe; yashasivxyAgu: \eng{the dinner went well} BoVjanakUTa cenAnxgi naDeyitu. 
\num{23} (\AmA) naDe; aMgiVkaqtavAgiru; anumati iru: \eng{anything goes} yAvudAdarU sari, naDeyutatxde. 
\num{24} samamxtavAgiru; opipxgeyAgiru: \eng{what he says goes} avanu heVLidudx samamxtavAgutatxde. 
\num{25} (paMdayx \mo vugaLalilx paMdAyxLugaLige, horaDuva saMjecnx koDuvAga) horaDi; ODi; ODalu pArxraMBisi. 
\num{26} (daMDaneyilalxde) tapipxsiko; (shikeSxyiMda) pArAgu. 
\num{27} hoVgu; mArATavAgu: \eng{go cheap} agagxda belege hoVgu, mArATavAgu. \eng{go for one rupee} oMdu rUpAyige hoVgu. 
\num{28} (haNada \vi) KacARgu; viniyoVgavAgu: \eng{whatever money he got, it all went in books} avanige baMda haNavelAlx pusatxkagaLige Ayitu, KacARyitu. 
\num{29} hoVgu; tayxkatxvAgu; keYbiTuTx hoVgu: \eng{the car must go} kAranunx tayxjisabeVku. 
\num{30} hoVgu; naSaTxvAgu; ilalxvAgu; aMtayxvAgu; konemuTuTx: \eng{my sight is going} nananx daqSiTx hoVgutitxde. \eng{our trade is going} namamx vAyxpAra aMtayxvAgutitxde. 
\num{31} radAdxgu; vajA Agu; radAdxgihoVgu: \eng{the peon must go} javAna vajA AgabeVku; javAnananunx vajA mADabeVku. 
\num{32} hoVgu; naSaTxvAgu; kaLeduhoVgu: \eng{my headache has gone} nananx talenoVvu hoVyitu. \eng{the next wicket went for nothing} oMdu ranUnx seVrisade muMdina vikeTuTx bidudxhoVyitu. 
\num{33} (\kanmu\ \BUkaq\ hAgU aneVka \pagu gaLalilx \parx) sAyu; satutxhoVgu: \eng{go the way of all the earth} (\engit{or} \eng{of all flesh)} matayxRda dAri tuLi; sAyi. \eng{go to a better world} utatxma loVkakekx hoVgu; sAyi. \eng{go to one's account or reward} tananx lekakx tiVrisalu yA purasAkxra paDeyalu hoVgu. sAyu. 
\num{34} bidudxhoVgu; muridubiVLu: \eng{the mast went in three places} kUvekaMba mUru kaDe muriyitu. 
\num{35} siVLi hoVgu; birukubiDu. 
\num{36} haridu kitutxhoVgu. 
\num{37} hoVgu; oMdaratatx horaDu: \eng{which road goes to Bangalore?} beMgaLUrige yAva dAri hoVgutatxde? 
\num{38} (yAvudAdarU udedxVshakAkxgi) horaDu; hoVgu; mADalu pArxraMBisu: \eng{I went to find him} avananunx kANalu hoVde. \eng{go on a pilgrimage} yAterx(gAgi) horaDu, hoVgu. \eng{go on an errand} kelasada meVle, kelasakAkxgi hoVgu. \eng{go on a spree} majA mADalu hoVgu. 
\num{39} (\kanmu\ \AmA) (mADuvaSuTx) daDaDxnAgiru; mUKaRnAgiru; aviveVkiyAgiru: \eng{don't go making him angry} avanige koVpa barisuvaSuTx mUKaRnAgabeVDa. 
\num{40} (\AmA\ yA \ame) (mADalu) muMduvari; toDagu: \eng{go jump in the lake} saroVvaradoLakekx dhumukuvaSuTx muMduvari (aviveVkiyAgu). \eng{go and catch a falling star} (aviveVkiyAgu eMbathaRdalilx) biVLutitxruva nakaSxtarxvanunx hoVgi hiDiduko. 
\num{41} mADalu pArxraMBisuvaMtiru; toDaguvudaralilxru. 
\num{42} mADutAtx -- iru, hoVgu: \eng{go shopping, fishing} vAyxpAra mADutAtx iru, miVnu hiDiyutAtx hoVgu. 
\num{43} Agu; -- aMte vatiRsu; -- aMte naDeduko: \eng{go bail for person} vayxkitxge jAmInAgu. 
\num{44} (adhikArigaLa, nAyxyAlayada -- muMde) hoVgu; ahavAlu koMDoyuyx: \eng{he is determined to go to a jury} avanu paMcAyitara muMde hoVgalu (tananx ahavAlanunx koMDoyayxlu) nidhaRrisidAdxne. 
\num{45} hoVgu; Asharxyisu; avalaMbisu; baLi hoVgu; neravu koVru: \eng{you must go to Aristotle for that} adakekx niVnu arisATxTalanalilxgeV hoVgabeVku. \eng{go to war or work} yudadhxkekx yA kelasakekx hoVgu. 
\num{46} (nidiRSaTx haMtadavaregU) hoVgu; muMduvari: \eng{will go as high as Rs. 1{,}000} oMdu sAvira rU.gaLavaregU hoVgutAtxne (koLuLxvavanu yA harAjinalilx bele sUcisuvAga heVLuva mAtu). \eng{went to a great expense} bahaLa KacuR mADida; bahaLa KaciRnavarege hoVda. \eng{went to a great trouble} bahaLa -- kaSaTx paTaTx, sharxmavahisida. 
\num{47} oLatUru; oLanugugx; oLahogu. 
\num{48} muLugu; taLa -- seVru, talupu, muTuTx: \eng{the ship went to the bottom} haDagu (muLugi) taLa seVritu. 
\num{49} (oLage) hiDisu; hoVgu; tUru; avakAsha, jAga -- paDediru: \eng{will not go in (to) the basket} buTiTxge hiDisuvudilalx; buTiTxyoLakekx hoVguvudilalx. 
\num{50} (saMKeyxya \vi) hoVgu; oLagoMDiru; BAgavAgiru; nisheyxVSavAgi yA sheVSasahitavAgi inonxMdu saMKeyxyalilx iru: \eng{6 into 12 goes twice} hanenxraDaralilx \eng{6} eraDu sAri hoVgutatxde. \eng{6 into 13 goes twice and one left over} hadimUraralilx \eng{6} eraDu sAri hoVgi oMdu uLiyutatxde (\eng{1} sheVSa uLiyutatxde). 
\num{51} (oMdu vasutx inonxMdaroLakekx) hiDisu; sariyAgi seVru; sariyAgu; sari hoVgu; sarihoMdu: \eng{this book goes on the third shelf} I pusatxka mUraneya baDuvige hiDisutatxde. 
\num{52} (bahumAna, jaya, AsitxpAsitx, adhikAra, \mo vugaLa \vi) obabxna pAlige, obabxnige -- hoVgu, baru. 
\num{53} (oMdu PalitAMshakAkxgi, udedxVshakAkxgi) Agu; viniyoVgavAgu; baLakeyAgu; upayoVgavAgu: \eng{fees do not go towards the sustenance of the school} shulakx sUkxlina nivaRhaNege viniyoVgavAguvudilalx. 
\num{54} pariNAmakekx -- oyuyx, neravAgu, sahAyakavAgu, agatayxvAda aMshavAgiru: \eng{the bones which go to form the head and trunk} ruMDa matutx muMDavanunx racisalu neravAguva mULegaLu, racisuva mULegaLu. 
\num{55} pariNAma -- taru, uMTumADu, yA uMTumADuvaMtiru: \eng{two things go to render this statement worthless} eraDu aMshagaLu I heVLikeyanunx beleyilalxdAdxgisutatxve. 
\num{56} seVri oMdAgu; motatxvAgu: \eng{12 inches go to the foot} hanenxraDu aMgula seVri oMdu aDiyAgutatxde. 
\num{57} hoVgu; talupu; visatxrisu; vAyxpisu: \eng{the difference goes deep} vayxtAyxsa bahaLa dUra hoVgutatxde. 
\num{58} (yAvudoV nidiRSaTx sithxtige) baru; talupu; hoVgu; tirugu; (yAvudoV sithxti yA rUpa) Agu: \eng{go brown} kaMdubaNaNxkekx tirugu. \eng{go blind} kuruDAgu. \eng{go mad} hucAcxgu. \eng{go sleep}{} nidedxhoVgu.{} 
\enum
\emng

\noindent
\gl{\sakirx}
\bmng
\bnum
\numi{1} (isipxVTinalilx) 
\banum
\alnum{a} turuPu heVLu. 
\alnum{b} (elegaLa gotAtxda) jote -- ideyenunx, ideyeMdu GoVSisu: \eng{go two diamonds} DayamaMDf ele jote heVLu. 
\eanum
\numie
\num{2} (\AmA) paNa oDuDx; bAji kaTuTx: \eng{I will go two rupees on number seven} nAnu ELaneya naMbarina meVle eraDu rUpAyi bAji kaTuTxtetxVne. 
\enum
\emng

\noindent
\gl{\pagu}
\bmng
\bnum
\num{1} \eng{as (a person or thing) goes} (vayxkitx yA vasutx) sAdhAraNavAgi tegedukoMDare; hoVlisidare: \eng{a good actor as actors go} elalx naTaroDane hoVlisidare oLeLxya naTaneV. 
\num{2} \eng{as} (\engit{or} \eng{so) far as it goes} (heVLikeyoMdara athaRvanunx bahaLa vAyxpakavAgi garxhisuvudara virudadhx ecacxrikeya mAtAgi) adu iruva maTiTxge, aSaTxra maTitxge, oMdu nidiRSaTx haMtadavarege hoVguva maTiTxge -- heVLuvudAdare; oMdu mitiyoLage adara viSayavAgi heVLabeVkeMdare. 
\num{3} \eng{as the Bull} (\engit{or} \eng{verse} \engit{or} \eng{catechism) goes} AjAcnxpatarx heVLuvudeVneMdare; A paThayxda parxkAra; A gAdeyaMte. 
\numi{4} \eng{be going to do (something)} 
\banum
\alnum{a} udedxVshisiru; saMkalipxsiru; nidhaRrisu; yoVjisiru; yoVcisiru: \eng{we are going to spend our holidays in Ooty} nAvu namamx rajAdinagaLanunx UTiyalilx kaLeyalu udedxVshisidedxVve. \eng{I am going to have my own way} nananx manasisxge toVridaMte vatiRsalu nidhaRrisidedxVne. \eng{we are going to buy the house with the money we have saved} kUDiTiTxruva haNadiMda nAvu A maneyanunx koLaLxlu yoVcisidedxVve. 
\alnum{b} AguvaMtiru; AgabahudAgiru; saMBavaniVyavAgiru: \eng{is there going to be a storm?} birugALiya saMBava EnAdarU ideyeV? 
\alnum{c} hatitxradalilxru; sadayxda BaviSayxdalilxru: \eng{I am going to tell you a story} nAnu ninagoMdu kathe heVLalidedxVne. 
\eanum
\numie
\num{5} \eng{from the word go} (\AmA) pArxraMBadiMdaleV; shuruviniMdaleV; AdiyiMdaleV; moTaTx modaliMdaleV. 
\num{6} \eng{go a-begging} tirupe beVDutAtx iru; BikeSx etutxtAtx iru. 
\hypertarget{go pagu7}{} 
\numi{7} \eng{go about} 
\banum
\alnum{a} oMdu kelasadalilx -- maganxvAgiru, toDagiru, uduyxkatxnAgiru. 
\alnum{b} (\vakaq kekx muMce \parx) (yAvudeV oMdanunx) aBAyxsa mADiko; cALi mADiko: \eng{goes about telling lies} suLuLx heVLikoMDu tiruguva aBAyxsa mADikoMDidAdxne. 
\alnum{c} (\nw) eduru dikakxnunx hiDi; dikukx badalAyisi virudadhx dikakxnunx hiDi. 
\alnum{d} oMdu sathxLadiMda inonxMdu sathxLakekx hoVgu; beVrebeVre jAgagaLige hoVgu: \eng{he is going about with that girl} avanu A huDugiya jote elalx kaDeyU sutAtxDutitxdAdxne. 
\alnum{e} (vadaMti, kathe, \mo vu) obabxniMda obabxrige haraDu, habubx: \eng{a story is going about that} adara bagegx oMdu kathe haraDide. 
\eanum
\numie
\num{8} \eng{go a-doing} (\pArxparx) \pagu \eng{(21)}. 
\num{9} \eng{go ahead} (EnoMdU saMdeVha, saMkoVca ilalxde oMdu kelasadalilx) muMduvari; mununxgugx; kelasa muMduvarisu. 
\numi{10} \eng{go a long way} 
\banum
\alnum{a} (\sA\ \eng{towards} jatege) tuMba pariNAma biVru, iru. 
\alnum{b} (AhAra, haNa, \mo vu) bahaLa kAla baru; bALike baru. 
\alnum{c} = \hyperlink{go pagu25}{?pagu? \((25)\)}. 
\eanum
\numie
\num{11} \eng{go along with} opupx; samamxtisu; adeV aBipArxya hoMdiru. 
\num{12} \eng{go and do (something)} (\kanmu) hoVgi (EnanonxV) mADuvaSuTx aviveVkiyAgu. 
\hypertarget{go pagu13}{} 
\numi{13} \eng{go around} 
\banum
\alnum{a} sadA joteyalilx ODADutitxru. 
\alnum{b} = \hyperlink{go pagu7}{?pagu? \((7)\)}. 
\alnum{c} rUDhiyiMda mADutitxru; vADikeyAgi mADuvudaralilx toDagiru. 
\eanum
\numie
\num{14} \eng{go as-you-please} abAdhita; anibaRMdhita; savxcaCxMda. 
\numi{15} \eng{go at} 
\banum
\alnum{a} meVle biVLu; AkarxmaNa mADu; keY mADu; halelx mADu. 
\alnum{b} joVrAgi AraMBisu; birusiniMda keYgoLuLx. 
\eanum
\numie
\numi{16} \eng{go away} 
\banum
\alnum{a} horaTu hoVgu. 
\alnum{b} (\kanmu\ vihAra \mo vugaLigAgi mane biTuTx) beVre Urige hoVgu; Uru biTuTx hoVgu. 
\eanum
\numie
\numi{17} \eng{go back on} 
\banum
\alnum{a} mAtige tapupx; vacanaBarxSaTxnAgu; koTaTx mAtinaMte naDedukoLaLxdiru. 
\eanum
\numie
\num{18} \eng{go} \hyperref{kandict_b.pdf}{B}{bail(1) pagu(7)}{$^1$bail for.} 
\numi{19} \eng{go behind} 
\banum
\alnum{a} hiMdiruvudanunx huDuku: \eng{go behind a person's words} vayxkitxya mAtugaLa hiMdiruva, mareyalilxruva athaRvanunx huDuku. 
\alnum{b} (tiVpuR \mo vugaLa \vi) AdhAragaLanunx, kAraNagaLanunx punaviRmasheR mADu. 
\eanum
\numie
\numi{20} \eng{go by} 
\banum
\alnum{a} (pakakxdalilx) hAdu hoVgu. 
\alnum{b} avakAshavanunx -- kaLeduko, tapipxsiko, ilalxdaMte mADiko: \eng{don't let this chance go by} I avakAshavanunx tapipxsikoLaLxbeVDa. 
\alnum{c} (inonxbabxra mAtu, aBipArxya, \mo vanunx) naMbu; necucx; anusarisu: \eng{don't go by what she says} avaLu heVLuvudanunx necicxkoLaLxbeVDa. \eng{that is a good rule to go by} anusarisalu adoMdu oLeLxya sUtarx. 
\alnum{d} kaLedu hoVgu; sari: \eng{time went by} kAla sariyitu. 
\alnum{e} (aBipArxya yA tiVmARna rUpisalu) AdhAravAgiTuTxko: \eng{have we enough evidence to go by} nAvu tiVmARna mADalu sAkaSuTx sAkASxyXdhAra ideyeV? 
\eanum
\numie
\num{21} \eng{go doing} mADalu hoVgu: \eng{go fishing} mInu hiDiyalu hoVgu. \eng{went shopping} sAmAnu koLaLxlu aMgaDige hoVgu. 
\hyperdef{G}{go(1) pagu(22)}{} 
\numi{22} \eng{go down} 
\banum
\alnum{a} (haDagina \vi) muLugu. 
\alnum{b} (yAvudoV oMdu niyamita kAladavaregU) muMduvari; hoVgu; hoVgi nilulx: \eng{the article goes down to the death of Ashoka} A leVKana ashoVkana maraNadavaregU hoVgutatxde. 
\alnum{c} (vijayiya eduru) keLakekx biVLu; soVtuhoVgu; bidudxhoVgu: \eng{go down fighting} kAdutAtx maDi; hoVrADutAtx keLage biVLu. 
\alnum{d} (cariterxyalilx baravaNigeya rUpadalilx) cirakAla -- iru, niMtiru, uLidiru, liKitarUpakekx iLi: \eng{down it must go in her book} adu avaLa pusatxkakekx iLiyaleVbeVku. 
\alnum{e} gaMTalalilxLi: \eng{I want no sauce or pickle to make it go down} adu gaMTalalilx iLiyalu nanage gojujx, upipxnakAyi yAvudU beVDa. 
\alnum{f} opipxgeyAgu; aMgiVkAra paDe: \eng{a poet who would not go down among readers of present day} iMdina Odugarige mecucxgeyAgada obabx kavi. 
\alnum{g} (\birx) (\AmA) (vaSaRda yA avadhiya koneyalilx vishavxvidAyxnilayavanunx) biTuTx hoVgu. 
\alnum{h} (sUyaR, caMdarx, nakaSxtarxgaLa \vi) muLugu; asatxmisu; asatxMgatavAgu. 
\alnum{i} (\AmA) jeYlige kaLuhisalapxDu: \eng{went down for ten years} hatutx vaSaR jeYluvAsa anuBavisida, jYlige kaLuhisalapxTaTx. 
\alnum{j} keDu; keTuTxhoVgu; hALAgu; guNa kaLeduko. 
\alnum{k} viPalavAgu; ayashasivxyAgu. 
\alnum{l} (samudarx, ale, gALi, \mo vugaLa \vi) shAMtavAgu; kaDimeyAgu: \eng{the flood went down} parxvAha kaDimeyAyitu. 
\alnum{m} (motatxda \vi) iLi; kaDimeyAgu: \eng{the coffee has gone down a lot} kAPiya iLuvari bahaLa iLidide. 
\alnum{n} (bele, mwlayx, \mo vugaLa \vi) iLi; kaLeduko. 
\alnum{o} (kaMpUyxTarf jAlada \vi) niSikxrXyavAgu; kelasa mADadiru; niMtuhoVgu. 
\eanum
\numie
\numi{23} \eng{go down with} 
\banum
\alnum{a} (\birx) kAyile biVLu; roVgakekx tutAtxgu: \eng{I went down with malaria} nAnu maleVriyA kAyilege tutAtxde. 
\alnum{b} (vivaraNa, samajAyiSi, kate, nATaka, \mo vugaLa \vi. Oduga, keVLuga, viVkaSxka, \mo varige) hiDisu; opipxgeyAgu; samamxtavAgu; aMgiVkaqtavAgu: \eng{the new play went down well with the provincial audiences} hosa nATaka haLiLx perxVkaSxkarige BajaRriyAgi hiDisitu. \eng{that explanation does not go down with us} A vivaraNe namage opipxgeyAguvudilalx. 
\eanum
\numie
\num{24} \eng{goes to show}. toVrisalu neravAgu: \eng{it goes to show that the Dutch are not the equals of the English} Dacacxru iMgilxSara samAnaralalx eMbudanunx adu toVrisutatxde. 
\hypertarget{go pagu25}{} 
\num{25} \eng{go far} tuMba yashasivxyAgu; saPalavAgu. 
\num{26} \eng{go fetch} (nAyige koDuva Ajecnx) hoVgi tegedukoMDu bA! 
\num{27} \eng{go great guns} balavAgi, tiVvarxvAgi yA yashasivxyAgi -- muMduvari, sAgu. 
\num{28} \eng{go halves or shares} saripAlu haMciko; adhaR adhaR haMciko. 
\numi{29} \eng{go in} 
\banum
\alnum{a} (sapxdhiRyAgi) (ATakekx) seVru. 
\alnum{b} (kirxkeTf ATadalilx) ATakikxLi; ininxMgfsx pArxraMBisu. 
\alnum{c} (sUyaR \mo vugaLa \vi) (moVDagaLalilx) mareyAgu; kANade hoVgu. 
\alnum{d} (koVNe, mane, \mo vugaLa oLakekx) parxveVshisu. 
\eanum
\numie
\num{30} \eng{go in and win?} (ATagArananunx huriduMbisuvAga heVLuva mAtu) bArisu! hoDi! gelulx! itAyxdi. 
\numi{31} \eng{go in for} 
\banum
\alnum{a} (yAvudeV oMdanunx) tananx udedxVshavAgi, guriyAgi, riVtiyAgi, tatatxvXvanAnxgi -- mADiko, iTuTxko, keYkoLuLx: \eng{Bhagat Singh deliberately went in for martyrdom} BagatfsiMganu udedxVshapUvaRkavAgi hutAtamxnAdanu. 
\alnum{b} (EnanAnxdarU) KariVdi mADu; koMDuko; vAyxpAra mADu: \eng{they do not go in for hats} avaru hAYxTugaLanunx koMDukoLuLxvudilalx. 
\alnum{c} (yAvudAdarU oMdu pariVkeSxge) hoVgu; kUru; kuLituko; seVru; aBayxthiRyAgu: \eng{1061 candidates went in for mathematics} gaNitashAsatxrXda pariVkeSxge \eng{1061} jana kuLitaru, pariVkeSx tegedukoMDaru. 
\eanum
\numie
\numi{32} \eng{go into} 
\banum
\alnum{a} (pAliRmeMTige) parxveVshisu; cunAyitanAgu. 
\alnum{b} (kasabanunx, vaqtitxyanunx) hiDi; avalaMbisu; keYkoLuLx; seVru. 
\alnum{c} (samAjadalilx) seVrutitxru; ODADutitxru; hoVgi barutitxru. 
\alnum{d} (kAyaRkalApagaLalilx) BAgavahisu. 
\alnum{e} oLapaDisiko: \eng{go into hysterics} hucucx AveVshakekx oLagAgu. 
\alnum{f} pariVkiSxsu; vicAra mADu; taniKe naDesu. 
\alnum{g} (duHKasUcane \mo) uDupu -- toDu, dharisu, hAkiko. 
\alnum{h} (yAvudaradeV) aMgavAgu; aMshavAgu; aMgaBAgavAgu. 
\eanum
\numie
\numi{33} \eng{go it} (\AmA) 
\banum
\alnum{a} hecucx veVgadiMda hoVgu. 
\alnum{b} birusAgi, birusiniMda -- mADu. 
\alnum{c} (\rUpa) suKaloVlupanAgu. 
\hyperdef{G}{go(1) pagu(34)}{} 
\eanum
\numie
\num{34} \eng{go it strong} (\AmA) atishayisu; atishayoVkitx baLasu; (mAtina \vi) bahaLa dUra hoVgu. 
\numi{35} \eng{go nap} 
\banum
\alnum{a} (nAyxpf eMba isipxVTu ATadalilx) elAlx aidu paTuTxgaLanunx gelulxtetxVneMdu GoVSisu. 
\alnum{b} (\rUpa) elalxvanUnx oMdeV paNavAgi oDuDx. 
\eanum
\numie
\num{36} \eng{go native} (biLi janAMgadavana \vi) deVshiVyanAgu; yAra madheyx vAsisutAtxnoV A deVshiVyara anAgarika jiVvanakarxma hiDi, anusarisu, aLavaDisiko. 
\numi{37} \eng{go off} 
\banum
\alnum{a} (raMgasathxLavanunx) biDu; (raMgadiMda) niSakxrXmisu. 
\alnum{b} horaDu; hoVgu; calisu: \eng{the boats went off} doVNigaLu muMdakekx horaTavu. 
\alnum{c} siDi; hAru; AsoPxVTisu. 
\alnum{d} (karxmeVNa) aLi; kiSxVNisu; ilalxvAgu; hoVgibiDu; anuBavadiMda, ariviniMda mareyAgu. 
\alnum{e} sAyu. 
\alnum{f} (nidedx, mUCeR, \mo vugaLiMda) arivu kaLeduko; jAcnxnatapupx; parxjAcnxhiVnavAgu: \eng{go off to sleep} nidedx hoVgu. \eng{go off in} (\engit{or} \eng{into) a faint} (\engit{or} \eng{fit)} mUCeR hoVgu. 
\alnum{g} beVgane -- mArATavAgi hoVgu, KacARgu: \eng{the tickets will go off with a rush} TikeTuTxgaLu BarATeyiMda mArATavAgibiDutatxve. 
\alnum{h} (cenAnxgi \mo\ riVtiyalilx) jarugu; naDe; neraveVru. 
\alnum{i} (\kanmu\ AhAra padAthaRgaLa \vi) keDu; keTuTxhoVgu; guNa kaLeduko: \eng{the milk has gone off} hAlu keTuTxhoVgide, huLi hiDidide. 
\alnum{j} pArxraMBavAgu; AraMBisu. 
\alnum{k} (\birx) (\AmA) devxVSisalu, iSaTxpaDadiralu shurumADu: \eng{I've gone off him} nAnu avananunx devxVSisalAraMBisidedxVne. 
\eanum
\numie
\num{38} \eng{go off at} (\AseTxrXV\ matutx nUyxsiZVlaMDf) (\ashi) bayuyx; CiVmArihAku. 
\num{39} \eng{go off well} (\engit{or} \eng{badly)} (udayxma \mo vugaLa \vi) cenAnxgi (yA keTaTxdAdxgi) naDeyutitxru. 
\numi{40} \eng{go on} 
\banum
\alnum{a} muMduvarisu; biDade naDesu; paTuTx hiDidu mADu: \eng{decided to go on with it} adaroDane muMduvariyuvaMte tiVmARnisida. \eng{went on trying} biDade parxyatanxpaDutatxleV hoVda. \eng{unable to go on} muMduvarisalAgade; muMduvariyalAgade. 
\alnum{b} (yAvudanenxV mADuvalilx) muMdina hejejx iDu; muMdina karxma tegeduko; muMduvari: \eng{he goes on to quote Vyasa} avanu muMduvaridu vAyxsananunx ulelxVKisutAtxne. 
\alnum{c} (nAcikegeVDina \mo\ riVtiyalilx) vatiRsu; naDeduko: \eng{she is playing the fool to go on in this style} iSuTx keTaTx riVti vatiRsutAtx avaLu aviveVkiyAgutitxdAdxLe. \eng{shameful, the way they went on} avaru vatiRsida riVti nAcikegeVDu. 
\alnum{d} (\AmA) bAyige baMdaMte -- ADu, bayuyx, dUSisu: \eng{went on and on at him} avanige bAyige baMdaMte bayuyxtatxleV idadxLu. 
\alnum{e} (\AmA) (raMgasathxLadalilx) pAtarxvahisu; raMgada meVle -- baru, kANisiko. 
\alnum{f} (kirxkeTf) boVlf mADalu pArxraMBisu. 
\alnum{g} (kelasadalilx tananx saradi baMdAga) mADu; nivaRhisu. 
\alnum{h} (\AmA) (vidhi rUpadalilx) bogaLabeVDa! bAyige baMda hAge haraTabeVDa! 
\alnum{i} (sidadhx uDupu \mo vugaLa \vi) hAkikoLuLxvavanige hiDisuvaSuTx doDaDxdAgiru, aLaLxkavAgiru. 
\alnum{j} vipariVta -- mAtADu, mAtu beLesu; agatayxvAdudakikxMta diVGaRvAgi mAtADu. 
\alnum{k} Agu; saMBavisu; jarugu. 
\alnum{l} (jilelx, parihAra nidhi, \mo vugaLa) bAbitxge biVLu; KaciRge seVru: \eng{go on the relief fund} parihAra nidhiya KaciRge seVru. 
\alnum{m} (\AmA) (\eng{go upon} saha) sAkaSxyXvAgi, purAveyAgi -- baLasu: \eng{police don't have anything to go on} sAkaSxyXvAgi baLasalu poliVsarige EnU ilalx. 
\alnum{n} (\AmA) (\kanmu\ niSeVdha vAkayxdalilx oMdara bagegx, bahaLa, savxlapx, \mo) Asakitx vahisu; gamanakoDu; tale keDisiko: \eng{don't go much on red hair} keMpu kUdalina bagegx hecucx tale keDisikoLaLxbeVDa. 
\alnum{o} (\AmA) (\eng{go on!}) (huriduMbisuvAga yA apanaMbike sUcisuvAga baLasuva udAgxra). 
\eanum
\numie
\num{41} \eng{go on the streets} (biVdi) sULeyAgu. 
\numi{42} \eng{go out} 
\banum
\alnum{a} (koThaDi, mane, \mo vanunx biTuTx) horakekx hoVgu; nigaRmisu. 
\alnum{b} davxMdavxyudadhx mADu. 
\alnum{c} (reVDiyoV, patirxke, \mo vugaLalilx) parxsAravAgu. 
\alnum{d} naMdi hoVgu. 
\alnum{e} parxNaya toVrutitxru. huDugiyanunx olisikoLaLxlu parxyatinxsiru. 
\alnum{f} (sakARrada \vi) hoVgu; uruLu; patanavAgu; padacuyxtavAgu; adhikAra biTuTxkoDu, tayxjisu. 
\alnum{g} (phAyxSanf \mo vugaLa \vi) baLake tapupx; rUDhitapupx; kAlasithxtige hiMde biVLu: \eng{hero -- worship doesn't seem to have gone out} vayxkitxpUje Iga baLake tapipxdaMte kANutitxlalx. 
\alnum{h} vasAhatige nelasalu hoVgu; valasehoVgu. 
\alnum{i} (\sA\ makakxLanunx noVDikoLuLxva gavaneRsfgaLAgi hoVguva huDugiyara \vi) udoyxVgakAkxgi mane biTuTx hoVgu. 
\alnum{j} (\AmA) samAjadalilx bere, kale. 
\alnum{k} (kelasagArara \vi) muSakxra hiDi; saMpu hUDu. 
\alnum{l} (haqdaya \mo vu perxVma, karuNe, \mo vugaLiMda) higugx; tuMbu; ububx: \eng{his heart went out for the beautiful girl} A celuvegAgi avana haqdaya perxVmadiMda ubibxtu. \eng{my heart goes out to them} avara bagegx nananx haqdaya karuNeyiMda tuMbutatxde. 
\alnum{m} (gAlfphx ATadalilx) saraNiya modala oMbatutx kuLigaLanunx ADu. 
\alnum{n} (\AmA) parxjecnxtapupx; mUCeRhoVgu. 
\eanum
\numie
\numi{43} \eng{go over} 
\banum
\alnum{a} pakASxMtara mADu yA matAMtara hoMdu; (tananx) pakaSxvanunx yA dhamaRvanunx badalAyisu. 
\alnum{b} (ATa, nATaka, \mo vugaLa \vi) yashasivxyAgu; jayagaLisu: \eng{the drama went over well in Mysore} meYsUrinalilx nATaka yashasivxyAyitu. 
\alnum{c} (ecacxrikeyiMda) noVDu; pariVkiSxsu; parishiVlisu: \eng{go over the accounts} lekakxpatarxgaLanunx parishiVlisu. 
\eanum
\numie
\numi{44} \eng{go round} 
\banum
\alnum{a} sutitxkoMDu baru. 
\alnum{b} (\AmA) BeVTi mADahoVgu. 
\alnum{c} baLasu; parxdakiSxNe mADu. 
\alnum{d} keY badalAyisu; oMdu keYyiMda inonxMdu keYge -- hoVgu, dATu. 
\alnum{e} sututxbaruvaSuTx iru. 
\alnum{f} (AhAra \mo vu) neravigelalx, samudAyakekxlalx -- sAkAgu, sAkAguvaMtiru: \eng{we have barely enough to go round} elalxrigU sAkAguvaSuTx namamxlilxlalx. 
\alnum{g} = \hyperlink{go pagu13}{?pagu? \((13)\)}. 
\eanum
\numie
\num{45} \eng{go sick} (seYnayx) kAyileyavara paTiTxyalilx dAKalAgu, dAKalu mADisiko. 
\numi{46} \eng{go slow} 
\banum
\alnum{a} (\kanmu) udedxVshapUvaRkavAgi nidhAnamADu, CAnasavAgi kelasa mADu. 
\alnum{b} (saMcAra niyaMtarxNada saMkeVta) nidhAnavAgi calisi! 
\eanum
\numie
\numi{47} \eng{go through} 
\banum
\alnum{a} vivaravAgi caciRsu; tapashiVlAgi vicAra mADu. 
\alnum{b} (ecacxrikeyiMda) parishiVlisu; parAmashiRsu. 
\alnum{c} (samAraMBa, vAcana, \mo vanunx) naDesu; neraveVrisu. 
\alnum{d} anuBavisu: \eng{go through hardships} kaSaTxgaLanunx anuBavisu. 
\alnum{e} (pusatxkada \vi) (aneVka AvaqtitxgaLalilx) parxkaTavAgu. 
\alnum{f} (\AmA) (haNa) KacuR mADu. 
\alnum{g} mugi; konegoLuLx; konemuTuTx; samApatxvAgu: \eng{the deal did not go through} vayxvahAra mugiyalilalx. 
\alnum{h} (masUde \mo vu) aMgiVkaqtavAgu; opipxge paDe: \eng{the bill did not go through} masUde aMgiVkaqtavAgalilalx. 
\alnum{i} KacuRmADibiDu; baLasi mugisu. 
\alnum{j} oLatUru; tUtu mADu; raMdharx kore. 
\alnum{k} huDuku; pariVkiSxsu: \eng{the police went through the pockets of the suspected thief} poliVsaru anumAnakokxLagAda kaLaLxna kisegaLanunx huDukidaru. 
\alnum{l} BAgavahisu; pAlugoLuLx: \eng{she made him go through both a civil and a religious marriage} avaLu avananunx kAyide vivAha hAgU dhAmiRka vivAha iveraDU vivAhavidhigaLalilx pAlugoLuLxvaMte mADidaLu. 
\alnum{m} (\AseTxrXV) (\ashi) taletapipxsikoMDu hoVgu; parAriyAgu. 
\eanum
\numie
\num{48} \eng{go through with} nisheyxVSavAgi mugisu; pUreYsu; konegoLisibiDu; samApitxgoLisu; saMpUNaRgoLisu; konegANisu: \eng{he is only going through with it as a duty} avanu adanunx keVvala kataRvayxveMdu BAvisi mugisutitxdAdxne. 
\numi{49} \eng{go to} 
\banum
\alnum{a} (vidhi rUpa) (\pArxparx) (budidhx heVLuvAga, apanaMbike matutx asahane sUcisuvAga \parx) sAku sAku! sumamxniru! sari, sari! biDu, biDu! 
\alnum{b} oTuTx motatxvAgu: \eng{twelve inches go to make a foot} hanenxraDu aMgulagaLu seVri oMdu aDi Agutatxde. 
\alnum{c} \eng{go to} (\engit{or} \eng{towards) make} Agisalu, uMTumADalu -- neravAgu, sahAyakavAgu, anukUlisu: \eng{what qualities go to the making of a statesman?} rAjaniVtijacnxnanAnxgisuva guNagaLu yAvuvu? \eng{this money can go towards the house you want to buy} niVnu koLaLxbeVkeMdiruva manege I haNa anukUlavAgabahudu. 
\eanum
\numie
\numi{50} \eng{go together} 
\banum
\alnum{a} oTiTxgiru; oTiTxge hoVgu; oDaniru; sahavatiRyAgiru; anuSaMgavAgiru. 
\alnum{b} hoMdikeyAgiru; sAmaMjasayxdiMdiru; susaMgatavAgiru: \eng{which of these colours go well together?} I baNaNxgaLalilx yAvuvu cenAnxgi hoMdikoLuLxtatxve? 
\alnum{c} (gaMDu heNiNxna \vi\ maduve mADikoLuLxva udedxVshadiMda) joteyalilx ODADu; joteyAgi aDADxDu, sutAtxDu. 
\eanum
\numie
\num{51} \eng{go to great expenses} (\engit{or} \eng{trouble)} bahaLa KacuRmADu (yA bahaLa toMdare tegeduko). 
\num{52} \eng{go to it!} (\AmA) (\kanmu\ vidhirUpadalilx) kelasa shuru mADu; pArxraMBisu. 
\num{53} \eng{go to pieces} cUrucUrAgu; oDeduhoVgu (\rUpa\ saha). 
\num{54} \eng{go to school} shikaSxNa paDe. 
\num{55} \eng{go to sea} nAvikanAgu. 
\numi{56} \eng{go to seed} 
\banum
\alnum{a} biVjavAgu; hUvu biDuvudu nilulx. 
\alnum{b} (\rUpa) vayxthaRvAgu; hALAgu. 
\eanum
\numie
\num{57} \eng{go to show} (\engit{or} \eng{prove)} toVrisalu (yA sAdhisalu) sahAyakavAgu. 
\num{58} \eng{go to somebody} obabxna vashakekx, sAvxdhiVnakekx, pAlige -- hoVgu: \eng{who did the property go to when the old man died?} muduka satAtxga Asitx yAra pAlige hoVyitu? \eng{the first prize went to Kavya} modalaneV bahumAna kAvayxLige hoVyitu. 
\num{59} \eng{go to stool} kakakxsige hoVgu. 
\num{60} \eng{go to the bar} vakiVlanAgu; lAyarAgu; nAyxyavAdiyAgu. 
\num{61} \eng{go to the country} (\birx) janamata arasu; sAvaRjanika cunAvaNeya mUlaka janABipArxya tiLiduko, pariVkiSxsu. 
\numi{62} \eng{go under} 
\banum
\alnum{a} (vayxkitx, haDagu, \mo vara \vi) muLugihoVgu. 
\alnum{b} muridu biVLu. 
\alnum{c} soVtu hoVgu; nAshavAgu. 
\eanum
\numie
\numi{63} \eng{go up} 
\banum
\alnum{a} (\birx) (\AmA) vishavxvidAyxnilayavanunx -- hatutx, seVru, parxveVshisu. 
\alnum{b} bele hecAcxgu; teVjiyAgu. 
\alnum{c} siDi; soPxVTisu. 
\alnum{d} (uri hiDidu yA hogeyiTuTx) AhutiyAgu; nAshavAgu. 
\eanum
\numie
\num{64} \eng{go well} (\engit{or} \eng{ill etc.)} (\sA\ \eng{with} oDane) oLeLxyadAgi (yA keTaTxdAgi) pariNamisu, Agu. 
\numi{65} \eng{go with} 
\banum
\alnum{a} joteyalilx hoVgu: \eng{I will go with you} nAnu ninanx jote barutetxVne, hoVgutetxVne. 
\alnum{b} jote iru; anugatavAgiru: \eng{disease often goes with squalor} roVga aneVka veVLe holasina jate irutatxde. 
\alnum{c} adeV pakaSx vahisu, aBipArxyapaDu: \eng{I won't go with you on that} A viSayadalilx nAnu ninanx aBipArxya hoMdiralAre. 
\alnum{d} opupx; samamxtisu. 
\alnum{e} hoMdiko; anuguNavAgiru. 
\alnum{f} samarasadiMdiru. 
\alnum{g} tAtapxyaR garxhisu. 
\alnum{h} (\AmA) (huDugana yA huDugiya \vi) (pArxyashaH maduveya daqSiTxyiMda) joteyalilx ODADu. 
\eanum
\numie
\numi{65} \eng{go without} 
\banum
\alnum{a} ilalxde -- hoVgu, iru. 
\alnum{b} ilalxdidadxrU heVgoV anusarisikoMDu, sudhArisikoMDu hoVgu: \eng{we shall just have to go without} adu ilalxdidadxrU nAvu heVgoV anusarisikoMDu hoVgabeVkAgide. 
\eanum
\numie
\numi{66} \eng{to go} 
\banum
\alnum{a} (\ame) (AhAra, upAhAra, \mo vugaLa \vi) mArATa mADida jAgadalilxlalxde beVreDe tegedukoMDu hoVgi tinanxlu: \eng{coffee and doughnuts to go} horagaDe tegedukoMDu hoVgi tinanxlu kAphi matutx kajAjxyagaLu. 
\alnum{b} uLidiru; mikikxru: \eng{two pages to go} inUnx eraDu puTagaLu uLidive. 
\hyperdef{G}{gopagu67}{} 
\eanum
\numie
\num{67} \eng{who goes there?} (kAvalugArana kUgu) yAradu? yAralilx hoVgutitxruvavaru? 
\enum
\emng

\noindent
\gl{\nuga}
\bmng
\bnum
\num{1} \eng{go all lengths} (yAvudAdarU oMdu kelasadalilx) eSuTx dUra beVkAdarU hoVgu, muMduvari. 
\num{2} \eng{go back on} (\engit{or} \eng{upon) one's word} mAtige tapupx; vacanaBaMgamADu. 
\num{3} \eng{go by default} (kakiSxgArana tapipxniMdAgi) mokadadxme vayxtirikatxvAgu, viroVdhavAgu: \eng{the case goes by default} (kakiSxdArana) geYruhAjariyiMdAgi mokadadxme avanige vayxtirikatxvAgi Agutatxde. 
\num{4} \eng{go by} (\engit{or} \eng{under) the name of} A hesaru hotitxru; -aMte kareyalapxDu. 
\numi{5} \eng{go for} 
\banum
\alnum{a} taralu hoVgu; tegedukoMDu baralu hoVgu. 
\alnum{b} paDeyalu parxyatinxsu: \eng{he is going for the championship} (sapxdheRyalilx) avanu cAYxMpiyanfgiri paDeyalu parxyatinxsutitxdAdxne. 
\alnum{c} (\ashi) keYmADu; meVlebiVLu; AkarxmaNa naDesu; halelxmADu: \eng{they went for the stranger with a vengeance} seVDiniMda avaru AgaMtukana meVle bidadxru. 
\alnum{d} iSaTx paDu; Ashisu: \eng{the kind of life you would go for} niVnu iSaTxpaDabahudAdaMtha jiVvana. \eng{that is the one I go for} nAnu iSaTxpaDuvudu adu. 
\alnum{e} aMtiru; adakekx samavAgiru; hAge BAvitavAgiru: \eng{go for nothing} (\engit{or} \eng{little)} beleyilalxdAgiru; upayoVgavilalxde hoVgu. 
\alnum{f} anavxyavAgu; anavxyisu: \eng{what I have said about him goes for you too} avana bagegx nAnu heVLiruvudu ninagU anavxyisutatxde. 
\eanum
\numie
\numi{6} \eng{go hot and cold} 
\banum
\alnum{a} javxra baru. 
\alnum{b} nAcikege oLagAgu. 
\eanum
\numie
\num{7} \eng{go it} \hyperref{kandict_a.pdf}{A}{alone(1) nuga}{$^1$alone}. 
\num{8} \eng{go off one's head} hucucxhiDi; talekeDu. 
\hypertarget{go nuga9}{} 
\num{9} \eng{go the way of all the earth} (\engit{or} \eng{of all flesh)} sAyu; satutxhoVgu. 
\num{10} \eng{go to a better world} = \hyperlink{go nuga9}{?nuga? \((9)\)}. 
\num{11} \eng{go to Canossa} modalu haTa hiDidu konege tagigxhoVgu, talebAgu. 
\num{12} \eng{go to Jericho (}\engit{or} \eng{Bath} \engit{or} \eng{blazes} \engit{or} \eng{hell)} (nananx kaNeNxdurige nilalxbeVDa eMbathaRdalilx) tolagihoVgu; nAshavAgi hoVgu. 
\num{13} \eng{go too} \hyperref{kandict_f.pdf}{F}{far(1) nuga(11)}{$^1$far}. 
\num{14} \eng{go to one's account} = \hyperlink{go nuga9}{?nuga? \((9)\)}. 
\num{15} \eng{go to one's} \hyperref{kandict_h.pdf}{H}{heart(1) nuga(8)}{heart}. 
\num{16} \eng{go to one's own place} = \hyperlink{go nuga9}{?nuga? \((9)\)}. 
\num{17} \eng{go to one's reward} = \hyperlink{go nuga9}{?nuga? \((9)\)}. 
\num{18} \eng{go to the country} sAvaRtirxka cunAvaNeya mUlaka janABipArxya keVLu, tiLiduko, janamata saMgarxhisu. 
\num{19} \eng{go the whole} \hyperref{kandict_h.pdf}{H}{hog(1) nuga(1)}{$^1$hog}. 
\num{20} \eng{go to the} \hyperref{kandict_d.pdf}{D}{devil(1) nuga(12)}{$^1$devil}. 
\num{21} \eng{go up the line} (seYnayx) mUlasAthxna biTuTx yudadhxraMgakekx muMduvari, hoVgu. 
\num{22} \eng{go with the tide} (\engit{or} \eng{times)} kAlakekx takakxMte naDe; hatutx jana naDeda hAge niVnU naDe; itararaMte niVnU vatiRsu. 
\num{23} \eng{how goes it?} \engit{or} \eng{how is it going?} \engit{or} \eng{how are things going?} heVge naDeyutAtx ide? heVge muMduvariyutitxde? 
\num{24} \eng{it goes without saying} idu sapxSaTxvAgiyeV ide; idara bagegx heVLadidadxrU athaRvAgutatxde. 
\num{25} \eng{tongue goes nineteen to the dozen} eDebiDade mAtanADu; oMdeV samane mAtanADu. 
\num{26} \eng{would not go to} (\engit{or} \eng{for to) do it} (\asaM) mADuvaSuTx aviveVkakekx hoVguvudilalx. 
\enum
\emng
\eentry

\bentry
\word[go(2)]{go}
\pron{goV}
\gl{\nA}
\expl{(\bava\ \eng{goes} \ucAcx\ goVsfZ).}
\bmng
% 
\bnum
\num{1} hoVguvudu; gamana. 
\num{2} nigaRmana; niSakxrXmaNa. 
\num{3} kecucx; hurupu; Apu; sAhasa; edegArike; satatxvX. 
\num{4} (\AmA) tiVvarx, birusina, joVrAda -- kelasa: \eng{it's all go} adelalx birusina kelasa. 
\num{5} (\AmA) aniriVkiSxta tiruvu, parisithxti: \eng{here's a go!} idoMdu aniriVkiSxta tiruvu. \eng{what a go!} eMtha parisithxti! 
\num{6} (\AmA) kaSaTxda kelasa; sharxmada kAyaR. 
\num{7} (\AmA) geluvu; yashasusx; jaya: \eng{make a go of it} adanunx gelilxsu, yashasivxgoLisu. 
\num{8} (\AmA) yatanx; parxyatanx; saradi; satiR; sULu: \eng{have a go at it} oMdu keY noVDu; parxyatinxsi noVDu. \eng{scored seven at one go} oMdeV ETige ELu aMka giTiTxsida. 
\num{9} (\AmA) oMdu Avaqtitx; oMdu salakekx baDisida madayx \mo vu. 
\num{10} yAvudeV kAyileya oMdu sututx, sULu, Avaqtitx. 
\enum
\emng

\noindent
\gl{\pagu}
\bmng
\hypertarget{go pagu1}{} 
\bnum
\num{1} \eng{all the go} (\AmA) kAlAnuguNavAdadudx; rUDhiyalilxruvudu; PAYxSanunx: \eng{he becomes all the go in the University} vishavxvidAyxnilayadalilx avaneV PAYxSaninxna savaRsavxvAgidAdxne. 
\num{2} \eng{at one go} oMdeV parxyatanxdalilx; oMdeV -- ETiyalilx, salakekx. 
\num{3} \eng{have a go at} (yAvudeV oMdanunx sAdhisalu) parxyatanx mADu; keYnoVDu. 
\num{4} \eng{it's a go} (\AmA) (opapxMda, karAru, \mo vu) samamxta; opipxgeyAgide. 
\numi{5} \eng{it's no go} 
\banum
\alnum{a} A kelasa asAdhayx. 
\alnum{b} parisithxti shoVcaniVya. 
\eanum
\numie
\num{6} \eng{near go} (\AmA) kUdaleLeyaSaTxralilx pArAdadudx; savxlapxdaralilx tapipxsikoMDadudx. 
\hyperdef{G}{go(2) pagu(7)}{} 
\numi{7} \eng{no-go} 
\banum
\alnum{a} asAdhayx; sAdhisalu asAdhayx. 
\alnum{b} (aDaDxgaTuTx, parxtibaMdhakAjecnx, \mo vugaLiMdAgi) parxveVshisalu asAdhayx; duSapxrXveVshayx; aparxveVshayx. 
\eanum
\numie
\numi{8} \eng{on the go} (\AmA) 
\banum
\alnum{a} saMtata gatiyalilx; sadA calisutAtx; niraMtara -- calaneyalilx, cala sithxtiyalilx. 
\alnum{b} biVLu sithxtiyalilx; iLigatiyalilx. 
\eanum
\numie
\num{9} \eng{quite the go} = \hyperlink{go pagu1}{?pagu? \((1)\)}. 
\enum
\emng
\eentry

\bentry
\word[go(3)]{go}
\pron{goV}
\gl{\gu}
\bmng
 (\AmA) 
\bnum
\num{1} sariyAgi kelasa mADutitxruva, sidadhxvAgiruva: \eng{the fuel system is go} iMdhana vayxvasethx sariyAgi kelasamADutitxde. 
\num{2} phAyxSaninxna: \eng{I am not a go person} nAnobabx phAyxSaninxna vayxkitxyalalx. 
\num{3} parxgatipara; Adhunika. 
\enum
\emng
\eentry

\bentry
\word[go(4)]{go}
\pron{goV}
\gl{\nA}
\bmng
 goV; manegaLiruva Palakada meVle biLi matutx kariya kalulx athavA kAyigaLiMda ADuva japAniVyara oMdu ATa. 
\emng
\eentry

\bentry
\word[goad(1)]{goad}
\pron{goVDf}
\gl{\nA}
\bmng
\bnum
\num{1} monegoVlu; cucucxgoVlu; tivigoVlu; danagaLanunx tividu curukugoLisuva, moneyuLaLx koVlu. 
\num{2} (\rUpa) coVdaka; yAtane koDuva, uderxVkisuva, yA huriduMbisuva -- vasutx. 
\enum
\emng
\eentry

\bentry
\word[goad(2)]{goad}
\pron{goVDf}
\gl{\sakirx}
\bmng
\bnum
\num{1} monegoVliniMda tivi, coVdisu. 
\hypertarget{goad(2)2}{} 
\num{2} reVgisu; keraLisu; uderxVkagoLisu. 
\num{3} ODisu; cucicx naDesu; ODuvaMte mADu. 
\enum
\emng

\noindent
\gl{\pagu}
\bmng
 \eng{goad on} = \hyperlink{goad(2)2}{$^2$goad \((2 \& 3)\)}. 
\emng
\eentry

\bentry
\word[go-ahead(1)]{go-ahead}
\pron{goVaheDf}
\gl{\gu}
\bmng
\bnum
\num{1} mununxgugxva; aDetaDeyilalxde, hiMdumuMdu noVDade muMdehoVguva. 
\num{2} dheYyaRshAli; sAhasiganAda: \eng{a go - ahead yankee peddler} sAhasiganAda yAMkiV tirugu vAyxpAri. 
\num{3} muMde hoVgalu sUcane niVDuva, sUcisuva: \eng{a go-ahead signal} muMde hoVgalu sUcisuva saMkeVta. 
\enum
\emng
\eentry

\bentry
\word[go-ahead(2)]{go-ahead}
\pron{goVaheDf}
\gl{\nA}
\bmng
\bnum
\num{1} muMduvarike sUcane yA anumati; muMde hoVgabahudeMba -- sUcane, anumati: \eng{they got the go-ahead on the construction work} kaTaTxDada kelasadalilx muMduvariyalu avarige anumati sikikxtu. 
\numi{2} (\ame) (\AmA) 
\banum
\alnum{a} mununxgugxva kirxye, utAsxha yA shakitx. 
\alnum{b} parxgati. 
\alnum{c} mahatAvxkAMkeSx. 
\eanum
\numie
\enum
\emng
\eentry

\bentry
\word{goal}
\pron{goVlf}
\gl{\nA}
\bmng
\bnum
\num{1} gurigaMba; reVsina koneya elelxyanunx toVrisuva gurutu. 
\num{2} (parxyatanxda, hebabxyakeya) guri; lakaSxyX; dheyxVya; udedxVsha. 
\num{3} gamayxsAthxna; talupabeVkAda sAthxna. 
\num{4} (kAlecxMDu \mo\ ATagaLalilx) (ceMDanunx tUrisabeVkAda) gurigaMbagaLu; goVlu. 
\num{5} (ideV riVti itara ATagaLalilx baLasuva) paMjara yA buTiTx. 
\num{6} (goVlinoLakekx ceMDanunx hoDedu paDeda) gelalxMka(gaLu); goVlu(gaLu). 
\num{7} (roVmanf \pArxparx) tiruvu kaMba; rathapaMdayxda pathadalilxya tiruvinalilx neTiTxruva kaMba. 
\enum
\emng

\noindent
\gl{\pagu}
\bmng
\bnum
\num{1} \eng{drop a goal} goVlu biDu; edurALiya ceMDanunx taDeyade yA hiDiyade, goVlinoLakekx tUruvaMte biDu. 
\num{2} \eng{kick a goal} goVlinoLakekx odi yA odudx goVlugaLisu. 
\num{3} \eng{make a goal} oMdu goVlu hoDi. 
\num{4} \eng{score a goal} oMdu goVlu -- gaLisu, giTiTxsu, saMpAdisu. 
\enum
\emng
\eentry

\bentry
\wordnospeech{goal average}{goal average}
\pron{?}
\gl{\nA}
\bmng
 goVlu sarAsari; paMdayxgaLa saraNigaLalilx oMdu taMDavu inonxMdara virudadhx hoDeda goVlugaLa sarAsari. 
\emng
\eentry

\bentry
\wordnospeech{goal difference}{goal difference}
\pron{?}
\gl{\nA}
\bmng
 goVlu vayxtAyxsa; eraDu yA hecicxna taMDagaLu gaLisida, hoDeda goVlugaLa motatxdalilxna vayxtAyxsa. 
\emng
\eentry

\bentry
\word{goalie}
\pron{goVli}
\gl{\nA}
\bmng
 (\AmA)  = \hyperlink{goalkeeper}{goalkeeper}. 
\emng
\eentry

\bentry
\word{goalkeeper}
\pron{goVlfkiVparf}
\gl{\nA}
\bmng
 goVli; goVlugAra; goVlurakaSxka; ceMDu goVlinoLakekx hoVgadaMte taDeyuva ATagAra. 
\emng
\eentry

\bentry
\word{goal-kick}
\pron{goVlfkikf}
\gl{\nA}
\bmng
 goVlukikukx; goVlodeta: 
\banum
\alnum{a} ceMDanunx edurALigaLu goVlige badalu goVlugereyiMda Acege odAdxga, rakaSxkanu goVlf parxdeVshadalilxTuTx penaliTx parxdeVshadiMdAcege (\eng{18} gajagaLu) odeyuva odeta. 
\alnum{b} (ragibx kAlecxMDATa) goVlu hoDeyuva parxyatanx; goVlu hoDeta. 
\eanum
\emng
\eentry

\bentry
\word{goalless}
\pron{goVlflisf}
\gl{\gu}
\bmng
\bnum
\num{1} (kAlecxMDATa) goVlilalxda; goVlurahita; eraDu kaDeyavarU oMdeV oMdu goVlanUnx hoDeyada. 
\num{2} gotutxguriyilalxda; udedxVsharahita: \eng{one evening, in my goalless wandering} oMdu saMje, nananx gotutxguriyilalxda aledATadalilx. 
\enum
\emng
\eentry

\bentry
\word{goal-line}
\pron{goVlfleYnf}
\gl{\nA}
\bmng
 goVlugere; goVlina elelxgere; ATada bayalina elelx gurutisuva, parxti jateya goVlukaMbagaLa neVradalilx udadxkUkx eraDU kaDe aDaDxvAgi eLediruva gere. 
\emng
\eentry

\bentry
\word{goal-minder}
\pron{goVlfmeYMDarf}
\gl{\nA}
\bmng
 (\ame)  = \hyperlink{goal-tender}{goal-tender}. 
\emng
\eentry

\bentry
\word{goal-mouth}
\pron{goVlfmwtf}
\gl{\nA}
\bmng
 goVlu bAyi; goVlukaMbagaLa naDuvaNa yA hatitxrada jAga. 
\emng
\eentry

\bentry
\word{goal-tender}
\pron{goVlfTeMDarf}
\gl{\nA}
\bmng
 (\ame) aisf hAki ATada goVlu rakaSxka, goVli. 
\emng
\eentry

\bentry
\word{goanna}
\pron{goVAyxna}
\gl{\nA}
\bmng
 (\AseTxrXV) varanasf kulada doDaDx halilx. 
\emng
\eentry

\bentry
\word{go-as-you-please}
\pron{goVAyxsfyupilxVsf}
\gl{\gu}
\bmng
 kaTuTxniTiTxlalxda; manabaMda; niyamabadadhxvalalxda; manasoV iceCxya. 
\emng
\eentry

\bentry
\word{goat}
\pron{goVTf}
\gl{\nA}
\bmng
\bnum
\num{1} meVke; ADu; hoVta; aja; meVSa. 
\num{2} (\bava dalilx) meVke jAti. 
\num{3} meVke jAtige seVrida pArxNi: \hyperref{kandict_m.pdf}{M}{mountain goat}{mountain goat}. 
\num{4} (\eng{Goat}) makara rAshi. 
\num{5} kacecxharuka; viSayalaMpaTa; vayxBicAri. 
\num{6} (\AmA) daDaDx; mUDha; muThAThxLa. 
\num{7} (\ame) balipashu; itarara tapupxgaLanunx hotutxkoMDavanu yA horisalapxTaTxvanu. 
\enum
\emng

\noindent
\gl{\nuga}
\bmng
\bnum
\num{1} \eng{get one's goat} (\ashi) reVgisu; koVpabaruvaMte mADu. 
\num{2} \eng{goat's wool} ilalxda vasutx; meVke uNeNx (kudure moTeTx, molada koMbu enunxvaMte). 
\num{3} \eng{play the} \hyperlink{giddy(1) nuga(2)}{$^1$giddy goat.} 
\num{4} \hyperref{kandict_s.pdf}{S}{sheep pagu(2)}{sheep and goats.} 
\enum
\emng
\eentry

\bentry
\word{goat-antelope}
\pron{goVTfAYxMTiloVpf}
\gl{\nA}
\bmng
 meVke jiMke; ADeraLe; meVke vaMshakekx seVrida jiMkeyaMtha pArxNi. 
\emng
\eentry

\bentry
\word{goatee}
\pron{goVTiV}
\gl{\nA}
\bmng
 (galalxda meVle beLesuva) hoVtana gaDaDx; hoVtana gaDaDxdaMtha gaDaDx. \imglink{goateefigure}{\raisebox{-0.15cm}[0pt][0pt]{\pdfimage width 0.5cm height 0.6cm{G_Pictures/goatee.jpg}}} 
\emng
\eentry

\bentry
\word{goat-god}
\pron{goVTfgADf}
\gl{\nA}
\bmng
 (girxVkara) pAyxnf deVvate. 
\emng
\eentry

\bentry
\word{goatherd}
\pron{goVTfhaDfR}
\gl{\nA}
\bmng
 meVkegAhi; meVke kAyuvavanu. 
\emng
\eentry

\bentry
\word{goatish}
\pron{goVTiSf}
\gl{\gu}
\bmng
\bnum
\num{1} meVkeya; meVkeyaMtha. 
\num{2} kAmuka; laMpaTa; viSayAsakatx; kAmaparxvaqtitxya. 
\enum
\emng
\eentry

\bentry
\word{goatishly}
\pron{goVTiSfli}
\gl{\kirxvi}
\bmng
\bnum
\num{1} meVkeyaMte; meVkeya hAge. 
\num{2} kAmukanaMte; kAmukana riVtiyalilx. 
\enum
\emng
\eentry

\bentry
\word{goatishness}
\pron{goVTiSfnisf}
\gl{\nA}
\bmng
\bnum
\num{1} meVketana; meVkeyaMtiruvudu. 
\num{2} kAmukate; viSaya laMpaTatana. 
\enum
\emng
\eentry

\bentry
\word{goatling}
\pron{goVTfliMgf}
\gl{\nA}
\bmng
 (oMderaDu vaSaRda) ADina mari. 
\emng
\eentry

\bentry
\wordnospeech{goat moth}{goat moth}
\pron{?}
\gl{\nA}
\bmng
 meVkenusi; meVSakiVTa; kAsiDeV vaMshakekx seVrida doDaDx kiVTa. 
\emng
\eentry

\bentry
\word{goat's-beard}
\pron{goVTfsxbiaDfR}
\gl{\nA}
\bmng
 (\savi) hoVtana gaDaDx: 
\banum
\alnum{a} meDoVsivxVTf eMba hesarina, gulAbi baLagada oMdu hUvina giDa. 
\alnum{b} sAyxlisxpheY eMdU, tarakAri siMpi eMdU kareyuva, oMdu tarakAri giDa. 
\alnum{c} kalxveVriya kulada oMdu shiliVMdharx. 
\eanum
\emng
\eentry

\bentry
\word{goatskin}
\pron{goVTfsikxnf}
\gl{\nA}
\bmng
\bnum
\num{1} ADina camaR. 
\num{2} ADina camaRdiMda mADida uDupu yA budadxli. 
\enum
\emng
\eentry

\bentry
\word{goatsucker}
\pron{goVTfsakarf}
\gl{\nA}
\bmng
 (\pArxvi) sivxphfTx hakikxyanunx hoVluva, oMdu iruLu hakikx. 
\emng
\eentry

\bentry
\word{goaty}
\pron{goVTi}
\gl{\gu}
\bmng
  = \hyperlink{goatish}{goatish}. 
\emng
\eentry

\bentry
\word[gob(1)]{gob}
\pron{gAbf}
\gl{\nA}
\bmng
 (\asaM) loVLegaDeDx; loVLe mudedx; loVLe vasutxvina garaNe uMDe, \udA\ uguLu, shelxVSamx. 
\emng
\eentry

\bentry
\word[gob(2)]{gob}
\pron{gAbf}
\gl{\akirx}
\expl{(\BU\ matutx \BUkaq\ \eng{gobbed,} \vakaq\ \eng{gobbing}).}
\bmng
(\asaM) uguLu. 
\emng
\eentry

\bentry
\word[gob(3)]{gob}
\pron{gAbf}
\gl{\nA}
\bmng
 (\ashi) amerikada nAvika, kalAsi. 
\emng
\eentry

\bentry
\word[gob(4)]{gob}
\pron{gAbf}
\gl{\nA}
\bmng
 (\ashi) bAyi. 
\emng
\eentry

\bentry
\word{gobang}
\pron{goVbAYxMgf}
\gl{\nA}
\bmng
goVbAYxMgf; cwkagaLiruva Palakada meVle ADuva oMdu bageya japAniVyara ATa. 
\emng
\eentry

\bentry
\word{gobbet}
\pron{gAbiTf}
\gl{\nA}
\bmng
\bnum
\num{1} (\pArxparx) (\kanmu\ hasiya mAMsada, AhArada) tuMDu; tuNuku; cUru; uMDe; mudedx. 
\num{2} (anuvAdakekx yA pariVkeSxyalilx TiVke TipapxNige, vAyxKeyxge koTaTx) garxMthada BAga; udadhxraNa. 
\enum
\emng
\eentry

\bentry
\word[gobble(1)]{gobble}
\pron{gAbflf}
\gl{\sakirx}
\bmng
\bnum
\num{1} (AhAravanunx) gapagapane, gabagabane -- tinunx; AturAturavAgi sadudx mADutAtx tinunx; gaTakakxne nuMgu, nuMgihAku; kabaLisu (\akirx\ saha). 
\num{2} (\ame) (\ashi) kasiduko; kitutxko; egarisu; hiDiduko; Pakakxne -- seLeduko, tegeduko: \eng{he gobbled up the clothes and set off} avanu baTeTxgaLanunx Pakakxne tegedukoMDu horaTubiTaTx. 
\num{3} sarasara Odu; beVgane Odu; atAyxseyiMda nuMgi hAkuvaMte Odu: \eng{bright girls can gobble up such books} jANa huDugiyaru aMtha pusatxkagaLanunx sarasara nuMgihAkabalalxru. 
\enum
\emng
\eentry

\bentry
\word[gobble(2)]{gobble}
\pron{gAbflf}
\gl{\nA}
\bmng
 (gAlfphx) ceMDu kuLiyoLakekx neVravAgi hoVguvaMtaha curuku hoDeta. 
\emng
\eentry

\bentry
\word[gobble(3)]{gobble}
\pron{gAbflf}
\gl{\akirx}
\bmng
\bnum
\num{1} (TakiR huMjada \vi) gaMTalalilx tananxdeV Ada vilakaSxNavAda dhavxnimADu, sadudx mADu. 
\num{2} (koVpa \mo vugaLiMda mAtanADuvAga) iMtha dhavxni mADu. 
\enum
\emng
\eentry

\bentry
\word{gobbledegook}
\pron{gAbalfDigu(gU)kf}
\gl{\nA}
\bmng
 (\ashi) vikaTaBASe; adhikAra vagaRda yA vaqtitxveYshiSaTxyXda ADaMbarayukatxvAda yA athaRvAgada pariBASe. 
\emng
\eentry

\bentry
\word{gobbledygook}
\pron{gAbalfDigu(gU)kf}
\gl{\nA}
\bmng
  = \hyperlink{gobbledegook}{gobbledegook}. 
\emng
\eentry

\bentry
\word{gobbler}
\pron{gAbalxrf}
\gl{\nA}
\bmng
\bnum
\num{1} nuMgALi; kabaLiga; AturaBakaSxka; gaTakakxne nuMguvava, kabaLisuvava; gapagapane tinunxvava; AturAturavAgi sadudx mADutAtx tinunxvava. 
\num{2} tukiR huMja. 
\enum
\emng
\eentry

\bentry
\word{gobby}
\pron{gAbi}
\gl{\nA}
\bmng
 (\ashi) 
\bnum
\num{1} (\ashi) = \hyperref{kandict_c.pdf}{C}{coastguard}{coastguard}. 
\num{2} amerikada nAvika, kalAsi. 
\enum
\emng
\eentry

\bentry
\wordRemoveSpace{Gobelin-tapestry}{Gobelin tapestry}
\pron{goVbalinf TAYxpisiTxrX}
\gl{\nA}
\bmng
 goVbalinf buTeTxVdAri vasatxrX; pAyxrisisxna goVbalinfsx kAKARneyalilx tayArAda yA A vasatxrXvanunx anukaraNe mADuva buTeTxVdAri vasatxrX, gavusu baTeTx. 
\emng
\eentry

\bentry
\word{gobemouche}
\pron{gAbfmUSf}
\gl{\nA}
\expl{(\bava\ \eng{gobemouches} \ucAcx\ adeV).}
\bmng
 sudidxgiVLa; sudidxguLi; sudidxbAka; sudidxpirxya; keVLidadxnenxlAlx sharxdedhxyiMda naMbuva sudidx perxVmi. 
\emng
\eentry

\bentry
\word{go-between}
\pron{goVbiTivxVnf}
\gl{\nA}
\bmng
\bnum
\num{1} madhayxsathx(gAra); saMdhAnakAra; rAji mADisuvavanu. 
\num{2} madhayxsathxgAra; pakaSxgaLa naDuve saMdeVsha, sUcane, \mo vanunx alilxMdililxge oyuyxvava. 
\num{3} kuMTaNi(ga); talehiDuka(ki). 
\enum
\emng
\eentry

\bentry
\word{goblet}
\pron{gAbilxTf}
\gl{\nA}
\bmng
\bnum
\num{1} (\pArxparx) giMDi; thAli; hiDigaLilalxda, boVguNiyAkArada, loVhada yA gAjina, kuDiyuva baTaTxlu. 
\num{2} (\kAparx) pAnapAterx. 
\num{3} madayxda baTaTxlu; piVThada gAju baTaTxlu; piVTha matutx kAMDa uLaLx gAjina loVTa. \imglink{gobletfigure}{\raisebox{-0.15cm}[0pt][0pt]{\pdfimage width 0.5cm height 0.6cm {G_Pictures/goblet.jpg}}} 
\enum
\emng
\eentry

\bentry
\word{goblin}
\pron{gAbilxnf}
\gl{\nA}
\bmng
tuMTa beVtALa; cAtukuTiTx; tuMTa devavx; vikAra rUpada tuMTa pishAci. 
\emng
\eentry

\bentry
\word{gobstopper}
\pron{gAbfsATxparf}
\gl{\nA}
\bmng
 doDaDx, gaTiTxyAda, ciVpuvaMtha sihi vasutx; bAyituMbuva miThAyi. 
\emng
\eentry

\bentry
\word{goby}
\pron{goVbi}
\gl{\nA}
\bmng
 (IjurekekxgaLu cakArxkAravAgi yA hiVrunaLikeyAgi kUDikoMDiruva) saNaNx mInu. 
\emng
\eentry

\bentry
\word{go-by}
\pron{goVbeY}
\gl{\nA}
\bmng
\bnum
\num{1} udedxVshapUvaRkavAda -- nilaRkaSxyX, upeVkeSx, asaDeDx. 
\num{2} beVkeMdeV noVDade biTuTxhoVguvudu, hAyudx hoVguvudu. 
\enum
\emng

\noindent
\gl{\pagu}
\bmng
 \eng{give the go-by to} 
\banum
\alnum{a} mIrihoVgu; hiMdakekx hAki hoVgu; muMde hoVgu. 
\alnum{b} hiMde biTuTx hoVgu. 
\alnum{c} tapipxsiko; jAriko; nuNuciko; sikakxde hoVgu. 
\alnum{d} nilaRkaSxyX toVrisu; upeVkiSxsu; asaDeDx mADu; kaDegaNisu; avagaNisu; tAtAsxra mADu. 
\alnum{e} gurutiradaMte, paricayavilalxdaMte vatiRsu. 
\eanum
\emng
\eentry

\bentry
\word{GOC}
\pron{?}
\gl{\saMkiSx}
\bmng
 \eng{General Officer Commanding.} 
\emng
\eentry

\bentry
\word{go-cart}
\pron{goVkATfR}
\gl{\nA}
\bmng
\bnum
\num{1} keYbaMDi; nUku baMDi; taLuLxgADi. 
\num{2} (\pArxparx) (makakxLige naDige kalisuva) naDegADi. 
\num{3} taLuLxkuciR; taLiLxkoMDu hoVgalu gAligaLanunx aLavaDisiruva kuciR. 
\enum
\emng
\eentry

\bentry
\word[god(1)]{god}
\pron{gADf}
\gl{\nA}
\bmng
\bnum
\num{1} deVvaru; deVva; deVvate. 
\num{2} deVvaru; deVvate; deYvasaMkeVtaveMdu yA deVvara AvAsasAthxnaveMdu BAvisi pUjisuva vigarxha, parxtime, pArxNi yA vasutx. 
\num{3} (deVvara) vigarxha; mUtiR; parxtime. 
\num{4} pUjayx (vayxkitx); deVvaru; shAlxGaniVya, mecicxge paDeda yA parxBAvashAli -- vayxkitx. 
\num{5} (nATakashAle) (\bava dalilx) `gAyxlari' parxBugaLu; meVlaTaTxda perxVkaSxkaru; (agagxda) meVlamxTaTxdalilx kuLitiruvavaru. 
\num{6} (\eng{God}) BagavaMta; paramAtamx; parameVshavxra; vishavxda saqSiTxkataR matutx pAlaka: \eng{the Lord God, Almighty God, God} parameVshavxra; savaRshakatx BagavaMta. 
\enum
\emng

\noindent
\gl{\pagu}
\bmng
\hyperdef{G}{god nuga(1)}{} 
\bnum
\num{1} \eng{act of God} deYvaGaTane; hatoVTige mIrida nisagaR shakitxgaLa kirxyeyiMdAda GaTane. 
\num{2} \eng{blind god} = \hyperlink{god pagu16}{?pagu? \((16)\)}. 
\num{3} \eng{by God} deVvarANe. 
\num{4} \eng{for God's sake} (bahaLavAgi goVgareyuvalilx) deVvarigAgi; deVvarigoVsakxra. 
\num{5} \eng{God bless -- me!, my soul!, my life!, you! etc.} (AshacxyaRsUcaka udAgxragaLu) abAbx deVvareV! deVvaru kApADali! BagavaMta oLeLxyadu mADali. 
\num{6} \eng{God damn (you, him, etc.)} (ninage, avanige, itAyxdi) deVvaru keDukanunxMTu mADali; (ninanxnunx itAyxdi) nAshamADali, narakakekx dUDali. 
\num{7} \eng{God forbid!} (deVvara dayeyiMda) hAgAgadirali! deVvaru -- tapipxsali, AgagoDadirali! 
\num{8} \eng{god from machine} = \hyperref{kandict_d.pdf}{D}{deus ex machina}{\it deus ex machina.} 
\num{9} \eng{God grant (that etc.)} (pArxthaRneyalilx) deVvareV (adanunx) neraveVrisu. 
\num{10} \eng{God help (you, him, etc)} deVvaru (ninage, avanige) sahAya mADali. 
\num{11} \eng{God in heaven good God, my God, oh God} (noVvu, aLalu yA koVpa sUcisuvAga baLasuva udAgxra) ayoyxV deVvareV! BagavaMta! 
\hyperdef{G}{god(1) pagu(12)}{} 
\hypertarget{god pagu12}{} 
\numi{12} \eng{God knows} 
\banum
\alnum{a} (nananx yA matayxRmAnavana arivige, manuSayxna arivige mIridudx eMbudanunx sUcisalu heVLuva mAtu) deVvarobabxnige gotutx; BagavaMta mAtarx balalx. 
\alnum{b} nananx heVLikege deVvareV sAkiSx; deVvara sAkiSxyAgi nAnu heVLuvudeVneMdare. 
\eanum
\numie
\numi{13} \eng{god of day} 
\banum
\alnum{a} sUyaR. 
\alnum{b} (\girxVpu) apAloV (sUyaRdeVva). 
\alnum{c} (\roVpu) phiVbasf. 
\eanum
\numie
\num{14} \eng{god of heaven} (\girxVpu) sUZyxsf, (\roVpu) jUpiTarf; savxgARdhipati; divasapxti. 
\num{15} \eng{god of hell} narakAdhipati; (\girxVpu) pUlxTo; Disf; yama. 
\hypertarget{god pagu16}{} 
\num{16} \eng{god of love} perxVmadeVvate; kAmadeVva; manamxtha; (\girxVpu) IrAsf, (\roVpu) kUyxpiDf. 
\num{17} \eng{god of the sea} samudarxdeVvate; (\girxVpu) posiVDAnf, (\roVpu) nepUcxnf. 
\num{18} \eng{god of wine} madayxdeVvate; (\girxVpu) bAyxkasf. 
\num{19} \eng{God's image} manuSayxdeVha; mAnava shariVra. 
\num{20} \eng{God's (own) country} BUsavxgaR; deVvadeVsha; (BUloVkada savxgaRveMdu parigaNisalapxTaTx) amerikA deVsha. 
\hypertarget{god pagu21}{} 
\num{21} \eng{God's plenty} (\AmA) samaqdidhx; dhaMDi. 
\num{22} \eng{God's quantity} = \hyperlink{god pagu21}{?pagu? \((21)\)}. 
\hyperdef{G}{god(1) pagu(23)}{} 
\num{23} \eng{God's truth} parama satayx. 
\num{24} \eng{God the Father, Son, Holy Ghost} deVvaru, deVvaputarx (kirxsatx), divAyxtamx -- eMba kerxYsatx tirxmUtiRgaLu. 
\num{25} \eng{God willing} deYveVceCx idadxre; BagavatasxMkalapxvidadxre; deYvasaMkalapxvidadxre; deVvara citatxvidadxre; deYvAnugarxhavidadxre. 
\num{26} \eng{God wot} (\pArxparx) = \hyperlink{god pagu12}{?pagu? \((12)\)}. 
\num{27} \eng{in God's} \hyperref{kandict_n.pdf}{N}{name(1) pagu(11)}{name}. 
\num{28} \hyperref{kandict_m.pdf}{M}{man(1) pagu(11)}{man of God.} 
\num{29} \eng{oh, my good, etc. god!} (noVvu, duHKa, koVpa -- sUcisuva udAgxragaLu) ayoyxV nananx deVvareV! hA deVvare! 
\num{30} \eng{play God} savaRshakatxnAgalu parxyatinxsu; deVvaraMte vatiRsatoDagu, ADatoDagu. 
\num{31} \eng{so help me God!} (pArxthaRneyalilx yA nAnu BASege tapupxvudilalx, nAnu satayxvanunx ADutitxdedxVne eMdu parxmANa mADuvalilx) deVvaru sahAya mADali! deVvaru naDesikoDali! deVvaru neraveVrisali! 
\num{32} \eng{thank God} (GaTane \mo vugaLu anukUlakaravAdudakekx mAtina madheyx mADuva saMtoVSada udAgxra) elalx deVvara kaqpe! badukiside deVvareV! deVvaru doDaDxvanu! 
\num{33} \eng{under God} (manuSayxna asAvxtaMtarxyXvanunx yA pUNaR sAvxtaMtarxyXkekx mitiyanunx tiLisuvAga heVLuva) deYvAdhiVna; BagavadadhiVna. 
\num{34} \eng{with God} satutx savxgaRdalilx; deVvara sAyujayxdalilx, sAmIpayxdalilx. 
\enum
\emng

\noindent
\gl{\nuga}
\bmng
\bnum
\num{1} \eng{feast for the gods} = \hyperlink{gods nuga3}{?nuga? \((3)\)}. 
\num{2} \eng{on the knees} \engit{or} \eng{in the lap of the gods} (manuSayxna adhiVnadalilxlalxde) deVvara keYyalilx; BagavaMtana adhiVnadalilx. 
\hypertarget{gods nuga3}{} 
\num{3} \eng{sight for the gods} divayxvAdadudx; deVvayoVgayx; deVvaBoVgayx; atayxMta manoVjacnxvAdadudx; savoRVtakxqqSaTxvAdadudx. 
\num{4} \eng{Ye gods! (and little fishes)!} (viDaMbanegAgi mADuva aNaka udAgxragaLu) deVvarugaLeV! 
\enum
\emng
\eentry

\bentry
\word[god(2)]{god}
\pron{gADf}
\gl{\sakirx}
\bmng
(\viparx) deVvarAgisu; deYvatavx kalipxsu; deYva padavigeVrisu; ArAdhisu. 
\emng

\noindent
\gl{\pagu}
\bmng
 \eng{god it} deVvaraMte vatiRsu, ADu. 
\emng
\eentry

\bentry
\word{God-awful}
\pron{gADfAphulf}
\gl{\gu}
\bmng
 (\ashi) ati BayaMkaravAda; tuMba asahayxvAda; tiVra ahitakaravAda: \eng{your affairs are in a god-awful mess} ninanx viSayagaLelalx BayaMkaravAda goMdaladalilxve. 
\emng
\eentry

\bentry
\word{godchild}
\pron{gADfceYlfDx}
\gl{\nA}
\bmng
 dhamaRshishu; dhamaRpitaniMda jAcnxnasAnxnakekx opipxsida gaMDu yA heNuNx magu. 
\emng
\eentry

\bentry
\word{God-dam(n)}
\pron{gADfDAYxmf}
\gl{\gu}
\bmng
  = \hyperlink{God-damned}{God-damned}. 
\emng
\eentry

\bentry
\word{God-damned}
\pron{gADfDAYxmfDx}
\gl{\gu}
\bmng
 hALu; hALAda; aniSaTx; daridarx: \eng{I'm so sick of their God-damned faces} nanage avara hALu muKagaLanunx noVDi ciTuTx hiDiduhoVgide. 
\emng
\eentry

\bentry
\word{god-daughter}
\pron{gADfDATarf}
\gl{\nA}
\bmng
 dhamaRputirx; dhamaRpitaniMda jAcnxnasAnxnakekx opipxsida heNuNx magu. 
\emng
\eentry

\bentry
\word{goddess}
\pron{gADisf}
\gl{\nA}
\bmng
\bnum
\num{1} deVvi; sitxrXVdeVvate. 
\num{2} deVvi; nalelx; pirxyatame; atayxMta mecicxnavaLu. 
\enum
\emng

\noindent
\gl{\pagu}
\bmng
\bnum
\num{1} \eng{goddess of corn} dhAnayx deVvate; (\roVpu) siVriVsfZ. 
\num{2} \eng{god of heaven} deVvarANi; savxgaRdeVvi; (\girxVpu) hiVrA; (\roVpu) jUno. 
\num{3} \eng{god of hell} naraka deVvate; (\girxVpu) porxVsapaRniV. 
\num{4} \eng{god of love} perxVmadeVvate; (\roVpu) viVnasf. 
\num{5} \eng{god of moon} caMdarxdeVvate; (\roVpu) DayAna. 
\num{6} \eng{god of wisdom} jAcnxnadeVvate; (\roVpu) minavaR. 
\enum
\emng
\eentry

\bentry
\word{godet}
\pron{goVDeV}
\gl{\nA}
\bmng
 oLavasatxrX; uDupu, keYgavasu, \mo vugaLoLakekx seVrisida tirxkoVNAkArada baTeTx, vasatxrX. 
\emng
\eentry

\bentry
\word{godetia}
\pron{gaDiVSa}
\gl{\nA}
\bmng
 goVDiVSa (sasayx); yatheVcaCxvAgi hUvu biDuva, oMdu vASiRka sasayx. 
\emng
\eentry

\bentry
\word{go-devil}
\pron{goVDevilf}
\gl{\nA}
\bmng
(\ame) goVDevilf; kaTiTxkoMDiruva peYpugaLa oLaBAgavanunx shucigoLisalu baLasuva sAdhana. 
\emng
\eentry

\bentry
\word[godfather(1)]{godfather}
\pron{gADfphAdarf}
\gl{\nA}
\bmng
\bnum
\num{1} dhamaRpita; dhamaRtaMde; maguvanunx jAcnxnasAnxnakekx opipxsi, adara paravAgi dhAmiRka vidhiya parxshenxgaLige utatxra niVDi, adara dhAmiRka shikaSxNada hoNe hotatx gaMDasu. 
\num{2} (\ame) (\rUpa) nAmadAta; hesaru koTaTxvanu; vayxkitxge, vasutxvige, yAra hesariTiTxdeyo avanu. 
\num{3} (\rUpa) hitasAdhaka; hitapoVSaka; sherxVysAsxdhaka; yAradeV sherxVyasisxge muKayx kAraNanAdavanu. 
\num{4} kAnUnubAhira saMsethxyanunx naDesutitxruvavanu. 
\enum
\emng

\noindent
\gl{\pagu}
\bmng
 \eng{my godfathers!} (\sw) BagavaMta! ayoyxV deVvareV! 
\emng
\eentry

\bentry
\word[godfather(2)]{godfather}
\pron{gADfphAdarf}
\gl{\sakirx}
\bmng
\bnum
\num{1} (yAvudeV oMdara) javAbAdxri vahisiko; hoNe hotutxko. 
\num{2} (yAvudeV oMdakekx) tananx hesaru -- niVDu, koDu. 
\num{3} (yAvudeV maguvige) dhamaRpitanAgu; dhamaRpitanAgi naDeduko. 
\enum
\emng
\eentry

\bentry
\word{god-fearing}
\pron{gADfphiariMgf}
\gl{\gu}
\bmng
deYvaBiVru; dhamaRBiVru; deYvaBiVtiyuLaLx; pArxmANikavAda dhamaRsharxdedhxyuLaLx, deYvaBakitxyuLaLx. 
\emng
\eentry

\bentry
\word{God-forsaken}
\pron{gADfphaseRVkanf}
\gl{\gu}
\bmng
 guNaleVshavilalxda; anAtha; dikikxlalxda; hALubidadx; hALu suriyuva; daridarx; yAvudeV oLeLxya aMshavU ilalxda. 
\emng

\noindent
\gl{\pagu}
\bmng
\bnum
\num{1} \eng{God-forsaken inhabitants} dikikxlalxda nivAsigaLu. 
\num{2} \eng{what a god-farsaken hole!} eMthA hALukoMpe! 
\enum
\emng
\eentry

\bentry
\word{godhead}
\pron{gADfheDf}
\gl{\nA}
\bmng
\bnum
\num{1} deVvarAgiruvudu; deVvateyAgiruvudu. 
\num{2} deVvatavx; deYvikate; deVva -- savxBAva, parxkaqti. 
\num{3} deYva; deVvate. 
\enum
\emng

\noindent
\gl{\pagu}
\bmng
 \eng{the Godhead} BagavaMta; parameVshavxra. 
\emng
\eentry

\bentry
\word{godhood}
\pron{gADfhuDf}
\gl{\nA}
\bmng
 deYvaguNa; deYvasithxti; deVvatavx; divayxte. 
\emng
\eentry

\bentry
\word{godless}
\pron{gADfli(le)sf}
\gl{\gu}
\bmng
\bnum
\num{1} deVvarilalxda; deVvarahita; deYvahiVna. 
\num{2} deVvaranunx opapxda, naMbada; niriVshavxravAdi; nAsitxka. 
\num{3} dhamaRrahita; adhAmiRka. 
\num{4} duSaTx; duruLa; niVca. 
\enum
\emng
\eentry

\bentry
\word{godlessness}
\pron{gADfli(le)sfnisf}
\gl{\nA}
\bmng
\bnum
\num{1} deVvarahitate; deVvarilalxdiruvike. 
\num{2} deYvanirAkaraNa; niriVshavxrate; nAsitxkate; deVvaranunx -- opapxdiruvudu, naMbadiruvudu. 
\num{3} AdhAmiRkate; dhamaRrahitate. 
\num{4} duSaTxte; duruLate; niVcatana. 
\enum
\emng
\eentry

\bentry
\word{godlike}
\pron{gADfleYkf}
\gl{\gu}
\bmng
\bnum
\num{1} deVvasadaqsha; deVvaraMtha; deVvateyaMtha; yAvudAdarU guNadalilx deVvaranunx, oMdu deVvateyanunx hoVluva. 
\num{2} deVvayoVgayx; deVvoVcita. 
\num{3} deVvasamAna. 
\enum
\emng
\eentry

\bentry
\word{godliness}
\pron{gADflinisf}
\gl{\nA}
\bmng
 dhamaRsharxdedhx; deYvaBakitx; Asitxka budidhx; dhAmiRkate. 
\emng
\eentry

\bentry
\word{godly}
\pron{gADfli}
\gl{\gu}
\bmng
 dhAmiRka; dhamaRniSaThx; deYvaBakitxyuLaLx; dhamaRsharxdedhxyuLaLx; Asitxka budidhxya. 
\emng
\eentry

\bentry
\word{godmother}
\pron{gADfmadarf}
\gl{\nA}
\bmng
 dhamaRmAte; dhamaRtAyi; kerxYsatxralilx maguvanunx jAcnxnasAnxnakekx opipxsi, adara paravAgi dhAmiRka parxshenxgaLige utatxra koDuva matutx adara dhAmiRka beLavaNigeya hoNe hotatx heMgasu. 
\emng
\eentry

\bentry
\word{godown}
\pron{goVDwnf}
\gl{\nA}
\bmng
 goVdAmu; maLige; gaDaMgu; koVThi; ugArxNa; pUvaR ESAyxdalilx, \kanmu iMDiyadalilx, sarakugaLanunx saMgarxhisiDuva mane. 
\emng
\eentry

\bentry
\word{godparent}
\pron{gADfpeVraMTf}
\gl{\nA}
\bmng
 dhamaRpita yA dhamaRmAte; kerxYsatxralilx maguvanunx jAcnxnasAnxnakekx opipxsi, adara paravAgi dhAmiRka parxshenxgaLige utatxra koTuTx maguvina dhAmiRka shikaSxNada hoNe hotatx gaMDasu yA heMgasu. 
\emng
\eentry

\bentry
\wordnospeech{God's Acre}{God's Acre}
\pron{?}
\gl{\nA}
\bmng
 (cacfR AvaraNadalilxruva) samAdhi BUmi. 
\emng
\eentry

\bentry
\wordnospeech{God's book}{God's book}
\pron{?}
\gl{\nA}
\bmng
 beYbalf (garxMtha). 
\emng
\eentry

\bentry
\wordnospeech{God's earth}{God's earth}
\pron{?}
\gl{\nA}
\bmng
 iDiV BUmi; samasatx BUmaMDala; saMpUNaR BUloVka. 
\emng
\eentry

\bentry
\word{godsend}
\pron{gADfseMDf}
\gl{\nA}
\bmng
 deYvadatatx; aniriVkiSxta adaqSaTx yA GaTane; aniriVkiSxta saMpAdane. lABa; deYva karuNisidudu: \eng{the rain after the drought was a godsent} jalakASxmada naMtara baMda maLe deYvadatatxvAdudu. 
\emng
\eentry

\bentry
\wordnospeech{God's gift}{God's gift}
\pron{?}
\gl{\nA}
\bmng
 deVvara vara; deYvadatatxvAdudu; deYva karuNisidudu. 
\emng
\eentry

\bentry
\word{godship}
\pron{gADfSipf}
\gl{\nA}
\bmng
 (\kanmu\ \hA) deVvatavx; deVvapadavi; deVva vayxkitxtavx. 
\emng
\eentry

\bentry
\word{godson}
\pron{gADfsanf}
\gl{\nA}
\bmng
 dhamaRputarx; dhamaRpitaniMda dhamaRsAnxnakekx opipxsida gaMDu magu. 
\emng
\eentry

\bentry
\word{Godspeed}
\pron{gADfsipxVDf}
\gl{\nA}
\bmng
 shuBa; yashasusx; swBAgayx; jaya; keSxVma; biVLogxLuLxva vayxkitxge koVruva `deVvaru yashasusx, swBAgayx karuNisali' eMba shuBAshaya. 
\emng
\eentry

\bentry
\word[godward(1)]{godward}
\pron{gADfvaDfR}
\gl{\kirxvi}
\bmng
\bnum
\num{1} deVvABimuKavAgi; deVvara kaDege. 
\num{2} deVvara viSayadalilx, saMbaMdhadalilx. 
\enum
\emng
\eentry

\bentry
\word[godward(2)]{godward}
\pron{gADfvaDfR}
\gl{\gu}
\bmng
deVvABimuKavAda; deVvara kaDeya; deVvara kaDe olida, tirugida. 
\emng
\eentry

\bentry
\word{godwards}
\pron{gADfvaDfs'R}
\gl{\kirxvi}
\bmng
  = \hyperlink{godward(1)}{$^1$godward}. 
\emng
\eentry

\bentry
\word{godwit}
\pron{gADfviTf}
\gl{\nA}
\bmng
 koMcehakikx tarahada, Adare savxlapx meVlakekx bAgida kokukxLaLx, niVra hakikx. 
\emng
\eentry

\bentry
\word{Godwottery}
\pron{gADfvATari}
\gl{\nA}
\bmng
 udAyxnAsakitx; toVTAsakitx; \kanmu\ toVTagArikeyalilx yA toVTagaLanunx noVDikoLuLxva bagegx atiyAda Asakitxvahisuvudu yA AsakitxvahisidaMte naTisuvudu. 
\emng
\eentry

\bentry
\word{goer}
\pron{goVarf}
\gl{\nA}
\bmng
\bnum
\num{1} hoVguvavanu yA hoVguvaMtha, hoVguva vasutx: \eng{good goer} curukAgi hoVguvava; shiVGarxgAmi. \eng{slow goer} nidhAnavAgi hoVguvava; maMdagAmi. 
\num{2} (oMdu kaDege) hoVguvavanu: \eng{church-goer} caciRge hoVguvavanu. \eng{theatre-goer} nATakamaMdirakekx hoVguvavanu. 
\num{3} lavalavikeyiMda, kaSaTxpaTuTx yA sevxVcACxcAradiMda vatiRsuva vayxkitx: \eng{a `banger' is a goer-a girl who'll do anything with anyone} `bAyxMgarf' eMdare sevxVcACxcAri -- yAra jote Enu beVkAdarU mADuvaMtha huDugi. 
\enum
\emng
\eentry

\bentry
\word{goes}
\pron{goVsfZ}
\gl{\kirx}
\bmng
\bnum
\num{1} \eng{go} kirxyApadada parxthama puruSa \Eva. 
\num{2} \eng{go} nAmapadada \bava. 
\enum
\emng
\eentry

\bentry
\word{goest}
\pron{goVisfTx}
\gl{\kirx}
\bmng
 (\pArxparx) \eng{go} kirxyApadada madhayxmapuruSa \Eva. 
\emng
\eentry

\bentry
\word{goeth}
\pron{goVitf}
\gl{\kirx}
\bmng
 (\pArxparx) \eng{go} kirxyApadada parxthama puruSa \Eva. 
\emng
\eentry

\bentry
\word[Goethean(1)]{Goethean}
\pron{gaTiRanf}
\gl{\gu}
\bmng
  = \hyperlink{Goethian(1)}{$^1$Goethian}. 
\emng
\eentry

\bentry
\word[Goethean(2)]{Goethean}
\pron{gaTiRanf}
\gl{\nA}
\bmng
  = \hyperlink{Goethian(2)}{$^2$Goethian}. 
\emng
\eentry

\bentry
\word[Goethian(1)]{Goethian}
\pron{gaTiRanf}
\gl{\gu}
\bmng
 (jamaRnf kavi) gayaTeya yA gayaTeyaMtha; gayaTeya leVKanagaLa yA leVKanagaLaMtha; gayaTeya aBipArxyagaLa yA aBipArxyagaLaMtha. 
\emng
\eentry

\bentry
\word[Goethian(2)]{Goethian}
\pron{gaTiRanf}
\gl{\nA}
\bmng
 (jamaRnf kavi) gayaTeya -- anuyAyi, mecucxga, aBimAni. 
\emng
\eentry

\bentry
\word[goffer(1)]{goffer}
\pron{goV(gA)pharf}
\gl{\sakirx}
\bmng
 (kalAbatu aMcu, jari alaMkAra, \mo vanunx kAyisida kabibxNadiMda otitx) nirige mADu; teretere mADu. 
\emng
\eentry

\bentry
\word[goffer(2)]{goffer}
\pron{goV(gA)pharf}
\gl{\nA}
\bmng
\bnum
\num{1} nirigeyotutx; nirigeyotatxlu baLasuva sAdhana. 
\num{2} (nirige \mo vakekx baLasuva) alaMkArada heNige. 
\enum
\emng
\eentry

\bentry
\word{goffered}
\pron{goV(gA)phaDfR}
\gl{\gu}
\bmng
 (pusatxkada raTiTxna aMcina \vi) otitx ubabxlaMkAra mADida. 
\emng
\eentry

\bentry
\word{go-getter}
\pron{goVgeTarf}
\gl{\nA}
\bmng
 (\ame) (\AmA) 
\bnum
\num{1} kAyaRsAdhaka; tAnu paDeyabayasuvudanunx paDeyuvava; toDagidadxnunx sAdhisuvavanu. 
\num{2} CAtivaMta; sAhasi; muMde bidudx hoVguvava; nugigx hoVguvavanu. 
\enum
\emng
\eentry

\bentry
\word[goggle(1)]{goggle}
\pron{gAgflf}
\gl{\sakirx}
\bmng
 (kaNuNxgaLanunx) pakakxkekx, oMdu kaDege, oMdu kaDeyiMda inonxMdu kaDege -- tirugisu, horaLisu, mAlisu. 
\emng

\noindent
\gl{\akirx}
\bmng
\bnum
\num{1} OreyAgi noVDu. 
\num{2} kaNuNx tirugisu, sutitxsu, horaLisu. 
\numi{3} (kaNuNxgaLa \vi) 
\banum
\alnum{a} sutatx tirugu; horaLu. 
\alnum{b} muMdakekx cAcu. 
\eanum
\numie
\num{4} (agalavAgi tereda kaNuNxgaLiMda) diTiTxsi noVDu; kaNuNx ubibxsi noVDu; kaNaNxraLisi noVDu. 
\enum
\emng
\eentry

\bentry
\word[goggle(2)]{goggle}
\pron{gAgflf}
\gl{\gu}
\bmng
 (kaNuNxgaLa \vi) 
\banum
\alnum{a} ubibxda; muMcAcida; ubibxkoMDiruva. 
\alnum{b} pUtiR terediruva, araLiruva. 
\alnum{c} sututxtitxruva; sutatx tirugutitxruva; horaLutitxruva. 
\eanum
\emng
\eentry

\bentry
\word[goggle(3)]{goggle}
\pron{gAgflf}
\gl{\nA}
\bmng
\bnum
\num{1} (\bava dalilx) gAgalusx; bisilu kananxDaka; taMpu kananxDaka; bisilina JaLa, dhULu, \mo vugaLiMda kaNuNxgaLanunx rakiSxsalu upayoVgisuva, bahumaTiTxge baNaNxda gAju, taMti jAlari, \mo vugaLuLaLx kananxDaka. \imglink{gogglesfigure}{\raisebox{-0.20cm}[0pt][0pt]{\pdfimage width 0.6cm height 0.6cm{G_Pictures/goggles.jpg}}} 
\num{2} kuriroVga; tatatxra beVne; giriki roVga; tatatxrisuva roVga. 
\enum
\emng
\eentry

\bentry
\word{goggle-box}
\pron{gAgflfbAkfsx}
\gl{\nA}
\bmng
 (\ashi) dUradashaRnada seTuTx, peTiTxge. 
\emng
\eentry

\bentry
\word{goggle-eyed}
\pron{gAgflfaiDf}
\gl{\gu}
\bmng
 ububxgaNiNxna; birugaNiNxna; araLugaNiNxna; horaLugaNiNxna; (\kanmu\ AshacxyaRdiMda) pUtiR araLida yA horaLuva kaNuNxgaLuLaLx. 
\emng
\eentry

\bentry
\word{goglet}
\pron{gAgilxTf}
\gl{\nA}
\bmng
 (BArata) maNuNxhUji; niVranunx taNaNxgiDalu, \sA\ maNiNxniMda mADida, udadx katitxna, hUji. 
\emng
\eentry

\bentry
\word{go-go}
\pron{goVgoV}
\gl{\gu}
\bmng
 (\AmA) 
\bnum
\num{1} savxcaCxMdada; sevxVcACxcArada; kaTuTxkaTaTxLeyilalxda. 
\num{2} curukAda; utAsxha tuMbida. 
\num{3} (sATxkf ekfsxceVMjina haNa hUDikeyalilx) saTeTxya; (belegaLa haNa hUDikeyalilx) saTeTxya; belegaLa tiVvarxvAda EriLitagaLiMda apAra lABa gaLisalu sATxku, SeVru, \mo vugaLanunx koLuLxva -- mAruva vayxvahArada. 
\num{4} (nataRka, nataRki, saMgiVta, \mo vugaLa \vi) birusina; curukina; hurupu, calana, OjasusxgaLiMda -- kUDida. 
\enum
\emng
\eentry

\bentry
\word{Goidel}
\pron{gAyaDxlf}
\gl{\nA}
\bmng
gAyaDxlf; kelfTx janAMgagaLige seVridava. 
\emng
\eentry

\bentry
\word[Goidelic(1)]{Goidelic}
\pron{gAyaDxlikf}
\gl{\gu}
\bmng
gAyfDalilxna; kelfTx janAMgagaLa; kelfTx janAMgagaLige saMbaMdhapaTaTx. 
\emng
\eentry

\bentry
\word[Goidelic(2)]{Goidelic}
\pron{gAyaDxlikf}
\gl{\nA}
\bmng
 gAyfDalalxra nuDi; kelfTx janAMgagaLa BASe. 
\emng
\eentry

\bentry
\word[going(1)]{going}
\pron{goViMgf}
\gl{\nA}
\bmng
\bnum
\num{1} hoVguvudu; gamana; horaDuvudu; teraLuvudu; naDeyuvudu; calisuvudu. 
\num{2} nigaRmana; hoVguvudu; parxyANa (mADuvudu). 
\num{3} parxgati; muMduvarike; munanxDe. 
\num{4} (naDage, kudure savAri, \kanmu\ kudure reVsige saMbaMdhisidaMte) nelada hada; hAdiya sithxti; dAriya sithxti: \eng{the going was bad} dAri cenAnxgiralilalx. 
\num{5} (gaMTeya) hoDeta. 
\num{6} (siDimadidxna) siDita(da shabadx). 
\num{7} (kelasada yA parxyANada) veVga yA vidhAna. \eng{70 miles an hour is good going} gaMTege \eng{70} meYligaLu oLeLxya veVga. 
\enum
\emng

\noindent
\gl{\pagu}
\bmng
 \eng{comings and goings} Agamana matutx nigaRmana; baruvudu matutx hoVguvudu; AyA matutx gayA (\rUpa\ saha): \eng{the comings and goings in the corridors of power} adhikArada naDavegaLalilxna AyA rAmf gayA rAmfgaLu. 
\emng

\noindent
\gl{\nuga}
\bmng
\bnum
\numi{1} \eng{going away} 
\banum
\alnum{a} nigaRmana; parxyANa; horaTu hoVguvudu. 
\alnum{b} madhucaMdarxkekx horaDuvudu. 
\eanum
\numie
\num{2} \eng{heavy going} nidhAnavAda yA kaSaTxsAdhayxvAda parxgati. 
\num{3} \eng{while the going is good} parisithxti anukUlavAgiruvAga; anukUla parisithxtiyalilx. 
\enum
\emng
\eentry

\bentry
\word[going(2)]{going}
\pron{goViMgf}
\gl{\gu}
\bmng
\bnum
\num{1} (yaMtarx, saMsethx, \mo vugaLa \vi) naDeyutitxruva; kelasadalilxruva; kAyaR mADutitxruva: \eng{set the clock going} gaDiyAra naDeyutitxruvaMte mADu. \eng{a going concern} cAlitx(yalilxruva) saMsethx; kelasa naDeyutitxruva, naDesutitxruva (udoyxVga) saMsethx. 
\num{2} iruva; siguva; doreyuva: \eng{one of the best fellows going} iruva atayxMta sherxVSaThxralilx obabxnu. \eng{there is cold beef going} doreyuva taNaNxneya goVmAMsa. 
\num{3} cAlitxyalilxruva; naDeyutitxruva; sadayxda: \eng{the going rate} sadayxda dara. 
\num{4} badukiruva; caTuvaTikeyiMdiruva. 
\num{5} horaTu hoVgutitxruva; nigaRmisutitxruva. 
\enum
\emng

\noindent
\gl{\pagu}
\bmng
\bnum
\num{1} \eng{get going} oMdeV samane, sithxravAgi (mAtanADalu, kelasa mADalu, ADalu, \mo vanunx) -- shurumADu, pArxraMBisu. 
\num{2} \eng{going for one} (\AmA) obabxra paravAgi mADutitxruva; oMdu pakaSx vahisi kelasa nivaRhisutitxruva. 
\num{3} \eng{going (on) fifteen} hadineYdaneya huTuTxhababx samIpisutitxruva; hadineYdu vaSaR tuMbutitxruva, Agaliruva. 
\num{4} \eng{going on for} (oMdu kAla, vayasusx, \mo vanunx) samIpisutitxruva: \eng{I shall be sixteen, going on for seventeen} nanage hadinAru tuMbi hadineVLu hatitxravAgutitxde. 
\numi{5} \eng{going to} 
\banum
\alnum{a} inenxVnu, sadayxdalelxV (mADaliruva, ADaliruva itAyxdi). 
\alnum{b} udedxVshisuva yA udedxVshisida. 
\alnum{c} bahushaH, pArxyashaH -- Agaliruva; saMBavaniVya. 
\eanum
\numie
\numi{6} \eng{to be going on with} 
\banum
\alnum{a} modalige; AraMBadalilx. 
\alnum{b} sadayxkekx. 
\eanum
\numie
\enum
\emng

\noindent
\gl{\nuga}
\bmng
\bnum
\num{1} \eng{going great} \hyperlink{gun(1) nuga(6)}{$^1$guns}. 
\num{2} \eng{going} \hyperref{kandict_s.pdf}{S}{strong(2) pagu(2)}{$^2$strong}. 
\enum
\emng
\eentry

\bentry
\word{going-over}
\pron{goViMgfOvarf}
\gl{\nA}
\bmng
\bnum
\num{1} (\AmA) pariVkeSx; parishiVlane. 
\num{2} (\ashi) cenAnxgi hoDeyuvudu, cacucxvudu. 
\num{3} (\ame) (\AmA) bayuyxvudu. 
\enum
\emng
\eentry

\bentry
\word{goings-on}
\pron{goViMgfs'Anf}
\gl{\nA}
\bmng
\bnum
\num{1} (\sA\ vicitarxvAda, aMtha, \mo) naDavaLike; vataRne. 
\num{2} GaTanegaLu; vidayxmAnagaLu; saMBavisuva, jaruguva -- saMgatigaLu. 
\enum
\emng
\eentry

\bentry
\word{goiter}
\pron{gAyaTxrf}
\gl{\nA}
\bmng
 (\ame)  = \hyperlink{goitre}{goitre}. 
\emng
\eentry

\bentry
\word{goitered}
\pron{gAyaTxDfR}
\gl{\gu}
\bmng
 (\ame)  = \hyperlink{goitred}{goitred}. 
\emng
\eentry

\bentry
\word{goitre}
\pron{gAyaTarf}
\gl{\nA}
\bmng
 (\birx) (\roVshA) gAyaTxru; gaLagaMDa; gaMTaluvALa; teYrAyfDx garxMthiya, aneVka veVLe katitxniMda joVlADuva, Uta. 
\emng
\eentry

\bentry
\word{goitred}
\pron{gAyaTxDfR}
\gl{\gu}
\bmng
 gAyfTarf, gaLagaMDa -- Ada; gaLagaMDa baMda. 
\emng
\eentry

\bentry
\word{goitrous}
\pron{gAiTarxsf}
\gl{\gu}
\bmng
\bnum
\num{1}  = \hyperlink{goitred}{goitred}. 
\num{2} gAyfTarfnaMtha; gaLagaMDadaMtha. 
\num{3} gAyfTarfna; gaLagaMDada. 
\num{4} (sathxLagaLa \vi) gAyfTarf haraDiruva; gaLagaMDa roVga taTiTxruva, taguliruva. 
\enum
\emng
\eentry

\bentry
\word{go-kart}
\pron{goVkATfR}
\gl{\nA}
\bmng
 (saNaNx cAsiyuLaLx, nAlukx cakarxgaLuLaLx, eraDu soTxrXVkina aMtadaRhana yaMtarxvuLaLx) kiru reVsfkAru. 
\emng
\eentry

\bentry
\word{Golconda}
\pron{gAlAkxMDa}
\gl{\nA}
\bmng
 goVlokxMDa; aishavxyaRda gaNi (\rUpa\ saha) (BAratada heYdarAbAdina baLiya nagaradiMda baMda \parx). 
\emng
\eentry

\bentry
\word[gold(1)]{gold}
\pron{goVlfDx}
\gl{\nA}
\bmng
\bnum
\num{1} cinanx; honunx; baMgAra; savxNaR; suvaNaR; heVma; kanaka; kAMcana. 
\num{2} cinanxda nANayxgaLu. 
\num{3} aishavxyaR; shirxVmaMtike; (dhana) saMpatutx; heVraLa haNa; apAra dhana. 
\num{4} savxNaRleVpa; cinanxda -- mulAmu, reVku, baNaNx; mulAmu mADuvudakAkxgi yA baNaNxvAgi upayoVgisuva cinanx. 
\num{5} hoMbaNaNx; baMgArada baNaNx. 
\num{6} (\rUpa) cinanx; baMgAra; savxNaR; aparaMji; parxshasatx vasutx; ujavxla, suMdara, amUlayx, belebALuva -- vasutxgaLu, vasatxrXgaLu, sAmagirx: \eng{a heart of gold} cinanxdaMtha haqdaya; udAra haqdayada vayxkitx. \eng{a voice of gold} honanxkaMTha; shirxVmaMtavAda kaMTha, dhavxni. \eng{she is pure gold} avaLu shudadhx aparaMji(yaMtha vayxkitx). 
\num{7} (bilulx videyx) hoMgurigaNuNx; savxNaRlakaSxyX; hoLeyuva (\sA\ giliVTu mADiruva) gurigaNuNx. 
\num{8}  = \hyperlink{gold medal}{gold medal}. 
\enum
\emng

\noindent
\gl{\pagu}
\bmng
 \eng{age of gold} = \hyperlink{golden age}{golden age}. 
\emng

\noindent
\gl{\nuga}
\bmng
\bnum
\num{1} \eng{all that glisters} (\engit{or} \eng{glitters) is not gold} hoLeyuvudelalx honanxlalx; beLaLxgiruvudelalx hAlalalx. 
\num{2} \eng{go off gold} suvaNaR parxmitiyanunx, suvaNaRmAnavanunx -- tore, tayxjisu, keYbiDu. 
\numi{3} \eng{worth one's weight in gold} 
\banum
\alnum{a} bahaLa belebALuva; atayxmUlayxvAda. 
\alnum{b} atuyxpayukatxvAda; bahaLa sahAyakavAda. 
\eanum
\numie
\enum
\emng
\eentry

\bentry
\word[gold(2)]{gold}
\pron{goVlfDx}
\gl{\gu}
\bmng
\bnum
\num{1} cinanxda; honinxna; suvaNaRda; pUtiRyAgi yA bahumaTiTxge cinanxdiMda mADida. 
\num{2} hoMbaNaNxda; savxNaRvaNaRda; cinanxdaMtha baNaNxvuLaLx. 
\num{3} (tagigxda beleya nANayx calAvaNeyalilx) savxNaRmAnada beleyalilx, savxNaRmAnakekx samanAda beleyalilx lekakx mADida: \eng{gold francs etc.} (bele tagagxde idAdxgina) savxNaRmAnada beleya phArxyXMkf nANayxgaLu \mo vu; cinanxda PArxMkugaLu \mo vu. 
\enum
\emng
\eentry

\bentry
\wordnospeech{gold amalgam}{gold amalgam}
\pron{?}
\gl{\nA}
\bmng
 rasabaMgAra; pAdarasa matutx cinanxgaLa misharxNa. 
\emng
\eentry

\bentry
\word[goldarn(1)]{goldarn}
\pron{gAlADxnfR}
\gl{\sakirx}
\bmng
 (\ame, \ashi) = \hyperref{kandict_d.pdf}{D}{damn(1)2}{$^1$damn (2)}. 
\emng
\eentry

\bentry
\word[goldarn(2)]{goldarn}
\pron{gAlADxnfR}
\gl{\gu}
\bmng
 (\ame, \ashi) = \hyperref{kandict_d.pdf}{D}{damned(1)2}{$^1$damned (2)}. 
\emng
\eentry

\bentry
\word[goldarn(3)]{goldarn}
\pron{gAlADxnfR}
\gl{\kirxvi}
\bmng
 (\ame, \ashi) = \hyperref{kandict_d.pdf}{D}{damned(2)}{$^2$damned}. 
\emng
\eentry

\bentry
\word{gold-beater}
\pron{goVlfDxbiVTarf}
\gl{\nA}
\bmng
 savxNaRreVkugAra; cinanxda reVku baDiyuvavanu. 
\emng
\eentry

\bentry
\wordnospeech{gold-beater's skin}{gold-beater's skin}
\pron{?}
\gl{\nA}
\bmng
\bnum
\num{1} cinanxda reVku camaR; cinanxda reVku baDiyuvAga reVkugaLanunx beVpaRDisalu baLasuva teLu camaR. 
\num{2} saNaNx gAyagaLanunx mucacxlu baLasuva camaRda pore. 
\enum
\emng
\eentry

\bentry
\wordnospeech{gold bloc}{gold bloc}
\pron{?}
\gl{\nA}
\bmng
 suvaNaRbaNa; suvaNaR parxmitiyanunx AdhAravAgiTuTxkoMDa nANayx vayxvasethxgaLuLaLx rASaTxrXgaLa baNa, kUTa. 
\emng
\eentry

\bentry
\wordnospeech{gold brick}{gold brick}
\pron{?}
\gl{\nA}
\bmng
 (\ashi) 
\bnum
\num{1} (horakekx mAtarx beleyuLaLxdAdxgi kANuva) (bariya) thaLukina vasutx. 
\num{2} (\ame) soVmAri; alasa. 
\enum
\emng
\eentry

\bentry
\word{goldcrest}
\pron{goVlfDxkerxsfTx}
\gl{\nA}
\bmng
 hoMjuTiTxga; savxNaRshiKa; hoMbaNaNxda juTuTxLaLx, oMdu ati saNaNx hakikx. 
\emng
\eentry

\bentry
\word{gold-digger}
\pron{goVlfDxDigarf}
\gl{\nA}
\bmng
\bnum
\num{1} baMgAra bedaka; savxNaRshoVdhaka; cinanxda shoVdhanegAgi Ageyuvava. 
\num{2} (\ame) (\ashi) honunx suliyuvavaLu; gaMDasaranunx maruLugoLisi haNa suliyuva olapugAti. 
\enum
\emng
\eentry

\bentry
\word{gold-dust}
\pron{goVlfDxDasfTx}
\gl{\nA}
\bmng
 honunxhuDi; hoMdhULu; savxNaRdhULi; cinanxda -- dhULu, puDi; aneVka veVLe sUkaSxmXkaNagaLa rUpadalilx dorakuva cinanx. 
\emng
\eentry

\bentry
\word{golden}
\pron{goVlaDxnf}
\gl{\gu}
\bmng
\bnum
\num{1} cinanxda; honinxna; suvaNaRda; cinanxdiMda mADida. 
\num{2} cinanx tuMbiruva, heVraLavAgiruva; suvaNaR -- vishiSaTx. 
\num{3} cinanx koDuva; savxNaRda; suvaNaRdAyi. 
\num{4} hoMbaNaNxda; cinanxda baNaNxda; savxNaRvaNaRda; cinanxdaMte hoLeyuva: \eng{golden hair} hoMbaNaNxda kUdalu. 
\num{5} bahaLa beleyuLaLx; amUlayx; parxshasatx; utakxqqSaTx; amoVGa; atuyxtatxma; sherxVSaThx; parxdhAna; parxmuKa; mahatavxvuLaLx: \eng{a golden remedy} saMjiVvini; paramwSadha; atuyxtatxma parihAra. \eng{a golden opportunity} suvaNARvakAsha; utakxqqSaTx avakAsha. \eng{a golden saying} cinanxdaMtha mAtu; mahatavxda ukitx; amoVGa vacana. 
\enum
\emng
\eentry

\bentry
\wordnospeech{golden age}{golden age}
\pron{?}
\gl{\nA}
\bmng
 suvaNaRyuga; cinanxda yuga: 
\banum
\alnum{a} cinanx, beLiLx, hitAtxLe, kabibxNa eMba nAlukx yugagaLalilx modalaneyadAda, manuSayxrelalx sadA saMtoVSaBaritarU pApavananxriyadavarU AgidadxreMdu BAvisiruva kAla, avadhi, yuga. 
\alnum{b} (rASaTxrXda, sAhitayxda) ati ucACxrXyakAla; ELigeya kAla; samaqdidhx kAla. 
\eanum
\emng
\eentry

\bentry
\word{golden-ager}
\pron{goVlaDxnfEjarf}
\gl{\nA}
\bmng
 (\ame) apara vayasakx; muduka; vayasAsxdava; vaqdadhx. 
\emng
\eentry

\bentry
\wordnospeech{golden balls}{golden balls}
\pron{?}
\gl{\nA}
\bmng
 mUru honanx ceMDugaLu; suvaNaR kaMdukatarxya (giravidArana saMkeVta). 
\emng
\eentry

\bentry
\wordnospeech{golden boy}{golden boy}
\pron{?}
\gl{\nA}
\bmng
 honanxhuDuga; cinanxda huDuga; savxNaRbAlaka; janapirxyanAda yA yashasusx gaLisida -- huDuga yA vayxkitx. 
\emng
\eentry

\bentry
\wordnospeech{golden chain}{golden chain}
\pron{?}
\gl{\nA}
\bmng
 = \hyperref{kandict_l.pdf}{L}{laburnum}{laburnum}. 
\emng
\eentry

\bentry
\wordnospeech{golden disc}{golden disc}
\pron{?}
\gl{\nA}
\bmng
 cinanxda taTeTx; savxNaRPalaka; aidu lakaSx gArxmaphoVnf taTeTxgaLu KacARda meVle, adanunx hADida vayxkitxge koDuva koDuge, parxshasitx. 
\emng
\eentry

\bentry
\wordnospeech{golden eagle}{golden eagle}
\pron{?}
\gl{\nA}
\bmng
 cinanxda garuDa; suvaNaR gaqdharx; hoMbaNaNxda talejuTuTxLaLx, doDaDx hadudx. 
\emng
\eentry

\bentry
\word{golden-eye}
\pron{goVlaDxnfai}
\gl{\nA}
\bmng
 hoMbAtu; bUyxsiphala kulakekx seVrida, oMdu kaDala bAtu. 
\emng
\eentry

\bentry
\wordnospeech{Golden Fleece}{Golden Fleece}
\pron{?}
\gl{\nA}
\bmng
 AsiTxrXya matutx sepxVnf deVshagaLa `neYTf' padaviya dajeR. 
\emng
\eentry

\bentry
\wordnospeech{golden girl}{golden girl}
\pron{?}
\gl{\nA}
\bmng
 honanxhuDugi; cinanxda huDugi; savxNaRbAle; janapirxyaLAda yA yashasusx gaLisida huDugi yA heMgasu. 
\emng
\eentry

\bentry
\wordnospeech{golden hamster}{golden hamster}
\pron{?}
\gl{\nA}
\bmng
 parxyoVgAlayadalilx yA maneyalilx sAkida, doDaDx iliyaMtha, kaMdu baNaNxda pArxNi. 
\emng
\eentry

\bentry
\wordnospeech{golden handshake}{golden handshake}
\pron{?}
\gl{\nA}
\bmng
 cinanxda -- keYkuluku, hasatxlAGava; suvaNaR hasatxlAGava; seVveyiMda vajA mADidAga ilalxve kaDADxya nivaqtitx mADidAga koDuva parihAra dhana. 
\emng
\eentry

\bentry
\wordnospeech{Golden Horde}{Golden Horde}
\pron{?}
\gl{\nA}
\bmng
 \eng{13}neV shatamAnadalilx pUvaR yUroVpina meVle dALi mADi AkarxmisikoMDa TATaRrara taMDa. 
\emng
\eentry

\bentry
\wordnospeech{Golden Horn}{Golden Horn}
\pron{?}
\gl{\nA}
\bmng
 goVlaDxnf hAnfR; tukiR deVshada isAtxMbulina baMdaru. 
\emng
\eentry

\bentry
\wordnospeech{golden jubilee}{golden jubilee}
\pron{?}
\gl{\nA}
\bmng
 cinanxda hababx; honanx hababx; suvaNaR mahoVtasxva: 
\banum
\alnum{a} rAjana paTATXBiSeVkada \eng{50}neV vASiRkoVtasxva. 
\alnum{b} yAvudeV GaTane, kaTaTxDa, \mo vugaLa \eng{50}neV vASiRkoVtasxva. 
\eanum
\emng
\eentry

\bentry
\wordnospeech{golden key}{golden key}
\pron{?}
\gl{\nA}
\bmng
 honanxkiVli; cinanxda kiVli, biVgada keY; yAvudeV kelasakekx odaguva aDacaNeyanunx nivArisalu baLasuva haNa. 
\emng
\eentry

\bentry
\word{golden-knop}
\pron{goVlaDxnfnApf}
\gl{\nA}
\bmng
 (\birx) saNaNx jiVruMDe. 
\emng
\eentry

\bentry
\wordnospeech{Golden legend}{Golden legend}
\pron{?}
\gl{\nA}
\bmng
 honanxkate; cinanxda kate; suvaNaR purANa; \eng{13}neV shatamAnada saMtara jiVvana cariterxgaLa oMdu saMgarxha(da hesaru). 
\emng
\eentry

\bentry
\wordnospeech{golden mean}{golden mean}
\pron{?}
\gl{\nA}
\bmng
\bnum
\num{1} suvaNaR mAdhayxma; cinanxda mAdhayxma; ati hecUcx alalxda, ati kaDimeyU alalxda -- madhayxmAgaR, hALata sUtarx, naDunele. 
\num{2}  = \hyperlink{golden section}{golden section}. 
\enum
\emng
\eentry

\bentry
\word{golden-mouthed}
\pron{goVlaDxnfmwdfDx}
\gl{\gu}
\bmng
vAgfJariya; vAgevxYKariyuLaLx; oLeLxya mAtugArikeya; vAgimxyAda; vAkfpaTutavxda. 
\emng
\eentry

\bentry
\wordnospeech{golden number}{golden number}
\pron{?}
\gl{\nA}
\bmng
 suvaNaRsaMKeyx; IsaTxrf dinavanunx gotutx mADalu muKayxveMdu parigaNisiruva, meTAnikf cAMdarxmAna cakarxdalilx vaSaRda karxmasaMKeyx. 
\emng
\eentry

\bentry
\wordnospeech{golden opinions}{golden opinions}
\pron{?}
\gl{\nA}
\bmng
 apAra mecucxge, gwrava; bahaLa oLeLxya aBipArxya. 
\emng
\eentry

\bentry
\wordnospeech{golden perch}{golden perch}
\pron{?}
\gl{\nA}
\bmng
 (\AseTxrXV) tinanxlu baLasuva, oMdu bageya sihiniVru mInu. 
\emng
\eentry

\bentry
\wordnospeech{golden rain}{golden rain}
\pron{?}
\gl{\nA}
\bmng
honanxmaLe; suvaNaRvaSaR; savxNaRdhAre; maLebANa; oMdu bageya bANa birusu. 
\emng
\eentry

\bentry
\wordnospeech{golden retriever}{golden retriever}
\pron{?}
\gl{\nA}
\bmng
 hoMbaNaNxda huDuku nAyi; hoMbaNaNxda meYgUdaluLaLx, koMda athavA gAyagoMDa pArxNi athavA pakiSxyanunx huDuki tegedukoMDu baruva nAyi. 
\emng
\eentry

\bentry
\wordnospeech{golden rod}{golden rod}
\pron{?}
\gl{\nA}
\bmng
 hoMgoVlu (sasayx); suvaNaR daMDa; koVlinaMtaha daMTU hoLeva hoMbaNaNxda hUgoMcalU uLaLx, sAyxliDeVgoV kulada sasayx. 
\emng
\eentry

\bentry
\wordnospeech{golden rule}{golden rule}
\pron{?}
\gl{\nA}
\bmng
\bnum
\num{1} `tananxMteyeV pararanUnx kANu' eMba beYbalilxna sUtarx (mAyxthUyx \eng{vii. 12}). 
\num{2} honanx sUtarx; suvaNaR sUtarx; yAvudeV kelasa \mo vugaLa mUla sUtarx, tatatxvX: \eng{the golden rule for eating and drinking is moderation} ananxpAnAdigaLa bagegx mUla sUtarxveMdare miti. 
\enum
\emng
\eentry

\bentry
\wordnospeech{golden section}{golden section}
\pron{?}
\gl{\nA}
\bmng
 honanxCeVda; suvaNaR CeVda; saraLareVKeya eraDu BAgagaLalilx cikakx BAga matutx oTuTx reVKeya guNalabadhxvu doDaDx BAgada vagaRkekx samavAgiruvaMte reVKeyanunx katatxrisuvudu. 
\emng
\eentry

\bentry
\wordnospeech{Golden State}{Golden State}
\pron{?}
\gl{\nA}
\bmng
 (\ame) kAyxliphoVniRya saMsAthxna. 
\emng
\eentry

\bentry
\wordnospeech{golden syrup}{golden syrup}
\pron{?}
\gl{\nA}
\bmng
 (\birx) hoMbaNaNxda kAkaMbiya (yA kabibxna hAlina pAkada) vANijayx nAma. 
\emng
\eentry

\bentry
\wordnospeech{golden wedding}{golden wedding}
\pron{?}
\gl{\nA}
\bmng
 vivAhada suvaNaR mahoVtasxva; maduveya cinanxda hababx; maduveya aivatatxneya vaSaRda utasxva. 
\emng
\eentry

\bentry
\word{gold-fever}
\pron{goVlfDxphiVvarf}
\gl{\nA}
\bmng
 honanx hucucx; savxNaR javxra; cinanxvanunx huDukikoMDu hoVguva hucucx, giVLu. 
\emng
\eentry

\bentry
\word{gold-field}
\pron{goVlfDxphiVlfDx}
\gl{\nA}
\bmng
 savxNaRkeSxVtarx; cinanxda parxdeVsha; cinanx doreyuva pArxMta. 
\emng
\eentry

\bentry
\word{goldfinch}
\pron{goVlfDxphiMcf}
\gl{\nA}
\bmng
 goVlfDxphiMcf; honanx hakikx; savxNaR pakiSx; hoLeyuva baNaNxda, haLadi macecxya rekekxyuLaLx, hADu hakikx. 
\emng
\eentry

\bentry
\word{goldfish}
\pron{goVlfDxphiSf}
\gl{\nA}
\bmng
 homImxnu; honanx mInu; suvaNaRmatasxyX; alaMkArakAkxgi koLagaLalilx, gAjupAterxgaLalilx iTuTx sAkuva, ciVnA deVshada, keMpu baNaNxda sihi niVru mInu. 
\emng

\noindent
\gl{\nuga}
\bmng
 \eng{goldfish bowl} EkAMtateyilalxda sathxLa yA saninxveVsha. 
\emng
\eentry

\bentry
\wordnospeech{gold foil}{gold foil}
\pron{?}
\gl{\nA}
\bmng
 savxlapx dapapxvAda cinanxda reVku, tagaDu. 
\emng
\eentry

\bentry
\word{goldilocks}
\pron{goVliDxlAkfsx}
\gl{\nA}
\bmng
\bnum
\num{1} hoMgUdalina vayxkitx; suvaNaRkeVshi. 
\num{2} oMdu bageya kAkapAda (sasayx). 
\num{3} hoMgoMcala hUgiDa; suvaNaRpuSapxgucaCx; kaMpAsiTiV vaMshakekx seVrida, koVlinaMtaha daMTU hoLeva hoMbaNaNxda hUgoMcalU uLaLx sasayx. 
\enum
\emng
\eentry

\bentry
\wordnospeech{gold leaf}{gold leaf}
\pron{?}
\gl{\nA}
\bmng
 cinanxda reVku, varaku; suvaNaRreVku; caMcapu reVku; cinanxda vatiRreVku; cinanxda teLuvAda tagaDu. 
\emng
\eentry

\bentry
\wordnospeech{gold medal}{gold medal}
\pron{?}
\gl{\nA}
\bmng
 (modalaneya bahumAnavAgi koDuva) cinanxda padaka; savxNaRpadaka. 
\emng
\eentry

\bentry
\word{gold-mine}
\pron{goVlfDxmeYnf}
\gl{\nA}
\bmng
\bnum
\num{1} cinanxda gaNi. 
\num{2} (\rUpa) saMpanUmxla; saMpatitxna nele; aishavxyaRda gaNi. 
\enum
\emng
\eentry

\bentry
\wordnospeech{gold of pleasure}{gold of pleasure}
\pron{?}
\gl{\nA}
\bmng
 saNaNx saNaNx haLadi hU biDuva, oMdu bageya vASiRka sasayx. 
\emng
\eentry

\bentry
\wordspecial{gold plate}{1}{1}{\hyperlink{gold-plate(1)}{\quad\textcolor{superscript}{$^2$}\eng{gold-plate}}}
\pron{?}
\gl{\nA}
\bmng
\bnum
\num{1} cinanxda pAterxgaLu; savxNaR pAterxgaLu. 
\num{2} cinanxda mulAmu hAkida sAmagirx; savxNaRleVpada padAthaR. 
\enum
\emng
\eentry

\bentry
\wordspecial{gold-plate}{1}{2}{\hyperlink{gold plate(1)}{\quad\textcolor{superscript}{$^1$}\eng{gold plate}}}
\pron{goVlfDxpelxVTf}
\gl{\sakirx}
\bmng
 cinanxda mulAmu hAku; savxNaRleVpa mADu. 
\emng
\eentry

\bentry
\wordnospeech{gold reserve}{gold reserve}
\pron{?}
\gl{\nA}
\bmng
 cinanxda dAsAtxnu; savxNaR saMgarxha; keVMdirxVya bAyxMku \mo vugaLa sheVKaraNeyalilxruva cinanxda nANayxgaLa yA gaTiTxgaLa dAsAtxnu. 
\emng
\eentry

\bentry
\word{gold-rush}
\pron{goVlfDxraSf}
\gl{\nA}
\bmng
 honunxgagxlu; hosa cinanxda parxdeVshagaLigAgi nugAgxTa, BAri saMKeyxyalilx valase hoVguvudu. 
\emng
\eentry

\bentry
\word{goldsmith}
\pron{goVlfDxsimxtf}
\gl{\nA}
\bmng
 akakxsAliga; patAtxra; soVnagAra; vAja(ravanu); savxNaRkAra; cinanxda kelasa mADuvavanu. 
\emng
\eentry

\bentry
\wordnospeech{goldsmith beetle}{goldsmith beetle}
\pron{?}
\gl{\nA}
\bmng
 soVnAra jiVruMDe; hoMbaNaNxda rekekx musukuLaLx jiVruMDe. 
\emng
\eentry

\bentry
\wordnospeech{gold standard}{gold standard}
\pron{?}
\gl{\nA}
\bmng
 (\athaRshA) suvaNaRmAna; nidiRSaTx tUkada cinanxvanunx nANayxda mUlamAnaveMdu parigaNisalAgiruva nANayx padadhxti. 
\emng
\eentry

\bentry
\wordnospeech{Gold Stick}{Gold Stick}
\pron{?}
\gl{\nA}
\bmng
\bnum
\num{1} (iMgelxMDinalilx) honanxgoVlu; savxNaRdaMDa; rASiTxrXVya samAraMBagaLalilx `leYphfgADfsxR'na yA `jeMTalfmanf-aTf-AmfsxR'na kAyxpaTxnf hiDidukoLuLxva, cinanxda koVlu. 
\num{2} honanxgoVlugAra; savxNaRdaMDadhAra; iMgelxMDinalilx rASiTxrXVya samAraMBagaLalilx cinanxda (mulAmina) daMDavanunx hiDidiruva adhikAri. 
\enum
\emng
\eentry

\bentry
\wordspecial{gold-thread}{1}{1}{\hyperlink{gold thread(1)}{\quad\textcolor{superscript}{$^2$}\eng{gold thread}}}
\pron{goVlfDxterxDf}
\gl{\nA}
\bmng
\bnum
\num{1} kApiTxsf kulada, teLuvAda haLadi beVruLaLx oMdu giDa. 
\num{2} auSadhige baLasuva I giDada beVru. 
\enum
\emng
\eentry

\bentry
\wordspecial{gold thread}{1}{2}{\hyperlink{gold-thread(1)}{\quad\textcolor{superscript}{$^1$}\eng{gold-thread}}}
\pron{?}
\gl{\nA}
\bmng
 jari; jaratAri; sutatxlU cinanxda taMti sutitxda, reVSemx \mo vugaLa nUlu, eLe. 
\emng
\eentry

\bentry
\word{golem}
\pron{goVlemf}
\gl{\nA}
\bmng
\bnum
\num{1} (yehUdayx purANadalilxna) boMbe manuSayx; jeVDimaNuNx \mo vugaLiMda mADi, alwkikavAgi jiVva tuMbida manuSAyxkaqti. 
\num{2} roVboVTf; yaMtarxmAnava. 
\enum
\emng
\eentry

\bentry
\word[golf(1)]{golf}
\pron{gAlfphx}
\gl{\nA}
\bmng
 gAlfphx (ATa); gAlfphx meYdAnada nAnA bageya sAvxBAvika yA kaqtaka aDacaNegaLanunx dATisi gAlfphx ceMDanunx beVre beVre dUradalilxruva \eng{9} yA \eng{18} badudxgaLanunx oMdAda meVle oMdaraMte AdaSuTx kaDime dAMDuhoDetagaLiMda tuMbuva, ibabxru yA eraDu joteyavaru ADuva oMdu bageya ceMDATa. 
\emng
\eentry

\bentry
\word[golf(2)]{golf}
\pron{gAlfphx}
\gl{\akirx}
\bmng
 gAlfphx (ATa) ADu. 
\emng
\eentry

\bentry
\word{golf-bag}
\pron{gAlfphxbAyxgf}
\gl{\nA}
\bmng
 gAlfphx ciVla; gAlfphx ATada dAMDu, ceMDugaLanunx hotutxkoMDu hoVgalu baLasuva ciVla.  \imglink{golf-bagfigure}{\raisebox{-0.15cm}[0pt][0pt]{\pdfimage width 0.6cm height 0.7cm{G_Pictures/golf-bag.jpg}}} 
\emng
\eentry

\bentry
\wordnospeech{golf ball}{golf ball}
\pron{?}
\gl{\nA}
\bmng
 gAlfphx ceMDu: 
\banum
\alnum{a} gAlfphx ATadalilx baLasuva ceMDu. 
\alnum{b} (\AmA) kelavu bageya viduyxtf beraLacucx yaMtarxgaLalilx akaSxragaLanunx peVparina meVlotatxlu baLasuva cikakx guMDu. 
\eanum
\emng
\eentry

\bentry
\wordnospeech{golf cart}{golf cart}
\pron{?}
\gl{\nA}
\bmng
 gAlfphxgADi: 
\banum
\alnum{a} gAlfphx dAMDugaLanunx oyayxlu baLasuva TArxli, taLuLxgADi. 
\alnum{b} gAlfphx ATagAraranUnx avara sAmAnu saraMjAmugaLanUnx sAgisalu baLasuva, moVTAru gADi. 
\eanum
\emng
\eentry

\bentry
\word{golf-club}
\pron{gAlfphxkalxbf}
\gl{\nA}
\bmng
\bnum
\num{1} gAlfphx dAMDu; gAlfphx ceMDanunx hoDeyuva dAMDu. 
\numi{2} gAlfphx kalxbubx: 
\banum
\alnum{a} gAlfphx (ATagArara) saMGa. 
\alnum{b} gAlfphx (ATagArara) saMGada sathxLa. 
\eanum
\numie
\enum
\emng
\eentry

\bentry
\word{golf-course}
\pron{gAlfphxkoVsfR}
\gl{\nA}
\bmng
  = \hyperlink{golf-links}{golf-links}. 
\emng
\eentry

\bentry
\word{golfer}
\pron{gAlaphxrf}
\gl{\nA}
\bmng
\bnum
\num{1} gAlfphx ATagAra. 
\num{2} = \hyperref{kandict_c.pdf}{C}{cardigan}{cardigan}. 
\enum
\emng
\eentry

\bentry
\word{golf-links}
\pron{gAlfphx liMkfsx}
\gl{\nA}
\bmng
 gAlfphx (ATada) meYdAna. 
\emng
\eentry

\bentry
\wordnospeech{golf widow}{golf widow}
\pron{?}
\gl{\nA}
\bmng
 gAlfphx vidhave; virAma kAlada hecucx BAgavanenxlAlx gAlfphx ATadalelxV kaLeyuvavana heMDati. 
\emng
\eentry

\bentry
\wordRemoveSpace{Golgi-apparatus}{Golgi apparatus}
\pron{gAlijx AYxpareVTasf}
\gl{\nA}
\bmng
 (\jiVvi) gAlijx parikara; jiVvakoVshadalilx naDeyuva saMshelxVSaNe matutx satxvana kAyaRgaLalilx pAlogxLuLxvudeMdu naMbalAgiruva, koVshadarxvadalilx kANabaruva vishiSaTx racanegaLu. 
\emng
\eentry

\bentry
\word{Goliath}
\pron{galeYatf}
\gl{\nA}
\bmng
\bnum
\num{1} rAkaSxsa; deYtayx. 
\hypertarget{Goliath(2)}{} 
\num{2} deYtayx etutxga; balavAda saMcAriV etutxyaMtarx. 
\hypertarget{Goliath(3)}{} 
\num{3} deYtayx kwrxMca; AphirxkAda doDaDx heranf pakiSx, kwrxMcapakiSx. 
\enum
\emng
\eentry

\bentry
\wordnospeech{Goliath beetle}{Goliath beetle}
\pron{?}
\gl{\nA}
\bmng
 deYtayx jiVruMDe; biLi paTeTxyuLaLx, AphirxkAda bahaLa doDaDx, kapupx jiVruMDe. 
\emng
\eentry

\bentry
\wordnospeech{Goliath crane}{Goliath crane}
\pron{?}
\gl{\nA}
\bmng
  = \hyperlink{Goliath(2)}{Goliath (2)}. 
\emng
\eentry

\bentry
\wordnospeech{Goliath heron}{Goliath heron}
\pron{?}
\gl{\nA}
\bmng
  = \hyperlink{Goliath(3)}{Goliath (3)}. 
\emng
\eentry

\bentry
\word{golliwog}
\pron{gAlivAgf}
\gl{\nA}
\bmng
\bnum
\num{1} vikAra boMbe; vikaTAkArada, guMguru kUdalina, \sA\ kapupx boMbe. 
\num{2} gumamx; becacx; gogagxyayx; becaRpapx. 
\num{3} vikAra vayxkitx. 
\enum
\emng
\eentry

\bentry
\word[gollop(1)]{gollop}
\pron{gAlapf}
\gl{\sakirx}
\bmng
 (\AmA) gabagabane tinunx; durAseyiMda yA AturAturavAgi kabaLisu, tinunx. 
\emng
\eentry

\bentry
\word[gollop(2)]{gollop}
\pron{gAlapf}
\gl{\nA}
\bmng
 (\AmA) AturAturavAda yA durAseya kabaLike, nuMgike. 
\emng
\eentry

\bentry
\word[golly(1)]{golly}
\pron{gAli}
\gl{\BAavayx}
\bmng
 (\kanmu\ niVgorxV jana baLasuva pada) deVvareV! deVvarANe! 
\emng
\eentry

\bentry
\word[golly(2)]{golly}
\pron{gAli}
\gl{\nA}
\bmng
  = \hyperlink{golliwog}{golliwog}(na \saMkiSx). 
\emng
\eentry

\bentry
\word{golosh}
\pron{galASf}
\gl{\nA}
\bmng
 \eng{galosh} padada rUpAMtara. 
\emng
\eentry

\bentry
\word{goluptious}
\pron{galapaSxsf}
\gl{\gu}
\bmng
 (\hA) 
\bnum
\num{1} rasavatAtxda; rucikara; bahu ruciyAda; bahaLa sihiyAda: \eng{goluptious draught} bahuruciyAda pAniVya. 
\num{2} AnaMdakara; saMtoVSakara; AmoVdakara. 
\enum
\emng
\eentry

\bentry
\wordnospeech{GOM}{GOM}
\pron{?}
\gl{\saMkiSx}
\bmng
 \eng{Grand Old Man.} 
\emng
\eentry

\bentry
\word{gombeen}
\pron{gAMbinf}
\gl{\nA}
\bmng
 (aileRMDf \parx) kusiVda; dubAri baDiDx; ati hecucx darada meVle sAla koDuvudu. 
\emng
\eentry

\bentry
\word{gombeen-man}
\pron{gAMbiVnfmAYxnf}
\gl{\nA}
\bmng
 baDiDxkoVra; kusiVdaka; dubAri baDiDxge sAla koDuvavanu. 
\emng
\eentry

\bentry
\word{gombroon}
\pron{gAMbUrxnf}
\gl{\nA}
\bmng
 (iMgelxMDina celisxV paTaTxNada maNiNxna pAterxgaLalilx anukarisiruva) paSiRyAda -- kuMbArike, kuMbAra sAmAnu, kuDikemaDike. 
\emng
\eentry

\bentry
\word{gomroon}
\pron{gAMrUnf}
\gl{\nA}
\bmng
  = \hyperlink{gombroon}{gombroon}. 
\emng
\eentry

\bentry
\word{Gomorrah}
\pron{gamAra}
\gl{\nA}
\bmng
\bnum
\num{1} keTUTxru; duSaTx paTaTxNa; niVcanagari. 
\num{2} duSaTx paTaTxNada mAdari. 
\enum
\emng
\eentry

\bentry
\wordwithhyphen{hyp-gon}{-gon}
\pron{-ganf}
\gl{\uparx}
\bmng
 -koVna; nidiRSaTx saMKeyxya koVnagaLiruva samatalAkaqtigaLanunx sUcisuva (\mo\ padagaLanunx racisalu baLasuva) \uparx: \eng{hexagon, polygon, n-gon.} 
\emng
\eentry

\bentry
\word{gonad}
\pron{goVnAYxDf}
\gl{\nA}
\bmng
 goVnAYxDf; janana garxMthi; aMDagaLanunx utapxtitxmADuva aMDAshaya, yA reVtANugaLanunx utapxtitx mADuva vaqSaNa. 
\emng
\eentry

\bentry
\word{gonadal}
\pron{goVneVDalf}
\gl{\gu}
\bmng
 goVnAYxDfna; janana garxMthiya yA janana garxMthige saMbaMdhisida. 
\emng
\eentry

\bentry
\word{gonadotrophic}
\pron{goVnaDaTArxphikf}
\gl{\gu}
\bmng
 (\shavi) goVnAYxDf parxcoVdaka; janana garxMthi parxcoVdaka; janana garxMthiyu kelasa mADuvaMte parxcoVdisuva. 
\emng
\eentry

\bentry
\word{gonadotropic}
\pron{goVnaDaTorxV(TArx)pikf}
\gl{\gu}
\bmng
  = \hyperlink{gonadotrophic}{gonadotrophic}. 
\emng
\eentry

\bentry
\word{gondola}
\pron{gAMDala}
\gl{\nA}
\bmng
\bnum
\num{1} gAMDola; (iTaliya venisfna niVru kAluvegaLalilx upayoVgisuva) hagura doVNi; naDuve koVNeyuLaLx, eraDu tudigaLU etatxravAgidudx hiMBAgada oMdu huTiTxniMda naDesuva, venisf nagarada kAluvegaLalilx saMcArakAkxgi upayoVgisuva, capapxTe taLada hagura doVNi.  \imglink{gondola-1figure}{\raisebox{-0.15cm}[0pt][0pt]{\pdfimage width 0.7cm height 0.5cm{G_Pictures/gondola-1.jpg}}} 
\num{2} (vAyunwkeyiMda iLibiTiTxruva) tUgudoTiTxlu. 
\hypertarget{gondola(3)}{} 
\num{3} tereda vAyxganunx; maTaTxsa taLada, meVlABxga mucicxlalxda, ukukx \mo vugaLanunx sAgisalu baLasuva reYlevx vAyxganunx. 
\num{4} calisutitxruva hagagx yA keVbalilxge aLavaDisiruva, eraDu athavA nAlukx jana kUruva vayxvasethxyiruva, himagalulx jArugaranunx koMDoyuyxva, buTiTx yA gADi. 
\num{5} (savxyaMseVvA aMgaDiyalilx vasutxgaLanunx parxdashiRsuvudakAkxgi EpaRDisiruva) biDuvina aMkaNa(gaLu). 
\enum
\emng
\eentry

\bentry
\wordnospeech{gondola car}{gondola car}
\pron{?}
\gl{\nA}
\bmng
 (\ame)  = \hyperlink{gondola(3)}{gondola (3)}. 
\emng
\eentry

\bentry
\word{gondolier}
\pron{gAMDaliarf}
\gl{\nA}
\bmng
 gAMDoliga; gAMDola (doVNiyanunx) naDesuvavanu, huTuTx hAkuvavanu. 
\emng
\eentry

\bentry
\word{gone}
\pron{gAnf}
\gl{\gu}
\bmng
\bnum
\num{1} agalida; hoVda; gata; nigaRta. 
\num{2} hALAda; keYtapipxda; Ase biTaTx; naSaTxvAda; keTuTxhoVda: \eng{a gone man} bALu hALu mADikoMDa manuSayx; pUtiR keTuTx hoVdavanu. \eng{a gone case} tiVra keTuTxhoVda vayxvahAra; keYtapipxda vayxvahAra. 
\hypertarget{gone(3)}{} 
\num{3} satutx hoVda; gatisida; maqta: \eng{limbs of the gone wretch} satatx pApiya avayavagaLu. 
\num{4} (kAlada \vi) kaLeda; Agi kaLedu hoVda; saMda; gata: \eng{sweet memories of gone springs} kaLeda vasaMtagaLa savi nenapugaLu. \eng{not until gone nine} oMbatutx gaMTe kaLeyuva muMce alalxde. 
\num{5} nitArxNada; nishayxkitxya; dubaRlavAda: \eng{a gone feeling} nishayxkitxya BAva. 
\num{6} susAtxda; baLalida; AyAsagoMDa. 
\hypertarget{gone(7)}{} 
\num{7} (\AmA) basirAda; gaBiRNiyAda: \eng{a woman seven months gone} ELu tiMgaLa gaBiRNi(yAda heMgasu). 
\num{8} (\AmA) tAtAkxlikavAgi hAjarilalxda, kANadAda. 
\enum
\emng

\noindent
\gl{\pagu}
\bmng
\bnum
\numi{1} \eng{be gone} 
\banum
\alnum{a} tolagAce! naDeyAce! tolagi hoVgu! 
\alnum{b} (\AmA) tAtAkxlikavAgi geYruhAjarAgiru. 
\eanum
\numie
\num{2} \eng{gone with child} = \hyperlink{gone(7)}{gone (7)}. 
\num{3} \eng{dead and gone} = \hyperlink{gone(3)}{gone (3)}. 
\num{4} \eng{gone away}! (nariya beVTeyalilx nariyu biladiMda horaTideyeMbudanunx sUcisuva udAgxra) horaTide! 
\numi{5} \eng{far gone} 
\banum
\alnum{a} bahaLa muMduvareda; bahaLa dUra hoVda; tiVra toDagisikoMDa; sikikxkoMDa. 
\alnum{b} AyAsagoMDa; susAtxda; nitArxNagoMDa. 
\alnum{c} sAyutitxruva; maraNa samIpisutitxruva. 
\hyperdef{G}{gone(1) pagu(6)}{} 
\eanum
\numie
\num{6} \eng{gone on} (\ashi) vAyxmoVhadiMdiruva; haMbala hacicxkoMDa; giVLu hiDida; tiVvArxsakatxnAgiruva; moVhakekx, perxVmakekx silukida: \eng{he is still gone on the girl who jilted him} tananxnunx vaMcisida huDugiya moVhakekx avanu inUnx silukidAdxne. 
\num{7} \eng{past and gone} gatisida; kaLeduhoVda; BUtakAlada. 
\enum
\emng
\eentry

\bentry
\wordnospeech{gone goose}{gone goose}
\pron{?}
\gl{\nA}
\bmng
  = \hyperlink{gone gosling}{gone gosling}. 
\emng
\eentry

\bentry
\wordnospeech{gone gosling}{gone gosling}
\pron{?}
\gl{\nA}
\bmng
 Ase keYbiTaTx vayxkitx yA vasutx. 
\emng
\eentry

\bentry
\word{goner}
\pron{gAnarf}
\gl{\nA}
\bmng
 (\ashi) Ase keYbiTaTx yA keTuTx hALAda -- vayxkitx, vasutx; pUtiR hALAda yA naSaTxvAda yA muLugihoVda -- vayxkitx, vasutx. 
\emng
\eentry

\bentry
\word{gonfalon}
\pron{gAnfphalanf}
\gl{\nA}
\bmng
\bnum
\num{1} paTiTxbAvuTa; aneVka veVLe aDaDxpaTiTxyiMda iLiyabiTaTx, hArADuva paTiTxgaLiMda kUDida bAvuTa, dhavxja, patAke. 
\num{2} (\ca) iTaliya parxjAparxButavxgaLa dhavxja. 
\enum
\emng
\eentry

\bentry
\word{gonfalonier}
\pron{gAnfphalaniarf}
\gl{\nA}
\bmng
\bnum
\num{1} paTiTx bAvuTagAra; dhavxjavAhaka; dAMguDiga; bAvuTa hiDiyuvavanu. 
\num{2} (\ca) (iTaliya kelavu parxjAparxButavxgaLalilx) parxdhAna nAyxyAdhipati; muKayx daMDAdhikAri. 
\enum
\emng
\eentry

\bentry
\word[gong(1)]{gong}
\pron{gAMgf}
\gl{\nA}
\bmng
\bnum
\num{1} jAgaTe; kaMsALa. 
\num{2} baTaTxlu gaMTe; boVguNi gaMTe; saNaNx boVguNi AkArada gaMTe. 
\num{3} (\ashi) bilelx; padaka. 
\enum
\emng
\eentry

\bentry
\word[gong(2)]{gong}
\pron{gAMgf}
\gl{\sakirx}
\bmng
\bnum
\num{1} (vAhanasaMcArada poliVsinavana \vi) jAgaTe hoDedu (moVTAru cAlakanige) nilulxvaMte sUcisu. 
\num{2} jAgaTe bArisi (vayxkitxyanunx) baruvaMte apapxNe mADu, karesu. 
\enum
\emng
\eentry

\bentry
\word{gongorism}
\pron{gAMgarisaZmf}
\gl{\nA}
\bmng
 gAMgarasheYli; vipayaRyagaLu, viroVdhoVkitx, pArxciVna kaqtigaLa sUcanegaLiMda kUDida, iMgelxMDina yUphUisaZmfge saMvAdiyAda, sepxVnina sAhitayx sheYli. 
\emng
\eentry

\bentry
\word{goniometer}
\pron{goVniAmiTarf}
\gl{\nA}
\bmng
 koVnamApaka; (moVjiNi, Kanija vijAcnxna, \mo vugaLalilx) koVnagaLanunx aLeyuva upakaraNa. 
\emng
\eentry

\bentry
\word{goniometric}
\pron{goVniameTirxkf}
\gl{\gu}
\bmng
 koVnamApanada; koVnagaLanunx aLeyuvudakekx saMbaMdhisida. 
\emng
\eentry

\bentry
\word{goniometrical}
\pron{goVniameTirxkalf}
\gl{\gu}
\bmng
  = \hyperlink{goniometric}{goniometric}. 
\emng
\eentry

\bentry
\word{goniometry}
\pron{goVniAmiTirx}
\gl{\nA}
\bmng
 koVnamApana; koVnagaLanunx aLeyuvudu. 
\emng
\eentry

\bentry
\word{gonna}
\pron{gAna}
\gl{(\asaM; yA amerikanf \AmA)}
\bmng
 \eng{going to} enunxvudara ADumAtina yA asaMsakxqqta \parx. 
\emng
\eentry

\bentry
\word{gonococcus}
\pron{gAnakAkasf}
\gl{\nA}
\expl{(\bava\ \eng{gonococci} \ucAcx\ gAnakAkeY).}
\bmng
 gAnakAkasf; shukalx meVhada bAyxkiTxVriya; gonoriya roVgakekx kAraNavAda kAkasf vagaRda bAyxkiTxVriya. 
\emng
\eentry

\bentry
\word{gonorrhea}
\pron{gAnariVa}
\gl{\nA}
\bmng
 (\ame)  = \hyperlink{gonorrhoea}{gonorrhoea}. 
\emng
\eentry

\bentry
\word{gonorrhoea}
\pron{gAnariVa}
\gl{\nA}
\bmng
 (\roVshA) gonoVriya; shukalxmeVha roVga; shukalx doVSa; huNANxda mUtarxnALa yA yoVniyiMda uritada kiVvu suriyuva oMdu aMTu meVharoVga. 
\emng
\eentry

\bentry
\word{gonorrhoeal}
\pron{gAnariValf}
\gl{\gu}
\bmng
\bnum
\num{1} shukalxmeVha roVgada yA adara saMbaMdhada. 
\num{2} shukalxmeVha roVgagarxsatx; shukalxmeVha tagulida. 
\enum
\emng
\eentry

\bentry
\word{goo}
\pron{gU}
\gl{\nA}
\bmng
 (\ashi) 
\bnum
\num{1} aMTu; aMTaMTAda, jiguTAda darxvayx. 
\num{2} (\rUpa) asahayx barisuva BAva; aMTaMTu BAva. 
\enum
\emng
\eentry

\bentry
\word[good(1)]{good}
\pron{guDf}
\gl{\gu}
\expl{(\tara\ \eng{better}, \tama\ \eng{best}). }
\bmng
\bnum
\num{1} cenAnxda; oLeLxya; utatxma; taqpitxkaravAda; samapaRkavAda; hadavAda; takakx; ucita; yoVgayx; apeVkiSxta -- guNagaLuLaLx, lakaSxNagaLuLaLx: \eng{a good fire} hadavAda beMki; ati saNaNxdAgali, kiSxVNavAgali alalxda uri. \eng{good health} utatxma AroVgayx; niVroVgate; roVgavilalxdiruvike. \eng{meat keeps good} mAMsa keDade oLeLxya sithxtiyalilxrutatxde. \eng{good soil} oLeLxya nela; PalavatAtxda jamInu. \eng{good theatre} utatxma raMga; nidiRSaTx mAdhayxmakekx opupxva manaraMjane \mo vu. 
\num{2} takakxSuTx; sAkaSuTx; sAkAguvaSuTx; apeVkiSxtavAdaSuTx: \eng{a good amount} sAkAguvaSuTx parxmANa. 
\num{3} (bariya sAMparxdAyika visheVSaNavAgi) oLeLxya; suMdara: \eng{the good town of India} BAratada suMdara paTaTxNa. 
\num{4} sAdhu; UjiRta; nAyxyasamamxta; loVkasamamxta: \eng{good law} sAdhuvAda shAsana; nAyxyasamamxtavAda kAnUnu, kAyide. 
\num{5} cenAnxda; sogasAda; ruciyAda: \eng{is good eating} tinanxlu sogasAdadudx. ruciyAdadudx. 
\num{6} mecacxtakakx; sutxtayxhaR; sotxVtArxhaR; parxshaMsAhaR; shAlxGaniVya; shAlxGayx; hirimevetatx; sherxVSaThx: \eng{good men and true} hirimevetatx necicxna jana; sherxVSaThxrU satayxsaMdharU Ada jana. 
\num{7} suMdara; AkaSaRkavAda; mATavAda: \eng{a good leg} mATavAda kAlu. \eng{a good figure} suMdaravAda Akaqti; mATavAda meY. 
\num{8} (\AmA) (parxshaMsAthaRka) KuSiyAda; moVjAda: \eng{good old days} hiMdina A dinagaLu, KuSiyAda dinagaLu; hiMdina moVjAda kAla!; oLeLxya haLeV kAla. 
\num{9} BajaRri: \eng{that's a good un!} (\ashi) BajaRri suLuLx! 
\numi{10} (saMboVdhaneyalilx) 
\banum
\alnum{a} vinayasUcakavAgi baLasuva pada: \eng{my good friend} olavina geLeya; pirxya mitarx; nalemxya neVhiga. 
\alnum{b} tananx doDaDxsitxke sUcisutatx doDaDx anugarxha mADutitxruvaMte toVrisikoLuLxvAga baLasuva pada: \eng{my good man!} ayAyx dore! O tamamx! 
\alnum{c} kapaTa vinaya ilalxve nAyxyavAda koVpa toVrisuvAga baLasuva pada: \eng{my good sir} mAnayxre; mahAshayare; mahAsAvxmi; mahaniVyare. 
\alnum{d} vinayasUcaka yA tirasAkxra misharxvAda aupacArika vaNaRneyalilx baLasuva pada: \eng{your good lady} tamamx kuTuMba, saMsAra; tamamx dhamaRpatinxyavaru. \eng{your good man} nimamx yajamAnaru. \eng{the good man} A saMBAvita; A gaqhasathx. 
\eanum
\numie
\num{11} oLeLxya; utatxma; utakxqqSaTx; sherxVSaThx; kuliVna: \eng{of good family} oLeLxya manetanadalilx huTiTxda; sadavxMshada utakxqqSaTx kulada. 
\hypertarget{good(1)12}{} 
\num{12} sariyAda; oLeLxya; ucita; sAdhu; sUkatx; yukatx; vihita; samayoVcita: \eng{I think, it is good to do} adanunx mADuvudu sariyeMdu, sUkatxveMdu, nanaganisutatxde. \eng{it is good that you came} niVnu baMdadudx oLeLxyadAyitu. \eng{it is good to be here} ililxruvudu samayoVcitavAgide. 
\num{13} (mecicxke yA samamxti sUcisuva udAgxravAgi) Bale! BeVSf BeVSf! shAhaBAsf! oLeLxyadu! sari! 
\num{14} neYtikavAgi oLeLxya; dhAmiRka; sadugxNada; sananxDateya: \eng{do one's good deed for the day} dinada oLeLxya kelasa mADu. \eng{good works} satAkxyaRgaLu; dAnadhamaRgaLu. 
\num{15} dayeyiMda kUDida; karuNeyiMda kUDida; oLeLxyatanadiMda kUDida; sadABxvada; upakAra savxBAvada; upakAriyAda: \eng{did me a good deed} nanagoMdu upakAra mADida. \eng{has always been good to me} nananx viSayadalilx yAvAgalU karuNeyiMda, sadABxvaneyiMda idAdxne. 
\num{16} (deVvara \vi) pArxthaRnegaLalilx, AshacxyaRsUcaka udAgxragaLu, \mo vugaLalilx baLasuva pada: \eng{thou great and good, thou art my father and my God!} O mahAmahima! O karuNALu! niVneV nananx taMde, niVneV nananx deYva! \eng{good God! good heavens! good gracious!} ayoyxV deVvare! ayoyxV BagavaMta! 
\num{17} (\kanmu\ maguvina \vi) oLeLxya (naDavaLikeyuLaLx); sadavxtaRneyuLaLx; toMdarekoDada: \eng{as good as gold} cinanxdaMtha; heVLidaMte keVLuva. 
\num{18} oLeLxya; saMtoVSada; saMtoVSakara; AnaMdadAyaka; taqpitxkara; hitavAda; shuBada: \eng{good news} saMtoVSada sudidx; shuBavAteR. \eng{it is good to be alive} badukiruvudu oLeLxyadu. 
\num{19} anukUlavAda; parxyoVjanakaravAda. 
\num{20} sugama; susUtarx; sumuKa; saliVsu; sarAga: \eng{things are in good train} elalx kelasagaLU sugamavAda riVtiyalilx naDedukoMDu hoVgutitxve. 
\num{21} obabxranonxbabxru kaMDAga matutx biTuTx hoVguvAga aupacArika shuBAshaya koVruvAga baLasuva pada: \eng{good morning} suparxBAta. 
\num{22} KuSiyAda; majAmADuva: \eng{have a good time} KuSiyAgiru; majAmADu. 
\num{23} hAyAda; suKavAda: \eng{have a good night} cenAnxgi nidedx mADu; rAtirx hAyAda niderx barali. 
\num{24} oLeLxya; hitavAda; parxshasatx: \eng{a good saying} sUkitx; hitavacana; oLeLxya, parxshasatxvAda gAde yA vacana. 
\num{25} AhAlxdakara; manoVraMjaka: \eng{as good as a play} oMdu nATakadaMteyeV, nATakadaSeTxV manoVraMjakavAda. 
\num{26} guNakAri: \eng{oil is good for burns} suTaTx gAyagaLige teYla guNakAri, oLeLxyadu. 
\num{27} AroVgayxkara: \eng{are acorns good to eat?} Okf haNuNxgaLu tinanxlu AroVgayxkaraveV? \eng{beer is not good for him} (\engit{or} \eng{for his health)} biyarf avanige (athavA avana AroVgayxkekx) oLeLxyadalalx. 
\num{28} hita; sherxVyasakxra. 
\num{29} (vayxkitxgaLa \vi) oLeLxya; kushala; nipuNa: \eng{a good driver} dakaSx cAlaka. \eng{a good actor} oLeLxya naTa. \eng{good English} sariyAda iMgilxSf. \eng{good at describing} vaNaRne \mo vanunx mADuvudaralilx nipuNa. 
\num{30} oLeLxya; takakx; sadaqsha; sariyAda; yoVgayx; oMdu udedxVshakekx anuguNavAda, anurUpavAda: \eng{has been a good wife to him} avanige oLeLxya, anurUpaLAda heMDatiyAgidAdxLe. 
\num{31} necacxbahudAda; naMbikege ahaRvAda; vishAvxsAhaR: \eng{a good customer} necacxbahudAda vayxkitx; vishAvxsAhaR girAki. 
\num{32} (haNakAsinalilx) BadarxvAda; gaTiTxyAda; salilxsabeVkAdudanenxlalx salilxsabalalx; surakiSxta; KAtariyAda: \eng{good debts} KAtari (vasUliya) sAlagaLu; KaMDita vApasAguva Baravaseyiruva sAlagaLu. 
\num{33} gaTiTx; suBadarx; shakatx: \eng{a good life} gaTiTx Ayususx, AyudARya; diVGaRkAla badukabalalx bALu; vime \mo vugaLa kaceVri aMgiVkarisuvaMtha Ayususx. 
\num{34} balavAda; oLeLxya; caloV: \eng{gave her a good beating} avaLige oLeLxya, caloV hoDeta koTaTx. 
\num{35} sAdhuvAda; sariyAda; yukatxvAda; samathaRniVyavAda; samaMjasavAda; anavxyisuva: \eng{did it for good reasons} samaMjasavAda kAraNagaLigAgi adanunx mADida. \eng{rule holds good} niyamavu anavxyisutatxde. 
\num{36} sariyAda; sakAraNavAda: \eng{a good excuse} sakAraNavAda neva. (\AmA) \eng{is a good bit better} takakxSuTx vAsi; savxlapx vAsi. 
\num{37} tavakisuva; AturapaDuva; tavxremADuva; tavxrepaDuva: \eng{have a good mind} (yAvudanenxV mADalu) manasusx tavakisutitxru. 
\num{38} (halavomemx guNavAcakagaLa hiMde AdhikayxvAcakavAgi) takakxSuTx; sAkaSuTx: \eng{went a good round pace} takakxSuTx dUra naDeda. \eng{will take a good long time} sAkaSuTx hecucx kAla hiDiyutatxde. 
\num{39} kaDimeyilalxda: \eng{played for a good hour} oMdu gaMTege kaDimeyilalxde ADida. \eng{it is a good three miles from the station} nilAdxNadiMda ililxge mUru meYlige kaDimeyeVnU ilalx. 
\num{40} (saMKeyx, parxmANa, \mo vugaLalilx) heVraLa; yatheVcaCx; sAkaSuTx hecicxna: \eng{a good deal of money} sAkaSuTx hecicxna haNa. \eng{a good many people} heVraLavAda jana. \eng{we have come a good way} nAvu bahaLa yA sAkaSuTx dUra baMdidedxVve. 
\enum
\emng

\noindent
\gl{\pagu}
\bmng
\bnum
\num{1} \eng{as good as} vasutxta: vAsatxvavAgi: \eng{he as good as told me so} nanagavanu vasutxtaH hAge heVLidaneMdeV enanxbeVku. \eng{as good as dead} satatxMteyeV. \eng{it is as good as done} adu AdaMteyeV; mugidaMteyeV. 
\hypertarget{good pagu2}{} 
\num{2} \eng{be good enough to do} dayaviTuTx mADu; mADuvaSuTx oLeLxya manasusx mADu. 
\num{3} \eng{be so good as to do} = \hyperlink{good pagu2}{?pagu? \((2)\)}. 
\numi{4} \eng{good for an amount} 
\banum
\alnum{a} haNakekx KAtariyuLaLx; haNa koDuvaneMdu dheYyaRvAgi naMbabahudAda. 
\alnum{b} (huMDi \mo vugaLa \vi)-aSuTx haNakAkxgi bareda. 
\hypertarget{good pagu5}{} 
\eanum
\numie
\num{5} \eng{good for you (him etc.)} (\AmA) (saMboVdhita vayxkitx yA sadari vayxkitxyu heVLidadxnunx yA mADidadxnunx opupxvalilx heVLuva udAgxra) (niVnu mADidudx) oLeLxyadAyitu! sari hoVyitu! sariyAgide! 
\num{6} \eng{good (old) man} BeVSf! (opipxge sUcisuva udAgxra). 
\num{7} \eng{good money} hecicxna kUli. 
\num{8} \eng{good o'n you (him, etc.)} (\AseTxrXV\ matutx nUyxsiZVlaMDf) = \hyperlink{good pagu5}{?pagu? \((5)\)}. 
\num{9} \eng{good times} oLeLxya kAla; ucACxrXyada avadhi; sirigAla; samaqdidhxya kAla. 
\num{10} \eng{how good of you!} niVveSuTx oLeLxyavaru! tamamxdu eSuTx doDaDx upakAra! eMtha mahoVpakAra! 
\num{11} \eng{in good spirits} geluviniMda; ulAlxsadiMda. 
\num{12} \eng{in good time} sakAladalilx; taDavAgirade; viLaMba mADirade. 
\num{13} \eng{not good enough} (\AmA) (mADalu, aMgiVkarisalu) takakxSuTx -- yoVgayxvAgilalx, ahaRvAgilalx, taqpitxkaravAgilalx; sharxmakekx sarisamavAdadadxlalx. 
\num{14} \eng{the good people} PeVrigaLu; kirukinanxriyaru; atipuTaTx atimAnuSa vayxkitxgaLu. 
\enum
\emng

\noindent
\gl{\nuga}
\bmng
\bnum
\numi{1} \eng{a good one} (\ashi) 
\banum
\alnum{a} naMbalasAdhayxvAda suLuLx yA atishayoVkitx. 
\alnum{b} BajaRri joVku \mo vu. 
\eanum
\numie
\num{2} \eng{all in good time} sakAladalilx; vihitakAladalilx; AgabeVkAda kAladalilx; muMde, BaviSayxdalilx, Adare Aturavilalxde, avasaravilalxde. 
\num{3} \eng{do me a good turn} (\engit{or} \eng{office)} nanagoMdu upakAra, sahAya mADu. 
\num{4} \eng{feel good} ulAlxsadiMdiru. 
\num{5} \eng{good at} (yAvudeV \vi) samathaR; dakaSx; gaTiTxga; kushala; nipuNa; shakatx: \eng{good at cricket} kirxkeTiTxnalilx gaTiTxga. 
\numi{6} \eng{good for} 
\banum
\alnum{a} iSaTxvuLaLx; manasusxLaLx. 
\alnum{b} shakitxyuLaLx; sAmathayxRvuLaLx: \eng{good for a ten mile walk} hatutx meYli naDeyalu iSaTxvuLaLx, sAmathayxRvuLaLx. 
\eanum
\numie
\num{7} \eng{have a good mind to} mADalu manasusx AturavAgiru, tavakisutitxru. 
\hyperdef{G}{good(1) nuga(8)}{} 
\numi{8} \eng{make good} 
\banum
\alnum{a} parihAra koDu; naSaTxtuMbu, kaTiTxkoDu. 
\alnum{b} (KacuRvecacxvanunx) pAvati mADu; salilxsu; teru. 
\alnum{c} (koTaTx mAtanunx, vAgAdxnavanunx) pAlisu; naDesu; neraveVrisu. 
\alnum{d} (udedxVshavanunx, udidxSaTx kAyaRvanunx) sAdhisu; pUreYsu. 
\alnum{e} (heVLikeyanunx, mAtanunx) mADi toVrisu. 
\alnum{f} (ApAdaneyanunx, AroVpavanunx) rujuvAtu mADikoDu; parxmANiVkarisu. 
\alnum{g} (sAthxnavanunx) paDedu dakikxsiko; vashadalilxTuTxkoMDiru; gaLisi dakikxsiko. 
\alnum{h} (kaLeduhoVda vasutxvige) parxtiyAgi (inonxMdanunx) koDu; badali koDu. 
\alnum{i} (oDeduhoVda yA hAnigoLagAda vasutxvanunx) saripaDisu. 
\alnum{j} (akamaRka \parx) yatinxsidadxnunx sAdhisu. 
\eanum
\numie
\num{9} \eng{put in a good word for} = \hyperlink{good nuga10}{?nuga? \((10)\)}. 
\hypertarget{good nuga10}{} 
\numi{10} \eng{say a good word for} 
\banum
\alnum{a} (obabxnigAgi, obabxna para) oMdu oLeLxya mAtu heVLu; shiPArasu mADu. 
\alnum{b} (obabxna para) vAdisu; vakAlatutx vahisu. 
\eanum
\numie
\num{11} \eng{take in good part} ninanx hitakAkxgiyeV eMdu tiLi; asamAdhAnapaTuTxkoLaLxbeVDa; koVpisikoLaLxbeVDa. 
\enum
\emng
\eentry

\bentry
\word[good(2)]{good}
\pron{guDf}
\gl{\kirxvi}
\bmng
 (\AmA) (\ame) cenAnxgi; sariyAgi; suKavAgi: \eng{doing pretty good} tuMba cenAnxgi idAdxne. 
\emng

\noindent
\gl{\nuga}
\bmng
 \eng{good and} (\AmA) tuMba; saKatf; bahaLa: \eng{raining good and hard} maLe bahaLa joVrAgi hoDeyutitxde. (\ame) \eng{I was good and angry} nAnu saKatf koVpadalilxdedx. 
\emng
\eentry

\bentry
\word[good(3)]{good}
\pron{guDf}
\gl{\nA}
\bmng
\bnum
\num{1} (\bava vAgi) oLeLxyavaru; sadugxNigaLu; suguNigaLu; satupxruSaru; sajajxnaru: \eng{the good and the bad alike respect him} oLeLxyavarU keTaTxvarU avananunx oMdeV riVtiyAgi gwravisutAtxre. 
\num{2} neYtikavAgi oLeLxyadu; oLitu; oLupx; maMgaLa; sherxVyasusx; shuBa: \eng{is a power for good} oLeLxyadanunx mADuva shakitx; shuBakAraka shakitx. 
\num{3} swKayx; hita; keSxVma: \eng{deceive him for his good} avana hitakAkxgiyeV avananunx vaMcisu. 
\num{4} lABa; PAyide; parxyoVjana; Pala: \eng{what good will it do?} adariMdeVnu lABa?, Pala? 
\num{5} daye; upakAra; sahAya. 
\numi{6} puruSAthaR: 
\banum
\alnum{a} bayasabahudAda, apeVkaSxNiVyavAda -- guri yA udedxVsha. 
\alnum{b} paDeyalu, sAdhisalu -- yoVgayxvAda vasutx. 
\eanum
\numie
\num{7} (\bava dalilx) cara Asitx; jiMdagi. 
\num{8} (\bava dalilx) vAyxpArada sarakugaLu; mAlugaLu; sAmAnu saraMjAmugaLu. 
\num{9} (\birx) (\bava dalilx) (\AmA) sAgaNe yA sAgAvaNe sarakugaLu; reYlu \mo vugaLalilx sAgisuva sAmAnugaLu, sarakugaLu: \eng{goods train} sAmAnina reYlu. 
\num{10} (\bava dalilx) (\AmA) sAmAnu saraMjAmu; saraku sAmagirx; odagisalu opipxda saraku, sAmAnu saraMjAmugaLu: \eng{deliver the goods} saraku odagisu (\rUpa\ saha.) 
\enum
\emng

\noindent
\gl{\pagu}
\bmng
\bnum
\num{1} \eng{to the good} nivavxLa lABada, uLitAyada, hecacxLada -- kaDege (iru); nivavxLa lABa (hoMdiru): \eng{we were Rs. 50 to the good} nAvu \eng{50} rUpAyi lABa paDedidedxvu. 
\num{2} \eng{by goods} sAmAnina reYlinalilx. 
\numi{3} \eng{do good (to)} 
\banum
\alnum{a} daye toVru. 
\alnum{b} oLitanunx mADu; dAnadhamaR mADu. 
\alnum{c} sahAya mADu; upakAra mADu. 
\eanum
\numie
\num{4} \eng{in good (with)} (\AmA) (obabxroDane) oLeLxya saMbaMdhadalilx. 
\num{5} \eng{the good} oLeLxyavaru; guNavaMtaru; sajajxnaru. 
\num{6} \eng{what good is it?} = \hyperlink{good2 pagu7}{?pagu? \((7)\)}. 
\hypertarget{good2 pagu7}{} 
\num{7} \eng{what is the good of it?} adariMdeVnu parxyoVjana, Pala? 
\enum
\emng

\noindent
\gl{\nuga}
\bmng
\hypertarget{good2 nuga1}{} 
\bnum
\num{1} \eng{be any good} EnAdarU parxyoVjanavAgu. 
\num{2} \eng{be much good} tuMbA parxyoVjanavAgu. 
\num{3} \eng{be no good} EneVnU parxyoVjanavAgadiru; niSapxrXyoVjakavAgiru. 
\num{4} \eng{be some good} alapxsavxlapx parxyoVjanavAgu. 
\num{5} \eng{come to good} oLeLxyadAgu; oLeLxya, utatxma -- Pala koDu; satapxriNAmavAgu. 
\num{6} \eng{come to no good} satapxriNAma niVDadiru. 
\num{7} \eng{to deliver the goods} opipxda, paNatoTaTx kelasavanunx mADu; niriVkeSxya maTaTxkekx baru. 
\num{8} \eng{for good (and all)} shAshavxtavAgi; kaDeyadAgi; koneyadAgi; aMtimavAgi; AKeYrAgi: \eng{we decided to settle there for good} nAvu alelxV shAshavxtavAgi nelesalu tiVmARnisidevu. 
\num{9} \eng{is after no good} avanu EnoV anathaR mADutitxdAdxne. 
\num{10} \eng{is upto no good} = \hyperlink{good2 nuga1}{?nuga? \((1)\)}. 
\hyperdef{G}{good(3) nuga(11)}{} 
\num{11} \eng{no good} EnoV keVDu; hAvaLi; anathaR. 
\num{12} \eng{piece of goods} (\hA) vayxkitx; AsAmi; isamu: \eng{she is a lovely piece of goods} Ake oLeLx celuvAda isamu. 
\enum
\emng
\eentry

\bentry
\wordnospeech{good breeding}{good breeding}
\pron{?}
\gl{\nA}
\bmng
 sadavxtaRne; shiSATxcAra; sananxDate; oLeLxya naDAvaLi; saBayx naDevaLike. 
\emng
\eentry

\bentry
\word[goodby(1)]{goodby}
\pron{guDfbeY}
\gl{\BAavayx}
\bmng
 (\ame)  = \hyperlink{goodbye(1)}{$^1$goodbye}. 
\emng
\eentry

\bentry
\word[goodby(2)]{goodby}
\pron{guDfbeY}
\gl{\nA}
\expl{(\bava\ \eng{goodbys}).}
\bmng
(\ame)  = \hyperlink{goodbye(1)}{$^1$goodbye}. 
\emng
\eentry

\bentry
\word[goodbye(1)]{goodbye}
\pron{guDfbeY}
\gl{\BAavayx}
\bmng
\hyperdef{G}{good-bye(1)}{} 
\bnum
\num{1} (obabxranonxbabxru biVLokxDuvAga yA dUravANi saMBASaNe mugisuvAga heVLuva mAtu) oLeLxyadu; hoVgi bA yA hoVgibarutetxVne; savxsitx; shuBamasutx; shuBavAgali; namasAkxra. 
\num{2} (\rUpa) (yAvudAdarU oMdu vasutxvanunx tolagisibiTATxga yA AKeYrAgi kaLedukoMDAga heVLuva mAtu) vidAya! 
\enum
\emng
\eentry

\bentry
\word[goodbye(2)]{goodbye}
\pron{guDfbeY}
\gl{\nA}
\expl{(\bava\ \eng{goodbyes}).}
\bmng
vidAya (koVrike); namasAkxra; biVLokxDuvAga yA dUravANi saMBASaNe mugisuvAga shuBAshaya koVruvudu. 
\emng
\eentry

\bentry
\wordnospeech{good fellow}{good fellow}
\pron{?}
\gl{\nA}
\bmng
\bnum
\num{1} sadugxNi; guNashAli; oLeLxyava. 
\num{2} sarasi; sahavAsapirxya; saMgashiVla. 
\num{3} senxVhapara; senxVhashiVla. 
\enum
\emng
\eentry

\bentry
\word{good-fellowship}
\pron{guDfpheloVSipf}
\gl{\nA}
\bmng
\bnum
\num{1} ulAlxsashiVlate; ulAlxsa parxkaqti. 
\num{2} saMgashiVlate. 
\num{3} senxVhashiVlate; senxVhaparate. 
\enum
\emng
\eentry

\bentry
\wordnospeech{good form}{good form}
\pron{?}
\gl{\nA}
\bmng
 shiSATxcAra; parxcalita AdashaRgaLiganuguNavAda vataRne. 
\emng
\eentry

\bentry
\word[good-for-nothing(1)]{good-for-nothing}
\pron{guDfpharfnatiMgf}
\gl{\gu}
\bmng
 kelasakekx bArada; niSapxrXyoVjaka; aparxyoVjaka. 
\emng
\eentry

\bentry
\word[good-for-nothing(2)]{good-for-nothing}
\pron{guDfpharfnatiMgf}
\gl{\nA}
\bmng
 niSapxrXyoVjaka; aparxyoVjaka; kelasakekx bArada vayxkitx. 
\emng
\eentry

\bentry
\word[good-for-nought(1)]{good-for-nought}
\pron{guDfpharfnATf}
\gl{\gu}
\bmng
  = \hyperlink{good-for-nothing(1)}{$^1$good-for-nothing}. 
\emng
\eentry

\bentry
\word[good-for-nought(2)]{good-for-nought}
\pron{guDfpharfnATf}
\gl{\nA}
\bmng
 = \hyperlink{good-for-nothing(2)}{$^2$good-for-nothing}. 
\emng
\eentry

\bentry
\wordnospeech{Good Friday}{Good Friday}
\pron{?}
\gl{\nA}
\bmng
(IsaTxrf hababxda hiMdina) pavitarx shukarxvAra; Esukirxsatxnanunx shilubege Erisidadxra (sAmxraka) dina. 
\emng
\eentry

\bentry
\word{good-hearted}
\pron{guDfhATiRDf}
\gl{\nA}
\bmng
 oLeLxya haqdayada; oLeLxya manasisxna; dayApara. 
\emng
\eentry

\bentry
\wordnospeech{good humour}{good humour}
\pron{?}
\gl{\nA}
\bmng
\bnum
\num{1} ulAlxsada manaHsithxti; ulAlxsa parxvaqtitx. 
\num{2} sarasate; sarasa savxBAva. 
\enum
\emng
\eentry

\bentry
\word{good-humoured}
\pron{guDfhUyxmaDfR}
\gl{\gu}
\bmng
\bnum
\num{1} ulAlxsa-manoVBAvada, parxvaqtitxya. 
\num{2} sarasa savxBAvada; sarasateya. 
\enum
\emng
\eentry

\bentry
\word{good-humouredly}
\pron{guDfhUyxmaDfRli}
\gl{\kirxvi}
\bmng
\bnum
\num{1} ulAlxsa -- manoVBAvadiMda, parxkaqtiyiMda. 
\num{2} sarasa savxBAvadiMda. 
\enum
\emng
\eentry

\bentry
\word{goodie}
\pron{guDi}
\gl{\nA}
\bmng
 \eng{goody} eMbudara rUpAMtara. 
\emng
\eentry

\bentry
\word{goodiness}
\pron{guDinisf}
\gl{\nA}
\bmng
 toVkeRya oLeLxyatana; soVgina sajajxnike; ati nayavAgi, bariya soVginiMda, madheyx talehAki, sapepxyAgi, ilalxve husiyAda BAva sUcisutatx guNashAli enisikoLuLxvike. 
\emng
\eentry

\bentry
\word{goodish}
\pron{guDiSf}
\gl{\gu}
\bmng
 savxlapxmaTiTxge, sumArAgi -- cenAnxgiruva. 
\emng
\eentry

\bentry
\word{goodliness}
\pron{guDflinisf}
\gl{\nA}
\bmng
 aMdavAgiruvike; suMdaravAgiruvike; swMdarayx. 
\emng
\eentry

\bentry
\word{good-looker}
\pron{guDflukarf}
\gl{\nA}
\bmng
 suMdarAMga yA suMdarAMgi; suMdara vayxkitx: \eng{she is a good-looker} avaLu suMdari, celuve. 
\emng
\eentry

\bentry
\word[good-looking(1)]{good-looking}
\pron{guDflukiMgf}
\gl{\gu}
\bmng
 aMdavAgiruva; celuvAda; suMdara; suPxradUrxpiyAda. 
\emng
\eentry

\bentry
\word[good-looking(2)]{good-looking}
\pron{guDflukiMgf}
\gl{\gu}
\bmng
 sadugxNiyAgi kANuva; guNashAliyeMdu toVruva. 
\emng
\eentry

\bentry
\wordnospeech{good looks}{good looks}
\pron{?}
\gl{\nA}
\bmng
 aMda; caMda; celuvu; swMdayaR. 
\emng
\eentry

\bentry
\wordnospeech{good luck}{good luck}
\pron{?}
\gl{\nA}
\bmng
 (oLeLxya) adaqSaTx; sudeYva; swBAgayx: \eng{good luck to you!} ninage oLeLxyadAgali! adaqSaTxshAliyAgu! BAgayxvaMtanAgu! shuBavAgali! jayavAgali! shuBamasutx! 
\emng
\eentry

\bentry
\word{goodly}
\pron{guDfli}
\gl{\gu}
\bmng
\bnum
\num{1} aMdavAda; caMdavAda; suMdara. 
\num{2} heVraLavAda; sumAru -- doDaDx gAtarxda, parxmANada: \eng{a goodly harvest} takakxSuTx heVraLavAda Pasalu, beLe. 
\num{3} (vayxMgayxvAgi) balu sogasAda; aduBxtavAda; BavayxvAda: \eng{here is a goodly watch indeed?} Bale! nijavAgiyU aduBxtavAda gaDiyAra! 
\enum
\emng
\eentry

\bentry
\word{goodman}
\pron{guDfmanf}
\gl{\nA}
\bmng
 (\birx) (\pArxparx) maneya oDeya; yajamAna; gaqhapati; gaMDa, taMde, itAyxdi. 
\emng
\eentry

\bentry
\wordnospeech{good money}{good money}
\pron{?}
\gl{\nA}
\bmng
\bnum
\num{1} (\AmA) dubAri kUli, veVtana, saMbaLa. 
\num{2} sAcA haNa; nijavAda haNa; (KoVTA alalxda) nijavAda nANayx. 
\num{3} oLeLxya haNa; beVre kAyaRkAkxgi sadupayoVga mADabahudAgidadx haNa: \eng{throw good money after bad} keTaTx haNada hiMde oLeLxya haNavanunx celulx; kaLedukoMDa haNavanunx matetx gaLisikoLaLxlu inanxSuTx haNa hALu mADiko. 
\enum
\emng
\eentry

\bentry
\wordspecial{good morning}{1}{1}{\hyperlink{good-morning(1)}{\quad\textcolor{superscript}{$^2$}\eng{good-morning}}}
\pron{?}
\gl{\BAavayx}
\bmng
 (beLigegx obabxranonxbabxru kaMDAga yA biTuTx hoVguvAga heVLuva udAgxra) suparxBAta! beLagina vaMdane! 
\emng
\eentry

\bentry
\wordspecial{good-morning}{1}{2}{\hyperlink{good morning(1)}{\quad\textcolor{superscript}{$^1$}\eng{good morning}}}
\pron{guDfmAniRMgf}
\gl{\nA}
\bmng
 suparxBAta (beLigegx obabxranonxbabxru saMdhisidAga yA agaluvAga heVLuva mAtu). 
\emng
\eentry

\bentry
\wordspecial{good morrow}{1}{1}{\hyperlink{goodmorrow(1)}{\quad\textcolor{superscript}{$^2$}\eng{goodmorrow}}}
\pron{?}
\gl{\BAavayx}
\bmng
 (\pArxparx)  = \hyperlink{good morning(1)}{$^1$good morning}. 
\emng
\eentry

\bentry
\wordspecial{goodmorrow}{1}{2}{\hyperlink{good morrow(1)}{\quad\textcolor{superscript}{$^1$}\eng{good morrow}}}
\pron{guDfmAroV}
\gl{\nA}
\bmng
 (\birx)  = \hyperlink{good-morning(1)}{$^2$good-morning}. 
\emng
\eentry

\bentry
\wordnospeech{good nature}{good nature}
\pron{?}
\gl{\nA}
\bmng
 oLeLxyatana; swjanayx; dayAparate; oLeLxya savxBAva; upakAra budidhx; pararigAgi sAvxthaRvanunx badigikukxva parxvaqtitx. 
\emng
\eentry

\bentry
\word{good-natured}
\pron{guDfneVcaDfR}
\gl{\gu}
\bmng
 oLeLxyatanada; dayApara; upakAra budidhxya. 
\emng
\eentry

\bentry
\word{good-naturedly}
\pron{guDfneVcaDfRli}
\gl{\kirxvi}
\bmng
 oLeLxyatanadiMda; upakAra budidhxyiMda. 
\emng
\eentry

\bentry
\wordnospeech{good neighbour}{good neighbour}
\pron{?}
\gl{\nA}
\bmng
 senxVhapara; senxVhadiMda vatiRsuvava. 
\emng
\eentry

\bentry
\wordnospeech{good neighbourhood}{good neighbourhood}
\pron{?}
\gl{\nA}
\bmng
 senxVhavataRne; geLetanada naDavaLike; senxVhadiMda naDedukoLuLxvudu. 
\emng
\eentry

\bentry
\wordnospeech{good neighbourliness}{good neighbourliness}
\pron{?}
\gl{\nA}
\bmng
  = \hyperlink{good neighbourhood}{good neighbourhood}. 
\emng
\eentry

\bentry
\word{goodness}
\pron{guDfnisf}
\gl{\nA}
\bmng
\bnum
\num{1} oLeLxyatana; suguNa; sadugxNa. 
\num{2} sherxVSaThxte; savxMtada yA hoVlikeyiMda kANuva hecucxgArike. 
\num{3} sahAya; upakAra; daye: \eng{have the goodness to do} dayemADi iSuTx mADu. 
\num{4} audAyaR. 
\num{5} (vasutxvina, viSayada) tiruLu; huruLu; sAra; satatxvX; shakitx. 
\num{6} (udAgxragaLalilx \eng{God} eMbudakekx parxtiyAgi baLasuva mAtAgi) deVvaru: \eng{goodness knows} deVvareV balalx! deVvarigeV gotutx! \eng{thank goodness}! deVvara daye! deVvaru kApADida! 
\enum
\emng

\noindent
\gl{\pagu}
\bmng
\bnum
\num{1} \eng{for goodness' sake} ninanx puNayxkekx; ninanx dhamaRkekx; ninanx damamxyayx; deVvara hesarinalilx ninanxnunx beVDikoLuLxtetxVne. 
\hypertarget{goodness pagu2}{} 
\num{2} \eng{goodness gracious!} (AshacxyaRvanunx, anAyxyada virudadhx koVpavanunx sUcisuva udAgxra) ayoyxV deVvare! 
\num{3} \eng{goodness me}! = \hyperlink{goodness pagu2}{?pagu? \((2)\)}. 
\num{4} \eng{I wish to goodness} deVvaranunx beVDikoLuLxtetxVne. 
\enum
\emng
\eentry

\bentry
\word{goodo}
\pron{guDoV}
\gl{\gu}
\bmng
 (\AseTxrXV\ matutx nUyxsiZVlaMDf)  = \hyperlink{good(1)12}{$^1$good (12)}. 
\emng
\eentry

\bentry
\wordnospeech{good offices}{good offices}
\pron{?}
\gl{\nA}
\bmng
\bnum
\num{1} madhayxsithxke; yAvudeV vivAdadalilx madhayxsathxgAranu salilxsida seVve, sahAya. 
\num{2} (\kanmu\ adhikAradalilxruvavana) parxBAva. 
\enum
\emng
\eentry

\bentry
\wordnospeech{good oil}{good oil}
\pron{?}
\gl{\nA}
\bmng
 (\AseTxrXV) (\ashi) vishAvxsAhaR sudidx; naMbalahaR mUladiMda baMda samAcAra, mAhiti. 
\emng
\eentry

\bentry
\wordnospeech{good question}{good question}
\pron{?}
\gl{\nA}
\bmng
 oLeLxya parxshenx; takaSxNa yA sulaBavAgi utatxrisalAgada parxshenx. 
\emng
\eentry

\bentry
\wordnospeech{good sense}{good sense}
\pron{?}
\gl{\nA}
\bmng
 viveVcane; viveVka; parijAcnxna; vayxvahAra jAcnxna. 
\emng
\eentry

\bentry
\wordnospeech{good temper}{good temper}
\pron{?}
\gl{\nA}
\bmng
 tALemx; sahane; manasasxmAdhAna; parxsananxcitatxte. 
\emng
\eentry

\bentry
\word{good-tempered}
\pron{guDfTeMpaDfR}
\gl{\gu}
\bmng
 sahanashiVla; tALemxya savxBAvada; sahaneya. 
\emng
\eentry

\bentry
\word{good-temperedly}
\pron{guDfTeMpaDfRli}
\gl{\kirxvi}
\bmng
 tALemxya manoVvaqtitxyiMda; sahanashiVlateyiMda; siDukilalxde. 
\emng
\eentry

\bentry
\wordnospeech{good thing}{good thing}
\pron{?}
\gl{\nA}
\bmng
\bnum
\num{1} oLeLxyadu; opipxgeyAda yA samamxti paDeda yAvudeV vasutx yA viSaya. 
\num{2} lABadAyaka -- vayxvahAra, vAyxpAra, saTATxvAyxpAra. 
\num{3} jANunxDi; caturoVkitx; cATUkitx; camatAkxravAda mAtu. 
\num{4} (\bava dalilx) savitinisugaLu; ruciyAda tiMDigaLu; madhuravAda BakaSxyXBoVjayxgaLu. 
\enum
\emng
\eentry

\bentry
\wordspecial{good time}{1}{1}{\hyperlink{good-time(1)}{\quad\textcolor{superscript}{$^2$}\eng{good-time}}}
\pron{?}
\gl{\nA}
\bmng
\bnum
\num{1} sakAla; vihita kAla; sariyAda, ucitavAda -- kAla. 
\num{2} oLeLxya kAla; ucACxrXya kAla; ELigeya kAla. 
\num{3} saMtoVSada kAla; KuSiya kAla; suKAvadhi; nemamxdiya kAla; oLeLxya kAla. 
\enum
\emng
\eentry

\bentry
\wordspecial{good-time}{1}{2}{\hyperlink{good time(1)}{\quad\textcolor{superscript}{$^1$}\eng{good time}}}
\pron{guDfTeYmf}
\gl{\gu}
\bmng
(vayxkitxya \vi) BoVgalAlasa; BoVgapipAseya; yAvudanUnx lekikxsade suKaBoVgagaLanunx arasi hoVguva: \eng{a good-time girl} BoVgalAlaseya huDugi. 
\emng
\eentry

\bentry
\word{goodwife}
\pron{guDfveYphf}
\gl{\nA}
\bmng
 (sAkxTelxMDinalilx) maneyoDati; gaqhiNi; maneya yajamAni. 
\emng
\eentry

\bentry
\word{goodwill}
\pron{guDfvilf}
\gl{\nA}
\bmng
\bnum
\num{1} (vayxkitxya bagegx) swhAdaR; sadAshaya; parxsananxte; oLeLxya manoVBAva; sadABxvane; vishAvxsa; pirxVti; aBimAna. 
\num{2} (obabxnige mADuva) upakAra; sahAya; anugarxha. 
\num{3} (saMtoVSadiMda niVDida) opipxge; samamxti; anumoVdane. 
\num{4} hAdiRkate; haqtUpxvaRkate. 
\num{5} utAsxha; hurupu. 
\num{6} (vAyxpAra \mo vugaLalilx sAthxpitavAgiruva) parxKAyxti; oLeLxya hesaru; hesaru matutx girAki. 
\num{7} vAyxpArada hakukx; vAyxpAra vahivATanunx mAribiDuvavanu tananx utatxrAdhikAriyAgi vAyxpAra naDesalu koLuLxvavanige niVDuva hakukx. 
\num{8} pagaDi; vAyxpArada hakakxnunx biTuTxkoDalu koDuva haNa. 
\enum
\emng
\eentry

\bentry
\wordnospeech{good works}{good works}
\pron{?}
\gl{\nA}
\bmng
 oLeLxya kelasagaLu; satAkxyaRgaLu; dAnadhamaRgaLu; dhamaRkAyaRgaLu. 
\emng
\eentry

\bentry
\word[goody(1)]{goody}
\pron{guDi}
\gl{\nA}
\bmng
\bnum
\num{1} (\pArxparx) keLavagaRda vayasAsxda heMgasu. 
\num{2} (bahuveVLe manetanada hesarina hiMde pUvaRpadavAgi \parx) \eng{goody Blake etc.} 
\enum
\emng
\eentry

\bentry
\word[goody(2)]{goody}
\pron{guDi}
\gl{\nA}
\bmng
\bnum
\num{1} sakakxre miThAyi; oMdu bageya miThAyi. 
\num{2} paramAyiSi vasutx, \kanmu\ tinanxlu cenAnxgiruvaMthadu. 
\num{3} (\AmA) oLeLxya vayxkitx, \kanmu\ oMdu kathe, sinimA, \mo vugaLa nAyaka. 
\enum
\emng
\eentry

\bentry
\word[goody(3)]{goody}
\pron{guDi}
\gl{\gu}
\bmng
 soVgina sajajxnikeya; sajajxnika soVgina; (atinayavAgi, bariya soVginiMda, madheyx talehAki, sapepxyAgi, ilalxve husiyAda BAva sUcisutatx) guNashAliyenisuva. 
\emng

\noindent
\gl{\pagu}
\bmng
 \eng{talk goody} swjanayxda soVgina mAtanADu; soVgina sajajxnikeyiMda mAtanADu; ati nayavAgi, bariya soVginiMda, madheyx tale hAki, sapepxyAgi, ilalxve husiyAda BAva sUcisutatx, guNashAliyenisikoLuLxva riVtiyalilx mAtanADu. 
\emng
\eentry

\bentry
\word[goody(4)]{goody}
\pron{guDi}
\gl{\nA}
\bmng
 soVgina sajajxna; soVgina swjanayx toVruva vayxkitx; oLeLxyavaneMdu naTisuva, toVrisikoLuLxva vayxkitx. 
\emng
\eentry

\bentry
\word[goody(5)]{goody}
\pron{guDi}
\gl{\BAavayx}
\bmng
 Bale! BeVSf! (mugadhx saMtoVSa yA AshacxyaR sUcisuva pada). 
\emng
\eentry

\bentry
\word{goody-goodiness}
\pron{guDiguDinisf}
\gl{\nA}
\bmng
  = \hyperlink{goodiness}{goodiness}. 
\emng
\eentry

\bentry
\word[goody-goody(1)]{goody-goody}
\pron{guDiguDi}
\gl{\gu}
\bmng
  = \hyperlink{goody(3)}{$^3$goody}. 
\emng
\eentry

\bentry
\word[goody-goody(2)]{goody-goody}
\pron{guDiguDi}
\gl{\nA}
\bmng
  = \hyperlink{goody(4)}{$^4$goody}. 
\emng
\eentry

\bentry
\word{gooey}
\pron{gUi}
\gl{\gu}
\expl{(\tara\ \eng{gooier}, \tama\ \eng{gooiest}).}
\bmng
(\ashi) 
\bnum
\num{1} aMTuva; aMTaMTAda. 
\num{2} (\rUpa) ati BAvuka; BAvAtireVkada. 
\enum
\emng
\eentry

\bentry
\word[goof(1)]{goof}
\pron{gUphf}
\gl{\nA}
\bmng
 (\ashi) 
\bnum
\num{1} pedadx; daDaDx. 
\num{2} (\kanmu\ ajAgarUkateyiMda uMTAda) tapupx; eDavaTuTx. 
\enum
\emng
\eentry

\bentry
\word[goof(2)]{goof}
\pron{gUphf}
\gl{\sakirx}
\bmng
 (\ashi) 
\bnum
\num{1} (kelasavanunx) holasebibxsu; hALumADu; hagaraNa mADu; galibili mADu; eDavaTuTx mADu: \eng{he goofed up one opportunity after another} avanu oMdAda meVle oMdaraMte avakAshagaLanunx hALumADikoMDa. 
\num{2} (\BUkaq dalilx) mAdaka vasutxgaLiMda, nashiVli padAthaRgaLiMda matutx barisu. 
\enum
\emng

\noindent
\gl{\akirx}
\bmng
 (\ashi) 
\bnum
\num{1} tapupxmADu; eDavaTuTx mADiko; aviveVka mADu. 
\num{2} kAlakaLe; kAlaharaNa mADu: \eng{we just goofed around till train time} reYlina veVLeya tanaka nAvu sumamxne kAlakaLedevu. 
\enum
\emng
\eentry

\bentry
\word{goof-ball}
\pron{gUphfbAlf}
\gl{\nA}
\bmng
\bnum
\num{1} gUphf guLige; nashiVli guLige; mAyxrihAvxna \mo\ yAvudeV mAdaka vasutxvina guLige. 
\num{2} gugugx; daDaDx; maMkudiNeNx. 
\enum
\emng
\eentry

\bentry
\word{go-off}
\pron{goVAphf}
\gl{\nA}
\bmng
 shuru; pArxraMBa; AraMBa; horaDuvudu: \eng{at the first go-off} moTaTxmodaleV; shuruvinalelxV; pArxraMBadalelxV. 
\emng
\eentry

\bentry
\word{goofy}
\pron{gUphi}
\gl{\gu}
\bmng
 (\ashi) pedadx; daDaDx; aviveVkada. 
\emng
\eentry

\bentry
\word{googly}
\pron{gUgilx}
\gl{\nA}
\bmng
 (kirxkeTf) gUgilx; `legf berxVkf'naMte bwlf mADida `Aphf berxVku'. 
\emng
\eentry

\bentry
\word{gook}
\pron{gU(gu)kf}
\gl{\nA}
\bmng
 (\ame) (\ashi) (\hiV) videVshiVya; \kanmu\ pUvaR ESAyx \mo vugaLiMda baMda kapupxvaNaRdava. 
\emng
\eentry

\bentry
\word{goon}
\pron{gUnf}
\gl{\nA}
\bmng
(\ashi) 
\bnum
\num{1} puMDa; gUMDA; kelasagAraranunx hedarisalu suligekoVraru neVmisikoLuLxva puMDa. 
\num{2} daDaDx; pedadx; egagx; muTAThxLa. 
\enum
\emng
\eentry

\bentry
\word{goop}
\pron{gUpf}
\gl{\nA}
\bmng
 (\ashi) daDaDx; maDiDx; modudx; gAMpa. 
\emng
\eentry

\bentry
\word{goopy}
\pron{gUpi}
\gl{\gu}
\bmng
modudxmodAdxda; pedudxpedAdxda. 
\emng
\eentry

\bentry
\word{goosander}
\pron{gUsAYxMDarf}
\gl{\nA}
\bmng
 garagasa bAtu; garagasadaMte cUpu kokukxLaLx, magaRsf magAyxRnasxrf kulakekx seVrida bAtu. 
\emng
\eentry

\bentry
\word[goose(1)]{goose}
\pron{gUsf}
\gl{\nA}
\bmng
 (\eng{5}ne athaRdalilx horatu \bava\ \eng{geese}). 
\bnum
\num{1} hebAbxtu; varaTe; tADigayx. 
\num{2} heNuNx varaTe. 
\num{3} varaTeya mAMsa. 
\num{4} modudxmaNi; heDaDx; gAMpa; daDaDx; maDiDx. 
\num{5} (\bava\ \eng{goose}) (bAtina katitxnaMte hiDiyuLaLx dajiRyavana) isitxrX peTiTxge. 
\enum
\emng

\noindent
\gl{\nuga}
\bmng
\bnum
\num{1} \eng{all his geese are swans} tananxdakekx, tananx yoVjanegaLu, senxVhitaru \mo vakekx -- mitimIrida bele kaTuTxtAtxne; tananx goDuDx hasuvanenxV kAmadheVnu anunxtAtxne; tananx bAtugaLanenxV rAja{ha}MsagaLenunxtAtxne. 
\num{2} \hyperref{kandict_c.pdf}{C}{cook(2) pagu(1)}{$^2$cook person's goose.} 
\num{3} \eng{kill the goose that lays the golden eggs} iMdina loVBakekx muMdina lABavanunx balikoDu. 
\hyperdef{G}{goose(1) nuga(4)}{} 
\num{4} \eng{sauce for goose is sauce for} \hyperlink{gander(1) nuga}{gander}. 
\num{5} \eng{can't say} \hyperref{kandict_b.pdf}{B}{boo(1) nuga}{$^1$boo to a goose.} 
\enum
\emng
\eentry

\bentry
\word[goose(2)]{goose}
\pron{gUsf}
\gl{\sakirx}
\bmng
 (\ashi) vayxkitxyanunx mamaR sAthxnadalilx, \kanmu\ jananAMgadalilx yA gudadAvxradalilx -- tivi, cucucx: \eng{as she was bending over her work-table..., a playful lab assistant goosed her} avaLu tananx kelasada meVjina meVle bagugxtitxdadxMte kiVTaleya parxyoVgAlaya sahAyakanu avaLa gudavanunx tivida. 
\emng
\eentry

\bentry
\word{gooseberry}
\pron{gusfZbari}
\gl{\nA}
\bmng
\bnum
\numi{1} gusfZberi: 
\banum
\alnum{a} reYbiVsf kulada muLuLx giDagaLalilx biDuva, nelilxkAyiyaMtha oMdu kAyi. 
\alnum{b} adara giDa, pode. 
\eanum
\numie
\num{2} (\pArxparx) gusfZberiyiMda tayArisida veYnu. 
\enum
\emng

\noindent
\gl{\pagu}
\bmng
 \eng{gooseberry fool} gusfZberi rasAyana. 
\emng

\noindent
\gl{\nuga}
\bmng
 \eng{play gooseberry} (hiriya heMgasina \vi) perxVmigaLa joVDiya usutxvAradAraLAgiru; (avaru naDate mIri hoVgadaMte) kAvalAgiru; (avarige) beVDavAda saMgAti, anapeVkiSxta sahacari -- Agiru. 
\emng
\eentry

\bentry
\wordnospeech{gooseberry bush}{gooseberry bush}
\pron{?}
\gl{\nA}
\bmng
\bnum
\num{1} gusfberi (giDada) pode. 
\num{2} gusfberi pode (makakxLu `nAnu heVge huTiTxde' eMdu keVLidAga kelavu veVLe gUsfberi giDada podeyiMda baMde eMdu heVLuvAga baLasuva parxyoVga). 
\enum
\emng
\eentry

\bentry
\wordnospeech{goose bumps}{goose bumps}
\pron{?}
\gl{\nA}
\bmng
 (\ame)  = \hyperlink{goose-flesh}{goose-flesh}. 
\emng
\eentry

\bentry
\word{goose-club}
\pron{gUsfkalxbf}
\gl{\nA}
\bmng
 bAtusaMGa; bAtukalxbubx; baDavarige kirxsfmasf (bAtu) mAMsa odagisalu saNaNx kaMtugaLalilx haNa koDuva kalxbubx, saMGa. 
\emng
\eentry

\bentry
\word{goose-egg}
\pron{gUsfegf}
\gl{\nA}
\bmng
 (\ame) (ATadalilx) sonenx -- aMka, sokxVru; kuMbaLakAyi. 
\emng
\eentry

\bentry
\word{goose-flesh}
\pron{gUsfphelxSf}
\gl{\nA}
\bmng
 (camaRda) roVmAMca (sithxti); ati caLiyiMda yA BayadiMda birumuLeLxdadx, naviredadx camaRda sithxti. 
\emng
\eentry

\bentry
\word{goose-foot}
\pron{gUsfphuTf}
\gl{\nA}
\expl{(\bava\ \eng{goose-foots}).}
\bmng
kADu Oma; huLi Oma; kinoVpoVDiyaM kulada oMdu sasayx. 
\emng
\eentry

\bentry
\word{goose-girl}
\pron{gUsfgalfR}
\gl{\nA}
\bmng
 bAtu huDugi; bAtugaLanunx sAkalu neVmisikoMDa huDugi. 
\emng
\eentry

\bentry
\word{goose-gog}
\pron{gusfZgAgf}
\gl{\nA}
\bmng
 (\birx) (\AmA)  = \hyperlink{gooseberry}{gooseberry}. 
\emng
\eentry

\bentry
\word{goose-grass}
\pron{gUsfgArxsf}
\gl{\nA}
\bmng
 = \hyperref{kandict_c.pdf}{C}{cleavers}{cleavers}. 
\emng
\eentry

\bentry
\word{goose-herd}
\pron{gUsfhaDfR}
\gl{\nA}
\bmng
 bAtugAhi; bAtu sAkuvavanu yA bAtu kAyuvavanu. 
\emng
\eentry

\bentry
\word{goose-neck}
\pron{gUsfnekf}
\gl{\nA}
\bmng
 bAtukatutx; bAtina katitxnaMtha vasutx. 
\emng
\eentry

\bentry
\word{goose-pimples}
\pron{gUsfpiMpalfsx}
\gl{\nA}
\bmng
=  \hyperlink{goose-flesh}{goose-flesh}. 
\emng
\eentry

\bentry
\word{goose-quill}
\pron{gUsfkivxlf}
\gl{\nA}
\bmng
 gariya leVKani, peVna; (\kanmu\ leVKanige baLasuva) varaTeya gari; bAtu gari. 
\emng
\eentry

\bentry
\word{goose-skin}
\pron{gUsfsikxnf}
\gl{\nA}
\bmng
  = \hyperlink{goose-flesh}{goose-flesh}. 
\emng
\eentry

\bentry
\wordnospeech{goose step}{goose step}
\pron{?}
\gl{\nA}
\bmng
 gUsfseTxpf; varaTe hejejx: 
\banum
\alnum{a} (hosadAgi seYnayxkekx seVruvavarige, \kanmu\ jamaRnf seYnayxdalilx baLakeyalilxdadx) maMDi seTedukoMDu hAkuva oMdu riVtiya dApu hejejx. 
\alnum{b} oMTikAlu tUgATa; oMdu kAlina meVle niMtu inonxMdu kAlanunx tUgADisuva oMdu riVtiya seYnika kavAyatu. 
\eanum
\emng
\eentry

\bentry
\word[goosey(1)]{goosey}
\pron{gUsi}
\gl{\nA}
\bmng
 daDaDx; pedadxMBaTaTx. 
\emng
\eentry

\bentry
\word[goosey(2)]{goosey}
\pron{gUsi}
\gl{\gu}
\bmng
 budidhxhiVna; daDaDx; egagx; maMka. 
\emng
\eentry

\bentry
\word[goosy(1)]{goosy}
\pron{gUsi}
\gl{\nA}
\bmng
  = \hyperlink{goosey(1)}{$^1$goosey}. 
\emng
\eentry

\bentry
\word[goosy(2)]{goosy}
\pron{gUsi}
\gl{\gu}
\bmng
  = \hyperlink{goosey(2)}{$^2$goosey}. 
\emng
\eentry

\bentry
\wordnospeech{GOP}{GOP}
\pron{?}
\gl{\saMkiSx}
\bmng
 (\ame) \eng{Grand Old Party (the Republican Party).} 
\emng
\eentry

\bentry
\word[gopher(1)]{gopher}
\pron{goVpharf}
\gl{\sakirx}
\bmng
  = \hyperlink{goffer(1)}{$^1$goffer}. 
\emng
\eentry

\bentry
\word[gopher(2)]{gopher}
\pron{goVpharf}
\gl{\nA}
\bmng
  = \hyperlink{goffer(2)}{$^2$goffer}. 
\emng
\eentry

\bentry
\word[gopher(3)]{gopher}
\pron{goVpharf}
\gl{\nA}
\bmng
\hyperdef{G}{gopher(3)a}{} goVParf: 
\banum
\alnum{a} utatxra matutx madhayx amerikada pashicxma pArxMtadalilx kANabaruva, doDaDx iliya gAtarxda, bila toVDuva oMdu daMshaka. 
\alnum{b} utatxra amerikada hululxgADinalilx kANabaruva, cikakx nela aLilu. 
\alnum{c} utatxra amerikada dakiSxNa karAvaLi parxdeVshadalilx kANabaruva, oMdu aDigiMta hecucx udadxda cipupxLaLx, rAtirxyalilx bila toVDuva oMdu bageya nela Ame, BUkUmaR. 
\eanum
\emng
\eentry

\bentry
\word[gopher(4)]{gopher}
\pron{goVpharf}
\gl{\nA}
\bmng
\bnum
\num{1} (\beY) goVpharf; noVa eMbuvana doVNiyanunx mADalu baLasida mara. 
\num{2}  = \hyperlink{gopher-wood}{gopher-wood}. 
\enum
\emng
\eentry

\bentry
\wordnospeech{Gopher snake}{Gopher snake}
\pron{?}
\gl{\nA}
\bmng
 = \hyperref{kandict_c.pdf}{C}{cribo}{cribo}. 
\emng
\eentry

\bentry
\wordnospeech{Gopher State}{Gopher State}
\pron{?}
\gl{\nA}
\bmng
 (\ame) minasoVTa saMsAthxna. 
\emng
\eentry

\bentry
\word{gopher-wood}
\pron{goVpharfvuDf}
\gl{\nA}
\bmng
 goVpharf mara; amerika saMyukatx saMsAthxnagaLa dakiSxNa parxdeVshadalilx beLeyuva, suvAsaneya biLiya hU biDuva, haLadi vaNaRdarxvayxvanunx koDuva, gaTiTxdAruvina oMdu bageya mara. 
\emng
\eentry

\bentry
\word{goral}
\pron{goVralf}
\gl{\nA}
\bmng
 goVralf; (oMdu jAtiya, iMDiya deVshada) sAraMga. 
\emng
\eentry

\bentry
\word{gorblimey}
\pron{goVbelxYRmi}
\gl{\BAavayx}
\bmng
 (\birx) (\tu) (AshacxyaR, koVpa, \mo vanunx sUcisuva udAgxra) nananx kaNuNx iMgihoVga! (\eng{God blind me} enunxvudara apaBarxMsha rUpa). 
\emng
\eentry

\bentry
\word{gorcock}
\pron{gAkARkf}
\gl{\nA}
\bmng
 (sAkxTalxMDf matutx utatxra iMgalxMDf) keMpu gwrxsf huMja; gwrxsf eMba oMdu bageya keMpu koVLi jAtiya gaMDu. 
\emng
\eentry

\bentry
\word{Gordian}
\pron{gADiRanf}
\gl{\gu}
\bmng
 (bicacxlArada kagagxMToMdanunx hAkidaneMdu parxKAyxtanAda, pArxciVna phirxjiyA deVshada rAjanAda) gADiRyasasxna. 
\emng

\noindent
\gl{\pagu}
\bmng
 \eng{Gordian knot} kagagxMTu; kaDugaMTu; biDisalArada toDaku, samaseyx: \eng{cut the gordian knot} (balaparxyoVgadiMda yA niyamagaLanunx badigotitx) gaMTu hari; ogaTu biDisu; samaseyx pariharisu. 
\emng
\eentry

\bentry
\wordRemoveSpace{Gordon-setter}{Gordon setter}
\pron{gADaRnf seTarf}
\gl{\nA}
\bmng
 kapupx matutx kaMdu baNaNxda, oMdu beVTe nAyi taLi. 
\emng
\eentry

\bentry
\word[gore(1)]{gore}
\pron{goVrf}
\gl{\nA}
\bmng
 (celilx) gaDeDx kaTiTxda, hepupxgaTiTxda -- rakatx. 
\emng
\eentry

\bentry
\word[gore(2)]{gore}
\pron{goVrf}
\gl{\nA}
\bmng
\bnum
\num{1} mumUmxle tuMDu; uDigeyalilx agalavanunx sarihoMdisalu baLasuva, tirxkoVnAkArada baTeTxya tuMDu. 
\num{2} (koDe, AkAshabuTiTx, gumamxTa, goVLa, \mo vugaLalilx) mumUmxleyAkArada yA bAlacaMdArxkaqtiya tuMDu. 
\num{3} nirigepaTiTx; nirige laMgada yA itara uDupina nirigegaLalilx, paTiTxgaLalilx oMdu. 
\enum
\emng
\eentry

\bentry
\word[gore(3)]{gore}
\pron{goVrf}
\gl{\sakirx}
\bmng
 mumUmxle tuMDu baTeTx aLavaDisi 
\banum
\alnum{a} holi; rUpa koDu. 
\alnum{b} kiridAgisu; agala kaDime mADu. 
\eanum
\emng
\eentry

\bentry
\word[gore(4)]{gore}
\pron{goVrf}
\gl{\sakirx}
\bmng
\bnum
\num{1} (koMbu, koVredADe, \mo vugaLiMda) iri; tivi; hAyi. 
\num{2} (cUpAda baMDegaLu) haDaganunx iri, hAyi. 
\enum
\emng
\eentry

\bentry
\word[gorge(1)]{gorge}
\pron{gAjfR}
\gl{\nA}
\bmng
\bnum
\num{1} (alaM.) oLagaMTalu. 
\num{2} nuMgidudx; kabalita; jaTharasathx; hoTeTxyalilxruvudu; hoTeTxyalilxruva padAthaRgaLu. 
\num{3} (koVTe) buruju bAgilu; burujina yA itara hora rakaSxkadiMda horakekx yA oLakekx hoVguva BAga. 
\num{4} hiMbAgilu; rakaSxNAvaraNada hiMdugaDeya mAgaR, parxveVsha. 
\num{5} (beTaTxgaLa naDuvaNa, \sA\ hoLe hariyutitxruva) kamari. 
\num{6} gALada ere(yAgi baLasuva gaTiTxvasutx). 
\num{7} (\ame) aDacu; kiridAda dAriyanunx mucicxruva maMjugaDeDx \mo vu. 
\enum
\emng

\noindent
\gl{\nuga}
\bmng
\bnum
\num{1} \eng{cast the gorge at} jugupesxyiMda, asahayxdiMda, heVsikeyiMda -- taLiLxbiDu, horahAku, nUkibiDu; Okarisi kakikxbiDu. 
\num{2} \eng{heave the gorge} (vAMtiyAguvudoV eMbaMte) Okarisu. 
\num{3} \eng{one's gorge rises at} noVDidare bahaLa asahayx, jugupesx huTuTxtatxde; kaMDare hoTeTx toLasutatxde. 
\enum
\emng
\eentry

\bentry
\word[gorge(2)]{gorge}
\pron{gAjfR}
\gl{\sakirx}
\bmng
\bnum
\num{1} kaMThapUtiR tinunx; kaTerxyAguvaMte, hoTeTx biriyuvaMte uNuNx. 
\num{2} gabagabane -- nuMgu, kabaLisu, mukukx. 
\num{3} aDacu. 
\num{4} higugxvaSuTx, biriyuvaSuTx -- tuMbu. 
\num{5} BatiR tuMbu. 
\enum
\emng

\noindent
\gl{\akirx}
\bmng
 (ati AsheyiMda) sikAkxpaTeTx tinunx; hoTeTxbAkanaMte uNuNx. 
\emng
\eentry

\bentry
\word[gorge(3)]{gorge}
\pron{gAjfR}
\gl{\nA}
\bmng
 gabagabane tinunxvudu; mitimiVri, kaTaTxreyAguvaMte uNuNxvudu. 
\emng
\eentry

\bentry
\word{gorgeous}
\pron{gAjaRsf}
\gl{\gu}
\bmng
\bnum
\num{1} bahukAMtiya; ujavxla vaNaRda. 
\num{2} atayxlaMkArada; jAjavxlayx mAnavAda. 
\num{3} veYBavada; ADaMbarada; kaNesxLeva. 
\num{4} (padaparxyoVgada \vi) alaMkArayuta; hoLeyuva; beragugoLisuva. 
\num{5} puSakxLavAda; samaqdadhxvAda: \eng{a gorgeous meal} puSakxLavAda BoVjana. 
\num{6} (\AmA) tuMba hitavAda; BajaRri: \eng{had a gorgeous time} hitavAgi kAlakaLeda. 
\enum
\emng
\eentry

\bentry
\word{gorgeously}
\pron{gAjaRsfli}
\gl{\kirxvi}
\bmng
\bnum
\num{1} bahukAMtiyutavAgi; ujavxla vaNaRdiMda. 
\num{2} atayxlaMkArayutavAgi; jAjavxlayxmAnavAgi. 
\num{3} (padaparxyoVgada \vi) alaMkArayutavAgi; beragugoLisuvaMte. 
\num{4} puSakxLavAgi; samaqdadhxvAgi. 
\num{5} (\AmA) tuMba hitavAgi; BajaRriyAgi. 
\enum
\emng
\eentry

\bentry
\word{gorgeousness}
\pron{gAjaRsfnisf}
\gl{\nA}
\bmng
\bnum
\num{1} bahukAMtiyutavAgiruvudu; ujavxla vaNaRtavx; jAjavxlayxmAnavAgiruvike. 
\num{2} atayxlaMkAra. 
\num{3} ADaMbarate; veYBavate. 
\num{4} (padaparxyoVgada \vi) alaMkArayukatxte. 
\num{5} puSakxLate; samaqdidhx. 
\num{6} (\AmA) balu hitavAgiruvike. 
\enum
\emng
\eentry

\bentry
\word[gorget(1)]{gorget}
\pron{gAjiRTf}
\gl{\nA}
\bmng
\bnum
\num{1} (\ca) koraLAkxpu; kaMThakApu. 
\num{2} talekavudi; talegavasu; talemusuku; (heMgasina) taleya matutx katitxna meVle hodedukoLuLxva vasatxrX. 
\num{3} sara; mAle; hAra. 
\num{4} katutx macecx; hakikx \mo vugaLa katitxna meVlina baNaNxda paTeTx, macecx yA baTuTx. 
\enum
\emng

\noindent
\gl{\pagu}
\bmng
 \eng{gorget patch} katutxpaTiTx gurutu; (miliTari samavasatxrXgaLige kutitxge paTiTxya meVle mADiruva) gurutu. 
\emng
\eentry

\bentry
\word[gorget(2)]{gorget}
\pron{gAjiRTf}
\gl{\nA}
\bmng
 gAjiRTf; ashamxriya shasatxrXcikitesxyalilx baLasuva, kAluveyAkArada, ukikxna salakaraNe, shasatxrX. 
\emng
\eentry

\bentry
\word{Gorgio}
\pron{gAjiRO}
\gl{\nA}
\expl{(\bava\ \eng{Gorgios}).}
\bmng
 (jipisx BASeyalilx) jipisxyalalxdava(Lu); jipisxyeVtara; ajipisx. 
\emng
\eentry

\bentry
\word{gorgon}
\pron{gAgaRnf}
\gl{\nA}
\bmng
\bnum
\num{1} (\girxVpu) gAgaRnf; (noVDidavaranunx) noVTadiMda kalAlxgisutitxdadx hAvugUdalina mUvaru soVdariyaralilx obabxLu (\kanmu\ meDusA). 
\num{2} ati kurUpi; GoVra rUpi; BayaMkara, vikArarUpina vayxkitx. 
\num{3} kurUpiNi; noVDalu asahayxvAguva heMgasu. 
\enum
\emng
\eentry

\bentry
\word[gorgonia(1)]{gorgonia}
\pron{gAgoRVnia}
\gl{\nA}
\expl{(\bava\ \eng{gorgoniae, gorgonias}).}
\bmng
gAgoRVniya; kaDala biVsaNige (eMdu kareyuva havaLada saMbaMdhada sUkaSxmX kaDala jiVvi). 
\emng
\eentry

\bentry
\word[gorgonia(2)]{gorgonian}
\pron{gAgoRVnianf}
\gl{\gu}
\bmng
 gAgaRnf savxrUpada; BayaMkara vikArada; asahayxvAguvaSuTx kurUpiyAda. 
\emng
\eentry

\bentry
\word[gorgonia(3)]{gorgonian}
\pron{gAgoRVnianf}
\gl{\gu}
\bmng
 gAgoRVniyada; gAgoRneVsiya vagaRda yA vagaRkekx seVrida. 
\emng
\eentry

\bentry
\word{gorgonian}
\pron{gAgoRVnianf}
\gl{\nA}
\bmng
 gAgoRVniyanf; gAgoRneVsiya vagaRkekx seVrida kaDala jiVvi. 
\emng
\eentry

\bentry
\word{gorgonise}
\pron{gAgaRneYsfZ}
\gl{\sakirx}
\bmng
  = \hyperlink{gorgonize}{gorgonize}. 
\emng
\eentry

\bentry
\word{gorgonize}
\pron{gAgaRneYsfZ}
\gl{\sakirx}
\bmng
\bnum
\num{1} (gAgaRninxnaMte) diTiTxsi noVDu. 
\num{2} BayadiMda satxbadhxgoLisu, nishecxVSaTx mADu, kalAlxgisu. 
\num{3} vashayxsupitx (\eng{hypnosis}) uMTumADu. 
\enum
\emng
\eentry

\bentry
\word{Gorgonzola}
\pron{gAgaRnfsoZVla}
\gl{\nA}
\bmng
 gAgaRnosxVla; hasuvina hAliniMda tayArisida oMdu bageya niVli ciVsu, giNuNx. 
\emng
\eentry

\bentry
\word{gorilla}
\pron{garila}
\gl{\nA}
\bmng
 gorilalx; Aphirxkada samaBAjaka vaqtatxda parxdeVshagaLalilxruva ciMpAMjiyaMtaha, Adare adakikxMta inUnx doDaDxdAda matutx aSATxgi marada Asharxya bayasada, oMdu bageya vAnara. \imglink{gorillafigure}{\raisebox{-0.10cm}[0pt][0pt]{\pdfimage width 0.7cm height 0.4cm {G_Pictures/gorilla.jpg}}} 
\emng
\eentry

\bentry
\word{gorily}
\pron{goVrili}
\gl{\kirxvi}
\bmng
 netatxru celilx; rakatxsikatxvAgi; rakatxleVpitavAgi; rakatxpAlxvitavAgi. 
\emng
\eentry

\bentry
\word[gormandise(1)]{gormandise}
\pron{gAmaRneDxYsfZ}
\gl{\nA}
\bmng
  = \hyperlink{gormandize(1)}{$^1$gormandize}. 
\emng
\eentry

\bentry
\word[gormandise(2)]{gormandise}
\pron{gAmaRneDxYsfZ}
\gl{\kirx}
\bmng
  = \hyperlink{gormandize(2)}{$^2$gormandize}. 
\emng
\eentry

\bentry
\word[gormandize(1)]{gormandize}
\pron{gAmaRneDxYsfZ}
\gl{\nA}
\bmng
  = \hyperlink{gourmandise}{gourmandise}. 
\emng
\eentry

\bentry
\word[gormandize(2)]{gormandize}
\pron{gAmaRneDxYsfZ}
\gl{\sakirx}
\bmng
 gabagabane nuMgu, mukukx; hoTeTxbAkanaMte tinunx (\akirx\ saha). 
\emng
\eentry

\bentry
\word{gormandizer}
\pron{gAmaRneDxYsaZrf}
\gl{\nA}
\bmng
 hoTeTxbAka; tiVnALi; bakAsura; tiMDipoVta. 
\emng
\eentry

\bentry
\word{gormless}
\pron{gAmfRlisf}
\gl{\gu}
\bmng
 (\AmA) pedadx; daDaDx; maMka; budidhxhiVna. 
\emng
\eentry

\bentry
\word{gormlessness}
\pron{gAmfRlisfnisf}
\gl{\nA}
\bmng
 pedadxtana; daDaDxtana; budidhxhiVnate; aviveVka. 
\emng
\eentry

\bentry
\word{gorse}
\pron{gAsfR}
\gl{\nA}
\bmng
 (\savi) gAsfR; yUlekfsx kulakekx seVrida, haLadi hU biDuva oMdu muLuLx pode. 
\emng
\eentry

\bentry
\word{Gorsedd}
\pron{gAseRdf}
\gl{\nA}
\bmng
 velfSx kavigaLU kelfTx dhamARdhikArigaLU (\kanmu\ \eng{eisteddfod} saBege muMce parxtinitayx) seVruva saBe. 
\emng
\eentry

\bentry
\word{gorsy}
\pron{gAsiR}
\gl{\gu}
\bmng
 gAsfR podeya, adakekx saMbaMdhisida, yA A pode beLediruva, tuMbida. 
\emng
\eentry

\bentry
\word{gory}
\pron{goVri}
\gl{\gu}
\bmng
\bnum
\num{1} netatxru celilxda; netatxralilx adidxda; rakatxsikatx; rakatxpAlxvita. 
\num{2} kole, narabali \mo vakekx saMbaMdhisida. 
\enum
\emng
\eentry

\bentry
\word{gosh}
\pron{gASf}
\gl{\BAavayx}
\bmng
deVvareV! deVvarANe! 
\emng

\noindent
\gl{\pagu}
\bmng
 \eng{by gosh} = \hyperlink{gosh}{gosh}. 
\emng
\eentry

\bentry
\word{goshawk}
\pron{gAsfhAkf}
\gl{\nA}
\bmng
gAsfhAkf; moVTu rekekxya doDaDx DeVge (jAti). \imglink{goshqwkfigure}{\raisebox{-0.15cm}[0pt][0pt]{\pdfimage width 0.5cm height 0.7cm {G_Pictures/goshqwk.jpg}}} 
\emng
\eentry

\bentry
\word{Goshen}
\pron{goVSenf}
\gl{\nA}
\bmng
 (\beY) beLakina yA samaqdidhxya sathxLa, parxdeVsha. 
\emng
\eentry

\bentry
\word{gosling}
\pron{gAsilxMgf}
\gl{\nA}
\bmng
\bnum
\num{1} mari varaTe; varaTeya mari. 
\num{2} ananuBavi; tale baliyadava(Lu); eLasu; anuBavavilalxda huDuga(gi). 
\enum
\emng
\eentry

\bentry
\word{go-slow}
\pron{goVsolxV}
\gl{\nA}
\bmng
 (audoyxVgika parxtiBaTaneyAgi mADuva) nidhAna kelasa; kelasadalilx nidhAna, CAnasa toVrisuvudu. 
\emng
\eentry

\bentry
\word{gospel}
\pron{gAsapxlf}
\gl{\nA}
\bmng
\bnum
\num{1} (kirxsatxnu upadeVshisida) shuBanuDi; suvAteR; osage (nuDi); oLunxDi. 
\num{2} kirxsatxna matutx avana apAsalara mata, tatatxvX. 
\num{3} kerxYsatx veVda; deYvoVkatx shAsatxrX. 
\num{4} pArxTeseTxMTf (mata) tatatxvX yA `ivAyxMjalikalf' tatatxvX. 
\num{5} (\eng{Gospel}) (hosa oDaMbaDikeya nAlukx apAsalara garxMthagaLalilx parxkaTavAgiruva) kirxsatxna jiVvana cariterx. 
\num{6} A (nAlukx) suvAteRgaLalolxMdu. 
\num{7} (parxBuBoVjana saMsAkxra kAladalilx Oduva) suvAteRgaLa oMdu BAga. 
\num{8} paramasatayx; satayxsayx satayx; veVdavAkayx; saMshayavilalxde dheYyaRvAgi naMbabahudAdadudx: \eng{takes his dreams for gospel} tananx kanasugaLeV veVdavAkayxgaLeMdu naMbutAtxne. 
\num{9} (obabxnu Acarisuva, naMbiruva yA upadeVshisuva) AdashaR; aBimata; tatavx; vicAra: \eng{the gospel of efficiency} kAyaRdakaSxteyeMba AdashaR. \eng{gospel of soap and water} savxcaCxteya tatatxvX; soVpu niVrina tatatxvX. 
\num{10}  = \hyperlink{gospel song}{gospel song}. 
\enum
\emng

\noindent
\gl{\pagu}
\bmng
 \eng{gospel oath} gAsepxlf parxmANa; suvAteR ANe; suvAteRgaLa sAkiSxyAgi mADida parxmANa, ANe. 
\emng
\eentry

\bentry
\word{gospel-book}
\pron{gAsapxlfbukf}
\gl{\nA}
\bmng
 suvAteR garxMtha; (parxBuBoVjanakAladalilx Oduva) suvAteRgaLiruva pusatxka. 
\emng
\eentry

\bentry
\word{gospeller}
\pron{gAsapxlarf}
\gl{\nA}
\bmng
 (\kerxY) suvAtAR -- pAThaka, vAcaka; suvAtAR Oduga; `kamuyxniyanf' saMsAkxradalilx suvAteR paThisuvavanu. 
\emng

\noindent
\gl{\pagu}
\bmng
 \eng{hot gospeller} 
\banum
\alnum{a} (\kerxY) `pUyxriTanf' kaTATxLu; kaTATx pUyxriTanf matAnuyAyi. 
\alnum{b} kaTATx parxcAraka; ugarx parxcAraka yA caLavaLigAra. 
\eanum
\emng
\eentry

\bentry
\word{gospel-shop}
\pron{gAsapxlfSApf}
\gl{\nA}
\bmng
(niMdAthaRkavAgi) `methaDisfTx' ArAdhana maMdira. 
\emng
\eentry

\bentry
\wordnospeech{gospel side}{gospel side}
\pron{?}
\gl{\nA}
\bmng
 gAsepxlf dikukx; suvAteR Odalu nilulxva, kamuyxniyanf veVdikeya (utatxrada) kaDe. 
\emng
\eentry

\bentry
\wordnospeech{gospel song}{gospel song}
\pron{?}
\gl{\nA}
\bmng
 suvAteRgiVte; suvAteR hADu; nAlavxru suvAtAR boVdhakara upadeVshagaLanunxLaLx hADu. 
\emng
\eentry

\bentry
\wordnospeech{gospel truth}{gospel truth}
\pron{?}
\gl{\nA}
\bmng
\bnum
\num{1} suvAteR satayx; suvAteRyalilxna satayx (vicAra)gaLu. 
\num{2} veVdavAkayx; paramasatayx; suvAteRyaSeTxV satayxvAdadudx. 
\enum
\emng
\eentry

\bentry
\word[gossamer(1)]{gossamer}
\pron{gAsamarf}
\gl{\nA}
\bmng
\bnum
\numi{1} gAsamarf: 
\banum
\alnum{a} (shAMtavAyuvinalilx teVlADutitxruva yA hasuru hulilxna meVle haraDikoMDiruva) cikakx jeVDara huLuvina bale yA hagura, teLu poreyaMtha padAthaR. 
\alnum{b} idara oMdu dAra, eLe. 
\alnum{c} ati navirAda, nisasxtavxda padAthaR; keVvala jALu padAthaR. 
\eanum
\numie
\num{2} bahu sUKaSxmX neyegx, jAla. 
\enum
\emng
\eentry

\bentry
\word[gossamer(2)]{gossamer}
\pron{gAsamarf}
\gl{\gu}
\bmng
\bnum
\num{1} gAsamarinaMtha. 
\num{2} ati navirAda, haguravAda. 
\num{3} (jeVDarabaleyaMte) jALAda; nisasxtavxda. 
\enum
\emng
\eentry

\bentry
\word{gossamered}
\pron{gAsamaDfR}
\gl{\gu}
\bmng
\bnum
\num{1} gAsamarinaMte ati -- haguravAda, teLuvAda, sapurAda, navirAda. 
\num{2} keVvala jALAda; nisasxtavxvAda. 
\enum
\emng
\eentry

\bentry
\word{gossamery}
\pron{gAsamari}
\gl{\gu}
\bmng
  = \hyperlink{gossamered}{gossamered}. 
\emng
\eentry

\bentry
\word[gossip(1)]{gossip}
\pron{gAsipf}
\gl{\nA}
\bmng
\bnum
\num{1} (\pArxparx) geLati; saligeya, Apatx, paricita vayxkitx (\kanmu\ heMgasu). 
\hypertarget{gossip(1)2}{} 
\num{2} haraTALi haraTemalalx(lilx); sudidxmalalx(lilx); goDuDx haraTeya sudidx heVLuva vayxkitx (\kanmu\ heMgasu). 
\num{3} goDuDxharaTe; kADuharaTe; buDavilalxda gALimAtu; biVdi mAtu; (\kanmu\ vayxkitxgaLa yA sAmAjika GaTanegaLa viSayavAgi) saliVsAgi aDiDx AtaMkavilalxde ADuva mAtu, bareyuva baravaNige yA haraTe. 
\enum
\emng
\eentry

\bentry
\word[gossip(2)]{gossip}
\pron{gAsipf}
\gl{\akirx}
\bmng
\bnum
\num{1} haraTehoDi; haraTekocucx; (kelasavilalxde kuLitu, saliVsAgi, haguravAgi) haraTu; joLuLx mAtanADu. 
\num{2} haraTeya sheYliyalilx bare. 
\enum
\emng
\eentry

\bentry
\wordnospeech{gossip column}{gossip column}
\pron{?}
\gl{\nA}
\bmng
 (vaqtatxpatirxke \mo vugaLalilxna) haraTe kAlaM; haraTe aMkaNa; vayxkitx. sAmAjika \mo\ GaTanegaLa bagegx buDavilalxde, aDiDx AtaMkagaLilalxde bareyuva barahagaLu. 
\emng
\eentry

\bentry
\word{gossiper}
\pron{gAsiparf}
\gl{\nA}
\bmng
  = \hyperlink{gossip(1)2}{$^1$gossip (2)}. 
\emng
\eentry

\bentry
\word{gossipry}
\pron{gAsipirx}
\gl{\nA}
\bmng
\bnum
\num{1} haraTe. 
\num{2} haraTe guMpu; haraTemalalxra guMpu. 
\enum
\emng
\eentry

\bentry
\word{gossipy}
\pron{gAsipi}
\gl{\gu}
\bmng
\bnum
\num{1} haraTeyaMtha. 
\num{2} haraTuva parxvaqtitxya; haraTepirxya: \eng{a gossipy farmer} haraTepirxya reYta. 
\num{3} haraTeyiMda tuMbida: \eng{a gossipy chronicle} haraTeyiMda tuMbida cariterx. 
\enum
\emng
\eentry

\bentry
\word{gossoon}
\pron{gAsUnf}
\gl{\nA}
\bmng
 (airalxMDf) huDuga; heYda; ALumaga. 
\emng
\eentry

\bentry
\word{got}
\pron{gATf}
\gl{\kirx}
\bmng
 \eng{get} dhAtuvina \BU\ matutx \BUkaq. 
\emng
\eentry

\bentry
\word{Goth}
\pron{gAtf}
\gl{\nA}
\bmng
\bnum
\num{1} gAtiVya; (\kirxsha\ \eng{3--5}neya shatamAnagaLalilx pUvaRpashicxma roVmanf sAmArxjayxgaLa oLanugigx, iTali, phArxnfsx, matutx sepxVnf deVshagaLalilx rAjayx sAthxpisida) oMdu jamaRnf kuladavanu. 
\num{2} oraTa; anAgarika; babaRra. 
\num{3} tiLivilalxdavanu; mUDha; ajacnx. 
\enum
\emng
\eentry

\bentry
\word{Gotham}
\pron{gATamf}
\gl{\nA}
\bmng
\bnum
\num{1} mUKaRpura; daDUDxru; shuMThara nagari; aviveVkigaLa, mUKaRra -- Uru. 
\num{2} (\ucAcx -- gATamf yA goVtamf). (\ame) (\AmA) nUyxyAkfR nagara. 
\enum
\emng

\noindent
\gl{\nuga}
\bmng
 \eng{wiseman of Gotham} daDUDxrina baqhasapxti; aviveVki; daDaDx. 
\emng
\eentry

\bentry
\word{Gothamite}
\pron{gATameYTf, goVtameYTf}
\gl{\nA}
\bmng
 (\ame) (\AmA) nUyxyAkiRga; nUyxyAkfR nagaradava(Lu). 
\emng
\eentry

\bentry
\word[Gothic(1)]{Gothic}
\pron{gAtikf}
\gl{\nA}
\bmng
\bnum
\num{1} gAtikf; gatikf; gAtf jana yA gAtf BASe. 
\num{2} (\vAshi) gAtikf sheYli, saMparxdAya; pashicxma yUroVpinalilx \kirxsha\ \eng{12--16}neya shatamAnagaLalilx baLakeyalilxdadx vAsutx sheYli (\kanmu\ cUpu kamAnina sheYli). 
\num{3} (mudarxNa) gAtikf; jamaRnf TeYpu; kapapxkaSxra. \imglink{gothicfigure}{\raisebox{-0.15cm}[0pt][0pt]{\pdfimage width 0.5cm height 0.5cm {G_Pictures/gothic.jpg}}} 
\enum
\emng
\eentry

\bentry
\word[Gothic(2)]{Gothic}
\pron{gAtikf}
\gl{\gu}
\bmng
\bnum
\num{1} `gAtiVya'; `gAtf' janara yA `gAtf' BASeya. 
\num{2} `gAtikf' (vAsutx) sheYliya (\kanmu\ cUpu kamAnina sheYliya). 
\num{3} oraTAda; anAgarika; asaMsakxqqta; vikaTavAda. 
\numi{4} (mudarxNa) gAtikf: 
\banum
\alnum{a} jamaRnf (TeYpina) kapapxkaSxrada. 
\alnum{b} akaSxrada meVle matutx keLage reVKAveYKarigaLilalxda. 
\eanum
\numie
\enum
\emng
\eentry

\bentry
\word{Gothically}
\pron{gAtikali}
\gl{\kirxvi}
\bmng
\bnum
\num{1} `gAtiVya'vAgi; gAtikf (vAsutx) sheYliyalilx. 
\num{2} oraTAgi; anAgarika riVtiyalilx; asaMsakxqqtavAgi; vikaTavAgi; citarxvicitarxvAgi. 
\enum
\emng
\eentry

\bentry
\word{Gothicise}
\pron{gAtiseYsfZ}
\gl{\kirx}
\bmng
  = \hyperlink{Gothicize}{Gothicize}. 
\emng
\eentry

\bentry
\word{Gothicism}
\pron{gAtisisaZmf}
\gl{\nA}
\bmng
\bnum
\num{1} `gAtiVyate'; gAtikf nuDigaTuTx. 
\num{2} (\vAshi) `gAtikf' parxvaqtitx yA sheYli(yA anukaraNe). 
\num{3} oraTutana; anAgarika, asaMsakxqqta, vikaTa -- naDavaLike. 
\enum
\emng
\eentry

\bentry
\word{Gothicize}
\pron{gAtiseYsfZ}
\gl{\sakirx}
\bmng
 gAtiVkarisu; madhayxyugada gAtikf sheYli, riVti, \mo vakekx aLavaDisu, parivatiRsu. 
\emng

\noindent
\gl{\akirx}
\bmng
 gAtiVkaqtavAgu; madhayxyugada gAtikf riVti \mo vanunx iSaTxpaDu. 
\emng
\eentry

\bentry
\word{Gothish}
\pron{gAtiSf}
\gl{\gu}
\bmng
 oraTa; anAgarika. 
\emng
\eentry

\bentry
\word{go-to-meeting}
\pron{goVTumITiMgf}
\gl{\gu}
\bmng
 (hAyxTu, uDupu, \mo vugaLa \vi) cacuR yoVgayx; cacoRVcita; caciRge hoVguvAga dharisuva, dharisatakakx. 
\emng
\eentry

\bentry
\word{gotta}
\pron{gATa}
\gl{}
\bmng
(\AmA\ yA \asaM) \eng{(have) got a} athavA \eng{(have) got to} eMbudaraapaBarxMsha rUpa.
\emng
\eentry

\bentry
\word{gotten}
\pron{gATanf}
\gl{\kirx}
\bmng
 (\ame \parx) \eng{get} dhAtuvina \BUkaq. 
\emng
\eentry

\bentry
\word{Gotterdammerung}
\pron{gaTaDeRmaruMgf}
\gl{\nA}
\bmng
\bnum
\num{1} (nAveRyavara purANa) deVvAsura yudadhx. 
\num{2} (\rUpa) (yAvudeV ALivxke, saMsethx, \mo vugaLa) saMpUNaR patana; avasAna. 
\enum
\emng
\eentry

\bentry
\word{got-up}
\pron{gATfapf}
\gl{\gu}
\bmng
 kaqtaka siMgArada; kaqtakAlaMkArada; pariNAma uMTumADalu yA moVsagoLisalu kaqtakavAgi siMgarisida yA rUpisida. 
\emng
\eentry

\bentry
\wordf{gouache}
\pron{guASf}
\gl{\nA}
\expl{\F}
\bmng
 (citarxkale) guASf: 
\banum
\alnum{a} niVrinalilx aredu, jeVnutupapx matutx goVMdina aMTinalilx maMdagoLisida apAradashaRka baNaNxgaLu. 
\alnum{b} I baNaNxgaLiMda citarx bareyuva sheYli. 
\alnum{c} hiVge bareda citarx. 
\eanum
\emng
\eentry

\bentry
\word{Gouda}
\pron{gwDa}
\gl{\nA}
\bmng
 `gwDa' (giNuNx): hAleMDina `gwDa' eMbalilx tayArisuva giNuNx. 
\emng
\eentry

\bentry
\word[gouge(1)]{gouge}
\pron{gw(gU)jf}
\gl{\nA}
\bmng
\bnum
\num{1} (maragelasa, shilapxkale, shasatxrXcikitesxgaLalilx baLasuva) uguruLi; oLabAgida aluguLaLx uLi. \imglink{gougesfigure}{\raisebox{-0.15cm}[0pt][0pt]{\pdfimage width 0.7cm height 0.5cm {G_Pictures/gouges.jpg}}} 
\num{2} (mara, loVha, \mo vugaLa meVle uLiyiMda mADida) gADi; toVDu; jADu; uLi toVDu. 
\num{3} (\ame) (\AmA) kaLaLxtana yA moVsa. 
\enum
\emng
\eentry

\bentry
\word[gouge(2)]{gouge}
\pron{gw(gU)jf}
\gl{\sakirx}
\bmng
\bnum
\num{1} (uguruLiyiMda) ketutx; katatxrisu. 
\num{2} (kAkaRnunx, nALiguMDiyanunx uguruLiyiMdaloV eMbaMte) ketutx; toVDu. 
\num{3} (\kanmu\ obabxna kaNaNxnunx hebebxTiTxniMda mITi) kiVLu; kitutxhAku. 
\numi{4} (\ame) (\AmA) 
\banum
\alnum{a} dagA hAku; moVsa mADu. 
\alnum{b} haNa kiVLu; duDuDx doVcu. 
\eanum
\numie
\enum
\emng

\noindent
\gl{\akirx}
\bmng
 (\AseTxrXV) Opalf shilegAgi (nela, BUmi) age. 
\emng
\eentry

\bentry
\word{goulash}
\pron{gUlAYx(lA)Sf}
\gl{\nA}
\bmng
\bnum
\num{1} gUlASf; mAMsa, tarakAri beVyisi cenAnxgi masAle hAki tayArisida oMdu tinisu. 
\num{2} (birxDfjx isipxVTf ATa) (elegaLa) maruhaMcike; punaH haMcuvudu; omemx haMcida elegaLanunx (ATagAraru) joVDisiTuTxkoMDa meVle kalasade hAgeyeV punaH haMcuvudu. 
\enum
\emng
\eentry

\bentry
\word{gourami}
\pron{guara(rA)mi}
\gl{\nA}
\bmng
 gurAmi: 
\banum
\alnum{a} AgenxVya ESAyxda, tinanxbahudAda doDaDx siVniVru mInu. 
\alnum{b} ideV jAtige seVrida, \sA\ jaloVdAyxnagaLalilxDuva, saNaNx mInu. 
\eanum
\emng
\eentry

\bentry
\word{gourd}
\pron{guaDfR}
\gl{\nA}
\bmng
\bnum
\num{1} gaDusAda sipepx matutx heVraLavAda tiruLiruva (kuMbaLa, kalalxMgaDi, soVre, \mo) doDaDxkAyi yA aMtaha kAyibiDuva baLiLx. 
\num{2} buruDe; (soVre, \mo\ kAyigaLa) tiruLu tegedu oNagisida, pAterxyaMtaha buruDe. 
\enum
\emng
\eentry

\bentry
\word{gourdful}
\pron{guaDfRphulf}
\gl{\nA}
\bmng
 oMdu buruDe; buruDe tuMbuvaSuTx yAvudeV padAthaR. 
\emng
\eentry

\bentry
\word[gourmand(1)]{gourmand}
\pron{guarfmaMDf, gUrfmAnf}
\gl{\gu}
\bmng
\bnum
\num{1} BoVjanapirxyateya; maqSATxnanxpirxyateya. 
\num{2} hoTeTxbAkatanada; tiVnALitanada. 
\enum
\emng
\eentry

\bentry
\word[gourmand(2)]{gourmand}
\pron{guarfmaMDf, gUrfmAnf}
\gl{\nA}
\bmng
\bnum
\num{1} BoVjanapirxya; BoVjanarasajacnx; maqSATxnanxpirxya. 
\num{2} hoTeTxbAka; tiVnALi; bakabakane tinunxvavanu. 
\enum
\emng
\eentry

\bentry
\word{gourmandise}
\pron{guarfmAMDiVsf}
\gl{\nA}
\bmng
\bnum
\num{1} hoTeTxbAkatana; tiVnALitana. 
\num{2} hoTeTxbAka aBAyxsagaLu. 
\enum
\emng
\eentry

\bentry
\word{gourmandism}
\pron{guarfmaMDisaZmf}
\gl{\nA}
\bmng
 BoVjanarasikate; maqSATxnanxpirxyate; rasagavaLadAse. 
\emng
\eentry

\bentry
\word{gourmet}
\pron{guameRV}
\gl{\nA}
\bmng
 BoVjanarasika; rasaBakaSxyX, BoVjayx, pAniVyagaLa rasajacnx. 
\emng
\eentry

\bentry
\word{gout}
\pron{gwTf}
\gl{\nA}
\bmng
\bnum
\num{1} saMdhivAta; deVhada saNaNx kiVlugaLu (\kanmu\ kAlina hebebxTuTx) Udi, kiVlugaLalilx siVmesuNaNxdaMtha uMDegaLuMTAgi thaTaTxne keraLuva oMdu roVga. 
\num{2} gwTf roVga; (gwTf noNagaLa mUlaka) goVdige tagaluva oMdu terada roVga. 
\num{3} (\kanmu\ rakatxda) toTuTx; kale; hani. 
\num{4} celilxdudx; siMcita; cimukisidudx. 
\enum
\emng

\noindent
\gl{\pagu}
\bmng
\bnum
\num{1} \eng{poor man's gout} alApxhAradiMda huTuTxva oMdu terada saMdhivAtaroVga. 
\num{2} \eng{rich man's gout} atAyxhAra matutx kuDitadiMda baruvudeMdu heVLuva oMdu terada vAtaroVga. 
\enum
\emng
\eentry

\bentry
\word{gout-fly}
\pron{gwTfphelxY}
\gl{\nA}
\bmng
 gwTfnoNa; goVdige oMdu bageya roVga barisuva noNa. 
\emng
\eentry

\bentry
\word{goutily}
\pron{gwTili}
\gl{\kirxvi}
\bmng
 saMdhivAtaroVgadiMda; saMdhivAta roVgadalilx. 
\emng
\eentry

\bentry
\word{goutiness}
\pron{gwTinisf}
\gl{\nA}
\bmng
 saMdhivAtate; saMdhivAta hiDidiruvike yA hiDidaMtiruvike. 
\emng
\eentry

\bentry
\word{goutweed}
\pron{gwTfviVDf}
\gl{\nA}
\bmng
 jerADfR eMba beVru. 
\emng
\eentry

\bentry
\word[gouty(1)]{gouty}
\pron{gwTi}
\gl{\gu}
\bmng
 saMdhivAtada; saMdhivAtakekx saMbaMdhisida. 
\emng
\eentry

\bentry
\word[gouty(2)]{gouty}
\pron{gwTi}
\gl{\nA}
\expl{(\bava\ \eng{gouties}).}
\bmng
meVlojxVDugaLu. 
\emng
\eentry

\bentry
\word{govern}
\pron{gavanfR}
\gl{\sakirx}
\bmng
\bnum
\num{1} (hakikxniMda) ALu; Adhipatayx naDesu. 
\num{2} ADaLita, adhikAra naDesu, nivaRhisu; rAjayxBAra mADu; (sakARrada, parxjegaLa) (niVtiniyama, kAyaR kalApagaLu, kArubAru, \mo vanunx) niraMkushavAgi yA saMvidhAnAtamxkavAgi naDesu, nivaRhisu. 
\num{3} (oMdu saMsethx \mo vugaLa) kAyaRnivaRhisu; kelasa kAyaRgaLanunx, kAyaRkarxmagaLanunx niyamabadadhxvAgi naDesu. 
\num{4} (koVTeya, paTaTxNada meVle) seYnikAdhikAra hoMdiru, paDediru. 
\num{5} (vayxkitx, avana kAyaRgaLu, GaTanegaLa gati, pariNAma, \mo vanunx) ALu; niyaMtirxsu; parxBAvagoZLisu; vayxvasethxgoLisu; rUpisu; nishacxyisu. 
\num{6} (kAyARvaLi) naDesu; nideRVshisu. 
\num{7} (\AtAmx) (yAvudoV oMdu riVtiyalilx) naDeduko; vatiRsu: \eng{let him know how to govern himself} tAnu heVge naDedukoLaLxbeVkeMbudanunx avanu aritukoLaLxli. 
\num{8} (koVpa \mo vanunx) aDagisu; nigarxhisu; damana mADu. 
\num{9} (tananxnunx tAnu) niyaMtirxsu; hatoVTiyalilxTuTxko: \eng{govern one's temper} koVpavanunx hatoVTiyalilxTuTxko. 
\num{10} (vayxvahAra niNaRyadalilx yAvudoV oMdakekx) niNARyakavAgiru; shAsanavAgi, vidhiyAgi, sUtarxvAgi, tatatxvXvAgi -- iru. 
\num{11} (vayxvahAra niNaRyadalilx oMdu saMdaBaR inonxMdakekx) anavxyavAgu; anavxyisuvaMtiru: \eng{the law stated there clearly governs this case} alilx ulelxVKisiruva kAnUnu (kAyide) I mokadadxmege sapxSaTxvAgi anavxyisutatxde. 
\num{12} (\vAyx) nAmapadadoDane, viBakitxyoDane -- anavxyavAgu, anavxyisu: \eng{a transitive verb governs a noun in the objective case} sakamaRka kirxyApadavu divxtiVyA viBakitxya nAmapadadoDane anavxyavAgutatxde. 
\enum
\emng

\noindent
\gl{\akirx}
\bmng
\bnum
\num{1} (koVTeyanunx, paTaTxNavanunx) seYnika adhikAradiMda ALu. 
\num{2} savxtaH ALu; sakARrada ADaLita naDesu: \eng{the king reigns but does not govern} rAja parxButavx naDesutAtxne Adare sakARra naDesuvudilalx; rAja doretana maDutAtxne, Adare rAjayxBAra mADuvudilalx, savxtaH ALuvudilalx (eMdare tAneV ADaLita naDesade, ADaLitagAraranunx neVmisutAtxne). 
\num{3} parxbala vAyxpitxyuLaLxdAdxgiru; keYmeVlAgiru; visheVSa parxBAva biVru: \eng{yet chance will govern at last} kaTaTxkaDeyalilx adaqSaTxda keYyeV meVlAgutatxde. 
\enum
\emng
\eentry

\bentry
\word{governability}
\pron{gavanaRbiliTi}
\gl{\nA}
\bmng
\bnum
\num{1} ALalu sAdhayxvAgiruvike. 
\num{2} nivaRhaNiVyate; nivaRhaNa, niyaMtarxNa -- sAdhayxte. 
\num{3} anavxya sAdhayxte; Anavxyikate. 
\enum
\emng
\eentry

\bentry
\word{governable}
\pron{gavanaRbflf}
\gl{\gu}
\bmng
\bnum
\num{1} ALivxkege oLapaDabalalx; ALabahudAda; adhikAra naDesalu sAdhayxvAguva. 
\num{2} nivaRhaNasAdhayx; nivaRhaNiVya. 
\num{3} niyaMtirxsabahudAda; hatoVTi mADalu sAdhayxviruva. 
\num{4} Anavxyika; anavxyasAdhayx: \eng{likely to be governable by prudent counsel} viveVkayuta salahege oLapaDuva saMBavavidudx. 
\enum
\emng
\eentry

\bentry
\word{governance}
\pron{gavanaRnfsx}
\gl{\nA}
\bmng
\bnum
\num{1} ALike; Adhipatayx; ADaLita. 
\num{2} adhikAra; parxButavx. 
\num{3} ADaLita -- riVti yA kAyaR. 
\num{4} hatoVTi; aMke; niyaMtarxNa. 
\enum
\emng
\eentry

\bentry
\word{governess}
\pron{gavaniRsf}
\gl{\nA}
\bmng
 (\kanmu\ maneyalilx makakxLige pATha heVLikoTuTx avara meVlivxcAraNe noVDikoLuLxva) gaqhashikaSxki; mane meVDaM. 
\emng
\eentry

\bentry
\word{governess-cart}
\pron{gavaniRsfkATfR}
\gl{\nA}
\bmng
 (\birx) (\ca) gavaneRsf gADi; eraDu eduru baduru piVThagaLiruva, eraDu cakarxda, haguravAda baMDi. 
\emng
\eentry

\bentry
\wordnospeech{governing body}{governing body}
\pron{?}
\gl{\nA}
\bmng
 (Asapxterx, pAThashAle, \mo vugaLa) adhikAra maMDali; ADaLita samiti; nivARhaka maMDali. 
\emng
\eentry

\bentry
\word{government}
\pron{gavanfRmeMTf, gavamaMTf}
\gl{\nA}
\bmng
\bnum
\num{1}  = \hyperlink{governance}{governance}. 
\num{2} Adhipatayx; rAjayx; pArxMta rAjayxpAlaru ALutitxruva deVshaBAga, pArxMta. 
\num{3} ADaLita -- vayxvasethx, kirxye yA karxma. 
\num{4} rAjayx (shAsana) padadhxti. 
\num{5} sakARra; maMtirxmaMDala; saciva saMpuTa. 
\num{6} (\vAyx) anavxya; AkAMkeSx; oMdu padakUkx adara AsharxyavAda inonxMdu padakUkx iruva saMbaMdha. 
\enum
\emng

\noindent
\gl{\pagu}
\bmng
\bnum
\num{1} \eng{form a Government} (parxdhAnamaMtirxya \vi) saciva saMpuTa racisu; maMtirxmaMDala EpaRDisu; sakARra racisu. 
\num{2} \eng{Government House} rAjayxpAla Bavana; rAjayxpAlara adhikaqta nivAsa. 
\num{3} \eng{Government paper, securities} sakARrada sAlapatarxgaLu, KajAneya huMDigaLu, \mo vu. 
\enum
\emng
\eentry

\bentry
\word{governmental}
\pron{gavanfRmeMTalf}
\gl{\gu}
\bmng
 sakARriV; sakARrada. 
\emng
\eentry

\bentry
\word{governmentally}
\pron{gavanfRmeMTali}
\gl{\kirxvi}
\bmng
sakARriVyavAgi; sakARrakekx saMbaMdhisida riVtiyalilx; sakARrada mUlakavAgi. 
\emng
\eentry

\bentry
\word{governor}
\pron{gavanaRrf}
\gl{\nA}
\bmng
 gavanaRru: 
\banum
\alnum{a} rAjayxpAla; ALuvavanu; rAjayxBAra mADuvavanu. 
\alnum{b} maMDalAdhipati; pArxMtAdhipati; oMdu pArxMtavanunx, paTaTxNavanunx ALalu niyamitanAda adhikAri. 
\alnum{c} (birxTiSf cakArxdhipatayxdalilx savxyamADaLita paDediruva) Dominiyaninxnalilx yA vasAhatinalilx samArxTara parxtinidhi. 
\alnum{d} (\ame\ saMyukatx saMsAthxnagaLalilx) parxti saMsAthxnada gavanaRru, kAyARdhayxkaSx. 
\alnum{e} dugARdhipati; koVTeya yA koVTeya seYnayxda adhipati. 
\alnum{f} (oMdu saMsethxya) ADaLita maMDaLiya sadasayx yA adhayxkaSx. 
\alnum{g} (baMdiVKAneya) muKAyxdhikAri. 
\hypertarget{governor(h)}{} 
\alnum{h} (\ashi) obabxna yajamAna; daNi. 
\alnum{i} (obabxna) taMde. 
\alnum{j} (saMboVdhaneyalilx) sAvxmI! yajamAnareV! 
\alnum{k} (\yaMshA) niyaMtarxka; yaMtarxda calane oMdeV samanAgiruvaMte adakekx odagisuva anila, habe, niVru, \mo vanunx niyaMtirxsuva sAdhana. 
\alnum{l} oMdu bageya gALada noNa. 
\eanum
\emng
\eentry

\bentry
\word{governorate}
\pron{gavanaRreVTf}
\gl{\nA}
\bmng
\bnum
\num{1} gavanaRrana (ALivxkeyalilxruva) pArxMta yA deVshada BAga (\kanmu\ tukiRya ATomanf sAmArxjayxda pArxMta matutx anaMtara IjipfTx). 
\num{2} gavanaRrana nivAsa. 
\num{3} gavanaRrfgiri; gavanaRrana adhikAra, hudedx. 
\enum
\emng
\eentry

\bentry
\wordnospeech{governor general}{governor general}
\pron{?}
\gl{\nA}
\bmng
 gavanaRrf janaralf; (tananx keY keLage maMDalAdhipatigaLanunxLaLx) mahAmaMDalAdhipati. 
\emng
\eentry

\bentry
\wordnospeech{governor generalship}{governor generalship}
\pron{?}
\gl{\nA}
\bmng
 gavanaRrf janaralf padavi, hudedx. 
\emng
\eentry

\bentry
\word{governorship}
\pron{gavanaRrfSipf}
\gl{\nA}
\bmng
 gavanaRrf padavi, hudedx. 
\emng
\eentry

\bentry
\wordnospeech{Govt.}{Govt.}
\pron{?}
\gl{\saMkiSx}
\bmng
 \eng{Government.} 
\emng
\eentry

\bentry
\word{gowan}
\pron{gwanf}
\gl{\nA}
\bmng
 (sAkxTflaMDf) DeYsi; oMdu bageya saNaNx biLi yA haLadi aDavi hUvu, araNayxpuSapx. 
\emng
\eentry

\bentry
\word{gowk}
\pron{gwkf}
\gl{\nA}
\bmng
 (\pArxparx) 
\bnum
\num{1} koVgile. 
\num{2} heDaDx; gAMpa; gugugx. 
\enum
\emng
\eentry

\bentry
\word[gown(1)]{gown}
\pron{gwnf}
\gl{\nA}
\bmng
\bnum
\num{1} gwnu; meVlaMgi; niluvaMgi; dogale; saDilavAda udadxneya meVluDupu (\kanmu\ heMgasina, \sA\ aMdacaMdagaLa toVkeRya uDige). 
\num{2} purAtana roVmanara ToVga, utatxriVya, meVlarive. 
\num{3} (AlaDxrfmanf, nAyxyAdhiVsha, vakiVla, pAdirx, vishavxvidAyxnilaya, kAleVju yA shAle, \mo vugaLa sadasayxru dharisuva, udoyxVgasUcakavAda yA avaravarige gotutx mADiruva beVre beVre AkAragaLa) niluvaMgi; gwnu. 
\enum
\emng

\noindent
\gl{\pagu}
\bmng
\bnum
\num{1} \eng{arms, gown} yudadhx matutx shAMti. 
\num{2} \eng{dinner gown} BoVjana niluvaMgi; UTa mADuvAga toDuva meVluDupu. 
\num{3} \eng{tea gown} TiVgwnu; upAhAra niluvaMgi. 
\num{4} \eng{town and gown} (\kanmu\ AkfsxphaDfR matutx keVMbirxjf nagaragaLalilxna) vishavxnidAyxnilayada sadaseyxVtararu matutx sadasayxru, nagaravAsigaLu matutx vishavxvidAyxnilayada sadasayxru. 
\enum
\emng
\eentry

\bentry
\word[gown(2)]{gown}
\pron{gwnf}
\gl{\sakirx}
\bmng
 (\kanmu\ \BUkaq dalilx) niluvaMgi toDu; meVlaMgi dharisu: \eng{beautifully gowned women} suMdaravAda niluvaMgi toTaTx heMgasaru. 
\emng
\eentry

\bentry
\word{gownsman}
\pron{gwnfs'manf}
\gl{\nA}
\bmng
 niluvaMgidhAri: 
\banum
\alnum{a} ayoVdha; seYnayxda seVveyalilxlalxdavanu; yudodhxVdoyxVgiyalalxdavanu. 
\alnum{b} vishavxvidAyxnilayada sadasayx. 
\eanum
\emng
\eentry

\bentry
\word{goy}
\pron{gAyf}
\gl{\nA}
\expl{(\bava\ \eng{goys} yA \eng{goyim} \ucAcx\ gAimf).}
\bmng
yehUdeyxVtara (vayxkitx). 
\emng
\eentry

\bentry
\wordnospeech{GP}{GP}
\pron{?}
\gl{\saMkiSx}
\bmng
\bnum
\num{1} \eng{general practitioner.} 
\num{2} \eng{Grand Prix.} 
\enum
\emng
\eentry

\bentry
\wordnospeech{GP. Capt.}{GP. Capt.}
\pron{?}
\gl{\saMkiSx}
\bmng
 \eng{Group Captain.} 
\emng
\eentry

\bentry
\wordnospeech{GPI}{GPI}
\pron{?}
\gl{\saMkiSx}
\bmng
 \eng{general paralysis of the insane.} 
\emng
\eentry

\bentry
\wordnospeech{GPO}{GPO}
\pron{?}
\gl{\saMkiSx}
\bmng
\bnum
\num{1} \eng{General Post Office.} 
\num{2} (\ame) \eng{Government Printing Office.} 
\enum
\emng
\eentry

\bentry
\wordnospeech{GR}{GR}
\pron{?}
\gl{\saMkiSx}
\bmng
 \eng{King George.} 
\emng
\eentry

\bentry
\wordnospeech{gr.}{gr.}
\pron{?}
\gl{\saMkiSx}
\bmng
\bnum
\num{1} \eng{grain(s).} 
\num{2} \eng{gram(s).} 
\num{3} \eng{grey}. 
\num{4} \eng{gross.} 
\enum
\emng
\eentry

\bentry
\word{Graafian}
\pron{gArxphianf}
\gl{\gu}
\bmng
 (Dacf aMgaracanAshAsatxrXjacnx) gArxphf eMbAtana, avanu kaMDuhiDida yA avanige saMbaMdhisida. 
\emng
\eentry

\bentry
\wordnospeech{Graafian follicle}{Graafian follicle}
\pron{?}
\gl{\nA}
\bmng
 gArxphfkoVsha; gArxphf kaMDuhiDida, sasatxniyoMdara aMDAshayadalilx pakavxvAgutitxruva aMDavanAnxvarisiruva ciVladaMtha oMdu racane. 
\emng
\eentry

\bentry
\wordnospeech{Graafian vesicle}{Graafian vesicle}
\pron{?}
\gl{\nA}
\bmng
  = \hyperlink{Graafian follicle}{Graafian follicle}. 
\emng
\eentry

\bentry
\word[grab(1)]{grab}
\pron{gArxYxbf}
\gl{\kirx}
\expl{(\BU\ matutx \BUkaq\ \eng{grabbed,} \vakaq\ \eng{Grabbing}).}
\bmng
\emng

\noindent
\gl{\sakirx}
\bmng
\bnum
\num{1} thaTaTxne hiDiduko; Pakakxne seLeduko; tuDuku: \eng{he grabbed me by the collar} avanu katutxpaTiTxyiMda nananunx thaTaTxne hiDidukoMDanu. 
\num{2} doVcu; anAyxyadiMda kasidiTuTxko, kitutxko; loVBadiMda sAvxdhiVnapaDisiko: \eng{to grab land} BUmiyanunx doVcikoLaLxlu. 
\num{3} serehiDi; vashapaDisiko; baMdhisu; hiDidu nililxsu: \eng{a criminal whom we won't be able to grab in a hurry} nAvu Aturadalilx baMdhisalu sAdhayxvAgada aparAdhi. 
\numi{4} (\ashi) (vayxkitxya) 
\banum
\alnum{a} gamanaseLe. 
\alnum{b} manasisxge hiDisu. 
\eanum
\numie
\enum
\emng

\noindent
\gl{\akirx}
\bmng
\bnum
\num{1} hiDiyalu keYhAku; kUDale hiDiduko: \eng{he grabbed at the opportunity} avanu avakAshavanunx kUDale hiDidukoMDanu. 
\num{2} (moVTAruvAhanada berxVkina \vi) joVrAgi yA thaTaTxne niMtu kulukuvaMte -- hiDi, vatiRsu, kelasa mADu. 
\enum
\emng
\eentry

\bentry
\word[grab(2)]{grab}
\pron{gArxYxbf}
\gl{\nA}
\bmng
\bnum
\num{1} Pakakxneya -- hiDita, muSiTx, tuDuku. 
\num{2} thaTaTxne tuDukuva parxyatanx; Pakakxne hiDidukoLuLxva parxyatanx. 
\num{3} kasidukoLuLxva aBAyxsa; tuDukATa; doVcATa. 
\num{4} (\kanmu\ rAjakiVyadalilx yA vANijayxdalilx) doVcikoLuLxva naDavaLike; apaharaNa niVti; suligeya -- vataRne, vayxvahAra. 
\num{5} baMdhisuvava yA baMdhisuvaMthadu; sere hiDiyuvava yA sere hiDiyuvaMthadu: \eng{land grab} BUmi kasiyuvava. 
\num{6} (\yaMshA) (BadarxvAgi hiDidukoLuLxva) kalxcucx; hiDike. 
\num{7} doVci -- sipxVTu; meVjiniMda kelavu elegaLanunx thaTaTxne egarisikoLuLxva oMdu bageya makakxLa isipxVTATa. 
\enum
\emng

\noindent
\gl{\nuga}
\bmng
 \eng{up for grabs} (\ame) (\ashi) sulaBavAgi sikukxva; sulaBavAgi hArisikoLaLxbahudAda: \eng{right now every position is up for grabs} sadayxdalilx elAlx sAthxnavU sulaBavAgi sikukxtatxde. 
\emng
\eentry

\bentry
\word{grab-bag}
\pron{gArxYxbfbAYxgf}
\gl{\nA}
\bmng
 = \hyperref{kandict_l.pdf}{L}{lucky dip}{lucky dip}. 
\emng
\eentry

\bentry
\word{grabber}
\pron{gArxYxbarf}
\gl{\nA}
\bmng
\bnum
\num{1} thaTaTxne -- hiDidukoLuLxvava, hiDidukoLuLxvaMthadudx, tuDukugAra. 
\num{2} doVcugAra; doVcuga; apahAraka; apaharaNakAra; kasidiTuTxkoLuLxvavanu yA kasidiTuTxkoLuLxvaMthadu. 
\enum
\emng
\eentry

\bentry
\word{grabble}
\pron{gArxYxbflf}
\gl{\akirx}
\bmng
\bnum
\num{1} taDakADu; taDavu; taDavarisu; (oMdu vasutxvigAgi) taDavutatx huDuku: \eng{grabbled about in her bag} avaLa ciVladalilx taDakADidaLu. 
\num{2} (keYkAlugaLanunx Uri, aneVka veVLe, yAvudeV vasutxvigAgi) haridADu. 
\enum
\emng
\eentry

\bentry
\word{grabby}
\pron{gArxYxbi}
\gl{\gu}
\bmng
\bnum
\num{1} (\AmA) doVcuva savxBAvada, parxvaqtitxya. 
\num{2} durAseya. 
\enum
\emng
\eentry

\bentry
\word{graben}
\pron{gArxbanf}
\gl{\nA}
\expl{(\bava\ \eng{grabens} yA adeV).}
\bmng
(\BUvi) tUgusatxra; eraDu satxraBaMgagaLu, padara viciCxtitxgaLa naDuve iruva tagugx. 
\emng
\eentry

\bentry
\wordnospeech{grab handle}{grab handle}
\pron{?}
\gl{\nA}
\bmng
 (moVTArukArinalilx parxyANikaru hatitx iLiyalu anukUlavAguvaMte ilalxve kAru calaneyalilxruvAga) savxsAthxnadalilx BadarxvAgiruvaMte hiDidukoLaLxlu aLavaDisiruva hiDi. 
\emng
\eentry

\bentry
\wordnospeech{grab rail}{grab rail}
\pron{?}
\gl{\nA}
\bmng
 hiDikaMbi; moVTAruvAhanadalilx niMta parxyANikaru hiDidukoLaLxlu anukUlavAguvaMte aLavaDisiruva kaMbi. 
\emng
\eentry

\bentry
\word[grace(1)]{grace}
\pron{gerxVsf}
\gl{\nA}
\bmng
\bnum
\num{1} (\kanmu\ aMgAMgagaLa swSaThxva, naDigeya ThiVvi, kAyaRvidhAna, BAvaBaMgi, sheYli -- ivugaLa nayanAjUkugaLige saMbaMdhisidaMte) celuvu; ceMda; cenunx; sogasu; lAvaNayx; lAlitayx; ramayxte; sobagu; beDagu; gADi; vilAsa; manoVharate; moVhakate. 
\num{2} opapx; aucitayx; mayARde; Ganate: \eng{cannot with any grace ask him} nAnu yAvudeV mayARde iTuTxkoMDu avananunx keVLalAre. \eng{have the grace to do this} idanunx mADuva Ganateyanunx toVru. 
\num{3} (yAvudeV kelasa maDuva) riVti; dhATi; ThiVvi. 
\num{4} AkaSaRka lakaSxNa; sogasu; BUSaNa; alaMkAra. 
\hypertarget{grace(1)5}{} 
\num{5} (\saM) beDagu savxra; raMjaka savxra; alaMkAra savxra; savxrameVLakAkxgali mAdhuyaRkAkxgali agatayxvalalxda, Adare raMjane hecicxsuva savxra. 
\num{6} (doDaDxvaru toVruva) upakAra; anugarxha; kaqpe; daye; parxsAda: \eng{be in one's good graces} yAvudeV vayxkitxya anugarxhakekx, kaqpege pAtarxnAgiru; vayxkitxya pirxVti gaLisiru. 
\num{7} (riyAyiti toVralu AdhAravAda) visheVSa -- aBimAna, vishAvxsa, anugarxha:\eng{act of grace} hakAkxgi keVLalAgadiruva savalatutx yA riyAyiti. 
\num{8} (\birx) (padavi sivxVkarisalu sadasayxra saBeya hAgU vishavxvidAyxnilaya kAleVjina) opipxge; anumati; anujecnx. 
\num{9} (vishavxvidAyxnilayada liKita shAsanagaLiMda) vinAyiti. 
\num{10} (\deVva) BagavaMtana nivAyxRja kaqpe, avAyxja karuNe; deYvAnugarxha. 
\num{11} (deVvara) udAdhxraka kaqpe; mukatxyXnugarxha; vimoVcaka karuNe (\rUpa\ saha). 
\num{12} (\deVva) (sUPxtiR, shakitxgaLanunx koTuTx AtomxVnanxtiyanunxMTumADuva) deYva parxBAva. 
\num{13} (\deVva) deYva anugarxhakekx, parxBAvakekx, kaqpege -- oLagAda sithxti. 
\num{14} deYvadatatx -- parxtiBe, sAmathayxR, kwshala, \mo vu. 
\num{15} vAyide mAPi; sAvakAsha riyAyiti; vAyide miVrida sAvakAshakekx anumati niVDuva riyAyati; taDamADalu opipxge koDuva upakAra: \eng{give a day's grace} (pAvatiya dina miVri) oMdu dina hecicxge avadhi koDu. 
\num{16} (UTakekx modalu yA UTada baLika heVLuva) saMkiSxpatx vaMdanApaRNe; BoVjana pArxthaRne; BoVjana vaMdane. 
\num{17} (\girxVpu) (\eng{Grace}) swMdayaRdAte; swMdayaR matutx sobagugaLanunx dayapAlisuva mUvaru soVdariyaralilx obabxLu. 
\num{18} (keVMbirxDfjx vishavxvidAyxnilaya) seneTf yA riVjeMTf hwsfna -- niNaRya. 
\num{19} kaSxme; karuNe; kaqpe; daye. 
\enum
\emng

\noindent
\gl{\pagu}
\bmng
\hyperdef{G}{grace(1)7}{} 
\bnum
\num{1} \eng{Act of grace} (pAliRmeMTf shAsanada mUlaka koDuva) vidhivatAtxda (\sA\ sAvaRtirxka) kaSxmApaNe; kaSxmA kAnUnu. 
\num{2} \eng{airs and graces} (gamana seLeyuvudakAkxgi yA AkaSaRNegAgi toVruva) onapu oyAyxra; aMdaceMda; beDagu binAnxNa(da naDavaLike). 
\num{3} \eng{by the grace of God} deVvara kaqpeyiMda; BagavadanugarxhadiMda (\kanmu\ doregaLa birudAvaLigaLalilx seVrisuva mAtugaLu). 
\num{4} \eng{days of grace} (huMDi yA jiVvavimA kaMtina \vi) (kAnUnu parxkAraveV dorakuva) hecicxna gaDavu, kAlAvakAsha; adhikAravadhi. 
\num{5} \eng{Her Grace} mahAmAneyx; DacesaLa birudanunx heVLuvAga baLasuva okakxNe. 
\num{6} \eng{His Grace} mahAmAnayx; DUyxkana yA AcfR biSapapxna birudanunx heVLuvAga baLasuva okakxNe. 
\num{7} \eng{in this year of grace} (\sA\ vayxMgayx parxyoVga) kerxYsatxdhamaR nelesi iSuTx vaSaRgaLAda meVlU 
\hyperdef{G}{grace(1)pagu(8)}{} 
\num{8} \eng{state of grace} kaqpApAtarxte; deYvakaqpege pAtarxnAgiruvike. 
\hyperdef{G}{grace(1) pagu(8)}{} 
\num{9} \eng{the Graces} (\girxVpu) suMdara deVvatA soVdariyaru; moVhana deVviyaru; swMdayaRvanUnx moVhakateyanUnx anugarxhisuva lAvaNayxvatiyarAda mUvaru deVvasoVdariyaru (agelxVya, yUphorxseYni, theVliya). 
\num{10} \eng{the graceth year of grace} \kirxsha... neya isaviyalilx. 
\num{11} \eng{with a bad grace} manasisxlalxde; iSaTxvilalxde; olalxda manasisxniMda; samAdhAnavilalxde; avinayadiMda: \eng{he apologized, but did so with a bad grace} avanu kaSxmApaNe keVLida, Adare olalxda manasisxniMda keVLida. 
\num{12} \eng{with a good grace} manaHpUvaRkaveMbaMte; manasisxruvaMte; iSaTxveMdu toVruvaMte. 
\num{13} \eng{Your Grace} (DUyxkananonxV DacesaLanonxV AcfR biSapananonxV saMboVdhisuvAga baLasuva okakxNe) mahAmAnayxreV. 
\enum
\emng

\noindent
\gl{\nuga}
\bmng
\bnum
\numi{1} \eng{fall from grace} 
\banum
\alnum{a} (\deVva) pApakekx matetx biVLu, iLi; avakaqpege guriyAgu; anugarxha kaLeduko. 
\alnum{b} (adhikAra sAthxnadalilxruva vayxkitxya) vishAvxsa, pirxVti -- kaLeduko: \eng{despite his good connections he fell from grace again} avanige oLeLxya saMbaMdhagaLidAdxgUyx avanu punaH (meVladhikAriya) vishAvxsa kaLedukoMDanu. 
\eanum
\numie
\num{2} \eng{in someone's good graces} (yAvudeV vayxkitxya) kaqpege pAtarxnAgiru: \eng{I have stayed in her good graces this long} iSuTx kAlavU avaLa kaqpege pAtarxnAgi uLididedxVne. 
\num{3} \eng{have the graces to} (yAvudaradeV bagegx) sAkaSuTx kataRvayx parxjecnx, mayARde -- hoMdiru yA toVru. 
\enum
\emng
\eentry

\bentry
\word[grace(2)]{grace}
\pron{gerxVsf}
\gl{\sakirx}
\bmng
\bnum
\num{1} shoVBisuvaMte mADu; alaMkarisu; BUSaNavAguvaMte mADu; aMdagoLisu: \eng{Einstein graced the chair of physics in Princeton} ainfseTxYnf avaru pirxnfsxTanfna BwtavijAcnxnada piVThavanunx alaMkarisidaru. 
\num{2} gwrava koDu; parxshasitx niVDu; (birudu \mo vugaLiMda) sanAmxnisu: \eng{he was graced with the title of Princeps} avanu pirxnesxpfsx birudiniMda sanAmxnisalapxTaTxnu. 
\num{3} kiVtiRtaru; gwravataru: \eng{discoveries which would have graced a century} oMdu shatamAnakekx kiVtiR taruvaMtha AviSAkxragaLu. 
\enum
\emng
\eentry

\bentry
\word{grace-cup}
\pron{gerxVsfkapf}
\gl{\nA}
\bmng
\bnum
\num{1} (BoVjana pArxthaRneya naMtara obabxriMdobabxrige sutatxlU kaLuhisuva) madayxda baTaTxlu. 
\num{2} vidAyapAna; biVLokxDuge kuDita; biVLokxLuLxvAga mADuva pAna. 
\enum
\emng
\eentry

\bentry
\word{graceful}
\pron{gerxVsfphulf}
\gl{\gu}
\bmng
 (\kanmu\ rUpa, calane, yA kirxyeyalilx) suMdaravAda; lalitavAda; AkaSaRkavAda; hitavAda. 
\emng
\eentry

\bentry
\word{gracefully}
\pron{gerxVsfphuli}
\gl{\kirxvi}
\bmng
 (\kanmu\ rUpa, calane yA kirxyeyalilx) suMdaravAgi; lAlitayxdiMda; AkaSaRkavAgi; hitavAgi. 
\emng
\eentry

\bentry
\word{gracefulness}
\pron{gerxVsfphulfnisf}
\gl{\nA}
\bmng
 (\kanmu\ rUpa, calane, yA kirxyeyalilx) lAlitayx; swMdarayx; AkaSaRkate; hitavAgiruvike. 
\emng
\eentry

\bentry
\word{graceless}
\pron{gerxVsflisf}
\gl{\gu}
\bmng
\bnum
\num{1} (\pArxparx\ yA \hA) niVtigeTaTx; niVtiBarxSaTx; viSayalaMpaTa; pApadiMda kUDida: \eng{a graceless rogue} niVtigeTaTx paTiMga. 
\num{2} nAcikegeTaTx; mAnageTaTx; mayARdeyilalxda; ashiSaTx; asaBayx: \eng{graceless behaviour} nAcikegeTaTx vataRne. 
\num{3} sobagilalxda; suMdaravAgirada; aMdavilalxda; caMdageTaTx; beDagilalxda; lAvaNayxrahita; nayanAjUkilalxda. 
\enum
\emng

\noindent
\gl{\pagu}
\bmng
 \eng{graceless florin} aMdageTaTx beLiLxnANayx; (\eng{1849}ralilx TaMkisida, \eng{D.G.} eMba akaSxragaLu biTuTxhoVda) iMgilxSf beLiLxnANayx. 
\emng
\eentry

\bentry
\word{gracelessly}
\pron{gerxVsflisfli}
\gl{\kirxvi}
\bmng
\bnum
\num{1} (\pArxparx\ yA \hA) niVtigeTaTx riVtiyalilx; niVtiBarxSaTxnAgi; viSayalaMpaTatanadiMda; pApapUritavAgi. 
\num{2} nAcikegeTaTxMte; mAnageTaTx riVtiyalilx; mayARdeyilalxde; aucitayxrahitavAgi; ashiSaTxteyiMda; asaBayxvAgi. 
\num{3} suMdaravAgilalxde; aMdavilalxde; beDagilalxde; lAvaNayxrahitavAgi; ceMdageTaTxMte; nayanAjUkilalxde. 
\enum
\emng
\eentry

\bentry
\word{gracelessness}
\pron{gerxVsflisfnisf}
\gl{\nA}
\bmng
\bnum
\num{1} (\pArxparx\ yA \hA) niVtigeTaTxtana; niVtiBarxSaTxte; viSayalaMpaTate; pApadiMda kUDiruvike. 
\num{2} nAcikegeVDitana; mAnageTaTx sithxti; mayARde ilalxdiruvike; aucitayxrAhitayx; ashiSaTxte; asaBayxte. 
\num{3} aMdageVDu; caMdageVDu; lAvaNayxrAhitayx; nayanAjUkilalxdiruvike. 
\enum
\emng
\eentry

\bentry
\word{grace-note}
\pron{gerxVsfnoVTf}
\gl{\nA}
\bmng
  = \hyperlink{grace(1)5}{$^1$grace (5)}. 
\emng
\eentry

\bentry
\word{gracile}
\pron{gArxYxsi(seY)lf}
\gl{\gu}
\bmng
\bnum
\num{1} teLuvAda; teLaLxneya; siVra; asiyoDalina. 
\num{2} lalita; koVmala; celuvAgi teLuvAda; aMdavAgi teLaLxgiruva: \eng{those gracile arms} A koVmala bAhugaLu. 
\enum
\emng
\eentry

\bentry
\word{gracility}
\pron{gArxYxsiliTi}
\gl{\nA}
\bmng
\bnum
\num{1} teLuvu; asi(tana); kaqshate. 
\num{2} (sAhitayxsheYli) nirADaMbara saraLate; alaMkAravilalxda saraLate; niralaMkaqta saraLate. 
\enum
\emng
\eentry

\bentry
\word[gracious(1)]{gracious}
\pron{gerxVSasf}
\gl{\gu}
\bmng
\bnum
\num{1} (\pArxparx) saMtoVSadAyaka; hitavAda; suKAvahavAda; haSaRdAyaka: \eng{a gracious gift} saMtoVSadAyaka koDuge. 
\num{2} (\kanmu\ kAvayxdalilx) dayeyuLaLx; upakAra savxBAvada; mayARdeya; vinayashiVla; viniVta; swjanayxshiVla: \eng{he was gracious to all ladies} Ata elalx mahiLeyara viSayadalilxyU swjanayxshiVlanAgidadxnu. 
\num{3} (tananx keLaginavarige) anugarxha toVruva; kaqpedoVri upakAra mADuva; audAyaRda sahane toVruva; (keLaginavara) hitabayasuva. 
\num{4} (Ganatevetatx vayxkitxgaLa \vi\ yA vayxMgayxvAgi yA hAsayxvAgi, \kanmu\ rAjavaMshada vayxkitxgaLa yA avara kAyaRgaLa \vi\ gwravasUcakavAgi baLasuva visheVSaNa): \eng{the gracious speech from the throne} doreya anugarxha BASaNa. 
\num{5} (deVvara \vi) kaqpe toVruva; dayAmaya; kaqpApUNaR; karuNApUrita. 
\enum
\emng
\eentry

\bentry
\word[gracious(2)]{gracious}
\pron{gerxVSasf}
\gl{\BAavayx}
\bmng
 (\eng{gracious God} eMbudaralilx \eng{God} adhAyxhAra mADida \parx) koVpa yA AshacxyaRsUcaka udAgxra: \eng{good gracious! my gracious! gracious me! gracious goodness!} 
\emng
\eentry

\bentry
\word{graciously}
\pron{gerxVSasfli}
\gl{\kirxvi}
\bmng
\bnum
\num{1} suMdaravAgi; celuvAgi; manoVharavAgi; ramayxvAgi; sobaginiMda; lAvaNayxdiMda; moVhakavAgi. 
\num{2} (naDavaLikeya \vi) ucita riVtiyalilx; gAMBiVyaRdiMda; BUSaNavAgi; mayARdeyiMda; swjanayxdiMda; sajajxnikeyiMda. 
\num{3} dayeyiMda; senxVhapUvaRkavAgi; anugarxha toVruvaMte; saMtoVSa niVDuvaMte; kaqpe toVri; upakAramADuva riVtiyalilx; audAyaRda sahaneyiMda. 
\enum
\emng
\eentry

\bentry
\word{graciousness}
\pron{gerxVSasfnisf}
\gl{\nA}
\bmng
\bnum
\num{1} moVhakate; AkaSaRNiVyate; ramayxte; celuvu; lAvaNayx. 
\num{2} (naDavaLikeya \vi) ucita vataRne; gAMBiVyaR; mayARde; BUSaNavAgi vatiRsuvike. 
\num{3} vinaya; viniVtate; saBayxte; swjanayx; sajajxnike. 
\num{4} (Iga \kanmu) anugarxha rUpada -- swjanayx, vinaya, saBayxte. 
\num{5} (deVvara) dayAparate; anugarxha; karuNAmayate; kaqpeyiMda kUDiruvudu. 
\enum
\emng
\eentry

\bentry
\word{grackle}
\pron{gArxYxkflf}
\gl{\nA}
\bmng
 doMbakAge; guDi kAge jAtiya nAnA bageya hakikxgaLu. 
\emng
\eentry

\bentry
\word{grad}
\pron{gArxYxDf}
\gl{\nA}
\bmng
 (\AmA)  = \hyperlink{graduate(1)}{graduate}. 
\emng
\eentry

\bentry
\word{gradate}
\pron{garxDeVTf}
\gl{\sakirx}
\bmng
\bnum
\num{1} karxmika CAyAMtara mADu; goVcaravAgadaMte karxmavAgi CAyAMtarisu; vaNARMtarisu; kaNiNxge kANadaMte karxmavAgi baNaNxda CAye badalAyisuvaMte mADu. 
\num{2} (aLate, parxmANa, \mo vugaLige anusAravAgi) sherxVNiVkarisu; meTaTxlu meTaTxlAgi aLavaDisu; dajeRdajeRyAgi EpaRDisu. 
\enum
\emng

\noindent
\gl{\akirx}
\bmng
 karxmika CAyAMtara hoMdu; kANisadaMte savxlapxsavxlapxvAgi vaNARMtara hoMdu; karxmavAgi baNaNxda CAye oMdariMda inonxMdakekx tirugu. 
\emng
\eentry

\bentry
\word{gradation}
\pron{garxDeVSanf}
\gl{\nA}
\bmng
\bnum
\num{1} (\sA\ \bava dalilx) parivataRneya yA munanxDeya -- haMta(gaLu). majalu(gaLu), meTaTxlu(gaLu). 
\num{2} (sAthxna, yoVgayxte, tiVkaSxNXte, vayxtAyxsa, \mo vugaLalilx) tAratamayx; vagaRsherxVNi; vagaRsaraNi: \eng{gradations between insect and man} kiVTa matutx mAnavana madheyx iruva vagaRsaraNi. 
\num{3} dajeR; vagaR; sherxVNi. 
\num{4} sherxVNiVkaraNa; dajeR aLavaDisuvike; sherxVNigaLalilx, dajeRgaLalilx aLavaDisuvudu. 
\num{5} (lalitakalegaLa \vi) karxmika vayxtAyxsa; karxmakarxmavAda CAyAMtarate, savxrAMtarate, rUpAMtarate, vaNARMtarate, \mo vu; edudx kANuvaMte karxmeVNa CAyAMtara, savxrAMtara, \mo vanunx hoMduvudu: \eng{what curvature is to lines, gradation is to shades and colours} reVKegaLige vakarxte heVgoV baNaNxgaLigU CAyegaLigU anukarxma CAyAMtarate hAge. 
\num{6} (\BAshA) savxravayxtayxya. 
\num{7} maTaTxsavAguvike; samataliVkaraNa; hariyuva niVru, have, \mo vugaLa kirxyeyiMda oMdu visAtxravAda parxdeVshadalilx BUmi hecucx kaDime maTaTxsavAguvike. 
\enum
\emng
\eentry

\bentry
\word{gradational}
\pron{garxDeVSanalf}
\gl{\gu}
\bmng
\bnum
\num{1} parivataRne yA munanxDeya -- haMtada, majalina, meTiTxlina. 
\num{2} tAratamayxkarxmada; vagaRsaraNiya; vagaRsherxVNiya. 
\num{3} vagiRVkaraNada; vagiRVkaraNada lakaSxNagaLuLaLx. 
\num{4} (lalitakalegaLalilx vaNaR, savxra, \mo vugaLa \vi) karxmika vayxtAyxsagaLanunxLaLx; CAyAMtaragaLanunxLaLx; CAyABeVdagaLiMda kUDida; savxrAMtara \mo vanunx hoMduva. 
\enum
\emng
\eentry

\bentry
\word{gradationally}
\pron{garxDeVSanali}
\gl{\kirxvi}
\bmng
\bnum
\num{1} sherxVNi sherxVNiyAgi; (parivataRne yA munanxDe) haMtahaMtavAgi; meTaTxlu meTaTxlAgi. 
\num{2} (lalitakalegaLalilx vaNaR, savxra, \mo vugaLa \vi) karxmika vayxtAyxsagaLiMda kUDi; karxmavAda CAyABeVda \mo vugaLiMda kUDi. 
\enum
\emng
\eentry

\bentry
\word[grade(1)]{grade}
\pron{gerxVDf}
\gl{\nA}
\bmng
\bnum
\num{1} (sAthxna, parxviVNate, guNa, bele, \mo vugaLalilx) dajeR; aMtasutx; majalu; maTaTx; sherxVNi; satxra. 
\num{2} (sAthxna, parxviVNate, \mo vugaLalilx samAnavAgiruva vayxkitxgaLa yA vasutxgaLa) vagaR: \eng{teachers of every grade} parxtiyoMdu vagaRda adhAyxpakaru. 
\num{3} (shAleyalilx) taragati; vagaR; iyatetx. 
\num{4} (danada taLiyiLisuvudaralilx nADuhasugaLa meVle utatxma taLiya gULigaLanunx hArisi paDeda) taLiBeVda; aDaDxtaLiya Pala. 
\num{5} (\pArxvi) maTaTx; vikasanada hAdiyalilx hecucx kaDime adeV haMtadalilx kavaloDedu baMda pArxNi samUha. 
\num{6} (\BAshA) savxravayxtayxya (sherxVNiyalilx saMbaMdhAthaRka) sAthxna; nidiRSaTx savxra yA padada mUlarUpavu savxra vayxtayxya sherxVNiyalilx paDediruva sAthxna. 
\num{7} OraDi; utAru; iLukalu; iLijAru. 
\num{8} OraDi mAna; parxvaNate; Erikeya yA iLitada parxmANa; OraDiya parxmANa; gatimAna. 
\num{9} guNAMka; vagARMka; shAleya adhayxyana, pariVkeSx, \mo vugaLalilx vidAyxthiRya sAdhaneya maTaTxvanunx sUcisuva aMka, naMbaru, akaSxra, pada, \mo vu \udA\ \eng{80\%}. \eng{``A'', ``Excellent'', ``Fair''.} 
\enum
\emng

\noindent
\gl{\pagu}
\bmng
\bnum
\num{1} \eng{at grade} (\ame) oMdeV maTaTxdalilx; samamaTaTxdalilx: \eng{a railroad crosses a highway at grade} oMdu reYlu dAriyu hedAdxriyanunx oMdeV maTaTxdalilx aDaDx hAyutatxde. 
\num{2} \eng{on the down grade} iLiyutatx; biVLutatx; iLimuKadalilx (\rUpa saha). 
\num{3} \eng{on the up grade} Erutatx; ErumuKadalilx (\rUpa saha): \eng{business is on the up grade} vAyxpAra aBivaqdidhxyAgutitxde. 
\num{4} \eng{over grade} (oMdu hedAdxri, reYlumAgaR, yA pAdacAri mAgaR inonxMdanunx aDADxhAyuvAga) meVlina maTaTxdalilx. 
\num{5} \eng{under grade} (eraDu mAgaRgaLu aDaDxhAyuvAga) keLagina maTaTxdalilx. 
\num{6} \eng{up to grade} apeVkiSxsida yA beVkAda guNamaTaTxda: \eng{this shipment is not up to grade} I haDagina saraku apeVkiSxsida guNamaTaTxdadxlalx. 
\enum
\emng

\noindent
\gl{\nuga}
\bmng
 \eng{make the grade} 
\banum
\alnum{a} teVgaRDeyAgu; yashasivxyAgu; jaya hoMdu; apeVkiSxta yA nidiRSaTx gurimuTuTx: \eng{his son could not make the grade in school} avana maga shAleyalilx teVgaRDeyAgalilalx. 
\alnum{b} oLeLxya maTaTx sAdhisu, muTuTx. 
\eanum
\emng
\eentry

\bentry
\word[grade(2)]{grade}
\pron{gerxVDf}
\gl{\sakirx}
\bmng
 vagiRVkarisu; viMgaDisu; dajeRgaLalilx, vagaRgaLalilx joVDisu; vagaRvagaRvAgi, guMpuguMpAgi -- viMgaDisu: \eng{a machine that grades two thousand eggs per hour} gaMTeyoMdakekx eraDu sAvira moTeTxgaLanunx dajeRgaLalilx viMgaDisuva yaMtarx. 
\bnum
\num{2} (maTaTx, dajeR badalAyisuvaMte) berasu; misharxNamADu; berake mADu: \eng{cider is again graded with other apple juices} seVbu madayxvanunx punaH itara seVbu rasagaLoDane beresalAgide. 
\num{3} baNaNx karxmeVNa badalAguvaMte baNaNx hAku, baNaNxkoDu: \eng{the sky is graded from the vapours of the horizon to the clear blue of the zenith} AkAshavu digaMtada dhUmavaNaRdiMda bAnenxtitxya nimaRla niVliyavaregU karxmeVNa badalAguva baNaNx paDedide. 
\num{4} (rasetx, kAluve, \mo vanunx) sulaBavAda OraDigaLige iLisu; Oretagigxsu; iLukalAgisu. 
\num{5} (dana taLiyiLisuvudaralilx) utatxma taLiya gULiyanunx -- hArisu, koDisu. 
\num{6} (\BAshA) (kamaRNi \parx) savxravayxtayxyadiMda mApaRDu. 
\enum
\emng

\noindent
\gl{\akirx}
\bmng
 karxmavAgi dajeR badalAgu; oMdu dajeRyiMda inonxMdakekx karxmeVNa -- hoVgu, Eru yA iLi. 
\emng

\noindent
\gl{\pagu}
\bmng
 \eng{grade up} (utatxma taLiya gULiyanunx hArisuva mUlaka danada) taLiyanunx hasanugoLisu; taLi meVlapxDisu. 
\emng
\eentry

\bentry
\wordnospeech{grade crossing}{grade crossing}
\pron{?}
\gl{\nA}
\bmng
 (\ame) = \hyperref{kandict_l.pdf}{L}{level crossing}{level crossing}. 
\emng
\eentry

\bentry
\word{gradely}
\pron{gerxVDfli}
\gl{\gu}
\bmng
 (\birx) (\pArxparx) 
\bnum
\num{1} GanavAda; atuyxtatxma; paramAyiSi; BajaRri; pakAkx; pUtiR taqpitx koDuva. 
\num{2} aMdavAda; celuvAda; suMdara; rUpavaMta; suPxradUrxpavuLaLx: \eng{this is a hard road for a gradely foot like that} aMtha celuvAda pAdakekx idu oraTu rasetx. 
\num{3} neYja; sAcA; acacx; nijavAda; satayxvAda; sariyAda; ucita; yukatx: \eng{my gradely name is Hari} nananx nijavAda hesaru hari. 
\enum
\emng
\eentry

\bentry
\word{grader}
\pron{gerxVDarf}
\gl{\nA}
\bmng
\bnum
\num{1} dajeRga; vagiRVkAri; dajeRyAgi, vagaRvAgi viMgaDisuva vayxkitx yA vasutx. 
\num{2} (shAleyalilx) nidiRSaTx taragatiya, iyatetxya vidAyxthiR: \eng{a fourth grader} nAlakxneya taragatiyava(Lu). 
\num{3} (\ame) maTaTxsayaMtarx; nelavanunx oMdeV maTaTxvAgiruvaMte yA kelasakekx beVkAdaSuTx OraDiyAgiruvaMte mADuva yaMtarx. 
\enum
\emng
\eentry

\bentry
\word{Gradgrind}
\pron{gArxYxDfgerxYMDf}
\gl{\nA}
\bmng
 BAvashUnayx vasutxniSaThx; Adarada BAvaneyilalxde keVvala vAsatxvAMshagaLa bagegx mAtarx Asakitxyiruva vayxkitx (cAlfsxR Dikanasxna `hADfR TeYmfsx' eMba kaqtiya oMdu pAtarxda hesaru). 
\emng
\eentry

\bentry
\word{gradient}
\pron{gerxVDiaMTf}
\gl{\nA}
\bmng
\bnum
\num{1} (rasetx, reYlu dAri, \mo vugaLalilx) Olu; vATa; parxvaNate; iLukalina parxmANa; OraDiya parxmANa; maTaTxmeYyoDane mADuva Olu. 
\num{2} (\BUvi) vATa; parxvaNate; tApa, otatxDa, viduyxdivxBava, viduyxtfkeSxVtarx, \mo\ yAvudeV aMshavu vayxtAyxsavAgutatx hoVguva dara. 
\enum
\emng
\eentry

\bentry
\word{gradin}
\pron{gerxVDinf}
\gl{\nA}
\bmng
\bnum
\num{1} meTaTxlu sAlu; soVpAnapaMkitx; piVThapaMkitx; pAvaTige sAlu; tagugxmeTaTxlugaLa oMdu sAlu; piVThasherxVNiya oMdu sAlu. 
\num{2} (\kerxY) pUjAveVdikeya hiMdiruva baDu. 
\enum
\emng
\eentry

\bentry
\word{gradine}
\pron{gerxVDiVnf}
\gl{\nA}
\bmng
  = \hyperlink{gradin}{gradin}. 
\emng
\eentry

\bentry
\word[gradual(1)]{gradual}
\pron{gArxYxDuyx(ju)alf}
\gl{\nA}
\bmng
 (\kerxY) 
\bnum
\num{1} (parxBuBoVjana ArAdhaneyalilx apAsalara patarxBAgagaLu matutx suvAtAR garxMthagaLa paThanagaLa madheyx hADuva) saMvAda giVta; parxtuyxtatxra giVta. 
\num{2} parxBuBoVjana saMsAkxrada saMgiVta pusatxka. 
\enum
\emng
\eentry

\bentry
\word[gradual(2)]{gradual}
\pron{gArxDuyx(ju)alf}
\gl{\gu}
\bmng
 karxmika; karxmakarxmavAgi baLasuva, Aguva, naDeyuva; meTaTxlu meTaTxlAgi, haMtahaMtavAgi Aguva; nidhAnavAgi, melalxmelalxge, karxmeVNa -- muMduvariyuva; tavxreyalalxda; thaTakakxne alalxda; melalxmelalxneya; anukarxmavAda: \eng{gradual improvement in health} AroVgayxdalilx karxmeVNa sudhAraNe. 
\emng

\noindent
\gl{\pagu}
\bmng
 \eng{gradual psalm} soVpAna sotxVtarx; meTiTxlu giVta; jerUsalemimxge hoVguva mAgaRdalilx yA mwMTf seZyAnana yA deVvAlayada meTaTxlugaLanunx hatutxvAga hiVbUrx yAtirxkaru hADuva (sotxVtarx giVta garxMthadalilx \eng{120}riMda \eng{134}ra varegina) hadineYdu sotxVtarxgiVtagaLalolxMdu (= \hyperref{kandict_s.pdf}{S}{Song of Degrees}{Song of Degrees}). 
\emng
\eentry

\bentry
\word{gradualism}
\pron{gArxYxDuyx(ju)alisaZmf}
\gl{\nA}
\bmng
 karxmika parivataRne; yAvudeV gurisAdhaneyanunx karxmakarxmavAgi mADabeVkeV horatu diDhiVrane alalx eMba niVti. 
\emng
\eentry

\bentry
\word{gradualist}
\pron{gArxYxDuyx(ju)alisfTx}
\gl{\nA}
\bmng
 karxmika parivataRnavAdi; yAvudeV guriyanunx haMtahaMtavAgi sAdhisabeVkeV horatu diDhiVrane alalx eMdu vAdisuvava. 
\emng
\eentry

\bentry
\word{gradually}
\pron{gArxYxDuyx(ju)ali}
\gl{\kirxvi}
\bmng
 savxlapxsavxlapxvAgi; karxmeVNa; karxmakarxmavAgi; meTaTxlu meTaTxlAgi; melalxmelalxne; anukarxmavAgi. 
\emng
\eentry

\bentry
\word{gradualness}
\pron{gArxYxDuyx(ju)alfnisf}
\gl{\nA}
\bmng
 karxmikate; karxmakarxmavAgi Aguvike; meTaTxlu meTaTxlAgi Aguvike; karxmeVNa muMduvariyuvike; anukarxmate. 
\emng
\eentry

\bentry
\word{graduand}
\pron{gArxYxDuyxAYxnfDx}
\gl{\nA}
\bmng
 (\birx) BAviV padaviVdhara; padaviVdharanAgaliruvavanu; vishavxvidAyxnilayada padaviyanunx paDeyaliruvavanu. 
\emng
\eentry

\bentry
\word[graduate(1)]{graduate}
\pron{gArxYxDuyx(ju)aTf}
\gl{\nA}
\bmng
\bnum
\num{1} (vishavxvidAyxnilayada) padaviVdhara. 
\num{2} (aLategeregaLanunx hAkiruva, auSadhavAyxpAriya) aLate pAterx; mApu. 
\num{3} (\ame) shAlA vAyxsaMgavanunx mugisidava(Lu). 
\enum
\emng
\eentry

\bentry
\word[graduate(2)]{graduate}
\pron{gArxYxDuyx(ju)ETf}
\gl{\sakirx}
\bmng
\bnum
\numi{1} (\ame) 
\banum
\alnum{a} (vishavxvidAyxnilaya) padavikoDu. 
\alnum{b} shAleyalilx adhayxyana mugisidudara gurutAgi saTiRphikeVTu, shAlA parxmANapatarx, parxmANapatarx koDu. 
\alnum{c} (\ame) (vishavxvidAyxnilaya \mo vugaLiMda) padaviVdharananAnxgisu; padaviVdharananAnxgi mADi kaLuhisu. 
\eanum
\numie
\num{2} aLate gurutisu; (aLate) aMkagaLAgi viBAgisu; BAgagaLanunx gurutumADu: \eng{the thermometer graduated according to the scale of Fahrenheit} phAyxranfhiVTfmAnakekx anusAravAgi aLate gurutu hAkida tApamApaka. 
\num{3} vagiRVkarisu; dajeRdajeRyAgi EpaRDisu; tAratamAyxnusAra viMgaDisu: \eng{they begin to graduate the ages past} avaru gatayugagaLanunx dajeRgaLAgi viMgaDisalu AraMBisutAtxre. 
\num{4} (terigeya) horeyanunx sherxVNige anusAravAgi -- haMcu, pAlu mADu: \eng{the proposal to graduate the income tax} AdAya terigeyanunx (oMdu sherxVNige anusAravAgi) haMcuva salahe. 
\num{5} (dArxvaNavanunx) (iMgisuva mUlaka) sAriVkarisu. 
\enum
\emng

\noindent
\gl{\akirx}
\bmng
\bnum
\num{1} (vishavxvidAyxnilayada) padavi paDe; padavi tegeduko; padavi sivxVkarisu. 
\num{2} (\ame) shAleyalilx Odu mugisidudara parxmANapatarx, saTiRphikeVTu tegeduko. 
\num{3} (rUpaka) ahaRte, yoVgayxte, dakaSxte saMpAdisalu shikaSxNa yA tarapeVti hoMdu: \eng{graduate as a saint} saMtanAgalu shikaSxNa paDe, tarabeVti hoMdu. 
\num{4} (visheVSavAgi \BUvi, \savi\ matutx \pArxvi gaLalilx) karxmakarxmavAgi mApaRDu; karxmeVNa badalAvaNeyAgu; savxlapxsavxlapxvAgi badalAyisu: \eng{sandstone graduates into the inferior conglomerates} maraLugalulx karxmeVNa kiVLudajeRya cUrugalulxMDegaLAgi badalAvaNeyAgutatxde. 
\num{5} (kelasakAyaR \mo vugaLalilx) utakxSaR paDe; meVlina satxrakekx Eru; unanxti sAdhisu; utatxma maTaTx talupu. 
\enum
\emng
\eentry

\bentry
\word{graduated}
\pron{gArxYxDuyx(ju)ETiDf}
\gl{\gu}
\bmng
\bnum
\num{1} (vishavxvidAyxnilayada) padaviyuLaLx; padavi paDediruva; padaviVdhara(rAda). 
\num{2} (mApakada \vi) aLate gurutu hAkida; karxmAMkita. 
\enum
\emng
\eentry

\bentry
\wordnospeech{graduate nurse}{graduate nurse}
\pron{?}
\gl{\nA}
\bmng
 (\ame) tarabeVti paDeda dAdi. 
\emng
\eentry

\bentry
\wordnospeech{graduate school}{graduate school}
\pron{?}
\gl{\nA}
\bmng
 (\ame) unanxtAdhayxyana viBAga; padaviVdhararu unanxta adhayxyanavanunx naDesalu sAthxpisiruva vishavxvidAyxnilayada viBAga, ilAKe. 
\emng
\eentry

\bentry
\word{graduation}
\pron{gArxYxDuyx(ju)ESanf}
\gl{\nA}
\bmng
\bnum
\num{1} aLate gurutisuvike; karxmAMkana; yAvudeV mApakavanunx aLate gurutugaLiMda viBajisuvudu: \eng{faulty graduation of the thermometer} tApamApakada doVSapUrita karxmAMkana. 
\num{2} aLate gurutugaLu; karxmAMkagaLu. 
\num{3} padavi sivxVkAra; shikaSxNada oMdu haMtada vidhivatAtxda mukAtxya; (\kanmu\ shAle, kAleVju, vishavxvidAyxnilayada) Dipolxma, yoVgayxtApatarx, yA padavi -- sivxVkAra. 
\num{4} (vishavxvidAyxnilayada) padavidAna samAraMBa; DipolxmA, yoVgayxtApatarx, yA padavi -- parxdAna. 
\num{5} vagiRVkaraNa; vagaRvayxvasethx; aMsha, dajeR, aMtasutx, majalu, vagaRgaLAgi mADida vayxvasethx: \eng{the abolition of the graduation rank} vagaRvayxvasethxya radidxyAti. 
\num{6} sAdhane; beLavaNige yA Ganateyalilx unanxtamaTaTxkekx Erike; utakxSaR: \eng{his graduation from the most brilliant childhood} atayxMta parxtiBApUNaR sheYshavadiMda Erike. 
\num{7} sAriVkaraNa; dArxvaNavanunx iMgisuva mUlaka sAra hecicxsuvike. 
\enum
\emng
\eentry

\bentry
\word{graduator}
\pron{gArxYxDuyx(ju)ETarf}
\gl{\nA}
\bmng
\bnum
\numi{1} karxmAMkaka: 
\banum
\alnum{a} karxmAMkanakAra; gAjina padAthaRgaLu, upakaraNa, \mo vakekx karxmAMkana mADuvavanu. 
\alnum{b} reVKA viBAjaka; neVra yA vakarxreVKeyanunx karxmavAgi saNaNx BAgagaLAgi viBAgisuva sAdhana. 
\eanum
\numie
\num{2} sAriVkAraka; beVga beVga iMgisuva mUlaka dArxvaNada sAra hecicxsuva sAdhana. 
\enum
\emng
\eentry

\bentry
\word{gradus}
\pron{gerxVDasf}
\gl{\nA}
\bmng
 (\ca) (lAYxTinf kavanagaLanunx bareyalu sahAyavAguvaMte pATha shAlegaLalilx upayoVgisuva) lAyxTinf CaMdashAyxsatxrXda niGaMTu. 
\emng
\eentry

\bentry
\word{Graecise}
\pron{girxVseYsf}
\gl{\kirx}
\bmng
  = \hyperlink{Graecize}{Graecize}. 
\emng
\eentry

\bentry
\word{Graecism}
\pron{girxVsisaZmf}
\gl{\nA}
\bmng
\bnum
\num{1} (\kanmu\ beVre BASeyalilx anukaraNa mADida) girxVkf nuDigaTuTx; girxVkf BASAmayARde. 
\num{2} girxVkftana; girxVkf veYshiSaTxyX; girxVkf manoVdhamaR, sheYli, aBivayxkitx, vidhAna, \mo vu. 
\num{3} ivugaLa anukaraNa. 
\enum
\emng
\eentry

\bentry
\word{Graecize}
\pron{girxVseYsfZ}
\gl{\sakirx}
\bmng
 girxVkaraMtAgisu; girxVkf acucx, savxBAva, lakaSxNa, rUpa -- koDu; girxVkf veYshiSaTxyX koDu. 
\emng

\noindent
\gl{\akirx}
\bmng
\bnum
\num{1} girxVkara pakaSxvahisu; girxVkara kaDe olavu toVru. 
\num{2} girxVkara anukaraNa mADu; girxVkaraMtAgu. 
\enum
\emng
\eentry

\bentry
\word{Graeco-}
\pron{girxVkoV-}
\gl{\sapUpa}
\bmng
 girxVkf- eMbathaRdalilx baLasuva \sapUpa: \eng{Graeco-Roman} girxVkf matutx roVmanf. 
\emng
\eentry

\bentry
\word{Graecomania}
\pron{girxVkoVmeVnia}
\gl{\nA}
\bmng
 girxVkf giVLu; girxVkf hucucx; girxVsina viSayagaLa bagegx atAyxsakitx, atayxBimAna. 
\emng
\eentry

\bentry
\word{Graecomaniac}
\pron{girxVkoVmeVniAYxkf}
\gl{\nA}
\bmng
 girxVkf hucucx hiDidavanu; girxVkf giVLinava; girxVsina viSayagaLa bagegx atAyxsakatxnAdava. 
\emng
\eentry

\bentry
\word[Graecophil(1)]{Graecophil}
\pron{girxVkoVphilf}
\gl{\nA}
\bmng
 girxVkf perxVmi; girxVsina vasutx, viSayagaLalilx pirxVtiyuLaLxva. 
\emng
\eentry

\bentry
\word[Graecophil(2)]{Graecophil}
\pron{girxVkoVphilf}
\gl{\gu}
\bmng
 girxVsf perxVmada; girxVkf perxVmada. 
\emng
\eentry

\bentry
\word{graffito}
\pron{gArxphiVToV}
\gl{\nA}
\expl{(\bava\ \eng{graffiti} \ucAcx\ garxphiVTiV). }
\bmng
(\sA\ \bava dalilx) 
\bnum
\num{1} giVru citarx; giVru baraha; giVcu citarx; goVDe \mo vugaLa meVle giVcida citarx yA baraha. 
\num{2} giVralaMkAra; bagebageya baNaNxgaLa keLameY kANuvaMte gilAvina meVlapxdaradalilx mADida giVrugaLa alaMkAra. 
\enum
\emng
\eentry

\bentry
\word[graft(1)]{graft}
\pron{gArxphfTx}
\gl{\nA}
\bmng
\bnum
\num{1} kasikoMbe; kasiToMge; kasikAMDada siVLinalilx seVrisida matotxMdaralilxyU jiVvarasa hariyuvaMte kaTiTxda koMbe yA kuDi. 
\num{2} (\shaveY) kasike; kasi mADi seVrisida Utaka BAga. \imglink{graftsfigure}{\raisebox{-0.25cm}[0pt][0pt]{\pdfimage width 0.5cm height 0.6cm {G_Pictures/grafts.jpg}}} 
\num{3} kasike; kasi mADuvike; kasi kaTuTxvike. 
\num{4} kasi jAga; kasi sathxLa; kasi kaTiTxda sathxLa; kasi saMyoVjisida sathxLa. 
\num{5} (\ashi) kaSaTxda kelasa; sharxmada duDime. 
\enum
\emng
\eentry

\bentry
\word[graft(2)]{graft}
\pron{gArxphfTx}
\gl{\sakirx}
\bmng
\bnum
\num{1} (kuDi, kone) kasimADu; kasikaTuTx; gUTikaTuTx; kalamumADu: \eng{to graft an old apple with scions of a better variety} oMdu haLe jAtiya seVbanunx utatxma terana seVbugaLoDane kasimADalu. 
\num{2} (\rUpa) (jiVvaMtikeyiMda kUDida yA viceCxVda mADalAgada Ekateyanunx rUpisuvaMte) bese; saMyoVjisu; kUDisu; oLaseVrisu; neDu; bigi: \eng{a hopeful ending was grafted on to the story} AshAdAyaka mukAtxyavanunx katege beseyalAyitu. 
\num{3} (kAMDadoLage) kasi -- neDu, iDu. 
\num{4} (\shaveY) neDu; nATi mADu, hAku; (sajiVva aMgAMshavanunx) kitutx nelegoLisu, tegedu beVreya BAgakekx yA beVreya pArxNige saMyoVjisu. 
\num{5} (\nw) (baLe, aguLi, \mo vanunx) saNaNx huriheNigeyiMda sututx. 
\enum
\emng

\noindent
\gl{\akirx}
\bmng
\bnum
\num{1} kasi(gaLanunx) oLakekx -- hAku, seVrisu, joVDisu. 
\num{2} kaSaTxpaTuTx duDi; sharxmisu. 
\enum
\emng
\eentry

\bentry
\word[graft(3)]{graft}
\pron{gArxphfTx}
\gl{\nA}
\bmng
 (\AmA) 
\bnum
\num{1} (rAjakiVyada yA vAyxpArada saMbaMdhadalilx paDeyuva) akarxma lABagaLu; anAyxyada saMpAdane; laMca; ruSuvatutx: \eng{no matter how much graft his subordinates may have garnered} avana keYkeLaginavaru eSuTx hecAcxgi lUTi hoDedidadxrU paravAyilalx. 
\num{2} (ivugaLanunx paDeyalu baLasuva) durAcAra; BaSATxcAra; \kanmu\ laMca: \eng{tried to clear the graft from the government} sakARradiMda BarxSATxcAravanunx tegeduhAkalu yatinxsidanu. 
\enum
\emng
\eentry

\bentry
\word[graft(4)]{graft}
\pron{gArxphfTx}
\gl{\akirx}
\bmng
\bnum
\num{1} akarxma lABa saMpAdanege parxyatinxsu; anAyxyada saMpAdanege AsepaDu, keYhAku. 
\num{2} akarxma lABa saMpAdane mADu; laMca tegeduko. 
\enum
\emng
\eentry

\bentry
\word{grafter}
\pron{gArxphaTxrf}
\gl{\nA}
\bmng
kasiga: 
\banum
\alnum{a} kasigAra; kasikaTuTxva vayxkitx. 
\alnum{b} kasikaTuTxva sAdhana, upakaraNa. 
\eanum
\emng
\eentry

\bentry
\word{grafting-clay}
\pron{gArxphiTxMgfkelxV}
\gl{\nA}
\bmng
 kasimaNuNx; kasikaTiTxda BAgavanunx mucacxlu baLasuva maNuNx mishirxta padAthaR. 
\emng
\eentry

\bentry
\word{grafting-wax}
\pron{gArxphiTxMgfvAYxkfsx}
\gl{\nA}
\bmng
 kasi meVNa; kasikaTiTxda BAgavanunx mucacxlu baLasuva meVNamishirxta padAthaR. 
\emng
\eentry

\bentry
\wordRemoveSpace{graham-bread}{graham bread}
\pron{gerxVamf berxDf}
\gl{\nA}
\bmng
 (\ame) oMdari hiDiyada goVdi hiTiTxniMda tayArisida berxDuDx. 
\emng
\eentry

\bentry
\wordnospeech{graham flour}{graham flour}
\pron{?}
\gl{\nA}
\bmng
 (\ame) oMdari hiDiyada goVdi hiTuTx. 
\emng
\eentry

\bentry
\word[grail(1)]{grail}
\pron{gerxVlf}
\gl{\nA}
\bmng
 (\pArxparx)  = \hyperlink{gradual(1)}{$^1$gradual}. 
\emng
\eentry

\bentry
\word[grail(2)]{Grail}
\pron{gerxVlf}
\gl{\nA}
\bmng
% 
\bnum
\num{1} (\pArxparx) kirxsatxnu aMtima BoVjanadalilx upayoVgisida matutx shilubegeVrisalapxTaTx kirxsatxna rakatxvanunx arimAthiyada joVsephfnu hiDidiTaTx taTeTx yA baTaTxlu. 
\num{2} (madhayxyugada viVrara) diVGARnevxVSaNege guriyAgidadx kirxsatxna aMtima pAnapAterx (\rUpa saha). 
\enum
\emng

\noindent
\gl{\pagu}
\bmng
 \eng{Holy Grail} = \hyperlink{grail(2)}{$^2$Grail}. 
\emng
\eentry

\bentry
\word[grail(3)]{grail}
\pron{gerxVlf}
\gl{\nA}
\bmng
 bAcaNige mADuvavana ara, anaR. 
\emng
\eentry

\bentry
\word[grain(1)]{grain}
\pron{gerxVnf}
\gl{\nA}
\bmng
\bnum
\num{1} kALu; dhAnayx; davasa; biVja. 
\num{2} (sAmUhika \Eva) goVdi yA adaraMtha AhAra niVDuva hululxsasayx yA adu biDuva Pala, davasa, dhAnayx yA kALu. 
\num{3} (sAmUhika \Eva) nidiRSaTx (bageya) dhAnayx. 
\num{4} (\bava dalilx) gasikALu; baTiTxyuLikeya kALu; gasimAluTx; sArAyi mADida meVle yA baTiTxyiLisida meVle uLiyuva moLeta jave goVdiya kasaru. 
\num{5} (maraLu, cinanx, lavaNa, koVvi madudx, suvAsanA darxvayx, \mo vugaLa) kALu; haraLu; rave; kaNa. 
\num{6} haraLu; kaNa; baMDeyalilxna yA loVhadalilxna nidiRSaTx AkAravilalxda, viBinanxvAda kaNagaLalilx yA haraLugaLalilx oMdu. 
\num{7} (\vAyA) rAkeTf eMjininalilx baLasuva GananoVdakada tuMDu. 
\num{8} gerxVnu; oMdu atayxMta cikakx tUkamAna (TArxyf padadhxtiyalilx \eng{1/480} aunusx). (aveDuRpAyfsx padadhxtiyalilx \eng{1/437.5} aunusx). 
\num{9} aNu; lava; leVsha; atayxlapx parimANa: \eng{without a grain of vanity, of love} leVshamAtarxvU oNahememxyilalxde, lavaleVshavU perxVmavilalxde. 
\num{10} (\ca) `kamiRsf' baNaNx; karumaMji baNaNx; kaDugeMpu vaNaRdarxvayx; `kamiRsf' yA `kAciniVlf' huLadiMda tayArisida baNaNx. 
\num{11} (\kAparx) baNaNx; raMgu; vaNaR: \eng{a robe of the darkest grain} atayxMta daTaTx vaNaRda meVlaMgi. 
\num{12} kaNakaNagaLAgiruva racane. 
\num{13} meVlemxY taritariyAgiruvudu, tarakalAgiruvudu; meVlemxYya oraTu. 
\num{14} meVlemxY cukekx cukekxyAgiruvudu, (baNaNx baNaNxda) macecx macecxyAgiruvudu. 
\num{15} (mAMsa, camaR, mara, kalulx, \mo vugaLalilx) reVKAvinAyxsa; kaNagaLa racane, vinAyxsa matutx gAtarx. 
\num{16} (marada yA kAgadada \vi) eLeracane; gerevinAyxsa; oMdu nidiRSaTx AkAradalilxruva eLegaLa, geregaLa racane yA vinAyxsa. 
\num{17} (kalilxdadxlu, kalulx, \mo vugaLalilx) pore (ELuva) padaragaLu. 
\num{18} (\rUpa) savxBAva; parxkaqti; manoVvaqtitx; manoVdhamaR; parxvaqtitx; olavu; Olu: \eng{against the grain} savxBAvakekx virudadxvAgi; manoVdhamaRkekx vayxtirikatxvAgi: \eng{the mind must not be made to work against the grain} manasasxnunx savxBAvakekx virudadhxvAgi kelasa mADuvaMte mADabAradu. 
\enum
\emng

\noindent
\gl{\pagu}
\bmng
\bnum
\numi{1} \eng{dye in grain} 
\banum
\alnum{a} kirumaMji baNaNxkaTuTx. 
\alnum{b} yAvudeV gaTiTxbaNaNx -- hAku, koDu. 
\alnum{c} nUlinalelxV baNaNx -- kaTuTx, koDu. 
\alnum{d} cenAnxgi baNaNxkaTuTx; pUtAR baNaNx koDu. 
\eanum
\numie
\num{2} \eng{Guinea grains} masAleyAgiyU auSadha sAmagirxyAgiyU baLasuva, pashicxma Aphirxkada sasayxda biVjakoVshagaLu. 
\numi{3} \eng{in grain} 
\banum
\alnum{a} (\rUpa) pakAkx; shudadhx; apapxTa (tirasAkxrapUvaRkavAgi \kanmu\ \eng{ass, fool,} \mo vugaLoDane): \eng{he was known to be a scoundrel in grain} avanu pakAkx paTiMganeMdu gotitxtutx. 
\alnum{b} acacxLiyada; aLisalAgada; gaTiTxbaNaNxda. 
\eanum
\numie
\num{4} \eng{large grain powder} doDaDx care; doDaDx kaNada koVvimadudx. 
\num{5} \eng{small grain powder} saNaNx care; cikakx kaNada koVvimadudx. 
\enum
\emng

\noindent
\gl{\nuga}
\bmng
\bnum
\num{1} \eng{against the grain} (obabxna) savxBAvakekx virudadhxvAgi; parxkaqtige vayxtirikatxvAgi. 
\num{2} \eng{with a grain of salt} savxlapx soVDi biTuTx; vimashaRka budidhxyiMda; jAgarUkateyiMda: \eng{take his predictions with a grain of salt} avanu heVLuva BaviSayxnuDigaLa viSayadalilx ecacxrikeyiMdiru. 
\enum
\emng
\eentry

\bentry
\word[grain(2)]{grain}
\pron{gerxVnf}
\gl{\sakirx}
\bmng
\bnum
\num{1} kaNakaNavAgi mADu; haraLu haraLAgisu: \eng{the sugar of this tree is capable of being grained} I marada sakakxreyanunx haraLuharaLAgi mADabahudu. 
\num{2} gaTiTx baNaNx hAku; gaTiTx baNaNx kaTuTx: \eng{grain a cloth} baTeTxge gaTiTx baNaNx kaTuTx. 
\num{3} (meVlemxYyanunx) tarakalu mADu; taritarimADu: \eng{the stone is grained by being rubbed against another stone} kalalxnunx inonxMdu kalilxna meVle ujijx adara meVlemxYyanunx tarakalu mADalAgide. 
\num{4} (pArxNiya camaRdiMda) kUdalu tege: \eng{he grained a beaver skin} avanu biVvarf pArxNiya camaRdiMda kUdalu tegedanu. 
\num{5} (yAvudeV meVlemxY meVle) marada yA amaqta shileya eLegaLa vinAyxsavanunx (baNaNxdalilx) citirxsu: \eng{care should be taken in graining maple} meVpalf dAruvina meVle eLegaLa racanA mAdariyanunx citirxsuvAga ecacxrike vahisabeVku. 
\enum
\emng

\noindent
\gl{\akirx}
\bmng
 kaNakaNavAgu; haraLuharaLAgu: \eng{to make the salt grain better} lavaNavanunx inUnx hecucx haraLu haraLAguvaMte mADalu. 
\emng
\eentry

\bentry
\word{grained}
\pron{gerxVnfDx}
\gl{\gu}
\bmng
\bnum
\num{1} kALuLaLx; dhAnayxvuLaLx; biVjavuLaLx. 
\num{2} kaNakaNavAgiruva (\sA\ \saMpa gaLalilx parxyoVga): \eng{fine-grained sand} saNaNx saNaNx kaNagaLuLaLx maraLu. 
\num{3} (Akaqti, racane yA meVlemxYgaLalilx) eLeyeLeyAda -- racaneyuLaLx, vinAyxsavuLaLx: \eng{wood and other grained materials} mara matutx itara eLeyeLeyAda racaneyuLaLx vasutxgaLu. 
\num{4} kaqtakavAgi tayArisida kaNakaNavAgiruva racane yA namUneyuLaLx. \eng{grained kid} kaNakaNavAgiruva namUneya marada toTiTx. 
\num{5} savxBAva, savxrUpa, guNalakaSxNavuLaLx (\sA\ saMyukatx padavAgi baLake): \eng{tough grained journalism} diTaTxtanada patirxkAvaqtitx. 
\enum
\emng
\eentry

\bentry
\wordnospeech{grain elevator}{grain elevator}
\pron{?}
\gl{\nA}
\bmng
 kALetutxga; dhAnayx etutxga; davasa dhAnayx \mo vanunx meVlakekxtutxva yaMtarx. 
\emng
\eentry

\bentry
\word[grainer(1)]{grainer}
\pron{gerxVnarf}
\gl{\nA}
\bmng
\bnum
\num{1} keVshanivAraka; camaRgaLa kUdalanunx tegeyuvudakAkxgi baLasuva sAdhana. 
\num{2} kaNaracaka; kaNavinAyxsadaMte kANuva hAge mADalu upayoVgisuva sAdhana, upakaraNa. 
\num{3} (upipxna tayArikeyalilx) (dArxvaNavanunx iMgisi) haraLu mADuva pAterx, kaDAyi. 
\num{4} (mudarxNoVdayxmadalilx) kalalxnunx ujijx kaNakaNavAgiruvaMte, taritariyAgiruvaMte mADuvava yA mADuva sAdhana. 
\num{5} manege baNaNx baLiyuvavana upakaraNa. 
\num{6} (yAvudeV meVlemxY meVle) marada yA shileya reVKAvinAyxsavanunx baNaNxdalilx anukarisuvava, citirxsuvava. 
\enum
\emng
\eentry

\bentry
\word[grainer(2)]{grainer}
\pron{gerxVnarf}
\gl{\nA}
\bmng
 mInu BajiRgAra; mInu BajiR upayoVgisuvava. 
\emng
\eentry

\bentry
\word{grain-leather}
\pron{gerxVnfledarf}
\gl{\nA}
\bmng
 (kUdalidadx camaRda kaDeyanunx) hadagoLisida togalu; nayamADida togalu. 
\emng
\eentry

\bentry
\word{grainless}
\pron{gerxVnflisf}
\gl{\gu}
\bmng
\bnum
\num{1} kALilalxda; dhAnayxvilalxda; biVjavilalxda. 
\num{2} (sAmUhika) goVdiyilalxda; davasavilalxda; kALukaDiDxyilalxda. 
\num{3} (maraLu, cinanx, lavaNa, koVvimadudx, suvAsanAdarxvayx, \mo vugaLa \vi) kALilalxda; haraLilalxda; raveyilalxda; kaNarahita. 
\num{4} (\kAparx) baNaNxvilalxda; vaNaRrahita. 
\num{5} meVlemxY oraTilalxda; meVlemxY doragAgilalxda; meVlemxY -- taritariyAgilalxda, tarakalAgilalxda; kaNakaNa racaneyilalxda. 
\num{6} (baNaNxbaNaNxda) cukekxyilalxda; macecx macecxyAgirada. 
\num{7} (mAMsa, camaR, mara, kalulx, \mo vugaLalilx) eLeyilalxda; kaNavinAyxsavilalxda. 
\num{8} (kalilxdadxlu, kalulx, \mo vugaLa \vi) poreyeVLuva padaragaLilalxda; padararahita. 
\enum
\emng
\eentry

\bentry
\word{grains}
\pron{gerxVnfs'}
\gl{\nA}
\bmng
 (\Eva) kavalu mInu BajiR yA ITigALa. 
\emng
\eentry

\bentry
\word{grainsick}
\pron{gerxVnfsikf}
\gl{\nA}
\bmng
 danada hoTeTx roVga; danagaLa modala hoTeTx ububxva roVga. 
\emng
\eentry

\bentry
\wordnospeech{grains of Paradise}{grains of Paradise}
\pron{?}
\gl{\nA}
\bmng
 savxgaRdhAnayx; ginikALu; masAleyAgiyU auSadha vasutxvAgiyU baLasuva, ApharxmomaM melegevxTa kulada, dakiSxNa Aphirxkada sasayxda biVjakoVsha. 
\emng
\eentry

\bentry
\word{grainy}
\pron{gerxVni}
\gl{\gu}
\bmng
\bnum
\num{1} kaNakaNavAgiruva; haraLu haraLAda; taritariyAda; kaNagUDida meVlemxY racaneya. 
\num{2} marada eLe vinAyxsavanunx hoVluva. 
\num{3} kALinaMtha: \eng{grainy particle} kALinaMtha haraLu. 
\num{4} kALu tuMbida; dhAnayx tuMbida; davasa tuMbida: \eng{grainy nest} kALu tuMbida gUDu. 
\enum
\emng
\eentry

\bentry
\word{graip}
\pron{gerxVpf}
\gl{\nA}
\bmng
 (sAkxTalxMDf) (gobabxravanunx etatxlu yA AlUgaDeDx \mo vanunx ageyalu upayoVgisuva, mUru yA nAlukx muLiLxna) kavalugudadxli; kavalugoVlu. 
\emng
\eentry

\bentry
\word{grallatorial}
\pron{gArxYxlaToVrialf}
\gl{\gu}
\bmng
 (\jiVvi) niDugAlu niVruhakikxgaLa; udadx kAlidudx, niVranunx taLiLxkoMDu naDeyuva pakiSxgaNada yA adakekx saMbaMdhisida. 
\emng
\eentry

\bentry
\word[gralloch(1)]{gralloch}
\pron{gArxYxlakf}
\gl{\nA}
\bmng
 satatx jiMkeya karuLu yA adara oLa aMgagaLu. 
\emng
\eentry

\bentry
\word[gralloch(2)]{gralloch}
\pron{gArxYxlakf}
\gl{\sakirx}
\bmng
 (jiMke \mo vugaLa) karuLu tege; karuLu siVLu; hoTeTx bagi. 
\emng
\eentry

\bentry
\word[gram(1)]{gram}
\pron{gArxYxmf}
\gl{\nA}
\bmng
\bnum
\num{1} kaDale. 
\num{2} beVLekALu; `dANa'; kuduregaLige meVvAgi baLasuva yAvudeV divxdaLa dhAnayx. 
\enum
\emng
\eentry

\bentry
\word[gram(2)]{gram}
\pron{gArxYxmf}
\gl{\nA}
\bmng
 gArxYxmf; meTirxkf (padadhxtiyalilx) tUkada EkamAna; pAyxrisisxnalilx parxmANavAgiTiTxruva kiloVgArxmfnalilx sAvirada oMdaneV BAga. 
\emng
\eentry

\bentry
\wordwithhyphen{hyp-gram}{-gram}
\pron{-gArxYxmf}
\gl{\uparx}
\bmng
 (nAmavAcakagaLanunx rUpisuvAga) -leVKana, -leVKa, -vaNaR eMba athaRdalilx baLasuva \uparx : 
\banum
\alnum{a} girxVkiniMdAda upasagiRVya samAsagaLu: \eng{anagram, diagram, epigram.} 
\alnum{b} nAmavAcaka samAsagaLu: \eng{chronogram, logogram.} 
\alnum{c} girxVkf saMKAyxvAcaka samAsagaLu: \eng{monogram, hexagram.} 
\alnum{d} girxVkf sAdaqshayx niyamavanunx ulalxMGisi Ada samAsagaLu: \eng{telegram, cablegram.} 
\eanum
\emng
\eentry

\bentry
\word{grama}
\pron{gArx(gArxYx, gerxV)ma}
\gl{\nA}
\bmng
 (amerikada saMyukatx saMsAthxnagaLa pashicxma matutx neYQutayx BAgagaLalilx beLeyuva) meVvina moVTuhulilxna bagegaLu. 
\emng
\eentry

\bentry
\wordnospeech{grama grass}{grama grass}
\pron{?}
\gl{\nA}
\bmng
  = \hyperlink{grama}{grama}. 
\emng
\eentry

\bentry
\word{gramarye}
\pron{gArxYxmari}
\gl{\nA}
\bmng
 (\pArxparx) kaNakxTuTx; moVDi; mATa; iMdarxjAla; yakiSxNi; perxVta videyx; aBicAra. 
\emng
\eentry

\bentry
\word{gram-atom}
\pron{gArxYxMAYxTamf}
\gl{\nA}
\bmng
 (\ravi, \BUvi) gArxYxmf paramANu; (yAvudeV dhAtuvina \vi) A dhAtuvina paramANu tUka eSoTxV aSuTx gArxYxmina A dhAtu. 
\emng
\eentry

\bentry
\word{gram-equivalent}
\pron{gArxYxMikivxvalaMTf}
\gl{\nA}
\bmng
 (\ravi) gArxYxmf samAnatUka; (yAvudeV dhAtu yA saMyukatxda \vi) A padAthaRda samAnatUka eSoTxV aSuTx gArxyXmf A padAthaR. 
\emng
\eentry

\bentry
\word{gramercy}
\pron{garxmasiR}
\gl{\BAavayx}
\bmng
 (\pArxparx) vaMdanegaLu; dhanayxvAdagaLu. 
\emng
\eentry

\bentry
\word{gram-force}
\pron{gArxYxmfphoVrfsx}
\gl{\nA}
\bmng
 (\Bwvi) gArxYxmf bala (balada EkamAna); oMdu gArxYxmf darxvayxrAshiya meVle parxyoVgisidAga gurutavx pariNAmada veVgoVtakxSaR (sekeMDige \eng{980.665} seMmiV.)vanunxMTumADabalalx bala. 
\emng
\eentry

\bentry
\word{graminaceous}
\pron{gArxYxmineVSasf}
\gl{\gu}
\bmng
\bnum
\num{1} hulilxna; taqNada. 
\num{2} hulilxnaMtha; taqNasadaqsha. 
\num{3} hululx hecAcxgi beLediruva; hulilxruva; hululx mucicxda; taqNavishiSaTx; taqNaBarita; taqNapUNaR. 
\enum
\emng
\eentry

\bentry
\word{gramineous}
\pron{gArxminiasf}
\gl{\nA}
\bmng
  = \hyperlink{graminaceous}{graminaceous}. 
\emng
\eentry

\bentry
\word{graminiferous}
\pron{gArxYxminipherasf}
\gl{\gu}
\bmng
 taqNi; hululx utapxtitx mADuva. 
\emng
\eentry

\bentry
\word{graminivorous}
\pron{gArxYxminivarasf}
\gl{\gu}
\bmng
 hululx tinunxva; hululx meVyuva; taqNAda; hululx dini. 
\emng
\eentry

\bentry
\word{gramma}
\pron{gArxma}
\gl{\nA}
\bmng
  = \hyperlink{grama}{grama}. 
\emng
\eentry

\bentry
\word{grammalogue}
\pron{gArxYxmalAgf}
\gl{\nA}
\bmng
\bnum
\num{1} (shiVGarxlipi) EkasAMkeVtika pada; oMdeV cihenxyiMda sUcitavAda pada. 
\num{2} akaSxrapada; cihenxya pada; padasUcaka cihenx; oMdu padakekx badalAgi baLasuva akaSxra, cihenx, lipi. 
\enum
\emng
\eentry

\bentry
\word{grammar}
\pron{gArxYxmarf}
\gl{\nA}
\bmng
\bnum
\num{1} vAyxkaraNa(shAsatxrX). 
\num{2} vAyxkaraNa garxMtha. 
\num{3} vAyxkaraNa sheYli, riVti; vAyxkaraNa rUpagaLanunx (vayxkitxyu) baLasuva riVti. 
\num{4} (vayxkitxya) vAyxkaraNa parxyoVga; vAyxkaraNashudadhxvAda yA shudadhxvalalxda -- mAtu yA baravaNige. 
\num{5} vAyxkaraNa -- shudadhxvAdadudx, samamxtavAdadudx; vAyxkaraNa niyamagaLa parxkAra sariyAdadudx. 
\num{6} BASA mayARde; BASA rUpagaLu matutx parxyoVgagaLu; nuDiya shabadxrUpagaLu, parxyoVgagaLu: \eng{Latin grammar} lAyxTinf BASA mayARde. 
\num{7} yAvudeV kaleya yA vijAcnxnada mUlAMshagaLu. mUlatatxvXgaLu, modala pAThagaLu. 
\enum
\emng

\noindent
\gl{\pagu}
\bmng
\bnum
\num{1} \eng{comparative grammar} tulanAtamxka vAyxkaraNa; eraDu yA hecicxna BASegaLa vAyxkaraNgaLigiruva saMbaMdha kurita adhayxyana. 
\hypertarget{grammar pagu2}{} 
\num{2} \eng{general grammar} sAmAnayx vAyxkaraNa shAsatxrX; tAtitxvXka vAyxkaraNashAsatxrX; sAvaRtirxka vAyxkaraNashAsatxrX; elalx BASegaLa vAyxkaraNa padadhxtigaLigU AdhAravAgiruvudeMdu naMbalAgiruva sAmAnayx tatatxvXgaLanunx kurita shAsatxrX. 
\num{3} \eng{historical grammar} cAritirxka vAyxkaraNa; aitihAsika vAyxkaraNa; oMdu BASeya shabadxrUpagaLa matutx yoVjaneya beLavaNigeya itihAsada adhayxyana. 
\num{4} \eng{philosophical grammar} = \hyperlink{grammar pagu2}{?pagu? \((2)\)}. 
\num{5} \eng{universal grammar} = \hyperlink{grammar pagu2}{?pagu? \((2)\)}. 
\enum
\emng
\eentry

\bentry
\word{grammarian}
\pron{garxmeVrianf}
\gl{\nA}
\bmng
\bnum
\num{1} veYyAkaraNi; vAyxkaraNajacnx. 
\num{2} BASAshAsatxrXjacnx; BASApaMDita. 
\enum
\emng
\eentry

\bentry
\word{grammarless}
\pron{gArxYxmarflisf}
\gl{\gu}
\bmng
 (BASe hAgU vayxkitx, mAtu, \mo vugaLa \vi\ saha) vAyxkaraNavilalxda; vAyxkaraNarahita. 
\emng
\eentry

\bentry
\wordnospeech{grammar school}{grammar school}
\pron{?}
\gl{\nA}
\bmng
 gArxYxmarf shAle: 
\banum
\alnum{a} (\birx) \eng{16}neya shatamAnadalilx lAyxTinf kalisalu (iMgelxMDinalilx) sAthxpisida pAThashAle (AnaMtara I bageya shAlegaLu BASegaLu, cariterx, vijAcnxna, \mo\ viSayagaLanunx boVdhisuva mAdhayxmika shAlegaLAdavu). 
\alnum{b} (\birx) BASe, cariterx, vijAcnxna, \mo vu paThayx viSayagaLAgiruva mAdhayxmika shAle. 
\alnum{c} (\ame) mAdhayxmika shAle; pArxthamika shAle matutx pwrxDha shAlegaLa naDuvaNa shAle. 
\eanum
\emng
\eentry

\bentry
\wordwithhyphen{hyp-grammatic}{-grammatic}
\pron{-garxmAYxTikf}
\gl{\uparx}
\bmng
 \eng{-gram} niMda aMtayxvAguva padagaLiMda guNavAcakagaLanunx racisalu baLasuva \uparx 
\emng
\eentry

\bentry
\word{grammatical}
\pron{garxmAYxTikalf}
\gl{\gu}
\bmng
\bnum
\num{1} vAyxkaraNada; vAyxkaraNakekx saMbaMdhapaTaTx: \eng{grammatical rules} vAyxkaraNada niyamagaLu. 
\num{2} vAyxkaraNabadadhxvAda; vAyxkaraNaniyamagaLanunx anusarisuva. 
\num{3} sUtArxnusAravAda; shAsatxrXbadadhxvAda; yAvudeV kaleya niSakxqqSaTx tatatxvXgaLige anuguNavAgiruva. 
\enum
\emng
\eentry

\bentry
\wordnospeech{grammatical gender}{grammatical gender}
\pron{?}
\gl{\nA}
\bmng
 vAyxkaraNa liMga; padavu sUcisuva vasutxvina nijavAda liMgavanunx Adharisade padada rUpa \mo vugaLanunx Adharisi nidhaRrisida padada liMga, \udA\ : saMsakxqqtadalilx `mitarxM' (senxVhita) eMbudu napuMsaka liMga, `dArAH' (heMDati) pulilxMga. 
\emng
\eentry


\bentry
\word{grammaticalize}
\pron{garxmAYxTikaleYsfZ}
\gl{\sakirx}
\bmng
=  \hyperlink{grammaticize}{grammaticize}. 
\emng
\eentry

\bentry
\word{grammatically}
\pron{garxmAYxTikali}
\gl{\kirxvi}
\bmng
\bnum
\num{1} vAyxkaraNabadadhxvAgi; vAyxkaraNasamamxtavAgi; vAyxkaraNaniyamagaLiganusAravAgi. 
\num{2} (yAvudeV kaleya) niyamagaLige anusAravAgi; shAsatxrXbadadhxvAgi. 
\enum
\emng
\eentry

\bentry
\wordnospeech{grammatical sense}{grammatical sense}
\pron{?}
\gl{\nA}
\bmng
 vAcAyxthaR; akaSxrAthaR; padashaH athaR; vAyxkaraNada niyamagaLanunx horatupaDisi uLida aMshagaLanunx parigaNanege tegedukoLaLxda athaR. 
\emng
\eentry

\bentry
\word{grammaticise}
\pron{garxmAYxTiseYsfZ}
\gl{\sakirx}
\bmng
  = \hyperlink{grammaticize}{grammaticize}. 
\emng
\eentry

\bentry
\word{grammaticize}
\pron{garxmAYxTiseYsfZ}
\gl{\sakirx}
\bmng
 vAyxkaraNabadadhxvAgisu; vAyxkaraNAnusAriyAgisu; vAyxkaraNasamamxtavAgi mADu; vAyxkaraNa niyamagaLanunx anusarisuvaMte mADu; vAyxkaraNa niyamagaLige iLisu. 
\emng
\eentry

\bentry
\word{gramme}
\pron{gArxYxmf}
\gl{\nA}
\bmng
  = \hyperlink{gram(2)}{$^2$gram}. 
\emng
\eentry

\bentry
\word{gram-molecule}
\pron{gArxYxmfmAlikUyxlf}
\gl{\nA}
\bmng
 gArxYxmf aNu; yAvudeV saMyukatx yA dhAtuvina aNu tUka eSoTxV aSuTx gArxYxmf tUguvaSuTx-A saMyukatx yA dhAtu. 
\emng
\eentry

\bentry
\word{gramophone}
\pron{gArxYxmaphoVnf}
\gl{\nA}
\bmng
 gArxmaphoVnu; dhavxni punarAvaqtitx mADuva yaMtarx. 
\emng
\eentry

\bentry
\word{gramophonic}
\pron{gArxYxmu(ma)phAnikf}
\gl{\gu}
\bmng
\bnum
\num{1} gArxYxmaphoVnina; gArxYxmaphoVnige saMbaMdhisida. 
\num{2} gArxmaphoVnina yA gArxyXmaphoVnf rekADiRna savxrUpada. 
\enum
\emng
\eentry

\bentry
\word{grampus}
\pron{gArxYxMpasf}
\gl{\nA}
\bmng
\bnum
\num{1} bojujx mInu; urubuva, niVru cimumxva, moMDu taleya haMdi mInina jAtiya mInu. \imglink{grampusfigure}{\raisebox{-0.15cm}[0pt][0pt]{\pdfimage width 0.8cm height 0.5cm {G_Pictures/grampus.jpg}}} 
\num{2} gaTiTxyAgi usirubiDuvava; joVrAgi sadudx mADutAtx usiru hoyuyxvava. 
\enum
\emng
\eentry

\bentry
\word{gran}
\pron{gArxYxnf}
\gl{\nA}
\bmng
 (\AmA\ yA makakxLa \parx) \eng{granny} yA \eng{grandmother} eMbudara harxsavxrUpa. 
\emng
\eentry

\bentry
\word{granadilla}
\pron{gArxYxnaDila}
\gl{\nA}
\bmng
 `pAyxSanf' haNuNx; (kirxsatxna yAtaneyanunx sUcisuvudeMdu BAvisuva) kelavu bageya hUgiDagaLa haNuNx. 
\emng
\eentry

\bentry
\word{granary}
\pron{gArxYxnari}
\gl{\nA}
\bmng
\bnum
\num{1} kaNaja; hageVvu; okikxda dhAnayxvaninxDuva ugArxNa (\rUpa\ saha). 
\num{2} hecAcxgi dhAnayx beLeyuva, \kanmu\ raphutx mADuva pArxMta, parxdeVsha. 
\enum
\emng
\eentry

\bentry
\word[grand(1)]{grand}
\pron{gArxYxnfDx}
\gl{\gu}
\bmng
\bnum
\num{1} (adhikArada birudugaLalilx) mahoVnanxta; mahA; parxdhAna' atuyxcacx; elalxvakikxMta meVlina: \eng{Grand Almoner, Falconer, etc.} mahA dAnAdhikAri, mahAsheyxVna shikaSxka (mahA DeVgegAra), \mo vu. 
\num{2} (\nAyxshA) doDaDx; mahA; sherxVSaThx; unanxta; GanavAda; mahatavxda; muKayx; parxmuKa; parxdhAna: \eng{grand jury} nAyxyadashiRgaLa mahAmaMDali. 
\num{3} atayxMta muKayxvAda; parxmuKa; mahatavxda; mahatatxra: \eng{that is the grand question} adeV atayxMta muKayx parxshenx, mahatavxda parxshenx. \eng{made a grand mistake} mahatatxravAda tapapxnunx mADida. 
\num{4} kaTaTxkaDeya; aMtima; oTiTxna; saNaNxpuTaTx aMshagaLanenxlalx seVrisida, oTiTxge kUDida: \eng{grand total} aKeYru jumAlx; oTuTx motatx; motatxgaLa motatx. \eng{grand finale} (aperAgaLu, kirxVDAkUTagaLu, \mo vugaLalilx) samApitx; upasaMhAra; ujavxla mukAtxya: \eng{the grand sum or result of his achievements} avana elalx sAdhanegaLa oTuTx Pala yA aMtima pariNAma. 
\num{5} (doDaDx kaTaTxDada BAgagaLanunx visheVSisuvalilx) muKayx; mahA; parxdhAna: \eng{the grand staircase, entrance, etc.} muKayx meTaTxlu (sAlina kaTaTxDa) BAga, mahAdAvxra, \mo vu. 
\num{6} (pherxMcf \pagu gaLalilx yA avugaLanunx anukarisida parxyoVgagaLalilx) doDaDx; mahA; Bavayx: \eng{grand army} doDaDx seYnayx; mahAseVne. 
\num{7} veYBava; saMBarxma \mo vugaLiMda Acarisida; BajaRriyAda; adUdxriyAgi naDesida: \eng{grand wedding festivities} veYBavayuta vivAha samAraMBagaLu. 
\num{8} (vayxkitxgaLa, avara sAmAnugaLa \vi) sogasAda; bahu ThiVviya; atayxlaMkArada; jAjavxlayxmAna; veYBavadiMda kUDida; BajaRri. 
\num{9} shirxVmaMta samAjakekx seVrida; unanxta samAja vagaRkekx seVrida; parxtiSiThxta vagaRkekx seVrida; sherxVSaThx vagaRda. 
\num{10} Bavayx; sogasAda; BAri; baqhatAtxda; ceVtoVhAri: \eng{the scene was grand} daqshayxvu BavayxvAgitutx. 
\num{11} (kalapxne, viSaya nirUpaNe, yA sheYligaLalilx) gaMBiVra; GanavAda; udAtatx; bhavayx: \eng{grand style} (udAtatx viSayagaLige takukxdAda) BavayxsheYli; mahAsheYli; gaMBiVra sheYli. 
\num{12} (vayxkitxgaLa \vi) udAtatx cAritarxyXda; dhiVroVdAtatx; sherxVSaThx; pUjayx; shAlxGayx: \eng{a grand old man} pUjayxvaqdadhx; pUjayx pitAmaha. 
\num{13} (\AmA) adUdxriya; BajaRri; bahaLa oLeLxya; taqpitxkaravAda: \eng{the ground was in grand condition} nelavu tuMba taqpitxkaravAda sithxtiyalilxtutx. \eng{a grand time} adUdxriya kAla. 
\num{14} (neMTatanada hesarugaLalilx) oMdu tale hiMdina yA muMdina: \eng{grand father} ajajx; tAta. \eng{grand son} momamxga; pwtarx. 
\enum
\emng

\noindent
\gl{\pagu}
\bmng
\bnum
\num{1} \eng{Grand Cross} (\birx) `gArxYxMDf kArxsf'; (neYTfhuDf padaviya) atuyxnanxta parxshasitx, padaka. 
\num{2} \eng{Grand Fleet} (\eng{1914--18}ra yudadhxdalilxna) parxdhAna birxTiSf nwkAseVne. 
\enum
\emng

\noindent
\gl{\nuga}
\bmng
 \eng{do the grand} ilalxda doDaDxsitxke toVrisiko; mere; parxtiSeThx toVrisu; oNajaMbadiMda biVgu. 
\emng
\eentry

\bentry
\word[grand(2)]{grand}
\pron{gArxYxnfDx}
\gl{\nA}
\bmng
\bnum
\num{1} doDaDx piyAno; doDaDx samatalAkArada piyAno vAdayx. 
\num{2} (\ashi) (\bava\ sAmAnayxvAgi adeV). oMdu sAvira pwMDugaLu, (\ame) DAlarugaLu, \mo vu. 
\enum
\emng
\eentry

\bentry
\word{grandad}
\pron{gArxYxnfDAYxDf}
\gl{\nA}
\bmng
 = \hyperref{kandict_g.pdf}{G}{grand-dad}{grand-dad}. 
\emng
\eentry

\bentry
\wordnospeech{grand air}{grand air}
\pron{?}
\gl{\nA}
\bmng
 GanagAMBiVyaR; Ganate (yA ThiVvi). 
\emng
\eentry

\bentry
\word{grandam}
\pron{gArxYxnADxYxmf}
\gl{\nA}
\bmng
\hypertarget{grandam(1)}{} 
\bnum
\numi{1} (\pArxparx) 
\banum
\alnum{a} ajijx. 
\alnum{b} pUvaRjaLu. 
\alnum{c} muduki; mudi heMgasu. 
\eanum
\numie
\num{2} ajijx; pArxNiya tAyiya tAyi. 
\enum
\emng
\eentry

\bentry
\word{grandame}
\pron{gArxYxneDxVmf}
\gl{\nA}
\bmng
  = \hyperlink{grandam(1)}{grandam (1)}. 
\emng
\eentry

\bentry
\word{grandaunt}
\pron{gArxYxnfDxAnfTx}
\gl{\nA}
\bmng
 taMdeya yA tAyiya cikakxmamx, doDaDxmamx, soVdara atetx. 
\emng
\eentry

\bentry
\word{grandchild}
\pron{gArxYxnf(nfDx)ceYlfDx}
\gl{\nA}
\bmng
 momamxgu. 
\emng
\eentry

\bentry
\wordnospeech{grand committee}{grand committee}
\pron{?}
\gl{\nA}
\bmng
 (\birx) (pAliRmeMTina) mahAsamiti; nAyxyAMga matutx vANijayxda masUdegaLanunx parishiVlisalu parxti adhiveVshanakUkx niyamisalapxDuva hwsf Aphf kAmanisxna eraDu sAthxyiV samitigaLalolxMdu. 
\emng
\eentry

\bentry
\word{grand-dad}
\pron{gArxYxnfDxDAYxDf}
\gl{\nA}
\bmng
 ajajx; tAta (\AmA\ yA makakxLa mAtinalilx; pirxVtiyalilx heVLuvAga baLasuva pada). 
\emng
\eentry

\bentry
\word{grand-daddy}
\pron{gArxYxnfDxDAYxDi}
\gl{\nA}
\bmng
  = \hyperlink{grand-dad}{grand-dad}. 
\emng
\eentry

\bentry
\word{granddaughter}
\pron{gArxYxnfDxDATarf}
\gl{\nA}
\bmng
 momamxgaLu; pwtirx. 
\emng
\eentry

\bentry
\word{grandducal}
\pron{gArxYxnfDxDUyxkalf}
\gl{\gu}
\bmng
\bnum
\num{1} `gArxYxMDf Daci'ya, doreya, rANiya. 
\num{2} `sAZrf' putarxna yA sAZrf putirxya. 
\enum
\emng
\eentry

\bentry
\wordnospeech{grand duchess}{grand duchess}
\pron{?}
\gl{\nA}
\bmng
\bnum
\num{1} `gArxYxMDf Daci' eMba hesarina yUroVpina kelavu saMsAthxnagaLa rANi. 
\num{2} (haLeya raSAyxda cakarxvatiR) `sAZrf'na magaLu. 
\enum
\emng
\eentry

\bentry
\wordnospeech{grand duke}{grand duke}
\pron{?}
\gl{\nA}
\bmng
\bnum
\num{1} `gArxYxMDf Daci' eMba hesarina yUroVpina kelavu saMsAthxnagaLa dore. 
\num{2} (haLeya raSAyxda cakarxvatiR) `sAZrf'na maga. 
\enum
\emng
\eentry

\bentry
\wordf{grande amoureuse}
\pron{gArxMDf amuaresfZR}
\gl{\nA}
\expl{\F}
\bmng
 mahAparxNayi; kAmuki; bahaLa parxNayAsakatx heMgasu. 
\emng
\eentry

\bentry
\wordf{grande dame}
\pron{gArxMDf DAmf}
\gl{\nA}
\expl{\F}
\bmng
mahA sitxrXV; unanxta vagaRda GanagaMBiVra heMgasu. 
\emng
\eentry

\bentry
\word{grandee}
\pron{gArxYxniDxV}
\gl{\nA}
\bmng
\bnum
\num{1} (sepxVnf yA poVcuRgalilxna) kuliVnoVtatxma; mahA shirxVmaMta. 
\num{2} utatxma dajeRya vayxkitx; unanxta padaviyavanu; Ganatevetatx vayxkitx; parxtiSAThxvaMta. 
\enum
\emng
\eentry

\bentry
\wordf{grande passion}
\pron{gArxMDf pAYxsAYxknuf}
\gl{\nA}
\expl{\F\ }
\bmng
BavayxparxNaya; mahAparxNaya; bahaLa BAvAviSaTx parxNaya (parxsaMga). 
\emng
\eentry

\bentry
\wordf{grande tenue}
\pron{gArxMDf TanUyx}
\gl{\nA}
\expl{\F\ }
\bmng
 pUNaR poVSAku; pUtiR dirisu; mahatatxvXda saMdaBaRgaLalilx dharisuva vidhuyxkatx uDupu. 
\emng
\eentry

\bentry
\word{grandeur}
\pron{gArxYxnfDayx(ja)rf}
\gl{\nA}
\bmng
\bnum
\num{1} (adhikAra, sAthxna, yA aMtasutxgaLa) aunanxtayx; hirime. 
\num{2} dhiVroVdAtatxte; unanxta cAritarxyX. 
\num{3} (rUpa yA pariNAmada) Bavayxte; mahime; parxBAva; Ganate; mahoVnanxti; udAtatxte; mahatatxvX. 
\num{4} GanagAMBiVyaR; aTaTxhAsa. 
\num{5} (jiVvana karxma, parisara, \mo vugaLa) veYBava; sogasu; shoVBe. 
\enum
\emng
\eentry

\bentry
\word{grandfather}
\pron{gArxYxnf(nfDx)phAdarf}
\gl{\nA}
\bmng
 ajajx; tAta; pitAmaha yA mAtAmaha. 
\emng

\noindent
\gl{\pagu}
\hyperdef{G}{grandfather clock}{}\bmng
 \eng{grandfather('s) clock} etatxragUDina gaDiyAra; etatxrada marada koVshadalilx BArada baTuTxgaLanunx tUgabiTuTx naDesuva doDaDx gaDiyAra. 
\emng
\eentry

\bentry
\wordRemoveSpace{Grand-Guignol}{Grand Guignol}
\pron{gArxnf giVnAyxlf}
\gl{\nA}
\bmng
 biVBatasxrUpaka; BiVkara daqshAyxvaLi; BiVkara yA GoVravAdudanunx oLagoMDa, aneVka veVLe tiVvarxvAgi BAvoVderxVkagoLisuva cikakx daqshayxgaLanunx anukarxmavAgi parxdashiRsuva nATaka. 
\emng
\eentry

\bentry
\word{grandiflora}
\pron{gArxYxMDipholxVra}
\gl{\gu}
\bmng
 baqhatupxSipx; vishAlapuSipx; doDaDx hUgaLanunx biDuva. 
\emng
\eentry

\bentry
\word{grandiloquence}
\pron{gArxYxMDilakavxnfsx}
\gl{\nA}
\bmng
\bnum
\num{1} vAgADaMbara; shabAdxDaMbara. 
\num{2} (baraha yA BASaNadalilx) ADaMbarada sheYli. 
\enum
\emng
\eentry

\bentry
\word{grandiloquent}
\pron{gArxYxMDilakavxMTf}
\gl{\gu}
\bmng
\bnum
\num{1} vAgADaMbarada; shabAdxDaMbarada: \eng{grandiloquent style} shabAdxDaMbarada sheYli. 
\num{2} baDAyi mAtina. 
\enum
\emng
\eentry

\bentry
\word{grandiloquently}
\pron{gArxYxMDilakavxMTfli}
\gl{\kirxvi}
\bmng
\bnum
\num{1} vAgADaMbaradiMda; shabAdxDaMbara sheYliyalilx. 
\num{2} baDAyi mAtiniMda. 
\enum
\emng
\eentry

\bentry
\word{grandiose}
\pron{gArxYxMDiOsf}
\gl{\gu}
\bmng
\bnum
\num{1} mahatavxpUNaRvAda; mahatavxdAdxgi -- iruva, toVruva, kANisuva. 
\num{2} mahatotxvxVdedxVshada; GanoVdedxVshada. 
\num{3} Bavayx parxmANadalilx -- EpaRDisida, yoVjisida: \eng{more grandiose than real life} nija jiVvanakikxMtalU Bavayx parxmANada. 
\num{4} ADaMbarada; ATAToVpada; DAMBika; baDAyiya: \eng{grandiose speeches} ATAToVpada BASaNagaLu. 
\enum
\emng
\eentry

\bentry
\word{grandiosely}
\pron{gArxYxMDiOsfli}
\gl{\kirxvi}
\bmng
\bnum
\num{1} mahatatxvXpUNaRvAgi; mahatavxdAdxgi iruvaMte, kANuvaMte. 
\num{2} GanoVdedxVshadiMda kUDi. 
\num{3} Bavayx parxmANadalilx yoVjisidaMte; BavayxvAgi EpaRDisiruva riVtiyalilx. 
\num{4} ADaMbara sheYliyalilx; ATAToVpadiMda; DAMBikavAgi. 
\enum
\emng
\eentry

\bentry
\word{grandiosity}
\pron{gArxYxMDiasiTi}
\gl{\nA}
\bmng
\bnum
\num{1} mahatavxpUNaRte; mahatavxdAdxgi iruvike yA kANisuvike. 
\num{2} GanoVdedxVsha hoMdiruvike. 
\num{3} BavayxvAgi yoVjisiruvudu. 
\num{4} ADaMbara; ATAToVpa; DAMBikate. 
\enum
\emng
\eentry

\bentry
\word{Grandisonian}
\pron{gArxYxMDisoVnianf}
\gl{\gu}
\bmng
 gArxYxMDisaninxnaMtha; swjanayx hAgU audAyaRdiMda kUDida (hadineMTaneya shatamAnada iMgilxSf kAdaMbarikAra ricaDfRsaninxna kAdaMbariya pAtarxda hesariniMda sUcitavAdadudx). 
\emng
\eentry

\bentry
\wordnospeech{grand jury}{grand jury}
\pron{?}
\gl{\nA}
\bmng
 (\ca\ yA \ame) (nAyxya vicAraNege ApAdanegaLanunx opipxsuva modalu, avugaLanunx vicArisuvudakAkxgi neVmisida \eng{12}riMda \eng{33} maMdiyanonxLagoMDa) nAyxyadashiRgaLa mahAmaMDali. 
\emng
\eentry

\bentry
\wordnospeech{grand lodge}{grand lodge}
\pron{?}
\gl{\nA}
\bmng
 (phirxVmeVsananxra matutx avaranunx anukarisuva saMsethxgaLa) ADaLita mahAmaMDali. 
\emng
\eentry

\bentry
\word{grandly}
\pron{gArxYxMDfli}
\gl{\kirxvi}
\bmng
\bnum
\num{1} veYBavadiMda; BajaRriyAgi; saMBarxmayutavAgi; adUdxriyAgi. 
\num{2} sogasAda riVtiyalilx; bahuThiVviyiMda; atayxlaMkAradiMda. 
\num{3} shirxVmaMtikeyiMda. 
\num{4} BavayxvAgi; BAriyAgi; ceVtoVhAriyAgi. 
\num{5} gaMBiVravAgi; udAtatx riVtiyalilx; GanavAda sheYliyalilx. 
\num{6} ADaMbaravAgi; ATAToVpadiMda. 
\enum
\emng
\eentry

\bentry
\word{grandma}
\pron{gArxYxnfDxmA}
\gl{\nA}
\bmng
  = \hyperlink{grandmama}{grandmama}. 
\emng
\eentry

\bentry
\wordf{grand mal}
\pron{gArxnf mAlf}
\gl{\nA}
\expl{\F\ }
\bmng
doDaDx beVne; mahAvAyxdhi; jAcnxna tapupxva, oMdu bageya tiVvarxvAda mUCeRroVga. 
\emng
\eentry

\bentry
\word{grandmama}
\pron{gArxYxnf(nfDx)mamA}
\gl{\nA}
\bmng
 ajijx. 
\emng
\eentry

\bentry
\wordnospeech{grand manner}{grand manner}
\pron{?}
\gl{\nA}
\bmng
 udAtatx sheYli; Bavayx sheYli; udAtatx viSayagaLige hoMduva sheYli. 
\emng
\eentry

\bentry
\wordnospeech{grand master}{grand master}
\pron{?}
\gl{\nA}
\bmng
 gArxYxMDf mAsaTxrf: 
\banum
\alnum{a} seYnayxda `neYTf' birudina dajeRya nAyaka. 
\alnum{b} phirxVmeVsananxra nAyaka; phirxVmeVsanf maMDalige seVrida pArxMtagaLoMdara nAyaka; phirxVmeVsanf pArxMtada muKayxsathx; phirxVmeVsanfsx, ADfpheloVsf, eMbiveV \mo\ swBArxtaq paMgaDagaLa nAyaka. 
\alnum{c} (caduraMgada ATadalilx) atuyxnanxta maTaTxda ATagAra; sherxVSaThx ATagAra. 
\eanum
\emng
\eentry

\bentry
\wordnospeech{Grand Monarch}{Grand Monarch}
\pron{?}
\gl{\nA}
\bmng
 phArxnisxna dore \eng{14}ne lUyi (\eng{1688--1715}). 
\emng
\eentry

\bentry
\word[grandmother(1)]{grandmother}
\pron{gArxYxnf(nfDx)madarf}
\gl{\nA}
\bmng
 ajijx; pitAmahi yA mAtAmahi. 
\emng

\noindent
\gl{\nuga}
\bmng
 \eng{teach your grandmother to suck eggs} ajijxge kemumx kalisu; miVnige Iju kalisu; tanagiMtalU hecucx anuBavavuLaLxvarige budidxvAda heVLu, tiLivaLike koDu. 
\emng
\eentry

\bentry
\word[grandmother(2)]{grandmother}
\pron{gArxYxnf(nfDx)madarf}
\gl{\sakirx}
\bmng
 ati upacAra mADu; tuMba akakxreyiMda AreYke mADu, hecAcxgi tininxsu; bahaLa mududxmADu. 
\emng
\eentry

\bentry
\word{grandmotherly}
\pron{gArxYxnf(nfDx)madarfli}
\gl{\gu}
\bmng
\bnum
\num{1} ajijxya yA ajijxya riVtiya. 
\num{2} saNaNx puTaTx vivaragaLa bagegx matutx amuKayxvAda kAnUnukaTaTxLegaLa bagegx atiyAda ecacxrike vahisuva. 
\enum
\emng
\eentry

\bentry
\wordnospeech{Grand National}{Grand National}
\pron{?}
\gl{\nA}
\bmng
 (\birx) (livarfpUlina) mahAkudure paMdayx; livarfpUlinalilx naDeyuva vASiRka kudure paMdayx, reVsu. 
\emng
\eentry

\bentry
\word{grandnephew}
\pron{gArxYxnf(nfDx)nevUyx(phuyx)}
\gl{\nA}
\bmng
 soVdara momamxga; soVdarana magana yA magaLa maga. 
\emng
\eentry

\bentry
\word{grandness}
\pron{gArxYxnfDxnisf}
\gl{\nA}
\bmng
\bnum
\num{1} veYBava(vAgiruvike); Bavayxte. 
\num{2} sogasAgiruvike; ujavxlate. 
\num{3} Ganate; udAtatxte. 
\num{4} BavayxkAyaR; mahatAkxyaR; mahoVnanxta kelasa: \eng{he did many grandnesses} avanu aneVka BavayxkAyaRgaLanunx mADidanu. 
\enum
\emng
\eentry

\bentry
\word{grand-niece}
\pron{gArxYxnf(nfDx)niVsf}
\gl{\nA}
\bmng
 soVdara momamxgaLu; soVdara magana yA soVdara soseya magaLu. 
\emng
\eentry

\bentry
\wordnospeech{grand old man}{grand old man}
\pron{?}
\gl{\nA}
\bmng
 pUjayxpitAmaha; vaqdadhxpitAmaha; rAjakiVya, kale, kirxVDe \mo\ yAvudeV keSxVtarxdalilx parxmuKanAgidudx \kanmu\ bahaLa kAla avugaLalilx kirxyAshiVlanAgidadx, bahaLa gwravAnivxtanAda vaqdadhxvayxkitx (\kanmu\ iMgelxMDinalilx parxdhAnigaLAgidadx gAlxDfsaTxnf yA caciRlf, kirxkeTiTxnalilx DabulxyX. ji. gerxVsfrige anavxyisutitxdudxdu). 
\emng
\eentry

\bentry
\wordnospeech{Grand Old Party}{Grand Old Party}
\pron{?}
\gl{\nA}
\bmng
 (\ame) ripabilxkanf pATiR. 
\emng
\eentry

\bentry
\wordnospeech{grand opera}{grand opera}
\pron{?}
\gl{\nA}
\bmng
 mahA geVya nATaka; mahA giVtanATaka; gArxYxMDf aperA; gadayxda saMBASaNegaLilalxde, udadxkUkx sAhitayxvanunx hADuva, saMgiVtada mUlaka athaR vivarisuva nATaka. 
\emng
\eentry

\bentry
\word{grandpa}
\pron{gArxYxnf(nfDx)pA}
\gl{\nA}
\bmng
  = \hyperlink{grandpapa}{grandpapa}. 
\emng
\eentry

\bentry
\word{grandpapa}
\pron{gArxYxnf(nfDx)papA}
\gl{\nA}
\bmng
tAta; ajajx. 
\emng
\eentry

\bentry
\word{grand-parent}
\pron{gArxYxnf(nfDx)peVranfTx}
\gl{\nA}
\bmng
 ajajx yA ajijx. 
\emng
\eentry

\bentry
\wordRemoveSpace{grand-passion}{grand passion}
\pron{gArxYxnfDx pAYxsAyxnf}
\gl{\nA}
\bmng
= \hyperlink{grande passion}{\it grande passion.} 
\emng
\eentry

\bentry
\wordnospeech{grand piano}{grand piano}
\pron{?}
\gl{\nA}
\bmng
doDaDx piyAno; samataladalilx taMtigaLanunx aLavaDisida doDaDx piyAnoV vAdayx. \imglink{grand pianofigure}{\raisebox{-0.15cm}[0pt][0pt]{\pdfimage width 0.7cm height 0.6cm {G_Pictures/grand piano.jpg}}} 
\emng
\eentry

\bentry
\wordRemoveSpace{Grand-Prix}{Grand Prix}
\pron{gArxnf pirxV}
\gl{\nA}
\bmng
 gArxnf pirxV; aMtararASiTxrXVya niyamagaLa parxkAra, beVre beVre deVshagaLalilx naDeyuva moVTArukAru yA seYkalugaLa sapxdheR, reVsu. 
\emng
\eentry

\bentry
\wordf{grand siecle}
\pron{gArxnf seYkflf}
\gl{\nA}
\expl{\F\ }
\bmng
 (phArxnisxna) mahAyuga; suvaNaRyuga; kAlxsikalf yuga; \eng{17}neV shatamAnadalilx \eng{14}ne lUyi ALida kAla, avadhi. 
\emng
\eentry

\bentry
\word{grandsire}
\pron{gArxYxnf(nfDx)seYarf}
\gl{\nA}
\bmng
\bnum
\num{1} (\kanmu) pArxNiya ajajx. 
\numi{2} (\pArxparx) 
\banum
\alnum{a} ajajx; tAta. 
\alnum{b} pUvaRja; pUviRka. 
\alnum{c} muduka; vaqdadhx. 
\eanum
\numie
\num{3} nAdaBeVda; vAdana vayxtAyxsa; gaDiyAravu \eng{1/4}, \eng{1/2} \mo\ GaMTegaLanunx sUcisuvAga oMdoMdu salavU beVre beVre nAdadiMda vAdana mADuva riVti. 
\enum
\emng
\eentry

\bentry
\wordnospeech{grand slam}{grand slam}
\pron{?}
\gl{\nA}
\bmng
\bnum
\num{1} (birxDfjx isipxVTATa) pUtiR paTuTx; elalx paTuTxgaLanUnx (hadimUru) gelulxvudu. 
\num{2} gArxYxMDfsAlxYxmf; savaR vijaya; Tenisf, gAlfphx, \mo\ ATagaLalilx vaSaRda oMdu nigadita avadhiyalilx elalx parxmuKa sapxdheRgaLalUlx gelulxvudu. 
\enum
\emng
\eentry

\bentry
\word{grandstand}
\pron{gArxYxnf(nfDx)sATxYxnfDx}
\gl{\nA}
\bmng
 (OTada paMdayx \mo vugaLalilx) parxdhAna perxVkaSxka veVdike; noVTakara muKayx aTaTxNe. \imglink{grandstandfigure}{\raisebox{-0.15cm}[0pt][0pt]{\pdfimage width 0.7cm height 0.5cm{G_Pictures/grandstand.jpg}}} 
\emng

\noindent
\gl{\pagu}
\bmng
\bnum
\num{1} \eng{grandstand finish} roVmAMcaka aMtayx, mukAtxya; yAvudeV ATadalilx bahaLa alapx aMtaradalilx gelulxvaMtha meYnavireVLisuva mukAtxya, aMtayx. 
\num{2} \eng{grandstand play} capApxLe ATa; saBikariMda capApxLe giTiTxsikoLuLxva daqSiTxyiMda ADuva ATa. 
\enum
\emng
\eentry

\bentry
\wordnospeech{grand style}{grand style}
\pron{?}
\gl{\nA}
\bmng
  = \hyperlink{grand manner}{grand manner}. 
\emng
\eentry

\bentry
\wordnospeech{grand tour}{grand tour}
\pron{?}
\gl{\nA}
\bmng
\bnum
\num{1} (\ca) mahAparxvAsa; shikaSxNapUraka parxvAsa; shikaSxNapUrakavAgi keYgoMDa yUroVpina muKayxnagara \mo vugaLa parxvAsa. 
\num{2} (\rUpa) vAyxpaka parxvAsa; \kanmu\ aneVka garxhagaLu \mo vugaLige mADuva hArATa. 
\enum
\emng
\eentry

\bentry
\word{granduncle}
\pron{gArxYxnfDxaMkflf}
\gl{\nA}
\bmng
 taMdeya, tAyiya cikakxpapx, doDaDxpapx yA soVdara mAva. 
\emng
\eentry

\bentry
\wordnospeech{grand vizier}{grand vizier}
\pron{?}
\gl{\nA}
\bmng
 parxdhAna vajiVra; (\kanmu\ hiMdina tukiR cakArxdhipatayxda) musilxM deVshada parxdhAni. 
\emng
\eentry

\bentry
\word{grange}
\pron{gerxVMjf}
\gl{\nA}
\bmng
\bnum
\num{1} (\pArxparx) kaNaja; hageVvu. 
\num{2} (holamanegaLu seVrida) haLiLx mane; okakxla mane. 
\enum
\emng
\eentry

\bentry
\word{grangerise}
\pron{gerxVMjareYsfZ}
\gl{\sakirx}
\bmng
  = \hyperlink{grangerize}{grangerize}. 
\emng
\eentry

\bentry
\word{grangerism}
\pron{gerxVMjarisaZmf}
\gl{\nA}
\bmng
 gerxVMjarike; gerxVMjarf padadhxti; (itara pusatxkagaLiMda katatxrisida citarxgaLiMda pusatxkakekx) hecucx citarxgaLanunx odagisuvudu. 
\emng
\eentry

\bentry
\word{grangerite}
\pron{gerxVMjareYTf}
\gl{\nA}
\bmng
  = \hyperlink{grangerizer}{grangerizer}. 
\emng
\eentry

\bentry
\word{grangerization}
\pron{gerxVMjareYseZVSanf}
\gl{\nA}
\bmng
 gerxVMjariVkaraNa; hecucx citarxgaLanonxdagisuvike; visheVSa sacitirxVkaraNa. 
\emng
\eentry

\bentry
\word{grangerize}
\pron{gerxVMjareYsfZ}
\gl{\sakirx}
\bmng
 gerxVMjariVkarisu; (aneVka veVLe, itara pusatxkagaLiMda katatxrisi tegeda citarxgaLu \mo vanunx seVrisi, pusatxkakekx) hecucx citarxgaLanonxdagisu; adhikavAgi sacitarxgoLisu. 
\emng
\eentry

\bentry
\word{grangerizer}
\pron{gerxVMjareYsaZrf}
\gl{\nA}
\bmng
 gerxVMjarika; (itara pusatxkagaLiMda katatxrisi tegeda citarxgaLu \mo vugaLiMda pusatxkakekx) hecucx citarxgaLanonxdagisuvava. 
\emng
\eentry

\bentry
\word{graniferous}
\pron{garxnipharasf}
\gl{\gu}
\bmng
 kALu biDuva; dhAnayxdAyi; kALinaMtha biVjavanunx biDuva. 
\emng
\eentry

\bentry
\word{graniform}
\pron{gArxYx(gerxV)niphAmfR}
\gl{\gu}
\bmng
 kALinAkArada; dhAnAyxkArada. 
\emng
\eentry

\bentry
\word{granite}
\pron{gArxYxniTf}
\gl{\nA}
\bmng
\bnum
\num{1} gArxyXneYTf (kalulx); kaTaTxDada kalulx, beNacu kalulx, aBarxka, \mo vugaLiMda kUDida, kaTaTxDagaLige upayoVgisuva, saPxTikAkaqtiya shile. 
\num{2} (\rUpa) kalulx; kaThoVrate; kaThinate; moMDutana; bagagxdiruvike: \eng{that granite -- hearted shipowner} A kalulx haqdayada haDagina mAliVka. 
\enum
\emng

\noindent
\gl{\pagu}
\bmng
 \eng{the granite city} (sAkxTelxMDina) abarfDiVnf nagara. 
\emng

\noindent
\gl{\nuga}
\bmng
 \eng{bite on granite} sharxmavelAlx vayxthaRvAgu; vaqthA sharxmapaDu; vayxthaRvAgi paTuTx hiDi; kagagxlulx kaDi. 
\emng
\eentry

\bentry
\word{granite-ware}
\pron{gArxYxniTfveVrf}
\gl{\nA}
\bmng
\bnum
\num{1} gArxYxneYTf pAterxgaLu; (gArxYxneYTf shileyananxnukarisi mADida) cukekx cukekxya kuMbAra sAmAnu. 
\num{1} oMdu bageya `enAyxmalf' kabibxNada pAterxgaLu. 
\enum
\emng
\eentry

\bentry
\word{granitic}
\pron{garxniTikf}
\gl{\gu}
\bmng
\bnum
\num{1} gArxYxneYTina; gArxYxneYTf savxrUpada; gArxYxneYTiniMdAda; gArxYxneYTuLaLx. 
\num{2} (niVrina \vi) gArxYxneYTf BUmiyiMda doreta. 
\enum
\emng
\eentry

\bentry
\word{granitiform}
\pron{garxniTiphAmfR}
\gl{\gu}
\bmng
 gArxYxneYTf sadaqsha; gArxYxneYTanunx hoVluva. 
\emng
\eentry

\bentry
\word[granitoid(1)]{granitoid}
\pron{gArxYxniTAyfDx}
\gl{\gu}
\bmng
 gArxYxneYTanunx hoVluva; gArxYxneYTina racaneyuLaLx; gArxYxneYTf racaneya. 
\emng
\eentry

\bentry
\word[granitoid(2)]{granitoid}
\pron{gArxYxniTAyfDx}
\gl{\nA}
\bmng
 gArxYxneYTanunx hoVluva shile; gArxYxneYTf racaneyuLaLx shile. 
\emng
\eentry

\bentry
\word{granivorous}
\pron{garxnivarasf}
\gl{\gu}
\bmng
 kALudini; kALutinunxva; dhAnAyxhAri. 
\emng
\eentry

\bentry
\word{grannie}
\pron{gArxYxni}
\gl{\nA}
\bmng
  = \hyperlink{granny}{granny}. 
\emng
\eentry

\bentry
\word{granny}
\pron{gArxYxni}
\gl{\nA}
\bmng
\bnum
\num{1} (saligeya, pirxVtiya, yA tAtAsxrada mAtAgi) ajajxmamx; mudukamamx; aDugUlajijx; haraTemalilx. 
\hypertarget{granny(2)}{} 
\num{2} toDakugaMTu; tapApxgi kuNike hAkida `riVphf' gaMTu. 
\enum
\emng
\eentry

\bentry
\wordnospeech{granny flat}{granny flat}
\pron{?}
\gl{\nA}
\bmng
 neMTariSaTxrigAgi savxyaMpUNaRvAgiruvaMte kaTiTxda maneya oMdu BAga. 
\emng
\eentry

\bentry
\wordnospeech{granny knot}{granny knot}
\pron{?}
\gl{\nA}
\bmng
  = \hyperlink{granny(2)}{granny (2)}. 
\emng
\eentry

\bentry
\wordnospeech{Granny Smith}{Granny Smith}
\pron{?}
\gl{\nA}
\bmng
 AseTxrXVliyada, oMdu bageya hasiru seVbu. 
\emng
\eentry

\bentry
\word[granolithic(1)]{granolithic}
\pron{gArxYxnalitikf}
\gl{\gu}
\bmng
 (gArxYxneYTf cUrugaLa matutx simeMTina berakeya) oMdu bageya gacicxna, kAMkirxVTina; gArxYxneYTf kAMkirxVTina. 
\emng
\eentry

\bentry
\word[granolithic(2)]{granolithic}
\pron{gArxYxnalitikf}
\gl{\nA}
\bmng
 gArxYxneYTf kAMkirxVTu; puDi mADida gArxYxneYTfniMda tayArisida oMdu bageya kAMkirxVTu. 
\emng
\eentry

\bentry
\word[grant(1)]{grant}
\pron{gArxnfTx}
\gl{\sakirx}
\bmng
\bnum
\num{1} (pArxthaRne \mo vanunx IDeVrisalu, salilxsalu, pUrayisalu) opupx; samamxtisu. 
\num{2} dayapAlisu; anugarxhisu; niVDu; (vasutx) hoMdalu (vayxkitxge) avakAshakoDu. 
\num{3} (sAvxmayxvanunx, hakakxnunx) vidhivihitavAgi -- niVDu, koDu, dayapAlisu, koDamADu, anugarxhisu. 
\num{4} (Asitxyanunx) kAnUnuriVtAyx -- vagARyisu, vahisikoDu. 
\num{5} vAdakekx AdhAravAgi (parxmeVyavanunx) -- aMgiVkarisu, iTuTxko, BAvisu, opipxko: \eng{I grant you} nAnu opipxkoLuLxtetxVne. 
\enum
\emng

\noindent
\gl{\pagu}
\bmng
\hyperdef{G}{granted(1) pagu}{} \eng{take for granted} 
\banum
\alnum{a} hAgeMdu iTuTxko, BAvisu; sidadhxvAdadedxMdu, parxmANa beVkilalxveMdu -- BAvisu; sAmAnayxvAgi opipxkoLaLxbahudAdadedxMdu tiLi; nijaveMdu, nidhARravAgideyeMdu -- parigaNisu; yathAkarxmadalilx AgatakakxdedxMdu tiLi. 
\alnum{b} ati paricayadiMda (oMdara) veYshiSaTxyXvanunx gamanisadiru, gwravisadiru, upeVkiSxsu. 
\eanum
\emng
\eentry

\bentry
\word[grant(2)]{grant}
\pron{gArxnfTx}
\gl{\nA}
\bmng
\bnum
\num{1} IDeVrisuvike; neraveVrisuvike; salilxsuvudu; pUreYke; mAnayx mADuvudu; samamxtisuvudu; opupxvudu. 
\num{2} niVDike; koDuvudu; niVDuvudu; dayapAlisuvudu; anugarxhisuvudu: \eng{the grant or refusal} mAnayx mADuvudu yA nirAkarisuvudu. 
\num{3} vidhivihita -- dAna, koDuge. 
\num{4} kAnUnuriVtAyx (Asitx, sAvxmayx, hakukx, \mo vugaLa) vagARvaNe. 
\num{5} datitx; koDuge; koDuvaLi; koTaTxdudx (\kanmu\ dAna, koDuge rUpavAgi koTaTx haNa): \eng{capitation grant} talAvAru koDuge; talegiSuTx eMdu koDuva haNa. 
\num{6} sananxdu vagaR patarx; dasetxYvajina mUlaka vagARyisuvudu. 
\enum
\emng
\eentry

\bentry
\word{grantable}
\pron{gArxnaTxbflf}
\gl{\gu}
\bmng
\bnum
\num{1} IDeVrisabalalx; neraveVrisi koDabalalx; salilxsabalalx; pUrayisabalalx; maninxsabalalx; opapxbalalx; samamxtisabalalx. 
\num{2} niVDabalalx; dayapAlisabalalx; anugarxhisabalalx. 
\num{3} (\nAyxshA) (sAvxmayx, hakukx) vidhivihitavAgi niVDabalalx, koDabalalx. 
\num{4} (\nAyxshA) (Asitx) kAnUnuriVtAyx vagARyisabalalx, vahisikoDabalalx. 
\num{5} (parxmeVya) vAdakekx AdhAravAgi -- aMgiVkarisabalalx, anumoVdisabalalx, iTuTxkoLaLxbalalx. 
\enum
\emng
\eentry

\bentry
\wordnospeech{grant-aided school}{grant-aided school}
\pron{?}
\gl{\nA}
\bmng
 (\birx) sahAyadhana paDeda shAle; sAvaRjanika nidhiyiMda savxlapx dhanasahAya paDeyutitxruva shAle. 
\emng
\eentry

\bentry
\word{grantee}
\pron{gArxniTxV}
\gl{\nA}
\bmng
 (\nAyxshA) (kAnUnu riVtAyx) parxtigArxhi; datitx, karxya, dAna -- paDeyuvava; Asitx, datitx, dAna -- gArxhi, sivxVkAri. 
\emng
\eentry

\bentry
\word{Granth}
\pron{garxMTf}
\gl{\nA}
\bmng
 garxMthasAhabf; sikakxra pavitarxgarxMtha, dhamaRgarxMtha. 
\emng
\eentry

\bentry
\word{grant-in-aid}
\pron{gArxMTfinfEDf}
\gl{\nA}
\bmng
 sahAya dhana; anudAna; sakARravu shAle \mo vugaLige koTaTx neravu haNa. 
\emng
\eentry

\bentry
\word{grantor}
\pron{gArxnATxrf}
\gl{\nA}
\bmng
 dAni; dAta; dAtaq; datitx; karxya, Asitx, dAna -- koDuvavanu. 
\emng
\eentry

\bentry
\wordf{gran turismo}
\pron{gArxyxnf TuarisfZmoV}
\gl{\nA}
\expl{\F\ (\bava\ \eng{gran turismos}).}
\bmng
parxvAsikAru; dUrada parxyANagaLige baLasuva, veVgavAgi hoVguva parxyANada kAru. 
\emng
\eentry

\bentry
\word{granular}
\pron{gArxYxnuyxlarf}
\gl{\gu}
\bmng
\bnum
\num{1} kaNakaNada; haraLu haraLAgiruva; kALukALAgiruva; kaNakaNadaMtiruva; haraLu haraLinaMtha; kALukALinaMtha. 
\num{2} kaNaracaneya; kaNa(gUDida) meVlemxYya; taritariyAda racaneyuLaLx. 
\enum
\emng
\eentry

\bentry
\word{granularity}
\pron{gArxYxnuyxlAYxriTi}
\gl{\nA}
\bmng
\bnum
\num{1} kaNakaNavAgiruvike; haraLuharaLAgiruvike. 
\num{2} kaNa(gUDida) meVlemxYiruvike; kaNakaNavAda racane; taritariyAgiruvike. 
\enum
\emng
\eentry

\bentry
\word{granularly}
\pron{gArxYxnuyxlarfli}
\gl{\kirxvi}
\bmng
\bnum
\num{1} kaNakaNavAgi; kaNakaNadaMtiruvaMte; haraLuharaLAgiruvaMte. 
\num{2} kaNa(gUDida) meVlemxY iruvaMte; taritariyAda racaneyuLaLxdAdxgi; kaNa(kaNavAda) racaneyiMda kUDi. 
\enum
\emng
\eentry

\bentry
\word{granulate}
\pron{gArxYxnuyxleVTf}
\gl{\sakirx}
\bmng
\bnum
\num{1} kaNakaNavAgi mADu; haraLuharaLAgi mADu. 
\num{2} meVlemxYyanunx -- tarakalu mADu, taritarimADu. 
\enum
\emng

\noindent
\gl{\akirx}
\bmng
\bnum
\num{1} kaNakaNavAgu; haraLuharaLAgu. 
\num{2} (gAya \mo vugaLa \vi) (mAyuvudara yA kUDikoLuLxvudara modalalilx) saNaNx saNaNx ububxgaLAgu; mAyi; Aru; kaNagUDu; seVriko; kUDiko. 
\enum
\emng
\eentry

\bentry
\word{granulation}
\pron{gArxYxnuyxleVSanf}
\gl{\nA}
\bmng
\bnum
\num{1} (sAmAnayx athaRdalilx) kaNavAguvike; haraLAguvike. 
\num{2} (meVlemxY) taritariyAguvike; macecxmacecxyAguvike. 
\num{3} (marada \vi) (yAvudeV mAdariya) eLegaLa racane. 
\num{4} kaNavAgalapxDuvike; haraLAgalapxDuvike. 
\num{5} (meVlemxY) taritariyAgalapxDuvike; macecxmacecxyAgalapxDuvike. 
\num{6} (marada \vi) eLegaLAgi racisalapxDuvudu. 
\num{7} (vasutxtaH) kaNa; haraLu. 
\num{8} (meVlemxYya) tarakalu; macecxmacecx. 
\num{9} (marada) eLe (vinAyxsa.) 
\numi{10} (\roVshA) 
\banum
\alnum{a} (gAya mAyuvAga) huNuNxgaLa meVle kaNadaMtha ububxgaLAgiruvudu. 
\alnum{b} (\bava dalilx) kaNadaMtha ububxgaLu. 
\eanum
\numie
\numi{11} (\jiVvi) 
\banum
\alnum{a} kALugaTuTxvike; giDagaLu, cipupxjiVvigaLu, muMtAduvugaLa horameYmeVle kALukALAgi kaNagaLu rUpugoLuLxvudu. 
\alnum{b} hAge uMTAda racane yA kaNagaLu. 
\eanum
\numie
\enum
\emng
\eentry

\bentry
\word{granulator}
\pron{gArxYxnuyxleVTarf}
\gl{\nA}
\bmng
\bnum
\num{1} kaNakAri; kaNakaNavAgi mADuvaMthavanu, mADuvaMthadudx; haraLu haraLAgi -- mADuvava, mADuvaMthadudx. 
\num{2} taritarimADuvava; tarakalu mADuvava. 
\num{3} (\kanmu) kaNayaMtarx; haraLuyaMtarx; kaNakaNavAgi mADuva yaMtarx. 
\num{4} tariyaMtarx; meVlemxYyanunx taritariyAgi mADuva yaMtarx. 
\enum
\emng
\eentry

\bentry
\word{granule}
\pron{gArxYxnUyxlf}
\gl{\nA}
\bmng
 saNaNx -- kALu, kaNa, haraLu. 
\emng
\eentry

\bentry
\word{granulocyte}
\pron{gArxnuyxlaseYTf}
\gl{\nA}
\bmng
 (\shavi) gArxYxnuyxloseYTf; tamamx seYTopAlxsaZmfnalilx kaNagaLanunx hoMdiruva bagebageya biLirakatx koVshagaLalolxMdu. 
\emng
\eentry

\bentry
\word{granulocytic}
\pron{gArxYxnuyxlasiTikf}
\gl{\gu}
\bmng
 (\shavi) gArxYxnuyxloseYTfna. 
\emng
\eentry

\bentry
\word{granulometric}
\pron{gArxYxnuyxlameTirxkf}
\gl{\gu}
\bmng
 maraLu \mo vugaLa mAdariyalilx (yAvudaradeV) kaNagaLa haMcikeya vinAyxsada yA adakekx saMbaMdhisida. 
\emng
\eentry

\bentry
\word{granulous}
\pron{gArxYxnuyxlasf}
\gl{\gu}
\bmng
  = \hyperlink{granular}{granular}. 
\emng
\eentry

\bentry
\word{grape}
\pron{gerxVpf}
\gl{\nA}
\bmng
\bnum
\num{1} (hasuru, bUdu, yA kapupx baNaNxda) dArxkiSx. 
\num{2}  = \hyperlink{grape-shot}{grape-shot}. 
\num{3} (\bava dalilx) (kudure \mo vugaLa gorasina meVlABxgadalilx yA danagaLa shAvxsakoVshAvaraNada poreya meVle beLeyuva) dArxkiSx geDeDx; dArxkiSx goMcalinAkArada dumARMsa. 
\enum
\emng

\noindent
\gl{\pagu}
\bmng
\hypertarget{grape pagu1}{} 
\bnum
\num{1} \eng{the grape} veYnu. 
\num{2} \eng{the juice of the grape} = \hyperlink{grape pagu1}{?pagu? \((1)\)}. 
\enum
\emng

\noindent
\gl{\nuga}
\bmng
\hyperdef{G}{grape nuga(1)}{} 
\hypertarget{grape nuga1}{} 
\bnum
\num{1} \eng{sour grapes} (bayasidudx keYgeTukadAga adanunx haLiyuvavana \vi\ vayxMgoyxVkitx) bariya huLidArxkiSx; kelasakekx bArada vasutx. 
\num{2} \eng{the grapes are sour} = \hyperlink{grape nuga1}{?nuga? \((1)\)}. 
\enum
\emng
\eentry

\bentry
\word{grape-brandy}
\pron{gerxVpfbArxYxniDx}
\gl{\nA}
\bmng
 dArxkiSxbArxMdi; beVre EnanUnx beresade, dArxkiSx haNaNxnunx yA veYnanunx mAtarx Asavisi mADida bArxMdi (madayx). 
\emng
\eentry

\bentry
\word{grape-cure}
\pron{gerxVpfkuyxarf}
\gl{\nA}
\bmng
 dArxkASx cikitesx; dArxkiSx haNaNxnunx dhArALavAgi koTuTx (kaSxya roVga \mo) roVgagaLanunx guNapaDisuva cikitesx. 
\emng
\eentry

\bentry
\word{grapefruit}
\pron{gerxVpfphUrxTf}
\gl{\nA}
\bmng
 (\bava\ adeV). cakokxVtada baLagada oMdu bageya haNuNx. 
\emng
\eentry

\bentry
\word{grape-house}
\pron{gerxVpfhwsf}
\gl{\nA}
\bmng
 dArxkiSx mane; dArxkASxgaqha; dArxkiSx beLesalu kaTiTxruva, anukUla havAguNa odagisuva kaTaTxDa. 
\emng
\eentry

\bentry
\wordnospeech{grape hyacinth}{grape hyacinth}
\pron{?}
\gl{\nA}
\bmng
 dArxkiSx hayasiMtf; \sA\ niVli hUgoMcalina, masakxri kulada saNaNx giDa. 
\emng
\eentry

\bentry
\word{grapery}
\pron{gerxVpari}
\gl{\nA}
\bmng
 dArxkiSxya -- toVTa yA mane. 
\emng
\eentry

\bentry
\word{grape-scissors}
\pron{gerxVpfsisaZrfsx}
\gl{\nA}
\bmng
 dArxkiSxgatatxri; dArxkiSx beLeya AraMBakAladalilx goMcalugaLanunx katatxrisi viraLagoLisuva yA tinunxvAga goMcalu katatxrisalu baLasuva katatxri. 
\emng
\eentry

\bentry
\word{grape-shot}
\pron{gerxVpfSATf}
\gl{\nA}
\bmng
 (\ca) cadaruguMDu; halavu saNaNx saNaNx guMDugaLanunx ciVla \mo vugaLalilx oTiTxge iTuTx PiraMgige hAki hoDeyutitxdudxdariMda sutatxlU siDiyutitxdadx cikakx guMDugaLu. 
\emng
\eentry

\bentry
\word{grape-stone}
\pron{gerxVpfsoTxVnf}
\gl{\nA}
\bmng
 dArxkiSxya biVja. 
\emng
\eentry

\bentry
\word{grape-sugar}
\pron{gerxVpfSugarf}
\gl{\nA}
\bmng
 dArxkiSx sakakxre; DekfsxTorxVsf yA gUlxkoVsf. 
\emng
\eentry

\bentry
\word{grapevine}
\pron{gerXpfveYnf}
\gl{\nA}
\bmng
\bnum
\num{1} dArxkiSxbaLiLx; dArxkASxlate. 
\num{2} dArxkiSxbaLiLx; sekxVTiMgf ATadalilx eraDu aDigaLanUnx himada meVliTuTx baLiLx baLiLxyAgi racisuva Akaqti. 
\hypertarget{grapevine(3)}{} 
\num{3} guTuTxbaLiLx; gupatxtaMti; gupatx (sudidx) mAdhayxma; rahasayx (vadaMti) mAdhayxma; vadaMtigaLu, beVre riVtiyalilx tiLiyalAgada rahasayx samAcAragaLu tiLiyabaruva yA parxcAravAguva, vayxkitxyiMda vayxkitxge bAyiya mUlakada \mo\ vidhAna yA mAdhayxma: \eng{I heard on the grapevine that he is to be promoted} avanige baDitx sigutatxdeMdu nAnu gupatx mAdhayxmadiMda keVLide. 
\enum
\emng
\eentry

\bentry
\wordnospeech{grapevine telegraph}{grapevine telegraph}
\pron{?}
\gl{\nA}
\bmng
  = \hyperlink{grapevine(3)}{grapevine (3)}. 
\emng
\eentry

\bentry
\word{grapey}
\pron{gerxVpi}
\gl{\gu}
\bmng
  = \hyperlink{grapy}{grapy}. 
\emng
\eentry

\bentry
\word[graph(1)]{graph}
\pron{gArxYxphf}
\gl{\nA}
\bmng
\bnum
\num{1} vakarx; gArxphu; gaNitadalilx oMdanonxMdu avalaMbisi vayxtAyxsavAgutitxruva eraDu parimANagaLigiruva saMbaMdhavanunx sUcisuva reVKe. \imglink{graphfigure}{\raisebox{-0.15cm}[0pt][0pt]{\pdfimage width 0.5cm height 0.5cm {G_Pictures/graph.jpg}}} 
\num{2} reVKAcitarx; gArxYxphu; nakeSx; vayxtAyxsavAgutitxruva yAvudeV parimANagaLigiruva saMbaMdhavanunx sUcisuva biMdugaLa samUha, gere, yA keSxVtarx. 
\enum
\emng
\eentry

%%%%%%%%%%%%%
\bentry
\word[graph(2)]{graph}
\pron{gArx(gArxYx)phf}
\gl{\sakirx}
\bmng
 (vakarxvanunx, nakeSxyanunx, yA reVKAcitarxvanunx) racisu; nimiRsu. 
\emng
\eentry

\bentry
\word[graph(3)]{graph}
\pron{gArxYx(gArx)phf}
\gl{\nA}
\bmng
 (\BAshA) dhavxnimeyanunx (\eng{phoneme}) yA itara vAgaMshavanunx sUcisuva cAkuSxSa saMkeVta, \kanmu\ akaSxra(gaLu). 
\emng
\eentry

\bentry
\wordwithhyphen{hyp-graph}{-graph}
\pron{-gArxphf}
\gl{\uparx}
\bmng
 -liKita, -leVKaka, leVKana eMba athaRgaLalilx baLasuva \uparx \udA: 
\bnum
\num{1} iMtha riVtiyalilx baredadudx, liKita eMba athaRdalilx: \eng{autograph, chirograph, holograph, lithograph, photograph.} 
\num{2} yAvudoV gotAtxda sAdhanada mUlaka dAKalisuva upakaraNa, yaMtarx eMba athaRdalilx: \eng{heliograph, seismograph, telegraph.} 
\num{3} iMtha riVtiyalilx bare, reVKisu eMba athaRdalilx: \eng{calligraph, hectograph.} 
\enum
\emng
\eentry

\bentry
\word{graphematic}
\pron{gArxYxphiVmAYxTikf}
\gl{\gu}
\bmng
 (\BAshA) 
\bnum
\num{1} lipimeya yA lipimege saMbaMdhisida; gArxYxphiVmina yA gArxYxphiVmige saMbaMdhisida. 
\num{2} lipiVya; akaSxriVya; lipiya yA lipige saMbaMdhisida. 
\enum
\emng
\eentry

\bentry
\word{grapheme}
\pron{gArxYxphiVmf}
\gl{\nA}
\bmng
 (\BAshA) 
\bnum
\num{1} lipime; gArxphiVmu; yAvudeV BASeya baravaNige vayxvasethxya mUlamAna, kaniSaThx EkamAna; athaRbadadhxvAgiruvaMte inUnx hecicxna saMKeyxya GaTakagaLAgi oDeyalAgada baravaNigeya aMsha; akaSxrime; vaNiRme. 
\num{2} lipi; akaSxra; vaNaR; BASeya dhavxnimeyoMdanunx barahadalilx saMkeVtisuva akaSxra \mo\ cihenxgaLu yA avugaLa vagaR. 
\enum
\emng
\eentry

\bentry
\word{graphemic}
\pron{garxphiVmikf}
\gl{\gu}
\bmng
  = \hyperlink{graphematic}{graphematic}. 
\emng
\eentry

\bentry
\wordwithhyphen{hyp-grapher}{-grapher}
\pron{-garxpharf}
\gl{\uparx}
\bmng
 iMtha-\eng{graphy} yA leVKanadalilx kushalanAdavanu eMbathaRdalilx \nA gaLanunx rUpisuva \uparx: \eng{geographer, radiographer.} 
\emng
\eentry

\bentry
\word[graphic(1)]{graphic}
\pron{gArxYxphikf}
\gl{\gu}
\bmng
\bnum
\num{1} reVKanada. 
\num{2} vaNaRcitarxda. 
\num{3} ketatxneya. 
\num{4} koretada; utikxVNaRda. 
\num{5} kaNiNxge kaTiTxdaMte vaNiRsida; sajiVvaveMbaMte vaNiRsida: \eng{a graphic account of the event} GaTaneya bagegx kaNiNxge kaTiTxdaMtha varadi. 
\num{6} barahada: baravaNigeya: leVKanada. 
\num{7} (KanijagaLa \vi) liKita; meVlemxY meVle yA siVLumeYgaLa meVle baravaNigeyaMtha gurutugaLiruva. 
\num{8} reVKAcitarxgaLa, nakeSxgaLa yA saMkeVtareVKegaLa yA avugaLige saMbaMdhisida. 
\num{9} gArxYxphikf kalegaLa yA avugaLige saMbaMdhisida. 
\enum
\emng
\eentry

\bentry
\word[graphic(2)]{graphic}
\pron{gArxYxphikf}
\gl{\nA}
\bmng
 gArxYxphikf kalAkaqti; gArxYxphikf vidhAnada mUlaka racisida kalAkaqti. 
\emng
\eentry

\bentry
\wordwithhyphen{hyp-graphic}{-graphic}
\pron{-gArxYxphikf}
\gl{\saupa}
\bmng
 iMtha-\eng{graphy}yA yA\eng{-graphy}yiMda eMba athaRda \saupa: \eng{geographic, photographic.} 
\emng
\eentry

\bentry
\word{graphical}
\pron{gArxYxphikalf}
\gl{\gu}
\bmng
\bnum
\num{1}  = \hyperlink{graphic(1)}{$^1$graphic}. 
\num{2} gArxYxphina; vakarxda; reVKAcitarxda. 
\enum
\emng
\eentry

\bentry
\wordwithhyphen{hyp-graphical}{-graphical}
\pron{-gArxYxphikalf}
\gl{\saupa}
\bmng
  = \hyperlink{hyp-graphic}{-graphic}. 
\emng
\eentry

\bentry
\word{graphically}
\pron{gArxYxphikali}
\gl{\kirxvi}
\bmng
\bnum
\num{1} citarxdalilxruvaMte; citirxsidaMte; susapxSaTxvAgi; kaNiNxge kaTiTxdaMte. 
\num{2} baravaNigeyalilx; liKitavAgi; leVKanada mUlaka. 
\num{3} citArxtamxkavAgi yA nakASxtamxkavAgi; citarxgaLu yA reVKAcitarxgaLiMda kUDi yA racitavAgi. 
\enum
\emng
\eentry

\bentry
\wordwithhyphen{hyp-graphically}{-graphically}
\pron{-gArxYxphikali}
\gl{\saupa}
\bmng
 iMtha-\eng{graphy}yiMda yA iMtha -- \eng{graphy}yA mUlaka eMbathaRdalilx baLasuva \saupa. 
\emng
\eentry

\bentry
\wordnospeech{graphic arts}{graphic arts}
\pron{?}
\gl{\nA}
\bmng
 (\bava) gArxYxphikf kalegaLu; reVKanakalegaLu: 
\banum
\alnum{a} capapxTeya talada yA samatalada mAdhayxmada meVle bareyuvudu, acocxtutxvudu, alaMkAra citarxracane, vaNaRcitarx racisuvudu, \mo\ kalegaLu. 
\alnum{b} phalaka, acucx, \mo vugaLiMda mUla citarxvoMdara nakalugaLanunx tegeyuva ketatxne, koreta, maradacucx, kalalxcucx, \mo\ kalegaLu yA vidhAnagaLu. 
\eanum
\emng
\eentry

\bentry
\wordnospeech{graphic formula}{graphic formula}
\pron{?}
\gl{\nA}
\bmng
 (\ravi) racanAsUtarx; aNuvinalilx paramANugaLu oMdaroDanoMdu baMdhisikoMDiruva karxmavanunx sUcisuva sUtarx. 
\emng
\eentry

\bentry
\word{graphics}
\pron{gArxYxphikfsx}
\gl{\nA}
\bmng
 (\bava) gArxYxPikfsx: 
\banum
\alnum{a} gaNita niyamagaLa parxkAra tirxvimitiVya kAyagaLa citarxgaLanunx divxvimitiVya talada meVle mUDisuva shAsatxrX. 
\alnum{b} nakeSxracane; reVKAcitarx nimARNa; kaMpUyxTarina sahAyadiMda reVKAcitarxgaLu, nakeSxgaLu, mAdarigaLu, \mo vanunx tayArisuvudu. 
\alnum{c} sacitarxte; mudirxta baravaNigeya jotege citarxgaLu, nakeSxgaLu, reVKAcitarxgaLu iruvudu. 
\alnum{d} gArxYxphikf kalegaLu. 
\eanum
\emng
\eentry

\bentry
\word{graphite}
\pron{gArxYxpheYTf}
\gl{\nA}
\bmng
 gArxYxpheYTf; siVsada kaDiDxgaLalilx, viduyxdAvxragaLalilx, mUsegaLalilx matutx herepadAthaRgaLalilx upayoVgisuva kAbaRninxna maqduvAda BinanxrUpagaLalolxMdu. 
\emng
\eentry

\bentry
\word{graphitic}
\pron{garxphiTikf}
\gl{\gu}
\bmng
 gArxYxpheYTinaMtha, adakekx saMbaMdhisida, adanonxLagoMDa yA adariMda janayxvAda. 
\emng
\eentry

\bentry
\word{graphitize}
\pron{gArxYxphiTeYsfZ}
\gl{\sakirx}
\bmng
 gArxYxpheYTAgisu. 
\emng

\noindent
\gl{\akirx}
\bmng
 gArxYxpheYTAgu. 
\emng
\eentry

\bentry
\word{graphological}
\pron{gArxYxphalAjikalf}
\gl{\gu}
\bmng
 (\BAshA) lipiVya hAgU lipishAsitxrXVya; lipigaLige hAgU lipishAsatxrXkekx saMbaMdhisida; mudirxta saMkeVtagaLa hAgU barehada padadhxtigaLa adhayxyanakekx saMbaMdhisida. 
\emng
\eentry

\bentry
\word{graphologist}
\pron{gArxYxphAlajisfTx}
\gl{\nA}
\bmng
 (\BAshA) lipishAsatxrXjacnx; mudirxta saMkeVtagaLa hAgU barehada padadhxtigaLanunx balalxva. 
\emng
\eentry

\bentry
\word{graphology}
\pron{garxphAlaji}
\gl{\nA}
\bmng
 hasAtxkaSxra -- videyx, kale; hasAtxkaSxra sAmudirxka; obabxna keYbarahavanunx noVDi avana shiVla, savxBAvagaLanunx Uhisuva kale. 
\bnum
\num{2} racanAsUtarx padadhxti; racanAsUtarxgaLanunx baLasuva padadhxti. 
\num{3} (\BAshA) lipishAsatxrX; mudirxta saMkeVtagaLa hAgU barehada padadhxtigaLa adhayxyana. 
\enum
\emng
\eentry

\bentry
\word{graphotype}
\pron{gArxYxphaTeYpf}
\gl{\nA}
\bmng
\bnum
\num{1} gArxYxphacucx; samatala mudarxNakAkxgi tayArisida ububx citarxda paDiyacucx. 
\num{2} iMtha paDiyacacxnunx tayArisuva vidhAna. 
\enum
\emng
\eentry

\bentry
\word{graphy}
\pron{gArxYxPi}
\gl{\nA}
\bmng
 (\BAshA)  = \hyperlink{graph(3)}{$^3$graph}. 
\emng
\eentry

\bentry
\wordwithhyphen{hyp-graphy}{-graphy}
\pron{-garxphi}
\gl{\saupa}
\bmng
\bnum
\num{1} reVKana yA leVKanada sheYligaLu eMbathaRda \nA gaLanunx sUcisuva \saupa. \udA: \eng{lithography, brachygraphy, stenography, calligraphy.} 
\num{2} vivaraNAtamxka vijAcnxnagaLa hesarugaLanunx rUpisuva \saupa. \udA: \eng{geography, bibliography.} 
\enum
\emng
\eentry

\bentry
\word{grapnel}
\pron{gArxYxpfnalf}
\gl{\nA}
\bmng
 gArxYxpfnalf: 
\banum
\alnum{a} koMDihagagx; koMDipAsha; yAvudeV vasutxvanunx (\kanmu\ shaturxnwkeyanunx) hiDideLedukoLaLxlu hagagx kaTiTx biVsuva kabibxNada koMDigaLuLaLx salakaraNe. 
\alnum{b} koMDilaMgaru; (doVNigaLanunx, AkAshabuTiTxgaLanunx hiDidu nililxsalu baLasuva) halavu koMDigaLuLaLx saNaNx laMgaru; pAtALagaraDi. \imglink{grapnel-bfigure}{\raisebox{-0.30cm}[0pt][0pt]{\pdfimage width 0.7cm height 0.6cm {G_Pictures/grapnel-b.jpg}}} 
\eanum
\emng
\eentry

\bentry
\word{grappa}
\pron{gArxYxpa}
\gl{\nA}
\bmng
 gArxYxpa; (veYnanunx tayArisida naMtara uLiyuva dArxkiSx) caraTavanunx baTiTxyiLisi tayArisida bArxMdi. 
\emng
\eentry

\bentry
\word[grapple(1)]{grapple}
\pron{gArxYxpflf}
\gl{\nA}
\bmng
\bnum
\num{1} =  \hyperlink{grapnel}{grapnel}. 
\num{2} kusitxgArara, jaTiTxgaLa (yA avara hiDitadaMtha) bigihiDita. 
\num{3} muSATxmuSiTx; hasAtxhasitx; keYkeY kALaga; keYkeY milAyisi hoVrADuvudu. 
\enum
\emng
\eentry

\bentry
\word[grapple(2)]{grapple}
\pron{gArxYxpflf}
\gl{\sakirx}
\bmng
\bnum
\num{1} gArxyXpfnalfnalilx (yA adaralilx hiDidaMte) hiDideLeduko yA bigiyAgi hiDi. 
\num{2} (keYgaLiMda) hiDiduko; bigi hiDi. 
\num{3} keYkeY hatutx, milAyisu. 
\enum
\emng

\noindent
\gl{\akirx}
\bmng
 keYkeY milAyisi -- hoVrADu, kALaga mADu. 
\emng

\noindent
\gl{\nuga}
\bmng
 \eng{grapple with} jayisalu, sAdhisalu yatinxsu yA edurisu: \eng{science grapples with such startling phenomena} vijAcnxnavu iMtaha becicxbiVLisuva vidayxmAnagaLanunx edurisalu yatinxsutatxde. 
\emng
\eentry

\bentry
\word{grappling}
\pron{gArxyxpilxMgf}
\gl{\nA}
\bmng
\bnum
\num{1} gArxYxpfnalfnalilx (yA adaralilx hiDidaMte) hiDideLedukoLuLxvudu yA bigiyAgi hiDiyuvudu. 
\num{2} (keYgaLiMda) hiDidukoLuLxvudu; bigi hiDiyuvudu. 
\num{3} hasAtxhasitx; keYkeY milAyisuvudu yA milAyisi hoVrADuvudu. 
\num{4} jayisalu, sAdhisalu -- yatinxsuvudu. 
\num{5} gArxYxpfnalf; hiDiyuva, kabibxNada salakaraNe. 
\enum
\emng
\eentry

\bentry
\word{grappling-hook}
\pron{gArxYxpilxMgf hukf}
\gl{\nA}
\bmng
  = \hyperlink{grapnel}{grapnel}. 
\emng
\eentry

\bentry
\word{grappling-iron}
\pron{gArxYxpilxMgf aianfR}
\gl{\nA}
\bmng
  = \hyperlink{grapnel}{grapnel}. 
\emng
\eentry

\bentry
\word{graptolite}
\pron{gArxYxpaTxleYTf}
\gl{\nA}
\bmng
 gArxYxpoTxleYTu; AdiBUyugada (\eng{Palaeozoic}) taLabaMDegaLalilx gurutugaLanunx biTiTxruva, paLeyuLike rUpadalilxruva, Iga aLiduhoVgiruva, oMdu sAgarada pArxNi. 
\emng
\eentry

\bentry
\word{grapy}
\pron{gerxVpi}
\gl{\gu}
\bmng
\bnum
\num{1} dArxkiSxya. 
\num{2} dArxkiSx baLiLxya. 
\num{3} dArxkiSxyiMda kUDida. 
\num{4} dArxkiSxya ruciya. 
\enum
\emng
\eentry

\bentry
\word[grasp(1)]{grasp}
\pron{gArxsfpx}
\gl{\sakirx}
\bmng
\bnum
\num{1} AturadiMda yA durAsheyiMda kasiduko; durAsheyiMda vashapaDisiko. 
\num{2} BadarxvAgi hiDi; bigiyAgi hiDi. 
\num{3} manasisxniMda -- garxhisu, ari: \eng{grasped the rudiments of the subject} viSayada mUla tatatxvXgaLanunx garxhisida, aritukoMDa. 
\enum
\emng

\noindent
\gl{\akirx}
\bmng
 hiDidukoLaLxlu yatinxsu: \eng{like a drowning man grasping at a straw} muLugutitxruvavanu oMdu hululxkaDiDxyanenxV hiDidukoLaLxlu yatinxsuvaMte. 
\emng

\noindent
\gl{\nuga}
\bmng
 \eng{grasp the nettle} muLuLx hiDi; kaSaTxvanunx yA apAyavanunx dheYyaRdiMda edurisu. 
\emng
\eentry

\bentry
\word[grasp(2)]{grasp}
\pron{gArxsfpx}
\gl{\nA}
\bmng
\bnum
\num{1} BadarxvAda hiDita; bigihiDita; kapimuSiTx: \eng{within one's grasp} hiDitakekx sikukxva, eTakuva. \eng{beyond one's grasp} hiDitakekx sikakxda, miVrida (\rUpa\ saha). 
\num{2} vasha; hatoVTi; hiDita; parxButavx; sAvxdhiVna: \eng{grasp of death} maqtuyxvasha. 
\num{3} garxhike; arivu; manavarike; samagarx jAcnxna; jAcnxnavAyxpitx; budidhx vAyxpitx: \eng{his mind takes into its grasp the immensity of the science} A shAsatxrXda agAdha viSayavanenxlalx avana manasusx samagarxvAgi garxhisutatxde, vashamADikoLuLxtatxde. 
\enum
\emng
\eentry

\bentry
\word{graspable}
\pron{gArxsapxbflf}
\gl{\gu}
\bmng
 ariyalu sAdhayxvAda; garxhisabalalx; gArxhayx. 
\emng
\eentry

\bentry
\word{grasping}
\pron{gArxsipxMgf}
\gl{\gu}
\bmng
\bnum
\num{1} BadarxvAgi hiDidukoLuLxva; bigihiDitada. 
\num{2} durAsheya; durAsheyiMda kUDida; loVBada; atAyxseya: \eng{a grasping nature} durAsheya savxBAva. 
\enum
\emng
\eentry

\bentry
\word{graspingly}
\pron{gArxsipxMgfli}
\gl{\kirxvi}
\bmng
\bnum
\num{1} durAsheyiMda(kUDi); atAyxseyiMda. 
\num{2} BadarxvAgi hiDidukoMDu yA hiDidukoLuLxvaMte. 
\enum
\emng
\eentry

\bentry
\word{graspingness}
\pron{gArxsipxMgfnisf}
\gl{\nA}
\bmng
 durAsheya budidhx, parxvaqtitx, savxBAva; loVBa. 
\emng
\eentry

\bentry
\word[grass(1)]{grass}
\pron{gArxsf}
\gl{\nA}
\bmng
\bnum
\num{1} hululx; taqNa. 
\num{2} (\savi) taqNavagaR; (AhAradhAnayxgaLu, joMDugaLu matutx bidirugaLanunx oLagoMDa yAvudeV) hulilxna jAti. 
\num{3} (\ashi) = \hyperref{kandict_a.pdf}{A}{asparagus}{asparagus}. 
\num{4} meVvu. 
\num{5} hululxgAvalu; goVmALa. 
\num{6} meVvina jamiVnu. 
\num{7} hululx (beLediruva) nela: \eng{keep off the grass} hululx biTuTx naDe; hululx tuLiyabeVDa; hulilxniMdAce iru. 
\num{8} (\gaNi) gaNiya bAyi yA bAyiya sutatxNa jAga. 
\num{9} (\ashi) = \hyperref{kandict_m.pdf}{M}{marijuana}{marijuana}. 
\num{10} (\ashi) poliVsf ilAKege aparAdhada yA aparAdhiya suLivu tiLisuvava. 
\enum
\emng

\noindent
\gl{\pagu}
\bmng
\bnum
\num{1} \eng{be at grass} hululxgAvalinalilx (meVyutatx) iru, meVvinalilxru. 
\num{2} \eng{go to grass} hululxgAvalige hoVgu; meVvige hoVgu. 
\hypertarget{grass pagu3}{} 
\num{3} \eng{put to grass} goVmALakekx kaLuhisu; meVvige hoVgu, kaLuhisu. 
\num{4} \eng{send to grass} = \hyperlink{grass pagu3}{?pagu? \((3)\)}. 
\num{5} \eng{turn to grass} goVmALakekx, meVvige aTuTx. 
\enum
\emng

\noindent
\gl{\nuga}
\bmng
\bnum
\num{1} \eng{at grass} kelasavilalxde; biDuvAgi; hAyAgi; virAmasuKa anuBavisutatx. 
\num{2} \eng{go to grass} (vayxkitx) hoDeta tiMdu biVLu; nelakukxruLu. 
\num{3} \eng{let the grass grow under one's feet} kelasadalilx cacacxravAgiru, cacacxravahisu, taDamADabeVDa, shiVGarxvAgi vatiRsu, Pakakxne avakAsha hiDiduko. 
\num{4} \eng{hear the grass grow} asAdhAraNavAda, loVkoVtatxravAda sUkaSxmXgarxhaNashakitx paDediru; kushAgarxbudidhxyuLaLxvanAgiru; aduBxtavAgi curukAgiru. 
\num{5} \eng{send to grass} (vayxkitxyanunx) hoDeduruLisu; nelakukxruLisu. 
\enum
\emng
\eentry

\bentry
\word[grass(2)]{grass}
\pron{gArxsf}
\gl{\sakirx}
\bmng
\bnum
\num{1} (parxdeVshavoMdara meVle) hululx beLesu; hululxhepupx beLe. 
\num{2} (agaseVnAru \mo vanunx) biLicisalu hulilxna meVle haravu, haraDu. 
\num{3} (edurALiyanunx) hoDeduruLisu; hoDedu nelada meVle keDavu, uruLisu. 
\num{4} (miVnanunx) daDakekx taru; daDada meVlakekx taru. 
\num{5} (hakikxyanunx guMDeVTiniMda) hoDedu keDavu; nelakekx biVLisu. 
\num{6} (\ame) (pArxNige yA pArxNigaLige meVyalu) hululxgAvalu, hululx -- odagisu. 
\num{7} (\ashi) (obabxnanunx) moVsadiMda opipxsu; (obabxnige) vishAvxsadorxVha mADu. 
\enum
\emng

\noindent
\gl{\akirx}
\bmng
(\ashi) poliVsarige (aparAdhada yA aparAdhiya) suLivu koDu. 
\emng
\eentry

\bentry
\word{grass-box}
\pron{gArxsfbAkfsx}
\gl{\nA}
\bmng
 hululxDababx; hululx koyuyxva yaMtarxdalilx katatxrisida hululx biVLuva, saMgarxhavAguva -- Dababx, peTiTxge. 
\emng
\eentry

\bentry
\word{grass-cloth}
\pron{gArxsfkAlxtf}
\gl{\nA}
\bmng
 hululxbaTeTx; nArubaTeTx; nAru \mo vugaLiMda heNeda baTeTx. 
\emng
\eentry

\bentry
\wordnospeech{grass court}{grass court}
\pron{?}
\gl{\nA}
\bmng
 hulilxna Tenisf -- koVTuR, kirxVDAMgaNa. 
\emng
\eentry

\bentry
\word{grasshopper}
\pron{gArxsfhAparf}
\gl{\nA}
\bmng
kupapxLisuva, kiVcuva -- miDate. \imglink{grasshopperfigure}{\raisebox{-0.15cm}[0pt][0pt]{\pdfimage width 0.9cm height 0.5cm{G_Pictures/grasshopper.jpg}}} 
\emng
\eentry

\bentry
\word{grassland}
\pron{gArxsflAYxMDf}
\gl{\nA}
\bmng
 hululxgAvalu; hululxBUmi; meVyisalu baLasuva, hulilxniMda Avarisida BUmi. 
\emng
\eentry

\bentry
\word{grassless}
\pron{gArxsflisf}
\gl{\gu}
\bmng
 hulilxlalxda; taqNarahita. 
\emng
\eentry

\bentry
\wordnospeech{grass of Parnassus}{grass of Parnassus}
\pron{?}
\gl{\nA}
\bmng
 kucicxnaMtha elegaLuLaLx, pAnARsiya kulada oMdu giDa. 
\emng
\eentry

\bentry
\wordnospeech{grass parakeet}{grass parakeet}
\pron{?}
\gl{\nA}
\bmng
 (\AseTxrXV) hululxgiLi; hecAcxgi hululxgAvalige baruva, ujavxla vaNaRda giLi. 
\emng
\eentry

\bentry
\wordnospeech{grass roots}{grass roots}
\pron{?}
\gl{\nA}
\bmng
\bnum
\num{1} mUlamaTaTx. 
\num{2} mUla; Akara. 
\num{3} (rAjakiVya) matadAraru; OTudAraru; OTu koDuva maMdi. 
\num{4} janasAmAnayxru; sAmAnayx janate; rAjakiVya nidhARra tegedukoLuLxvudariMda dUraviruva, Adare A nidhARragaLa pariNAmakekx oLagAguva sAmAnayx jana. 
\num{5} (deVshavoMdara) gArxmiVNa parxdeVsha; gArxmAMtara parxdeVsha; kaqSi valaya. 
\num{6} gArxmiVNa janate; gArxmAMtara jana; rAjikiVya, sAmAjika yA AthiRka vagaRvAgi gArxmAMtara parxdeVshagaLalilx vAsisuva jana, maMdi. 
\enum
\emng
\eentry

\bentry
\wordnospeech{grass skirt}{grass skirt}
\pron{?}
\gl{\nA}
\bmng
 hululxlaMga; udadxvAda hululxgaLanUnx elegaLanUnx soMTapaTiTxge kaTiTx mADida tuMDulaMga. 
\emng
\eentry

\bentry
\word{grass-snake}
\pron{gArxsfsenxVkf}
\gl{\nA}
\bmng
\bnum
\num{1} sututx paTiTxyuLaLx yUroVpina (sAmAnayx) hAvu. 
\num{2} amerika saMyukatx saMsAthxnada (sAmAnayx) hasuru hAvu. 
\enum
\emng
\eentry

\bentry
\wordnospeech{grass widow}{grass widow}
\pron{?}
\gl{\nA}
\bmng
 tAtAkxlika vidhave; haMgAmi vidhave; maneyiMda savxlapx kAla geYruhAjarAgiruvavana yA Uralilxlalxdavana heMDati. 
\emng
\eentry

\bentry
\wordnospeech{grass widower}{grass widower}
\pron{?}
\gl{\nA}
\bmng
 tAtAkxlika vidhura; haMgAmi vidhura; maneyiMda savxlapx kAla geYruhAjarAgiruvavaLa yA UralilxlalxdavaLa gaMDa. 
\emng
\eentry

\bentry
\word{grass-wrack}
\pron{gArxsfrAYxkf}
\gl{\nA}
\bmng
 = \hyperref{kandict_e.pdf}{E}{eel-grass}{eel-grass}. 
\emng
\eentry

\bentry
\wordnospeech{grass wren}{grass wren}
\pron{?}
\gl{\nA}
\bmng
 (\AseTxrXV) amiToVniRsf kulada oMdu saNaNx hADuhakikx. 
\emng
\eentry

\bentry
\word{grassy}
\pron{gArxsi}
\gl{\gu}
\bmng
\bnum
\num{1} hulilxna. 
\num{2} hululx mucicxruva, kavidiruva. 
\num{3} hululx heVraLavAgiruva. 
\num{4} hululx beLediruva. 
\num{5} hulilxnaMtha. 
\enum
\emng
\eentry

\bentry
\word[grate(1)]{grate}
\pron{gerxVTf}
\gl{\nA}
\bmng
\hypertarget{grate(1)1}{} 
\bnum
\numi{1} (\viparx) jAlari: 
\banum
\alnum{a} samAMtaravAgi yA oMdanonxMdu aDaDxhAyuvaMte joVDisiruva marada yA loVhada saraLugaLa vayxvasethx. 
\alnum{b} (\daqvi) vivataRneya (\eng{diffraction}) mUlaka roVhitavanunx utapxtitx mADalu joVDisiruva samAMtara taMtigaLa taMDa yA gAjina meVle eLediruva samAMtara geregaLu. 
\eanum
\numie
\num{2} (beMkigUDinalilx, oleyalilx yA kulumeyalilx uruvalu horakekx biVLadaMte taDeyuva) saraLina cwkaTuTx; (beMkiya) saraLu taDe. 
\num{3} ole yA kulume. 
\enum
\emng
\eentry

\bentry
\word[grate(2)]{grate}
\pron{gerxVTf}
\gl{\sakirx}
\bmng
\bnum
\num{1} turi; here; oraTAda horameY meVle ujujxvudariMda saNaNx saNaNx kaNgaLanAnxgi, rajagaLanAnxgi mADu. 
\num{2} (halulx) kaDi; mase. 
\num{3} kakaRsha shabadxvAguvaMte oMdara meVle ujujx. 
\enum
\emng

\noindent
\gl{\akirx}
\bmng
\bnum
\num{1} manasisxge -- ahitavAgu, kirikiriyuMTumADu: \eng{you have a knack for choosing what grates on the mind} manasisxge ahitavAguvaMthadanenxV Arisuvudaralilx niVnu kushala. 
\num{2} kakaRsha shabadxvAguvaMte oMdara meVle ujujx. 
\num{3} gaDasAgi, kakaRshavAgi, paruSavAgi -- shabadxvAgu, sadAdxgu: \eng{a grating laugh} gaDasAda, kakaRshavAda nagu. 
\num{4} (kiVlu \mo vu) karerxnunx; kiruguTuTx. 
\enum
\emng
\eentry

\bentry
\word{grated}
\pron{gerxVTiDf}
\gl{\gu}
\bmng
\bnum
\numi{1} jAlariyuLaLx: 
\banum
\alnum{a} samAMtaravAgi yA oMdanonxMdu aDaDx hAyuvaMte joVDisiruva marada yA loVhada saraLugaLuLaLx. 
\alnum{b} vivataRneya (\eng{diffraction}) mUlaka roVhitavanunx utapxtitxmADalu samAMtara geregaLanenxLediruva. 
\eanum
\numie
\num{2} (beMkigUDina yA oleya \vi) saraLina taDe yA cwkaTuTx hAkiruva. 
\enum
\emng
\eentry

\bentry
\word{grateful}
\pron{gerxVTfphulf}
\gl{\gu}
\bmng
\bnum
\num{1} (iMdirxyagaLige yA manasisxge) opupxva; hitavAda; pirxyavAda; suKakara; AhAlxdakara; ApAyxyamAna (Iga vasutxgaLanunx yA viSayagaLanunx mAtarx kuritu \parx). 
\num{2} (vayxkitxge, upakArakAkxgi) kaqtajacnx; kaqtajacnxteyuLaLx; upakAra samxraNeyuLaLx; ABAriyAda. 
\enum
\emng
\eentry

\bentry
\word{gratefully}
\pron{gerxVTfphuli}
\gl{\kirxvi}
\bmng
\bnum
\num{1} hitavAgi; pirxyavAgi; suKakaravAgi; AhAlxdakaravAgi. 
\num{2} kaqtajacnxtApUvaRkavAgi. 
\enum
\emng
\eentry

\bentry
\word{gratefulness}
\pron{gerxVTfphulfnisf}
\gl{\nA}
\bmng
 kaqtajacnxtA -- budidhx, BAva, manoVvaqtitx, guNa. 
\emng
\eentry

\bentry
\word{grateless}
\pron{gerxVTflisf}
\gl{\gu}
\bmng
 (beMkigUDina yA oleya \vi) saraLutaDe ilalxda. 
\emng
\eentry

\bentry
\word{grater}
\pron{gerxVTarf}
\gl{\nA}
\bmng
 turiyuva maNe; hereyuva maNe; heracu maNe. 
\emng
\eentry

\bentry
\word{graticule}
\pron{gArxYxTikUyxlf}
\gl{\nA}
\bmng
jAlike: 
\banum
\alnum{a} (\daqvi) dUradashaRka yA itara duyxti upakaraNagaLa mUlaka viVkiSxsuva vasutxgaLanunx aLeyalu yA avugaLa sAthxnagaLanunx gurutisalu anukUlavAguvaMte upakaraNadoLage aLavaDisiruva aDaDx matutx udadx geregaLu. 
\alnum{b} (saveRV) nakeSxgeregaLu; nakeSxreVKegaLu; nakeSxya meVle akASxMsha, reVKAMshagaLanunx sUcisuva geregaLu. 
\eanum
\emng
\eentry

\bentry
\word{gratification}
\pron{gArxYxTiphikeVSanf}
\gl{\nA}
\bmng
\bnum
\num{1} (\pArxparx) parxtiPala; (\sA) parxtiPalavAgi koTaTx rusumu, haNa, yA koDuge. 
\num{2} (\pArxparx) laMca; ruSuvatutx. 
\num{3} saMtoVSa; taqpitx; daNivu; AnaMda. 
\num{4} koVridadxnunx salilxsi saMtoVSapaDisuvudu. 
\num{5} (obabxru) bayasidadxnunx neraveVrisuvudu. 
\num{6} (Ashe, manoVBAva, manoVvaqtitxgaLanunx) taDeyilalxde -- hoVgabiDuvudu; savxcaCxMdavAgi hariyabiDuvudu. 
\enum
\emng
\eentry

\bentry
\word{gratify}
\pron{gArxYxTipheY}
\gl{\sakirx}
\bmng
\bnum
\num{1} (\pArxparx) \sA\ (haNarUpadalilx) parxtiPala yA saMBAvane, rusumu yA shulakx, koDuge yA bahumAna -- koDu. 
\num{2} (\pArxparx) laMca koDu; ruSuvatutx koDu. 
\num{3} saMtoVSapaDisu; taNisu; taqpitxpaDisu; AnaMdagoLisu; haSaRgoLisu. 
\num{4} koVridadxnunx salilxsi saMtoVSapaDisu. 
\num{5} (obabxru) bayasidadxkekx samamxtisu; (obabxra) bayake neraveVrisu. 
\num{6} (Ashe, manoVBAva, manaHparxvaqtitx -- ivugaLanunx) taDeyilalxde hoVgabiDu; savxcaCxMdavAgi hariyabiDu. 
\enum
\emng
\eentry

\bentry
\word{gratifying}
\pron{gArxYxTipheYiMgf}
\gl{\gu}
\bmng
 saMtoVSakara; taqpitxkara; AnaMdakara. 
\emng
\eentry

\bentry
\word{gratifyingly}
\pron{gArxYxTipheYiMgfli}
\gl{\kirxvi}
\bmng
 saMtoVSakaravAgi; taqpitxkaravAgi; AnaMdakaravAgi. 
\emng
\eentry

\bentry
\wordf{gratin}
\pron{gArxTAYxnf}
\gl{\nA}
\expl{\F\ }
\bmng
\bnum
\num{1} berxDf puDiyanunx udurisi yA mosarina kene tiVDi eraDu kaDeyU kAvu koTuTx garigari padara ELuvaMte beVyisuva vidhAna. 
\num{2} hiVge tayArisida BakaSxyX. 
\enum
\emng

\noindent
\gl{\pagu}
\bmng
 \eng{au gratin} (\ucAcx -- O gArxYxTAYxnf) berxDf puDi udurisi yA mosarina kene tiVDi eraDu kaDeyU kAvu koTuTx garigari padara ELuvaMte beVyisi tayArisida. 
\emng
\eentry

\bentry
\word{grating}
\pron{gerxVTiMgf}
\gl{\nA}
\bmng
  = \hyperlink{grate(1)1}{$^1$grate \((1 \& 2)\)}. 
\emng
\eentry

\bentry
\word{gratingly}
\pron{gerxVTiMgfli}
\gl{\kirxvi}
\bmng
 gaDasAgi; kakaRshavAgi; paruSavAgi. 
\emng
\eentry

\bentry
\word[gratis(1)]{gratis}
\pron{gerxV(gArx, gArxYx)Tisf}
\gl{\kirxvi}
\bmng
 pukakxTeyAgi; ucitavAgi; biTiTx; muPatAtxgi; parxtiyAgi EnanUnx tegedukoLaLxde. 
\emng
\eentry

\bentry
\word[gratis(2)]{gratis}
\pron{gerxV(gArxYx, gArx)Tisf}
\gl{\gu}
\bmng
 pukakxTe; biTiTx yA muPatAtxgi koTaTx yA mADida. 
\emng
\eentry

\bentry
\word{gratitude}
\pron{gArxYxTiTUYxDf}
\gl{\nA}
\bmng
\bnum
\num{1} upakAra samxraNe; kaqtajacnxte. 
\num{2} parxtuyxpakAra budidhx. 
\enum
\emng
\eentry

\bentry
\word{gratuitous}
\pron{garxTUyxiTasf}
\gl{\gu}
\bmng
\bnum
\num{1} saMpAdisade, bele koDade, biTiTx -- paDeda yA koTaTx; muPatAtxgi doreta yA koTaTx. 
\num{2} koVrada; beVDada; anapeVkiSxtavAda. 
\num{3} samathiRsalAgada; asamathaRniVya. 
\num{4} udedxVshaveV ilalxda. 
\num{5} akAraNa; niSAkxraNa; niniRmitatx: \eng{a gratuitous lie} niSAkxraNavAda suLuLx; yAva kAraNavU ilalxde heVLida suLuLx. 
\enum
\emng
\eentry

\bentry
\word{gratuitously}
\pron{garxTUyxiTasfli}
\gl{\kirxvi}
\bmng
\bnum
\num{1} pukakxTeyAgi; biTiTx; muPatAtxgi. 
\num{2} anapeVkiSxtavAgi. 
\num{3} asamathaRniVyavAgi. 
\num{4} akAraNavAgi; niSAkxraNavAgi; vinA kAraNa. 
\enum
\emng
\eentry

\bentry
\word{gratuitousness}
\pron{garxTUyxiTasfnisf}
\gl{\nA}
\bmng
\bnum
\num{1} pukakxTetana; biTiTxtana; muPatfgiri. 
\num{2} anapeVkiSxtate. 
\num{3} asamathaRniVyate. 
\num{4} akAraNate; niniRmitatxte; nirudidxSaTxte. 
\enum
\emng
\eentry

\bentry
\word{gratuity}
\pron{garxTUyxiTi}
\gl{\nA}
\bmng
\bnum
\num{1} (keLadajeRyavana sahAya seVvegaLanunx parigaNisi dAtanu koDuva) haNarUpada koDuge; inAmu; bakiSxVsu. 
\num{2} (\birx) gArxyXcUyxTi; (seYnayx visajaRneyAdAga, seVveyiMda nivaqtatxnAdAga, yA beVre kelavu saMdaBaRgaLalilx koDuva) koDuge; ucita dhana. 
\num{3} (keTaTx athaRdalilx) laMca. 
\enum
\emng
\eentry

\bentry
\word{gratulate}
\pron{gArxYxTuyxleVTf}
\gl{\kirx}
\bmng
 (\pArxparx) = \hyperref{kandict_c.pdf}{C}{congratulate}{congratulate}. 
\emng
\eentry

\bentry
\word{gratulation}
\pron{gArxTuyxleVSanf}
\gl{\nA}
\bmng
 (\pArxparx) = \hyperref{kandict_c.pdf}{C}{congratulation}{congratulation}. 
\emng
\eentry

\bentry
\word{gratulatory}
\pron{gArxYxTuyxleVTari}
\gl{\gu}
\bmng
 aBinaMdisuva; inonxbabxna jaya \mo vakAkxgi saMtoVSa sUcisuva. 
\emng
\eentry

\bentry
\word{graunch}
\pron{gArxMcf}
\gl{\sakirx}
\bmng
\bnum
\num{1} kiruguTiTxsu; shabadxmADisu; raTaraTisu. 
\num{2} hiVge kiruguTiTxsi, raTaraTisi (yAvudeV yaMtarxvanunx) keDisu. 
\enum
\emng

\noindent
\gl{\akirx}
\bmng
 kiruguTuTx, raTaraTisuva shabadxmADu. 
\emng
\eentry

\bentry
\word{gravamen}
\pron{garxveVmenf}
\gl{\nA}
\expl{(\bava\ \eng{gravamens} yA \eng{gravamina}).}
\bmng
\bnum
\num{1} (\kanmu\ kuMdukorategaLa yA akarxmagaLa bagegx) dUru; manavi; ajiR. 
\num{2} parxdhAna AroVpa; dUrina AroVpada, ApAdaneya -- tiruLu, sAra yA ati kaTuvAda BAga. 
\num{3} (kerxYsatxmaThada akarxmagaLanunx yA anAyxyagaLanunx kurita) pAdirxsaBeya keLamaMDaliyiMda meVlamxMDalige kaLuhisuva manavi, ajiR. 
\enum
\emng
\eentry

\bentry
\word{gravamina}
\pron{garxveVmina}
\gl{\nA}
\bmng
 \eng{gravamen} padada \bava. 
\emng
\eentry

\bentry
\word[grave(1)]{grave}
\pron{gerxVvf}
\gl{\nA}
\bmng
 samAdhi; goVri; shavavanunx hULalu ageda guMDi, kuLi yA adara meVlaNa dibabx yA sAmxraka. 
\bnum
\num{2} satatxsithxti; maqtAvasethx; sAvu; maraNa; maqtuyx. 
\num{3} maqtuyxloVka. 
\num{4} (\rUpa) samAdhi (rUpada yAvudeV vasutx, Asharxya): \eng{watery grave} jalasamAdhi. \eng{the grave of many reputations} eSoTxV janara kiVtiR samAdhi, hesaru maNuNx pAlAguvaMte mADiruva viSaya. 
\num{5} (AlUgeDeDx \mo vanunx hULalu ageda) sAluguMDi. 
\enum
\emng

\noindent
\gl{\nuga}
\bmng
\bnum
\numi{1} \eng{dig grave of} 
\banum
\alnum{a} samAdhi toVDu; goVri -- age, tege. 
\alnum{b} obabxna patanakekx kAraNavAgu; obabxnanunx uruLisu. 
\eanum
\numie
\num{2} \eng{have one foot in the grave} sAvina hositxlalilxru; UralolxMdu kAlu kADalolxMdu kAlu; Uru hoVgu enunx kADu bA enunx; samAdhiyalilx oMdu kAliTiTxru; bahaLa vayasAsxgiru. 
\hyperdef{G}{grave(1) nuga(3)}{} 
\num{3} \eng{make one turn in his grave} satatxvana samAdhiyalilx naraLuvaMte mADu; satatxvanigU asahayxvAguvaMtha AGAtavuMTumADu (avanu badukidAdxga heVsikoLuLxtitxdadxMtha \vi). 
\num{4} \eng{secret as the grave} samAdhiguTiTxna; ati rahasayxvAda. 
\num{5} \eng{someone walking on my grave} yAroV nananx samAdhiya meVle naDeyutitxdAdxre aninxsutatxde (obabxnige EkeMdu heVLalAgade meY naDukavuMTAdAga heVLuva mAtu.). 
\enum
\emng
\eentry

\bentry
\word[grave(2)]{grave}
\pron{gerxVvf}
\gl{\sakirx}
\expl{(\BUkaq\ \eng{graven} yA \eng{graved}).}
\bmng
\bnum
\num{1} (\pArxparx) hULu (\BUkaq\ \eng{graved} mAtarx). 
\num{2} (\vAshi) (hinenxleyAgi baLasuva sAmagirxyanunx, parxtimeyanunx) kore; ketutx; kaDe; kaMDarisu (\BUkaq\ \eng{graven} yA \eng{graved}): \eng{graven image} ketitxda, kaDeda -- parxtime, vigarxha, mUtiR. 
\num{3} (\rUpa) (manasisxnalilx, manasisxna meVle) ketutx; kore; acocxtutx (\BUkaq\ \eng{graven, graved}). 
\enum
\emng
\eentry

\bentry
\word[grave(3)]{grave}
\pron{gerxVvf}
\gl{\gu}
\bmng
\bnum
\num{1} muKayx; parxmuKa; parxdhAna. 
\num{2} Gana; tUkavAda; gurutara. 
\num{3} tiVvarx AloVcane agatayxvAda. 
\numi{4} (tapupxgaLu, kaSaTxgaLu, hoNegArikegaLu matutx roVgada yA apAyada lakaSxNagaLu, \mo vugaLa \vi) 
\banum
\alnum{a} edurisalAgada; asAdhayx; dugaRma. 
\alnum{b} BayaMkara; BiVkara; digilu huTiTxsuva; gAbari huTiTxsuva. 
\alnum{c} ugarx; tiVvarx; gaMBiVra; viSama. 
\eanum
\numie
\num{5} (vayxkitxgaLu, avara naDate, muKaBAva, mAtu, naDavaLike, \mo vugaLa \vi) gaMBiVra; nirADaMbara; saraLa; beDagu -- baNaNx ilalxda; thaLathaLisada; thaLukupaLukilalxda. 
\num{6} (sAthxyiya \vi) madhayxma yA maMdarxsAthxyiya; tAravalalxda; gaMBiVra: \eng{the thicker the string, the graver the tone} taMti dapapxnAgidadxSUTx nAda gaMBiVravAgirutatxde. 
\num{7} (\BAshA) (savxrada \vi) anudAtatx: \eng{grave accent} anudAtatx savxra. 
\enum
\emng
\eentry

\bentry
\word[grave(4)]{grave}
\pron{gArxvf}
\gl{\nA}
\bmng
  = \hyperlink{grave accent}{grave accent}. 
\emng
\eentry

\bentry
\word[grave(5)]{grave}
\pron{gerxVvf}
\gl{\sakirx}
\bmng
 (\ca) (haDagu nelada meVle yA oNa kaTeTxyalilx iruvAga adara taLakekx hatitxkoMDu rAshigUDiruva kasa suTuTx, TArf baLidu) haDagina taLa cokakxTa mADu. 
\emng
\eentry

\bentry
\wordRemoveSpace{grave-accent}{grave accent}
\pron{gArxvf AYxkesxMTf}
\gl{\nA}
\bmng
 (mUru bageya savxragaLalilx oMdAda) anudAtatx savxra yA adara cihenx (') 
\emng
\eentry

\bentry
\word{grave-clothes}
\pron{gerxVvfkolxVdfs'}
\gl{\nA}
\bmng
 heNada hodike; shavavasatxrX; shavavanunx hULuvAga adakekx hodisuva baTeTxgaLu. 
\emng
\eentry

\bentry
\word{grave-digger}
\pron{gerxVvfDigarf}
\gl{\nA}
\bmng
\bnum
\num{1} goVri toVDuga; samAdhi toVDuvavanu. 
\num{2} kirxmikiVTagaLa deVhagaLu \mo vanunx tamamx mari huLugaLa AhArakAkxgi hUtiDalu guMDi toVDuva kiVTajAti. 
\enum
\emng
\eentry

\bentry
\word{grave-goods}
\pron{gerxVvfguDfs'}
\gl{\nA}
\bmng
 samAdhi vasutxgaLu; goVri padAthaRgaLu; pArxciVna samAdhigaLalilx heNagaLoMdige sikikxda vasutxgaLu. 
\emng
\eentry

\bentry
\word[gravel(1)]{gravel}
\pron{gArxYxva(vf)lf}
\gl{\nA}
\bmng
\bnum
\num{1} (dArigaLanUnx rasetxgaLanUnx mADalu baLasuva) jalilx (kalulx); garasu; kaMkare; gArxYxvalulx. 
\num{2} (\BUvi, gaNi.) jalilxsatxra (\kanmu\ cinanxviruvaMtahudu). 
\num{3} (\roVshA) ashamxriVroVga; kalulxbeVne; mUtarxpiMDagaLalilx matutx mUtarxkoVshadalilx haraLugaLu seVrikoLuLxva roVga. 
\enum
\emng

\noindent
\gl{\pagu}
\bmng
 \eng{pay gravel} lABa sikukxvaSuTx cinanxviruva jalilxsatxra. 
\emng
\eentry

\bentry
\word[gravel(2)]{gravel}
\pron{gArxYxva(vf)lf}
\gl{\sakirx}
\expl{(\BU\ matutx \BUkaq\ \eng{gravelled,} \vakaq\ \eng{gravelling}).}
\bmng
\bnum
\num{1} garasu hAku; jalilx haravu. 
\num{2} tabibxbubx mADu; digaBxrXme hiDisu; EnU toVradaMte mADu. 
\enum
\emng
\eentry

\bentry
\word{gravel-blind}
\pron{gArxYxvalfbelxYnfDx}
\gl{\gu}
\bmng
 hecucx kaDime pUNaR kuruDAda. 
\emng
\eentry

\bentry
\word{graveless}
\pron{gerxVvflisf}
\gl{\gu}
\bmng
 samAdhiyilalxda; goVriyilalxda. 
\emng
\eentry

\bentry
\word{gravelly}
\pron{gArxYxvali}
\gl{\gu}
\bmng
\bnum
\num{1} garasugUDida yA garasiniMdAda. 
\num{2} ashamxriVroVgadaMtha yA ashavxriVroVgadiMda uMTAda. 
\num{3} kakaRshavAda; ahitavAda: \eng{gravelly voice} kakaRsha dhavxni. 
\enum
\emng
\eentry

\bentry
\word{gravely}
\pron{gerxVvfli}
\gl{\kirxvi}
\bmng
 GanavAgi; gaMBiVravAgi; beDagubaNaNx ilalxde; thaLathaLa ilalxde; thaLakupaLakilalxde; nirADaMbaravAgi; saraLavAgi. 
\emng
\eentry

\bentry
\word{graver}
\pron{gerxVvarf}
\gl{\nA}
\bmng
\bnum
\num{1} kaMDariga; utikxVNaRka; ketutxgAra; ketutxvavanu; koreyuvavanu; kaDeyuvavanu. 
\num{2} = \hyperref{kandict_b.pdf}{B}{burin}{burin}. 
\enum
\emng
\eentry

\bentry
\word{Graves}
\pron{gArxvf}
\gl{\nA}
\expl{(\bava\ adeV, \ucAcx\ gArxvfs').}
\bmng
phArxnisxna gArxvf DisiTxrXkfTxnalilx tayArisuva tiVkaSxNXte kaDimeyAda, biLiya (kelavomemx keMpaneya) veYnu. 
\emng
\eentry

\bentry
\wordRemoveSpace{Graves'-disease}{Graves' disease}
\pron{gerxVvfsx DisiZVsfZ}
\gl{\nA}
\bmng
 gerxVvfsx beVne, kAyile; teYrAyfDx garxMthi Udi, nADi baDita hecAcxgi, kaNuNxguDeDxgaLu muMcAcikoLuLxva oMdu bageya gaLagaMDa roVga. 
\emng
\eentry

\bentry
\word{gravestone}
\pron{gerxVvfsoTxVnf}
\gl{\nA}
\bmng
 goVrikalulx; samAdhishile; hesaru \mo vanunx ketitx, goVriya taleya yA kAlina kaDe neDuva kalulx. 
\emng
\eentry

\bentry
\word[Gravettian(1)]{Gravettian}
\pron{garxveTianf}
\gl{\gu}
\bmng
 garxveTiyanf; phArxnisxna lagarxveTf eMbalilx doreta avasheVSagaLiMda patetxyAda, pUvaR shilAyugada yA adakekx saMbaMdhisida. 
\emng
\eentry

\bentry
\word[Gravettian(2)]{Gravettian}
\pron{garxveTianf}
\gl{\nA}
\bmng
\bnum
\num{1} garxveTiyanf; phArxnisxna lagarxveTf eMbalilx doreta avasheVSagaLiMda patetxyAda pUvaRshilAyugada saMsakxqqti. 
\num{2} I saMsakxqqtiya puruSa yA sitxrXV. 
\enum
\emng
\eentry

\bentry
\word[graveward(1)]{graveward}
\pron{gerxVvfvaDfR}
\gl{\kirxvi}
\bmng
 sAvinatatx; maqtuyxloVkadatatx; maqtuyxloVkada kaDege. 
\emng
\eentry

\bentry
\word[graveward(2)]{graveward}
\pron{gerxVvfvaDfR}
\gl{\gu}
\bmng
 maqtuyxloVkadatatx tirugida; AsananxmaraNa. 
\emng
\eentry

\bentry
\word{graveyard}
\pron{gerxVvfyADfR}
\gl{\nA}
\bmng
 masaNa; shamxshAna; olalxkADu; hULugADu. 
\emng
\eentry

\bentry
\word{gravid}
\pron{gArxYxviDf}
\gl{\gu}
\bmng
 (sAhitayxka yA \pArxvi) basirAda; gaBiRNiyAda. 
\emng
\eentry

\bentry
\word{gravimeter}
\pron{garxvimiTarf}
\gl{\nA}
\bmng
 gurutavxmApaka: 
\banum
\alnum{a} darxva yA Gana padAthaRgaLa vishiSaTx gurutavxvanunx nidhaRrisuva oMdu sAdhana. 
\alnum{b} beVre beVre sathxLagaLalilx sithxra darxvayxrAshiyoMdara tUkagaLanunx aLate mADuva mUlaka gurutavx keSxVtarxdalilxruva vayxtAyxsagaLanunx aLatemADuva sAdhana. 
\eanum
\emng
\eentry

\bentry
\word{gravimetric}
\pron{gArxYxvimeTirxkf}
\gl{\gu}
\bmng
 gurutavxmApakada: 
\banum
\alnum{a} tUkavanunx aLate mADuva yA adakekx saMbaMdhisida. 
\alnum{b} gurutavxkeSxVtarxgaLalilxruva vayxtAyxsagaLa yA avakekx saMbaMdhisida. 
\eanum
\emng
\eentry

\bentry
\word{gravimetry}
\pron{garxvimiTirx}
\gl{\nA}
\bmng
 gurutavxmApana; tUkamApana. 
\emng
\eentry

\bentry
\wordf{gravitas}
\pron{gArxYxviTA(TAYx)sf}
\gl{\nA}
\expl{\Latin}
\bmng
Gana gAMBiVyaR; gaMBiVra -- naDate, cayeR, BAva. 
\emng
\eentry

\bentry
\word{gravitate}
\pron{gArxyxviTeVTf}
\gl{\sakirx}
\bmng
 (vajarxkAkxgi ageyuvAga) tUkavAda kalulxgaLu taLakekx hoVguvaMte (kaMkareyanunx) kalaku. 
\emng

\noindent
\gl{\akirx}
\bmng
\bnum
\num{1} gurutAvxkaSaRNeyiMda oMdu kAyada dikikxnalilx sAgu. 
\num{2} biVLu; keLakekx baru; iLi; taLaseVru. 
\num{3} yAvudeV parxBAva keVMdarxkekx yA keVMdarxda kaDege AkaSiRtavAgu, tirugisalapxDu. 
\enum
\emng
\eentry

\bentry
\word{gravitation}
\pron{gArxYxviTeVSanf}
\gl{\nA}
\bmng
\bnum
\numi{1} gurutavx: 
\banum
\alnum{a} BUmiya AkaSaRNeya kAraNa keLagaDe iLiyuvike: \eng{supply of water by gravitation} gurutavxdiMda niVranunx odagisuvudu. 
\alnum{b} vishavxdalilxna yAvudeV eraDu kAyagaLu oMdanonxMdu AkaSiRsuva Bwta kAraNa. 
\eanum
\numie
\num{2} (yAvudeV dikikxnalilx sAguva) parxvaqtitx; olavu. 
\enum
\emng

\noindent
\gl{\pagu}
\bmng
 \eng{law of gravitation} gurutavx niyama; vishavxdalilxna yAvudeV eraDu kAyagaLu tamamx rAshigaLa guNalabadhxkekx anuloVmavAda matutx eraDara naDuvaNa aMtarada vagaRkekx viloVmavAda baladiMda oMdanonxMdu AkaSiRsutatxve eMba nUyxTananxna niyama. 
\emng
\eentry

\bentry
\word{gravitational}
\pron{gArxYxviTeVSanalf}
\gl{\gu}
\bmng
 gurutavxda; gurutavxkekx saMbaMdhisida. 
\emng
\eentry

\bentry
\wordnospeech{gravitational constant}{gravitational constant}
\pron{?}
\gl{\nA}
\bmng
 gurutavx sithxra; nUyxTananxna gurutavx niyamAnusAra eraDu kAyagaLa naDuvaNa AkaSaRNa balakUkx A eraDu kAyagaLa rAshigaLa guNalabadhx hAgU avugaLa naDuvaNa aMtarada vagaRgaLigiruva niSapxtitxgU iruva saMbaMdhavanunx sUcisuva sithxra saMKeyx. 
\emng
\eentry

\bentry
\wordnospeech{gravitational field}{gravitational field}
\pron{?}
\gl{\nA}
\bmng
 gurutavxkeSxVtarx: 
\banum
\alnum{a} yAvudeV kAyada gurutavx kANabaruva parxdeVsha. 
\alnum{b} nidiRSaTx sathxLadalilx gurutavxda tiVvarxte. 
\eanum
\emng
\eentry

\bentry
\word{gravitative}
\pron{gArxYxviTeVTivf}
\gl{\gu}
\bmng
 gurutivxVya; gurutavxda, adara pariNAmadiMda Ada, yA adakekx saMbaMdhisida. 
\emng
\eentry

\bentry
\word{gravity}
\pron{gArxYxviTi}
\gl{\nA}
\bmng
\bnum
\num{1} Ganate; gAMBiVyaR. 
\num{2} pArxdhAnayx; pArxmuKayx; mahatavx. 
\num{3} gurutavx; gurutara sithxti; laGuvAgilalxdiruvike; tiVvarxte: \eng{the gravity of the illness} kAyileya tiVvarxte. \eng{the gravity of his behaviour} avana naDavaLikeya gurutavx. \eng{the gravity of the responsibility} hoNegArikeya gurutara sithxti. 
\num{4} tUka; BAra. 
\numi{5} gurutavx: 
\banum
\alnum{a} yAvudeV kAyavu BUmiya keVMdarxda kaDege AkaSiRtavAguva bala. 
\alnum{b} A balada parxmANa. 
\alnum{c} oMdu kAyavu inonxMdariMda AkaSiRtavAguva tiVvarxte. 
\eanum
\numie
\enum
\emng

\noindent
\gl{\pagu}
\bmng
\hyperdef{G}{gravity pagu}{} \eng{specific gravity} vishiSaTx gurutavx; sApeVkaSx sAMdarxte; yAvudeV gAtarxda padAthaRda tUkakUkx adeV gAtarxda niVrina (anilagaLa \vi vAyuvina) tUkakUkx iruva parxmANa. 
\emng
\eentry

\bentry
\wordnospeech{gravity feed}{gravity feed}
\pron{?}
\gl{\nA}
\bmng
 gurutavx uNike; gurutavxda neraviniMda padAthaRvanunx odagisuvudu. 
\emng
\eentry

\bentry
\wordnospeech{gravity wave}{gravity wave}
\pron{?}
\gl{\nA}
\bmng
 gurutavx taraMga: 
\banum
\alnum{a} gurutavxda parxBAvadiMda darxvada meVlemxYyalilx uMTAguva ale. 
\alnum{b} gurutavx keSxVtarxda EriLitagaLu aleyaMte parxsAravAguva vidayxmAna. 
\eanum
\emng
\eentry

\bentry
\word{gravure}
\pron{garxvuyxarf}
\gl{\nA}
\bmng
 = \hyperref{kandict_p.pdf}{P}{photogravure(1)}{photo gravure} (eMbudara saMkiSxpatx rUpa). 
\emng
\eentry

\bentry
\word{gravy}
\pron{gerxVvi}
\gl{\nA}
\bmng
\bnum
\num{1} mAMsarasa; beVyisuvAgalU anaMtaravU mAMsadiMda osaruva rasagaLu. 
\num{2} mAMsarasadalilx itara sAmagirxgaLanunx hAki mADida aDuge. 
\num{3} diDhiVrf duDuDx; aniriVkiSxtavAgi yA saMpAdisade odagida haNa. 
\enum
\emng
\eentry

\bentry
\wordnospeech{gravy beef}{gravy beef}
\pron{?}
\gl{\nA}
\bmng
 mAMsarasakAkxgi beVyisida danada kAlina BAga. 
\emng
\eentry

\bentry
\word{gravy-boat}
\pron{gerxVviboVTf}
\gl{\nA}
\bmng
 mAMsarasa baDisuva doVNiyAkArada pAterx, donenx. 
\emng
\eentry

\bentry
\wordnospeech{gravy train}{gravy train}
\pron{?}
\gl{\nA}
\bmng
 (\ashi) sulaBavAgi duDuDx sikukxva mUla. 
\emng
\eentry

\bentry
\word[gray(1)]{gray}
\pron{gerxV}
\gl{\nA}
\bmng
 (\Bwvi) gerxV; hiVrikoLaLxlAda vikiraNavanunx aLeyalu baLasuva mAna (= kiloVgArxmfge oMdu jUlf). 
\emng
\eentry

\bentry
\word[gray(2)]{gray}
\pron{gerxV}
\gl{\gu}
\bmng
 (\ame)  = \hyperlink{grey(1)}{$^1$grey}. 
\emng
\eentry

\bentry
\word[gray(3)]{gray}
\pron{gerxV}
\gl{\nA}
\bmng
 (\ame)  = \hyperlink{grey(2)}{$^2$grey}. 
\emng
\eentry

\bentry
\word[gray(4)]{gray}
\pron{gerxV}
\gl{\kirx}
\bmng
 (\ame)  = \hyperlink{grey(3)}{$^3$grey}. 
\emng
\eentry

\bentry
\word{graywacke}
\pron{gerxVvAYxkf}
\gl{\nA}
\bmng
 (\ame)  = \hyperlink{greywacke}{greywacke}. 
\emng
\eentry

\bentry
\word[graze(1)]{graze}
\pron{gerxVsfZ}
\gl{\sakirx}
\bmng
\bnum
\num{1} (beLeyutitxruva hululx \mo vanunx danagaLige, kurigaLige) meVyisu. 
\num{2} (danagaLa \vi\ hulalxnunx) meVyu. 
\num{3} (danagaLanunx) kAvalinalilx meVyisu (\akirx\ saha). 
\num{4} (kAvalinalilx meVyutitxruva) dana kAyu (\akirx saha). 
\num{5} (danagaLanunx) kAvalige aTuTx, biDu; kAvalinalilx meVyabiDu (\akirx\ saha). 
\enum
\emng

\noindent
\gl{\akirx}
\bmng
\bnum
\num{1} (beLeyutitxruva hululx \mo vugaLanunx danagaLu) meVyu. 
\num{2} (danagaLu) kAvalu meVyu. 
\enum
\emng
\eentry

\bentry
\word[graze(2)]{graze}
\pron{gerxVsfZ}
\gl{\sakirx}
\bmng
\bnum
\num{1} hAduhoVguvAga haguravAgi tAku, soVku. 
\num{2} ujijxkoMDu hoVguvAga (camaR \mo vanunx) taracu. 
\enum
\emng

\noindent
\gl{\akirx}
\bmng
\bnum
\num{1} (deVhaBAga) taracihoVgu; taracugAyavAgu. 
\num{2} soVkikoMDu, ujijxkoMDu-hoVgu. 
\enum
\emng
\eentry

\bentry
\word[graze(3)]{graze}
\pron{gerxVsfZ}
\gl{\nA}
\bmng
 ujujxgAya; taracugAya. 
\emng
\eentry

\bentry
\word{grazier}
\pron{gerxVsiZarf, gerxVSaZrf}
\gl{\nA}
\bmng
\bnum
\num{1} pashupAla; danasAkaNegAra; mArATakAkxgi danagaLanunx meVyisuvavanu, sAkuvavanu. 
\num{2} (\AseTxrXV) kuri sAkuvava; kuruba. 
\enum
\emng
\eentry

\bentry
\word{graziery}
\pron{gerxVSaZri}
\gl{\nA}
\bmng
\bnum
\num{1} dana meVyisuva kelasa. 
\num{2} hululxgAvalu. 
\enum
\emng
\eentry

\bentry
\word{grazing}
\pron{gerxVsiZMgf}
\gl{\nA}
\bmng
\bnum
\num{1} meVyisuvudu. 
\num{2} meVyuvudu. 
\enum
\emng
\eentry

\bentry
\word[grease(1)]{grease}
\pron{girxVsf}
\gl{\nA}
\bmng
\bnum
\num{1} (jiMke yA itara beVTeya pArxNiya) kobubx; carabi; meVdasusx. 
\num{2} satatx pArxNigaLa karagisida (\kanmu\ meduvAgiruvAgina) kobubx. 
\num{3} girxVsu; (\kanmu\ cAlana swkayaRkAkxgi hereyAgi, kaMdaneyAgi baLasuva) jiDiDxna yA kobibxna padAthaR; carabi. 
\num{4} uNeNxyalilxna jiDuDx. 
\num{5} jiDuDx toLeyuva uNeNx; jiDuDxtupapxTa. 
\num{6} kudureya himamxDiya oMdu roVga. 
\enum
\emng

\noindent
\gl{\pagu}
\bmng
\hypertarget{grease pagu1}{}. 
\pagu \eng{$(1)$}
\bnum
\num{1} \eng{in grease} (AhArakAkxgi) kolalxlu hadavAda; kobubx tuMbiruva. 
\num{2} \eng{in pride of grease} = \hyperlink{grease pagu1}{?pagu? \((1)\)}. 
\num{3} \eng{in prime of grease} = \hyperlink{grease pagu1}{?pagu? \((1)\)}. 
\num{4} \eng{wool in the grease} uNeNx tuMbida tupapxTa; kuriya meYmeVlina uNeNx. 
\enum
\emng
\eentry

\bentry
\word[grease(2)]{grease}
\pron{girxVsf(sfZ)}
\gl{\sakirx}
\bmng
\bnum
\num{1} jiDaDxnunx -- hacucx, baLi, savaru, leVpisu; girxVsf hAku. 
\num{2} (bANale \mo vakekx) kobubx baLi. 
\num{3} jiDiDxniMda koLemADu; jiDuDx jiDuDx mADu. 
\num{4} jiDiDxniMda jAruvaMte mADu: \eng{grease the wheels} cakarxgaLige jiDuDxhacicx jAruvaMte mADu. 
\num{5} kudurege himamxDi roVga -- barisu, baruvaMte mADu. 
\enum
\emng

\noindent
\gl{\nuga}
\bmng
\bnum
\num{1} \eng{grease palm of} keY becacxge mADu; laMca koDu. 
\num{2} \eng{grease the wheels} (\rUpa) (\kanmu\ laMcakoTuTx) kelasa saliVsAgi naDeyuvaMte mADiko. 
\num{3} \eng{like greased lightning} atayxMta veVgavAgi; miMcige kaMdane hAkidaMte. 
\enum
\emng
\eentry

\bentry
\word{grease-box}
\pron{girxVsfbAkfsx}
\gl{\nA}
\bmng
 (reYlina cakarxkekx here odagisalu tagulisiruva) carabi peTiTxge; girxVsu peTiTxge. 
\emng
\eentry

\bentry
\word{grease-gun}
\pron{girxVsfganf}
\gl{\nA}
\bmng
 girxVsfganunx; yaMtarxgaLa beVriMgu \mo vugaLige girxVsanunx hAkalu baLasuva saNaNx paMpu. 
\emng
\eentry

\bentry
\wordnospeech{grease monkey}{grease monkey}
\pron{?}
\gl{\nA}
\bmng
 (\ashi) yaMtarxkamiR; mekAYxnikukx; yaMtarxda kelasa mADuvava. 
\emng
\eentry

\bentry
\word{grease-paint}
\pron{girxVsfpeVMTf}
\gl{\nA}
\bmng
 girxVsfbaNaNx; naTanaTiyara muKagaLige baLiyalu baLasuva oMdu bageya baNaNx. 
\emng
\eentry

\bentry
\word{greaser}
\pron{girxVsarf}
\gl{\nA}
\bmng
\bnum
\num{1} girxVsu leVpaka; jiDuDx hacucxga; yaMtarxBAgagaLige jiDuDx hacucxvavanu, baLiyuvavanu. 
\num{2} (\kanmu\ ja{ha}jinalilx, eMjinina) beMki, uri noVDikoLuLxvavanu. 
\num{3} (\ashi) (\ame) mekisxko, deVshadavanu yA sApxyXniSf amerikanf. 
\num{4} haDaginalilxna eMjiniyaru. 
\num{5} girxVsuleVpaka; moVTAru vAhanagaLa BAgagaLige girxVsanunx hacacxlu baLasuva sAdhana. 
\num{6} (\ashi) AkeSxVpAhaR vayxkitx; \kanmu\ beNeNx hacucxvava, hogaLu BaTaTx. 
\num{7} (\ashi) moVTAru seYkalulxgaLalilx hoVguva taMDadavaralilx obabx. 
\enum
\emng
\eentry

\bentry
\word{grease-trap}
\pron{girxVsfTArxYxpf}
\gl{\nA}
\bmng
 girxVsuroVdhaka; carabitaDe; moVrigaLalilx haridu baruva girxVsanunx, jiDaDxnunx taDehiDidu nililxsuva sAdhana. 
\emng
\eentry

\bentry
\word{greasily}
\pron{girxVsili}
\gl{\kirxvi}
\bmng
\bnum
\num{1} jiDuDxjiDADxgi. 
\num{2} (\rUpa) beNeNx beNeNxyAgi; ati vinayadiMda; kaqtaka nayadiMda; vinayaveVSadiMda. 
\enum
\emng
\eentry

\bentry
\word{greasiness}
\pron{girxVsinisf}
\gl{\nA}
\bmng
\bnum
\num{1} jiDuDxtana; jiDuDx jiDADxgiruvike. 
\num{2} (\rUpa) beNeNx beNeNxyAgi vatiRsuvudu; ati vinaya; kaqtaka naya; vinayADaMbara; vinayaveVSa. 
\enum
\emng
\eentry

\bentry
\word{greasy}
\pron{girxVsi}
\gl{\gu}
\bmng
\bnum
\num{1} jiDuDxbaLida; girxVsu savarida. 
\num{2} jiDuDxLaLx. 
\num{3} jiDiDxniMdAda. 
\num{4} jiDiDxnaMtha. 
\num{5} ati jiDiDxna; jiDuDxjiDADxda. 
\num{6} (tupapxTada \vi) jiDuDx toLeyada. 
\num{7} (kudureya \vi) himamxDi roVga tagulida. 
\num{8} (kesariniMdaloV, teVvadiMdaloV) jArutitxruva; jArikeya. 
\num{9} (naDavaLikeya yA muKaBAvada \vi) beNeNxbeNeNxyAgi vatiRsuva; ati vinayada; kaqtaka nayada; vinayaveVSada. 
\enum
\emng
\eentry

\bentry
\wordnospeech{greasy fritillary}{greasy fritillary}
\pron{?}
\gl{\nA}
\bmng
 oMdu jAtiya ciTeTx. 
\emng
\eentry

\bentry
\wordnospeech{greasy pole}{greasy pole}
\pron{?}
\gl{\nA}
\bmng
 jiDuDxgaMba; eNeNxgaMba; kirxVDegaLalilx hatutxvudakAkxgi yA naDeyuvudakAkxgi jiDuDx baLidiruva kaMba. 
\emng
\eentry

\bentry
\wordnospeech{greasy spoon}{greasy spoon}
\pron{?}
\gl{\nA}
\bmng
 (\ashi) kaLape hoVTelu yA agagxda hoVTelu; agagxvAda matutx kaLapeyAda hoVTelu. 
\emng
\eentry

\bentry
\word[great(1)]{great}
\pron{gerxVTf}
\gl{\gu}
\bmng
\bnum
\numi{1} (\sA\ AshacxyaR, mecicxke, tirasAkxra, sadAgarxha, \mo vanunx sUcisuva) doDaDx; BAri; bahu; vAyxpaka; mahA: 
\banum
\alnum{a} \eng{made a great blot} doDaDx citutx, kale mADibiTaTx. 
\alnum{b} (ADumAtinalilx inonxMdu \gu da hiMde) \eng{a great big stick} doDaDx, BAri doNeNx. 
\alnum{c} (jAtigaLalilx yA vayxkitxgaLalilx hecucx doDaDxdara visheVSaNavAgi) doDaDx: \eng{great A, Z} doDaDx \eng{A, Z} (akaSxragaLu). 
\eanum
\numie
\num{2} asAdhAraNa; hecicxna; bahaLa; visheVSa: \eng{take great care} visheVSa ecacxra tALu.: \eng{of great popularity} asAdhAraNa janapirxyateya. 
\num{3} muKayx; parxmuKa; parxdhAna: \eng{a great necessity} muKayx Avashayxkate. 
\num{4} unanxta; ucacx; udAtatx; sherxVSaThx. 
\num{5} mAnayx; gwravAnivxta; saMBAvita. 
\num{6} mahatavxda; pavaR: \eng{a great occasion} mahatavxda saMdaBaR. \eng{a great moment} pavaRkAla. 
\num{7} agarx; agarxgaNayx: \eng{the Great Powers} agarxrASaTxrXgaLu; parxmuKa rASaTxrXgaLu. 
\num{8} (A hesarinavaralilx aitihAsikavAgi parxmuKa eMbathaRdalilx) ati viKAyxta: \eng{Alexander the Great} alekAsxMDarf mahAshaya. 
\num{9} (birudugaLalilx visheVSaNavAgi) sherxVSaThx: \eng{the Great Mogul} mogalf sherxVSaThx. 
\num{10} sherxVSaThx; asAdhAraNa sAmathayxR, parxtiBe, bwdidhxka ilalxve vAyxvahArika guNAtishayagaLu, shiVlada aunanxtayx yA pavitarxte ivu uLaLx: \eng{a great judge} satayxsaMdha yA nAyxyaniSaThx nAyxyAdhipati. \eng{a great painter} parxtiBAnivxta citarxkAra. \eng{the truly great man} nijavAgiyU mahAnf vayxkitx, loVkoVtatxra puruSa. \eng{great thoughts} unanxta ciMtanagaLu. 
\numi{11} (AKAyxtakavAgi) 
\banum
\alnum{a} (oMdu kelasa, keSxVtarxdalilx) ati kushala, catura. 
\alnum{b} (oMdu viSayadalilx) doDaDx paMDita; mahAvidAvxMsa. 
\eanum
\numie
\num{12} bahu samapaRka, taqpitxkara; utatxma: \eng{wouldn't it be great if....? ....} Agidadxre bahu samapaRkavAgutitxtatxlalxve? 
\num{13} (hesarisidaMtha) hesaru salulxva; hesarige takakx; anavxthaRka. 
\num{14} (kataqRboVdhaka \nA gaLoDane) kAyaRvanunx, kaqtayxvanunx -- ati doDaDx parxmANadalilx mADuva; shudadhx; pakAkx; mahAnf: \eng{a great scoundrel} pakAkx badAmxSf; shudadhx niVca. \eng{a great dancer} mahAnf nataRka. 
\num{15} (\eng{uncle, aunt, nephew, niece} eMbuvugaLige yA \eng{grand} oDane kUDida naMTatana heVLuva padagaLige omemxyoV hecucx bAriyoV pUvaRpadavAgi) oMdu tale hiMdina yA muMdina: \eng{great grandfather} mutatxjajx. 
\num{16} (AshacxyaRsUcaka udAgxravAgi) \eng{Great god!} ayoyxV deVvareV! BagavaMta! paramAtamx! 
\enum
\emng

\noindent
\gl{\pagu}
\bmng
\bnum
\num{1} \eng{a great deal} bahaLa; tuMba; bahu; hecucx parxmANa. 
\num{2} \eng{a great many} BAri saMKeyxya; aneVkAneVka. 
\num{3} \eng{a great while ago} bahukAlada hiMde. 
\num{4} \eng{Great Caesar!} ayoyxV deVvareV! 
\num{5} \eng{Great} \hyperref{kandict_c.pdf}{C}{charter(1) pagu(2)}{$^1$charter}.
\num{6} \eng{Great} \hyperref{kandict_c.pdf}{C}{circle(1) pagu(1)}{$^1$circle}. 
\num{7} \eng{Great} \hyperref{kandict_d.pdf}{D}{dane pagu}{dane}. 
\num{8} \eng{Great} \hyperref{kandict_d.pdf}{D}{deal(1) pagu(3)}{$^1$deal}. 
\num{9} \eng{Great} \hyperref{kandict_d.pdf}{D}{divide(2) nuga}{$^2$divide}. 
\num{10} \eng{Greater Britain} gerxVTf birxTanf matutx adara vasAhatugaLu (vivaraNAtamxka hesaru, adhikaqta hesaralalx). 
\numi{11} \eng{greatest common measure} 
\banum
\alnum{a} mahatatxma sAmAnayx apavataRna. 
\alnum{b} atayxdhika sAmAnAyxMsha. 
\eanum
\numie
\num{12} \eng{great go} keVMbirxjfna bi.e. padaviya aMtima pariVkeSx. 
\num{13} \eng{Great} \hyperref{kandict_s.pdf}{S}{seal(3) pagu(6)}{$^3$seal}. 
\num{14} \eng{live to a great age} bahuvaSaR baduku. 
\num{15} \eng{the great Commoner} hiriya viliyamf piTf [sapatxvASiRka yudadhxda kAladalilx \eng{(1757--1763)} birxTaninxna parxdhAniyAgidadxvanu]. 
\num{16} \eng{the greatest happiness of the greatest number} beMtaM eMba tatatxvXjacnxna parxmuKa tatatxvX (noVDi \eng{Benthamism}). 
\num{17} \eng{the great majority} adhikAMsha; bahupAlu. 
\num{18} \eng{the great unpaid} saMbaLavilalxda nAyxyAdhipatigaLu. 
\num{19} \eng{the great unwashed} meYtoLeyadavaru; doMbi jana; kiVLu jana. 
\enum
\emng

\noindent
\gl{\nuga}
\bmng
\bnum
\num{1} \eng{great with child} (\pArxparx) basirAda; tuMbu gaBiRNiyAda. 
\num{2} \eng{the great world} unanxta samAja; samAjada unanxta -- vaqtatx, valaya, vagaR. 
\enum
\emng
\eentry

\bentry
\word[great(2)]{great}
\pron{gerxVTf}
\gl{\nA}
\bmng
\bnum
\num{1} (\gaparx) mahApuruSa; sherxVSaThx vayxkitx. 
\num{2} doDaDxdAdadudx; sherxVSaThxvAdadudx; mahatAtxdadudx. 
\num{3} (\bava dalilx \eng{Greats}) AkfsxphaDiRna bi.e. aMtima pariVkeSx \kanmu\ kAlxsikalf sAhitayx matutx tatatxvXshAsatxrXgaLa Anarfsx pariVkeSx. 
\enum
\emng

\noindent
\gl{\pagu}
\bmng
\bnum
\num{1} \eng{great and small} hiriyarU kiriyarU yA doDaDxvU cikakxvU. 
\num{2} \eng{the great} (\bava) mahAvayxkitxgaLu; mahApuruSaru; mahaniVyaru; mahAmahimaru. 
\num{3} \eng{the greatest} (\ashi) asAdhAraNa vayxkitx. 
\enum
\emng
\eentry

\bentry
\wordnospeech{Great Assize}{Great Assize}
\pron{?}
\gl{\nA}
\bmng
 (kerxYsatxralilx) (deVvaru mADuva, aMtima) mahA vicAraNe(yA dina). 
\emng
\eentry

\bentry
\wordnospeech{Great Bear}{Great Bear}
\pron{?}
\gl{\nA}
\bmng
 mahABalUlxka eMba hesarina doDaDx nakaSxtarxmaMDala; sapatxSiRmaMDala. 
\emng
\eentry

\bentry
\wordnospeech{Great Bible}{Great Bible}
\pron{?}
\gl{\nA}
\bmng
 gerxVTf beYbalf; mahA beYbalulx; kavarfDeVlf eMbAta iMgilxSfge BASAMtarisida beYbalulx \eng{(1539)}. 
\emng
\eentry

\bentry
\wordnospeech{Great Britain}{Great Britain}
\pron{?}
\gl{\nA}
\bmng
 gerxVTf birxTanf; iMgelxMDf, sAkxTfleMDf, veVlfsxgaLu seVri Ada rAjayx, deVsha. 
\emng
\eentry

\bentry
\word{greatcoat}
\pron{gerxVTfkoVTf}
\gl{\nA}
\bmng
 BAri niluvaMgi; doDaDx kapani. 
\emng
\eentry

\bentry
\word{greaten}
\pron{gerxVTfnf}
\gl{\kirx}
\bmng
 (\pArxparx) \sakirx. 
\bnum
\num{1} gAtarxdalilx yA aLateyalilx -- doDaDxdu mADu, hecicxsu. 
\num{2} unanxtagoLisu; parxmuKagoLisu; meVlemxgeVrisu; hecacxLavuMTumADu; hirimegoLisu. 
\enum
\emng

\noindent
\gl{\akirx}
\bmng
 hirimegoLuLx; meVlemxgoLuLx; unanxtanAgu; udAtatxnAgu. 
\emng
\eentry

\bentry
\wordnospeech{great game}{great game}
\pron{?}
\gl{\nA}
\bmng
\bnum
\num{1} gAlfphx (ATa). 
\num{2} beVhugArike. 
\enum
\emng
\eentry

\bentry
\wordnospeech{great gross}{great gross}
\pron{?}
\gl{\nA}
\bmng
 doDaDx gorxVsu; hanenxraDu gorxVsu; \eng{144} Dajanunx. 
\emng
\eentry

\bentry
\word{great-hearted}
\pron{gerxVTfhATiRDf}
\gl{\gu}
\bmng
\bnum
\num{1} kececxdeya; dhiVra. 
\num{2} doDaDx manasisxna; udAra savxBAvada; vishAlahaqdayada. 
\enum
\emng
\eentry

\bentry
\wordnospeech{great house}{great house}
\pron{?}
\gl{\nA}
\bmng
 doDaDx mane; haLiLx \mo vugaLa, muKayxvAda mane. 
\emng
\eentry

\bentry
\wordnospeech{Great Inquest}{Great Inquest}
\pron{?}
\gl{\nA}
\bmng
  = \hyperlink{Great Assize}{Great Assize}. 
\emng
\eentry

\bentry
\word{greatly}
\pron{gerxVTfli}
\gl{\kirxvi}
\bmng
\bnum
\num{1} tuMba; bahaLa; bahu. 
\num{2} bahumaTiTxge; bahaLa maTiTxge: \eng{greatly esteemed} bahaLa gwravakekx pAtarxnAda. \eng{greatly superior} bahumeVlAda. \eng{I should greatly prefer} nanage hecucx iSaTx. 
\num{3} GanavAgi; udAtatxvAgi; unanxta riVtiyalilx; udAravAgi. 
\enum
\emng
\eentry

\bentry
\word{greatness}
\pron{gerxVTfnisf}
\gl{\nA}
\bmng
\bnum
\num{1} baqhatavx; (gAtarx, visAtxra, parxmANadalilx) doDaDxdAgiruvudu. 
\num{2} hirime; meVlemx; aunanxtayx; Ganate; gAMBiVyaR; pArxmuKayx. 
\num{3} (sahajavAda) udArate; udAtatx guNa. 
\enum
\emng
\eentry

\bentry
\wordnospeech{great organ}{great organ}
\pron{?}
\gl{\nA}
\bmng
 mahA AgaRnf (vAdayx); eraDu yA hecucx kiVlikeY maNigaLiruva AgaRnf vAdayxdalilx parxdhAnavAda kiVlikeYmaNe. 
\emng
\eentry

\bentry
\wordnospeech{Great Power}{Great Power}
\pron{?}
\gl{\nA}
\bmng
 mahA rASaTxrX; seYnika shakitx yA rAjakiVya parxBAva matutx BAri aMtara rASiTxrXVya pArxmuKayxvuLaLx rASaTxrX. 
\emng
\eentry

\bentry
\wordnospeech{Great Russian}{Great Russian}
\pron{?}
\gl{\nA}
\bmng
\bnum
\num{1} soVviyatf raSAyxda muKayxvAda janAMgadavanu, \kanmu\ soVviyatf okUkxTada utatxra yA madhayx BAgagaLavanu(Lu). 
\num{2} (yUkerxVniyanf matutx beYloraSayxnfgaLanunx biTuTx uLida) raSayxnf BASe. 
\enum
\emng
\eentry

\bentry
\wordnospeech{Great War}{Great War}
\pron{?}
\gl{\nA}
\bmng
 (modalaneya) mahAyudadhx \eng{(1914--18)}. 
\emng
\eentry

\bentry
\wordnospeech{Great White Way}{Great White Way}
\pron{?}
\gl{\nA}
\bmng
 gerxVTf veYTf veV; mahAshevxVtapatha; citarxmAdarigaLu hecAcxgiruva, nUyxyAkfR nagarada, rAtirxya hotutx JagaJagisuva diVpagaLiMda alaMkaqtavAgiruva rasetx. 
\emng
\eentry

\bentry
\word{greave}
\pron{girxVvf}
\gl{\nA}
\bmng
 (\sA\ \bava dalilx) kaNakAlina kApu; jaMGAkavaca. 
\emng
\eentry

\bentry
\word{greaves}
\pron{girxVvfsfZ}
\gl{\nA}
\bmng
 (\bava) (nAyi \mo vugaLa AhArakAkxgi yA miVnina ereyAgi baLasuva) carabiya, nArunArAda kaLape; carabiya kaLape. 
\emng
\eentry

\bentry
\word{grebe}
\pron{girxVbf}
\gl{\nA}
\bmng
\bnum
\num{1} girxVbf; giDaDx meY, joVlu camaRda kAlebxraLuLaLx, moVTu bAlagaLuLaLx, muLugu hakikxya jAtigaLu. 
\num{2} (alaMkArakAkxgi baLasuva) iMtha hakikxgaLa garigaLu. 
\enum
\emng
\eentry

\bentry
\word[Grecian(1)]{Grecian}
\pron{girxVSanf}
\gl{\gu}
\bmng
 girxVsina; girxVkf (\vAshi da matutx muKada AkArada \vi horatu Iga \viparx). 
\emng

\noindent
\gl{\pagu}
\bmng
\bnum
\num{1} \eng{Grecian bend} (\birx) \eng{1870}ra sumArinalilx parxcalitavAgidadx naDageya kaqtaka BaMgi; girxVkf gatutx. 
\num{2} \eng{Grecian gift} = \hyperlink{Greek gift}{Greek gift}. 
\num{3} \eng{Grecian knot} girxVkf jaDegaMTu; girxVkf sheYliya heMgasara turubu, gaMTu. 
\num{4} \eng{Grecian nose} girxVkf mUgu; naDuve ububx tagigxlalxde haNeyiMda mATavAgi iLiyuva mUgu. 
\num{5} \eng{Grecian profile} girxVkf mUgina mATavanunx etitx toVrisuva girxVkf muKada pAshavxRnoVTa yA adara citarx. 
\num{6} \eng{grecian slippers} (\birx) girxVkf joVDu (pwrasatxyXru baLasuvaMtha joVDige vAyxpArada hesaru). 
\enum
\emng
\eentry

\bentry
\word[Grecian(2)]{Grecian}
\pron{girxVSanf}
\gl{\nA}
\bmng
\bnum
\num{1} girxVkf paMDita; girxVkf BASAparxviVNa. 
\num{2} (\birx) `kerxYsfTxsX hAsipxTalf' shAleyalilx atayxMta meVlatxragatiya vidAyxthiR. 
\enum
\emng
\eentry

\bentry
\word{Grecise}
\pron{girxVseYsfZ}
\gl{\kirx}
\bmng
  = \hyperlink{Graecize}{Graecize}. 
\emng
\eentry

\bentry
\word{Grecism}
\pron{girxVsisaZmf}
\gl{\nA}
\bmng
  = \hyperlink{Graecism}{Graecism}. 
\emng
\eentry

\bentry
\word{Grecize}
\pron{girxVseYsfZ}
\gl{\kirx}
\bmng
  = \hyperlink{Graecize}{Graecize}. 
\emng
\eentry

\bentry
\word{Greco-}
\pron{girxVkoV-}
\gl{\sapUpa}
\bmng
  = \hyperlink{Graeco-}{Graeco-}. 
\emng
\eentry

\bentry
\word{greed}
\pron{girxVDf}
\gl{\nA}
\bmng
\bnum
\num{1} (\kanmu\ aishavxyaRkAkxgi yA AhArakAkxgi) taNivilalxda Ashe; taqpitxyAgada Ase. 
\num{2} atAyxshe; durAshe; loVBa. 
\enum
\emng
\eentry

\bentry
\word{greedily}
\pron{girxVDili}
\gl{\kirxvi}
\bmng
\bnum
\num{1} ati hasiviniMda. 
\num{2} hoTeTxbAkatanadiMda. 
\num{3} atAyxsheyiMda; lABada durAsheyiMda; atiloVBadiMda; elalxvanUnx doVcikoLuLxva savxBAvadiMda. 
\num{4} (mADalu) AturadiMda; utAsxhadiMda; tiVvArxBilASeyiMda. 
\enum
\emng
\eentry

\bentry
\word{greediness}
\pron{girxVDinisf}
\gl{\nA}
\bmng
\bnum
\num{1} AhAra pAniVyagaLigAgi ati Ashe, Atura. 
\num{2} tinunxvudu, kuDiyuvudaralilx ati Atura. 
\num{3} hoTeTxbAkatana; tiMDipoVtatana. 
\num{4} aishavxyaRkAkxgi ati Ashe; dhanaloVBa; durAshe; elalxvanUnx doVcikoLuLxva savxBAva. 
\num{5} (yAvudeV \vi) ati Atura, utAsxha; tiVvArxBilASe. 
\enum
\emng
\eentry

\bentry
\word{greedy}
\pron{girxVDi}
\gl{\gu}
\bmng
\bnum
\num{1} ati hasivina. 
\num{2} gabagabane tinunxva; AturAturavAgi nuMguva. 
\num{3} hoTeTxbAkatanada; tiMDipoVtatanada. 
\num{4} atAyxsheya; durAsheya; elalxvanUnx doVcikoLuLxva. 
\num{5} (mADalu) AturavuLaLx; utAsxhavuLaLx; tiVvArxBilASeyuLaLx. 
\enum
\emng
\eentry

\bentry
\word{greedy-guts}
\pron{girxVDigaTfsx}
\gl{\nA}
\bmng
 (\asaM) hoTeTxbAka; tiVnALi; bakAsura. 
\emng
\eentry

\bentry
\word{greegree}
\pron{girxVgirxV}
\gl{\nA}
\bmng
 Aphirxkada -- yaMtarx, tAyita, rakeSx, taDe. 
\emng
\eentry

\bentry
\word[Greek(1)]{Greek}
\pron{girxVkf}
\gl{\nA}
\bmng
\bnum
\num{1} girxVkf; girxVsf deVshakekx yA girxVsf buDakaTiTxge seVridavanu. 
\num{2} (\kerxY) girxVkf caciRna sadasayx; girxVkf caciRge seVridavanu. 
\num{3} (\pArxparx) girxVkf caciRna sadasayx. 
\num{4} (\pArxparx) kuyukitxgAra; kutaMtirx; moVsagAra; Thakakx. 
\num{5} girxVkf BASe. 
\num{6} (BASe, baravaNige, \mo vugaLa \vi) athaRvAgada viSaya: \eng{Greek to me} nananx pAlige girxVkf; nananx arivige miVridudx; nanageVnU tiLiyadudx. 
\enum
\emng
\eentry

\bentry
\word[Greek(2)]{Greek}
\pron{girxVkf}
\gl{\gu}
\bmng
\bnum
\num{1} girxVsina. 
\num{2} girxVsina janara; girxVkara. 
\num{3} helenikf saMsakxqqtiya. 
\num{4} girxVkf BASeya. 
\num{5} girxVkf BASeganuguNavAda. 
\num{6} girxVkf BASeyalilx bareda yA heVLida. 
\enum
\emng

\noindent
\gl{\pagu}
\bmng
\bnum
\num{1} \eng{Greek Fathers} girxVkf BASeyalilx bareda kerxYsatxmata garxMthakAraru. 
\num{2} \eng{Greek} \hyperref{kandict_f.pdf}{F}{fire(1) pagu(3)}{$^1$fire}. 
\enum
\emng
\eentry

\bentry
\wordnospeech{Greek Church}{Greek Church}
\pron{?}
\gl{\nA}
\bmng
 girxVkf cacuR; kAnfsATxyXMTinoVpalfna hiriya pAdirxya adhikAravanonxpipxkoMDu, \kanmu\ girxVsina, raSayxda matutx tukiR cakArxdhipatayxda kerxYsatxranonxLagoMDa cacuR. 
\emng
\eentry

\bentry
\wordnospeech{Greek cross}{Greek cross}
\pron{?}
\gl{\nA}
\bmng
 girxVkf shilube; aDADxhAyuva paTiTxgaLa udadxvu samAnavAgiruva shilube (citarxkekx \hyperref{kandict_c.pdf}{C}{cross(1)}{$^1$cross} noVDi). 
\emng
\eentry

\bentry
\wordnospeech{Greek gift}{Greek gift}
\pron{?}
\gl{\nA}
\bmng
 girxVkf dAna; keVDanunx bayasi koTaTx dAna; dudARna; aniSaTxdAna. 
\emng
\eentry

\bentry
\word{Greekless}
\pron{girxVkflisf}
\gl{\gu}
\bmng
 girxVkf tiLiyada; girxVkf BASeya jAcnxna ilalxda. 
\emng
\eentry

\bentry
\word[green(1)]{green}
\pron{girxVnf}
\gl{\gu}
\bmng
\bnum
\num{1} hasuru; hasirAda; hacacxneya; hasiya hululx, kaDala niVru, pacecx \mo vugaLa baNaNxda. 
\num{2} sopupx beLediruva. 
\num{3} ele tuMbiruva. 
\num{4} (meYbaNaNx, muKa CAyeya \vi) biLicikoMDa; nisetxVja; pAMDura; roVgada CAyeya. 
\num{5} (\rUpa) karubina; asUyeya; mAtasxyaRda. 
\num{6} kAyipalayxda; sopipxna. 
\num{7} mAgilalxda; haNANxgilalxda; kAyAda. 
\num{8} eLasAda; hiVcAda; piVcAda. 
\num{9} (hasirugUDi) baliyutitxruva; beLeyutitxruva. 
\num{10} ceYtanayx tuMbiruva; naLanaLisutitxruva. 
\num{11} bADilalxda; kaMdirada. 
\num{12} eLeya; apakavx; balitilalxda; beLavaNige ilalxda. 
\num{13} ananuBavi; anuBavavilalxda; ati mugadhx; sulaBavAgi moVsa hoVguva: \eng{was not so green as to expect a suspicious man look suspicious} saMshayAsapxda vayxkitx saMshayAsapxdavAgiyeV kANutAtxneMdu niriVkiSxsuvaSuTx avanu mugadhxnAgiralilalx. 
\num{14} hasiya; oNagirada. 
\num{15} pakavx mADirada; hadagoLisirada. 
\num{16} (inUnx) mAyadiruva; hasi: \eng{a green wound} hasi gAya. 
\enum
\emng

\noindent
\gl{\pagu}
\bmng
\bnum
\num{1} \eng{a green Christmas} (yA \eng{Winter} yA \eng{Yule}) hitavAda (himavilalxda) kirxsfmasf kAla. 
\num{2} \eng{a green old age} tiVra haNANxgada, utAsxhadiMda kUDida mupupx. 
\num{3} \eng{a green season} maMjilalxda, hitavAda kAla. 
\enum
\emng

\noindent
\gl{\nuga}
\bmng
\hypertarget{green(1)nuga}{} 
\bnum
\num{1} \eng{green fingers} toVTagArikeyalilx kwshala. 
\num{2} \eng{in the green tree} (\beY) oLeLxya deseyalilx; swKayx sithxtiyalilx; anukUla saninxveVshadalilx. 
\enum
\emng
\eentry

\bentry
\word[green(2)]{green}
\pron{girxVnf}
\gl{\nA}
\bmng
\bnum
\num{1} hasurAgiruvudu. 
\num{2} hasiru BAga. 
\num{3} hasuru (baNaNx). 
\num{4} hasuru baTeTx yA uDuge: \eng{dressed in green} hasiru baTeTx uTaTx. 
\num{5} (\rUpa) (naMbi moVsa hoVguva) maMkutanada CAye; maDiDxtanada cihenx; gAMpatanada gurutu: \eng{do you see any green in my eye?} ninage nananx kaNiNxnalilx gAMpatanada guruteVnAdarU kANisutitxdeyeV? nAneVnu gugugxveMdu eNisideyoV? 
\num{6} (\sA\ visheVSaNadoDane) hasuru baNaNx, raMgu: \eng{mineral green} Kanija hasuru. 
\num{7} (ywvanada) hurupu; shakitx: \eng{in the green} navaywvanadalilx; tAruNayxda Baradalilx. 
\num{8}  = \hyperlink{greenery}{greenery}. 
\num{9} (\bava dalilx) (beVyisuva muMcina yA beVyisida) kAyipalayxgaLu; tarakAri (sopupx \mo vu). 
\num{10} sAvaRjanika hululxmeYdAna, hululxmALa: \eng{village green} haLiLxya hululx meYdAna. 
\num{11} (visheVSa) udedxVshakAkxgi baLasuva hululx nela (\kanmu\ visheVSaNagaLoMdige): \eng{putting-green, bowling green} itAyxdi. 
\num{12} (gAlfphx) kuLiyanunx sututxvarediruva, tuMDAgi katatxrisida hulilxna parxdeVsha, AvaraNa. 
\num{13} (gAlfphx) = \hyperref{kandict_f.pdf}{F}{fairway(3)}{fairway (3)}. 
\num{14} (\ashi) haNa; rokakx; duDuDx. 
\num{15} kiVLadxjeRya mAYxrihAvxna. 
\num{16} (\bava dalilx) saMBoVga meYthuna. 
\num{17} (sUnxkarf \mo\ ATagaLalilxna) hasiru ceMDu. 
\num{18} aileRMDanunx saMkeVtisuva (hasiru) baNaNx. 
\num{19}  = \hyperlink{green light}{green light}. 
\enum
\emng

\noindent
\gl{\pagu}
\bmng
 \eng{through the green} (gAlfphx ATadalilx) pArxraMBika hoDetada jAgakUkx guLiya sutatxNa hulilxna AvaraNakUkx naDuvaNa parxdeVsha. 
\emng
\eentry

\bentry
\word[green(3)]{green}
\pron{girxVnf}
\gl{\sakirx}
\bmng
\bnum
\num{1} hasuru baNaNx hAku, baLi. 
\num{2} hasuru kale (uMTu)mADu. 
\num{3} (\ashi) moVsagoLisu; ToVpi hAku; vaMcisu: \eng{attempts were made to green me} nanage ToVpi hAkuva parxyatanxgaLu naDeduvu. 
\enum
\emng

\noindent
\gl{\akirx}
\bmng
\bnum
\num{1} (\kanmu\ payirupacecxgaLiMda) hasureVru; hacacxgAgu. 
\num{2} hasuru kaleyAgu. 
\enum
\emng
\eentry

\bentry
\word{greenback}
\pron{girxVnfbAyxkf}
\gl{\nA}
\bmng
 (\ame) 
\bnum
\num{1} (amerikada saMyukatx saMsAthxnada kAyide parxkAra calAvaNeyalilxruva) noVTu; amerikada saMyukatx saMsAthxnada yA rASiTxrXVya bAyxMkinavaru horaDisiruva noVTu. 
\num{2} hasiru beninxga; hasiru beninxna yAvudeV pArxNi (\udA\ miVnu). 
\enum
\emng
\eentry

\bentry
\wordnospeech{green belt}{green belt}
\pron{?}
\gl{\nA}
\bmng
 hasuru valaya; Ura sutatxlina udAyxnagaLu, terapiTaTx hasuru bayalu, \mo vu. 
\emng
\eentry

\bentry
\wordnospeech{Green Beret}{Green Beret}
\pron{?}
\gl{\nA}
\bmng
 (\AmA) hasiru ToVpi; birxTiSf yA amerikada kiSxparxdALi paDeya yoVdha. 
\emng
\eentry

\bentry
\word{green-blind}
\pron{girxVnfbelxYnfDx}
\gl{\gu}
\bmng
 hasuruguruDu; hasuru baNaNxgaLanunx garxhisalAgada daqSiTx paTalavuLaLx. 
\emng
\eentry

\bentry
\wordnospeech{green card}{green card}
\pron{?}
\gl{\nA}
\bmng
 hasiru kADuR; tamamx moVTAru kArugaLanunx horadeVshakekx tegedukoMDu hoVgalu cAlakaru iTuTxkoLaLxbeVkAda aMtararASiTxrXVya vimA patarx. 
\emng
\eentry

\bentry
\wordnospeech{green cheese}{green cheese}
\pron{?}
\gl{\nA}
\bmng
 hasiru ciVsu: 
\banum
\alnum{a} mosariniMda mADida ciVsu, giNuNx. 
\alnum{b} kapURrada eleya kaSAya berasidadxriMda hasirubaNaNx tirugida ciVsu. 
\alnum{c} hadakekx barada, apakavx ciVsu, giNuNx. 
\eanum
\emng
\eentry

\bentry
\wordnospeech{Green Cloth}{Green Cloth}
\pron{?}
\gl{\nA}
\bmng
(birxTiSf) (aramaneya gaqhakaqtayxda bakiSxyAda) lADfR siTxVvaDaRna viBAga. 
\emng

\noindent
\gl{\pagu}
\bmng
 \eng{Board of Green Cloth} = \hyperlink{Green Cloth}{Green Cloth}. 
\emng
\eentry

\bentry
\wordnospeech{green crop}{green crop}
\pron{?}
\gl{\nA}
\bmng
 hasiru meVvu; hasiru payiru; oNagisade, hasirAgiruvAgaleV AhAravAgi baLasuva hasihululx, sopupxsede, \mo vu. 
\emng
\eentry

\bentry
\wordnospeech{green drake}{green drake}
\pron{?}
\gl{\nA}
\bmng
 alApxyuvAda oMdu bageya kiVTa, hasirunoNa. 
\emng
\eentry

\bentry
\wordnospeech{green earth}{green earth}
\pron{?}
\gl{\nA}
\bmng
 hasiru maNuNx; hasiru raMgugaLige AdhAravAgi upayoVgisuva, \kanmu\ kabibxNada silikeVTugaLanonxLagoMDa, neYsagiRka maNuNx. 
\emng
\eentry

\bentry
\word{greener}
\pron{girxVnarf}
\gl{\nA}
\bmng
 (\ashi) hosaba; ananuBavi (\kanmu\ kelasa huDukikoMDu hosadAgi baMda parasathxLadavanu). 
\emng
\eentry

\bentry
\word{greenery}
\pron{girxVnari}
\gl{\nA}
\bmng
 hasiruvANi; vanarAji; payirupacecx; hasuru hululx; giDamara, baLiLx, \mo vu; hasuru sasayxgaLu. 
\emng
\eentry

\bentry
\word{greenery-yallery}
\pron{girxnariyAYxlari}
\gl{\gu}
\bmng
 (\AmA) (kaqtakavAgi) hasiru matutx haLadi baNaNxgaLanunx iSaTxpaDuva. 
\emng
\eentry

\bentry
\wordnospeech{green eye}{green eye}
\pron{?}
\gl{\nA}
\bmng
 karubu; hoTeTxkicucx; mAtasxyaR; asUye. 
\emng
\eentry

\bentry
\word{greeneyed}
\pron{girxVnfaiDf}
\gl{\gu}
\bmng
 karubuva; hoTeTx kicicxna; asUyeyuLaLx; matasxra budidhxyuLaLx. 
\emng

\noindent
\gl{\pagu}
\bmng
 \eng{greeneyed monster} = \hyperlink{green eye}{green eye}. 
\emng
\eentry

\bentry
\wordnospeech{green fat}{green fat}
\pron{?}
\gl{\nA}
\bmng
 hasuru kobubx; (rasikaru mecucxva) Ameya kobubx. 
\emng
\eentry

\bentry
\word{greenfinch}
\pron{girxVnfphinfcx}
\gl{\nA}
\bmng
 hasiru gubibx, phiMcf; hasiru matutx haLadi garigaLiruva gubibxyaMtha, yUroVpina oMdu hakikx. 
\emng
\eentry

\bentry
\word{greenfly}
\pron{girxVnfphelxY}
\gl{\nA}
\bmng
 (\birx) hasiru heVnu; oMdu terana giDaheVnu. 
\emng
\eentry

\bentry
\word{greengage}
\pron{girxVnfgeVjf}
\gl{\nA}
\bmng
 hasuru palxmf; duMDuduMDAgiruva, sogasAda ruciya, hasuru `palxmf' haNuNx. 
\emng
\eentry

\bentry
\wordnospeech{green goose}{green goose}
\pron{?}
\gl{\nA}
\bmng
 eLevaraTe; hatutx, hanenxraDu vAragaLa veVLegAgaleV AhArakekx baLasalu sidadhxvAgiruva, kobibxda varaTe, bAtu. 
\emng
\eentry

\bentry
\word{greengrocer}
\pron{girxVnfgorxVsarf}
\gl{\nA}
\bmng
 hasaruvANi vAyxpAri; kAyipalayxda vAyxpAri; tarakAri vAyxpAri; haNuNxhaMpalu, kAyipalayxgaLa cilalxre vAyxpAri. 
\emng
\eentry

\bentry
\word{greengrocery}
\pron{girxVnfgorxVsari}
\gl{\nA}
\bmng
\bnum
\num{1} (kAyipalayx, haNuNxhaMpalu mAruva) tarakAri aMgaDi. 
\num{2} (cilalxreyAgi vAyxpAravAguva) haNuNx tarakArigaLu. 
\num{3} haNuNxhaMpalu, kAyipalayx -- vAyxpAra. 
\enum
\emng
\eentry

\bentry
\word{greenhead}
\pron{girxVnfheDf}
\gl{\nA}
\bmng
\bnum
\num{1} hasirutale noNa; TAYxbaniDeV vaMshada, oMdu riVtiya kacucxva noNa. 
\num{2} hasiru tale iruve; noVyuvaMte kacucxva, AseTxrXVliyada iruve. 
\enum
\emng
\eentry

\bentry
\word{greenheart}
\pron{girxVnfhATfR}
\gl{\nA}
\bmng
 hasirudAru mara; gayAnadalilx beLeyuva, hasiru CAyeya dAru koDuva, oMdu bageya mara. 
\emng
\eentry

\bentry
\word{greenhorn}
\pron{girxVnfhAnfR}
\gl{\nA}
\bmng
 tiLigeVDi; ananuBavi; daDaDx; gugugx. 
\emng
\eentry

\bentry
\word{greenhouse}
\pron{girxVnfhwsf}
\gl{\nA}
\bmng
 hasirumane; (koVmala sasayxgaLanunx beLesuva yA giDagaLa beLavaNigeyanunx tavxritagoLisalu baLasuva) gAjina mane. 
\emng
\eentry

\bentry
\wordnospeech{greenhouse effect}{greenhouse effect}
\pron{?}
\gl{\nA}
\bmng
 hasirumane parxBAva, pariNAma; parxyoVgagaLige matitxtara udedxVshagaLige sasayxgaLa kaqSi mADalu upayoVgisuva gAjina manegaLalilx AguvaMte sUyaRrashimx nirAtaMkavAgi oLakekx baralu matutx parxtiPalita rakAtxtiVta kiraNagaLu horahoVgalu aDacaNeyuMTAgi oLagina vAyuvina tApa Eruvudu. 
\emng
\eentry

\bentry
\word{greening}
\pron{girxVniMgf}
\gl{\nA}
\bmng
 hasuru seVbu; pakavxvAdAga hasurAgiruva oMdu bageya seVbu. 
\emng
\eentry

\bentry
\word{greenish}
\pron{girxVniSf}
\gl{\gu}
\bmng
 nasu hasurubaNaNxda; tusu hasurAda. 
\emng
\eentry

\bentry
\word{greenkeeper}
\pron{girxVnfkiVparf}
\gl{\nA}
\bmng
 gAlfphx meYdAna pAlaka; gAlfphx meYdAnada usutxvAri noVDikoLuLxvava. 
\emng
\eentry

\bentry
\word{Greenlander}
\pron{girxVnflaMDarf}
\gl{\nA}
\bmng
 girxVnfleMDiga; girxVnfleMDfnalilx huTiTxdava yA vAsisuvava. 
\emng
\eentry

\bentry
\word[Greenlandic(1)]{Greenlandic}
\pron{girxVnflAYxMDikf}
\gl{\gu}
\bmng
 (AkfRTikf divxVpa) girxVnfleMDina. 
\emng
\eentry

\bentry
\word[Greenlandic(2)]{Greenlandic}
\pron{girxVnflAYxMDikf}
\gl{\nA}
\bmng
girxVnfleMDina BASe. 
\emng
\eentry

\bentry
\wordRemoveSpace{Greenland-whale}{Greenland whale}
\pron{girxVnflaMDf veVlf}
\gl{\nA}
\bmng
girxVnfleMDina timiMgila; AkfRTikf sAgaradalilxna timiMgila. 
\emng
\eentry

\bentry
\wordnospeech{green leek}{green leek}
\pron{?}
\gl{\nA}
\bmng
 hasiru muKada (AseTxrXVliyada) giLi. 
\emng
\eentry

\bentry
\word{greenlet}
\pron{girxVnfliTf}
\gl{\nA}
\bmng
 oMdu bageya saNaNx, hasiru baNaNxda (amerikanf) hADuhakikxya jAti. 
\emng
\eentry

\bentry
\wordnospeech{green light}{green light}
\pron{?}
\gl{\nA}
\bmng
\bnum
\num{1} hasirudiVpa; rasetx, haLi, \mo vugaLa meVle muMduvariyalu koDuva hasiru diVpada saMjecnx. 
\num{2} (\AmA) hasuru diVpa; (yAvudAdarU yoVjaneya \vi) kelasa naDeyali eMba -- anumati, opipxge, aMgiVkAra, maMjUrAti. 
\enum
\emng
\eentry

\bentry
\wordnospeech{green linnet}{green linnet}
\pron{?}
\gl{\nA}
\bmng
  = \hyperlink{greenfinch}{greenfinch}. 
\emng
\eentry

\bentry
\word{greenly}
\pron{girxVnfli}
\gl{\kirxvi}
\bmng
\bnum
\num{1} hasurAgi; hasuru baNaNxdiMda. 
\num{2} hasuriniMda tuMbi; (giDamaragaLiMda) nibiDavAgi: \eng{the earth broke greenly into spring} BUmi hasuriniMda tuMbi vasaMta Qutu baMtu. 
\num{3} (anuBavavilalxdadxriMda) oDoDxDADxgi; pedudxpedAdxgi. 
\num{4} (\rUpa) hocacx hosatAgi; hacacxhasirAgi; ceYtanayxpUNaRvAgi. 
\enum
\emng
\eentry

\bentry
\word{green-man}
\pron{girxVnfmanf}
\gl{\nA}
\bmng
  = \hyperlink{greenkeeper}{greenkeeper}. 
\emng
\eentry

\bentry
\wordnospeech{green manure}{green manure}
\pron{?}
\gl{\nA}
\bmng
(sasayxgaLanunx beLesi matutx biDuva) hasuru gobabxra; hasuruvANi gobabxra. 
\emng
\eentry

\bentry
\wordnospeech{Green Mountain State}{Green Mountain State}
\pron{?}
\gl{\nA}
\bmng
 amerikada saMyukatx saMsAthxnagaLalilx oMdAda varfmAMTf (\eng{Vermont}) saMsAthxna. 
\emng
\eentry

\bentry
\word{greenness}
\pron{girxVnfnisf}
\gl{\nA}
\bmng
\bnum
\num{1} hasurAgiruvike; hasurutana. 
\num{2} hasuru; hasuruhululx, giDamara, vanarAji, \mo vu. 
\num{3} (haNuNx \mo vugaLa \vi) kAyAgiruvike; apakavx sithxti. 
\num{4} ananuBava; eLasutana. 
\num{5} niSakxpaTate; ati mugadhxte; (sulaBavAgi moVsahoVguva) BoVLetana; saraLatana. 
\enum
\emng
\eentry

\bentry
\wordnospeech{Green Paper}{Green Paper}
\pron{?}
\gl{\nA}
\bmng
 haritapxtarx; hasiru patarx; hasiru kAgada; sakARravoMdu yAvudeV AshAvxsane niVDade, muMdiDuva tananx kAyaRkarxmagaLa, yoVjanegaLa tAtAkxlikavAda pArxyoVgika yA pariVkASxthaRka varadi. 
\emng
\eentry

\bentry
\wordnospeech{Green Peak}{Green Peak}
\pron{?}
\gl{\nA}
\bmng
 hasuru marakuTuka; haLadi pukakx matutx keMpu tale iruva, yUroVpina oMdu doDaDx marakuTuka hakikx. 
\emng
\eentry

\bentry
\wordnospeech{green plover}{green plover}
\pron{?}
\gl{\nA}
\bmng
 = \hyperref{kandict_l.pdf}{L}{lapwing}{lapwing}. 
\emng
\eentry

\bentry
\wordnospeech{green pound}{green pound}
\pron{?}
\gl{\nA}
\bmng
 hasirupwMDu; yUroVpiyanf ekanAmikf kamUyxniTige seVrida rASaTxrXgaLa kaqSi utApxdakarige salilxsuva haNada vinimayada EkamAna. 
\emng
\eentry

\bentry
\word{green-room}
\pron{girxVnfrUmf}
\gl{\nA}
\bmng
 neVpathayx; girxVnf rUmu; veVSada koThaDi; (raMgasathxLadalilxlalxdAga) naTanaTiyaru iruva koVNe. 
\emng
\eentry

\bentry
\word{greensand}
\pron{girxVnfsAYxMDf}
\gl{\nA}
\bmng
\bnum
\num{1}  = \hyperlink{green earth}{green earth}. 
\num{2} (I bageya maNiNxniMdAda) hasiru maraLugalulx. 
\num{3} hasiru maraLugalilxna satxra. 
\enum
\emng
\eentry

\bentry
\wordnospeech{greens fee}{greens fee}
\pron{?}
\gl{\nA}
\bmng
 gAlfphx meYdAnadalilx sadasayxra atithigaLu ATavADidare koDabeVkAda rusumu, phiVsu. 
\emng
\eentry

\bentry
\word{greenshank}
\pron{girxVnfSAYxMkf}
\gl{\nA}
\bmng
 doDaDx sAyxMDfpeYparf hakikx. 
\emng
\eentry

\bentry
\word{greensick}
\pron{girxVnfsikf}
\gl{\gu}
\bmng
 = \hyperref{kandict_c.pdf}{C}{chlorotic}{chlorotic}. 
\emng
\eentry

\bentry
\word{greensickness}
\pron{girxVnfsikfnisf}
\gl{\nA}
\bmng
 = \hyperref{kandict_c.pdf}{C}{chlorosis}{chlorosis}. 
\emng
\eentry

\bentry
\word{green-stick}
\pron{girxVnfsiTxkf}
\gl{\nA}
\bmng
 (visheVSavAgi eLe makakxLige) elubu muridAga oMdu kaDe cipepxdudx hoVgi inonxMdu kaDe bAgihoVguva oMdu bageya elubu murita. 
\emng
\eentry

\bentry
\word{greenstone}
\pron{girxVnfsoTxVnf}
\gl{\nA}
\bmng
\bnum
\num{1} hasirugalulx; hasirushile; jAvxlAmuKiyiMda horabaruva, phelfsApxrf matutx hAnfRbelxMDfgaLuLaLx oMdu bageya kalulx. 
\num{2} (nUyxsiZVleMDfnalilx dorakuva, oDavegaLu \mo vugaLige baLasuva) oMdu bageya `jeVDf' shile. 
\enum
\emng
\eentry

\bentry
\word{greenstuff}
\pron{girxVnfsaTxphf}
\gl{\nA}
\bmng
\bnum
\num{1} sasayxgaLu; hasuru beLe. 
\num{2} (hasuru) kAyipalayx; sopupx; tarakAri. 
\enum
\emng
\eentry

\bentry
\word{greensward}
\pron{girxVnfsAvxDfR}
\gl{\nA}
\bmng
 (tekekxgaTiTxda, teMDegaTiTxda) hululxnela. 
\emng
\eentry

\bentry
\wordnospeech{green table}{green table}
\pron{?}
\gl{\nA}
\bmng
 jUju meVju; jUjATada meVju. 
\emng
\eentry

\bentry
\word{green-tail}
\pron{girxVnfTeVlf}
\gl{\nA}
\bmng
  = \hyperlink{green-tail}{grannom}. 
\emng
\eentry

\bentry
\wordnospeech{green tea}{green tea}
\pron{?}
\gl{\nA}
\bmng
 hasuru -- caha, TiV; Aviyalilx oNagisi tayArisida TiV eleyiMda mADida caha. 
\emng
\eentry

\bentry
\word{greenth}
\pron{girxVnftx}
\gl{\nA}
\bmng
 (sAhitayxka) = \hyperref{kandict_v.pdf}{V}{verdure}{verdure}. 
\emng
\eentry

\bentry
\wordnospeech{green thumb}{green thumb}
\pron{?}
\gl{\nA}
\bmng
 (\AmA)  = \hyperlink{green(1)nuga}{$^1$green ?nuga? \((1)\)}. 
\emng
\eentry

\bentry
\wordnospeech{green turtle}{green turtle}
\pron{?}
\gl{\nA}
\bmng
 hasirAme; AhArakekx baLasuva, hasiru cipipxna Ame. 
\emng
\eentry

\bentry
\wordnospeech{green vitriol}{green vitriol}
\pron{?}
\gl{\nA}
\bmng
 girxVnf viTirxyalf; pherasf salephxVTina haraLugaLu. 
\emng
\eentry

\bentry
\word{greenweed}
\pron{girxVnfviVDf}
\gl{\nA}
\bmng
 raMgina giDa; vaNaRsasayx; baTeTxgaLige baNaNx hAkalu baLasuva, oMdu bageya haLadi hU biDuva, genisaTx TiMkoTxVriya eMba oMdu bageya haLadi hU biDuva podegiDa. 
\emng
\eentry

\bentry
\word{Greenwich}
\pron{girx(gerx)nijf(cf)}
\gl{\nA}
\bmng
 girxnijf; AgenxVya laMDaninxna oMdu paTaTxNa (hiMde ililxdadx rAyalf KagoVLa viVkaSxNAlayavanunx Iga hasfTxRmAnfsUkfsx eMbalilxge sAgisalAgide). 
\emng
\eentry

\bentry
\wordnospeech{Greenwich civil time}{Greenwich civil time}
\pron{?}
\gl{\nA}
\bmng
  = \hyperlink{Greenwich time}{Greenwich time}. 
\emng
\eentry

\bentry
\wordnospeech{Greenwich mean time}{Greenwich mean time}
\pron{?}
\gl{\nA}
\bmng
  = \hyperlink{Greenwich time}{Greenwich time}. 
\emng
\eentry

\bentry
\wordnospeech{Greenwich time}{Greenwich time}
\pron{?}
\gl{\nA}
\bmng
 girxVnicf kAlamAna; girxVnicfna madhAyxhanx reVKege sariyAda sarAsari kAla (iMgelxMDf matutx itara kelavu deVshagaLalilx baLake). 
\emng
\eentry

\bentry
\word{greenwood}
\pron{girxVnfvuDf}
\gl{\nA}
\bmng
 (beVsageya) hasurugADu (\kanmu\ deVshaBarxSaTxra, kaLaLxkAkara -- vAsasathxLa, kAyaRkeSxVtarx). 
\emng
\eentry

\bentry
\word[greeny(1)]{greeny}
\pron{girxVni}
\gl{\gu}
\bmng
  = \hyperlink{greenish}{greenish}. 
\emng
\eentry

\bentry
\word[greeny(2)]{greeny}
\pron{girxVni}
\gl{\nA}
\bmng
  = \hyperlink{greenhorn}{greenhorn}. 
\emng
\eentry

\bentry
\word{greeny-}
\pron{girxVni-}
\gl{\sapUpa}
\bmng
 hasiru eMbathaRda \sapUpa: \eng{greeny - yellow} hasirumisharx haLadi. 
\emng
\eentry

\bentry
\word{greenyard}
\pron{girxVnfyADfR}
\gl{\nA}
\bmng
 (toMDu pArxNigaLanunx kUDuva) doDiDx; ropapx. 
\emng
\eentry

\bentry
\word[greet(1)]{greet}
\pron{girxVTf}
\gl{\sakirx}
\bmng
\bnum
\num{1} (namasakxrisi) kushala parxshenx mADu. 
\num{2} (mAtiniMda yA aBinayadiMda) aBinaMdisu: \eng{was greeted with acclamation} jayaGoVSadoMdige aBinaMdisalAyitu. 
\num{3} saMdhisidAga yA baMdu iLidAga (senxVhaBAvada yA alalxda) mAtugaLiMda, aMgacAlanegaLiMda edurugoLuLx. 
\num{4} (jayakAradoMdige) sAvxgatisu, purasakxrisu yA aMgiVkarisu: \eng{shouts of assent greeted the resolution} niNaRyavanunx jayakAragaLoMdige aMgiVkarisalAyitu. 
\num{5} (noVTa, daqshayx, \mo vu kaNiNxge, kivige) biVLu; goVcaravAgu; kaNiNxge idirAgu; edudx toVru: \eng{a wide extent of sea greets the eye} samudarxda vishAla harahu kaNaNx muMde edudx toVrutatxde. kaNiNxge goVcaravAgutatxde. 
\enum
\emng
\eentry

\bentry
\word[greet(2)]{greet}
\pron{girxVTf}
\gl{\akirx}
\bmng
 (sAkxTalxMDf \parx) aLu; goVLADu; parxlApisu. 
\emng
\eentry

\bentry
\word{greeting}
\pron{girxVTiMgf}
\gl{\nA}
\bmng
\bnum
\num{1} (namasakxrisi) kushala parxshenx mADuvudu. 
\num{2} (mAtiniMda yA aBinayadiMda) aBivaMdisuvudu. 
\num{3} saMdhisidAga yA baMdu iLidAga (senxVhaBAvada yA alalxda) mAtugaLiMda, aMgacAlanegaLiMda edurugoLuLxvudu. 
\num{4} sAvxgata; jayakAradoMdige sAvxgatisuvudu, aMgiVkarisuvudu. 
\enum
\emng
\eentry

\bentry
\word{greetings}
\pron{girxVTiMgfs'}
\gl{\nA}
\bmng
 (\bava) shuBAshayagaLu; shuBakAmanegaLu; osageya mAtugaLu: \eng{Deepavali, Christmas greetings} diVpAvaLiya, kirxsfmasfna -- shuBAshayagaLu. 
\emng
\eentry

\bentry
\word{greffier}
\pron{gerxphiarf}
\gl{\nA}
\bmng
 (\kanmu\ iMgelxMDinAceya paradeVshagaLalilx matutx cAnalf divxVpagaLalilx) rijisATxrXrf; noVTari. 
\emng
\eentry

\bentry
\word{gregarious}
\pron{girxgeVriasf}
\gl{\gu}
\bmng
\bnum
\num{1} guMpAgi vAsisuva; samUhavAsi; maMdegUDi vAsisuva. 
\num{2} saMGajiVviyAda; sahavAsapirxya. 
\num{3} (\savi) goMcalugoMcalAgi beLeyuva. 
\num{4} maMdeya; hiMDina; guMpina. 
\num{5} sAmUhika; sAmudAyika: \eng{mere religious zeal is a gregarious thing} dhamaRvanunx kurita aMdhasharxdedhx keVvala oMdu sAmudAyika parxvaqtitx. 
\enum
\emng
\eentry

\bentry
\word{gregariously}
\pron{girxgeVriasfli}
\gl{\kirxvi}
\bmng
\bnum
\num{1} maMdegUDi vAsisutatx; guMpAgi jiVvisutatx. 
\num{2} saMGajiVviyAgi; sahavAsapirxyanA(LA)gi. 
\num{3} (\savi) goMcalugoMcalAgi. 
\num{4} guMpAgi. 
\num{5} sAmUhikavAgi. 
\enum
\emng
\eentry

\bentry
\word{gregariousness}
\pron{girxgeVriasfnisf}
\gl{\nA}
\bmng
\bnum
\num{1} sAmUhika vAsa; maMdegUDi vAsisuvike; guMpAgi jiVvisuvike. 
\num{2} saMGajiVvana; sahavAsapirxyate. 
\num{3} (\savi) goMcalugoMcalAgiruvike. 
\num{4} guMpinalilxruvike; guMpupirxyate. 
\num{5} sAmudAyikate; samudAyapirxyate. 
\enum
\emng
\eentry

\bentry
\word[grege(1)]{grege}
\pron{gerxVSfZ}
\gl{\gu}
\bmng
 nasuhaLadi kaMdubaNaNxkUkx bUdubaNaNxkUkx naDuvaNa baNaNxda. 
\emng
\eentry

\bentry
\word[grege(2)]{grege}
\pron{gerxVSfZ}
\gl{\nA}
\bmng
 nasu haLadi kaMdubaNaNx hAgU bUdu baNaNxgaLa naDuvaNa baNaNx. 
\emng
\eentry

\bentry
\word[Gregorian(1)]{Gregorian}
\pron{girxgoVrianf}
\gl{\gu}
\bmng
\bnum
\num{1} girxgari gAnada yA girxgari gAnadaMtha; poVpf oMdaneya girxgari (\kirxsha\ \eng{540--604}) racisidanenanxlAda (roVmanf kAyxtholikf) cacfR dhamARcaraNeya saMdaBaRdalilx hADuva, saraLagAnada yA A gAnakekx saMbaMdhisida. 
\num{2} girxgari sAthxpisida. 
\enum
\emng
\eentry

\bentry
\word[Gregorian(2)]{Gregorian}
\pron{girxgoVrianf}
\gl{\nA}
\bmng
  = \hyperlink{Gregorian chant}{Gregorian chant}. 
\emng
\eentry

\bentry
\wordnospeech{Gregorian calendar}{Gregorian calendar}
\pron{?}
\gl{\nA}
\bmng
 girxgari paMcAMga; \eng{1582}ralilx jUliyanf paMcAMgavanunx tididx \eng{13}neV poVpf girxgari tayArisida paMcAMga. 
\emng
\eentry

\bentry
\wordnospeech{Gregorian chant}{Gregorian chant}
\pron{?}
\gl{\nA}
\bmng
 oMdaneV girxgari racisidanenanxlAda dhamARcaraNeya hADu yA A (hADina) rAgadalilxruvudu. 
\emng
\eentry

\bentry
\wordnospeech{Gregorian epoch}{Gregorian epoch}
\pron{?}
\gl{\nA}
\bmng
 (\eng{1582}riMda pArxraMBavAguva) girxgari shake. 
\emng
\eentry

\bentry
\wordnospeech{Gregorian style}{Gregorian style}
\pron{?}
\gl{\nA}
\bmng
 girxgariya \eng{(1582)} hosa (paMcAMga) padadhxti. 
\emng
\eentry

\bentry
\wordnospeech{Gregorian telescope}{Gregorian telescope}
\pron{?}
\gl{\nA}
\bmng
 girxgari dUradashaRka; girxgoVriyanf dUradashaRka; divxtiVyaka kananxDiyiMda parxtiPalanagoMDa beLaku pArxthamika kananxDiyalilxna raMdharxda muKAMtara hoVguvaMte \eng{17}neya shatamAnada vijAcnxni je. girxgari racisida parxtiPalana dUradashaRka. 
\emng
\eentry

\bentry
\wordnospeech{Gregorian tones}{Gregorian tones}
\pron{?}
\gl{\nA}
\bmng
 girxgari giVtagaLu; roVmanf kAyxtholikf caciRnalilx pArxthaRnA giVtagaLeMdu niyamisalAda eMTu saraLa gAnada kaqtigaLu. 
\emng
\eentry

\bentry
\word{gregory-powder}
\pron{gerxgaripwDarf}
\gl{\nA}
\bmng
 gerxgari puDi; vireVcakavAgi baLasuva, reVvalf cininx beVrina puDi. 
\emng
\eentry

\bentry
\word[greige(1)]{greige}
\pron{gerxVSf}
\gl{\gu}
\bmng
  = \hyperlink{grege(1)}{$^1$grege}. 
\emng
\eentry

\bentry
\word[greige(2)]{greige}
\pron{gerxVSfZ}
\gl{\nA}
\bmng
  = \hyperlink{grege(2)}{$^2$grege}. 
\emng
\eentry

\bentry
\word{gremial}
\pron{girxVmialf}
\gl{\nA}
\bmng
 toDebaTeTx; Uru vasatxrX; kerxYsatx dhamARcaraNeya saMdaBaRgaLalilx biSapf guruvina toDeya meVle hAkuva reVSemx baTeTx. 
\emng
\eentry

\bentry
\word{gremlin}
\pron{gerxmilxnf}
\gl{\nA}
\bmng
 (\ashi) yaMtarx \mo vugaLige toMdare mADuvudeMdu naMbalAgidadx oMdu tuMTa pishAci, devavx. 
\emng
\eentry

\bentry
\word{grenade}
\pron{girxneVDf}
\gl{\nA}
\bmng
gerxneVDu: 
\hypertarget{grenade(a)}{} 
\banum
\alnum{a} keYsiDiguMDu; keYbAMbu. 
\hypertarget{grenade(b)}{} 
\alnum{b} koVviyiMda hArisabahudAda guMDu. 
\alnum{c} guMpu cadurisuvudakokxV beMki ArisuvudakokxV matetx yAvudoV udedxVshakAkxgiyoV rAsAyanikavanunx siMpaDisalu upayoVgisuva, rAsAyanika tuMbida gAjina buruDe. 
\eanum
\emng

\noindent
\gl{\pagu}
\bmng
\bnum
\num{1} \eng{hand-grenade} = \hyperlink{grenade(a)}{grenade(a)}. 
\num{2} \eng{rifle grenade} = \hyperlink{grenade(b)}{grenade(b)}. 
\enum
\emng
\eentry

\bentry
\word{grenadier}
\pron{gerxnaDiarf}
\gl{\nA}
\bmng
\bnum
\num{1} (\ca) keYbAMbu yA siDiguMDu eseyuva seYnika. 
\num{2} (\birx) \eng{Grenadiers} yA \eng{Grenadier Guards} gerxnaDiyarf daLagaLu; iMgelxMDina aramaneya seYnayxda padAti daLagaLalilx modala daLa. 
\num{3} gerxneVDiyarf (hakikx); kapupx matutx keMpu garigaLiruva dakiSxNa Aphirxkada giVjagana hakikx. 
\hyperdef{G}{grenadier(4)}{} 
\num{4} makwrxriDeV vaMshada, udadxvAda, kiridAgutatx hoVguva deVhavU cUpubAlavU uLaLx, ALasamudarxda miVnu. 
\enum
\emng
\eentry

\bentry
\word{grenadilla}
\pron{gerxnaDila}
\gl{\nA}
\bmng
  = \hyperlink{granadilla}{granadilla}. 
\emng
\eentry

\bentry
\word[grenadine(1)]{grenadine}
\pron{gerxnaDinf}
\gl{\nA}
\bmng
 gerxnaDinf; koVLiya pakekxmAMsakekx haMdiya kobubx hAki tayArisida oMdu KAdayx padAthaR. 
\emng
\eentry

\bentry
\word[grenadine(2)]{grenadine}
\pron{gerxnaDiVnf}
\gl{\nA}
\bmng
 gerxnaDiVnf; uDupugaLige baLasuva, reVSemxya yA reVSemx matutx uNeNxya baTeTx. 
\emng
\eentry

\bentry
\word[grenadine(3)]{grenadine}
\pron{gerxnaDiVnf}
\gl{\nA}
\bmng
 gerxnaDiVnf; haqdayoVtetxVjakavAda dALiMbe \mo vugaLa sharabatutx. 
\emng
\eentry

\bentry
\wordRemoveSpace{Gresham's-law}{Gresham's law}
\pron{gerxSamfsx lA}
\gl{\nA}
\bmng
 gerxSamf -- niyama, sUtarx; oMdeV beleya eraDu nANayxgaLu calAvaNeyalilxruvAga kaDime nija mwlayxda nANayxvu hecucx nijamwlayxda nANayxvanunx horadUDi tAneV calAvaNeyalilx nilulxtatxdeMdu parxtipAdisuva sUtarx, niyama. 
\emng
\eentry

\bentry
\word{gressorial}
\pron{gerxsoVrialf}
\gl{\gu}
\bmng
 (\pArxvi) 
\bnum
\num{1} naDageya. 
\num{2} naDagege anukUlisida. 
\enum
\emng
\eentry

\bentry
\wordRemoveSpace{Gretna-Green-marriage}{Gretna Green marriage}
\pron{gerxTanx girxVnf mAYxrijf}
\gl{\nA}
\bmng
 (\ca) gerxTanx girxVnf maduve; iMgelxMDiniMda ODihoVgi, kAnUnina parxkAra taMde tAyiyara opipxge beVkilalxda, sAkxTelxMDina gaDiya haLiLx `gerxTanx girxVnf' eMbalilx mADikoMDa maduve. 
\emng
\eentry

\bentry
\word{grew}
\pron{gUrx}
\gl{\kirx}
\bmng
 \eng{grow} kirxyApadada BUtarUpa. 
\emng
\eentry

\bentry
\word[grey(1)]{grey}
\pron{gerxV}
\gl{\gu}
\bmng
\bnum
\num{1} bUdu (baNaNxda); nare; biLupigU kapipxgU madhayxsathxvAda bUdiyaMtha yA siVsadaMtha baNaNxda. 
\num{2} nasugatatxlAda; masukAda; mabubxmabAbxda. 
\num{3} moVDa kavida. 
\num{4} gelavilalxda; ulAlxsavilalxda; AshAdAyakavalalxda; nirutAsxhakara: \eng{a grey report} nirutAsxhakara varadi. 
\num{5} maMkAda; nisetxVja. 
\num{6} (vayxkitx yA vayxkitxya kUdalina \vi) (mupupx \mo vugaLiMda) nareta; biLupAda; nare tirugida: \eng{a grey old man} kUdalu nareta mudiya. 
\num{7} pArxciVna; purAtana; anAdi kAlada. 
\num{8} mupipxna; mupApxda; vaqdAdhxpayxda. 
\num{9} nurita; parisharxmavuLaLx; paLagida; oLeLxya anuBavavuLaLx; paripakavxvAda: \eng{grey wisdom} paripakavx viveVka. 
\num{10} (vayxkitxya \vi) ajAcnxta; yAreMdu tiLiyada. 
\enum
\emng

\noindent
\gl{\nuga}
\bmng
 \eng{the grey mare is the better horse} sAvira kudure saradAranAdarU mane heMDati kAsadAra; heMDati gaMDana meVle adhikAra naDesutAtxLe. 
\emng
\eentry

\bentry
\word[grey(2)]{grey}
\pron{gerxV}
\gl{\nA}
\bmng
\bnum
\num{1} bUdu vasatxrX; bUdu uDupu, baTeTxgaLu: \eng{dressed in grey} bUdu baTeTx uTaTx. 
\num{2} muMjAneya yA saMjeya -- nasuku, beYgu, mabubx. 
\num{3} bUdu baNaNx. 
\num{4} bUdu vaNaRdarxvayx. 
\num{5} bUdugudure. 
\num{6} (\ame) (niVgorxV \ashi) biLiya; shevxVta vaNiRVya. 
\enum
\emng

\noindent
\gl{\pagu}
\bmng
\hypertarget{Grey pagu1}{} 
\bnum
\num{1} \eng{Scot Greys} (birxTiSf seVneya) `DarxgUnfsx' eMba eraDaneya ashavxdaLa. 
\num{2} \eng{the Greys} = \hyperlink{Grey pagu1}{?pagu? \((1)\)}. 
\enum
\emng
\eentry

\bentry
\word[grey(3)]{grey}
\pron{gerxV}
\gl{\sakirx}
\bmng
\bnum
\num{1} bUdu baNaNxkekx tirugisu; bUdAgisu. 
\num{2} (\CA) (gAjina meVlemxYyanunx) bUdugoLisu; masukumADu. 
\num{3} (\CA) (viSama citarxda meVle masuku gAjaninxDuva mUlaka CAyAcitarxvanunx) masukumADu; bUdugoLisu. 
\enum
\emng

\noindent
\gl{\akirx}
\bmng
\bnum
\num{1} bUdubaNaNxvAgu; bUdAgu; bUdubaNaNxkekx tirugu. 
\num{2} masukAgu. 
\enum
\emng
\eentry

\bentry
\wordnospeech{grey area}{grey area}
\pron{?}
\gl{\nA}
\bmng
 bUduvalaya; sUtarx, niyama, \mo vugaLa mUlaka sapxSaTxvAgi beVpaRDisi vivarisalAgada saMdigadhx keSxVtarx. 
\emng
\eentry

\bentry
\word{grey-back}
\pron{gerxVbAYxkf}
\gl{\nA}
\bmng
\bnum
\num{1} bUdubeninxga; bUdubeninxna pArxNi yA pakiSx. 
\num{2} (\ame) (\ca) kanfpheDareVTf seYnika; amerikada aMtayuRdadhxda kAladalilx EpaRTiTxdadx dakiSxNada saMsAthxnagaLa okUkxTada seYnayxda seYnika. 
\enum
\emng
\eentry

\bentry
\word{greybeard}
\pron{gerxVbiaDfR}
\gl{\nA}
\bmng
\bnum
\num{1} dADi naretavanu; muduka. 
\num{2} (madayx tuMbiDuva) doDaDx jADi. 
\num{3} (\birx) = \hyperref{kandict_c.pdf}{C}{clematis}{clematis}. 
\enum
\emng
\eentry

\bentry
\word{greycing}
\pron{gerxVsiMgf}
\gl{\nA}
\bmng
 (\birx) (\AmA) \eng{greyhound racing} eMbudara \saMkiSx. 
\emng
\eentry

\bentry
\wordnospeech{grey cells}{grey cells}
\pron{?}
\gl{\nA}
\bmng
  = \hyperlink{grey matter}{grey matter}. 
\emng
\eentry

\bentry
\word{greycoat}
\pron{gerxVkoVTf}
\gl{\nA}
\bmng
 bUdaMgiyavanu, \kanmu\ kaMbarfleMDina saNaNx reYta. 
\emng
\eentry

\bentry
\wordnospeech{grey crow}{grey crow}
\pron{?}
\gl{\nA}
\bmng
 bUdukAge; juTuTxkAge; taleya meVle juTiTxruva kAge. 
\emng
\eentry

\bentry
\wordnospeech{grey drake}{grey drake}
\pron{?}
\gl{\nA}
\bmng
 (\birx) bUdunoNa; oMdu bageya dinajiVvi (kiVTa). 
\emng
\eentry

\bentry
\wordnospeech{grey eminence}{grey eminence}
\pron{?}
\gl{\nA}
\bmng
 = \hyperref{kandict_e.pdf}{E}{eminence grise}{\it eminence grise.} 
\emng
\eentry

\bentry
\wordnospeech{grey eye}{grey eye}
\pron{?}
\gl{\nA}
\bmng
 bUdugaNuNx; bUdubaNaNxda pApeporeyuLaLx kaNuNx. 
\emng
\eentry

\bentry
\wordnospeech{Grey Friar}{Grey Friar}
\pron{?}
\gl{\nA}
\bmng
 phArxnisxsakxnf paMthada saMnAyxsi. 
\emng
\eentry

\bentry
\wordnospeech{grey goose}{grey goose}
\pron{?}
\gl{\nA}
\bmng
 bUduvaraTe; nAlukx tiMgaLu tuMbuvudaroLage sAyisi masAle tuMbade tinunx varaTe. 
\emng
\eentry

\bentry
\word{greyheaded}
\pron{gerxVheDiDf}
\gl{\gu}
\bmng
\bnum
\num{1} tale nareta; mudiyAda. 
\num{2} bahukAla duDida, seVve mADida. 
\num{3} haLeya; pArxciVna; purAtana. 
\enum
\emng
\eentry

\bentry
\word{grey-hen}
\pron{gerxVhenf}
\gl{\nA}
\bmng
 heNuNx kari gwrxsf (hakikx). 
\emng
\eentry

\bentry
\word{greyhound}
\pron{gerxVhwMDf}
\gl{\nA}
\bmng
 gerxVhwMDf: 
\banum
\alnum{a} etatxravAda, teLaLxneya, udadxkAlina, tiVkaSxNX daqSiTxya beVTenAyi. \imglink{greyhoundfigure}{\raisebox{-0.15cm}[0pt][0pt]{\pdfimage width 0.8cm height 0.6cm {G_Pictures/greyhound.jpg}}} 
\alnum{b} iMtha nAyiya jAti. 
\eanum
\emng
\eentry

\bentry
\word{greyhound-racing}
\pron{gerxVhwMDfreVsiMgf}
\gl{\nA}
\bmng
 beVTenAyi jUju; paMdayx kaTaTxlu yAMtirxka molavanunx beVTe nAyigaLiMda benanxTiTxsuva oMdu bageya Adhunika vinoVda kirxVDe. 
\emng
\eentry

\bentry
\word{greyish}
\pron{gerxViSf}
\gl{\gu}
\bmng
\bnum
\num{1} nasubUdubaNaNxda; madhayxsathx bUdubaNaNxda: \eng{of greyish robes} bUdu vasatxrXgaLu. 
\num{2} (baNaNxda \vi) tiLiyAda bUdumishirxta: \eng{of greyish blue} bUdu niVliya. 
\enum
\emng
\eentry

\bentry
\word{greylag}
\pron{gerxVlAYxgf}
\gl{\nA}
\bmng
 yUroVpina sAmAnayx kADuvaraTe. 
\emng
\eentry

\bentry
\wordnospeech{greylag goose}{greylag goose}
\pron{?}
\gl{\nA}
\bmng
  = \hyperlink{greylag}{greylag}. 
\emng
\eentry

\bentry
\word{greyly}
\pron{gerxVli}
\gl{\kirxvi}
\bmng
 nasubUdAgi; bUdubaNaNxdalilx; madhayxsathx bUdubaNaNxvAgi; tiLibUdu baNaNxdalilx. 
\emng
\eentry

\bentry
\wordnospeech{grey matter}{grey matter}
\pron{?}
\gl{\nA}
\bmng
\bnum
\num{1} (\aMrashA) (meduLina caTuvaTikeya BAgada) bUdudarxvayx. 
\num{2} budidhx; budidhxvaMtike; meVdhAvitana. 
\enum
\emng
\eentry

\bentry
\wordnospeech{grey monk}{grey monk}
\pron{?}
\gl{\nA}
\bmng
 = \hyperref{kandict_c.pdf}{C}{Cistercian(1)}{Cistercian}. 
\emng
\eentry

\bentry
\word{greyness}
\pron{gerxVnisf}
\gl{\nA}
\bmng
\bnum
\num{1} bUdubaNaNxvAgiruvudu; bUdubaNaNx taLediruvike. 
\num{2} (kUdalu) naretiruvike. 
\num{3} nasuku; mabubx. 
\num{4} maMku; nirutAsxha; gelavilalxdiruvike. 
\num{5} pArxciVnate. 
\enum
\emng
\eentry

\bentry
\wordnospeech{grey squirrel}{grey squirrel}
\pron{?}
\gl{\nA}
\bmng
 bUdu aLilu; \eng{19}neV shatamAnadalilx yUroVpige taMda amerikada aLilu. 
\emng
\eentry

\bentry
\word{greystone}
\pron{gerxVsoTxVnf}
\gl{\nA}
\bmng
 bUdugalulx; bUdubaNaNxda jAvxlAmuKi shile. 
\emng
\eentry

\bentry
\word{greywacke}
\pron{gerxVvAYxkf}
\gl{\nA}
\bmng
 (\BUvi) duMDaneya kalulxharaLU maraLU seVri Agiruva, cUrugalulxMDe. 
\emng
\eentry

\bentry
\word{grid}
\pron{girxDf}
\gl{\nA}
\bmng
\bnum
\numi{1} girxDf jAlari: 
\banum
\alnum{a} saraLugaLanunx aDaDxDaDxkUkx ududxdadxkUkx samAMtaravAgi aLavaDisiruva cwkaTuTx. 
\alnum{b} adeV riVti taMtigaLiMda tayArisida cwkaTuTx. 
\alnum{c} kavATada dhanadAvxrakUkx taMtuvigU naDuvaNa taMti bale. 
\alnum{d} miliTari nakeSxgaLalilx sathxLa nideRVshanakAkxgi eLediruva cwkaLi manegaLu. 
\alnum{e} viduyxcaCxkitx sarabarAjigAgi EpaRDisiruva vAhaka taMtigaLa bale. 
\alnum{f} (aDigege, suDuvudakekx baLasuva) saraLu taDe. 
\alnum{g} anilavanunx sarabarAju mADalu EpaRDisiruva koLavegaLa vayxvasethx. 
\eanum
\numie
\num{2} gerejAla; geregUDu; reVKAkoVSaThx; reVKAjAlari; oMdanonxMdu laMbavAgi aDaDxhAyuva, samAMtaradalilxruva matutx saMKeyxgaLanunx hAkiruva reVKegaLa, \kanmu\ nakeSxyalilx sathxLagaLanunx gotutxpaDisalu baLasuva reVKegaLa jAla. 
\num{3} reVsu gere, paTeTx; paMdayxda kArugaLu horaDuvAga yAva yAva sathxLadalilx nilalxbeVku enunxvudanunx sUcisalu kArugaLa pathada meVle baLidiruva reVKegaLa jAla, gerepaTeTxgaLu. 
\num{4}  = \hyperlink{gridiron}{gridiron (1, 2, 3)}. 
\hypertarget{grid(5)}{} 
\num{5} Aya -- vayxvasethx, racane; paTaTxNda rasetxgaLanunx racisiruva AyAkArada vinAyxsa, vayxvasethx. 
\enum
\emng
\eentry

\bentry
\word{gridded}
\pron{girxDiDf}
\gl{\gu}
\bmng
 jAlariyuLaLx; jAlarivishiSaTx; jAlariyAguva yA jAlariyiMda Avarisiruva. 
\emng
\eentry

\bentry
\word[griddle(1)]{griddle}
\pron{girxDflf}
\gl{\nA}
\bmng
\bnum
\num{1} (doVse hoyuyxva) kabibxNada guMDu kAvali, heMcu. 
\num{2} gaNi kelasadavana taMti (taLaviruva) jaraDi, jalalxDi, cwkaTuTx yA oMdari. 
\enum
\emng
\eentry

\bentry
\word[griddle(2)]{griddle}
\pron{girxDflf}
\gl{\sakirx}
\bmng
 (aduranunx) jalalxDi hiDi; jaraDi hiDi; oMdariyADu. 
\emng
\eentry

\bentry
\word[gride(1)]{gride}
\pron{gerxYDf}
\gl{\akirx}
\bmng
 karakarane, caracarane -- koyiyx, here. 
\emng

\noindent
\gl{\pagu}
\bmng
 \eng{grides its way} karakarane koyudxkoMDu hoVgutatxde. 
\emng
\eentry

\bentry
\word[gride(2)]{gride}
\pron{gerxYDf}
\gl{\nA}
\bmng
 karakara shabadx; kakaRsha dhavxni. 
\emng
\eentry

\bentry
\word{gridiron}
\pron{girxDfaianfR}
\gl{\nA}
\bmng
\bnum
\num{1} (keMDada meVliTuTx mAMsa yA miVnanunx suDalu baLasuva) kaMbi jAlari; saraLu cwkaTuTx. 
\num{2} (\nw) tole cwkaTuTx; haDagukaTeTxyalilx haDagige AsareyAda samAnAMtara tolegaLa cwkaTuTx. 
\num{3} (\ame) (ADuva jAgada elelxyanunx eraDu samAnAMtara reVKegaLa mUlaka gurutu mADiruva) kAlecxMDu (phuTfbAlf) meYdAna. 
\num{4} (raMgasathxLa) meVlacxpapxra; meVlfcwkaTuTx; keLage iLi biDuva citarxda parade \mo vugaLige AsareyAda, raMgasathxLada meVlABxgadalilxya tolegaLa vayxvasethx. 
\num{5} nAvikaseVneya oMdu vinAyxsa badalAvaNe. 
\num{6}  = \hyperlink{grid-iron pendulum}{grid-iron pendulum}. 
\num{7}  = \hyperlink{grid(5)}{grid (5)}. 
\enum
\emng
\eentry

\bentry
\wordnospeech{grid-iron pendulum}{grid-iron pendulum}
\pron{?}
\gl{\nA}
\bmng
 jAlari loVlaka; loVlakada udadx parxBAvataH sithxravAgiruvaMte eraDu beVre beVre loVhagaLa asama visatxraNeyanunx baLasikoMDu tayArisuva oMdu bageya loVlaka. 
\emng
\eentry

\bentry
\word{grief}
\pron{girxVphf}
\gl{\nA}
\bmng
\bnum
\num{1} koragu; aLalu; vayxsana; duHKa; shoVka; saMtApa. 
\num{2} tiVvarx vayxthe. 
\num{3} aLalige, vayxthege kAraNa; kaSaTx; vipatutx. 
\enum
\emng

\noindent
\gl{\nuga}
\bmng
\bnum
\numi{1} \eng{come to grief} 
\banum
\alnum{a} kaSaTxkekx sikukx; duHKakekx oLagAgu; vipatitxge sikukx. 
\alnum{b} (ATadalilx) soVlu; bidudxhoVgu. 
\eanum
\numie
\num{2} \eng{good} (\engit{or} \eng{great) grief} (AshacxyaR, apAya, \mo vanunx sUcisuva udAgxra) ayoyxV! abAbx! 
\enum
\emng
\eentry

\bentry
\word{grievance}
\pron{girxVvanfsx}
\gl{\nA}
\bmng
\bnum
\num{1} saMkaTa; duHKa; piVDe: \eng{grievances illegally inflicted} akarxmavAgi koDuva piVDegaLu. 
\num{2} koVpa; siDuku; asamAdhAna; asaMtoVSa; ataqpitx (ivugaLiMduMTAguva manasisxna sithxti): \eng{a grievance against whistlers in public} sAvaRjanika sathxLagaLalilx shiLuLx hAkuvavara meVlina asamAdhAna, siTuTx. 
\num{3} (tanage anAyxyavAgideyeMba) aLalu; beVgudi; manadubabxra; ataqpitx; asamAdhAna; saMkaTagaLanunxMTumADiruvareMba dUru: \eng{the grievance of taxation without representation} tamage pArxtinidhayx koDade tamimxMda terige vasUli mADutitxdAdxreMba dUru, aLalu, beVgudi. 
\num{4} (kelasagArara \vi) vaqtitxyalilxna ananukUla, aswkarayx \mo vugaLu; kaSaTxniSuThxragaLu; kuMdukorategaLu: \eng{failure to respect seniority was a major grievance} seVvA hiritana uLaLxvarige purasAkxra koDuvudilalxveMba dUru (kelasagArara) kuMdukorategaLalilx oMdu muKayx aMshavAgide. 
\num{5} (yajamAnana vataRneya virudadhx kAmiRkara) asamAdhAna; dUru. 
\enum
\emng
\eentry

\bentry
\word[grieve(1)]{grieve}
\pron{girxVvf}
\gl{\sakirx}
\bmng
 duHKa, saMtApa -- uMTumADu; koragisu; aLalisu. 
\emng

\noindent
\gl{\akirx}
\bmng
 koragu; saMtApa, duHKa, vayxsana -- paDu; saMkaTapaDu. 
\emng
\eentry

\bentry
\word[grieve(2)]{grieve}
\pron{girxVvf}
\gl{\nA}
\bmng
 (sAkxTiSf \parx) (kaqSikeSxVtarx \mo vugaLa) meVlivxcAraka; maNegAra. 
\emng
\eentry

\bentry
\word{grievous}
\pron{girxVvasf}
\gl{\gu}
\bmng
\bnum
\num{1} keDukumADuva; bahaLa kaSaTx, toMdare, vipatutx, \mo vugaLanunxMTumADuva. 
\num{2} (noVvu \mo vugaLa \vi) tiVvarx; asahaniVya; tALalAgada; duBaRra. 
\num{3} dUSaNiVya; GoVra; paramapAtakada. 
\num{4} duHKakara. 
\enum
\emng

\noindent
\gl{\pagu}
\bmng
 \eng{grievous bodily harm} (\nAyxshA) tiVvarxvAda gAya, peTuTx, AGAta, deVhApAya. 
\emng
\eentry

\bentry
\word{grievously}
\pron{girxVvasfli}
\gl{\kirxvi}
\bmng
\bnum
\num{1} keDukanunxMTumADuva hAge; kaSaTxdAyakavAgi; vipatAkxraka riVtiyalilx. 
\num{2} tiVvarxvAgi; asahaniVyavAgi; duBaRra riVtiyalilx. 
\num{3} dUSaNiVya riVtiyalilx; GoVravAgi. 
\num{4} duHKakaravAgi. 
\enum
\emng
\eentry

\bentry
\word{griff}
\pron{girxphf}
\gl{\nA}
\bmng
 (\birx) (\ashi) (naMbalahaRvAda) sudidx; samAcAra. 
\emng
\eentry

\bentry
\word[griffin(1)]{griffin}
\pron{girxphinf}
\gl{\nA}
\bmng
 (\birx) (\ashi) (bAji \mo vugaLalilxna yA yAvudaradeV) suLivu; sUcane. 
\emng
\eentry

\bentry
\word[griffin(2)]{griffin}
\pron{girxphinf}
\gl{\nA}
\bmng
 girxphinf; gaqdharxsiMha; siMhahadudx; hadidxna tale, rekekxgaLu matutx siMhada oDalu uLaLxdedxMdu girxVkaru naMbidadx oMdu, pwrANika pArxNi. \imglink{griffinfigure}{\raisebox{-0.15cm}[0pt][0pt]{\pdfimage width 0.9cm height 0.5cm {G_Pictures/griffin.jpg}}} 
\emng
\eentry

\bentry
\word[griffon(1)]{griffon}
\pron{girxphanf}
\gl{\nA}
\bmng
\bnum
\num{1}  = \hyperlink{griffin(1)}{$^1$griffin}. 
\hypertarget{griffon(1)2}{} 
\num{2} `jipfsx' kulada oMdu bageya raNahadudx. 
\enum
\emng

\noindent
\gl{\pagu}
\bmng
 \eng{griffon vulture} = \hyperlink{griffon(1)2}{$^1$griffon(2)}. 
\emng
\eentry

\bentry
\word[griffon(2)]{griffon}
\pron{girxphanf}
\gl{\nA}
\bmng
 `Teriyarf' jAtiya nAyiyaMtha, oraTu kUdalina oMdu nAyi jAti. 
\emng
\eentry

\bentry
\word[grift(1)]{grift}
\pron{girxphfTx}
\gl{\nA}
\bmng
 (\ame) (\ashi) 
\bnum
\num{1} moVsa, vaMcane, \mo vugaLa mUlaka haNa paDeyuvudu. 
\num{2} moVsada saMpAdane; dagA haNa. 
\enum
\emng
\eentry

\bentry
\word[grift(2)]{grift}
\pron{girxphfTx}
\gl{\akirx}
\bmng
 (\ame) (\ashi) moVsa, vaMcane, \mo vugaLa mUlaka haNa saMpAdisu, paDe. 
\emng
\eentry

\bentry
\word{grifter}
\pron{girxphaTxrf}
\gl{\nA}
\bmng
 (\ame) (\ashi) 
\bnum
\num{1} (jUjinalilx) moVsagAra; dagAkoVra; vaMcaka; haNada jUjinalilx moVsamADuvavanu. 
\num{2} upaparxdashaRnakAra; sakaRsusx, saMte, \mo vugaLalilx upaparxdashaRnagaLanunx, \kanmu\ jUjATavanunx EpaRDisuvavanu. 
\enum
\emng
\eentry

\bentry
\word{grig}
\pron{girxgf}
\gl{\nA}
\bmng
\bnum
\num{1} saNaNx hAvumInu. 
\num{2} miDate yA cimamxMDe huLu. 
\enum
\emng

\noindent
\gl{\nuga}
\bmng
\hypertarget{Grig nuga1}{} 
\bnum
\num{1} \eng{lively as a grig} bahaLa ulAlxsadiMda; bahaLa KuSiyiMda; atayxMta caTuvaTikeyiMda. 
\num{2} \eng{merrily as a grig} = \hyperlink{Grig nuga1}{?nuga? \((1)\)}. 
\enum
\emng
\eentry

\bentry
\word{gri-gri}
\pron{girxVgirxV}
\gl{\nA}
\bmng
  = \hyperlink{greegree}{greegree}. 
\emng
\eentry

\bentry
\word[grill(1)]{grill}
\pron{girxlf}
\gl{\sakirx}
\bmng
\bnum
\num{1} (kabibxNada saraLucwkaTiTxna meVle) beVyisu; suDu. 
\num{2} (\rUpa) suDuvudariMda citarxhiMse koDu. 
\num{3} (\rUpa) suDu; tiVvarxshAKakekx, tApakekx gurimADu. 
\num{4} (\kanmu\ poliVsinavaru) bahuvAgi parxshenx keVLi piVDisu, hiMsisu, hiMsege gurimADu. 
\num{5} (siMpi \mo vanunx) cipipxnoMdige beVyisu. 
\enum
\emng

\noindent
\gl{\akirx}
\bmng
\bnum
\num{1} (kabibxNada saraLu cwkaTiTxna meVle) beVyu; suDu. 
\num{2} (\rUpa) suDuvudariMda citarxhiMsegoLagAgu. 
\num{3} suDu; tiVvarx shAKakekx, tApakekx -- oLagAgu: \eng{you were walking in the cool shadow of the woods, while I sat grilling} nAnu (bisiliniMda) suTuTxhoVgutatx kuLitidAdxga, niVnu kADina taMpu neraLinalilx naDeyutitxdedx. \eng{a grilling hot day} suDusuDu hagalu. 
\enum
\emng
\eentry

\bentry
\word[grill(2)]{grill}
\pron{girxlf}
\gl{\nA}
\bmng
\bnum
\num{1} (saraLukAvaliya meVle) beVyisida, suTuTx tayArisida -- AhAra, BakaSxyX. 
\hypertarget{grill(2)2}{} 
\num{2} mAMsada hoVLu \mo vanunx suTuTx baDisuva koVNe. 
\hypertarget{grill(2)3}{} 
\num{3} anwpacArika hoVTelu, resoTxrAMTu. 
\enum
\emng
\eentry

\bentry
\word[grill(3)]{grill}
\pron{girxlf}
\gl{\nA}
\bmng
\bnum
\num{1}  = \hyperlink{gridiron}{gridiron (1)}. 
\num{2} girxlulx; shAKavu keLaBAgakekx parxsAravAguvaMte kukakxrige aLavaDisiruva upakaraNa, \udA\ gAYxsf banaRru, hATf pelxVTu, \mo vu. 
\enum
\emng
\eentry

\bentry
\word[grill(4)]{grill}
\pron{girxlf}
\gl{\nA}
\bmng
  = \hyperlink{grille}{grille}. 
\emng
\eentry

\bentry
\word{grillage}
\pron{girxlijf}
\gl{\nA}
\bmng
 tolegaLa hadigaTuTx; husineladalilx kaTaTxDakekx taLahadiyAgiralu baLasuva aDaDxtolegaLa BAravAda cwkaTuTx. 
\emng
\eentry

\bentry
\word{grille}
\pron{girxlf}
\gl{\nA}
\bmng
\bnum
\num{1} kaMbi cwkaTuTx; saraLu jAlari; kaMbi tere; aDaDx kaMbigaLa mare; kaMbijAlari; \kanmu\ baMdavaranunx noVDalu anukUlisuvaMte, kAnevxMTugaLalilx kerxYsatx sanAyxsiniyaranunx BeVTigArariMda beVpaRDisalu bAgilige hAkiruva kaMbi, tere. 
\num{2} (hiMde hwsf Aphf kAmanfsx Bavanadalilx sitxrXVyara piVThasAlugaLa muMde hAkidadx) aDaDx kaMbi tere. 
\num{3} (Tenisf) (goVDeyalilxna cwkanAda) saMdu; kaMDi. 
\num{4} (mInugArike) mInu moTeTxgaLanunx marimADuva cwkaTuTx. 
\num{5} girxlulx; muMjAlari; moVTAru vAhanada reVDiyeVTaranunx rakiSxsalu adara muMde hAkiruva loVhada aDaDxkaMbigaLu. 
\enum
\emng
\eentry

\bentry
\word{grilled}
\pron{girxlfDx}
\gl{\gu}
\bmng
 beVyisida; suTaTx. 
\emng
\eentry

\bentry
\word{griller}
\pron{girxlarf}
\gl{\nA}
\bmng
\bnum
\num{1} (swTxnalilx beVyisuva, suDuva) jAlari. 
\num{2} beVyisuvavanu; aDigeyavanu. 
\enum
\emng
\eentry

\bentry
\word{grill-room}
\pron{girxlfrUmf}
\gl{\nA}
\bmng
\bnum
\num{1}  = \hyperlink{grill(2)2}{$^2$grill (2)}. 
\num{2}  = \hyperlink{grill(2)3}{$^2$grill (3)}. 
\enum
\emng
\eentry

\bentry
\word{grilse}
\pron{girxlfsx}
\gl{\nA}
\bmng
 girxlfsx; tananx jiVvanadalilx motatx modala bArige mari mADalu samudarxdiMda nadige hiMdiruguva eLeya vayasisxna sAmanf mInu. 
\emng
\eentry

\bentry
\word{grim}
\pron{girxmf}
\gl{\gu}
\bmng
\bnum
\num{1} kaThiNa; nidaRya; kUrxra: \eng{a grim battle} kUrxra kALaga. 
\num{2} urimoVreya, musuDiya; ugarxmuKada: \eng{a grim man loving duty more than humanity} mAnavakoVTigiMta tanage kataRvayxveV hececxnunxva urimusuDiyava. 
\num{3} (aneVkaveVLe maqtuyxvina \vi) karALa; rwdarx; GoVra. 
\num{4} aniSaTxkara; aniSaTxsUcaka: \eng{a grim threat} aniSaTxsUcaka bedarike. 
\num{5} kaThoVra; BayAnaka; BayaMkara; BiVkara; vikaTa; vikAravAda: \eng{has a grim truth in it} adaralilx oMdu kaThoVra satayxvide. \eng{a grim smile} vikAravAda husinage. 
\num{6} asahayxkara; ahitakaravAda. 
\enum
\emng

\noindent
\gl{\pagu}
\bmng
 \eng{hold on like grim death} GoVra maqtuyxvinaMte bigiyAgi, balavAgi hiDiduko. 
\emng
\eentry

\bentry
\word[grimace(1)]{grimace}
\pron{girxmeV(ma)sf}
\gl{\nA}
\bmng
\bnum
\num{1} (kirukuLadiMda Ada anuBava, jugupesx, \mo vanunx toVrisuva) soTaTx moVre; gaMTumoVre; siDimoVre. 
\num{2} (nagu huTiTxsuva) halulx kirita; muKa ceVSeTx; kapimUti. 
\num{3} husi muKaBAva; soVgina -- muKaBAva, muKacayeR. 
\num{4} soVgina, avalakaSxNada muKaBAva \mo vanunx toVrisuvudu: \eng{grace at Paris may appear grimace at London} pAYxrisisxnalilx lakaSxNa enisikoMDadudx laMDaninxnalilx avalakaSxNavAgi toVrabahudu. 
\enum
\emng
\eentry

\bentry
\word[grimace(2)]{grimace}
\pron{girxmeV(ma)sf}
\gl{\akirx}
\bmng
\bnum
\num{1} muKa soTaTxge mADiko, muri. 
\num{2} halulxkiri; aNakisu. 
\enum
\emng
\eentry

\bentry
\word{grimacer}
\pron{girxmeV(ma)sarf}
\gl{\nA}
\bmng
\bnum
\num{1} muKa soTaTxge mADuvavanu. 
\num{2} halulx kiriga; aNakisuvavanu. 
\enum
\emng
\eentry

\bentry
\word{grimalkin}
\pron{girxmAYx(mA)likxnf}
\gl{\nA}
\bmng
\bnum
\num{1} mudi (heNuNx) bekukx. 
\num{2} hoTeTxkicicxna muduki. 
\enum
\emng
\eentry

\bentry
\word[grime(1)]{grime}
\pron{gerxYmf}
\gl{\nA}
\bmng
 (meVlemxYyalilx, \kanmu\ camaRdalilx, hatitxkoMDiruva) ilalxNa; masi; koLe; kashamxla. 
\emng
\eentry

\bentry
\word[grime(2)]{grime}
\pron{gerxYmf}
\gl{\sakirx}
\bmng
\bnum
\num{1} masihacucx; kapupxbaLi. 
\num{2} koLe mADu; malina mADu; holegeDisu. 
\enum
\emng
\eentry

\bentry
\word{griminess}
\pron{gerxYminisf}
\gl{\nA}
\bmng
 masi, koLe, kashamxla -- hatitxruvudu; masi baLidiruvike. 
\emng
\eentry

\bentry
\word{grimly}
\pron{girxmfli}
\gl{\kirxvi}
\bmng
\bnum
\num{1} kaThiNavAgi; nidaRyavAgi; kUrxra riVtiyalilx. 
\num{2} uri muKadiMda; siDimoVreyiMda: \eng{that man who receives you so grimly} ninanxnunx urimuKadiMda edurugoLuLxva A vayxkitx. 
\num{3} BayAnakavAgi; GoVravAgi; vikAravAgi: \eng{faces grimly tattooed} vikAravAgi hacecx hoyadx muKagaLu. 
\num{4} kaThinavAgi; niSuThxravAgi; tiVvarxvAgi. 
\enum
\emng
\eentry

\bentry
\wordRemoveSpace{Grimm's-law}{Grimm's law}
\pron{girxmfsx lA}
\gl{\nA}
\bmng
 girxmf sUtarx, niyama; (aitihAsika BASAvijAcnxnadalilx) jamAYxRnikf BASegaLalilx padagaLu oMdu BASeyiMda inonxMdakekx baruvAga Aguva vayxMjana vayxtAyxsagaLanunx kurita niyama. 
\emng
\eentry

\bentry
\word{grimness}
\pron{girxmfnisf}
\gl{\nA}
\bmng
\bnum
\num{1} kaThiNate; kaThoVrate; nidaRyate; kwrxyaR. 
\num{2} vikaqtavAda, ugarxvAda rUpa iruvike; karALa savxrUpa. 
\num{3} BayAnakatavx; BayaMkaratavx. 
\num{4} vikaTatavx; vikAratavx. 
\num{5} niSuThxrate; tiVvarxte. 
\enum
\emng
\eentry

\bentry
\word{grimy}
\pron{gerxYmi}
\gl{\gu}
\bmng
 (meVlemxYyalilx, \kanmu\ camaRda meVle) ilalxNa, koLe, kashamxla -- hatitxruva; koLeyAda; masibaLidiruva; masiyAda; malinavAda. 
\emng
\eentry

\bentry
\word[grin(1)]{grin}
\pron{girxnf}
\gl{\kirx}
\expl{(\BU\ matutx \BUkaq\ \eng{grinned,} \vakaq\ \eng{grinning}).}


\noindent
\gl{\sakirx}
\bmng
 (tirasAkxra, samAdhAna, mecucxge, \mo vanunx) halulx kiridu sUcisu, vayxkatxpaDisu: \eng{the surgeon grinned approbation} shasatxrXveYdayxnu halulx kiridu mecucxge vayxkatxpaDisida. 
\emng

\noindent
\gl{\akirx}
\bmng
 (noVviniMda, balAtAkxradiMda, balavaMtavAda yA pedudxtanada naguviniMda) halulxbiDu; halulxkiri. 
\emng

\noindent
\gl{\pagu}
\bmng
\hyperdef{G}{grin(1) pagu}{} \eng{grin through a horse -- collar} (haLiLxya ATada paMdayxdalilx) halulx kiriyuva jUjATavADu. 
\emng

\noindent
\gl{\nuga}
\bmng
\bnum
\num{1} \eng{grin and bear it} yatanxvilalx, tALiko; beVreVnU upAyavilalxdadxriMda sahisiko. 
\num{2} \eng{grin like a} \hyperref{kandict_c.pdf}{C}{Cheshire cat nuga}{cheshire cat}. 
\enum
\emng
\eentry

\bentry
\word[grin(2)]{grin}
\pron{girxnf}
\gl{\nA}
\bmng
 halulxkirita; halulx kiriyuvudu. 
\emng

\noindent
\gl{\pagu}
\bmng
\hypertarget{Grin pagu1}{} 
\bnum
\num{1} \eng{on the broad grin} doDaDxdAgi halulxbiTuTx, halulxkiridu. 
\num{2} \eng{on the grin} = \hyperlink{Grin pagu1}{?pagu? \((1)\)}. 
\enum
\emng
\eentry

\bentry
\word[grind(1)]{grind}
\pron{gerxYnfDx}
\gl{\kirx}
\bmng
(\BU\ matutx \BUkaq\ \eng{ground}).
\emng

\noindent
\gl{\sakirx}
\bmng
\bnum
\num{1} (biVsuva kalulx, halulx, \mo vugaLiMda) puDi yA hiTuTx mADu; are. 
\num{2} kADu; piVDisu; hiMsisu; upadarxva koDu; aredu biDu; sulidu, tera horisi toMdarepaDisu: \eng{grinding tyranny} piVDaka parxButavx; aredubiDuva dabAbxLike. 
\num{3} (biVsi) hiTuTxmADu. 
\num{4} mase; sANehiDi; ujijx nayamADu yA cUpu mADu: \eng{grind diamonds} vajarxgaLanunx ujijx nayamADu. 
\num{5} (biVsuva kalalxnunx) biVsu; tirugisu. 
\num{6} (`haDiRguDiR' \mo\ vAdayxgaLa) hiDi tirugisu. 
\num{7} (`haDiRguDiR', bAYxralf AgaRnf, \mo\ vAdayxgaLanunx) nuDisu. 
\num{8} (pATha \mo vanunx vidAyxthiRge) aredu hoyiyx; kaSaTxpaTuTx kalisu: \eng{after grinding him in Greek and Latin} avanige girxVkf matutx lAYxTinf aredu hoyadx meVle. 
\num{9} (kiruguTuTxvaMte) ujujx; tikukx; mase; tiVDu (\akirx\ saha): \eng{the ship was grinding on the rocks} haDagu baMDegaLige tiVDutitxtutx, maseyutitxtutx. 
\num{10} kaTakaTane (halulx) mase, kaDi. 
\enum
\emng

\noindent
\gl{\akirx}
\bmng
\bnum
\num{1} (biVsi, aredu) puDimADu; hiTuTx mADu. 
\num{2} EkaparxkAravAgi, oMdeV samane -- duDi, heNagu, kaSaTxpaDu: \eng{our fellows grind on the river or in the gymnasium} namamxvaru nadiyalolxV garaDimaneyalolxV oMdeV samane heNagutAtxre. 
\num{3} (guruvina baLi) kaSaTxpaTuTx Odu, videyx kali: \eng{after grinding with tutors} upAdhAyxyara baLi kaSaTxpaTuTx OdiyAda meVle. 
\enum
\emng

\noindent
\gl{\nuga}
\bmng
\bnum
\num{1} \eng{an} \hyperref{kandict_a.pdf}{A}{axe(1) nuga(1)}{axe to grind.} 
\num{2} \eng{grind out an oath} halulxhalulxkaDiyutatx shapisu, shapatha mADu. 
\num{3} \eng{grind the faces of the poor} baDavaranunx -- tuLi, aredubiDu, piVDisu. 
\enum
\emng
\eentry

\bentry
\word[grind(2)]{grind}
\pron{gerxYnfDx}
\gl{\nA}
\bmng
\bnum
\num{1} biVsuvudu. 
\num{2} ujujxvudu; tikukxvudu; sANe hiDiyuvudu; areta; maseta. 
\num{3} puDi dapapx; biVsida, puDimADida kaNagaLa gAtarx. 
\num{4} kaSaTxvAda, beVsara hiDisuva, oMdeV tarada -- kelasa, duDime. 
\num{5} vAyxyAma saMcAra; vAyxyAmakAkxgi naDeyuvudu. 
\num{6} (haLaLxkoLaLx, beVli, \mo vanunx hAri ODabeVkAda) kudureya paMdayx, jUju. 
\num{7} (keVMbirxjf) doVNi dATu; kaDavu. 
\num{8} (\ame) adhayxvasAyi; hagalU iruLU kaSaTxpaTuTx Oduva vidAyxthiR. 
\num{9} (\ashi) saMBoVga; meYthuna. 
\num{10} (\ashi) nitaMba BarxmaNa; kuMDe tirugaNe; naqtayxdalilx nataRkanu (yA nataRkiyu) nitaMbagaLanunx sutatxlU tirugisuvudu. 
\enum
\emng

\noindent
\gl{\nuga}
\bmng
 \eng{the daily grind} (\AmA) obabxna -- dinaMparxtiya duDime; deYnaMdina kelasa kAyaR. 
\emng
\eentry

\bentry
\word{grinder}
\pron{gerxYnaDxrf}
\gl{\nA}
\bmng
\bnum
\num{1} davaDehalulx; areyuva halulx. 
\num{2} gerxYMDaru; biVsuva, hiTuTxmADuva, sANe hiDiyuva yaMtarx. 
\num{3} biVsuva kalilxna meVlina hoVLu. 
\num{4} (\kanmu\ samAsagaLalilx) biVsuvavanu yA sANegAra: \eng{knife grinder} cUri sANe hiDiyuvava. \eng{organ grinder} haDiRguDiR \mo\ AgaRnf vAdayxvanunx biVdiyalilx saMpAdanegAgi nuDisuvava. 
\num{5} pariVkeSxge tayAri koDuvavanu; uruhacicxsuvavanu; uruhoDesuvavanu; bAyipATha kalisuvavanu; kaMThapATha mADisuvavanu; gaTiTx mADisuvavanu. 
\num{6} (\ame) kaSaTxpaTuTx duDiyuva vidAyxthiR. 
\enum
\emng
\eentry

\bentry
\word{grindery}
\pron{gerxYnaDxri}
\gl{\nA}
\bmng
 (\birx) 
\bnum
\num{1} joVDu holiyuvavana, camAmxrana kAyaRgaLu. 
\num{2} camAmxrana, samagArana elalx bageya salakaraNegaLu. 
\num{3} sANe mane; maseyeDe; (Ayudha \mo vanunx) sANe hiDiyuva sathxLa. 
\enum
\emng
\eentry

\bentry
\word{grind-stone}
\pron{gerxYnfDxsoTxVnf}
\gl{\nA}
\bmng
\bnum
\num{1} areyuva, biVsuva kalulx. 
\num{2} merugukalulx; merugukoDuva sANe kalulx. 
\num{3} (haritagoLisuva) maseyuva kalulx; masegalulx; sANekalulx. 
\num{4} sANeshile; sANekalulx \mo vugaLigAgi baLasuva kalulxjAti. 
\enum
\emng

\noindent
\gl{\nuga}
\bmng
 \eng{hold} (\engit{or} \eng{keep) one's nose to the grind-stone} sadA duDisu; biDuvilalxdaMte kelasa mADisu. 
\emng
\eentry

\bentry
\word{gringo}
\pron{girxMgoV}
\gl{\nA}
\expl{(\bava\ \eng{gringos}).}
\bmng
(sApxYxniSf amerikanf deVshagaLalilxna \parx) videVshiVya; horaginavanu (\kanmu\ birxTanf yA amerikadavanu). 
\emng
\eentry

\bentry
\word[grip(1)]{grip}
\pron{girxpf}
\gl{\nA}
\bmng
\bnum
\num{1} (bigi) hiDita; BadarxvAgi hiDidiruvudu; balivxDi (\gArx). 
\num{2} hiDiyuva shakitx; garxhaNashakitx. 
\num{3} keYkulukuva riVti. 
\num{4} (EnanAnxdarU) hiDidukoLuLxva riVti; keY hiDitada varase, paTuTx: \eng{overlapping grip} mucucx hiDita; (gAlfphx ATadalilx balageY kiruberaLu eDageY toVruberaLanunx mucucxvaMte dAMDanunx hiDiyuvudu). 
\num{5} (\rUpa) hiDita; sAvxdhiVna; hatoVTi; vasha; vajarxmuSiTx: \eng{unable to escape the grip of his old habits} avana haLeya aBAyxsagaLa vajarxmuSiTxyiMda tapipxsikoLaLxlArade. 
\num{6} (bwdidhxka) garxhike; garxhaNa sAmathayxR, shakitx; viSayavoMdanunx aritukoLuLxva yA sAvxdhiVnapaDisikoLuLxva shakitx: \eng{wanting in intellectual grip} bwdidhxka garxhike sAlada. 
\num{7} AkaSaRNe; gamanavanunx seLediDuva shakitx. 
\num{8} (yaMtarx \mo vugaLalilxruva) kacucxva BAga. 
\num{9} (Ayudha \mo vugaLa) hiDi; hiDike; hiDiya BAga. 
\num{10} (\ame) (parxyANikana) hiDiciVla; keYciVla. 
\num{11} = \hyperref{kandict_h.pdf}{H}{hair-grip}{hair-grip}. 
\enum
\emng

\noindent
\gl{\pagu}
\bmng
\bnum
\num{1} \eng{get a grip on oneself} tananx meVle hatoVTi iTuTxko; saMyama sAdhisu; ecacxra vahisu. 
\num{2} \eng{lengthen one's grip} bAyxTu, doNeNx, \mo vanunx meVle hiDiduko; avugaLa hoDeyuva tudiyiMda dUradalilx hiDiduko. 
\num{3} \eng{lose one's grip} bAyxTu, doNeNx, \mo vanunx keLage hiDiduko; avugaLa hoDeyuva tudige hatitxradalilx hiDiduko. 
\num{4} \eng{shorten one's grip} (yAvudaradeV meVle) hatoVTi kaLeduko. 
\enum
\emng

\noindent
\gl{\nuga}
\bmng
\hypertarget{grip nuga1}{} 
\bnum
\numi{1} \eng{at grips with} 
\banum
\alnum{a} muKAmuKiyAgi; keY keY milAyisi; malAlxmalilxyAgi. 
\alnum{b} (viSaya, samaseyxyanunx) neVravAgi yA daqDhavAgi edurisi. 
\eanum
\numie
\num{2} \eng{come} (\engit{or} \eng{get) to grips with} = \hyperlink{grip nuga1}{?nuga? \((1)\)}. 
\enum
\emng
\eentry

\bentry
\word[grip(2)]{grip}
\pron{girxpf}
\gl{\kirx}
\bmng
(\BU\ matutx \BUkaq\ \eng{gripped,} \vakaq\ \eng{gripping}).
\emng

\noindent
\gl{\sakirx}
\bmng
\bnum
\num{1} (BadarxvAgi, bigiyAgi, balavAgi) hiDi; hiDiduko. 
\num{2} (kokukx, mUti, halulx, \mo vugaLiMda) kacicx hiDi. 
\num{3} AkaSiRsu; gamana seLe; manasasxnunx hiDidiDu: \eng{the pathos of the play gripped the beholders} nATakada karuNarasa perxVkaSxkara gamanavanunx seLediTiTxtu. 
\enum
\emng

\noindent
\gl{\akirx}
\bmng
 bigihiDi; kacucx: \eng{the anchor grips} laMgaru kacucxtatxde, hiDidukoLuLxtatxde. 
\emng
\eentry

\bentry
\word[gripe(1)]{gripe}
\pron{gerxYpf}
\gl{\sakirx}
\bmng
\bnum
\num{1} BadarxvAgi hiDi(duko); bigi hiDi. 
\num{2} hiMsisu; upadarxva koDu; piVDisu. 
\num{3} (\nw) (doVNiyanunx hagagxgaLiMda) kaTuTx; bigi. 
\num{4} hoTeTxnulita taru; hoTeTxshUle, jaTharashUle -- uMTumADu. 
\enum
\emng

\noindent
\gl{\akirx}
\bmng
\bnum
\num{1} BadarxvAgi hiDi; bigihiDi. 
\num{2} hoTeTx nulitavanunx, jaTharashUleyanunx -- anuBavisu. 
\num{3} (haDagu) cukAkxNi tirugisidarU gALiya kaDege baru. 
\num{4} (\ashi) dUru; goNagu: \eng{people are always griping about kids hanging around the wrong places} irabArada sathxLadalilx suLidADuva makakxLa bagegx jana yAvAgalU dUrutitxdAdxre. 
\enum
\emng
\eentry

\bentry
\word[gripe(2)]{gripe}
\pron{gerxYpf}
\gl{\nA}
\bmng
\bnum
\num{1} bigi hiDiyuvudu; bigihiDita. 
\num{2} (\pArxparx) hiDita; vasha; sAvxdhiVna. 
\num{3} (\bava dalilx) hoTeTxshUle; jaTharashUle; hoTeTxnulita. 
\num{4} (Ayudha \mo vugaLa \vi) hiDi; hiDike. 
\num{5} (\nw) (\bava dalilx) (doVNiyanunx adara sAthxnadalilx) bigidu nililxsuva hagagxgaLu. 
\num{6} (\ashi) dUru. 
\enum
\emng

\noindent
\gl{\pagu}
\bmng
 \eng{in the gripe of} hiDitadalilx; adhiVnadalilx. 
\emng
\eentry

\bentry
\word{gripe-water}
\pron{gerxYpfvATarf}
\gl{\nA}
\bmng
 gerxYpfvATaru; makakxLa vAtahara auSadhi. 
\emng
\eentry

\bentry
\word{grippe}
\pron{girxpf}
\gl{\nA}
\bmng
 `infPulxyenAs'' roVga; phUlx. 
\emng
\eentry

\bentry
\word{gripper}
\pron{girxparf}
\gl{\nA}
\bmng
\bnum
\num{1} hiDike; yAvudanAnxdarU BadarxvAgi hiDidukoLaLxlu baLasuva koMDi, kalxcucx, \mo\ yAvudeV sAdhana. 
\num{2} hiDike tayArisuvava. 
\enum
\emng
\eentry

\bentry
\word{gripsack}
\pron{girxpfsAYxkf}
\gl{\nA}
\bmng
 (\ame) parxyANada ciVla yA sUTfkeVsu. 
\emng
\eentry

\bentry
\word{grisaille}
\pron{girxseY(seZV)lf}
\gl{\nA}
\bmng
\banum
\alnum{a} ububxshilapxvanunx. hoVluvaMte EkavaNaRdalilx (\sA\ bUdu baNaNxdalilx) citarxgaLanunx racisuva vidhAna. 
\alnum{b} I vidhAnadalilx racisida vaNaRcitarx. 
\eanum
\emng
\eentry

\bentry
\word{griseofulvin}
\pron{girxsiZOphulivxnf}
\gl{\nA}
\bmng
 girxsiyoVphulivxnf; huLukaDiDxya cikitesxge baLasuva oMdu parxti-jiVvaka, AyxMTibayATikf, \eng{$\bg\rm C\eg_\bg 17\eg\bg\rm H\eg_\bg 17\eg\bg\rm O\eg_6\bg\rm CL\eg$}. 
\emng
\eentry

\bentry
\word{griseous}
\pron{girxsiZasf}
\gl{\gu}
\bmng
 (\jiVvi) nasuniVli yA bUdu(baNaNxda). 
\emng
\eentry

\bentry
\word{grisette}
\pron{girxseZTf}
\gl{\nA}
\bmng
 (hiMdina kAladalilx bUdu vasatxrX dharisutitxdadx) pherxMcf kAmiRkavagaRda huDugi. 
\emng
\eentry

\bentry
\word{gris-gris}
\pron{girxsfgirxsf}
\gl{\nA}
\bmng
  = \hyperlink{greegree}{greegree}. 
\emng
\eentry

\bentry
\word{griskin}
\pron{girxsikxnf}
\gl{\nA}
\bmng
 haMdiya ToMkada mAMsa (\kanmu\ neNavilalxdudx). 
\emng
\eentry

\bentry
\word{grisly}
\pron{girxsfZli}
\gl{\gu}
\bmng
\bnum
\num{1} GoVra; BayaMkara; BayAnaka; BiVkara; bahuBiVtiyanunxMTumADuva. 
\num{2} (mUDha) Baya huTiTxsuva. 
\enum
\emng
\eentry

\bentry
\word{grison}
\pron{girxsaZnf}
\gl{\nA}
\bmng
 viVsalf kulada oMdu mAMsAhAri pArxNi. 
\emng
\eentry

\bentry
\word{grissini}
\pron{girxsiVniV}
\gl{\nA}
\bmng
 (\bava) garigari berxDuDx kaDiDxgaLu; udadxvAda, teLuvAda kaDiDxgaLa AkAradalilx mADida garigariyAda berxDuDx. 
\emng
\eentry

\bentry
\word[grist(1)]{grist}
\pron{girxsfTx}
\gl{\nA}
\bmng
\bnum
\num{1} biVsuva kALu, dhAnayx. 
\num{2} (sArAyi mADalu arediTiTxruva) moLe(yisida) dhAnayx. 
\enum
\emng

\noindent
\gl{\nuga}
\bmng
\bnum
\num{1} \eng{all is grist that comes to his mill} avanu keYge sikikxdadxnenxlalx sAvxhA mADutAtxne; tanage parxtikUlavAdudanunx anukUlakAkxgiyeV baLasikoLuLxtAtxne. 
\num{2} \eng{brings grist to the mill} (vAyxpAra, kasabu, \mo vugaLa \vi) lABa tarutatxde; lABadAyakavAgutatxde. 
\enum
\emng
\eentry

\bentry
\word[grist(2)]{grist}
\pron{girxsfTx}
\gl{\nA}
\bmng
 (nUlina, hagagxda) gAtarx; dapapx. 
\emng
\eentry

\bentry
\word{gristle}
\pron{girxsflf}
\gl{\nA}
\bmng
= \hyperref{kandict_c.pdf}{C}{cartilage}{cartilage}. 
\emng

\noindent
\gl{\pagu}
\bmng
 \eng{in the gristle} eLeya; baliyada; mULeya badalu maqdavxsithxyuLaLx. 
\emng
\eentry

\bentry
\word{gristly}
\pron{girxsfli}
\gl{\gu}
\bmng
 maqdavxsithxyiMda kUDida. 
\emng
\eentry

\bentry
\word[grit(1)]{grit}
\pron{girxTf}
\gl{\nA}
\bmng
\bnum
\num{1} (\kanmu\ toMdare koDuva yA yaMtArxdigaLalilx aDacikoLuLxva) kalilxna yA maraLina saNaNx kaNagaLu. 
\hypertarget{grit(1)2}{} 
\num{2} oraTu maraLugalulx. 
\num{3} kalilxna eLe yA racane. 
\num{4} (\AmA) (cAritarxyXda, shiVlada) dADhayxR; sethxYyaR. 
\num{5} edegArike; kececxde; diTaTxtana; CAti. 
\num{6} (kaSaTx) sahiSuNxte; tALike. 
\enum
\emng
\eentry

\bentry
\word[grit(2)]{grit}
\pron{girxTf}
\gl{\kirx}
\expl{(\BU\ matutx \BUkaq\ \eng{gritted,} \vakaq\ \eng{gritting}).}


\noindent
\gl{\sakirx}
\bmng
\bnum
\num{1} (halulx) mase; kaDi. 
\num{2} (maMjugaDeDxya rasetx \mo vugaLa meVle) kalilxna, maraLina saNaNx kaNagaLanunx haraDu. 
\enum
\emng

\noindent
\gl{\akirx}
\bmng
\bnum
\num{1} karakaraguTuTx; karakara shabadx mADu. 
\num{2} karakaraguTuTxtatx -- sAgu, hoVgu, calisu. 
\enum
\emng
\eentry

\bentry
\word{grits}
\pron{girxTfsx}
\gl{\nA}
\bmng
 (\bava) 
\bnum
\num{1} (tavaDu yA hoTuTx tegeda, hiTuTx mADada) `OTfsx' kALu. 
\num{2} `OTfsx' tari. 
\enum
\emng
\eentry

\bentry
\word{grit-stone}
\pron{girxTfsoTxVnf}
\gl{\nA}
\bmng
  = \hyperlink{grit(1)2}{$^1$grit (2)}. 
\emng
\eentry

\bentry
\word{grittiness}
\pron{girxTinisf}
\gl{\nA}
\bmng
\bnum
\num{1} kalilxna, maraLina kaNagaLiruva sithxti. 
\num{2} (shiVlada) dADhayxR; sethxYyaR. 
\num{3} diTaTxtana; CAti. 
\num{4} (kaSaTx) sahiSuNxte. 
\enum
\emng
\eentry

\bentry
\word{gritty}
\pron{girxTi}
\gl{\gu}
\bmng
\bnum
\num{1} kalilxna, maraLina -- saNaNx kaNagaLa, kaNagaLiruva. 
\num{2} (kalilxna, maraLina) saNaNx kaNagaLaMtiruva, avugaLanunx hoVluva. 
\num{3} (shiVlada) dADhayxRda; sethxYyaRvuLaLx. 
\num{4} kececxdeyuLaLx; diTaTx; CAtiyuLaLx. 
\num{5} (kaSaTx)sahiSuNxvAda; sahisuva. 
\enum
\emng
\eentry

\bentry
\word{grizzle}
\pron{girxsfZlf}
\gl{\akirx}
\bmng
 (\birx) (\AmA) (\kanmu\ makakxLu bikakxLisi, gadagxdisi, kuyoyxV eMdu) aLu; ragaLemADu. 
\emng
\eentry

\bentry
\word{grizzled}
\pron{girxsfZlfDx}
\gl{\gu}
\bmng
\bnum
\num{1} nareta; nare yA bUdu baNaNxda. 
\num{2} naregUdalina. 
\enum
\emng
\eentry

\bentry
\word[grizzly(1)]{grizzly}
\pron{girxsfZli}
\gl{\gu}
\bmng
\bnum
\num{1} nasubUdubaNaNxda; nare baNaNxda. 
\num{2} naregUdalina. 
\enum
\emng
\eentry

\bentry
\word[grizzly(2)]{grizzly}
\pron{girxsfZli}
\gl{\nA}
\bmng
  = \hyperlink{grizzly bear}{grizzly bear}. 
\emng
\eentry

\bentry
\wordnospeech{grizzly bear}{grizzly bear}
\pron{?}
\gl{\nA}
\bmng
bUdukaraDi; amerikada BayaMkaravAda oMdu doDaDx karaDijAti. \imglink{grizzly-bearfigure}{\raisebox{-0.15cm}[0pt][0pt]{\pdfimage width 0.8cm height 0.6cm {G_Pictures/grizzly-bear.jpg}}} 
\emng
\eentry

\bentry
\word[groan(1)]{groan}
\pron{gorxVnf}
\gl{\sakirx}
\bmng
 mulukutatx, naraLutatx, koragutatx -- heVLu, ADu, vayxkatxpaDisu: \eng{will groan out some prayer} naraLutAtx (EnanonxV) moreyiDutAtxne. 
\emng

\noindent
\gl{\akirx}
\bmng
\bnum
\num{1} mulugu; naraLu; vayxthepaDu; saMkaTapaDu; koragu; halubu; (noVvu, duHKa, asamamxti, \mo vanunx sUcisuva) ALavAda, asapxSaTxvAda dhavxni mADu: \eng{groaned with rage and frustration} korxVdha hAgU AshABaMgadiMda naraLida. 
\num{2} piVDitanAgiru; saMkaTapaDu; horalArade naraLu: \eng{groan under injustice} anAyxyadiMda jajaRritanAgi naraLu. \eng{shelf groans with books} baDu pusatxkada horeyiMda naraLutitxde; baDuvina meVle horalAradaSuTx pusatxkagaLive. 
\num{3} (\pArxparx) yAvudAdarU vasutxvigAgi -- haMbalisu, atAyxsheyiMda apeVkiSxsu, bAyibAyibiDu, tahatahisu: \eng{groaning to be with her again} punaH avaLoMdige irabeVkeMdu haMbalisutAtx. 
\enum
\emng

\noindent
\gl{\pagu}
\bmng
\bnum
\num{1} \eng{groan down} (naraLATada, ataqpitx sUcakavAda kUgugaLiMda BASaNakAra \mo varanunx) bAyimucicxsu; sumamxnAgisu: \eng{the consuls were groaned down} kAnasxlfranunx (naraLikeya kUginiMda) bAyi mucicxsidaru. 
\num{2} \eng{groaning board} tinisugaLu iDi kirida UTada meVju. 
\num{3} \eng{groan inwardly} oLagoLageV -- koragu, vayxthepaDu. 
\enum
\emng
\eentry

\bentry
\word[groan(2)]{groan}
\pron{gorxVnf}
\gl{\nA}
\bmng
 muluku; naraLATa; naraLike; saMkaTadhavxni; koragina kUgu. 
\emng
\eentry

\bentry
\word{groaningly}
\pron{gorxVniMgfli}
\gl{\kirxvi}
\bmng
 mulukutAtx; naraLutAtx; vayxthepaDutAtx; koraginalilx; saMkaTadiMda; halubutAtx. 
\emng
\eentry

\bentry
\word{groat}
\pron{gorxVTf}
\gl{\nA}
\bmng
 (\ca) 
\bnum
\num{1} (iMgelxMDinalilx \eng{1351--1662}ravarege calAvaNeyalilxdadx. \eng{4} peni beleya) oMdu beLiLxya nANayx. 
\num{2} (\pArxparx) alapx motatx. 
\enum
\emng

\noindent
\gl{\pagu}
\bmng
 \eng{don't care a groan} oMdu kavaDeyaSUTx lakaSxyX mADe. 
\emng
\eentry

\bentry
\word{groats}
\pron{gorxVTfsx}
\gl{\nA}
\bmng
 (\bava) hoTuTx kaLeda (kelavu sala nucucx mADida) dhAnayx (\kanmu\ OTfsx). 
\emng
\eentry

\bentry
\word{Grobian}
\pron{gorxVbianf}
\gl{\nA}
\bmng
 oraToraTAda, nayanAjUkilalxda manuSayx. 
\emng
\eentry

\bentry
\word{grocer}
\pron{gorxVsarf}
\gl{\nA}
\bmng
 kirANi vAyxpAri; panasAre, palasaraku vAyxpAri; jinasi, dinasi -- vAyxpAri; saMbAra, jinasi, oNahaNuNxgaLu, beNeNx, TiV, hiTuTx, sakakxre, \mo\ gaqhakaqtayxda sAmAnugaLanunx mAruvavanu. 
\emng

\noindent
\gl{\pagu}
\bmng
 \eng{grocer's itch} sakakxre isabu; sakakxreyanunx muTATxDuvudariMda baruva -- karapANi, isabu. 
\emng
\eentry

\bentry
\word{grocery}
\pron{gorxVsari}
\gl{\nA}
\bmng
\bnum
\num{1} (\sA\ \bava dalilx) palasaraku; kirANi sAmAnu. 
\num{2} kirANi vAyxpAra. 
\num{3} (\ame) dinasi, kirANi -- aMgaDi. 
\enum
\emng
\eentry

\bentry
\word{groceteria}
\pron{gorxVseTiaria}
\gl{\nA}
\bmng
 (\ame) savxyaMseVveya dinasi yA kirANi aMgaDi. 
\emng
\eentry

\bentry
\word[grog(1)]{grog}
\pron{gArxgf}
\gl{\nA}
\bmng
\bnum
\num{1} gArxgf (madayx); niVru beresida madayx. 
\num{2} (\AseTxrXV\ matutx nUyxsiZVlaMDf) biyarU seVridaMte yAvudeV madayx. 
\num{3} gArxgf (saMtoVSa) kUTa; `gArxgf' kuDitada kUTa. 
\enum
\emng
\eentry

\bentry
\word[grog(2)]{grog}
\pron{gArxgf}
\gl{\kirx}
\expl{(\BU\ matutx \BUkaq\ \eng{grogged,} \vakaq\ \eng{grogging}).}
\bmng
(\sakirx\ madayxda KAli piVpAyige) bisiniVru hAkiTuTx mara hiVrikoMDidadx madayx saMgarxhisu: \eng{some traders grog the empty casks} kelavu vAyxpArigaLu KAli piVpAyiyiMda `gArxgf' saMgarxhisutAtxre. 
\emng

\noindent
\gl{\akirx}
\bmng
 `gArxgf' madayx kuDi. 
\emng
\eentry

\bentry
\word{grog-blossom}
\pron{gArxgfbAlxsamf}
\gl{\nA}
\bmng
 gArxgf moDave; ati kuDitadiMda mUgina meVle kANisikoLuLxva moDave yA keMpu raMgu. 
\emng
\eentry

\bentry
\word{grogginess}
\pron{gArxginisf}
\gl{\nA}
\bmng
\bnum
\num{1} (\pArxparx) matutx; amalu; kuDida, amaleVrida sithxti. 
\num{2} nishayxkitx; nitArxNasithxti. 
\num{3} tatatxrike; asithxrate; tUrATa. 
\enum
\emng
\eentry

\bentry
\word{groggy}
\pron{gArxgi}
\gl{\gu}
\bmng
\bnum
\num{1} (\pArxparx) kuDida; amaleVrida. 
\num{2} (\pArxparx) kuDitakekx bidadx. 
\num{3} (kudureya \vi) muMgAlugaLalilx tArxNavilalxda. 
\num{4} tatatxrisuva; tUrADuva. 
\enum
\emng
\eentry

\bentry
\word{grogram}
\pron{gArxgarxmf}
\gl{\nA}
\bmng
 gArxgarxmf (baTeTx); goVMdiniMda peDasugoLisida, oraTAda reVSemxya, ADina kUdalina uNeNxya yA elalxvU misharxvAgiruva baTeTx. 
\emng
\eentry

\bentry
\word[groin(1)]{groin}
\pron{gArxinf}
\gl{\nA}
\bmng
\bnum
\num{1} (\aMrashA) toDesaMdu; gejejx; kibobxTeTxya taLaBAgakUkx toDeya meVlABxgakUkx iruva haLaLx, maDike. 
\num{2} (\vAshi) (cAvaNi kamAnugaLa) kUDaMcu; eraDu cAvaNi kamAnugaLu oMdakokxMdu saMdhisi Ada aMcu.  \imglink{groin2-3figure}{\raisebox{-0.15cm}[0pt][0pt]{\pdfimage width 0.5cm height 0.6cm {G_Pictures/groin2-3.jpg}}} 
\num{3} (\vAshi) kamAnu cAvaNige AdhAravAda kamAnu. 
\enum
\emng
\eentry

\bentry
\word[groin(2)]{groin}
\pron{gArxinf}
\gl{\sakirx}
\bmng
 (\vAshi) kUDukamAnanunx iTuTx cAvaNi kaTuTx, racisu. 
\emng
\eentry

\bentry
\word[groin(3)]{groin}
\pron{gArxinf}
\gl{\nA}
\bmng
 (\ame)  = \hyperlink{groyne(1)}{$^1$groyne}. 
\emng
\eentry

\bentry
\word[groin(4)]{groin}
\pron{gArxinf}
\gl{\sakirx}
\bmng
  = \hyperlink{groyne(2)}{$^2$groyne}. 
\emng
\eentry

\bentry
\word{groining}
\pron{gArxiniMgf}
\gl{\nA}
\bmng
 (\vAshi) kUDu kamAnanunx iTuTx cAvaNi kaTuTxvudu; kUDu kamAnina cAvaNi racane. 
\emng
\eentry

\bentry
\word{grommet}
\pron{gArxmiTf}
\gl{\nA}
\bmng
  = \hyperlink{grummet}{grummet}. 
\emng
\eentry

\bentry
\word{gromwell}
\pron{gArxmfvelf}
\gl{\nA}
\bmng
 (\savi) gArxmevxlf; litosapxmaRmf kulakekx seVrida, kalilxnaMtha biVjagaLiruva, hiMde auSadhiyalilx baLasutitxdadx, oMdu bageya giDa. 
\emng
\eentry

\bentry
\word[groom(1)]{groom}
\pron{gUrxmf}
\gl{\nA}
\bmng
\bnum
\num{1} (\birx) (iMgelxMDinalilx) aramaneya ADaLitada obabx adhikAri. 
\num{2} kAsadAra; kudureya ALu; sAhaNi; ashavxpAla; kuduregaLanunx noVDikoLuLxvavanu. 
\num{3} maduvaNiga; madumaga. 
\num{4} (\pArxparx) seVvaka; gaMDALu. 
\enum
\emng

\noindent
\gl{\pagu}
\bmng
\bnum
\num{1} \eng{Groom in waiting} (\birx) (rAjana parijanaralilx) obabx adhikAri. 
\num{2} \eng{Groom of the stole} (\birx) iMgelxMDina aramaneyalilx vasAtxrXlaMkAra, sejejx, \mo vanunx noVDikoLuLxva adhikAri; shayAyxdhikAri; sejejx adhikAri. 
\enum
\emng
\eentry

\bentry
\word[groom(2)]{groom}
\pron{gUrxmf}
\gl{\sakirx}
\bmng
\bnum
\num{1} (kudureyanunx) noVDiko; kudureya usutxvAri vahisu; kudurege mAliVsu mADu; dANa yA meVvu tininxsu. 
\num{2} (vayxkitx \mo varanunx) aMdagoLisu; niVTu mADu; cenAnxgi, lakaSxNavAgi kANisuvaMte siMgarisu. 
\num{3} (obabxnanunx yAvudAdarU sAthxna \mo vakekx) tayArumADu; tarapeVtu mADu: \eng{was being groomed as a presidential candidate} adhayxkaSx sAthxnakekx obabx aBayxthiRyAgi tayAru mADalAgidadx, tarapeVtu koDalAgidadx. 
\num{4} (\AtAmx) tayArAgu; sidadhxvAgu: \eng{grooming for dinner} UTakekx sidadhxvAgutatx. 
\enum
\emng

\noindent
\gl{\pagu}
\bmng
 \eng{well-groomed} (manuSayxra \vi) (kUdalu, gaDaDx, \mo vanunx bAci, suMdaravAgi uDige toDige dharisi) niVTu mADikoMDa; aMda mADikoMDiruva; cenAnxgi siMgarisikoMDa. 
\emng
\eentry

\bentry
\word{groomsman}
\pron{gUrxmfsfZmanf}
\gl{\nA}
\bmng
 aLiyana geLeya; saMparxdAyAnusAra maduvaNigana joteyalilxruva avana avivAhita senxVhita. 
\emng
\eentry

\bentry
\word[groove(1)]{groove}
\pron{gUrxvf}
\gl{\nA}
\bmng
\bnum
\num{1} (\kanmu\ muMdakekx calisuvaMte mADalu yA adeV AkArada ububx adakekx hoMduvaMtAgalu mara, loVha, \mo vugaLalilx koreda, mADida) toVDu; gADi. 
\num{2} (rUDhiya) jADu; aBAyxsada naDavaLike; rUDhibidadx kelasakAyaR; nitayx vidhAna; nitayxgaTaTxle; saMparxdAya; vaqdAdhxcAra. 
\num{3} korakalu; savakalu dAri. 
\num{4} (obabxna savxBAva, sAmathayxR, manoVdhamaR, \mo vugaLige takakx) sAthxna, hudedx, \mo vu: \eng{found his groom in advertising} jAhiVrAtu kelasa sikikxdadxriMda avanige tananx sAmathayxRkekx takakx hudedx doretaMtAyitu. 
\num{5} (\ashi) atuyxtatxma -- sithxti, deshe, manoVdhamaR; hitavAda, sariyAda -- sithxti: \eng{a hot bath and a drink will put you back in the groove} bisiniVru sAnxna, oMdiSuTx pAna -- ivu ninanxnunx hitavAda sithxtige oyuyxtatxve. 
\num{6} (\ashi) oLeLxya jAsfZ saMgiVta. 
\enum
\emng

\noindent
\gl{\nuga}
\bmng
 \eng{in the groove} (\ashi) 
\banum
\alnum{a} tuMba cenAnxgi parxdashaRna niVDutitxruva. 
\alnum{b} oLeLxya parxdashaRnavanunx mecucxva yA mecucxva manoVdhamaRvuLaLx. 
\alnum{c} paramAyiSiyAda; sogasAda; BajaRriyAda. 
\alnum{d} ulAlxsapUNaR; utAsxhaBarita. 
\alnum{e} sadayxda janapirxya phAyxSanAnxgiruva; atAyxdhunika; navoVnava. 
\eanum
\emng
\eentry

\bentry
\word[groove(2)]{groove}
\pron{gUrxvf}
\gl{\sakirx}
\bmng
\bnum
\num{1} (yAvudaralelxV) gADi, toVDu, \mo vanunx -- kore, mADu. 
\num{2} gADi, toVDu, \mo vanunx mADi (yAvudakekxV) seVrisu, kUrisu. 
\num{3} (\ashi) (vayxkitxge) KuSikoDu; AnaMda uMTumADu. 
\enum
\emng

\noindent
\gl{\akirx}
\bmng
 (\ashi) 
\bnum
\num{1} sogasAgiru. 
\num{2} oLeLxya jAsfZ saMgiVtada suKa anuBavisu. 
\num{3} muMduvari; parxgati paDe; aBivaqdidhxhoMdu. 
\num{4} (obabx vayxkitxyoDane) cenAnxgi hoMdikoMDu hoVgu. 
\enum
\emng
\eentry

\bentry
\word{groovily}
\pron{gUrxvili}
\gl{\kirxvi}
\bmng
\bnum
\num{1} rUDhiya jADinalilx; sAMparxdAyika manoVBAvadiMda. 
\num{2} (\ashi) BajaRriyAgi; atuyxtatxmavAgi parxdashaRna niVDutatx. 
\enum
\emng
\eentry

\bentry
\word{grooviness}
\pron{gUrxvinisf}
\gl{\nA}
\bmng
 rUDhibidadx kelasadalelxV toDagiruvike; vaqdAdhxcArada savxBAva; sAMparxdAyika, kaMdAcArada -- manoVBAva. 
\emng
\eentry

\bentry
\word{groovy}
\pron{gUrxvi}
\gl{\gu}
\bmng
\bnum
\num{1} rUDhiya jADina; nitayxvidhAnada; oMdeV jADina; hoVda dAriyalelxV hoVguva; savakalu dAriyalilx hoVguva: \eng{school masters are sometimes groovy} upAdhAyxyaru kelavu veVLe savakalu dAri hiDidu naDeyuva savxBAvadavaru. 
\num{2} gADiya; toVDina. 
\num{3} gADiyanunx, toVDanunx -- hoVluva. 
\num{4} (\ashi) sogasAda; cenAnxda; BajaRriyAda. 
\enum
\emng
\eentry

\bentry
\word{grope}
\pron{gorxVpf}
\gl{\akirx}
\bmng
\bnum
\num{1} (katatxleyalilx heVgoV hAge) taDakADu; taDavu; taDavarisu. 
\num{2} kuruDanaMte huDukADu (\rUpa\ saha). 
\enum
\emng

\noindent
\gl{\sakirx}
\bmng
\bnum
\num{1} taDavi yA heVgoV dAri kaMDuko. 
\num{2} (\ashi) (obabxna) jananAMgavanunx neVvarisu, pirxVtiyiMda savaru, mudAdxDu, cuMbisu. 
\enum
\emng

\noindent
\gl{\pagu}
\bmng
 \eng{grope one's way} 
\banum
\alnum{a} taDakADutatx naDe; dAri huDuku. 
\alnum{b} (\rUpa) dAri yA mAgaR tiLiyade huDukutitxru. 
\eanum
\emng
\eentry

\bentry
\word{groper}
\pron{gorxVparf}
\gl{\nA}
\bmng
 (\kanmu\ \AseTxrXV matutx nUyxsiZVlaMDf)  = \hyperlink{grouper}{grouper}. 
\emng
\eentry

\bentry
\word{gropingly}
\pron{gorxVpiMgfli}
\gl{\kirxvi}
\bmng
\bnum
\num{1} taDakADutAtx; taDavutAtx; taDavarisuvaMte. 
\num{2} (kuruDanaMte) huDukADutAtx. 
\enum
\emng
\eentry

\bentry
\word{grosbeak}
\pron{gorxVsfbiVkf}
\gl{\nA}
\bmng
 niVLacaMcu; udadxneya balavAda kokukxLaLx kelavu saNaNx pakiSxjAtigaLu. 
\emng
\eentry

\bentry
\word{groschen}
\pron{gArx(gorxV)Sanf}
\gl{\nA}
\bmng
 gorxVSanf: 
\banum
\alnum{a} (\ca) jamaRniya saNaNx beLiLxnANayx. 
\alnum{b} AsiTxrXyAda kaniSaThx beleya oMdu nANayx, \eng{1/100} SiliMgf. 
\alnum{c} (\AmA) jamaRniya \eng{10} phenigf nANayx. 
\eanum
\emng
\eentry

\bentry
\word{grosgrain}
\pron{gorxVgerxVnf}
\gl{\nA}
\bmng
 reVSemxhuri baTeTx; reVSemx \mo vugaLa dArada gereyiruva, hurihuriyAda baTeTx. 
\emng
\eentry

\bentry
\wordf{gros point}
\pron{gorxV pAvxknuf(nf)}
\gl{\nA}
\expl{\F\ }
\bmng
kAyxnfvAsf kasUti; kAyxnfvAsina meVle aDaDx holigegaLanunx hAki mADuva kasUti (kelasa). 
\emng
\eentry

\bentry
\word[gross(1)]{gross}
\pron{gorxVsf}
\gl{\nA}
\bmng
(\bava\ adeV). gorxVsu; \eng{12} Dajanunx. 
\emng

\noindent
\gl{\nuga}
\bmng
 \eng{by the gross} gorxVsugaTaTxle; sArAsagaTAgi; BAri parxmANadalilx. 
\emng
\eentry

\bentry
\word[gross(2)]{gross}
\pron{gorxVsf}
\gl{\gu}
\bmng
\bnum
\num{1} hulusAgi, gAdAgi, turugAgi -- beLeda; samaqdadhxvAda: \eng{gross shoots} hulusAda kuDigaLu. 
\num{2} mitimIri tiMdu Udida; guDANa, DoLuLx -- hoTeTxya; bojujx bojAjxda; asahayxvAgi bojujx beLeda. 
\num{3} edudx kANuva; ati sapxSaTxvAda; visapxSaTxvAda; kaNuNx kukukxvaMtha: \eng{one gross error after another} oMdAda meVloMdu kaNiNxge kukukxvaMtha tapupxgaLu. 
\num{4} oTiTxna; pUrA; kaLetagaLu seVrirada; nivavxLavalalxda: \eng{gross earnings} oTuTx AdAya. 
\num{5} daTaTx; sAMdarx; nibiDa; Gana: \eng{a gross fog} daTaTxvAda maMju, kAvaLa. 
\num{6} pAthiRva; pArxkaqta; sUthxla; iMdirxyagoVcara; aiMdirxyaka; (keVvala) aihika; pArxpaMcika: \eng{spirits of purest light at first, now grown gross by sinning} modalu parishudadhx teVjasisxniMda tuMbidadx AtamxgaLu Iga pApada PalavAgi keVvala pAthiRvavAgive. 
\num{7} (AhAra \mo vu) kiVLetxrada; oraTAda; koLakAda; asahayxkara; Okarike huTiTxsuva: \eng{fish, oil and such gross commodities} mInu, eNeNx, modalAda koLaku padAthaRgaLu. 
\num{8} (iMdirxya \mo) sUkaSxmXvalalxda; jaDa; maMda: \eng{our eyes are too gross to discern the workmanship of nature} namamx kaNuNxgaLu parxkaqtiya kalAvaMtikeyanunx garxhisalAgadaSuTx maMdavAgive. 
\num{9} (riVti niVtigaLalilx) nayanAjUkilalxda; oraTAda; asaMsakxqqta; ashilxVla; asaBayx; niVtigeTaTx: \eng{society of high culture, but in morals lax, even gross} atiyAda nayanAjUkina samAja, Adare adu niVtiyalilx saDila, aSeTxV alalx, asaBayx. 
\enum
\emng

\noindent
\gl{\pagu}
\bmng
\bnum
\numi{1} \eng{gross feeder} 
\banum
\alnum{a} kadananxBakaSxka; rUkASxhAri; koLaku AhAra tinunxvavanu. 
\alnum{b} vipariVtadini; heVraLavAgi gobabxravanunx bayasuva giDa \mo vu. 
\eanum
\numie
\num{2} \eng{in (the) gross} (vivaragaLige hoVgade) sUthxlavAgi; oTiTxna meVle; oTATxre. 
\enum
\emng
\eentry

\bentry
\word[gross(3)]{gross}
\pron{gorxVsf}
\gl{\sakirx}
\bmng
oTuTx lABavAgi utApxdisu yA gaLisu. 
\emng

\noindent
\gl{\pagu}
\bmng
 \eng{gross up} oTuTx motatxkekxVrisu; terige \mo vugaLanunx kaLeyuvudakikxMta modalina motatxkekx (nivavxLa motatxvanunx) Erisu, hecicxsu. 
\emng
\eentry

\bentry
\wordnospeech{gross domestic product}{gross domestic product}
\pron{?}
\gl{\nA}
\bmng
 samagarx aMtadeRVshiVya utApxdane; nivavxLa deVshiVya utapxnanx; oMdu vaSaRda avadhiyalilx deVshadalilx utApxdisida seVve hAgU utapxnanxgaLa mwlayx (idaralilx beVre beVre deVshagaLiMda baruva nivavxLa AdAyavanunx seVrisuvudilalx). 
\emng
\eentry

\bentry
\word{grossly}
\pron{gorxVsfli}
\gl{\kirxvi}
\bmng
\bnum
\num{1} (\pArxparx) daTaTxvAgi; nibiDavAgi. 
\num{2} (\pArxparx) sapxSaTxvAgi; KacitavAgi; niKaravAgi. 
\num{3} atiyAgi; vipariVtavAgi; pUrA; tiVrA; atayxMta: \eng{grossly ignorant} tiVrA ajAcnxniyAgi. 
\num{4} asaBayxvAgi; ashiSaTx riVtiyalilx; oraTAgi: \eng{grossly clad} asaBayxvAgi baTeTx dharisida. 
\num{5} (tiMDi, kuDitagaLa \vi) asahayxvAgi; atiyAgi; vipariVtavAgi. 
\num{6} sUthxlavAgi; oTATxre; oTiTxna meVle. 
\enum
\emng
\eentry

\bentry
\wordnospeech{gross national product}{gross national product}
\pron{?}
\gl{\nA}
\bmng
 samagarx rASiTxrXVya utApxdane; nivavxLa rASiTxrXVya utapxnanx; oTuTx nADina utApxdane; oMdu deVshadalilx oMdu vaSaRdalilx utapxtitx mADida padAthaRgaLu hAgU seVvegaLa oTuTx mwlayx. 
\emng
\eentry

\bentry
\word{grossness}
\pron{gorxVsfnisf}
\gl{\nA}
\bmng
\bnum
\num{1} (\pArxparx) sUthxlatavx; bojujx. 
\num{2} atiyAgiruvike; vipariVtavAgiruvudu. 
\num{3} (\pArxparx) sAMdarxte; nibiDate. 
\num{4} (AhAra, mAtukate, \mo vugaLa \vi) ashiSaTxte; asaBayxte; oraTutana; koLakutana; nayanAjUkilalxdiruvudu. 
\num{5} huMbatana; pedudxtana; gAMpatana. 
\enum
\emng
\eentry

\bentry
\wordf{grosso modo}
\pron{gArxsoV moVDoV}
\gl{\kirxvi}
\expl{\It}
\bmng
hecucxkaDame; aMdAjinalilx; sumAru. 
\emng
\eentry

\bentry
\word{grot}
\pron{gArxTf}
\gl{\nA}
\bmng
 (\kAparx)  = \hyperlink{grotto}{grotto}. 
\emng
\eentry

\bentry
\word[grotesque(1)]{grotesque}
\pron{gorxVTesfkx}
\gl{\nA}
\bmng
\bnum
\num{1} vikaTa -- citarx, shilapx; (manuSayxna matutx pArxNigaLa AkaqtigaLanunx eleguMpina latAkaqtiyoDane citarx vicitarxvAgi seVrisi racisida) vikaTAlaMkAra citarx yA shilapx. \imglink{grotesque-1figure}{\raisebox{-0.15cm}[0pt][0pt]{\pdfimage width 0.7cm height 0.4cm {G_Pictures/grotesque-1.jpg}}} 
\num{2} (Igina baLake) vikaqta citarx; tamASeyAgiruvaMte, hAsAyxsapxdavAguvaMte vikAra mADida kaqti, racane. 
\enum
\emng
\eentry

\bentry
\word[grotesque(2)]{grotesque}
\pron{gorxVTesfkx}
\gl{\gu}
\bmng
\bnum
\num{1} (\vAshi) vikaTa -- citarx yA shilapx sheYliya. 
\num{2} vakarxvakarx; vijAtiVya; vikaTAlaMkArada; ativicitarxvAda; virUpagoLisida; aSATxvakarxvAda. 
\num{3} asaMbadadhx; apahAsayxkara; hAsAyxsapxdavAda; asaMgata; akaTavikaTa. 
\enum
\emng
\eentry

\bentry
\word{grotesquely}
\pron{gorxVTesfkxli}
\gl{\kirxvi}
\bmng
\bnum
\num{1} vikaTavAgi; aSATxvakarxvAgi; ati vicitarxvAgi. 
\num{2} asaMbadadhxvAgi; hAsAyxsapxda riVtiyalilx; asaMgatavAgi; akaTavikaTavAgi. 
\enum
\emng
\eentry

\bentry
\word{grotesqueness}
\pron{gorxVTesfkxnisf}
\gl{\nA}
\bmng
\bnum
\num{1} vikaTate; vakarxte; vikAravAgiruvudu. 
\num{2} asaMbadadhxte; asAMgatayx. 
\enum
\emng
\eentry

\bentry
\word{grotesquerie}
\pron{gorxVTesakxri}
\gl{\nA}
\bmng
\bnum
\num{1} (sAmUhikavAgi) vikaTa vasutxgaLu. 
\num{2} vikaTakAyaR; vicitarxvataRne. 
\num{3} vikaTa -- lakaSxNa guNa; aSATxvakarxte; ativeYcitarxyX. 
\enum
\emng
\eentry

\bentry
\word{grotto}
\pron{gArxToV}
\gl{\nA}
\expl{(\bava\ \eng{grottos} yA \eng{grottoes}).}
\bmng
\bnum
\num{1} gArxToV; (sAvxBAvikavAgiyeV) suMdara guhe; aMdavAda gavi. 
\num{2} suMdara kaqtaka guhe; siMgarisida kaqtaka guhe, gavi. 
\num{3} gavi koVNe; kapepxcipupx \mo vugaLiMda alaMkarisida, guheyananxnukarisi alaMkAra mADida koVNe \mo vu. 
\enum
\emng
\eentry

\bentry
\word{grottoed}
\pron{gArxToVDf}
\gl{\gu}
\bmng
\bnum
\num{1} siMgarisida, aMdavAda -- gaviyalilxruva. 
\num{2} suMdara guheyAgisida; aMdavAda guheyAgi parivatiRsiruva; siMgarisida gaviyaMte racitavAda; guhAkaqtiyalilxruva; gaviyaMtiruva. 
\enum
\emng
\eentry

\bentry
\word{grotty}
\pron{gArxTi}
\gl{\gu}
\bmng
 (\birx) (\ashi) holasAda; kacaDavAda; vikaTavAda; ahitakaravAda: \eng{the house was grotty} mane holasAgitutx. 
\emng
\eentry

\bentry
\word[grouch(1)]{grouch}
\pron{gwrxcf}
\gl{\akirx}
\bmng
 (\AmA) goNaguTuTx; goNagADu. 
\emng
\eentry

\bentry
\word[grouch(2)]{grouch}
\pron{gwrxcf}
\gl{\nA}
\bmng
\bnum
\num{1} goNaguTuTxvava. 
\num{2} ataqpatx (vayxkitx). 
\num{3} goNagATa. 
\enum
\emng
\eentry

\bentry
\word{grouchy}
\pron{gwrxci}
\gl{\gu}
\bmng
 goNaguTuTxva; goNagATada. 
\emng
\eentry

\bentry
\word[ground(1)]{ground}
\pron{gwrxnfDx}
\gl{\nA}
\bmng
\bnum
\num{1} kaDala taLa; samudarxda taLa. 
\num{2} (\bava dalilx) \kanmu\ kAPiya -- caraTa, gasi, gaSuTx. 
\num{3} = \hyperref{kandict_e.pdf}{E}{earth(9a)}{$^1$earth (9a)}. 
\num{4} AdhAra; asitxBAra; bunAdi; nelegaTuTx. 
\num{5} (sariyAda) kAraNa; (samathaR) AdhAra: \eng{on the ground of} (yAvudoV oMdara) kAraNadiMda; nepadiMda; AdhArada meVle. \eng{on public grounds} sAvaRjanika kAraNagaLiMda. 
\num{6} keLapadara; taLasatxra. 
\num{7} (yAvudaradeV) taLaBAga; AdhAra BAga. 
\num{8} (kasUti kelasa, vaNaRcitarxracane, \mo vugaLalilxna) taLa; meVlemxY; citarx kelasa mADuva meYBAga. 
\num{9} (citarx \mo vugaLalilxna) KAli jAga; citarx kelasavanunx mADade uLidiruva BAga. 
\num{10} (vasutxvina, citarxda) talavaNaR; keLabaNaNx; AdhAra vaNaR; itara baNaNxgaLige AdhAravAgi yA hinenxleyAgi modalu hacicxda baNaNx. 
\num{11} (vaNaRcitarxdalilxna) parxdhAna vaNaR yA vaNaRCAye; parxmuKavAgi yA edudx kANuva baNaNx yA adara CAye. 
\num{12} (ketatxneyalilx) meVlemxY leVpa; talaleVpa; sUjiyiMda koreyuva munanx meVlemxYge savaruva padAthaR. 
\num{13} BUmi; nela; BUmiya meVlemxY. 
\num{14} (\bava\ dalilx) AvaraNa; sututx nela; aMgaNa; beVli \mo\ AvaraNa hAki alaMkArakAkxgali vihArakAkxgali iTuTxkoMDiruva, manege seVrida parxdeVsha. 
\num{15} (oMdu gotAtxda) BUBAga; BUparxdeVsha; BUparxdeVshadalilxna -- sAthxna, haravu, yA dUra. 
\num{16} oMdu gotAtxda udedxVshakAkxgi iTiTxruva -- parxdeVsha, bayalu, jamInu: \eng{fishing grounds} mInugArike parxdeVsha; mInu hiDiyalu nigadiyAda sathxLa. \eng{cricket grounds} kirxkeTf meYdAna. 
\num{17} (obabxnige seVrida) BUmi kANi; jamInu; BU Asitx. 
\num{18} (kirxkeTf) nilulxjAga; \kanmu\ bAyxTugArana kaDeya vikeTiTxgU avana muMdina pApiMgf geregU naDuvaNa jAga: \eng{in his ground} tananx nilulxjAgadoLage. 
\num{19} (\birx) koThaDi \mo vugaLa nela. 
\num{20}  = \hyperlink{ground-staff}{ground-staff}. 
\enum
\emng

\noindent
\gl{\pagu}
\bmng
\eng{classic ground} parxsidadhx sathxLa; aitihAsika keSxVtarx. 
\emng

\noindent
\gl{\nuga}
\bmng
\bnum
\num{1} \eng{above ground} jiVvaMtavAgiruva; badukiruva. 
\num{2} \eng{be dashed to the ground} = \hyperlink{ground1 nuga5}{?nuga? \((5)\)}. 
\num{3} \hyperref{kandict_b.pdf}{B}{break(1) pagu(4)}{$^1$break ground.} 
\num{4} \eng{cover much ground} (vicAraNe, varadi, \mo vugaLu) bahaLa vAyxpakavAgiru; bahaLa viSayagaLanonxLagoMDiru. 
\hypertarget{ground1 nuga5}{} 
\num{5} \eng{down to the ground} (\AmA) pUtiRyAgi; saMpUNaRvAgi. 
\num{6} \eng{fall to the ground} (oMdu saMkalapx, yoVjane, parxtiVkeSx, Ase) vayxthaRvAgu; niSapxrXyoVjakavAgu; bidudx hoVgu; dhavxMsavAgi biDu; maNuNxgUDu. 
\num{7} \eng{forbidden ground} niSidadhx vasutx; tAyxjayx viSaya; keY hAkabArada viSaya. 
\numi{8} \eng{from the ground up} 
\banum
\alnum{a} kAliniMda taleyavarege; taLadiMda tudiyavarege; karxmavAgi: \eng{he decided to learn the business from the groud up} avanu A udayxmavanunx taLadiMda meVlinavarege saMpUNaRvAgi kaliyalu nidhaRrisida. 
\alnum{b} (\AmA) saMpUNaRvAgi; pUtiRyAgi; taLasapxshiRyAgi; AdayxMtavAgi: \eng{he knows the subject from the group up} avanu A shAsatxrXvanunx pUtiRyAgi balalx. 
\hypertarget{ground(1) nuga(9)}{} 
\eanum
\numie
\numi{9} \eng{gain ground} 
\banum
\alnum{a} muMduvari; aBivaqdidhx hoMdu; sithxti balapaDisiko. 
\alnum{b} (yAvudAdarU aBipArxya, vAda, padadhxti, \mo vu janaralilx) vAyxpisu; haraDu; habubx: \eng{the custom is gaining ground} A padadhxtiyu Iga (janaralilx) habubxtitxde. 
\eanum
\numie
\num{10} \eng{get in on the ground floor} kaMpaniya sAthxpakarige anavxyisuva niyamagaLa parxkAraveV oMdu kaMpani \mo vakekx parxveVsha paDe. 
\num{11} \eng{get off the ground} (\AmA) yashasivxyAgi AraMBavAgu. 
\num{12} \eng{give} (\engit{or} \eng{lose) ground} himemxTuTx; hiMjari; biTuTxkoDu; hiMde biVLu; kiSxVNisu; avanatiyAgu. 
\numi{13} \eng{go to ground} 
\banum
\alnum{a} (nAyi, nari, \mo vugaLa \vi) bila -- hogu, seVru. 
\alnum{b} (vayxkitxya \vi) kaNamxreyAgu; sAvaRjanika gamanadiMda hiMdege. 
\eanum
\numie
\num{14} \eng{into the ground etc.} (\AmA) susAtxguvavarege. 
\num{15} \eng{on firm, solid, etc. ground} suBadarx neleyalilx; sariyAda takaRvanunx Adharisi, baLasi. 
\num{16} \eng{on one's ground} (obabxnige cenAnxgi paricayaviruva) keSxVtarx; parxdeVsha: \eng{meet the enemy on his own ground} shaturxvina keSxVtarxdalelxV avananunx edurisu. 
\num{17} \eng{on the ground} (sAvxrasayx GaTane, caTuvaTike, \mo vu) saMBavisida, naDeda -- sathxLadalilx: \eng{very soon reporters were on the ground to get the story} sudidxgAgi varadigAraru A kUDaleV sathxLakekx baMdaru. 
\num{18} \eng{shift one's ground} nele badalAyisu; vAda, udedxVsha, \mo vanunx badalisu. 
\num{19} \eng{stand one's ground} nela kacicx nilulx; nele biDadiru; tananx vAda, udedxVsha, \mo vanunx badalisadiru. 
\num{20} \eng{take ground} (haDagina \vi) nela -- kacucx, hatutx, hiDi. 
\hyperdef{G}{ground(1) nuga(21)}{} 
\num{21} \eng{thin on the ground} hecAcxgilalxda; bahu saMKeyxyalilxrada; aparUpavAda. 
\numi{22} \eng{touch ground} 
\banum
\alnum{a} (bariV haraTe \mo vanunx biTuTx) muKayx viSayakekx baru. 
\alnum{b} (haDagina \vi) nela kacucx. 
\eanum
\numie
\enum
\emng
\eentry

\bentry
\word[ground(2)]{ground}
\pron{gwrxnfDx}
\gl{\gu}
\bmng
\bnum
\num{1} (hakikxgaLa hesarugaLalilx) nelada meVle vAsisuva; nelada; BUcara: \eng{ground sparrow} nelada gubabxcicx. 
\numi{2} (pArxNigaLa \vi) 
\banum
\alnum{a} nelavAsi; BUvAsi; nelada; bila koreyuva. 
\alnum{b} BUshAyi; nelada meVle malaguva. 
\eanum
\numie
\numi{3} (giDagaLa \vi) 
\banum
\alnum{a} kurucalu; giDaDx; moVTu. 
\alnum{b} nelada meVle -- habubxva, parxsarisuva, baLiLxvariyuva. 
\eanum
\numie
\enum
\emng
\eentry

\bentry
\word[ground(3)]{ground}
\pron{gwrxnfDx}
\gl{\sakirx}
\bmng
\bnum
\num{1} (saMsethx, tatatxvX, naMbike, \mo vanunx yAvudAdarU nishacxyAMsha yA AdhArada meVle) sAthxpisu; parxtiSiThxsu; nelegoLisu (\sA\ kamaRNi \parx): \eng{well grounded} oLeLxya AdhAravuLaLx. 
\num{2} (kasUti \mo vakekx) AdhAratala EpaRDisu; hinenxle hoMdisu. 
\num{3} (\kanmu\ shasAtxrXsatxrXgaLanunx) nelada meVliDu; keLagiDu. 
\num{4} = \hyperref{kandict_e.pdf}{E}{earth(2)3}{$^2$earth (3)}. 
\num{5} haDaganunx -- nelakacicxsu, daDa hatitxsu. 
\num{6} (vimAna, veYmAnika) hAradaMte -- taDe (hiDi), aDiDx mADu. 
\enum
\emng

\noindent
\gl{\akirx}
\bmng
\bnum
\num{1} nelakikxLi; nelakekx -- tAgu, biVLu; keLakikxLi. 
\num{2} (haDagu) nelakacucx; taLa -- hiDi, hatutx. 
\enum
\emng
\eentry

\bentry
\word[ground(4)]{ground}
\pron{gwrxnfDx}
\gl{\kirx}
\bmng
 \eng{grind} dhAtuvina \BU\ matutx \BUkaq\ : \eng{ground glass} ujijxda gAju; ujijx apAradashaRkavAgi mADida gAju. 
\emng
\eentry

\bentry
\word{groundage}
\pron{gwrxniDxjf}
\gl{\nA}
\bmng
 nelasuMka; baMdaru suMka; samudarxtiVradalilx niMtiruva yA baMdarinoLakekx baMda haDagige hAkuva suMka. 
\emng
\eentry

\bentry
\word{ground-ash}
\pron{gwrxnfDxAYxSf}
\gl{\nA}
\bmng
\bnum
\num{1} AYxSf marada sasi. 
\num{2} AYxSf (sasiyiMda mADida) keYbetatx, baDige. 
\enum
\emng
\eentry

\bentry
\word[ground-bait(1)]{ground-bait}
\pron{gwrxnfDxbeVTf}
\gl{\nA}
\bmng
 taLada ere; mInu hiDiyuva kaDe mInanunx AkaSiRsalu taLakekx eseda ere (\rUpa\ saha). 
\emng
\eentry

\bentry
\word[ground-bait(2)]{ground-bait}
\pron{gwrxnfDxbeVTf}
\gl{\sakirx}
\bmng
 mInanunx AkaSiRsalu taLakekx ere ese (\rUpa saha). 
\emng
\eentry

\bentry
\word{ground-based}
\pron{gwrxnfDxbeVsfTx}
\gl{\gu}
\bmng
 (\kanmu\ vAyuyAnakekx saMbaMdhisidaMte) BUsaMbaMdhi; nilAdxNa \mo\ nelada meVlina kelasagaLige, viSayagaLige saMbaMdhisida. 
\emng
\eentry

\bentry
\wordnospeech{ground bass}{ground bass}
\pron{?}
\gl{\nA}
\bmng
 (\saM) maMdarxBAga; oMdu saMgiVtakaqtiyalilx baruva, udadxkUkx punarAvataRnegoLuLxva, maMdarxsavxradalilx hADalu udedxVshisida BAga. 
\emng
\eentry

\bentry
\word{ground-box}
\pron{gwrxnfDxbAkfsx}
\gl{\nA}
\bmng
 nelagurucalu; toVTada pAtigaLa (maDigaLa) aMcinalilx beLesuva, bAkfsx eMba kurucalu giDa. 
\emng
\eentry

\bentry
\word{groundcherry}
\pron{gwrxnfDxceri}
\gl{\nA}
\bmng
\bnum
\num{1} kubajx, kuLaLx -- ceri; giDaDxjAtiya ceri (giDa) yA haNuNx. 
\num{2} phisalisf kulada giDa. 
\enum
\emng
\eentry

\bentry
\word{ground-colour}
\pron{gwrxnfDxkalarf}
\gl{\nA}
\bmng
\bnum
\num{1} (baNaNxda) parxthama leVpa; modala baLita. 
\num{2} (itara baNaNxgaLiMda bareda citarxkekx hinenxleyAda) AdhAravaNaR; nelebaNaNx. 
\enum
\emng
\eentry

\bentry
\wordnospeech{ground control}{ground control}
\pron{?}
\gl{\nA}
\bmng
 (\vAyA) BU niyaMtarxNa; vimAnavu AkAshadiMda nelakekx iLiyuvudanunx BUmiyiMda niyaMtirxsuvudu. 
\emng
\eentry

\bentry
\word{grounder}
\pron{gwrxnaDxrf}
\gl{\nA}
\bmng
\bnum
\num{1} sAthxpaka; kAraka; sAthxpisuvavanu yA sAthxpisuvudu; uMTu mADuvava yA uMTu mADuvudu. 
\num{2} (kirxkeTf) nelahoDeta; nelada meVle uruLutAtx hoVguvaMte hoDeda ceMDu. 
\enum
\emng
\eentry

\bentry
\word{ground-fish}
\pron{gwrxnfDxphiSf}
\gl{\nA}
\bmng
 taLamInu; niVrina taLadalilx vAsisuva mInu. 
\emng
\eentry

\bentry
\word{ground-fishing}
\pron{gwrxnfDxphiSiMgf}
\gl{\nA}
\bmng
 taLakekx ere esedu mInu hiDiyuvudu. 
\emng
\eentry

\bentry
\wordnospeech{ground floor}{ground floor}
\pron{?}
\gl{\nA}
\bmng
 nela aMtasutx; horagina nelamaTaTxdalilxruva koVNegaLu \mo vu. 
\emng

\noindent
\gl{\nuga}
\bmng
 \eng{get in on the ground floor} (yAvudeV yoVjane, kaMpani, \mo vugaLige) sAthxpakaralilx, parxvataRkaralilx obabxnAgi parxveVsha paDeyuvudu; sAthxpaka, parxvataRka -- sadasayxnAguvudu. 
\emng
\eentry

\bentry
\wordnospeech{ground frost}{ground frost}
\pron{?}
\gl{\nA}
\bmng
 nelahima; BUmiya meVlemxY meVle yA meVlamxNiNxna meVle haraDiruva hima. 
\emng
\eentry

\bentry
\wordnospeech{ground game}{ground game}
\pron{?}
\gl{\nA}
\bmng
 (\birx) nelabeVTe; bila, biruku, \mo vugaLalilx vAsisuva, beVTege tutAtxguva mola \mo\ pArxNigaLu. 
\emng
\eentry

\bentry
\wordnospeech{ground glass}{ground glass}
\pron{?}
\gl{\nA}
\bmng
 apAradashaRka gAju; ujujxvudeV \mo\ vidhAnagaLiMda apAradashaRkavAgisida gAju. 
\emng
\eentry

\bentry
\word{groundhog}
\pron{gwrxnfDxhAgf}
\gl{\nA}
\bmng
\bnum
\num{1} = \hyperref{kandict_a.pdf}{A}{aardvark}{aardvark}. 
\num{2} gwrxMDfhAgf; aLilu (iNaci) vaMshakekx seVrida mAmaRTf eMba amerikada oMdu daMshaka. 
\enum
\emng

\noindent
\gl{\pagu}
\bmng
 \eng{Groundhog Day} (\ame) gwrxMDfhAgf dina; phebarxvari \eng{2}ne tAriVKu (I dinadaMdu gwrxMDfhAgf daMshakavu tananx neraLanunx noVDidare inUnx \eng{6} vAra caLigAla muMduvariyutatxde, adu aMdu noVDadidadxre caLigAla beVga mugidu hoVgutatxde eMdu oMdu naMbike). 
\emng
\eentry

\bentry
\word{ground-ice}
\pron{gwrxnfDxaisf}
\gl{\nA}
\bmng
taLada -- niVgaRlulx, maMjugaDeDx; niVrina taLadalilx kaTiTxda niVgaRlulx. 
\emng
\eentry

\bentry
\word{grounding}
\pron{gwrxniDxMgf}
\gl{\nA}
\bmng
\bnum
\num{1} pArxthamika, mUla -- shikaSxNa (koDuvudu). 
\num{2} (oMdu \vi) taLahadi; taLapAya; asitxBAra: \eng{a good grounding in mathematics} gaNitadalilx balavAda taLahadi. 
\num{3} (ripeVrigAgi haDaganunx) nela, daDa -- hatitxsuvudu. 
\enum
\emng
\eentry

\bentry
\wordnospeech{ground landlord}{ground landlord}
\pron{?}
\gl{\nA}
\bmng
 (\birx) niveVshana pati; (mane kaTiTxkoLaLxlu sathxLavanunx geVNige koTiTxruva) niveVshanada -- oDeya, mAliVka, daNi. 
\emng
\eentry

\bentry
\word{groundless}
\pron{gwrxnfDxlisf}
\gl{\gu}
\bmng
 niSAkxraNa; nirAdhAra; taLabuDavilalxda. 
\emng
\eentry

\bentry
\word{groundlessly}
\pron{gwrxnfDxlisfli}
\gl{\kirxvi}
\bmng
 niSAkxraNavAgi; nirAdhAravAgi; taLabuDavilalxda riVtiyalilx. 
\emng
\eentry

\bentry
\word{groundlessness}
\pron{gwrxnfDxlisfnisf}
\gl{\nA}
\bmng
 niSAkxraNate; nirAdhAravAgiruvudu; taLabuDavilalxdiruvike. 
\emng
\eentry

\bentry
\wordnospeech{ground level}{ground level}
\pron{?}
\gl{\nA}
\bmng
 (\BUvi) 
\bnum
\num{1} nelamaTaTx; nelahaMta; BUmaTaTx. 
\num{2} paramANuvinalilxna ilekATxrXnf, nUyxkilxyasf, \mo vugaLa vividha shakitx haMtagaLalilx atayxMta keLaginadu. 
\enum
\emng
\eentry

\bentry
\word{groundling}
\pron{gwrxnfDxliMgf}
\gl{\nA}
\bmng
\bnum
\num{1}  = \hyperlink{ground-fish}{ground-fish}. 
\num{2} kurucalu giDa yA habubxgiDa; baLiLx; late. 
\num{3} nAlAkxNe siVTinavanu; pAmara perxVkaSxka; tiVrA keLadajeRya, kiVLu aBiruciya -- perxVkaSxka, noVTaka. 
\num{4} pAmaravAcaka; kiVLu aBiruciyuLaLx Oduga. 
\num{5} (\kanmu\ vimAnadalilxruvavanige virudadhxvAgi) BUmiya meVliruvava; BUmisathx; nelasathx; nelada meVliruvavanu. 
\enum
\emng
\eentry

\bentry
\word{groundman}
\pron{gwrxnfDxmanf}
\gl{\nA}
\bmng
  = \hyperlink{groundsman}{groundsman}. 
\emng
\eentry

\bentry
\word{groundnote}
\pron{gwrxnfDxnoVTf}
\gl{\nA}
\bmng
 (oMdu savxragArxmada) AdhArasavxra; mUla (maMdarx) savxra. 
\emng
\eentry

\bentry
\word{groundnut}
\pron{gwrxnfDxnaTf}
\gl{\nA}
\bmng
\bnum
\num{1} (\birx) nelagaDale; sheVMgA; kaDalekAyi. 
\num{2} (amerikada) iMtahadeV oMdu sasayxjAti. 
\enum
\emng
\eentry

\bentry
\word{ground-pine}
\pron{gwrxnfDxpeYnf}
\gl{\nA}
\bmng
\bnum
\num{1} bUyxgalf eMba, dhUpada vAsaneya oMdu yUroVpiyanf mUlike. 
\num{2} = \hyperref{kandict_c.pdf}{C}{club-moss}{club-moss}. 
\enum
\emng
\eentry

\bentry
\word{ground-plan}
\pron{gwrxnfDxpAlxYxnf}
\gl{\nA}
\bmng
\bnum
\num{1} (kaTaTxDada) mUlanakeSx; taLa (kaTuTx) nakeSx; mUla reVKAkaqti. 
\num{2} (yAvudaradeV) sAmAnayx kalapxne; sUthxla rUpa. 
\enum
\emng
\eentry

\bentry
\word{ground-rent}
\pron{gwrxnfDxrenfTx}
\gl{\nA}
\bmng
 niveVshanada bADige; nelagaMdAya; BUbADige; niveVshanada mAlikanige koDuva bADige. 
\emng
\eentry

\bentry
\wordnospeech{ground rule}{ground rule}
\pron{?}
\gl{\nA}
\bmng
 mUla sUtarx; mUlaBUtavAda tatatxvX. 
\emng
\eentry

\bentry
\wordnospeech{ground sea}{ground sea}
\pron{?}
\gl{\nA}
\bmng
 horaLagxDalu; birugaDalu; kuSxbadhxsamudarx; vayxkatx kAraNavilalxde parxkuSxbadhxvAda samudarx. 
\emng
\eentry

\bentry
\word[groundsel(1)]{groundsel}
\pron{gwrxnfDxsalf}
\gl{\nA}
\bmng
 (paMjarada hakikxgaLige AhAravAgi baLasuva) senesiyo kulada giDa. 
\emng
\eentry

\bentry
\word[groundsel(2)]{groundsel}
\pron{gwrxnfDxsalf}
\gl{\nA}
\bmng
\bnum
\num{1} (\vAshi) taLa dimimx; asitxBArakAkxgi baLasuva marada tole. 
\num{2} hositxlu; marada cwkaTiTxna (bAgilu \mo vugaLa) atayxMta taLaBAga. 
\enum
\emng
\eentry

\bentry
\word{groundsheet}
\pron{gwrxnfDxSiVTf}
\gl{\nA}
\bmng
 nelahAsu; nelaparade; BUmiya meVle hAsalu baLasuva, niVru hogada hALe. 
\emng
\eentry

\bentry
\word{groundsman}
\pron{gwrxnfDxsfZmanf}
\gl{\nA}
\bmng
 (kirxkeTf \mo\ ATagaLu) meYdAnapAla; kirxVDAMgaNa pAla; kirxVDAMgaNavanunx susithxtiyalilx iDuvavanu. 
\emng
\eentry

\bentry
\wordnospeech{ground speed}{ground speed}
\pron{?}
\gl{\nA}
\bmng
 (\vAyA) nelaveVga; BUveVga; BUmige sApeVkaSxvAgi vimAnavu calisuva veVga. 
\emng
\eentry

\bentry
\word{ground-squirrel}
\pron{gwrxnfDxsikxvXralf}
\gl{\nA}
\bmng
 neladaLilu; oMdu bageya aLilu. 
\emng
\eentry

\bentry
\word{ground-staff}
\pron{gwrxnfDxsATxYxphf}
\gl{\nA}
\bmng
\bnum
\num{1} BU sibabxMdi; nelasibabxMdi; vimAnada hArATadalilx BAgavahisada, nelada meVle kelasa mADuva, nilAdxNada eMjiniyarf \mo\ sibabxMdi. 
\num{2} (\birx) kirxkeTf kalxbibxnavaru haNakoTuTx neVmisikoMDa ATagAraru. 
\enum
\emng
\eentry

\bentry
\wordnospeech{ground state}{ground state}
\pron{?}
\gl{\nA}
\bmng
 (\Bwvi) =  \hyperlink{ground level}{ground level}. 
\emng
\eentry

\bentry
\wordnospeech{ground stroke}{ground stroke}
\pron{?}
\gl{\nA}
\bmng
 (lAnfTenisf) neladeVTu; nelada hoDeta; ceMDu puTida naMtara BUmige samiVpadalelxV hoDeda hoDeta. 
\emng
\eentry

\bentry
\wordnospeech{ground swell}{ground swell}
\pron{?}
\gl{\nA}
\bmng
 (BAri birugALiya yA BUkaMpada PalavAgi samudarxda taLadiMdaleV ELuva taraMgagaLiMdAda) kuSxbadhxte; alolxVla kalolxVla. 
\emng
\eentry

\bentry
\wordnospeech{ground torpedo}{ground torpedo}
\pron{?}
\gl{\nA}
\bmng
 taLasoPxVTaka; taLasiDi; haDagukore; kaDala taLadalilx neTaTx nwkA soPxVTaka, haDagusiDi. 
\emng
\eentry

\bentry
\wordnospeech{ground water}{ground water}
\pron{?}
\gl{\nA}
\bmng
 aMtajaRla; oLaniVru; BUmiya meVlapxdarada oLagiLidu, bAvi, cilume, \mo vugaLige odaguva niVru. 
\emng
\eentry

\bentry
\wordnospeech{ground wave}{ground wave}
\pron{?}
\gl{\nA}
\bmng
 (reVDiyoV) neladale; BUtaraMga; BUmiya meVlemxY meVle parxsAravAguva reVDiyoV ale. 
\emng
\eentry

\bentry
\word{groundwork}
\pron{gwrxnfDxvakfR}
\gl{\nA}
\bmng
\bnum
\num{1} (\sA\ \rUpa) AdhAra; asitxBAra; taLapAya; nelagaTuTx. 
\num{2} (oMdu padAthaRdalilx seVriruva) parxdhAnavasutx; muKAyxMsha. 
\num{3} (kasUtiyiMda yA beVre alaMkAradiMda maremADadiruva) sAmAnayx oDalu; hinenxle. 
\enum
\emng
\eentry

\bentry
\word{groundy}
\pron{gwrxniDx}
\gl{\gu}
\bmng
\bnum
\num{1} (kAphiya \vi) caraTavuLaLx; gasiyuLaLx; gaSiTxniMda kUDida. 
\num{2} (kAphiya \vi) maNiNxna ruci, vAsane iruva. 
\enum
\emng
\eentry

\bentry
\wordnospeech{ground zero}{ground zero}
\pron{?}
\gl{\nA}
\bmng
 sonenx nela; shUnayx BUmi; siDiyutitxruva bAMbina, \kanmu\ paramANu bAMbina, taLaBAgadalilxruva nela, BUmi. 
\emng
\eentry

\bentry
\word[group(1)]{group}
\pron{gUrxpf}
\gl{\nA}
\bmng
\bnum
\num{1} guMpu; samUha; neravi; puMja; taMDa; samudAya; oTiTxge iruva manuSayxru yA vasutxgaLu. 
\num{2} paMgaDa; guMpu; samudAya; oMdu taragatige seVrida yA seVrisida vayxkitxgaLu yA vasutxgaLu. 
\num{3} guMpu; oMdeV oDetanadalilxruva vANijayx kaMpanigaLu. 
\num{4} (rAjakiVyadalilx divxpakaSx parxButavx ilalxdiruva shAsanasaBegaLalilx, pakaSxkikxMta cikakxdAda) oLapaMgaDa; oLaguMpu. 
\num{5} samudAya citarx; rAshicitarx; eraDu yA hecicxna citarxgaLu, vasutxgaLu beVre beVreyAgidadxrU parasapxra hoMdikoMDu Agiruva samagarx racane, vinAyxsa. 
\num{6} vAyupaDeya yA vimAnataMDada -- oMdu BAga, viBAga. 
\num{7} janapirxya saMgiVta(gArara) taMDa. 
\enum
\emng
\eentry

\bentry
\word[group(2)]{group}
\pron{gUrxpf}
\gl{\sakirx}
\bmng
\bnum
\num{1} guMpu seVrisu; guMpAgisu; guMpugUDisu; samUhisu; samUhagoLisu. 
\num{2} samUhisu; guMpinoDane, guMpinalilx -- seVrisu. 
\num{3} (citarxgaLu, baNaNxgaLu, \mo vanunx) suvayxvasithxtavAgi, hoMdikeyAda samaSiTxrUpa baruvaMte aLavaDisu, (avugaLige) samanivxta rUpa koDu. 
\num{4} vagiRVkarisu. 
\enum
\emng

\noindent
\gl{\akirx}
\bmng
 (\viparx) guMpinalilx seVru; guMpu seVru; guMpAgu; guMpugUDu. 
\emng
\eentry

\bentry
\word{groupage}
\pron{gUrxpijf}
\gl{\nA}
\bmng
 vagiRVkaraNa mADuvudu. 
\emng
\eentry

\bentry
\wordnospeech{group annuity}{group annuity}
\pron{?}
\gl{\nA}
\bmng
 (jiVvavime) sAmudAyika vaSARshana; oMdu guMpige, vagaRkekx, samudAyakekx seVrida udoyxVgigaLu nivaqtatxrAda meVle vaSARshana paDeyuvaMte mADuva vimAvayxvasethx. 
\emng
\eentry

\bentry
\wordnospeech{group captain}{group captain}
\pron{?}
\gl{\nA}
\bmng
 (veYmAnika daLadalilx) taMDanAyaka; veYmAnika taMDada adhikAri. 
\emng
\eentry

\bentry
\word{grouper}
\pron{gUrxparf}
\gl{\nA}
\bmng
 seraniDeV vaMshada, KAdayxvAda, oMdu samudarxmInu. 
\emng
\eentry

\bentry
\word{groupie}
\pron{gUrxpi}
\gl{\nA}
\bmng
 (\ashi) 
\bnum
\num{1}  = \hyperlink{group captain}{group captain}. 
\num{2} gUrxpi; parxvAsiV saMgiVtagArara taMDavanunx hiMbAlisuva huDugi. 
\enum
\emng
\eentry

\bentry
\wordnospeech{group insurance}{group insurance}
\pron{?}
\gl{\nA}
\bmng
 (jiVvavime) guMpu vime; sAmudAyika vimApadadhxti; viBinanx deVhaparxkaqtiya hAgU vayasisxna vayxkitxgaLu sAmudAyikavAgi oMdeV opapxMdakakxnusAravAgi vimA swlaBayx paDeyuva vayxvasethx. 
\emng
\eentry

\bentry
\wordnospeech{group marriage}{group marriage}
\pron{?}
\gl{\nA}
\bmng
 (\sashA) sAmudAyika dAMpatayx; guMpugAhaRsathxyX; (anAgarika samAjagaLalilx) oMdu gaMDu taMDa oMdu heNuNx taMDavanunx maduveyAgi oMdeV dAMpatayx paMgaDavAgi uLiyuva vayxvasethx. 
\emng
\eentry

\bentry
\wordnospeech{group practice}{group practice}
\pron{?}
\gl{\nA}
\bmng
 samUha veYdayx(vaqtitx); aneVka maMdi veYdayxru oTATxgi seVri roVgigaLa roVgavanunx kaMDuhiDidu cikitesx niVDuva padadhxti. 
\emng
\eentry

\bentry
\wordnospeech{group representation}{group representation}
\pron{?}
\gl{\nA}
\bmng
 (sakARra) sAmudAyika pArxtinidhayx; guMpupArxtinidhayx; oMdu ADaLita maMDaLiyalilx BwgoVLika sAthxnakikxMta hecAcxgi hitAsakitxgaLa AdhArada meVle pArxtinidhayx niVDuva padadhxti. 
\emng
\eentry

\bentry
\wordnospeech{group sex}{group sex}
\pron{?}
\gl{\nA}
\bmng
 sAmUhika saMBoVga; guMpu meYthuna; EkakAladalilx ibabxrigiMta hecucx jana BAgavahisuva saMBoVga kirxye. 
\emng
\eentry

\bentry
\wordnospeech{group therapy}{group therapy}
\pron{?}
\gl{\nA}
\bmng
 sAmUhika cikitesx; samAnavAda roVgavuLaLx roVgigaLanunx oTiTxge seVrisi mADuva auSadhoVpacAra, cikitesx. 
\emng
\eentry

\bentry
\wordnospeech{group velocity}{group velocity}
\pron{?}
\gl{\nA}
\bmng
 samUha veVga; guMpu veVga; oMdu taraMgada yA taraMga samUhada shakitxyu parxyANa mADuva veVga. 
\emng
\eentry

\bentry
\word[grouse(1)]{grouse}
\pron{gwrxsf}
\gl{\nA}
\bmng
 (\bava\ adeV). gwrxsf: 
\banum
\alnum{a} koVLiya baLagada, TeTorxniDeV vaMshakekx seVrida, tupapxLu kAlina hakikx. \imglink{grousefigure}{\raisebox{-0.15cm}[0pt][0pt]{\pdfimage width 0.6cm height 0.5cm {G_Pictures/grouse.jpg}}} 
\hyperdef{G}{grouse(1)b}{} 
\hypertarget{grouse(1)b}{} 
\alnum{b} birxTiSf divxVpagaLalilx kaMDubaruva, beVTegAraru hiDiyuva keMpu gwrxsf hakikx. 
\alnum{c} gwrxsf (hakikxya) mAMsa. 
\eanum
\emng

\noindent
\gl{\pagu}
\bmng
 \eng{red grouse} = \hyperlink{grouse(1)b}{$^1$grouse (b)}. 
\emng
\eentry

\bentry
\word[grouse(2)]{grouse}
\pron{gwrxsf}
\gl{\nA}
\bmng
 (\ashi) goNagATa. 
\emng
\eentry

\bentry
\word[grouse(3)]{grouse}
\pron{gwrxsf}
\gl{\akirx}
\bmng
goNagu(TuTx); goNagADu. 
\emng
\eentry

\bentry
\word{grouser}
\pron{gwrxsarf}
\gl{\nA}
\bmng
 goNaguTuTxvava; goNagADuvava. 
\emng
\eentry

\bentry
\word[grout(1)]{grout}
\pron{gwrxTf}
\gl{\nA}
\bmng
 gwrxTu; (saMdugaLanunx mucacxlu hAkuva) teLaLxneya niVrugAre. 
\emng
\eentry

\bentry
\word[grout(2)]{grout}
\pron{gwrxTf}
\gl{\sakirx}
\bmng
 niVrugAre hAki mucicx samamADu; gwrxTu hAki naya mADu. 
\emng
\eentry

\bentry
\word[grout(3)]{grout}
\pron{gwrxTf}
\gl{\sakirx}
\bmng
 (haMdigaLu maNuNx \mo vanunx) mUtiyiMda -- age, toVDu, etitxhAku, ebibxsu (\akirx\ saha) (\rUpa\ saha). 
\emng
\eentry

\bentry
\word[grout(4)]{grout}
\pron{gwrxTf}
\gl{\nA}
\bmng
 toVpu; maragaLa cikakx guMpu. 
\emng
\eentry

\bentry
\word{groved}
\pron{gorxVvfDx}
\gl{\gu}
\bmng
\bnum
\num{1} toVpugaLiruva. 
\num{2} toVpAgisiruva; guMpuguMpAgi maragaLanunx neTiTxruva; toVpu(gaLanunx) beLesiruva. 
\num{3} (hakikxya \vi) toVpuvAsi; toVpinalilx vAsisuva: \eng{the groved nightingales} toVpinalilx vAsisuva neYTiMgeVlf hakikxgaLu. 
\enum
\emng
\eentry

\bentry
\word{grovel}
\pron{gArxvalf}
\gl{\akirx}
\expl{(\BU\ matutx \BUkaq\ \eng{grovelled,} \vakaq\ \eng{grovelling}).}
\bmng
\bnum
\num{1} muKavaDiyAgi malagu; Dabubx malagu; boVralu biVLu. 
\num{2} aMgalAcu; deYnayx toVru; aDiyALaMte vatiRsu: \eng{grovel in the dirt} (\engit{or} \eng{dust)} koLakinalilx (yA dhULinalilx) bidudx horaLu; atideYnayx toVru; aMgalAcu. 
\enum
\emng
\eentry

\bentry
\word{groveler}
\pron{gorxVvalarf}
\gl{\nA}
\bmng
 (\ame)  = \hyperlink{groveller}{groveller}. 
\emng
\eentry

\bentry
\word{groveless}
\pron{gorxVvflisf}
\gl{\gu}
\bmng
 toVpugaLilalxda; maragiDagaLilalxda. 
\emng
\eentry

\bentry
\word{groveling}
\pron{gorxVvaliMgf}
\gl{\gu}
\bmng
 (\ame)  = \hyperlink{grovelling}{grovelling}. 
\emng
\eentry

\bentry
\word{groveller}
\pron{gorxVvalarf}
\gl{\nA}
\bmng
 (deYnayxdiMda) horaLuvavanu; aDiyALaMte vatiRsuvavanu: \eng{mere worms and grovellers as we are} keVvala huLugaLaMte, deYnayxdiMda horaLuva nAvu. 
\emng
\eentry

\bentry
\word{grovelling}
\pron{gorxVvaliMgf}
\gl{\gu}
\bmng
 hiVna; diVna; tucaCx; nikaqSaTx. 
\emng
\eentry

\bentry
\word{grovellingly}
\pron{gorxVvaliMgfli}
\gl{\kirxvi}
\bmng
 hiVnavAgi; aMgalAcutatx; deYnayxdiMda; tucaCxvAgi; nikaqSaTx riVtiyalilx. 
\emng
\eentry

\bentry
\word{grovy}
\pron{gorxVvi}
\gl{\gu}
\bmng
\bnum
\num{1} toVpina; toVpige saMbaMdhisida yA toVpinaMtiruva. 
\num{2} toVpu tuMbida; maragiDagaLa guMpiniMda, toVpiniMda -- tuMbiruva. 
\num{3} toVpina naDuvaNa; maragiDagaLa guMpina madheyx iruva. 
\num{4} toVpuvAsiyAda; toVpinalilx -- nelasiruva, vAsisuva. 
\enum
\emng
\eentry

\bentry
\word{grow}
\pron{gorxV}
\gl{\kirx}
\bmng
(\BU\ \eng{grew} \ucAcx\ gUrx; \BUkaq\ \eng{grown, have} yA \eng{be} joteyalilx).
\emng

\noindent
\gl{\sakirx}
\bmng
\bnum
\num{1} (\kaparx) (yAvudAdarU peYru, beLe, \mo vugaLiMda) tuMbiru; mucicxru; Avarisiru: \eng{the field was grown up with corn} hola dhAnayxda peYriniMda tuMbitutx. 
\num{2} (giDagaLu, haNuNx, uNeNx, \mo vanunx vayxvasAyadiMda, kaqSiyiMda) beLesu; utapxtitxmADu. 
\num{3} (gaDaDx \mo vanunx) beLesu. 
\enum
\emng

\noindent
\gl{\akirx}
\bmng
\bnum
\num{1} (sajiVva sasayxdaMte) beLe; aBivaqdidhxyAgu. 
\num{2} (sasayxda \vi) badukiru; jiVvisiru; jiVva taLediru. 
\num{3} (\hA) (nijiRVva vasutx \mo vugaLa \vi) (oMdu sathxLadalilx) iru; dore; sikukx. 
\num{4} moLe; konaru; moLakeyiDu; moLakeyeVLu. 
\num{5} moLe; meVlakekxVLu; udaBxvisu; utapxnanxvAgu. 
\num{6} sAvxBAvikavAgi -- huTuTx, janisu. 
\num{7} taleyetutx; udayisu. 
\num{8} (gAtarxdalilx, etatxradalilx) beLe; doDaDxdAgu. 
\num{9} (motatx, dajeR, shakitx, adhikAragaLalilx) hecAcxgu; beLe; adhikavAgu; vadhiRsu. 
\num{10} karxmeVNa Agu; karxmakarxmavAgi -- beLe, aBivaqdidhxyAgu: \eng{grow rich} karxmeVNa dhanikanAgu. 
\num{11} (neladalilx, maNiNxnalilx) huTuTx; moLe; saMpUNaRvAgi beLe, AkArapaDe, aBivaqdidxge baru: \eng{mosquitoes grow in swamps} soLeLxgaLu jwguparxdeVshadalilx huTuTxtatxve. 
\num{12} (padadhxti, saMparxdAya, \mo vugaLa \vi) taledoVru; taleyetutx; vADikege, rUDhige -- baru: \eng{a wicked practice had grown up} oMdu duSaTx padadhxti rUDhige baMditutx. \eng{a troublesome situation had grown up} oMdu toMdareya parisithxti taledoVritutx. 
\enum
\emng

\noindent
\gl{\pagu}
\bmng
\bnum
\numi{1} \eng{growing pains} 
\banum
\alnum{a} beLavaNige noVvu; beLeyuva kAladalilx cikakxvarige sAmAnayxvAgi (kAlinalilx) kANisakoLuLxva naragaLa noVvu. 
\alnum{b} (\rUpa) bAlAriSaTx; bAlagarxha; udayxma \mo vugaLa beLavaNigeyalilx modalu modalige kANisikoLuLxva kaSaTxnaSaTxgaLu. 
\eanum
\numie
\num{2} \eng{growing season} beLavaNige kAla; vadhaRnakAla; beLeyuva kAla; maLe matutx tApagaLu giDamaragaLu beLeyalu anuvu mADikoDuva kAla. 
\num{3} \eng{grow into one, together, etc.} oMdAgi beLe; oTATxgi, oMdAgi -- kUDiko. 
\numi{4} \eng{grow out of} 
\banum
\alnum{a} baTeTx, pAdarakeSx, \mo vugaLanunx mIri beLe; (baTeTx, pAdarakeSx, \mo vugaLa \vi) hAkikoLuLxvavanige cikakxdAgu. 
\alnum{b} yAvudaradeV pariNAmavAgiru; PalavAgiru. 
\alnum{c} doDaDxvanAgu; beLe; pwrxDhanAgu; huDugATa \mo vugaLanunx uLisikoLuLxva vayasusx dATu, mIru. 
\eanum
\numie
\numi{5} \eng{grow up} 
\banum
\alnum{a} doDaDxvanAgu; pwrxDhanAgu; vayasakxnAgu. 
\alnum{b} (\kanmu\ vidhirUpadalilx) viveVkadiMda vatiRsu. 
\alnum{c} (padadhxti, saMparxdAya) huTuTx; beLe; parxcalitavAgu; rUDhige baru. 
\eanum
\numie
\enum
\emng

\noindent
\gl{\nuga}
\bmng
\bnum
\numi{1} \eng{grow downwards} 
\banum
\alnum{a} iLi; keLagaDege hoVgu. 
\alnum{b} iLi; iLiduhoVgu; kaDimeyAgu; avanati hoMdu. 
\eanum
\numie
\numi{2} \eng{grows on} (\engit{or} \eng{upon)} 
\banum
\alnum{a} (obabxna meVle) hecucx hecucx parxBAva biVruvaMtAgu, hiDita sAdhisu, sAvxmayx paDeduko: \eng{a bad habit grows on a man} yAvudAdarU duraBAyxsa obabxna meVle bahaLa parxBAva biVruvaMtAgutatxde. 
\alnum{b} AkaSiRsu; mecucxgeyAgu; Asakitx -- keraLisu, seLe; (manasasxnunx) serehiDi; pirxVti, vishAvxsa, \mo vanunx gaLisu: \eng{music grows and grows on the more we listen} nAvu saMgiVtavanunx keVLidaSUTx adu namamx Asakitxyanunx seLeyutatxde. \eng{the picture grows on me} A citarx nananx manasasxnunx sere hiDiyutatxde. 
\eanum
\numie
\enum
\emng
\eentry

\bentry
\word{growable}
\pron{gorxVabflf}
\gl{\gu}
\bmng
 beLesabahudAda; beLeyalAguva; beLesalu shakayxvAda. 
\emng
\eentry

\bentry
\word{grower}
\pron{gorxVarf}
\gl{\nA}
\bmng
\bnum
\num{1} (oMdu gotAtxda riVtiyalilx) beLeyuva sasayx. 
\num{2} beLegAra; beLeyuvavanu; beLe mADuvavanu: \eng{fruit grower} haNuNx beLegAra; haNuNx beLeyuvavanu. 
\enum
\emng

\noindent
\gl{\pagu}
\bmng
\bnum
\num{1} \eng{fast grower} beVga beVga, kiSxparxvAgi beLeyuva giDa. 
\num{2} \eng{free grower} savxtaMtarxvAgi, Asharxyavilalxde -- beLeyuva giDa. 
\enum
\emng
\eentry

\bentry
\word[growl(1)]{growl}
\pron{gwrxlf}
\gl{\sakirx}
\bmng
 gurerxnunxtatx (EnanAnxdarU) -- vayxkatxpaDisu, sUcisu, heVLu: \eng{she growls out her discontent} avaLu guruguTuTxtAtx tananx ataqpitx sUcisutAtxLe. 
\emng

\noindent
\gl{\akirx}
\bmng
\bnum
\num{1} (pArxNigaLu) gurerxnunx; koVpadiMda guruguru shabadxmADu. 
\num{2} (PiraMgi, BUkaMpa, guDugu, \mo vugaLa \vi) moLagu; guDugu. 
\num{3} (manuSayxra \vi) (koVpadiMda) guruguTuTx; gurerxnunx; asamAdhAnadiMda gurerxnunxtAtx goNagADu yA reVgADu. 
\enum
\emng
\eentry

\bentry
\word[growl(2)]{growl}
\pron{gwrxlf}
\gl{\nA}
\bmng
\bnum
\num{1} gurf; pArxNigaLu guruguTuTxva shabadx. 
\num{2} (PiraMgi, BUkaMpa, guDugu, \mo vugaLa) moLagu; guDugATa; guDugu. 
\num{3} (manuSayxra \vi) (koVpadiMda) guruguTuTxvike; gurerxnunxva shabadx; asamAdhAnada goNagATa; guruguTuTxva reVgATa. 
\enum
\emng
\eentry

\bentry
\word{growler}
\pron{gwrxlarf}
\gl{\nA}
\bmng
\bnum
\num{1} guruguTuTxga; guruguTuTxva, gurerxnunxva -- pArxNi, vayxkitx, \mo vu. 
\num{2} (\birx) (\ca) (nAlukx cakarxda) bADige savAri baMDi. 
\num{3} kelavu mInu jAtigaLu. 
\num{4} cikakx himaguDaDx; niVrinalilx teVluva saNaNx niVgaRlalx guDaDx. 
\num{5} (biyaru \mo vanunx tuMbuva) siVse; hUji; kUja. 
\num{6} (\Bwvi) gwrxlaru; moTaku maMDala\eng{(short circuit)}vAgiruva taMti suruLigaLanunx patetxhacucxva oMdu sAdhana. 
\enum
\emng
\eentry

\bentry
\word{growlery}
\pron{gwrxlari}
\gl{\nA}
\bmng
\bnum
\num{1} moreta; guDugu; guruguTuTxva shabadx. 
\num{2} (pArxNigaLu guruguTuTxva) gavi; guhe. 
\num{3} (\hA) koVpagaqha; munisukoVNe; yArAdarU EkAMtavAgi koVpa tApAdigaLanunx vayxkatxpaDisatakakx savxMta koVNe. 
\enum
\emng
\eentry

\bentry
\word{grown}
\pron{gorxVnf}
\gl{\gu}
\bmng
 beLeda; vayasakx; vayasisxge, pArxyakekx baMda: \eng{grown man} pwrxDha; pArxyakekx baMda manuSayx. 
\emng
\eentry

\bentry
\word[grown-up(1)]{grown-up}
\pron{gorxVnfapf, gorxVnapf}
\gl{\gu}
\bmng
 beLeda; hareyada; vayasisxge, pArxyakekx baMda. 
\emng
\eentry

\bentry
\word[grown-up(2)]{grown-up}
\pron{gorxVnfapf, gorxVnapf}
\gl{\nA}
\bmng
 pwrXDha; vayasakx; pArxyakekx baMdavanu; beLedu doDaDxvanAdavanu. 
\emng
\eentry

\bentry
\word{growth}
\pron{gorxVtf}
\gl{\nA}
\bmng
\bnum
\num{1} beLavaNige; beLeyuvudu; beLavu; beLeta; aBivaqdidhx; vaqdidhx. 
\num{2} beLe -- mADuvudu, tegeyuvudu. 
\num{3} beLe; Pasalu; payiru. 
\num{4} (\roVshA) dumARMsa; anAroVgayxkara beLeta. 
\enum
\emng

\noindent
\gl{\pagu}
\bmng
\bnum
\num{1} \eng{full growth} tuMbu beLavaNige; pUNARBivaqdidhx; pUNaRgAtarx; beLeyuvudara pUNaR parxmANa. 
\num{2} \eng{of foreign growth} videVshadalilx beLeda. 
\enum
\emng
\eentry

\bentry
\wordnospeech{growth industry}{growth industry}
\pron{?}
\gl{\nA}
\bmng
 aBivaqdidhxshiVla keYgArike; itara keYgArikegaLigiMta veVgavAgi beLeyutitxruva keYgArike. 
\emng
\eentry

\bentry
\wordnospeech{growth stock}{growth stock}
\pron{?}
\gl{\nA}
\bmng
 vadhaRka sATxku, SeVru; beLeyuva sATxku, SeVru; (lABadalilx sAkaSuTx BAgavanunx udayxmada visatxraNege beLesikoLuLxva niVtiyiMdAgi kaMpaniya sATxku yA SeVrina mwlayx takaSxNakekx hecacxde) muMde hecacxbahudeMdu niriVkiSxsabahudAda sATxku, SeVru. 
\emng
\eentry

\bentry
\word[groyne(1)]{groyne}
\pron{gArxinf}
\gl{\nA}
\bmng
 karegoVDe; tiVragoVDe; taDegaTuTx; taDegoVDe; aDaDxkaTeTx; taDedaMDe; alegaLu taMda maraLu, kalulx, \mo vu karemIri oLanugagxdaMte taDeyuvudakAkxgi kaTiTxda marada cwkaTuTx yA agalavAda tagugxgoVDe. 
\emng
\eentry

\bentry
\word[groyne(2)]{groyne}
\pron{gArxinf}
\gl{\sakirx}
\bmng
 (karAvaLiyalilx) taDegoVDe, taDegaTuTx -- hAku; aDaDxkaTeTx nimiRsu. 
\emng
\eentry

\bentry
\word[grub(1)]{grub}
\pron{garxbf}
\gl{\nA}
\bmng
\bnum
\num{1} lAvAR; kiVTada mari; marihuLu. 
\num{2} kaMbaLihuLu; koVrihuLu. 
\num{3} (\pArxparx) daDaDxnAda kaSaTxjiVvi; katetxyaMte duDiyuvavanu. 
\num{4} (\pArxparx) jiVtagAranaMte duDiyuva barahagAra. 
\num{5} (\pArxparx) aMdageVDi; koLaka; shavxpaca. 
\num{6} (\ashi) tiMDi; tinisu; UTa; AhAra. 
\enum
\emng
\eentry

\bentry
\word[grub(2)]{grub}
\pron{garxbf}
\gl{\kirx}
\expl{(\BU\ matutx \BUkaq\ \eng{grubbed,} \vakaq\ \eng{grubbing}).}
\bmng
\emng

\noindent
\gl{\sakirx}
\bmng
\bnum
\num{1} meVle meVle -- age, toVDu, kukukx. 
\num{2} kULe, beVru, \mo vanunx kitutx (nelavanunx) cokakxTapaDisu. 
\num{3} (kULe, beVru, \mo vanunx) kitutxhAku. 
\num{4} (nelavanunx) agedu -- horadege, shoVdhisu, huDuku. 
\num{5} (\rUpa) pusatxka \mo vugaLalilx -- huDuku, huDuki tege. 
\num{6} (\ashi) tinunx; uNuNx. 
\num{7} (UTada girAki \mo varige) UTakikxDu; uNabaDisu. 
\enum
\emng

\noindent
\gl{\akirx}
\bmng
\bnum
\num{1} huDukADu; taDakADu; anevxVSaNe mADu. 
\num{2} meYmuriya duDi; gulAma cAkari mADu; katetxcAkari mADu. 
\num{3} (\ashi) tinunx; AhAra seVvisu. 
\enum
\emng
\eentry

\bentry
\word{grub-axe}
\pron{garxbfAYxkfsx}
\gl{\nA}
\bmng
 kULekoDali; kULe tegeyalu baLasuva koDali. 
\emng
\eentry

\bentry
\word{grubber}
\pron{garxbarf}
\gl{\nA}
\bmng
\bnum
\num{1} ageyuvavanu; toVDuvavanu. 
\num{2} jiVtadALu; katetx cAkara. 
\num{3} (\ashi) haNavanunx lapaTAyisuvavanu; duDuDx tiMduhAkuvavanu. 
\num{4} kULe \mo vanunx kitutx hAkuva sAdhana, salakaraNe. 
\enum
\emng
\eentry

\bentry
\word{grubbily}
\pron{garxbili}
\gl{\kirxvi}
\bmng
 koLakukoLakAgi. 
\emng
\eentry

\bentry
\word{grubbiness}
\pron{garxbinisf}
\gl{\nA}
\bmng
 koLakutana; gabubxtana; kashamxlate; shavxpaca sithxti. 
\emng
\eentry

\bentry
\word{grubby}
\pron{garxbi}
\gl{\gu}
\bmng
\bnum
\num{1} huLumarigaLa; huLumarigaLiMda tuMbida. 
\num{2} koLakAda; gabAbxda; kashamxlavAda. 
\enum
\emng
\eentry

\bentry
\word{grub-hoe}
\pron{garxbfhoV}
\gl{\nA}
\bmng
 kULegudadxli; kULe kiVLuva gudadxli. 
\emng
\eentry

\bentry
\word{grub-hook}
\pron{garxbfhukf}
\gl{\nA}
\bmng
 kULekokekx; kULe, moVTu, \mo vanunx kitutxhAkalu baLasuva sAdhana. 
\emng
\eentry

\bentry
\word{grub-screw}
\pron{garxbfsUkxrX}
\gl{\nA}
\bmng
 boVLusUkxrX; taleyilalxda (Adare sUkxrXDerxYvarf iTuTx tirugisalu gADi iruva) sUkxrX. 
\emng
\eentry

\bentry
\word[grub-stake(1)]{grub-stake}
\pron{garxbfseTxVkf}
\gl{\sakirx}
\bmng
 (\ame, \ashi) ananxbaTeTx odagisu; (hiMde KanijashoVdhakanige odagisutitxdadxMte obabxnige siguva lABadalilx pAlanunx niVDuvudakekx parxtiPalavAgi yA adaralilx iMtiSuTx pAlu koDabeVkeMba opapxMdada meVrege avanige) davasadhAnayx, baTeTxbare, \mo vanunx niVDu. 
\emng
\eentry

\bentry
\word[grub-stake(2)]{grub-stake}
\pron{garxbfseTxVkf}
\gl{\nA}
\bmng
 (\ame) (\ashi) ananxbaTeTx; hiMde KanijashoVdhakanige mADutitxdadxMte obabxnige sikukxva lABadalilx pAlanunx niVDuvudakekx parxtiPalavAgi yA adaralilx iMtiSuTx pAlu koDabeVkeMba opapxMdada meVrege odagisuva davasadhAnayx, baTeTxbare, \mo vu. 
\emng
\eentry

\bentry
\word(Grub Street[1]){Grub Street}
\pron{?}
\gl{\nA}
\bmng
 garxbfsiTxrXVTf: 
\banum
\alnum{a} (laMDaninxna) gariVba garxMthakataRru, matutx jiVtada leVKakaru vAsisutitxdadx parxdeVsha. 
\alnum{b} baDa garxMthakAraru matutx jiVtada leVKakaru. 
\eanum
\emng
\eentry

\bentry
\word(Grub Street[2]){Grub Street}
\pron{?}
\gl{\gu}
\bmng
\bnum
\num{1} gariVba garxMthakataRra yA jiVtada leVKakara. 
\num{2} kiVLudajeRya; kaDimeguNada; keLamaTaTxda: \eng{a Grub Street book} keLamaTaTxda pusatxka. 
\enum
\emng
\eentry

\bentry
\word[grudge(1)]{grudge}
\pron{garxjf}
\gl{\sakirx}
\bmng
\bnum
\num{1} (vasutxvanunx vayxkitxge) koDalu, dAnamADalu, biTuTxkoDalu -- samamxtisadiru, anumatisadiru. 
\num{2} kosari kosari koDu; kApaRNayxpaDu. 
\num{3} tALadiru; sahisadiru; asUyepaDu; karubu: \eng{you come to grudge even the sun for shining} sUyaRnu beLaguvudanUnx niVnu karubutitxVye. 
\enum
\emng
\eentry

\bentry
\word[grudge(2)]{grudge}
\pron{garxjf}
\gl{\nA}
\bmng
\bnum
\num{1} Cala; devxVSa; hagetana; viroVdha; asamAdhAna: \eng{have} (\ame\ \eng{hold) a grudge against} obabxna virudadhx devxVSa sAdhisu. \eng{bear (one) a grudge} obabxna \vi\ viroVdha tALu. \eng{owe (one) a grudge} (obabxna meVle) hagetana sAdhisu. 
\num{2} asUye; karubu. 
\enum
\emng
\eentry

\bentry
\word{grudging}
\pron{garxjiMgf}
\gl{\gu}
\bmng
 manasisxlalxda; samAdhAnavilalxda; olalxda manasisxna: \eng{a grudging acceptance} olalxda manasisxna opipxta. 
\emng
\eentry

\bentry
\word{grudgingly}
\pron{garxjiMgfli}
\gl{\kirxvi}
\bmng
 olalxda manasisxniMda; manasisxlalxda riVtiyalilx; samAdhAnavilalxde; beVkubeVDadaMte. 
\emng
\eentry

\bentry
\word[gruel(1)]{gruel}
\pron{gUrxalf}
\gl{\nA}
\bmng
\bnum
\num{1} (roVgigaLigAgi rave, hiTuTx, \mo vanunx hAlinalAlxgali, niVrinalAlxgali beVyisi mADuva) gaMji; aMbali; kaLi. 
\num{2} (\birx) (\pArxparx) shikeSx; daMDane. 
\num{3} (\birx) (\pArxparx) soVlu; apajaya; parABava. 
\enum
\emng

\noindent
\gl{\pagu}
\bmng
\bnum
\numi{1} \eng{have} (\engit{or} \eng{get) one's gruel} 
\banum
\alnum{a} shikeSxge guriyAgu. 
\alnum{b} ugarxvAda soVlanunx hoMdu yA koleyAgu. 
\eanum
\numie
\numi{2} \eng{give one his gruel} 
\banum
\alnum{a} (obabxnige) shikeSxkoDu. 
\alnum{b} (obabxnanunx) soVlisu. 
\alnum{c} (obabxnanunx) koMdu hAku. 
\eanum
\numie
\enum
\emng
\eentry

\bentry
\word[gruel(2)]{gruel}
\pron{gUrxalf}
\gl{\sakirx}
\expl{(\BU\ matutx \BUkaq\ \eng{gruelled,} \vakaq\ \eng{gruelling}).}
\bmng
 (\birx) shakitxguMdisu; dubaRlagoLisu; balahiVnamADu; nishayxkitxgoLisu. 
\emng
\eentry

\bentry
\word[{grue(l)ling}(1)]{grue(l)ling}
\pron{gUrxaliMgf}
\gl{\nA}
\bmng
 shakitxhArxsaka; shakitxguMdisuva, susutxmADuva yAvudeV vidhAna yA anuBava. 
\emng
\eentry

\bentry
\word[{grue(l)ling}(2)]{grue(l)ling}
\pron{gUrxaliMgf}
\gl{\gu}
\bmng
\bnum
\num{1} susutx mADuva; shakitxguMdisuva; nishayxkitxgoLisuva; balahiVnamADuva. 
\num{2} tArxsadAyaka; bahukaSaTxkaravAda: \eng{a grue(l)ling race} kaDukaSaTxda paMdayx. 
\enum
\emng
\eentry

\bentry
\word{gruesome}
\pron{gUrxsamf}
\gl{\gu}
\bmng
 BayaMkara; biVBatasx; asahayx; GoVra. 
\emng
\eentry

\bentry
\word{gruesomely}
\pron{gUrxsamfli}
\gl{\kirxvi}
\bmng
 BayaMkaravAgi; biVBatasxvAgi; asahayxvAda riVtiyalilx. 
\emng
\eentry

\bentry
\word{gruesomeness}
\pron{gUrxsamfnisf}
\gl{\nA}
\bmng
 BayaMkarate; biVBatasxte; asahayxvAgiruvike. 
\emng
\eentry

\bentry
\word{gruff}
\pron{garxphf}
\gl{\gu}
\bmng
\bnum
\num{1} siDukina; gadaruva. 
\num{2} moTakumAtina; mitanuDiya; mitavAkikxna; hecucx mAtilalxda; mitavAda mAtina. 
\num{3} oraTutanada; oraTu naDateya. 
\num{4} oraTu daniya; kakaRshadaniya. 
\enum
\emng
\eentry

\bentry
\word{gruffly}
\pron{garxphfli}
\gl{\kirxvi}
\bmng
\bnum
\num{1} siDukiniMda; gadaruva riVtiyalilx. 
\num{2} mitanuDiyiMda; moTaku mAtinalilx; hecucx mAtilalxde. 
\num{3} oraTAgi; oraTu naDateyiMda. 
\num{4} kakaRshadaniyalilx; oraTu daniyalilx. 
\enum
\emng
\eentry

\bentry
\word{gruffness}
\pron{garxphfnisf}
\gl{\nA}
\bmng
\bnum
\num{1} gadarisikoLuLxvike; siDukutana. 
\num{2} mitaBASitavx; hecucx mAtilalxdiruvudu. 
\num{3} oraTutana; oraTu naDavaLike. 
\num{4} oraTudani; kakaRshadani. 
\enum
\emng
\eentry

\bentry
\word[grumble(1)]{grumble}
\pron{garxMbflf}
\gl{\sakirx}
\bmng
\bnum
\num{1} goNagu; asamAdhAna -- paDu, toVru, parxkaTisu. 
\num{2} AkeSxVpaNe mADutatx, dUrutatx -- heVLu; asamAdhAnadiMda heVLu. 
\enum
\emng

\noindent
\gl{\akirx}
\bmng
\bnum
\num{1} goNagu; goNaguTuTx. 
\num{2} asapxSaTxvAgi gurerxnunx. 
\num{3} (guDugu \mo vugaLa \vi) moLagu; guDugu; gajiRsu; gajaRne mADu. 
\num{4} dUru; AkeSxVpaNe mADu. 
\enum
\emng
\eentry

\bentry
\word[grumble(2)]{grumble}
\pron{garxMbflf}
\gl{\nA}
\bmng
\bnum
\num{1} goNagATa; goNagu. 
\num{2} dUru; AkeSxVpaNe. 
\enum
\emng
\eentry

\bentry
\word{grumbler}
\pron{garxMbalxrf}
\gl{\nA}
\bmng
\bnum
\num{1} dUriga; AkeSxVpaNegAra; dUruvavanu. 
\num{2} goNaga; goNaguvavanu. 
\enum
\emng
\eentry

\bentry
\word{grumbling}
\pron{garxMbilxMgf}
\gl{\gu}
\bmng
\bnum
\num{1} AkeSxVpisuva; dUruva. 
\num{2} goNaguTuTxva; goNagADuva. 
\enum
\emng
\eentry

\bentry
\wordnospeech{grumbling appendix}{grumbling appendix}
\pron{?}
\gl{\nA}
\bmng
 (\AmA) aMtarxpucaCxroVga (\eng{appendicitis}) Agade padeV padeV hoTeTx noVvu uMTumADuva karuLuvALa. 
\emng
\eentry

\bentry
\word{grumblingly}
\pron{garxMbilxMgfli}
\gl{\kirxvi}
\bmng
\bnum
\num{1} dUrutAtx; AkeSxVpisutAtx. 
\num{2} goNaguTuTxtAtx; goNagADutAtx. 
\enum
\emng
\eentry

\bentry
\word{grume}
\pron{gUrxmf}
\gl{\nA}
\bmng
 (\veYshA) 
\bnum
\num{1} rakatxda garaNe, hepupx, gaDeDx. 
\num{2} aMTu; sinxgadhxdarxva; jiguTudarxva; gaTiTxyAda yA aMTaMTAda darxva. 
\enum
\emng
\eentry

\bentry
\word{grummet}
\pron{garxmiTf}
\gl{\nA}
\bmng
\bnum
\numi{1} garxmeTuTx: 
\banum
\alnum{a} (\nw) bigiyuvudakAkxgi yA doVNiya huTaTxnunx doVNiya pakakxkekx lagatitxsuvudakAkxgi yA metetxyAgi baLasuva hurihagagxda kuNike. 
\alnum{b} (loVhada vasutxvina raMdharxdalilx tUruva) viduyxdAvxhakada sutatx hAkuva viduyxdorxVdhaka vASaru, bilelx. 
\alnum{c} (seYnikana ToVpiyu baLukadaMte adaroLage hAkuva) rababxrf, loVha, baTeTx, \mo vugaLiMda mADida -- uMgura. 
\eanum
\numie
\enum
\emng
\eentry

\bentry
\word{grumous}
\pron{gUrxmasf}
\gl{\gu}
\bmng
\bnum
\num{1} rakatxda garaNeyiMda -- kUDida, Ada. 
\num{2} sinxgadhxdarxvadiMda -- kUDida, Ada. 
\num{3} rakatxda garaNeyanunx hoVluva; rakatxda garaNeyaMtha. 
\num{4} sinxgadhxdarxvavanunx hoVluva; sinxgadhxdarxvadaMtha. 
\enum
\emng
\eentry

\bentry
\word{grump}
\pron{garxMpf}
\gl{\nA}
\bmng
\bnum
\num{1} (\AmA) siDuka; muMgoVpi. 
\num{2} (\bava dalilx) reVgATa; reVgu; reVgina keraLu. 
\enum
\emng
\eentry

\bentry
\word{grumpily}
\pron{garxMpili}
\gl{\kirxvi}
\bmng
 koVpadiMda; muMgoVpadalilx; siDukiniMda; reVgibiVLutAtx. 
\emng
\eentry

\bentry
\word{grumpiness}
\pron{garxMpinisf}
\gl{\nA}
\bmng
 muMgoVpa; siDukutana; koVpasavxBAva; reVgATa. 
\emng
\eentry

\bentry
\word{grumpish}
\pron{garxMpiSf}
\gl{\gu}
\bmng
  = \hyperlink{grumpy}{grumpy}. 
\emng
\eentry

\bentry
\word{grumpy}
\pron{garxMpi}
\gl{\gu}
\bmng
 muMgoVpada; koVpa savxBAvada; siDukina; reVgibiVLuva. 
\emng
\eentry

\bentry
\word{Grundy}
\pron{garxMDi}
\gl{\nA}
\bmng
 mahAkaMdAcAri; atireVkada shiSATxcAri; vADikeya kaMdAcAravanunx, naDenuDigaLalilx ati shiSaTxteyanunx, mahAmaDivaMtikeyanunx parxtinidhisuva vayxkitx. 
\emng

\noindent
\gl{\pagu}
\bmng
 \eng{Mrs. Grundy} = \hyperlink{Grundy}{Grundy}. 
\emng
\eentry

\bentry
\word{Grundyism}
\pron{garxMDiisaZmf}
\gl{\nA}
\bmng
 mahAshiSaTxte; atireVkada shiSATxcAra; kaqtaka saMBAvitatana; kaTuTxniTATxda kaMdAcAra. 
\emng
\eentry

\bentry
\word{grunion}
\pron{garxnayxnf}
\gl{\nA}
\bmng
 tatitxyiDalu samudarxtiVrakekx baruva kAYxliphoVniRyada oMdu saNaNx mInu. 
\emng
\eentry

\bentry
\word[grunt(1)]{grunt}
\pron{garxMTf}
\gl{\sakirx}
\bmng
guruguTuTxtAtx, asamAdhAnadiMda heVLu. 
\emng

\noindent
\gl{\akirx}
\bmng
\bnum
\num{1} (haMdiya \vi) Durukf enunx; roMkiDu; guruguTuTx. 
\num{2} (haMdiyaMte gurerxnunxtatx) asamAdhAna, asamamxti, AyAsa, \mo vanunx -- toVrisu. 
\enum
\emng
\eentry

\bentry
\word[grunt(2)]{grunt}
\pron{garxMTf}
\gl{\nA}
\bmng
\bnum
\num{1} haMdiya `Durukf' shabadx; (haMdiya) roMke. 
\hypertarget{grunt(2)2}{} 
\num{2} Durukf mInu; pamaDasiDeV vaMshada, hiDidukoMDare Durukf eMba sadudxmADuva, amerikada mInu. 
\enum
\emng
\eentry

\bentry
\word{grunter}
\pron{garxMTarf}
\gl{\nA}
\bmng
\bnum
\num{1} (haMdiyaMte) Durukf enunxvava(Lu), guruguTuTxvavanu(Lu). 
\num{2} haMdi. 
\num{3}  = \hyperlink{grunt(2)2}{$^2$grunt (2)}. 
\enum
\emng
\eentry

\bentry
\word{Grunth}
\pron{garxMtf}
\gl{\nA}
\bmng
 \eng{Granth} eMbudara rUpAMtara. 
\emng
\eentry

\bentry
\word{gruntled}
\pron{garxMTflfDx}
\gl{\gu}
\bmng
 (\AmA) KuSiyAda; taqpatxnAda. 
\emng
\eentry

\bentry
\word{Gruyere}
\pron{gUrxyeVrf}
\gl{\nA}
\bmng
 gUrxyeVrf (giNuNx); modalige sivxTasxleRMDina hasugaLa hAlina, haLadi biLacu baNaNxda, saNaNx gUDugaLuLaLx giNuNx. 
\emng
\eentry

\bentry
\word{gryphon}
\pron{girxphanf}
\gl{\nA}
\bmng
  = \hyperlink{griffin(2)}{$^2$griffin}. 
\emng
\eentry

\bentry
\word{grysbok}
\pron{girxsfbAkf}
\gl{\nA}
\bmng
 \da\ Aphirxkada, bUdu baNaNxda, saNaNx jiMke. 
\emng
\eentry

\bentry
\wordnospeech{gs.}{gs.}
\pron{?}
\gl{\saMkiSx}
\bmng
 (\birx) \eng{guineas.} 
\emng
\eentry

\bentry
\word{G-string}
\pron{jiVsiTxrXMgf}
\gl{\nA}
\bmng
\bnum
\num{1} (\saM) jiV -- taMti; \eng{G -- }savxravanunx miDisuva piTiVlu \mo vugaLa taMti. 
\hypertarget{G-string(2)}{} 
\num{2} (amerikanf, iMDiyanf meVLa gAyaki \mo varu dharisuva) kwpiVna; laMgoVTi. 
\num{3} uDidAra; kwpiVna kaTaTxlu soMTakekx kaTiTxkoLuLxva dAra. 
\enum
\emng
\eentry

\bentry
\word{G-suit}
\pron{jiVsU(sUyx)Tf}
\gl{\nA}
\bmng
 jiV -- poVSAku; jiV -- sUTu; atayxMta veVgadalilx hoVguvAga vimAnacAlakaru \mo varu hAkikoLuLxva, gurutAvxkaSaRNa balagaLanunx taDedukoLuLxvaMte tayArisida, deVhada vividha aMgagaLa meVle tAneV tAnAgi tuMbikoLuLxvaMtha ciVlagaLuLaLx poVSAku. 
\emng
\eentry

\bentry
\wordnospeech{GT}{GT}
\pron{?}
\gl{\saMkiSx}
\bmng
 \eng{gran turismo.} 
\emng
\eentry

\bentry
\wordnospeech{Gt.}{Gt.}
\pron{?}
\gl{\saMkiSx}
\bmng
 \eng{Great.} 
\emng
\eentry

\bentry
\word{guacharo}
\pron{gAvxcaroV}
\gl{\nA}
\expl{(\bava\ \eng{guacharos}).}
\bmng
\da\ amerikada -- eNeNxhakikx, teYlapakiSx. 
\emng
\eentry

\bentry
\word{guaiac}
\pron{gevxYAyXkF}
\gl{\nA}
\bmng
gevxYAyXkF: 
\banum
\alnum{a} gevxYakamf kulada kelavu jAtiya maragaLiMda dorakuva, auSadhiyalilx baLasuva, kaMdu hasuru dAru. 
\alnum{b} gevxYakamf kulada kelavu maragaLiMda baruva rALa, aMTu. 
\alnum{c} adariMda tayArisida auSadhi. 
\eanum
\emng
\eentry

\bentry
\word{guaiacum}
\pron{gevxYakamf}
\gl{\nA}
\bmng
\bnum
\num{1} gevxYakamf; vesfTx iMDiVsfnalilx iruva, \sA\ niVli hU biDuva oMdu sasayxkula. 
\num{2}  = \hyperlink{guaiac}{guaiac}. 
\enum
\emng
\eentry

\bentry
\word{guan}
\pron{gAvxnf}
\gl{\nA}
\bmng
 gAvxnf; kArxYxsiDeV vaMshakekx seVrida, koVLi baLagada, uSaNxvalaya amerikada oMdu hakikx. 
\emng
\eentry

\bentry
\word{guana}
\pron{gAvxna}
\gl{\nA}
\bmng
\bnum
\num{1} uDa; marada meVle vAsisuva, doDaDx jAtiya halilx. 
\num{2} yAvudeV jAtiya doDaDx halilx (= \hyperref{kandict_i.pdf}{I}{iguana}{iguana}.) 
\enum
\emng
\eentry

\bentry
\word{guanaco}
\pron{gavxnAkoV}
\gl{\nA}
\expl{(\bava\ \eng{guanacos}). }
\bmng
gavxnAko; \da\ amerikada keMgaMdu baNaNxda uNeNx koDuva lAma (pArxNi). \imglink{guanacofigure}{\raisebox{-0.15cm}[0pt][0pt]{\pdfimage width 0.8cm height 0.6cm {G_Pictures/guanaco.jpg}}} 
\emng
\eentry

\bentry
\word{guanine}
\pron{gAvxniVnf}
\gl{\nA}
\bmng
 (\jiVra) gAvxniVnf; elalx jiVvigaLalUlx kaMDubaruva, \eng{DNA} matutx \eng{RNA}gaLa saMyoVjaneyalilx oMdu GaTakavAgiruva, puyxriVnfna janayxvAda neYTorxVjanfyukatx saMyukatx. 
\emng
\eentry

\bentry
\word[guano(1)]{guano}
\pron{gAvxnoV}
\gl{\nA}
\bmng
\bnum
\num{1} (\bava\ \eng{guanos}). gAvxnoV; (\kanmu\ peru deVshada baLiya divxVpagaLalilx doreyuva, gobabxravAgi baLasuva) kaDalu koVLiya hikekx; hikekxgobabxra. 
\num{2} oMdu kaqtaka gobabxra, \kanmu\ mIniniMda tayArisidudx. 
\enum
\emng
\eentry

\bentry
\word[guano(2)]{guano}
\pron{gAvxnoV}
\gl{\sakirx}
\bmng
 gAvxnoV gobabxra hAku; gAvxnoV gobabxradiMda Palavatutx mADu. 
\emng
\eentry

\bentry
\word{guarani}
\pron{gAvxrani}
\gl{\nA}
\bmng
\bnum
\num{1} (\eng{Guarani}) gAvxrani; dakiSxNa amerikada iMDiyananxra buDakaToTxMdara BASe yA sadasayx. 
\num{2} gAvxrani; paragevxV deVshada haNa GaTaka. 
\enum
\emng
\eentry

\bentry
\word[guarantee(1)]{guarantee}
\pron{gAYxraMTiV}
\gl{\nA}
\bmng
\bnum
\num{1} gAyxraMTidAra; gAyxraMTi koDuvavanu; hoNegAra; hoNe hotatxvanu; jAmInudAra; KAtaridAra. 
\num{2} gAyxraMTi; hoNe; jAmInu; KAtari. 
\num{3} gAyxraMTi; IDu; KAtari; AdhAra; niyamapAlanegAgi yA (yAvudoMdara) sithxrate, shAshavxtate, Badarxte, \mo vakAkxgi hoNeyAgi koTiTxdudx, iTaTxdudx, iruvudu, \kanmu\ dAKale: \eng{wealth is no guarantee for happiness} haNadiMda suKa laBisuvudeMba KAtari ilalx. 
\num{4} gAyxraMTigArxhi; hoNe paDedavanu; yArige hoNe koTiTxdeyoV avanu. 
\enum
\emng
\eentry

\bentry
\word[guarantee(2)]{guarantee}
\pron{gAyxraMTiV}
\gl{\sakirx}
\bmng
\bnum
\num{1} (karAru \mo vugaLa pAlanege vasutxvina sAcAtanakekx) hoNeyAgu; hoNeyAgiru; gAyxraMTiyAgu; javAbAdxranAgiru; BaravasekoDu; KAtari koDu; oMdu vasutxvina Badarxtege, sithxratege -- naMbike niVDu, Baravase koDu, KAtari koDu. 
\num{2} saMBavavideyeMdu, saMBavisuvudeMdu -- Baravase koDu, mAtukoDu, KaMDitavAgi heVLu: \eng{I guarantee that I will be there} nAnu alilxrutetxVneMdu KaMDitavAgi heVLutetxVne. 
\num{3} (oMdu vasutxvina sAvxdhiVnavanunx, adanunx paDeyuva hakakxnunx obabxnige koDisuvudAgi) gAyxraMTi koDu; BaravasekoDu. 
\num{4} (apAya \mo vu AgadaMte) gAyxraMTi koDu; Baravase koDu; javAbAdxri vahisu. 
\enum
\emng
\eentry

\bentry
\wordnospeech{guarantee fund}{guarantee fund}
\pron{?}
\gl{\nA}
\bmng
 gAyxraMTi haNa; IDu haNa; IDu TheVvaNi; naSaTx saMBavisabahudAda saMdaBaRdalilx naSaTx tuMbikoLuLxvudakAkxgi IDiTaTx motatx. 
\emng
\eentry

\bentry
\word{guarantor}
\pron{gAYxraMTa(TA)rf}
\gl{\nA}
\bmng
 hoNegAra; jAmInugAra; hoNe hotitxruvavanu. 
\emng
\eentry

\bentry
\word[guaranty(1)]{guaranty}
\pron{gAYxraMTi}
\gl{\nA}
\bmng
\bnum
\num{1} (obabxnu koDabeVkAda sAlakAkxgali mADabeVkAda avashayx kataRvayxkAkxgali javAbAdxriyanunx vahisuvudAgi baravaNigeya mUlaka yA beVre riVtiyalilx matotxbabxnu koTaTx) Baravase; gAYxraMTi; hoNe; KAtari; jAmInu; havAli. 
\num{2} gAYxraMTi; hoNege -- AdhAra, AdhAra mADidudx. 
\enum
\emng
\eentry

\bentry
\word[guaranty(2)]{guaranty}
\pron{gAyxraMTi}
\gl{\sakirx}
\bmng
  = \hyperlink{guarantee(2)}{$^2$guarantee}. 
\emng
\eentry

\bentry
\word[guard(1)]{guard}
\pron{gADfR}
\gl{\nA}
\bmng
\bnum
\num{1} (katitxvarise, muSiTxkALaga, \mo vugaLalilx) rakaSxNABaMgi; kApu varise; AtamxrakaSxNegAgi nilulxva BaMgi yA mADuva calane. 
\num{2} (kirxkeTf) bAyxTina kApunele; rakaSxNe sAthxna; vikeTiTxna rakaSxNegAgi bAyxTanunx hiDiyuva, iDuva -- jAga, sathxLa. 
\num{3} kAvalu; pahare; jAgarUka sithxti; ecacxra; ecacxrike(yiMdiruvudu). 
\num{4} rakaSxka; pAlaka; kApADuvavanu. 
\num{5} kAvalugAra; kAvaliruvavanu; pahareyavanu. 
\num{6} (\ca) (TapAlu baMDiya) rakaSxNAdhikAri. 
\num{7} (\birx) (reYlina) gADuR; meVlivxcAraka. 
\num{8} (\ame) jeYlu pahareyava; seremaneya kAvalugAra. 
\num{9} (bAsekxTf bAlf yA amerikada kAlecxMDATagaLalilx) rakaSxka; rakaSxNe ATagAra. 
\num{10} (\bava dalilx) (iMgelxMDina) aramaneya seYnayx: \eng{foot guards} padAtigaLu; kAlALu seYnikaru. \eng{horse guards} rAvutaru; ashAvxroVhi seYnikaru. 
\num{11} (oMdu sathxLada, vayxkitxya) kApu paDe; rakaSxNA daLa; rakaSxNegAgiruva seYnikaru. 
\num{12} beMgAvalu paDe; aMgarakaSxkadaLa; aMgarakaSxka (seYnika)ru; beMgAvalu BaTaru. 
\num{13} seYnayxda (parxteyxVka) kApu -- daLa, paDe. 
\num{14} (\sA\ saMyukatx padagaLalilx) kApu; rakeSx; keDukAgadaMte, apAya saMBavisadaMte rakiSxsuva salakaraNe: \eng{fire-guard} beMkikApu. \eng{trigger-guard} baMdUkina kudure kApu. 
\num{15} (baLasuvavana keYyanunx rakiSxsuva) katitxhiDikeya BAga. 
\enum
\emng

\noindent
\gl{\pagu}
\bmng
\bnum
\num{1} \eng{advance guard} muMgApina paDe. 
\num{2} \eng{be on guard} = \hyperlink{guard pagu5}{?pagu? \((5)\)}. 
\hyperdef{G}{guard(1) pagu(3)}{} 
\num{3} \eng{guard of honour} gwravarakeSx; (gwrava toVrisuvudakAkxgiruva) gwravadaLa. 
\num{4} \eng{give guard =} \hyperlink{guard pagu10}{?pagu? \((10)\)}. 
\hypertarget{guard pagu5}{} 
\num{5} \eng{keep guard} kAvaliru; pahare iru. 
\num{6} \eng{mount guard} kAvalu vahisiko. 
\num{7} \eng{rear guard} hiMgApina paDe. 
\num{8} \eng{relieve guard} pahare vahisiko; inonxbabxna pahareya avadhi mugida meVle tananx pahareya saradi vahisiko. 
\num{9} \eng{stand guard} = \hyperlink{guard pagu5}{?pagu? \((5)\)}. 
\hypertarget{guard pagu10}{} 
\num{10} \eng{take guard} (kirxkeTf) ATagAranu bAyxTina rakaSxNAjAga nidhaRrisu; kApu neleyAgi yA sathxLadalilx bAyxTanunx UrabeVkeMdu nidhaRrisu. 
\enum
\emng

\noindent
\gl{\nuga}
\bmng
\bnum
\num{1} \eng{off one's guard} (akarxmaNa, Akasimxka, \mo vugaLiMda rakiSxsikoLuLxvudaralilx) huSAru tapipx; ecacxrike tapipx(hoVgi). 
\num{2} \eng{on one's guard} (AkarxmaNa, akasAmxtf GaTane, haThAtAtxda savxMta perxVraNe, \mo vanunx edurisalu) tananx huSArinalilx, ecacxrikeyiMda, sidadhxvAgi -- idudx. 
\enum
\emng
\eentry

\bentry
\word[guard(2)]{guard}
\pron{gADfR}
\gl{\sakirx}
\bmng
\bnum
\num{1} kAvaliru; kApiru; pahareyiru; hoVgi baruvavaranunx hatoVTiyalilxTuTxkoLaLxlu (bAgilu \mo vanunx) kAdiru. 
\num{2} rakaSxNe koDu; kApiDu. 
\num{3} (apAthaRvAgadaMte, durupayoVgavAgadaMte, vivaraNegaLiMda yA karAru kaTaTxLegaLu, \mo vugaLiMda) BadarxpaDisu; rakaSxNe kalipxsu. 
\num{4} (\veYshA) kApinwSadhi koDu; rakaSxkwSadhi niVDu; auSadhadoDane doVSanivArakagaLanunx seVrisi koDu. 
\num{5} (AloVcane, mAtu, \mo vanunx) hatoVTiyalilxDu; niyaMtarxNadalilxDu; hiDitadalilxTuTxko: \eng{he that guards his mouth keeps his life} bAyi hiDitadalilxTuTxkoMDavanu baduki uLidAnu. 
\enum
\emng

\noindent
\gl{\akirx}
\bmng
\bnum
\num{1} (katitxvarise) kApina varise baLasu; ETu tapipxsiko. 
\num{2} (yAvudaradeV virudadhx) ecacxrike vahisu; muMjAgarxtAkarxma keYkoLuLx. 
\num{3} (caduraMga) kApukoDu; (oMdu kAyanunx yA peVdeyanunx matotxMdariMda) rakiSxsu. 
\num{4} (isipxVTu) kApele -- iLi, ese; eleyoMdanunx rakiSxsalu inonxMdu ele hAku. 
\enum
\emng
\eentry

\bentry
\word{guardant}
\pron{gADaRMTf}
\gl{\gu}
\bmng
\bnum
\num{1} (\pArxparx) rakiSxsuva; kApiDuva; rakaSxka; rakaSxNemADuva; joVpAnavAgi noVDikoLuLxva: \eng{guardant sword} kApukatitx; rakASxKaDagx. 
\num{2} (maqgada \vi) perxVkaSxkABimuKa; noVDuvavana kaDe pUtiR muKa tirugisida. 
\num{3} (\vaMlAM) perxVkaSxkABimuKa; deVhavanunx pakakxkUkx muKavanunx saMpUNaRvAgi noVTakanatatx tirugisidaMteyU ketitxda, citirxsida, bareda: \eng{a lion guardant} perxVkaSxkABimuKa siMha; noVTakana kaDe pUtiR muKa tirugisida BaMgiyalilx citirxsida, ketitxda -- siMhAkaqti. 
\enum
\emng
\eentry

\bentry
\word{guard-boat}
\pron{gADfRboVTf}
\gl{\nA}
\bmng
\bnum
\num{1} (baMdarina nwkAdaLakekx kAvalu sariyAgideyeV enunxvudanunx noVDikoLuLxva) gasutx doVNi; pahare doVNi; gasutx tiruguva doVNi. 
\num{2} (suMkada yA kAvxraMTeYnf niyamagaLanunx sariyAgi pAlisalu neVmisiruva) pahare doVNi; (sakARrada) baMdaru doVNi. 
\enum
\emng
\eentry

\bentry
\word{guard-book}
\pron{gADfRbukf}
\gl{\nA}
\bmng
 kApupusatxka; hecucxvari hALegaLu, patarxgaLu, \mo vanunx seVrisalu anukUlavAgiruvaMte racisiruva pusatxka. 
\emng
\eentry

\bentry
\word{guard-chain}
\pron{gADfRceVnf}
\gl{\nA}
\bmng
 kApu sarapaNi; kiseya gaDiyAra, padaka, \mo vanunx BadarxpaDisuva sarapaNi. 
\emng
\eentry

\bentry
\word{guarded}
\pron{gADiRDf}
\gl{\gu}
\bmng
\bnum
\num{1} kAvaliruva; rakiSxta; kAvalugAraniMda (yA avaniMdaloV eMbaMte) rakiSxsalapxTiTxruva, kAyalapxTiTxruva. 
\num{2} (mAtu \mo vugaLa \vi) ecacxrikeya; huSArina; ecacxrada. 
\enum
\emng
\eentry

\bentry
\word{guardedly}
\pron{gADiRDfli}
\gl{\kirxvi}
\bmng
 (AloVcane, mAtu, \mo vugaLa \vi) (joVpAnavAgi) tUkamADi; (huSArAgi) aLedutUgi; ecacxrikeyiMda. 
\emng
\eentry

\bentry
\word{guardedness}
\pron{gADiRDfnisf}
\gl{\nA}
\bmng
\bnum
\num{1} (AloVcane, mAtukate, \mo vanunx) hatoVTiyalilxDuvudu; niyaMtirxsuvudu; hiDitadalilxDuvudu. 
\num{2} ecacxrikeyiMdiruvudu; joVpAnavAgiruvike; huSArinalilxruvudu. 
\enum
\emng
\eentry

\bentry
\word{guardee}
\pron{gADiRV}
\gl{\nA}
\bmng
 (\birx) (\AmA) ThAkuThiVkAda -- kAvalugAra, pahareyava, gADuR. 
\emng
\eentry

\bentry
\word{guardhouse}
\pron{gADfRhwsf}
\gl{\nA}
\bmng
 kApumane; rakASxgaqha: 
\banum
\alnum{a} seVneya kApudaLada nelamane. 
\alnum{b} baMdigaLanunx kUDihAkuva mane. 
\eanum
\emng
\eentry

\bentry
\word{guardian}
\pron{gADiRanf}
\gl{\nA}
\bmng
\bnum
\num{1} kAyuvavanu; kApinavanu. 
\num{2} rakaSxka. 
\num{3} pAlaka; poVSaka. 
\num{4} (\birx) gADiRyanf; diVnarakaSxka; caciRna adhikAra parxdeVsha(pAYxriSf)dalilx yA oMdu pArxMtadalilx baDavara hitakAkxgi iruva kAyidegaLanunx kAyaRgatamADalu cunAyitavAda maMDaliya sadasayx. 
\num{5} \eng{(Guardian)} gADiRyanf; oMdu vAtARpatirxkeya hesaru. 
\num{6} (\nAyxshA) pAlaka; rakaSxka; tananx vahivATugaLanunx naDesikoLaLxlu ashakatxnAgiruva apArxpatxvayasakxna yA huTATx modadxnAgiruvavana rakaSxNeya, yA avana Asitxya rakaSxNeya yA iveraDara javAbAdxri uLaLxvanu. 
\num{7} phArxnisxsakxnf kAnevxMTina -- hiriya, adhipati. 
\enum
\emng
\eentry

\bentry
\wordnospeech{guardian angel}{guardian angel}
\pron{?}
\gl{\nA}
\bmng
 (obabx vayxkitxya, oMdu sathxLada) aBimAna deVvate; rakaSxka deVvate. 
\emng
\eentry

\bentry
\word{guardianship}
\pron{gADiRanfSipf}
\gl{\nA}
\bmng
\bnum
\num{1} rakaSxkana, pAlakana -- adhikAra, sAthxna. 
\num{2} (kAnUniganusAravAda) pAlakatana. 
\num{3} pAlane; rakaSxNe; kApu. 
\enum
\emng

\noindent
\gl{\pagu}
\bmng
 \eng{under the guardianship of the laws} kAnUnugaLu niVDuva rakaSxNeyalilx. 
\emng
\eentry

\bentry
\word{guardless}
\pron{gADfRlisf}
\gl{\gu}
\bmng
\bnum
\num{1} arakiSxta; kApilalxda; rakaSxNe ilalxda; kApADuvavarilalxda; dikikxlalxda; gatiyilalxda: \eng{he thought her guardless} avaLu dikikxlalxdavaLeMdu avanu BAvisida. 
\num{2} ecacxrike ilalxda; huSAru tapipxda. 
\num{3} kApilalxda; rakaSxNAsAdhanavilalxda: \eng{guardless sword} kApilalxda katitx. 
\enum
\emng
\eentry

\bentry
\word{guard-rail}
\pron{gADfRreVlf}
\gl{\nA}
\bmng
 kApukaMbi; kaTAMjana; ODADuvavaru, noVTa noVDuvavaru bidudxhoVgadaMte aDaDxvAgi hAkiruva keYkaMbi. 
\emng
\eentry

\bentry
\word{guard-ring}
\pron{gADfRriMgf}
\gl{\nA}
\bmng
 (inonxMdu uMgura bidudxhoVgadaMte taDeyuva) taDeyuMgura; otutxMgura. 
\emng
\eentry

\bentry
\word{guardroom}
\pron{gADfRrUmf}
\gl{\nA}
\bmng
  = \hyperlink{guardhouse}{guardhouse}. 
\emng
\eentry

\bentry
\word{guard-ship}
\pron{gADfRSipf}
\gl{\nA}
\bmng
 kApuhaDagu; rakaSxNAnwke; baMdarina rakaSxNegiruva matutx tamamx tamamx haDagugaLanunx seVrikoLuLxvavaregU nAvikarige Asharxya koDuva yudadhxnwke. 
\emng
\eentry

\bentry
\word{guardsman}
\pron{gADfsxRmanf}
\gl{\nA}
\bmng
 (\kanmu\ iMgelxMDina aramaneya) kApudaLada, rakASxdaLada sipAyi, \kanmu\ hudedxVdAra. 
\emng
\eentry

\bentry
\word{guard-tent}
\pron{gADfRTeMTf}
\gl{\nA}
\bmng
 kApuDeVre: 
\banum
\alnum{a} seVneya kApudaLada DeVre. 
\alnum{b} baMdigaLanunx kUDihAkuva DeVre. 
\eanum
\emng
\eentry

\bentry
\word{Guarnerius}
\pron{gAvxniRariasf}
\gl{\nA}
\bmng
 gAvxniRVriyasf; iTaliya kerxmoVna eMba nagaradalilx \eng{17--18}neV shatamAnada gAvxneRVri eMbAta yA Atana vaMshajaru tayArisida piTiVlu. 
\emng
\eentry

\bentry
\word{guava}
\pron{gAvxva}
\gl{\nA}
\bmng
\bnum
\num{1} siVbehaNuNx; peVrala(haNuNx); ceVpehaNuNx; bikekxhaNuNx. 
\num{2} siVbemara. 
\enum
\emng
\eentry

\bentry
\word{guayule}
\pron{gevxYyUli}
\gl{\nA}
\bmng
 gavxyUli; rababxrinaMte upayoVgisabahudAda padAthaRvanunx niVDuva, mekisxkoV deVshada oMdu giDa. 
\emng
\eentry

\bentry
\word{gubbins}
\pron{gabinfsx}
\gl{\nA}
\bmng
\bnum
\num{1} kelasakekx bArada vasutx; niSapxrXyoVjaka padAthaR. 
\num{2} kasa; kacaDa. 
\num{3} salakaraNe; upakaraNa. 
\num{4} (\birx) (\AmA) daDaDx; egagx; gugugx. 
\enum
\emng
\eentry

\bentry
\word{gubernatorial}
\pron{gUyxbanaRToVrialf}
\gl{\gu}
\bmng
 (\ame) gavanaRrana; rAjayxpAla(ka)na. 
\emng
\eentry

\bentry
\word{guddle}
\pron{gaDflf}
\gl{\kirx}
\bmng
(sAkxTalxMDf) 
\emng

\noindent
\gl{\sakirx}
\bmng
(keYgaLiMda niVrina daDadalilx yA kalulxgaLa saMdiyalilx) taDakutAtx mInu hiDi. 
\emng

\noindent
\gl{\akirx}
\bmng
 hiVge mInigAgi taDakADu. 
\emng
\eentry

\bentry
\word{guddler}
\pron{gaDalxrf}
\gl{\nA}
\bmng
 mInu hiDiyalu niVrina daDadalilx yA kalulxgaLa saMdiyalilx keYyADisuvava, taDakuvava. 
\emng
\eentry

\bentry
\word[gudgeon(1)]{gudgeon}
\pron{gajanf}
\gl{\nA}
\bmng
\bnum
\num{1} gajanunx; gALakekx baLasuva oMdu bageya saNaNx sihiniVru mInu. 
\num{2} ati mugadhx; ati naMbikeyavanu; sulaBavAgi moVsa hoVguvavanu. 
\enum
\emng
\eentry

\bentry
\word[gudgeon(2)]{gudgeon}
\pron{gajanf}
\gl{\nA}
\bmng
\bnum
\num{1} (cAlaka daMDa, acucx, \mo vugaLa tudiyalilx cakarx \mo vanunx tagulisalu aLavaDisiruva) tirugANi; tirugaNi gUTa. 
\num{2} tirugaNi kaNuNx, baLe; bAgilanunx kaMbada tirugANige baMdhisuva bAgilina kaNuNx, baLe. 
\num{3} nAveya cukAkxNi tiruguva puTa, oraLu. 
\num{4} kUDumoLe; saMyoVjaka moLe; eraDu kalulx dimimxgaLu \mo vanunx seVrisuva moLe. 
\enum
\emng
\eentry

\bentry
\word{godgeon-pin}
\pron{gajanfpinf}
\gl{\nA}
\bmng
 cAlakadaMDada moLe; (\kanmu) cAlakadaMDavanUnx kUDu daMDavanUnx (\eng{connecting rod}) kUDisuva tirugANi moLe. 
\emng
\eentry

\bentry
\word{Guebre}
\pron{giV(geV)barf}
\gl{\nA}
\bmng
jaratUSapxrX matadavanu; jaratUSaTx matAvalaMbi; pAsiR. 
\emng
\eentry

\bentry
\wordRemoveSpace{guelder-rose}{guelder rose}
\pron{gelaDxrf roVsf}
\gl{\nA}
\bmng
 himagoMDe giDa; kene biLupina hUvugaLa goMDegaLanunx biDuva pode. 
\emng
\eentry

\bentry
\word{Guelf}
\pron{gevxlfphx}
\gl{\nA}
\bmng
  = \hyperlink{Guelph}{Guelph}. 
\emng
\eentry

\bentry
\word{Guelfic}
\pron{gevxlipxkf}
\gl{\gu}
\bmng
  = \hyperlink{Guelphic}{Guelphic}. 
\emng
\eentry

\bentry
\word{Guelfism}
\pron{gevxlipxsaZmf}
\gl{\nA}
\bmng
  = \hyperlink{Guelphism}{Guelphism}. 
\emng
\eentry

\bentry
\word{Guelph}
\pron{gevxlfphx}
\gl{\nA}
\bmng
 gevxlfphx; madhayxyugadalilx cakarxvatiRge virudadhxvAgi poVpfge beMbala koTaTx iTaliya rAjakiVya pakaSxda sadasayx. 
\emng
\eentry

\bentry
\word{Guelphic}
\pron{gevxliphxkf}
\gl{\gu}
\bmng
 gevxliphxVya; gevxlfphx pakaSxda. yA adakekx saMbaMdhisida. 
\emng
\eentry

\bentry
\word{Guelphism}
\pron{gevxliphxsaZmf}
\gl{\nA}
\bmng
\bnum
\num{1} gevxlfphx -- vAda, sidAdhxMta; cakarxvatiRgiMta poVpaneV sAvaRBwma eMdu vAdisida, madhayxyugada iTaliya gevxlfphx rAjikiVya sidAdhxMta, dhoVraNe. 
\num{2} gevxlfphx pakaSxkekx seVriruvudu; gevxlfphx pakiSxVyate. 
\enum
\emng
\eentry

\bentry
\word{guenon}
\pron{ganAknuf(nf)}
\gl{\nA}
\bmng
 ganAnf; udadxneya bAlavuLaLx Aphirxkada oMdu bageya koVti. 
\emng
\eentry

\bentry
\word[guerdon(1)]{guerdon}
\pron{gaDaRnf}
\gl{\nA}
\bmng
 (\kAparx) parxtiPala; pAritoVSika; bahumAna. 
\emng
\eentry

\bentry
\word[guerdon(2)]{guerdon}
\pron{gaDaRnf}
\gl{\sakirx}
\bmng
 (\kAparx) parxtiPala niVDu; pAritoVSika koDu; bahumAna niVDu. 
\emng
\eentry

\bentry
\word{guerdonless}
\pron{gaDaRnflisf}
\gl{\gu}
\bmng
 parxtiPalavilalxda; pAritoVSakarahita; bahumAnavilalxda. 
\emng
\eentry

\bentry
\wordf{gueridon}
\pron{geriDanf}
\gl{\nA}
\expl{\F\ }
\bmng
(\sA\ ketatxne mADiruvaMtha) saNaNx AlaMkArika meVju yA baDu. 
\emng
\eentry

\bentry
\word{Guernsey}
\pron{ganfRsiZ}
\gl{\nA}
\bmng
\bnum
\num{1} ganfRsiZ; (iMgilxSf kaDalAgxluveyalilxruva) cAnalf divxVpagaLalolxMdu. 
\num{2} \eng{(guernsey)} dapapx heNigeya, uNeNxya (\sA\ niVliya) jesiR, SaraTu, \mo vu. 
\num{3} (\AseTxrXV) phuTfbAlf SaraTu; kAlecxMDATadalilx hAkikoLuLxva SaTuR. 
\num{4} ganfRsiZ divxVpada sAkudanada taLi yA I taLiya dana. 
\enum
\emng
\eentry

\bentry
\wordnospeech{Guernsey lily}{Guernsey lily}
\pron{?}
\gl{\nA}
\bmng
 ganfRsiZ neYdile; \da\ AphirxkadiMda baMda oMdu bageya viSa muMgali hUvina giDada jAti. 
\emng
\eentry

\bentry
\word{guerilla}
\pron{garila}
\gl{\nA}
\bmng
  = \hyperlink{guerrilla}{guerrilla}. 
\emng
\eentry

\bentry
\word{guerrilla}
\pron{garila}
\gl{\nA}
\bmng
 gerilalx; cikakx cikakx taMDagaLalilx shaturxvina meVle haThAtatxne dALi mADuvudu, shaturxvina saMpakaR vayxvasethxgaLanunx nAshagoLisuvudeV modalAda kaqtayxgaLa mUlaka shaturxvige kATa koDuva savxtaMtarx yoVdhara taMDadavanu. 
\emng
\eentry

\bentry
\wordnospeech{guerrilla war}{guerrilla war}
\pron{?}
\gl{\nA}
\bmng
 gerilalxyudadhx; gerilalx riVtiya yA gerilalxgaLoDane yA gerilalxgaLiMda yudadhx mADuvudu. 
\emng
\eentry

\bentry
\wordnospeech{guerrilla warfare}{guerrilla warfare}
\pron{?}
\gl{\nA}
\bmng
  = \hyperlink{guerrilla war}{guerrilla war}. 
\emng
\eentry

\bentry
\word[guess(1)]{guess}
\pron{gesf}
\gl{\sakirx}
\bmng
\bnum
\num{1} (aLate yA vivaravAda lekAkxcAra mADade) aMdAju mADu; UhekaTuTx; Uhisu. 
\num{2} (\akirx\ saha) saMBAvayxveMdu BAvisu; AgabahudeMdu, saMBavaveMdu -- yoVcisu; AloVcisu. 
\num{3} (oMdara savxrUpa) gotitxdeyeMdu -- aMduko, BAvisu. 
\num{4} (oMdara bagegx) UhAkalapxne mADu; Uhe kalipxsu. 
\num{5} UhApoVhamADu. 
\num{6} (oMdu \vi aBipArxya sariyoV tapopxV) dheYyaRvAgi heVLibiDu (\akirx\ saha). 
\num{7} (toDakAda rahasayxkekx utatxravanunx, samaseyxge parihAravanunx) sariyAgi -- takiRsu, Uhisu, anumAnisu (\akirx\ saha). 
\enum
\emng

\noindent
\gl{\pagu}
\bmng
\bnum
\num{1} \eng{guess at} (yAvudeV \vi) Uhisu; UhemADu. 
\num{2} \eng{I guess} (\ame) (adu saMBavaveMdu) nanage aninxsutatxde. toVrutatxde. 
\num{3} \eng{keep person guessing} (\AmA) obabxnanunx saMshayadalilx, anishicxtateyalilx iTiTxru. 
\enum
\emng
\eentry

\bentry
\word[guess(2)]{guess}
\pron{gesf}
\gl{\nA}
\bmng
(sarisumArAda, hecucxkaDame sariyAda) aMdAju; Uhe; kalapxne. 
\emng

\noindent
\gl{\pagu}
\bmng
\hypertarget{guess pagu1}{} 
\bnum
\num{1} \hyperref{kandict_a.pdf}{A}{anybody nuga(2)}{anybody's guess.} 
\num{2} \eng{anyone's guess} = \hyperlink{guess pagu1}{?pagu? \((1)\)}. 
\num{3} \eng{by guess} UheyiMda; aMdAjiniMda; lekAkxcAravilalxde. 
\num{4} \eng{by guess and by God(frey)} bariya aMdAjiniMda; keVvala UhAtamxkavAgi; sariyAda lekAkxcAra ilalxde: \eng{some of these surveys were done completely by guess and by God(frey)} I moVjaNigaLalilx kelavanunx bariya aMdAjiniMda mADalAyitu. 
\num{5} \eng{have another guess coming} eNike tapApxgiru; Uhe tapApxgiru; tiLidukoMDadudx tapApxgiru. 
\num{6} \eng{miss one's guess} (\ame) tapupx eNike mADu; tapupx lekAkxcAra mADu. 
\num{7} \eng{my guess is} nanage idu hecucx kaDime Kacitavenisutatxde, KaMDitavenisutatxde. 
\enum
\emng
\eentry

\bentry
\word{guesstimate}
\pron{gesiTxmeVTf}
\gl{\nA}
\bmng
 (\AmA) UhA aMdAju; Uhe matutx takaR eraDanUnx avalaMbisi mADida aMdAju. 
\emng
\eentry

\bentry
\word{guess-rope}
\pron{gesfroVpf}
\gl{\nA}
\bmng
  = \hyperlink{guest-rope}{guest-rope}. 
\emng
\eentry

\bentry
\word{guesswork}
\pron{gesfvakfR}
\gl{\nA}
\bmng
\bnum
\num{1} Uhe; UhAkAyaR; UheyeV AdhAravAda kAyaRvidhAna. 
\num{2} Uhisuvudu; Uhana. 
\num{3} UheyiMda baMdadudx. 
\enum
\emng
\eentry

\bentry
\word{guest}
\pron{gesfTx}
\gl{\nA}
\bmng
\bnum
\num{1} (obabxra maneyalilx satAkxra paDeyuva yA UTakekx baMdiruva) atithi; aBAyxgata. 
\num{2} hoVTelinalilx, KAnAvaLiyalilx vAsisutitxruvavanu yA taMgiruvavanu. 
\num{3} paroVpajiVvi(yAda pArxNi yA sasayx). 
\num{4} atithi naTa, naTi, \mo varu; oMdu nidiRSaTx kaMpani \mo vugaLige seVrade, AhAvxnitarAda naTa, naTi \mo varu. 
\enum
\emng

\noindent
\gl{\pagu}
\bmng
\bnum
\num{1} \eng{guest of honour} gwravAnivxta atithi; visheVSa gwravagaLanunx paDeda atithi. 
\num{2} \eng{paying guest} haNa koTuTx UTa mADuvava(Lu). 
\enum
\emng
\eentry

\bentry
\word{guest-chamber}
\pron{gesfTxceVMbarf}
\gl{\nA}
\bmng
 atithikoVNe; atithi aBAyxgatarigAgi miVsaliTaTx koThaDi, rUmu. 
\emng
\eentry

\bentry
\word{guest-house}
\pron{gesfTxhwsf}
\gl{\nA}
\bmng
 (meVladxjeRya) atithi gaqha; biDadi; atithigaLigAgi kaTiTxruva mane. 
\emng
\eentry

\bentry
\word{guestimate}
\pron{gesiTxmaTf}
\gl{\nA}
\bmng
  = \hyperlink{guesstimate}{guesstimate}. 
\emng
\eentry

\bentry
\word{guest-night}
\pron{gesfTxneYTf}
\gl{\nA}
\bmng
 atithisatAkxrada rAtirx; kalxbubx, kAleVju KAnAvaLi, \mo vugaLalilx atithigaLanunx autaNa, vinoVda, \mo vugaLiMda satakxrisuva rAtirx. 
\emng
\eentry

\bentry
\word{guest-room}
\pron{gesfTxrUmf}
\gl{\nA}
\bmng
  = \hyperlink{guest-chamber}{guest-chamber}. 
\emng
\eentry

\bentry
\word{guest-rope}
\pron{gesfTxroVpf}
\gl{\nA}
\bmng
\bnum
\num{1} (eLedoyuyxtitxruva doVNiyanunx sitxmitagoLisalu adakekx kaTuTxva) eraDaneya hagagx. 
\num{2} iLi hagagx; pakakxdalilx barutitxruva doVNigaLu hiDidukoLuLxvudakAkxgi haDagina horage iLiyabiTiTxruva yA udadxkUkx kaTiTxruva hagagx. 
\enum
\emng
\eentry

\bentry
\word{guestship}
\pron{gesfTxSipf}
\gl{\nA}
\bmng
 atithitana; aBAyxgatatana. 
\emng
\eentry

\bentry
\word{guff}
\pron{gaphf}
\gl{\nA}
\bmng
(\ashi) buruDe; bogaLe; huruLilalxda mAtu. 
\emng
\eentry

\bentry
\word[guffaw(1)]{guffaw}
\pron{gaphA}
\gl{\nA}
\bmng
 birunagu; keVkenagu; gahagahisi naguva nagu; oraToraTAda ababxrada, atigaTiTxyAda-nagu. 
\emng
\eentry

\bentry
\word[guffaw(2)]{guffaw}
\pron{gaphA}
\gl{\akirx}
\bmng
gahagahisi nagu; ababxrisi nagu; keVke hAki nagu. 
\emng

\noindent
\gl{\sakirx}
\bmng
keVke hAki, ababxrisi, gahagahisi nagutatx heVLu, nuDi. 
\emng
\eentry

\bentry
\word[guggle(1)]{guggle}
\pron{gagflf}
\gl{\kirx}
\bmng
 \eng{gurgle} padada rUpAMtara. 
\emng
\eentry

\bentry
\word[guggle(2)]{guggle}
\pron{gagflf}
\gl{\nA}
\bmng
 \eng{gurgle} padada rUpAMtara. 
\emng
\eentry

\bentry
\word{guichet}
\pron{giVSeV}
\gl{\nA}
\bmng
\bnum
\num{1} paTiTxcwkaTuTx; jAlari cwkaTuTx; ududxdadxvAgi yA aDaDxDaDxvAgi hAyuvaMte paTiTxgaLanunx joVDisiruva cwkaTuTx. 
\num{2} TikeTf (koDuva kaceVriya) kiTaki. 
\enum
\emng
\eentry

\bentry
\word{guidable}
\pron{geYDabflf}
\gl{\gu}
\bmng
\bnum
\num{1} mAgaRdashaRniVya; dAritoVrisalu sAdhayxvAda; mAgaRdashaRna mADabahudAda. 
\num{2} nideRVshana, niyaMtarxNa mADabahudAda; nideRVshana sAdhayx yA niyaMtarxNa sAdhayx. 
\enum
\emng
\eentry

\bentry
\word{guidance}
\pron{geYDanfsx}
\gl{\nA}
\bmng
\bnum
\num{1} dAri toVrisuvudu; mAgaRdashaRna (mADuvudu); muMde naDedu naDesuvudu; muMdoyuyxvudu; mAgaR nideRVshisuvudu. 
\num{2} niyaMtarxNa yA nideRVshana. 
\num{3} digadxshaRna; digadxshaRna mADuvudu. 
\num{4} kArubAru naDesuvudu; kAyaRBAra nivaRhaNe. 
\enum
\emng
\eentry

\bentry
\word[guide(1)]{guide}
\pron{geYDf}
\gl{\nA}
\bmng
\bnum
\num{1} mAgaRdashiR; mAgaRdashaRka; dAritoVruga; dAri toVrisuvavanu. 
\num{2} parxvAsi mAgaRdashaRka; parxyANikanige yA parxvAsige haNa paDedu dAri, perxVkaSxNiVya sathxLa, \mo vanunx toVrisuvavanu. 
\num{3} (\kanmu\ sivxTasxlaRMDf \mo\ deVshagaLalilx) vaqtitxpara pavaRtAroVhi; pavaRtAroVhaNa vaqtitxyavanu. 
\num{4} (\bava\ dalilx) mAgaRdashaRka daLa; mAgaRdashaRna mADalu yA pUvaRBAvi parishiVlana naDesalu, kelavu seYnayxgaLalilx racisiruva visheVSa daLa, taMDa. 
\numi{5} geYDu: 
\banum
\alnum{a} (seYnayx) tananx calanavalanagaLiMda daLada itara seYnikaru tamamx calanavalanagaLanunx hoMdisikoLuLxvaMte, niyaMtirxsikoLuLxvaMte mAgaRdashaRna niVDuva seYnika; mAgaRdashaRka seYnika. 
\alnum{b} mAgaRdashaRka vAhana; itara vAhanagaLu tamamx calana valanagaLanunx sari hoMdisikoLuLxvaMte mAgaRdashaRna niVDuva vAhana. 
\alnum{c} mAgaRdashaRka nwke; itara nwkegaLu tamamx calanavalanagaLanunx sarihoMdisikoLuLxvaMte mAgaRdashaRna niVDuva haDagu. 
\eanum
\numie
\num{6} salahegAra; salahAkAra; vayxkitxge hitanuDi heVLuvavanu. 
\num{7} AdashaR; parxmANa; mAgaRdashaRka tatatxvX: \eng{scripture is our guide} shAsatxrXveV namage parxmANa. \eng{the feelings are a bad guide} BAvanegaLu keTaTx mAgaRdashaRka (tatatxvX)gaLu. 
\num{8} (\eng{Guide}) (\birx) galfR geYDu; bAlikA camU; camU bAlike. 
\num{9} geYDu; keYpiDi; mAgaRdashiR; mAgaRdashaRna (pusatxka); oMdu viSayada vAyxsaMgakekx sahAyakavAguva, adara mUla tatatxvXgaLa paricaya mADikoDuva pusatxka. 
\hypertarget{guide(1)10}{} 
\num{10} (oMdu nagara, adara cacuR, vasutx saMgarxhAlaya, \mo vugaLa bagege tiLivaLike koDuva) geYDu; mAgaRdashaRka (pusatxka). 
\num{11} (\yaMshA) cAlakadaMDa; yAvudeV yaMtarxda yA yaMtarxBAgada calanavanunx niyaMtirxsuva kaMbi, daMDa, \mo vu. 
\num{12} (\yaMshA) niyaMtarxka; salakaraNegaLa kAyaRvanunx niyaMtirxsuva sAdhana. 
\num{13} nideRVshaka; oMdu sAthxnavanunx gurutisuva yA kaNiNxge nideRVshakavAgiruva vasutx. 
\enum
\emng

\noindent
\gl{\pagu}
\bmng
\hyperdef{G}{guide(1) pagu}{} \eng{King's or Queen's Guide} rAjana yA rANiya mAgaRdashiR; rAjana yA rANiya AsAthxnadalilx atayxMta hecicxna pariNati, dakaSxte uLaLx adhikArige koDuva mAgaRdashiR padavi. 
\emng
\eentry

\bentry
\word[guide(2)]{guide}
\pron{geYDf}
\gl{\sakirx}
\bmng
\bnum
\num{1} dAritoVrisu; mAgaRdashiRyAgu; mAgaR nideRVshisu; pathatoVru; mAgaRdashaRna mADu. 
\num{2} (dAri toVrisalu) muMde hoVgu; muMdALAgu. 
\num{3} (vAhana, vasutx, pArxNi, vayxkitx, GaTanegaLu, \mo vugaLa gatiyanunx, mAgaRvanunx) rUpisu; niyaMtirxsu; nideRVshisu. 
\num{4} (vayxkitxgaLu, kirxyegaLu, aBipArxyagaLu, \mo vugaLige) digadxshaRnamADu. 
\num{5} (sUcanegaLu, tatatxvXgaLu, aMtaHperxVraNegaLu, \mo vugaLa \vi) mAgaR nideRVshisu; sariyAda disheyalilx vatiRsuvaMte mADu: \eng{Lord guide my judgement!} deVvaru nananx niNaRya shakitxyanunx nideRVshisali! 
\num{6} (rASaTxrX \mo vugaLa) kAyaR nivaRhisu; kArubAru naDesu; vayxvahAragaLanunx nivaRhisu. 
\enum
\emng
\eentry

\bentry
\word{guidebook}
\pron{geYDfbukf}
\gl{\nA}
\bmng
  = \hyperlink{guide(1)10}{$^1$guide (10)}. 
\emng
\eentry

\bentry
\wordnospeech{guided missile}{guided missile}
\pron{?}
\gl{\nA}
\bmng
 nideRVshita kiSxpaNi; kiSxpaNiyu guriyatatx dhAvisutitxruvAga reVDiyoV dUraniyaMtarxNadiMdaloV savxyaMcAlita gurisAdhaka vayxvasethxyiMdaloV adara pathavanunx badalAyisuva vayxvasethx uLaLx kiSxpaNi. 
\emng
\eentry

\bentry
\word{guide-dog}
\pron{geYDfDAgf}
\gl{\nA}
\bmng
 mAgaRdashiR nAyi; kuruDanige dAritoVrisuvaMte tarapeVti koTaTx nAyi. 
\emng
\eentry

\bentry
\wordnospeech{guided tour}{guided tour}
\pron{?}
\gl{\nA}
\bmng
 geYDuparxvAsa; mAgaRdashiR parxvAsa; mAgaRdashiR sahita parxvAsa; parxvAsigarige perxVkaSxNiVya sathxLagaLanunx toVrisalu mAgaRdashiR iruva parxvAsa. 
\emng
\eentry

\bentry
\word{guideless}
\pron{geYDflisf}
\gl{\gu}
\bmng
\bnum
\num{1} mAgaRdashaRkanilalxda. 
\num{2} niyaMtarxrakahita. 
\enum
\emng
\eentry

\bentry
\word{guide-line}
\pron{geYDfleYnf}
\gl{\nA}
\bmng
 (\rUpa) mAgaRdashaRna sUtarx; nideRVshaka tatatxvX; muMdina karxmada bagege mAgaRdashaRna yA sUcane. 
\emng
\eentry

\bentry
\word{guide-post}
\pron{geYDfpoVsfTx}
\gl{\nA}
\bmng
 = \hyperref{kandict_s.pdf}{S}{signpost(1)}{signpost}. 
\emng
\eentry

\bentry
\word{guider}
\pron{geYDarf}
\gl{\nA}
\bmng
\bnum
\num{1} mAgaRdashiR; itararige mAgaRdashaRna niVDuvava. 
\num{2} \eng{(Guider)} bAlikAcamUvina vayasakx nAyaki; galfR geYDf nAyaki. 
\enum
\emng
\eentry

\bentry
\word{guide-rope}
\pron{geYDfroVpf}
\gl{\nA}
\bmng
 niyaMtarxka hagagx; nideRVshaka hagagx: 
\banum
\alnum{a} bakayaMtarxda (yA kerxVnina) horage, adanunx beVkAdaMte tirugisalikAkxgi kaTiTxruva saNaNx hagagx. 
\alnum{b} oMdu balUnu yA AkAshanwkeyu hArutitxruva etatxravanunx sithxravAgiDuvudakAkxgi nelada udadxkUkx eLedukoMDu hoVguvaMte adakekx kaTiTxda hagagx. 
\alnum{c} AkAshanwkeyu hAruva muMce adanunx sitxmitagoLisuva hagagxgaLalolxMdu. 
\eanum
\emng
\eentry

\bentry
\word{guiding-stick}
\pron{geYDiMgfsiTxkf}
\gl{\nA}
\bmng
 (citarxkAra tananx balageYge AsareyAgi eDageYyalilx hiDidukoLuLxva) AsarekaDiDx; AsaregoVlu; AdhArapaTiTx. 
\emng
\eentry

\bentry
\word{guidon}
\pron{geYDanf}
\gl{\nA}
\bmng
 (rAvutara patAkeyAgi baLasuva, tirxkoVnAkAradalilxdudx tudiyalilx moneyAgiruva) kirubAvuTa. 
\emng
\eentry

\bentry
\word{Guignol}
\pron{giVnAyxlf}
\gl{\nA}
\bmng
\bnum
\num{1}  = \hyperlink{Grand Guignol}{Grand Guignol}. 
\num{2} paMcf matutx jUDi boMbeyATa. 
\enum
\emng
\eentry

\bentry
\word{Guignolesque}
\pron{giVnAyxlesfkx}
\gl{\gu}
\bmng
\bnum
\num{1} biVBatasx nATakadaMtha; BiVkara daqshAyxvaLiyaMtha. 
\num{2} paMcf matutx jUDi boMbeyATadaMtha. 
\enum
\emng
\eentry

\bentry
\word{guild}
\pron{gilfDx}
\gl{\nA}
\bmng
\bnum
\num{1} parasapxra sahAya saMGa; samAnoVdedxVsha saMGa; parasapxra sahAyakAkxgiyoV elalxrigU samAnavAda udedxVsha sAdhisalikAkxgiyoV EpaRDisikoMDa saMGa. 
\num{2} (madhayxyugada) vaqtitx -- saMGa, sherxVNi; kushalakamiRgaLa yA vAyxpArigaLa saMGa. 
\enum
\emng
\eentry

\bentry
\word{guilder}
\pron{gilaDxrf}
\gl{\nA}
\bmng
 gilaDxrf: 
\banum
\alnum{a} (\ca) nedarflaMDfsx matutx jamaRni deVshagaLalilx calAvaNeyalilxdadx oMdu cinanxda nANayx. 
\alnum{b} nedarflaMDisxna oMdu nANayx. 
\eanum
\emng
\eentry

\bentry
\word{guildhall}
\pron{gilfDxhAlf}
\gl{\nA}
\bmng
\bnum
\num{1} vaqtitxsaMGa Bavana; madhayxyugada vaqtitxsaMGa yA vAyxpAri saMGa seVrutitxdadx sathxLa. 
\num{2} (\birx) (\eng{the Guildhall} saha) (sakARri rAtirxBoVjanagaLu, purasaBA kAyaRkalApagaLu, \mo vugaLAgi baLasuva) laMDaninxna purasaBA Bavana. 
\num{3} puraBavana; nagarapAlikA Bavana; nagara saBeya kAyaRkalApagaLanunx naDesalu seVruva sathxLa. 
\enum
\emng
\eentry

\bentry
\word{guile}
\pron{geYlf}
\gl{\nA}
\bmng
\bnum
\num{1} dorxVha. 
\num{2} moVsa; vaMcane; kaqtirxma; kapaTa; dagAKoVratana; dagalAbxji. 
\num{3} kutaMtarx; haMcike; hUTa; pitUri; oLasaMcu. 
\enum
\emng
\eentry

\bentry
\word{guileful}
\pron{geYlfphulf}
\gl{\gu}
\bmng
\bnum
\num{1} dorxVhada; dorxVhadiMda kUDida. 
\num{2} moVsada; vaMcaka; kapaTa; dagAKoVratanada; dagalAbxjiya. 
\num{3} kutaMtarxda; haMcike, hUTa, pitUri, oLasaMcu, \mo vugaLiMda kUDida. 
\enum
\emng
\eentry

\bentry
\word{guilefully}
\pron{geYlfphuli}
\gl{\kirxvi}
\bmng
\bnum
\num{1} dorxVhabudidhxyiMda; dorxVhadiMda kUDi. 
\num{2} moVsadiMda; vaMcakatanadiMda; kapaTatanadiMda. 
\num{3} kutaMtarxdiMda; haMcike, hUTa, pitUri, oLasaMcu, \mo vugaLiMda. 
\enum
\emng
\eentry

\bentry
\word{guileless}
\pron{geYlflisf}
\gl{\gu}
\bmng
\bnum
\num{1} dorxVhabudidhxyilalxda; dorxVhavilalxda. 
\num{2} moVsagArikeyilalxda; vaMcakatanavilalxda; niSakxpaTa. 
\num{3} kutaMtarxvilalxda; haMcike, hUTa, pitUri, oLasaMcu -- ilalxda. 
\enum
\emng
\eentry

\bentry
\word{guillemot}
\pron{gilimaTf}
\gl{\nA}
\bmng
 yuriya yA sephasf kulada kaDalabAtu. 
\emng
\eentry

\bentry
\word{guilloche}
\pron{giloVSf}
\gl{\nA}
\bmng
 (\vAshi) heNige alaMkAra; jaDe yA paTiTx heNedaMte racisida vAsutxshilapxda yA loVhada alaMkAra vidhAna.  \imglink{guillochefigure}{\raisebox{-0.15cm}[0pt][0pt]{\pdfimage width 0.9cm height 0.5cm {G_Pictures/guilloche.jpg}}} 
\emng
\eentry

\bentry
\word[guillotine(1)]{guillotine}
\pron{gilaTiVnf}
\gl{\nA}
\bmng
 giloTiVnf: 
\banum
\alnum{a} talegaDuka (yaMtarx); shiraceCxVdaka(yaMtarx); BAravAda katitxyalagu gADigaLalilx jArikoMDu iLiyuvaMte aLavaDisiruva talekaDiyuva yaMtarx. \imglink{guillotine-1figure}{\raisebox{-0.15cm}[0pt][0pt]{\pdfimage width 0.5cm height 0.8cm {G_Pictures/guillotine-1.jpg}}} 
\alnum{b} kirunAlage \mo vanunx katatxrisalu baLasuva shasatxrXcikitesxya upakaraNa. 
\alnum{c} kAgada \mo vanunx katatxrisalu baLasuva (vividha) yaMtarxsAdhana. 
\alnum{d} (\birx) (pAliRmeMTf) aDiDxya kaDita; aDiDx nivAraNe; parxtiroVdha nivAraNe; masUdeya BAgagaLu \mo vugaLa caceRyalilx, sadasayxru beVkeMdeV oDuDxva aDiDxyiMda kAlaviLaMbavAguvudanunx tapipxsalu voVTumADuva kAlagaLanunx modaleV gotutxpaDisuva vidhAna. 
\eanum
\emng
\eentry

\bentry
\word[guillotine(2)]{guillotine}
\pron{gilaTiVnf}
\gl{\sakirx}
\bmng
 giloTiVnfge hAku: 
\banum
\alnum{a} giloTiVniniMda tale katatxrisu. 
\alnum{b} (\birx) (pAliRmeMTf) giloTiVnf vidhAnavanunx parxyoVgisi aDiDxyanunx tapipxsu. 
\eanum
\emng
\eentry

\bentry
\word{guilt}
\pron{gilfTx}
\gl{\nA}
\bmng
\bnum
\num{1} (nideRVshisida yA sUcisida) tapipxta; aparAdha; tapipxtavanunx mADiruvudu. 
\num{2} tapipxtasathxte; aparAdhitavx; doVSitavx. 
\num{3} tapipxtasathx manoVBAva; aparAdhi manoVBAva. 
\num{4} shikASxhaRte; daMDayxte; daMDaniVyate; daMDAhaRte; shikiSxsalu ahaRvAgiruvudu. 
\enum
\emng
\eentry

\bentry
\wordnospeech{guilt complex}{guilt complex}
\pron{?}
\gl{\nA}
\bmng
 (\mashA) aparAdha parxjecnx; doVSaBAvane; doVSAshaMke; tapipxtaBAva; yAvudoV tapipxtavanunx mADida BAvane, saMKe. 
\emng
\eentry

\bentry
\word{guiltily}
\pron{giliTxli}
\gl{\kirxvi}
\bmng
 tapipxtasathxnaMte; aparAdhiyaMte. 
\emng
\eentry

\bentry
\word{guiltiness}
\pron{giliTxnisf}
\gl{\nA}
\bmng
\bnum
\num{1} tapipxtasathxtana; aparAdhitana; doVSitavx. 
\num{2} shikASxhaRte; daMDayxte; daMDAhaRtavx; shikeSx anuBavisalu yoVgayxnAgiruvudu. 
\num{3} aparAdhi (manoV)BAva; doVSAshaMke. 
\enum
\emng
\eentry

\bentry
\word{guiltless}
\pron{gilfTxlisf}
\gl{\gu}
\bmng
\bnum
\num{1} tapipxtavilalxda; tapipxtasathxnalalxda; niraparAdhi; aparAdha mADirada; nidoRVSi; doVSiyalalxda; aparAdhiyalalxda; niraparAdhiyAda. 
\num{2} (yAvudeV viSayada) arivilalxda; tiLivaLike, jAcnxna, paricaya, gaMdha -- ilalxda: \eng{guiltless of grammar} vAyxkaraNada gaMdhaveV ilalxda. 
\num{3} (yAvudeV vasutxvanunx) hoMdilalxda; paDedilalxda: \eng{guiltless of moustache} miVseyilalxda. 
\enum
\emng
\eentry

\bentry
\word{guiltlessly}
\pron{gilfTxlisfli}
\gl{\kirxvi}
\bmng
tapipxtavilalxdaMte; aparAdhavilalxdaMte; doVSarahitanAgi; tapipxtasathxnalalxde; nidoRVSiyAgi; niraparAdhiyAgi. 
\emng
\eentry

\bentry
\word{guiltlessness}
\pron{gilfTxlisfnisf}
\gl{\nA}
\bmng
 tapipxtavilalxdiruvike; aparAdhavilalxdiruvike; doVSa rahitate; tapipxtasathxnalalxdiruvike; doVSiyalalxdiruvike; nidoRVSate; niraparAdhitavx. 
\emng
\eentry

\bentry
\word{guilty}
\pron{giliTx}
\gl{\gu}
\bmng
\bnum
\num{1} tapipxta mADiruva; tapipxtasathxnAda; aparAdha mADiruva; aparAdhiyAda; takisxVru mADiruva; takisXVrudAranAda; doVSavesagiruva; doVSiyAda. 
\num{2} shikASxhaRnAda; shikeSxge yoVgayxnAda; daMDAhaR; daMDaniVya; daMDayx; daMDanege ahaRnAda. 
\num{3} aparAdha parxjecnxyuLaLx; tananx tapipxtada ariviruva; tananx tapipxtavanunx balalx; tAnu tapipxtasathxneMdu arita; tapipxta manasisxna: \eng{guilty conscience} aparAdha parxjecnx; kaLaLxmanasusx. 
\hypertarget{guilty(4)}{} 
\num{4} (yAvudeV gotAtxda, nidiRSaTx) aparAdha mADiruva. 
\num{5} aparAdha perxVrita; aparAdha mADuva savxBAvadiMda, BAvaneyiMda -- perxVritavAda, parxcoVditavAda. 
\enum
\emng

\noindent
\gl{\pagu}
\bmng
\bnum
\num{1} \eng{guilty, not guilty} (koVTuR vicAraNegaLalilx koDuva tiVpuRgaLalilx) tapipxtasathx, nidoRVSi. 
\num{2} \eng{guilty of} = \hyperlink{guilty(4)}{guilty (4)}. 
\enum
\emng
\eentry

\bentry
\word{guimp}
\pron{giMpf}
\gl{\nA}
\bmng
  = \hyperlink{gimp(1)}{gimp} padada rUpAMtara. 
\emng
\eentry

\bentry
\word{guinea}
\pron{gini}
\gl{\nA}
\bmng
\bnum
\num{1} (\eng{Guinea}) gini; Aphirxkada pashicxma tiVrada oMdu BAga. 
\num{2} (\ca) gini; hiMdina oMdu birxTiSf cinanxda nANayx. 
\num{3} (\birx) gini; Iga vividha vaqtitxgaLavarege parxtiPala koDuvudaralilx, caMdA koDuvudaralilx, citarxgaLu, kuduregaLu, jamInugaLu, \mo vugaLa bele niNaRyisuvudaralilx lekAkxcArakAkxgi mAtarx baLasuva, \eng{1.05} pwMDina beleya, nAmfkeVvAsetx nANayx. 
\enum
\emng
\eentry

\bentry
\word{guinea-corn}
\pron{ginikAnfR}
\gl{\nA}
\bmng
 = \hyperref{kandict_d.pdf}{D}{durra}{durra}. 
\emng
\eentry

\bentry
\word{guinea-fowl}
\pron{giniphwlf}
\gl{\nA}
\bmng
 gini koVLi; yUroVpinalilx sAkuva, navilu baLagada, namiDa (\kanmu\ namiDa melagirxsf) kulakekx seVrida, selxVTu baNaNxda, biLi cukikxya garigaLuLaLx pakiSx. \imglink{guinea-fowlfigure}{\raisebox{-0.15cm}[0pt][0pt]{\pdfimage width 0.8cm height 0.6cm {G_Pictures/guinea-fowl.jpg}}} 
\emng
\eentry

\bentry
\wordnospeech{Guinea grains}{Guinea grains}
\pron{?}
\gl{\nA}
\bmng
  = \hyperlink{grains of Paradise}{grains of Paradise}. 
\emng
\eentry

\bentry
\word{guinea-hen}
\pron{ginihenf}
\gl{\nA}
\bmng
  = \hyperlink{guinea-fowl}{guinea-fowl}. 
\emng
\eentry

\bentry
\word{guinea-pig}
\pron{ginipigf}
\gl{\nA}
\bmng
\bnum
\num{1} giniyili; yUroVpu \mo\ kaDegaLalilx Iga mudidxna pArxNiyAgi sAkuva yA veYdayxkiVya parxyoVgapArxNiyAgi baLasuva, kAyxviya kulakekx seVrida, \da\ amerikada oMdu bageya daMshaka. \imglink{guinea-pigfigure}{\raisebox{-0.15cm}[0pt][0pt]{\pdfimage width 0.7cm height 0.5cm {G_Pictures/guinea-pig.jpg}}} 
\num{2} (\birx) (\pArxparx) ginigArxhaka; oMdu gini rusumu tegedukoLuLxvavanu (\kanmu\ oMdu kaMpaniya DeYrekaTxrf yA upapAdirx). 
\num{3} parxyoVgapArxNi; veYdayxkiVya parxyoVgagaLalilx parxyoVgakekx guriyAguva vayxkitx yA pArxNi. 
\enum
\emng
\eentry

%%%%%%%%%%%%
\bentry
\wordnospeech{Guinea worm}{Guinea worm}
\pron{?}
\gl{\nA}
\bmng
\bnum
\num{1} ginihuLu; nAru huNiNxna huLu; uSaNxvalayadalilx manuSayxna camaRdalilx seVrikoLuLxva, oMdu paroVpajiVvi. 
\num{2} nAruhuNuNx; ginihuLuviniMda baruva, oMdu bageya camaRroVga. 
\enum
\emng
\eentry

\bentry
\word{guipure}
\pron{giVpuarf}
\gl{\nA}
\bmng
 dapapx -- kare, aMcu, paTiTx; (reVSemx, uNeNx yA linanf baTeTxya tuMDugaLanunx dapapx kasUtiyiMda heNedu mADida), alaMkArada kare, aMcupaTiTx. 
\emng
\eentry

\bentry
\word{guise}
\pron{geYsfZ}
\gl{\nA}
\bmng
\bnum
\num{1} (\pArxparx) uDupu; uDupina sheYli. 
\num{2} horatoVkeR; hora rUpa; bAhayx rUpa. 
\num{3} veVSa; soVgu; kaqtaka rUpa; naTane: \eng{under} (\engit{or} \eng{in) the guise of} veVSadalilx; soVginalilx. 
\enum
\emng
\eentry

\bentry
\word{guiser}
\pron{geYsaZrf}
\gl{\nA}
\bmng
 (sAkxTalxMDf matutx \kanu iMgalxMDf) mUkanaTa; mUkABinayakAra; \kanmu\ hAyxloVyiVnf hababxdalilx (akoTxVbarf \eng{31}) mUkanATakagaLalilx naTisuvava. 
\emng
\eentry

\bentry
\word[guitar(1)]{guitar}
\pron{giTArf}
\gl{\nA}
\bmng
 giTAru; Aru taMtigaLuLaLx; meTaTxlugaLuLaLx, beraLiniMda yA mITudaMta \mo vugaLiMda nuDisuva oMdu taMtiVvAdayx. \imglink{guitarfigure}{\raisebox{-0.15cm}[0pt][0pt]{\pdfimage width 0.6cm height 0.7cm {G_Pictures/guitar.jpg}}} 
\emng

\noindent
\gl{\pagu}
\bmng
 \eng{electric guitar} viduyxdigxTAru; oLagaDe mekorxVphoVnf iruva giTAru. 
\emng
\eentry

\bentry
\word[guitar(2)]{guitar}
\pron{giTArf}
\gl{\akirx}
\expl{(\BU\ matutx \BUkaq\ \eng{guitarred,} \vakaq\ \eng{guitarring}).}
\bmng
giTArf -- nuDisu, vAdana mADu. 
\emng
\eentry

\bentry
\word{guitarist}
\pron{giTArisfTx}
\gl{\nA}
\bmng
 giTArf vAdaka; giTArf nuDisuvava. 
\emng
\eentry

\bentry
\word[Gujarati(1)]{Gujarati}
\pron{gUjarATi}
\gl{\nA}
\bmng
 gujarAti: 
\banum
\alnum{a} (BAratada) gujarAtinalilx huTiTxdava yA vAsisuva. 
\alnum{b} gujarAtina BASe. 
\eanum
\emng
\eentry

\bentry
\word[Gujarati(2)]{Gujarati}
\pron{gUjarATi}
\gl{\gu}
\bmng
 gujarAti: 
\banum
\alnum{a} (BAratada) gujarAtina. 
\alnum{b} gujarAti BASeya. 
\eanum
\emng
\eentry

\bentry
\word{gulch}
\pron{galfcx}
\gl{\nA}
\bmng
 (\ame) (\kanmu\ niVru hariyutitxruva) kamari; kaMdara. 
\emng
\eentry

\bentry
\word{gulden}
\pron{gu(gU)laDxnf}
\gl{\nA}
\bmng
  = \hyperlink{guilder}{guilder}. 
\emng
\eentry

\bentry
\word[gules(1)]{gules}
\pron{gUyxlfs'}
\gl{\nA}
\bmng
 (\kanmu\ \vaMlAM) (\sA\ nAmapadada muMde \parx) keMpu (baNaNx). 
\emng
\eentry

\bentry
\word[gules(2)]{gules}
\pron{gUyxlfs'}
\gl{\gu}
\bmng
 (\sA\ \vaMlAM) keMpu; keMpAda; keMpu baNaNxda. 
\emng
\eentry

\bentry
\word[gulf(1)]{gulf}
\pron{galfphx}
\gl{\nA}
\bmng
 (\BUgoV) (vishAla haravu matutx kiridAda muKaBAga iruva) kolilx; KAri. 
\bnum
\num{2} Dogaru; korakalu; kamari; kaMdara; ALavAda haLaLx. 
\num{3} (\kAparx) pAtALa; samudarxda -- taLAtaLa, agAdha ALa. 
\num{4} (\pArxparx) suLi; jalAvataR; yAvudanenxV nuMgi biDuvaMthadu. 
\num{5} (dATalAgada) gaDi; elelx; aMtara. 
\num{6} (\birx) (\vivi da \ashi) riyAyati padavi; AnasfR padavi pariVkeSxyalilx napAsAda, AdarU pAsAgalu ahaRneMba aBayxthiRge koDuva padavi. 
\num{7} (BAvagaLu, aBipArxyagaLu, \mo vugaLa) agAdha aMtara; bahaLavAda vayxtAyxsa. 
\enum
\emng
\eentry

\bentry
\word[gulf(2)]{gulf}
\pron{galfphx}
\gl{\sakirx}
\bmng
\bnum
\num{1} muLugisibiDu; nuMgi hAkibiDu. 
\num{2} (\birx) (\vivi da \ashi) riyAyiti padavi koDu. 
\enum
\emng
\eentry

\bentry
\wordnospeech{Gulf Stream}{Gulf Stream}
\pron{?}
\gl{\nA}
\bmng
 galfphx siTxrXVmf; uSaNx parxvAha; mekisxko kolilxyiMda aTAlxMTikf sAgaradalilx yUroVpina kaDege haridu baruva uSoNxVdaka parxvAha. 
\emng
\eentry

\bentry
\word{gulf-weed}
\pron{galfphxviVDf}
\gl{\nA}
\bmng
 = \hyperref{kandict_s.pdf}{S}{sargasso}{sargasso}. 
\emng
\eentry

\bentry
\word[gull(1)]{gull}
\pron{galf}
\gl{\nA}
\bmng
(hakikx); udadx rekekxya, jAlapAdada, visheVSavAgi kaDalaMcinalilx vAsisuva, lAyxriDeV vaMshada, \sA\ biLi baNaNxda, mutitxna baNaNxdiMda hiDidu kapupx baNaNxdavaregU vividha baNaNxda meYhodikeyuLaLx, hoLapu kokikxna kaDala hakikxya jAti. 
\emng
\eentry

\bentry
\word[gull(2)]{gull}
\pron{galf}
\gl{\nA}
\bmng
 (\pArxparx\ yA \pArxM) (sulaBavAgi naMbi moVsa hoVguva) maMka; gugugx; bepupxtakakxDi. 
\emng
\eentry

\bentry
\word[gull(3)]{gull}
\pron{galf}
\gl{\sakirx}
\bmng
 naMbisi -- moVsagoLisu, vaMcisu. 
\emng
\eentry

\bentry
\word{gullery}
\pron{galari}
\gl{\nA}
\bmng
 moVsa; vaMcane; kutaMtarx. 
\emng
\eentry

\bentry
\word{gullet}
\pron{galiTf}
\gl{\nA}
\bmng
\bnum
\num{1} ananxnALa; AhAranALa. 
\num{2} gaMTalu; kaMTha. 
\enum
\emng
\eentry

\bentry
\word{gullibility}
\pron{galibiliTi}
\gl{\nA}
\bmng
 sulaBavaMcaniVyate; sulaBavAgi naMbi moVsahoVguva dwbaRlayx; naMbi keDuva -- maMkutana, heDaDxtana, bepupxtana. 
\emng
\eentry

\bentry
\word{gullible}
\pron{galibflf}
\gl{\gu}
\bmng
 sulaBavaMcaniVya; sulaBavAgi naMbisi moVsagoLisabahudAda. 
\emng
\eentry

\bentry
\word{gullish}
\pron{galiSf}
\gl{\gu}
\bmng
 maMkutanada; bepupxtanada; sulaBavAgi naMbi moVsahoVguva savxBAvada. 
\emng
\eentry

\bentry
\word{gull-wing}
\pron{galfviMgf}
\gl{\nA}
\bmng
 galf rekekx: 
\banum
\alnum{a} (moVTAru kArina \vi) meVlakekx -- bicicxkoLuLxva, teredukoLuLxva bAgilu. 
\alnum{b} (vimAnada \vi) cikakxdAda oLameY BAgavu oDaliniMda meVlakekx OTavAgidudx udadxvAda hora meY BAgavu maTaTxsavAgiruva rekekx. 
\eanum
\emng
\eentry

\bentry
\word[gully(1)]{gully}
\pron{gali}
\gl{\nA}
\bmng
\bnum
\num{1} (hariyuva niVriniMda saveda) korakalu; kamari; kaMdara; haLaLx. 
\numi{2} (kaqtakavAgi nimiRsida ALavAda) 
\banum
\alnum{a} kAluve. 
\alnum{b} caraMDi; moVri. 
\alnum{c} bacacxlu. 
\alnum{d} bacacxluguMDi. 
\eanum
\numie
\num{3} (kirxkeTf) gali; pAyiMTfgU silxpfsxgU naDuvaNa phiVlfDx mADuva parxdeVsha. 
\num{4} (\AseTxrXV\ matutx nUyxsiZVlaMDf) nadiya kaNive. 
\enum
\emng
\eentry

\bentry
\word[gully(2)]{gully}
\pron{gali}
\gl{\sakirx}
\bmng
\bnum
\num{1} ALavAda doDaDx caraMDigaLanunx -- tege, toVDu, mADu. 
\num{2} hariyuva niVriniMda savesi kAluvegaLanunx mADu; korakalu kAluve mADu. 
\enum
\emng
\eentry

\bentry
\word[gully(3)]{gully}
\pron{gali}
\gl{\nA}
\bmng
 (\birx) doDaDx -- cUri, cAku. 
\emng
\eentry

\bentry
\word{gully-drain}
\pron{galiDerxVnf}
\gl{\nA}
\bmng
 oLacaraMDi moVri; rasetxya caraMDiya rocucxniVru doDaDx oLacaraMDige hoVguva moVri. 
\emng
\eentry

\bentry
\word{gully-hole}
\pron{galihoVlf}
\gl{\nA}
\bmng
 caraMDi guMDi; rocucxniVru biVdiyiMda caraMDiyoLakekx hoVguvaMte rasetxya pakakxdalilx mADiruva guMDi. 
\emng
\eentry

\bentry
\word{gully-trap}
\pron{galiTArxYxpf}
\gl{\nA}
\bmng
 caraMDi taDe; galiTArxYxpf; koLave caraMDigaLalilx iTiTxruva, keTaTx gALiya taDe. 
\emng
\eentry

\bentry
\word{gulosity}
\pron{gUyxlAsiTi}
\gl{\nA}
\bmng
 (sAhitayxka \parx) hoTeTxbAkatana; tiMDipoVtatana. 
\emng
\eentry

\bentry
\word[gulp(1)]{gulp}
\pron{galfpx}
\gl{\sakirx}
\bmng
 (AturadiMda, ati AsheyiMda, parxyAsadiMda) nuMgibiDu; guTukisiko. 
\emng

\noindent
\gl{\akirx}
\bmng
\bnum
\num{1} kuDiyalArade kuDi; kaSaTxdiMda guTukisiko. 
\num{2} (doDaDx guTukanunx kuDiyuvAga yA hAge kuDiyuvAga AguvaMte) usiru hiDiduko; gaMTalu aDaciko. 
\enum
\emng

\noindent
\gl{\pagu}
\bmng
 \eng{gulp back} (\engit{or} \eng{down) sobs, tears} bikikxbikikx aLuvudanunx yA kaNiNxVranunx -- taDeduko, aDagisiko, nuMgiko. 
\emng
\eentry

\bentry
\word[gulp(2)]{gulp}
\pron{galfpx}
\gl{\nA}
\bmng
\bnum
\num{1} guTuku; nuMguvudu; guTukisikoLuLxvudu: \eng{drained it at one gulp} adanunx oMdeV guTukige kuDidubiTeTx. 
\num{2} nuMguvike; guTukisikoLuLxvike; nuMguva parxyatanx, kirxye. 
\num{3} gukukx; tututx yA guTuku; oMdu salakekx nuMgabahudAda parxmANa. 
\enum
\emng
\eentry

\bentry
\word{gulpingly}
\pron{galipxMgfli}
\gl{\kirxvi}
\bmng
 gaTagaTane; oMdeV guTukige. 
\emng
\eentry

\bentry
\word{gulpy}
\pron{galipx}
\gl{\gu}
\bmng
 (doDaDx guTukanunx nuMguvAga AguvaMte) usiru hiDidukoLuLxva; gaMTalu aDacikoLuLxva. 
\emng
\eentry

\bentry
\word[gum(1)]{gum}
\pron{gamf}
\gl{\nA}
\bmng
(\sA\ \bava dalilx) osaDu; bAyalilx halulxgaLu hugidiruva gaTiTx mAMsa. 
\emng
\eentry

\bentry
\word[gum(2)]{gum}
\pron{gamf}
\gl{\nA}
\bmng
\bnum
\num{1} goVMdu; baMTe; aMTu. 
\num{2} kaNaNx -- kasaru, pisuru. 
\num{3} gamf; goVMdu miThAyi; jileTinf \mo vugaLiMda tayArisida pAradashaRkavAda gaTiTx miThAyi. 
\hypertarget{gum(2)4}{} 
\num{4} goVMdu (sarxvisuva) mara. 
\num{5} baMke roVga; goVMdu beVne; haNuNx giDagaLalilx aMTu suriyuva roVga. 
\num{6} (\ame)  = \hyperlink{gumboot}{gumboot}. 
\num{7} (\ame) = \hyperref{kandict_c.pdf}{C}{chewing-gum}{chewing-gum}. 
\num{8} reVSemx meVNa; reVSemx aMTu; hasi yA saMsakxrisada reVSemx eLegaLa sutatx iruva meVNadaMtha aMTu padAthaR. 
\num{9} peTorxVliyaM \mo vugaLalilxruva gaSuTx. 
\enum
\emng
\eentry

\bentry
\word[gum(3)]{gum}
\pron{gamf}
\gl{\kirx}
\expl{(\vakaq\ \eng{gumming,} \BU\ matutx \BUkaq\ \eng{gummed}).}
\bmng
\emng

\noindent
\gl{\sakirx}
\bmng
\bnum
\num{1} goVMdiniMda, goVMdu baLidu -- raTATxgisu, raTiTxnaMtAgisu. 
\num{2} goVMdu -- hAku, baLi, hacucx, leVpisu. 
\num{3} goVMdiniMda (yAvudanenxV) BadarxvAgi -- aMTisu, seVrisu, baMdhisu. 
\enum
\emng

\noindent
\gl{\akirx}
\bmng
 (haNiNxna giDada yA marada \vi) (roVgadiMda) goVMdu surisu, aMTu -- soVrisu, sarxvisu, jinugu, osaru. 
\emng

\noindent
\gl{\pagu}
\bmng
 \eng{gum up} aDacaNe mADu; aDiDxpaDisu; hALu mADu; ekakxhuTiTxsu; holabugeDisu; susUtarxvAgi naDeyuvudara madheyx parxveVshisu: \eng{gum up the works} kelasagaLige aDiDxpaDisu. 
\emng
\eentry

\bentry
\word[gum(4)]{gum}
\pron{gamf}
\gl{\nA}
\bmng
 (\asaM) (ANegaLu, udAgxragaLu, \mo vugaLalilx) deVvaru: \eng{by gum} deVvarANe! 
\emng
\eentry

\bentry
\word{gumbo}
\pron{gaMboV}
\gl{\nA}
\expl{(\bava\ \eng{gumbos}).}
\bmng
(\ame) 
\bnum
\num{1} beMDe(kAyi, giDa). 
\num{2} beMDekAyi sAru; beMDekAyi biVjagaLiMda gaTiTxyAgisida esaru kaTuTx. 
\num{3} (\eng{Gumbo}) gaMboV; amerikada lUyisiyAna saMsAthxnada kariyaru matutx kirxyoVlf niVgorxVgaLa ADuBASe. 
\enum
\emng
\eentry

\bentry
\word{gumboot}
\pron{gamfbUTf}
\gl{\nA}
\bmng
 rababxrf bUTu. 
\emng
\eentry

\bentry
\wordnospeech{gum dragon}{gum dragon}
\pron{?}
\gl{\nA}
\bmng
 (auSadhagaLalilx, kAyxliko mudarxNa, \mo vugaLalilx baLasuva, AsaTxrXgalusf baLagada sasayxgaLalilx doreyuva) oMdu bageya biLiya yA keMpu aMTu. 
\emng
\eentry

\bentry
\wordnospeech{gum juniper}{gum juniper}
\pron{?}
\gl{\nA}
\bmng
 shaMku rALa; koVnipharf rALa; vAniRSf tayArikeyalUlx gugugxLavAgiyU baLasuva, utatxra AphirxkAda koVnipharf marada rALa. 
\emng
\eentry

\bentry
\word{gumlah}
\pron{gaMlA}
\gl{\nA}
\bmng
 raMjaNige; haMDe; iMDiyada doDaDx gAtarxda maNiNxna pAterx, doDaDx maDake yA jADi. 
\emng
\eentry

\bentry
\word{gumma}
\pron{gamA}
\gl{\nA}
\expl{(\bava\ \eng{gummas, gummata}).}
\bmng
(\roVshA) gamamx; siphilisf gaMTu; upadaMsha (siphilisf roVga) da pariNAmavAgi uMTAguva, rababxrinaMtha geDeDx. 
\emng
\eentry

\bentry
\word{gummatous}
\pron{gamaTasf}
\gl{\gu}
\bmng
 (\roVshA) gamamx pUrita; siphilisf gaMTugaLiMda kUDida; upadaMshada geDeDxgaLiMda kUDida. 
\emng
\eentry

\bentry
\word{gummily}
\pron{gamili}
\gl{\kirxvi}
\bmng
 halilxlalxde; halilxlalxdaMte; nidaRMtavAgi; daMtarahitavAgi. 
\emng
\eentry

\bentry
\word{gumminess}
\pron{gaminisf}
\gl{\nA}
\bmng
\bnum
\num{1} aMTaMTAgiruvike; jiguTujiguTAgiruvike; aMTutana. 
\num{2} goVMdu -- tuMbiruvike, sarxvisuvike. 
\num{3} (kAlugaLa, kAluharaDugaLa \vi) Uta; bAvu. 
\enum
\emng
\eentry

\bentry
\word{gummite}
\pron{gameYTf}
\gl{\nA}
\bmng
 (\BUvi) gameYTu; yureVniyamf, thoVriyamf matutx siVsada AkesxYDugaLuLaLx oMdu Kanija. 
\emng
\eentry

\bentry
\word[gummy(1)]{gummy}
\pron{gami}
\gl{\gu}
\bmng
\bnum
\num{1} aMTaMTAda; jiguTu jigaTAda. 
\num{2} goVMdu tuMbida, sarxvisuva. 
\num{3} (kAlugaLa, kAluharaDugaLa \vi) UdikoMDa; bAtukoMDa. 
\enum
\emng
\eentry

\bentry
\word[gummy(2)]{gummy}
\pron{gami}
\gl{\gu}
\bmng
 halilxlalxda; daMtarahita; nidaRMta. 
\emng
\eentry

\bentry
\word[gummy(3)]{gummy}
\pron{gami}
\gl{\nA}
\bmng
\bnum
\num{1} (\AseTxrXV) shAMtasAgarada saNaNx SAkfR mInu. 
\num{2} (\AseTxrXV\ matutx nUyxsiZVlaMDf) halilxlalxda kuri; nidaRMta kuri; halulxgaLanunx kaLedukoMDa yA kaLedukoLuLxtitxruva kuri. 
\enum
\emng
\eentry

\bentry
\word{gump}
\pron{gaMpf}
\gl{\nA}
\bmng
 (\AmA)  = \hyperlink{gumption}{gumption}. 
\emng
\eentry

\bentry
\word{gumption}
\pron{gaMpaSxnf}
\gl{\nA}
\bmng
\bnum
\num{1} (\AmA) kecucx; CAti; diTaTxtana; dASiTxkatana; mununxgugxva sAhasa. 
\num{2} (kiSxparx) vayxvahAra -- jAcnxna, kwshala, cAtuyaR. 
\num{3} (citarxkale) vaNaRvAhaka; vaNaR mAdhayxma; baNaNxgaLanunx yAva vasutxvinalilx beresalAguvudoV aMtha AdhAravasutx. 
\num{4} (citarxkArana) vaNaRmisharxNa kwshala; baNaNxgaLanunx tayArisuva kale. 
\enum
\emng
\eentry

\bentry
\wordnospeech{gum resin}{gum resin}
\pron{?}
\gl{\nA}
\bmng
goVMdu rALa; goVMdinoDane beresida sasayxrALa, \udA\ \eng{gamboge.} 
\emng
\eentry

\bentry
\word{gumshoe}
\pron{gamfSU}
\gl{\nA}
\bmng
 (\ame) 
\bnum
\num{1}  = \hyperlink{galosh}{galosh}. 
\num{2} (\AmA) rahasayx calanavalana yA rahasayxkirxye; guTATxgi, rahasayxvAgi kelasa mADuvudu, calisuvudu. 
\num{3} (\AmA) patetxVdAra; beVhugAra yA poliVsinava. 
\enum
\emng
\eentry

\bentry
\word{gum-tree}
\pron{gamfTirxV}
\gl{\nA}
\bmng
 = \hyperlink{gum(2)4}{$^2$gum (4)}. 
\emng

\noindent
\gl{\nuga}
\bmng
 \eng{up a gum-tree} bahaLa ikakxTiTxnalilx sikikxkoMDu; dikukxgANada peVcinalilx; tuMba kaSaTxdalilx. 
\emng
\eentry

\bentry
\word[gun(1)]{gun}
\pron{ganf}
\gl{\nA}
\bmng
\bnum
\num{1} ganunx; baMdUku; koVvi; madudx yA beVre soPxVTakadiMda guMDu \mo\ kiSxpaNigaLanunx hArisuva loVha koLave. 
\num{2} ganunx; PiraMgi;toVpu; baMdUku; tupAki; koVvi; reYphalulx. 
\num{3} (\ame) rivAlavxru. 
\num{4} (paMdayxgaLanunx) pArxraMBisuva pisUtxlu. 
\num{5} picakAri; siMpaNi; kiVTanAshaka, girxVsf, ilekATxrXnfgaLu, \mo vanunx nidiRSaTx dikikxnalilx siMpaDisuva sAdhana. 
\num{6} SikAri taMDadavanu; koVvi beVTegArara taMDadavanu. 
\num{7} (\ame)  = \hyperlink{gunman}{gunman}. 
\num{8} (\bava\ dalilx) (nwkA \ashi) PiraMgi (hArisuva) adhikAri. 
\enum
\emng

\noindent
\gl{\nuga}
\bmng
\bnum
\num{1} \eng{at gun point} koVvi toVrisi; baMdUku bedarikeyiMda; baMdUkiniMda kolulxvudAgi hedarisi. 
\numi{2} \eng{beat} (\engit{or} \eng{jump) the gun} 
\banum
\alnum{a} saMkeVta, sUcane koDuvudakikxMta muMceyeV horaDu. 
\alnum{b} (nigadita) kAlakekx modaleV (kAyaR) pArxraMBisu. 
\eanum
\numie
\num{3} \eng{big gun} (\ashi) doDaDxmanuSayx; parxmuKa vayxkitx; gaNayx. 
\num{4} \eng{blow great guns} (birugALiya \vi) raBasadiMda, parxcaMDavAgi -- biVsu. 
\num{5} \eng{give engine, motor vehicle the gun} (\AmA) eMjinu, moVTAru vAhanagaLige veVga koDu, avugaLa veVga hecicxsu. 
\hypertarget{gun(1) nuga(6)}{} 
\num{6} \eng{going great guns} vijayadatatx, geluvinatatx -- BaradiMda sAgutitxru. 
\hyperdef{G}{gun(1) nuga(7)}{} 
\num{7} \eng{son of a gun} (\AmA) kuSxdarx; niVca; tirasAkxrAhaR vayxkitx. 
\hyperdef{G}{gun(1) nuga(8)}{} 
\num{8} \eng{stand} (\engit{or} \eng{stick) one's guns} tananx niluvanunx biDadiru; paTuTx hiDidu sAdhisu. 
\num{9} \eng{sure as a gun} (baMdUkinaMte) guritapapxde; KaMDita; nisasxMdeVhavAgi. 
\enum
\emng
\eentry

\bentry
\word[gun(2)]{gun}
\pron{ganf}
\gl{\kirx}
\expl{(\BU\ matutx \BUkaq\ \eng{gunned,} \vakaq\ \eng{gunning}).}
\bmng
\emng

\noindent
\gl{\sakirx}
\bmng
\bnum
\num{1} (obabxna, oMdara meVle) baMdUku -- hArisu, hoDe. 
\num{2} baMdUkiniMda hoDedu keDavu, uruLisu. 
\num{3} (\AmA) (eMjinunx \mo vugaLa) veVga hecicxsu. 
\enum
\emng

\noindent
\gl{\akirx}
\bmng
\bnum
\num{1} koVvi beVTeyADu; baMdUkiniMda beVTeyADu; baMdUkina sameVta beVTege hoVgu. 
\num{2} baMdUkiniMda hoDe; koVvi -- hoDe, hArisu. 
\enum
\emng

\noindent
\gl{\pagu}
\bmng
\bnum
\num{1} \eng{gun down} baMdUkiniMda hoDedu hAku. 
\numi{2} \eng{gun for} 
\banum
\alnum{a} (kolalxlu yA hAni mADalu) baMdUku sameVta -- arasu, huDukikoMDu hoVgu. 
\alnum{b} (\rUpa) (obabxna meVle) halelx naDesalu, (obabxnige) hAni mADalu, yA (obabxnanunx) kolalxlu -- hoVgu, parxyatinxsu. 
\alnum{c} (\rUpa) huDuku; arasu; paDedukoLaLxlu parxyatinxsu. 
\eanum
\numie
\enum
\emng
\eentry

\bentry
\word{gunboat}
\pron{ganfboVTf}
\gl{\nA}
\bmng
PiraMginwke; BAravAda PiraMgigaLiruva saNaNx yudadhxnwke. 
\emng

\noindent
\gl{\pagu}
\bmng
 \eng{gunboat diplomacy} PiraMginwke rAjataMtarx; seVnA balaparxyoVgada bedarikeyanonxDuDxva rAjataMtarx. 
\emng
\eentry

\bentry
\word{gun-carriage}
\pron{ganfkAYxrijf}
\gl{\nA}
\bmng
 PiraMgi baMDi; toVpugADi; PiraMgiyanunx uDAyisalu, sAgisalu baLasuva, gAligaLiruva cwkaTuTx. 
\emng
\eentry

\bentry
\word{gun-cotton}
\pron{ganfkATanf}
\gl{\nA}
\bmng
 siDihatitx; seluyxloVsf neYTerxVTf; seluyxloVsanunx, \kanmu\ hatitxyanunx, neYTirxkf matutx salUphxyXrikf AmalxgaLiMda saMsakxrisi tayArisida sophxVTaka. 
\emng
\eentry

\bentry
\wordnospeech{gun crew}{gun crew}
\pron{?}
\gl{\nA}
\bmng
 PiraMgi paDe; goVlaMdAju daLa; PiraMgi cAlaka daLa;. 
\emng
\eentry

\bentry
\wordnospeech{gun dog}{gun dog}
\pron{?}
\gl{\nA}
\bmng
 koVvi (beVTe)nAyi; koVvigaLiMda beVTeyADuvAga SikAri mADuvavara jate hoVguvaMte tarapeVti koTaTx nAyi. 
\emng
\eentry

\bentry
\word{gun-fight}
\pron{ganfpheYTf}
\gl{\nA}
\bmng
 (\ame) (\AmA) koVvi kALaga; PiraMgi yudadhx; koVvi yA PiraMgigaLanunx baLasi mADuva yudadhx. 
\emng
\eentry

\bentry
\word{gun-fire}
\pron{ganfpheYarf}
\gl{\nA}
\bmng
\bnum
\num{1} baMdUku hArisuvudu. 
\num{2} (\kanmu\ seYnayxdalilx matutx nwkeyalilx, beLagegx matutx saMje hotutxgaLanunx sUcisalu hArisuva) veVLeya guMDu. 
\num{3} sAlu hoDeta; PiraMgiya yA baMdUkina sAlinalilx oMdoMdu baMdUkU sAlAgi hoDeyuva hoDeta. 
\enum
\emng
\eentry

\bentry
\word{gunge}
\pron{gaMjf}
\gl{\nA}
\bmng
  = \hyperlink{gunk}{gunk}. 
\emng
\eentry

\bentry
\word{gung-ho}
\pron{gaMgfhoV}
\gl{\gu}
\bmng
 utAsxhi; utAsxhaBarita; utAsxhapUNaR; utusxka. 
\emng
\eentry

\bentry
\word{gun-harpoon}
\pron{ganfhApURnf}
\gl{\nA}
\bmng
 baMdUku ITi; koVvi ITi; (timiMgila \mo vanunx tividu hiDiyalu keYyiMdalalxde) baMdUkiniMda hArisuva ITi. 
\emng
\eentry

\bentry
\word{gunk}
\pron{gaMkf}
\gl{\nA}
\bmng
 (\ashi) aMTaMTAda yA darxvarUpadalilxruva vasutx. 
\emng
\eentry

\bentry
\word{gun-layer}
\pron{ganfleVarf}
\gl{\nA}
\bmng
 koVvi gurikAra; doDaDx baMdUkada guri -- iDuvava, tirugisuvava. 
\emng
\eentry

\bentry
\word{gunless}
\pron{ganflisf}
\gl{\gu}
\bmng
 baMdUkilalxda; koVvi rahita. 
\emng
\eentry

\bentry
\word{gun-lock}
\pron{ganflAkf}
\gl{\nA}
\bmng
 koVvi cApu; baMdUkina madadxnUnx guMDanUnx hArisuva baMdUkina salakaraNe. 
\emng
\eentry

\bentry
\word{gunman}
\pron{ganfma(mAyx)nf}
\gl{\nA}
\bmng
\bnum
\num{1} koVvikoVra; baMdUku hArisi bedarisuva yA kolulxva daroVDeKoVra. 
\num{2} baMdUkudhAri; koVvidhAri; baMdUku hiDidiruvavanu yA hArisuvudaralilx kushalanAdavanu. 
\num{3} koVvigAra; baMdUku tayArisuvava. 
\enum
\emng
\eentry

\bentry
\word{gun-metal}
\pron{ganfmeTalf}
\gl{\nA}
\bmng
\bnum
\num{1} (hiMde baMdUku, PiraMgi, \mo vanunx tayArisuvalilx baLasutitxdadx) baMdUku loVha; koVvi loVha; tAmarx matutx tavara yA tAmarx matutx satu -- ivugaLiMda tayArisida oMdu misharxloVha. 
\num{2} maMkAda bUduniVli baNaNx. 
\enum
\emng
\eentry

\bentry
\wordnospeech{gun moll}{gun moll}
\pron{?}
\gl{\nA}
\bmng
\bnum
\num{1} koVvi perxVyasi; kaLaLxru, daroVDekoVraru, \mo vara upapatinx. 
\num{2} koVvigAtiR; baMdUkudhAriNi; baMdUku hiDidu aparAdhagaLanunx naDesuva heMgasu. 
\enum
\emng
\eentry

\bentry
\word{gunned}
\pron{ganfDx}
\gl{\gu}
\bmng
 tupAkisajijxta; koVvi, baMdUku, PiraMgi, \mo vugaLiMda kUDida, sajijxtavAda. 
\emng
\eentry

\bentry
\word[gunnel(1)]{gunnel}
\pron{ganalf}
\gl{\nA}
\bmng
 koVvimInu; hAvu mIninAkAkArada saNaNx kaDala mInu. 
\emng
\eentry

\bentry
\word[gunnel(2)]{gunnel}
\pron{ganalf}
\gl{\nA}
\bmng
 = \hyperlink{gunwale}{gunwale}. 
\emng
\eentry

\bentry
\word{gunner}
\pron{ganarf}
\gl{\nA}
\bmng
\bnum
\num{1} (\kanmu\ seYnikana hudedxya hesarAgi) goVlaMdAja toVpugAra; PiraMgidAra; PiraMgi -- seYnika yA adhikAri. 
\num{2} (\nw) toVpadhikAri; toVpuKAne, madidxna mane, \mo vugaLa adhikAri. 
\num{3} (kADupArxNi, hakikx, \mo vugaLa) beVTegAra; SikAri(yava). 
\num{4} vimAna koVvigAra; baMdUku hArisuva vimAna cAlaka vagaRda sadasayx. 
\enum
\emng

\noindent
\gl{\pagu}
\bmng
 \eng{Master Gunner} (doreya) AyudhAgArada adhikAri; shasAtxrXsatxrXda ugArxNavanunx noVDikoLuLxva adhikAri. 
\emng

\noindent
\gl{\nuga}
\bmng
\bnum
\num{1} \eng{gunner's daughter} (\hA) caDiPiraMgi; caDiyeVTanunx koDalu nAvikaranunx bigiyutitxdadx PiraMgi. 
\num{2} \eng{kiss or marry the gunner's daughter} caDiyeVTu -- tinunx, hoDesiko. 
\enum
\emng
\eentry

\bentry
\word{gunnera}
\pron{ganara}
\gl{\nA}
\bmng
 ganara: 
\banum
\alnum{a} alaMkArakAkxgi beLesuva, suMdaravAda doDaDx doDaDx elegaLanunx biDuva giDagaLa oMdu kula. 
\alnum{b} I kulada giDa. 
\eanum
\emng
\eentry

\bentry
\word{gunnery}
\pron{ganari}
\gl{\nA}
\bmng
\bnum
\num{1} toVpugArike; PiraMgi videyx; doDaDx PiraMgigaLanunx nimiRsuva matutx nivaRhisuva videyx; BAriV PiraMgigaLa nimARNa matutx nivaRhaNa. 
\num{2} PiraMgi hoDeta; toVpu, baMdUku -- hArisuvudu. 
\num{3} (sAmUhikAthaRdalilx) PiraMgigaLu; toVpugaLu; baMdUkugaLu; tupAkigaLu. 
\enum
\emng
\eentry

\bentry
\word{gunning}
\pron{ganiMgf}
\gl{\nA}
\bmng
\bnum
\num{1} (\kanmu\ beVTeya pArxNige) baMdUku hoDeyuvudu. 
\num{2} (SikAri doVNiyalilx kuLitu mADuva) kADu hakikxya beVTe. 
\enum
\emng

\noindent
\gl{\pagu}
\bmng
 \eng{go gunning} SikArige hoVgu; beVTege hoVgu. 
\emng
\eentry

\bentry
\word{gunny}
\pron{gani}
\gl{\nA}
\bmng
 (\sA\ seNabunArina) goVNiciVla yA goVNi taTuTx. 
\emng
\eentry

\bentry
\word{gun-pit}
\pron{ganfpiTf}
\gl{\nA}
\bmng
 (shaturxgaLa PiraMgi hoDetadiMda tamamxnUnx, tamamx PiraMgigaLanUnx rakiSxsikoLaLxlu toVDiruva) PiraMgi guNi; toVpuguMDi. 
\emng
\eentry

\bentry
\word{gun-play}
\pron{ganfpelxV}
\gl{\nA}
\bmng
% 
\bnum
\num{1} PiraMgi, baMdUku, \mo vugaLa baLake. 
\num{2} PiraMgi kAdATa; koVvi kALaga; baMdUku hoDedATa; PiraMgi, baMdUku, \mo vugaLanunx baLasi kAdADuvudu. 
\enum
\emng
\eentry

\bentry
\word{gun-powder}
\pron{ganfpwDarf}
\gl{\nA}
\bmng
% 
\bnum
\num{1} (peTulxpupx, gaMdhaka matutx ididxlu beresi tayArisida) koVvimadudx; baMdUku siDimadudx. 
\num{2} saNaNx saNaNx kaNagaLaMtiruva oMdu utakxqqSaTxvAda bageya TiV sopupx. 
\enum
\emng

\noindent
\gl{\pagu}
\bmng
 \eng{gun-powder plot} siDimadidxna saMcu; ganfpwDarf pAlxTf; \eng{1605}neya naveMbarf \eng{5}ralilx laMDaninxnalilxna pAliRmeMTina kaTaTxDavanunx madidxniMda dhavxMsagoLisalu naDesida oMdu pitUri. 
\emng
\eentry

\bentry
\word{gunpower}
\pron{ganfpwarf}
\gl{\nA}
\bmng
 baMdUkubala; PiraMgibala; laBayxviruva PiraMgigaLa oTuTx saMKeyx. 
\emng
\eentry

\bentry
\word{gunroom}
\pron{ganfrUmf}
\gl{\nA}
\bmng
 (\birx) 
\bnum
\num{1} toVpu koVNe; yudadhxnwkeyalilx kiriya adhikArigaLige biDAravoV, lephiTxneMTfgaLige UTada sathxLavoV Ada koThaDi. 
\num{2} baMdUku koThaDi; maneyalilx SikAri baMdUkanunx iDuva koThaDi. 
\enum
\emng
\eentry

\bentry
\word{gun-runner}
\pron{ganfranarf}
\gl{\nA}
\bmng
 baMdUku (\mo vugaLa) kaLaLxsAgaNedAra; nAyxyabAhiravAgi deVshadoLakekx baMdUku \mo\ shasAtxrXsatxrXgaLanunx sAgisuvavanu. 
\emng
\eentry

\bentry
\word{gun-running}
\pron{ganfraniMgf}
\gl{\nA}
\bmng
 baMdUku (\mo vugaLa) kaLaLxsAgaNe. 
\emng
\eentry

\bentry
\word{gunship}
\pron{ganfSipf}
\gl{\nA}
\bmng
 shasAtxrXsatxrXsajijxta helikApaTxru yA vimAna. 
\emng
\eentry

\bentry
\word{gunshot}
\pron{ganfSATf}
\gl{\nA}
\bmng
\bnum
\num{1} baMdUku hoDeta; baMdUkiniMda hoDeda guMDu. 
\num{2} baMdUku dUra; koVvidUra; koVviyaLate; baMdUku guMDu talapuva dUra: \eng{within gunshot} baMdUku dUradoLage; koVviyaLateyoLage. 
\enum
\emng
\eentry

\bentry
\word{gun-shy}
\pron{ganfSeY}
\gl{\gu}
\bmng
 (\kanmu\ beVTenAyiya \vi) koVvi hedarikeya; baMdUkuBayada; baMdUkina shabadxkekx hedaruva. 
\emng
\eentry

\bentry
\word{gun-site}
\pron{ganfseYTf}
\gl{\nA}
\bmng
 koVvinele; baMdUku jAga; baMdUkanunx sAthxpisiruva (\sA\ sutatxlU rakaSxNeyodagisida) sathxLa, jAga. 
\emng
\eentry

\bentry
\word{gun-slinger}
\pron{ganfsilxMgarf}
\gl{\nA}
\bmng
  = \hyperlink{gunman}{gunman}. 
\emng
\eentry

\bentry
\word{gunsmith}
\pron{ganfsimxtf}
\gl{\nA}
\bmng
 koVvikamAmxra; baMdUkugAra; PiraMgi, toVpu, \mo vugaLanunx tayArisuvavanu yA ripeVri mADuvavanu. 
\emng
\eentry

\bentry
\word{gun-stock}
\pron{ganfsATxkf}
\gl{\nA}
\bmng
 baMdUku buDa; koVvibuDa; baMdUku naLikeyanunx kUrisiruva marada BAga. 
\emng
\eentry

\bentry
\word{gunter}
\pron{gaMTarf}
\gl{\nA}
\bmng
\bnum
\num{1} gaMTaru; gaMTarf (aLate) kaDiDx; moVjiNiya matutx nwkAgati shAsatxrXda lekakxgaLanunx sugamagoLisalu baLasuva, aLategere lAgaridamumxgaLa gaNanareVKe, \mo vanunx gurutumADiruva, eraDu aDi udadxda capapxTe aLatepaTiTx. 
\num{2} gaMTarf; netitxkUve yA adara paTa; keLakUveya meVle baLegaLa mUlaka meVlakUkx keLakUkx sariyabalalx netitxkUve yA adara paTa. 
\enum
\emng
\eentry

\bentry
\wordnospeech{Gunter's chain}{Gunter's chain}
\pron{?}
\gl{\nA}
\bmng
 gaMTarf sarapaNi; \eng{66} aDi udadxda moVjiNi sarapaNi. 
\emng
\eentry

\bentry
\wordnospeech{Gunter's scale}{Gunter's scale}
\pron{?}
\gl{\nA}
\bmng
  = \hyperlink{gunter}{gunter(1)}. 
\emng
\eentry

\bentry
\word{gunwale}
\pron{ganalf}
\gl{\nA}
\bmng
 (\kanmu\ cikakx) haDagina yA doVNiya pakakxda meVlaMcu. 
\emng

\noindent
\gl{\pagu}
\bmng
\bnum
\num{1} \eng{gunwale to} niVrina maTaTxkekx samanAgi. 
\num{2} \eng{gunwale under} niVrina maTaTxda keLage. 
\enum
\emng
\eentry

\bentry
\word{gunyah}
\pron{ganAyx}
\gl{\nA}
\bmng
 (\AseTxrXV) (AdivAsigaLa) joVpaDi; guDisalu; guDilu. 
\emng
\eentry

\bentry
\word{gup}
\pron{gapf}
\gl{\nA}
\bmng
 (\AmA) guDuDxharaTe; kADuharaTe. 
\emng
\eentry

\bentry
\word[guppy(1)]{guppy}
\pron{gapi}
\gl{\nA}
\bmng
gaPi; jaloVdAyxnagaLalilx sAkuva, vesfTxiMDiVsfna saNaNx mInu. 
\emng
\eentry

\bentry
\word[guppy(2)]{guppy}
\pron{gapi}
\gl{\nA}
\bmng
 gapi; niVrina parxtiroVdhavanunx tagigxsuvaMte mIninAkAradalilx racisida, gALiyanunx odagisuva vayxvasethxyuLaLx, jalAMtagARmi nwke. 
\emng
\eentry

\bentry
\word{gurdwara}
\pron{gadAvxRra}
\gl{\nA}
\bmng
 gurudAvxra; siKaKxra deVvasAthxna. 
\emng
\eentry

\bentry
\word{gurgitation}
\pron{gagiRTeVSanf}
\gl{\nA}
\bmng
\bnum
\num{1} ukekxVruvudu; aleyaleyAgi-tonedATa, hoyAdxTa: \eng{the gurgitation of the waves} alegaLu ukekxVruvudu. 
\num{2} ukekxVruva yA buLabuLa kudita yA adara shabadx. 
\enum
\emng
\eentry

\bentry
\word[gurgle(1)]{gurgle}
\pron{gagfRlf}
\gl{\sakirx}
\bmng
 guLuguLisutAtx heVLu: \eng{the woman gurgled her greetings} mahiLeyu avaLa vaMdanegaLanunx guLuguLisutAtx heVLidaLu. 
\emng

\noindent
\gl{\akirx}
\bmng
 (siVseyiMda niVru suriyuvAga, kalulxgaLa madheyx niVru hariyuvAga) guLuguLisu; juLu juLisu; juLajuLa, guLaguLa -- shabadx mADu: \eng{he sent the wine gurgling down his throat} avanu gaMTalinalilx guLuguLu sadudx mADutAtx madayxvanunx kuDidanu. 
\emng
\eentry

\bentry
\word[gurgle(2)]{gurgle}
\pron{gagfRlf}
\gl{\nA}
\bmng
 juLujuLu; guLuguLu sadudx. 
\emng
\eentry

\bentry
\word{gurjun}
\pron{gajaRnf}
\gl{\nA}
\bmng
 gajaRnf mara; auSadhiyAgi yA vAniRSf tayArikeyalilx gugugxla darxva yA eNeNxyanunx koDuva, DipaTxrokApaRsf kulakekx seVrida, pUvaR iMDiyAda mara. 
\emng
\eentry

\bentry
\wordnospeech{gurjun balsam}{gurjun balsam}
\pron{?}
\gl{\nA}
\bmng
 gajaRnf (marada) gugugxla. 
\emng
\eentry

\bentry
\word[gurk(1)]{gurk}
\pron{gakfR}
\gl{\kirx}
\bmng
 (\AmA) = \hyperref{kandict_b.pdf}{B}{belch(1)}{$^1$belch}. 
\emng
\eentry

\bentry
\word[gurk(2)]{gurk}
\pron{gakfR}
\gl{\nA}
\bmng
 (\AmA) = \hyperref{kandict_b.pdf}{B}{belch(2)}{$^2$belch}. 
\emng
\eentry

\bentry
\word{Gurkha}
\pron{gakaR, guakaR}
\gl{\nA}
\bmng
 gUKaR; neVpAlada hiMdU rAjavaMshakekx seVridavanu. 
\emng
\eentry

\bentry
\word{gurnard}
\pron{ganaRDfR}
\gl{\nA}
\bmng
 ganaRDfR; muLuLxgaLiMda kUDida udadxneya mUtiyiruva, TirxgilxDeV vaMshakekx seVrida, oMdu kaDala mInu. 
\emng
\eentry

\bentry
\word{gurnet}
\pron{ganiRTf}
\gl{\nA}
\bmng
  = \hyperlink{gurnard}{gurnard}. 
\emng
\eentry

\bentry
\word{guru}
\pron{gurU}
\gl{\nA}
\bmng
\bnum
\num{1} guru; AcAyaR; hiMdUgaLa AdhAyxtimxka guru yA dhAmiRka paMthada muKaMDa. 
\num{2} guru; parxBAvashAliyAda -- upadeVshaka, boVdhaka. 
\num{3} guru; Bakitx gwravagaLiMda purasakxqqtanAda salahegAra, budidhx heVLuvava. 
\enum
\emng
\eentry

\bentry
\word[gush(1)]{gush}
\pron{gaSf}
\gl{\sakirx}
\bmng
 (niVranunx) haThAtatxne yA yatheVcaCxvAgi harisu, horaDisu, cimimxsu, surisu (mAtu, anukaMpa, \mo vugaLa \vi \rUpa saha): \eng{the broken pipe gushed a stream of water} murida koLave niVrina hoLeyanunx harisitu. 
\emng

\noindent
\gl{\akirx}
\bmng
\bnum
\num{1} (thaTaTxne yA tuMbu parxvAhadaMte) nugugx; hari; suri; cimumx; horaDu (aneVka veVLe BASaNa, karuNe, pirxVti, \mo vugaLa \vi \rUpa): \eng{the blood gushed from the wound} gAyadiMda rakatx cimimxtu. 
\num{2} ukikx hariyuvaMte, BAvagadagxdateya soVginiMda mAtanADu, vatiRsu: \eng{girls who gush over film stars} sinimA naTara bagegx ukikx hariyuvaMte, atuyxtAsxhadiMda mAtanADuva huDugiyaru. 
\enum
\emng
\eentry

\bentry
\word[gush(2)]{gush}
\pron{gaSf}
\gl{\nA}
\bmng
\bnum
\num{1} parxvAha; surita; harivu; cimumxge. 
\num{2} (darxvada) thaTaTxneya yA yatheVSaTx -- visajaRne, horacelilxke. 
\num{3} hAge visajiRsalapxTaTx darxvada pariNAma. 
\num{4} (\rUpa) cimumxge; thaTaTxneya horahomimxke. 
\num{5} (\AmA) BAvaneya ukekxVrike; BAvoVdevxVgada parxdashaRna. 
\enum
\emng
\eentry

\bentry
\word{gusher}
\pron{gaSarf}
\gl{\nA}
\bmng
\bnum
\num{1} ukekxVruva vayxkitx yA vasutx; ukikxhariyuvaMte, BAvagadagxdateyiMda -- mAtanADuvava, vatiRsuvava. 
\num{2} eNeNxbugegx; teYlabugegx; paMpumADade teYla tAneVtAnAgi cimumxva eNeNx bAvi. 
\enum
\emng
\eentry

\bentry
\word{gushing}
\pron{gaSiMgf}
\gl{\gu}
\bmng
\bnum
\num{1} (tuMbuparxvAhadaMte) nugugxva; hariyuva; suriyuva; cimumxva: \eng{gushing fountains} cimumxva bugegxgaLu. 
\num{2} (darxvavanunx) yatheVcaCxvAgi -- harisuva, horaDisuva, cimimxsuva, surisuva: \eng{gushing eyes} kaNiNxVru surisuva kaNuNxgaLu. 
\num{3} (\rUpa) ukekxVruva; tuMbi hariyuva; BAvAveVgavanunx parxdashiRsuva. 
\num{4} yatheVcaCxvAda; tuMbi tuLukuva. 
\enum
\emng
\eentry

\bentry
\word{gushingly}
\pron{gaSiMgfli}
\gl{\kirxvi}
\bmng
\bnum
\num{1} (tuMbu parxvAha) nugugxvaMte; hariyuvaMte; suriyuvaMte; cimumxvaMte. 
\num{2} (\rUpa) ukekxVruvaMte; tuMbi hariyuvaMte; BAvAveVgavanunx parxdashiRsuvaMte. 
\num{3} yatheVcaCxvAgi; tuMbi tuLukuvaMte. 
\enum
\emng
\eentry

\bentry
\word{gushy}
\pron{gaSi}
\gl{\gu}
\bmng
 ukikxhariyuvaMte atiyAgi -- mAtanADuva, vatiRsuva; ukikx hariyuva mAtu, vataRne, \mo\ lakaSxNavuLaLx: \eng{gushy poetry} (BAvAtireVkadiMda, shabadxbAhuLayxdiMda) ukikx hariyuva kAvayx; atiBAvuka kAvayx. 
\emng
\eentry

\bentry
\word{gusset}
\pron{gasiTf}
\gl{\nA}
\bmng
\bnum
\num{1} mUlepaTiTx; uDupina yAvudAdarU BAgavanunx balagoLisalu yA higigxsalu adaroLagiDuva paTiTx. \imglink{gussets-1figure}{\raisebox{-0.15cm}[0pt][0pt]{\pdfimage width 0.7cm height 0.6cm {G_Pictures/gussets-1.jpg}}} 
\num{2} (kaTaTxDada koVnaBAgavanunx balapaDisalu hAkuva) kabibxNada mUlekaTuTx. 
\enum
\emng
\eentry

\bentry
\word{gusseted}
\pron{gasiTeDf}
\gl{\gu}
\bmng
\bnum
\num{1} mUlepaTiTx yA paTiTxgaLu -- uLaLx. 
\num{2} kabibxNada -- mUlekaTuTx yA kaTuTxgaLu uLaLx. 
\enum
\emng
\eentry

\bentry
\word{gussy}
\pron{gasi}
\gl{\sakirx}
\bmng
 (\ashi) niVTAda uDigetoDige hAkiko; ThAkuThiVkAgu. 
\emng
\eentry

\bentry
\word[gust(1)]{gust}
\pron{gasfTx}
\gl{\nA}
\bmng
\bnum
\num{1} hoyAgxLi; nugugxgALi; thaTaTxne birusiniMda nugugxva gALi. 
\num{2} (maLe, beMki, shabadx yA BAvoVderxVkagaLa) hoDeta; jaDita; soPxVTa; raBasada homimxke; uderxVka; udevxVga: \eng{unruly gusts of passion} hadudx miVrida rAgoVderxVkagaLu. 
\enum
\emng
\eentry

\bentry
\word[gust(2)]{gust}
\pron{gasfTx}
\gl{\akirx}
\bmng
\bnum
\num{1} (gALiya \vi) thaTaTxne birusAgi -- biVsu, nugugx. 
\num{2} (maLe, beMki, shabadx, rAgaBAvagaLa \vi) hoDi; soPxVTisu; raBasavAgi -- homumx, keraLu, uderxVkagoLuLx: \eng{pride gusts up again} gavaR matetx keraLutatxde. 
\enum
\emng
\eentry

\bentry
\word[gust(3)]{gust}
\pron{gasfTx}
\gl{\nA}
\bmng
 (\pArxparx yA \kAparx) 
\bnum
\num{1} rasaneVMdirxya; AsAvxdanashakitx. 
\num{2} visheVSa ruci; aBiruci; vishiSATxBiruci; iSaTx(vAda) ruci: \eng{have a gust of} iSaTxpaDu; mecucx. \eng{gust of the things of the world} pArxpaMcika vasutxgaLalilx visheVSa aBiruci. 
\num{3} ruci; savi; rasa. 
\enum
\emng
\eentry

\bentry
\word{gustation}
\pron{gaseTxVSanf}
\gl{\nA}
\bmng
\bnum
\num{1} rucinoVDuvudu; rucigarxhaNa; AsAvxdane. 
\num{2} ruci(garxhaNa) shakitx; AsAvxdanashakitx. 
\enum
\emng
\eentry

\bentry
\word{gustative}
\pron{gasaTxTivf}
\gl{\gu}
\bmng
\bnum
\num{1} rucigArxhaka; ruci garxhisuva; rucinoVDuva. 
\num{2} ruci saMbaMdhi; rucinoVDuvudakekx, rucigarxhaNakekx -- saMbaMdhisida. 
\enum
\emng
\eentry

\bentry
\word{gustatory}
\pron{gasaTxTari}
\gl{\gu}
\bmng
 rucisaMbaMdhi; rucigarxhaNakekx yA rasaneVMdirxyakekx saMbaMdhisida: \eng{gustatory nerve} ruci nara; rucigArxhaka nara; nAligeyalilxruva, ruciyanunx garxhisuva nara. 
\emng
\eentry

\bentry
\word{gustily}
\pron{gasiTxli}
\gl{\kirxvi}
\bmng
\bnum
\num{1} (gALi, maLe, birugALi, \mo vu) birusAgi biVsuvaMte; thaTaTxne baruva hAge. 
\num{2} (gALi, maLe, \mo vugaLa) thaTaTxneya hoDetadaMte; birusu biVsina riVtiyalilx. 
\num{3} (shabadx, nagu, \mo vu) thaTaTxne soPxVTisuvaMte; Pakakxne biriyuvaMte. 
\num{4} nirathaRka mAtiniMda kUDidaMte; poLuLx mAtiniMda; soVgumAtinalilx. 
\num{5} hurupiniMda; utAsxhadiMda; KuSiyAgi; ulAlxsadiMda; utAsxhapUNaRvAgi; ulAlxsaBaritavAgi. 
\enum
\emng
\eentry

\bentry
\word{gustiness}
\pron{gasiTxnisf}
\gl{\nA}
\bmng
\bnum
\num{1} (gALi, maLe, \mo vugaLa) birusAda biVsu. 
\num{2} (shabadx, nagu, \mo vugaLu) thaTaTxne soPxVTisuvike. 
\num{3} (mAtina) nirathaRkate; poLuLx. 
\num{4} hurupu; utAsxha; ulAlxsa; KuSi. 
\enum
\emng
\eentry

\bentry
\word{gusto}
\pron{gasoTxV}
\gl{\nA}
\expl{(\bava\ \eng{gustoes}).}
\bmng
\bnum
\num{1} (\pArxparx) rasa; ruci; AsAvxdane; rasAsAvxda: \eng{enjoy the full gusto of} pUNaRruciyanunx, rasavanunx savi. 
\num{2} utAsxha; humamxsusx; umeVdu; KuSi: \eng{he describes the adventure with enormous gusto} avanu sAhasavanunx atuyxtAsxhadiMda vaNiRsutAtxne. 
\num{3} kalAkwshala; kaqtiyanunx racisuva sheYli, riVti. 
\num{4} (veYyakitxka) aBiruci; ruci; olavu. 
\enum
\emng
\eentry

\bentry
\word{gusty}
\pron{gasiTx}
\gl{\gu}
\bmng
\bnum
\num{1} (gALi, maLe, birugALi, \mo vugaLaMte) birusAgi biVsuva; thaTaTxne baruva. 
\num{2} (gALi, maLe, \mo vugaLa) thaTaTxneya hoDetada; birusu biVsina: \eng{a gusty day} birusu gALiya dina. 
\num{3} (shabadx, nagu, \mo vugaLu) thaTaTxne soPxVTisuva; Pakakxne biriyuva. 
\num{4} nirathaRka mAtiniMda kUDida; poLuLxmAtina; soVgumAtiniMda tuMbida: \eng{an evening of gusty speech making} nirathaRka BASaNa mADuvudaralilx kaLeda saMje. 
\num{5} hurupina; ulAlxsada; umeVdina; KuSiyAda; utAsxhaBarita: \eng{a gusty woman} hurupina heMgasu. 
\enum
\emng
\eentry

\bentry
\word[gut(1)]{gut}
\pron{gaTf}
\gl{\nA}
\bmng
\bnum
\num{1} (\bava dalilx) (\kanmu\ pArxNigaLa) karuLu; aMtarx. 
\num{2} (yAvudeV vasutxvina) oLaBAga; aMtaBARga; oLagiruvudu. 
\num{3} karuLu; aMtarx; keLa ananxnALada nidiRSaTx BAga. 
\num{4} (\bava dalilx) (\asaM) (hasivina sAthxnavAda) hoTeTx: \eng{gave the man a poke in the guts} A manuSayxna hoTeTx tivide. 
\num{5} (\bava dalilx) (\AmA) kecucx; dheYyaR; shiVlashakitx; cAritarxyXbala; satatxvX; dhaqti; sahana shakitx; tAkatutx; dADhayxR; damumx: \eng{he alone has the guts to grapple with the enemy} shaturxvinoDane malAlxmalilxyAgi hoVrADuva kecucx avanige mAtarx ide. 
\num{6} (pArxNigaLa karuLiniMda mADida, piTiVlu taMti, bAyxTina huri yA shasatxrXcikitesxyalilx holiyuva hurigAgi baLasuva) karuLu -- huri, nara, taMti, taMtu. 
\num{7} (reVSemxhuLugaLa karuLiniMda mADida) mInugALada huri. 
\num{8} jalasaMdhi; jalakaMTha; samudarx, saroVvara, \mo\ eraDu jalarAshigaLanunx kUDisuva ikakxTATxda jalamAgaR. 
\num{9} (AkfsxphaDfR matutx keVMbirxjfgaLalilx) doVNipaMdayx mAgaRdalilx nadigaLa tiruvu. 
\num{10} kaNive; ikakxTuTx hAdi. 
\num{11} (biVdiya) ONi; kirusaMdi; cikakx galilx. 
\enum
\emng

\noindent
\gl{\nuga}
\bmng
\bnum
\num{1} \eng{has no guts in it} adaralilx nijavAda tiruLu yA satatxvX ilalx; adaralilx nijavAgi belebALatakakxdudx yAvudU ilalx. 
\num{2} \eng{hate person's guts} (\AmA) (obabxnanunx) tiVvarxvAgi devxVSisu. 
\num{3} \eng{sweat} (\engit{or} \eng{work) one's guts out} (\AmA) bahaLa sharxmisu; atiyAgi duDi, kelasamADu; bahaLa bevaru surisu. 
\enum
\emng
\eentry

\bentry
\word[gut(2)]{gut}
\pron{gaTf}
\gl{\gu}
\bmng
\bnum
\num{1} mUlaBUtavAda: \eng{a gut issue} mUlaBUtavAda viSaya. 
\num{2} neYsagiRkavAda; sAvxBAvikavAda; parxkaqtisahajavAda: \eng{a gut reaction} parxkaqtisahajavAda yA sAvxBAvikavAda parxtikirxye. 
\enum
\emng
\eentry

\bentry
\word[gut(3)]{gut}
\pron{gaTf}
\gl{\sakirx}
\expl{(\BU\ matutx \BUkaq\ \eng{gutted,} \vakaq\ \eng{gutting}).}
\bmng
\bnum
\num{1} (mInina) karuLu tege. 
\num{2} (\kanmu\ beMkiyiMda) (mane \mo vugaLa) oLajoVDaNegaLanunx kitutxhAku, tegeduhAku, nAshamADu: \eng{fire gutted the building} beMkiyu kaTaTxDada oLaBAgavanenxlAlx nAshamADitu. 
\num{3} (garxMtha \mo vugaLa) sAra garxhisu; tiruLu -- hiDi, tege: \eng{I am pleased when I see my books well gutted} nananx garxMthagaLa sAravatAtxda BAgagaLa Ayekxyanunx noVDidAga nanage saMtoVSavAguvudu. 
\enum
\emng
\eentry

\bentry
\word{gutless}
\pron{gaTflisf}
\gl{\gu}
\bmng
 (\AmA) 
\bnum
\num{1} damimxlalxda; dheYyaRvilalxda; shakitxyilalxda. 
\num{2} daqDhasaMkalapxvilalxda. 
\enum
\emng
\eentry

\bentry
\word{got-rot}
\pron{gaTfrATf}
\gl{\nA}
\bmng
 = \hyperref{kandict_r.pdf}{R}{rot-gut}{rot-gut}. 
\emng
\eentry

\bentry
\word{gutser}
\pron{gaTasxrf}
\gl{\nA}
\bmng
 (\AseTxrXV\ matutx nUyxsiZVlaMDf) (\AmA) joVrAgi, saKatAtxgi -- biVLuvudu; BAri patana. 
\emng
\eentry

\bentry
\word{gutsily}
\pron{gaTisxli}
\gl{\kirxvi}
\bmng
\bnum
\num{1} hoTeTxbAkatanadiMda; durAsheyiMda. 
\num{2} kecicxniMda; dheYyaRdiMda. 
\enum
\emng
\eentry

\bentry
\word{gutsiness}
\pron{gaTisxnisf}
\gl{\nA}
\bmng
\bnum
\num{1} hoTeTxbAkatana; durAshe. 
\num{2} dheYyaR; kecucx. 
\enum
\emng
\eentry

\bentry
\word{gutsy}
\pron{gaTisx}
\gl{\gu}
\bmng
 (\AmA) 
\bnum
\num{1} durAsheya; hoTeTxbAkatanada. 
\num{2} dheYyaRda; kecicxna. 
\enum
\emng
\eentry

\bentry
\word{guttae}
\pron{gaTiV}
\gl{\nA}
\bmng
 (\bava) loVlakamAle; sAluloVlaka; loVlakapaMkitx; \kanmu\ DoVrikf vAsutxshilapxdalilx, alaMkArada loVlakagaLa sAlu. 
\emng
\eentry

\bentry
\word{gutta-percha}
\pron{gaTapacaR(kaR)}
\gl{\nA}
\bmng
 pacaR vaqkaSxda goVMdu; gaTapacaR; malaya divxVpagaLalilxya halavu bageya maragaLiMda rasavAgi sarxvisuva, bUdubaNaNxda aMTu. 
\emng
\eentry

\bentry
\word{guttate}
\pron{gaTeVTf}
\gl{\gu}
\bmng
 (\jiVvi) macecxmacecxyAgiruva; cukekxcukekxyAgiruva; boTuTx boTATxgiruva. 
\emng
\eentry

\bentry
\word[gutter(1)]{gutter}
\pron{gaTarf}
\gl{\nA}
\bmng
\bnum
\num{1} (\viparx) toVDu; niVru dAri; niVrina jADu; hariyuva niVriniMdAda jADu. 
\num{2} sUrubAne; sUru doVNi; maLeya niVru haridu hoVguvaMte sUrina keLagiTiTxruva marige. 
\num{3} caraMDi; moVri; gaTAra; rasetxya badiyalilxruva kAluve. 
\num{4} kAluve; tUbu; darxva hariduhoVguvudakAkxgi mADiruva tereda nAle. 
\num{5} gADi; caDi; toVDu. 
\num{6} (pusatxkagaLalilxna) eraDu eduru baduru puTagaLa naDuve iruva mAjiRnfgaLa jAga. 
\hypertarget{gutter(1)7}{} 
\num{7} halAkx -- koMpe, guMDi; caraMDi; gaTAra; koLace, aneYtikate, aparAdha \mo\ ayoVgayx badukina yA naDateya sithxti yA AvAsa: \eng{the language of the gutter} gaTArada BASe; halAkxmAtu. 
\enum
\emng

\noindent
\gl{\nuga}
\bmng
\bnum
\num{1} \eng{take child etc., out of gutter} magu \mo vanunx baDatanadiMda biDisu, meVletutx, pArumADu; dAridarxyXda parisaradiMda magu \mo varanunx biDisu. 
\num{2} \eng{the gutter} = \hyperlink{gutter(1)7}{$^1$gutter (7)}. 
\enum
\emng
\eentry

\bentry
\word[gutter(2)]{gutter}
\pron{gaTarf}
\gl{\sakirx}
\bmng
\bnum
\num{1} (niVrina hariviniMda) kAluve toVDu; caraMDi tege; moVri mADu: \eng{a heavy rain guttering the field} holadalilx kAluve mADuvaMtha BAri maLe. 
\num{2} caraMDi, moVri -- odagisu. 
\num{3} (kaNiNxVru \mo vugaLa \vi) gere (gurutu) mADu. 
\enum
\emng

\noindent
\gl{\akirx}
\bmng
\bnum
\num{1} koVDi hari; kAluveyAgi hari. 
\num{2} (moVMbatitxya \vi) (oMdu kaDe kAluve bidudx, meVNa \mo vu haridu) karagihoVgu: \eng{the candles flickered and guttered down} moVMbatitxgaLu miNuku miNukutAtx uridu beVga karagihoVdavu. 
\enum
\emng
\eentry

\bentry
\word{gutter-child}
\pron{gaTarfceYlfDx}
\gl{\nA}
\bmng
 moVrimagu; gaTArada shishu; caraMDikUsu: 
\banum
\alnum{a} BikAri magu; dikikxlalxda magu; caraMDigaLalelxV suLidADuva magu. 
\alnum{b} kiVLu huTiTxna yA kiVLu shikaSxNada magu. 
\eanum
\emng
\eentry

\bentry
\word{guttering}
\pron{gaTariMgf}
\gl{\gu}
\bmng
\bnum
\num{1} moVri mADuvike; haLaLxtoVDi kAluve mADuvike. 
\num{2} moVri; caraMDi. 
\num{3} gADi toVDuvike; gere mADuvike. 
\num{4} moVMbatitxyiMda karagi bidadx meVNa. 
\enum
\emng
\eentry

\bentry
\wordnospeech{gutter journalism}{gutter journalism}
\pron{?}
\gl{\nA}
\bmng
 (kiVLu aBiruciya) gaTAra patirxkoVdayxma; kocecx, rocucx -- patirxkoVdayxma. 
\emng
\eentry

\bentry
\wordnospeech{gutter press}{gutter press}
\pron{?}
\gl{\nA}
\bmng
 (kiVLu yA tucaCx aBiruciyanunx taqpitxpaDisuva) gaTAra patirxkegaLu; rocucx, kocecx -- patirxkegaLu. 
\emng
\eentry

\bentry
\word{guttersnipe}
\pron{gaTarfsenxYpf}
\gl{\nA}
\bmng
 biVdimagu; BikArimagu; dikikxlalxde, manemaTha ilalxde biVdigaLalilx aledukoMDiruva huDuga. 
\emng
\eentry

\bentry
\word{guttle}
\pron{gaTflf}
\gl{\sakirx}
\bmng
 mukukx; hoTeTxbAkanaMte tinunx (\akirx\ saha). 
\emng
\eentry

\bentry
\word{guttler}
\pron{gaTalxrf}
\gl{\nA}
\bmng
 hoTeTxbAka; tiVnALi; tiMDi poVta. 
\emng
\eentry

\bentry
\word[guttural(1)]{guttural}
\pron{gaTaralf}
\gl{\gu}
\bmng
\bnum
\num{1} gaMTalina. 
\num{2} (dhavxniya \vi) kaMThayx; gaMTalinalilx yA nAlageya hiMBAga matutx tAluviniMda huTuTxva. 
\enum
\emng
\eentry

\bentry
\word[guttural(2)]{guttural}
\pron{gaTaralf}
\gl{\nA}
\bmng
 kaMThayx(vaNaR); kaMThayx dhavxni; gaMTalinalilx yA nAligeya hiMBAga matutx tAluviniMda huTuTxva vaNaR yA dhavxni, \udA\ : ka, ga. 
\emng
\eentry

\bentry
\word{gutturalism}
\pron{gaTaralisaZmf}
\gl{\nA}
\bmng
 (dhavxniya \vi) kaMThayxte; kaMThayx guNa, lakaSxNa. 
\emng
\eentry

\bentry
\word{gutturalize}
\pron{gaTaraleYsfZ}
\gl{\sakirx}
\bmng
\bnum
\num{1} kaMThayxvAgi ucacxrisu. 
\num{2} (dhavxniyanunx) kaMThayxvAgisu. 
\enum
\emng
\eentry

\bentry
\word{gutturally}
\pron{gaTarali}
\gl{\kirxvi}
\bmng
 kaMThayxvAgi. 
\emng
\eentry

\bentry
\word{gutturo-}
\pron{gaTaroV-}
\gl{\sapUpa}
\bmng
 kaMTha matutx --, gaMTalu hAgU... eMba athaRda padagaLa racaneyalilx baLasuva \sapUpa: \eng{gutturo-maxillary} gaMTalu matutx davaDegaLige saMbaMdhisida. 
\emng
\eentry

\bentry
\word{guv}
\pron{gavf}
\gl{\nA}
\bmng
 (\birx) (\asaM\ yA \AmA)  = \hyperlink{governor(h)}{governor (h \& i)}. 
\emng
\eentry

\bentry
\word{guv'nor}
\pron{gavfnarf}
\gl{\nA}
\bmng
 (\birx) (\asaM\ yA \AmA)  = \hyperlink{governor(h)}{governor (h \& i)}. 
\emng
\eentry

\bentry
\word[guy(1)]{guy}
\pron{geY}
\gl{\nA}
\bmng
 geY; bigi -- hagagx, sarapaNi, \mo vu; etutxyaMtarxda hore \mo vanunx OladaMte sitxmitapaDisuva, yA DeVre, kaMba, \mo vanunx adara sathxLadalilx BadarxvAgi kaTiTx nililxsuva hagagx, sarapaNi, \mo vu. 
\emng
\eentry

\bentry
\word[guy(2)]{guy}
\pron{geY}
\gl{\sakirx}
\bmng
 hagagx, sarapaNi, \mo vugaLiMda kaTuTx, bigi, BadarxpaDisu. 
\emng
\eentry

\bentry
\word[guy(3)]{guy}
\pron{geY}
\gl{\nA}
\bmng
\bnum
\num{1} (\birx) (iMgelxMDinalilx parxti naveMbarf \eng{5}neya tAriVKu suDuva) geYphAkfsx eMbuvana Akaqti, boMbe. 
\num{2} (\birx) vikAraveVSi; vikAravAda veVSa dharisida manuSayx. 
\num{3} (\ashi) ALu; AsAmi; girAki. 
\num{4} (\birx) (\ashi) (tale tapipxsikoMDu) ODi hoVguvudu; parAriyAguvudu; kaMbi kiVLuvudu. 
\enum
\emng

\noindent
\gl{\nuga}
\bmng
\bnum
\num{1} \eng{do a guy} kaNamxreyAgu; mAyavAgu. 
\num{2} \eng{give the guy to} parAriyAgu; kaMbikiVLu. 
\enum
\emng
\eentry

\bentry
\word[guy(4)]{guy}
\pron{geY}
\gl{\sakirx}
\bmng
\bnum
\num{1} boMbe mADi parxdashiRsu; Akaqti mADi toVrisu; parxtikaqtiyAgi mADi mereyisu. 
\num{2} kucoVdayx, geVli -- mADu. 
\enum
\emng

\noindent
\gl{\akirx}
\bmng
 (\ashi) ODihoVgu; parAriyAgu. 
\emng
\eentry

\bentry
\word{guzzle}
\pron{gasfZlf}
\gl{\sakirx}
\bmng
\bnum
\num{1} gabagabane, bakabakane tinunx yA gaTagaTane kuDi (\akirx\ saha). 
\num{2} (haNa \mo vanunx) nuMgi niVrukuDi; tinunxvudaralilx, kuDitadalilx -- niVgibiDu. 
\enum
\emng
\eentry

\bentry
\word{guzzler}
\pron{gasfZlarf}
\gl{\nA}
\bmng
\bnum
\num{1} tiMdu, kuDidu hALu mADuvavanu. 
\num{2} (haNa \mo vanunx) nuMgi niVru kuDiyuvavanu. 
\enum
\emng
\eentry

\bentry
\word{gwyniad}
\pron{givxniaDf}
\gl{\nA}
\bmng
 givxniyaDf; iMgelxMDina veVlfsx parxdeVshada utatxra BAgadalilxna bAlA saroVvaradalilx sikukxva, sAmanf baLagada, biLiya mAMsada oMdu bageya mInu. 
\emng
\eentry

\bentry
\wordnospeech{Gy}{Gy}
\pron{?}
\gl{\saMkiSx}
\bmng
  = \hyperlink{gray(1)}{$^1$gray}. 
\emng
\eentry

\bentry
\word[gybe(1)]{gybe}
\pron{jeYbf}
\gl{\sakirx}
\bmng
 (haDagu gALiya dikikxnalilx ODutitxruvAga yA gALi haDaganunx raBasavAgi taLiLxkoMDu hoVgutitxruvAga, hAyiyanunx yA hAyidimimxyanunx) haDagina oMdu pakakxdiMda inonxMdu pakakxkekx -- tUgu, tUgi hoVguvaMte mADu, tUgi badalAyisu. 
\emng

\noindent
\gl{\akirx}
\bmng
\bnum
\num{1} (haDaganunx gALi raBasadiMda taLiLxkoMDu hoVgutitxruvAga, hAyi yA hAyidimimx) haDagina oMdu pakakxdiMda inonxMdu pakakxkekx -- tUgi hoVgu, tUgi badalAyisu. 
\num{2} (haDagu, nAvikaru, \mo vugaLa \vi) gALiya dikikxnalilx haDagu ODutitxruvAga, hAyi yA hAyidimimx haDagina oMdu pakakxdiMda inonxMdu pakakxkekx tUgi hoVguvaMte haDagina -- patha badalAyisu, dikukx tirugisu. 
\enum
\emng
\eentry

\bentry
\word[gybe(2)]{gybe}
\pron{jeYbf}
\gl{\nA}
\bmng
 (haDaganunx gALi raBasadiMda oyuyxtitxruvAga adara hAyiyanunx yA hAyidimimxyanunx) haDagina oMdu pakakxdiMda inonxMdu pakakxkekx -- tUgisuvudu, tUgi(koMDu) hoVguvaMte mADuvudu, tUgi badalAyisuvudu. 
\emng
\eentry

\bentry
\word{gyle}
\pron{geYlf}
\gl{\nA}
\bmng
 geYlf: 
\banum
\alnum{a} oMdu sala hudugu hAki tayArisida biyarf parxmANa. 
\alnum{b} biyarf tayArisalu hudugu hAkiruva mAlfTx misharxNa. 
\alnum{c} biyarfna hudugu toTiTx. 
\eanum
\emng
\eentry

\bentry
\word{gym}
\pron{jimf}
\gl{\nA}
\bmng
 (\AmA) \eng{gymnasium, gymnastics} padagaLa \saMkiSx. 
\emng
\eentry

\bentry
\word{gymkhana}
\pron{jiMkAna}
\gl{\nA}
\bmng
 (mUlataH \AMiM) 
\bnum
\num{1} vAyxyAmaraMga; ATada meYdAna; aMgasAdhanege anukUlagaLanunx odagisiruva sAvaRjanika sathxLa. 
\num{2} vAyxyAma (kirxVDA) parxdashaRna. 
\num{3} jiMkAna; kuduregaLu, kudure savAraru, vAhanagaLu, \mo vugaLu BAgavahisuva sapxdheR. 
\enum
\emng
\eentry

\bentry
\word{gymnasial}
\pron{jimenxVsialf}
\gl{\gu}
\bmng
 ucacxshikaSxNa tarabeVti shAleya; (vishavxvidAyxnilayagaLige vidAyxthiRgaLanunx tarabeVtu mADuva) meVlu dajeRya -- vidAyxshAleya yA vidAyxshAlege saMbaMdhisida. 
\emng
\eentry

\bentry
\word{gymnasium}
\pron{jimenxVsiZamf, jimenxVsayxZmf}
\gl{\nA}
\expl{(\bava\ \eng{gymnasiums, gymnasia}).}
\bmng
\bnum
\num{1} vAyxyAmashAle; garaDi(mane); aMgasAdhane gaqha; tAliVmuKAne. 
\num{2} (\ucAcx\ jimAnxsiZumf saha.) ucacxshikaSxNa tarabeVti shAle; (yUroVpinalilx, \kanmu\ jamaRniyalilx) vishavxvidAyxnilayagaLige vidAyxthiRgaLanunx tarabeVtu mADuva meVludajeRya vidAyxshAle. 
\enum
\emng
\eentry

\bentry
\word{gymnast}
\pron{jimAnxYxsfTx}
\gl{\nA}
\bmng
 aMgasAdhaka; vAyxyAmapaTu; garaDisAdhaka; aMgasAdhaneyalilx parxviVNa. 
\emng
\eentry

\bentry
\word[gymnastic(1)]{gymnastic}
\pron{jimAnxYxsiTxkf}
\gl{\gu}
\bmng
\bnum
\num{1} (deYhika shikaSxZNa, caTuvaTike, sharxma, shisutx, \mo vugaLanonxLagoMDa) aMgasAdhaneya; vAyxyAmada; kasaratitxna. 
\num{2} (\rUpa) budidhxsAdhaneya; budidhx kasaratitxna; mAnasika shikaSxNa yA tarabeVtiyiMda kUDida. 
\enum
\emng
\eentry

\bentry
\word[gymnastic(2)]{gymnastic}
\pron{jimAnxYxsiTxkf}
\gl{\nA}
\bmng
\bnum
\num{1} (tarabeVtiya yA shisitxna karxmaveMdu parigaNisuva) shikaSxNa karxma; sAdhane; vAyxyAma; kasaratutx: \eng{grammar is a good gymnastic} vAyxkaraNavu budidhxge oLeLxya kasaratutx, vAyxyAma. 
\num{2} aMgasAdhaneya (oMdu) sAhasa. 
\num{3} aMgasAdhane(yeMba oMdu shikaSxNa karxma). 
\enum
\emng
\eentry

\bentry
\word{gymnastically}
\pron{jimAnxYxsiTxkali}
\gl{\kirxvi}
\bmng
 (budidhx yA aMgakekx) sAdhaneyAgi; kasaratitxnaMte 
\emng
\eentry

\bentry
\word{gymnastics}
\pron{jimAnxYxsiTxkfsx}
\gl{\nA}
\bmng
 (\bava) (kelavomemx \Eva vAgi \parx) (mAMsaKaMDagaLanunx balisuva, \kanmu\ garaDi maneyalilx mADuvaMtha) aMgasAdhanegaLu; kasaratutxgaLu; vAyxyAmagaLu. 
\emng
\eentry

\bentry
\word{gymno-}
\pron{jimanx-}
\gl{\sapUpa}
\bmng
 (\jiVvi) naganx, hodikeyilalxda, anAcACxdita, anAvaqta eMba athaRdalilx baLasuva \sapUpa. 
\emng
\eentry

\bentry
\word{gymnosophist}
\pron{jimAnxsaphiZsfTx}
\gl{\nA}
\bmng
\bnum
\num{1} (hiMdU) batatxle saMnAyxsi; naganx saMnAyxsi; digaMbara; atayxlapx hodikeyiMdaloV hodikeyilalxdeyoV saMcarisutatx dhAyxnAsakatxnAgirutitxdadx, hiMdina kAlada, oMdu hiMdU sAdhu paMgaDadavanu. 
\num{2} virakatx; saMnAyxsi. 
\num{3} anuBAvi. 
\enum
\emng
\eentry

\bentry
\word{gymnosophy}
\pron{jimAnxsaphiZ}
\gl{\nA}
\bmng
 batatxle saMnAyxsi paMtha; naganxsaMnAyxsi sidAdhxMta. 
\emng
\eentry

\bentry
\word{gymnosperm}
\pron{jimanxsapxmfR}
\gl{\nA}
\bmng
 (\savi) jimonxVsapxmfR; anAvaqta biVji; parxkaTabiVji; naganxbiVji; biVjakoVshadoLagaDe irade horage parxkaTavAgiruva biVjagaLanunx utapxtitx mADuva vagaRda sasayx. 
\emng
\eentry

\bentry
\word{gymnospermous}
\pron{jimanxsapxmaRsf}
\gl{\gu}
\bmng
 naganx biVjiya; jimonxVsapxmfRge saMbaMdhisida; anAvaqta biVjagaLuLaLx. 
\emng
\eentry

\bentry
\word{gymp}
\pron{giMpf}
\gl{\nA}
\bmng
  = \hyperlink{gimp(1)}{$^1$gimp}. 
\emng
\eentry

\bentry
\word{gym-slip}
\pron{jimfsilxpf}
\gl{\nA}
\bmng
  = \hyperlink{gymtunic}{gymtunic}. 
\emng
\eentry

\bentry
\word{gymtunic}
\pron{jimfTUyxnikf}
\gl{\nA}
\bmng
 jimf uDige; shAleya huDugiyaru hAkikoLuLxva, toVLilalxda, maMDiyavarege baruva, naDupaTiTxyuLaLx uDige. 
\emng
\eentry

\bentry
\word{gyn-}
\pron{ji(gi)nf-}
\gl{\sapUpa}
\bmng
 heMgasu, heNuNx, sitxrXV eMbathaRda \sapUpa. 
\emng
\eentry

\bentry
\word{gynaeceum}
\pron{jeY(geY)ni(V)siVamf}
\gl{\nA}
\bmng
\bnum
\num{1} (girxVkf matutx roVmf \pArxca) aMtaHpura; janAna; (maneyalilx) heMgasara koVNegaLu. 
\num{2} (\savi)  = \hyperlink{gynoecium}{gynoecium}. 
\enum
\emng
\eentry

\bentry
\word{gynaeco-}
\pron{geYnakoV-}
\gl{\sapUpa}
\bmng
 heMgasina, heMgasara, sitxrXVyara eMbathaRgaLalilx baLasuva \sapUpa. 
\emng
\eentry

\bentry
\word{gynaecocracy}
\pron{geYnikAkarxsi}
\gl{\nA}
\bmng
 heNANxLivxke; sitxrXVparxButavx; heMgasara rAjayxBAra. 
\emng
\eentry

\bentry
\word{gynaecologic}
\pron{geYnikalAjikf}
\gl{\gu}
\bmng
 (\ame)  = \hyperlink{gynaecological}{gynaecological}. 
\emng
\eentry

\bentry
\word{gynaecological}
\pron{geYnikalAjikalf}
\gl{\gu}
\bmng
 sitxrXVroVgashAsitxrXVya; sitxrXVroVgashAsatxrXda. 
\emng
\eentry

\bentry
\word{gynaecologist}
\pron{geYnikAlajisfTx}
\gl{\nA}
\bmng
 sitxrXVroVgashAsatxrXjacnx; sitxrXVroVgatajacnx(jecnx); sitxrXVroVgashAsatxrXdalilx tajacnx(Lu). 
\emng
\eentry

\bentry
\word{gynaecology}
\pron{geYnikAlaji}
\gl{\nA}
\bmng
 sitxrXVroVgashAsatxrX; heMgasaru, avarige vishiSaTxvAda roVgagaLu, \mo vanunx kurita veYdayxshAsatxrX viBAga. 
\emng
\eentry

\bentry
\word{gynaecomastia}
\pron{geYnikoVmAYxsiTxa}
\gl{\nA}
\bmng
 sitxrXVkucatavx; gaMDasina satxnagaLu heMgasina molegaLaMte doDaDxdAguva sithxti. 
\emng
\eentry

\bentry
\word{gynandromorph}
\pron{geYnAYxMDarxmAphfR}
\gl{\nA}
\bmng
 (\jiVvi) uBayaliMgi; divxliMgi; sitxrXV matutx puruSa lakaSxNagaLuLaLxdudx (\kanmu\ kiVTa). 
\emng
\eentry

\bentry
\word{gynandromorphic}
\pron{geYnAYxMDarxmAphiRkf}
\gl{\gu}
\bmng
 (\kanmu\ kiVTada \vi) divxliMgada; uBayaliMgada; deVhadalilx sitxrXV matutx puruSa lakaSxNagaLuLaLx. 
\emng
\eentry

\bentry
\word{gynandromorphism}
\pron{geYnAYxMDarxmAphiRsaZmf}
\gl{\nA}
\bmng
 (\kanmu\ kiVTada \vi) uBayaliMgitavx; divxliMga(lakaSxNa)te; deVhadalilx sitxrXV matutx puruSa lakaSxNagaLiruvike. 
\emng
\eentry

\bentry
\word{gynandrous}
\pron{geYnAYxMDarxsf}
\gl{\gu}
\bmng
 (\savi) uBaya keVsari; jarigiDagaLalilx, AkiRDuDxgaLalilx iruvaMte, shalAkeyU puMkeVsaragaLU oMdAgi kUDikoMDiruva. 
\emng
\eentry

\bentry
\word{gynecium}
\pron{jeY(geY)ni(V)siVamf}
\gl{\nA}
\bmng
  = \hyperlink{gynaeceum}{gynaeceum}. 
\emng
\eentry

\bentry
\word{gyneco-}
\pron{geYnakoV-}
\gl{\sapUpa}
\bmng
 (\ame) \eng{= gynaeco.} 
\emng
\eentry

\bentry
\word{gynecocracy}
\pron{geYnikAkarxsi}
\gl{\nA}
\bmng
 (\ame)  = \hyperlink{gynaecocracy}{gynaecocracy}. 
\emng
\eentry

\bentry
\word{gunecology}
\pron{geYnikAlaji}
\gl{\nA}
\bmng
 (\ame)  = \hyperlink{gynaecology}{gynaecology}. 
\emng
\eentry

\bentry
\word{gynoecium}
\pron{geYniVsiamf}
\gl{\nA}
\expl{(\bava\ \eng{gynoecia} \ucAcx= geYniVsia).}
\bmng
(\savi) geYniVsiyamf; jAyAMga; sitxrXVkeVsaragaLa samUha. 
\emng
\eentry

\bentry
\word{gynophore}
\pron{jeY(geY)naphoVrf, jinaphoVrf}
\gl{\nA}
\bmng
\bnum
\num{1} (\savi) aMDAshaya vaqMta; aMDAshayada toTuTx. 
\num{2} (\pArxvi) aMkuradhAri; aMbali mInina baLagada kelavu jalacara pArxNigaLa deVhadalilxna mogugxLaLx shAKe. 
\enum
\emng
\eentry

\bentry
\wordwithhyphen{hyp-gynous}{-gynous}
\pron{-gi(ji)nasf}
\gl{\saupa}
\bmng
 (\savi) shalAkegaLiruva eMbathaRdalilx baLasuva \saupa: \eng{monogynous, androgynous.} 
\emng
\eentry

\bentry
\word[gyp(1)]{gyp}
\pron{jipf}
\gl{\nA}
\bmng
 (\birx) (keVMbirxjf matutx DahaRmf kAleVjugaLalilxna) kAleVjina javAna. 
\emng
\eentry

\bentry
\word[gyp(2)]{gyp}
\pron{jipf}
\gl{\nA}
\bmng
(\AmA) piVDe; upadarxva; kATa. 
\emng

\noindent
\gl{\pagu}
\bmng
 \eng{give one gyp} upadarxva koDu; kATakoDu; noVvuMTumADu yA beYyu. 
\emng
\eentry

\bentry
\word[gyp(3)]{gyp}
\pron{jipf}
\gl{\sakirx}
\expl{(\BU\ matutx \BUkaq\ \eng{gypped,} \vakaq\ \eng{gypping}). }
\bmng
moVsa mADu; dagA hAku; vaMcisu. 
\emng
\eentry

\bentry
\word[gyp(4)]{gyp}
\pron{jipf}
\gl{\nA}
\bmng
 moVsa; dagA; vaMcane. 
\emng
\eentry

\bentry
\word{gyppy}
\pron{jipi}
\gl{\nA}
\bmng
  = \hyperlink{gippy}{gippy}. 
\emng
\eentry

\bentry
\word{gypseous}
\pron{jipisxasf}
\gl{\gu}
\bmng
\bnum
\num{1} jipasxmfnaMtha. 
\num{2} jipasxmf uLaLx. 
\enum
\emng
\eentry

\bentry
\word{gypsiferous}
\pron{jipisxpharasf}
\gl{\gu}
\bmng
 jipasxmfdhAri; jipasxmamxnunx oLagoMDa. 
\emng
\eentry

\bentry
\word{gypsography}
\pron{jipAsxgarxphi}
\gl{\nA}
\bmng
 jipasxM ketatxne; jipasxmimxna meVle ketatxne mADuva kale. 
\emng
\eentry

\bentry
\word{gypsophila}
\pron{jipAsxphila}
\gl{\nA}
\bmng
 jipAsxphila; cikakxcikakx sUkaSxmXvAda saMkiVNaR biLi hUgaLanunx biDuva oMdu sasayxkula. 
\emng
\eentry

\bentry
\word{gypsous}
\pron{jipasxsf}
\gl{\gu}
\bmng
  = \hyperlink{gypseous}{gypseous}. 
\emng
\eentry

\bentry
\word{gypsum}
\pron{jipasxmf}
\gl{\nA}
\bmng
 jipasxmf; pAlxsaTxrf Aphf pAYxrisf tayArisalu baLasuva, jaliVkaqta kAYxlisxyamf salePxVTf Kanija, \eng{$\bg\rm CaSo\eg_4. 2\bg\rm H\eg_2\bg\rm O\eg$}. 
\emng
\eentry

\bentry
\word{gypsy}
\pron{jipisx}
\gl{\nA}
\bmng
\hyperdef{G}{gypsy(1)}{} 
\bnum
\num{1} jipisx; korava; `roVmani' eMdu karedukoLuLxva, BAratada neleyavarAgidadx, ashudadhx hiMdiV BASeyanAnxDuva, buTiTx heNige, kudure vAyxpAra, kaNi heVLuvudu, \mo\ jiVvanoVpAyagaLuLaLx, oMdeDe nilalxde aleyutitxruva, kapupx oDalU kUdalU uLaLx, oMdu janAMgadavanu. 
\num{2} jipisx; jipisxgaLanunx hoVluvavanu yA avaraMte jiVvana naDesuvavanu. 
\num{3} (tamASeyAgi) taMTekoVri heMgasu yA shAyxmala vaNaRda heMgasu. 
\enum
\emng
\eentry

\bentry
\wordnospeech{gypsy bonnet}{gypsy bonnet}
\pron{?}
\gl{\nA}
\bmng
 jipisx ToVpi; muduru pakakxpaTiTxgaLuLaLx oMdu doDaDx ToVpi. 
\emng
\eentry

\bentry
\word{gypsydom}
\pron{jipisxDamf}
\gl{\nA}
\bmng
 (sAmUhikavAgi) jipisx jagatutx; jipisx loVka; jipisxgaLu. 
\emng
\eentry

\bentry
\word{gypsyfy}
\pron{jipisxpheY}
\gl{\sakirx}
\bmng
 jipisxVkarisu; lakaSxNadalilx yA savxBAvadalilx jipisxyaMte mADu. 
\emng
\eentry

\bentry
\word{gypsyhood}
\pron{jipisxhuDf}
\gl{\nA}
\bmng
 jipisxtana; jipisxsithxti; jipisxyAgiruvike. 
\emng
\eentry

\bentry
\word{gypsyish}
\pron{jipisxiSf}
\gl{\gu}
\bmng
 jipisxyaMtaha; jipisx sadaqsha; jipisxyanunx hoVluva. 
\emng
\eentry

\bentry
\word{gypsyism}
\pron{jipisxisaZmf}
\gl{\nA}
\bmng
 jipisxdhamaR; jipisxgaLa jiVvanakarxma matutx vaqtitx yA udoyxVga. 
\emng
\eentry

\bentry
\wordnospeech{gypsy moth}{gypsy moth}
\pron{?}
\gl{\nA}
\bmng
 jipisx pataMga; elegaLige tuMba hAnikArakavAda, goMcalu kUdaliruva oMdu bageya pataMga. 
\emng
\eentry

\bentry
\wordnospeech{gypsy rose}{gypsy rose}
\pron{?}
\gl{\nA}
\bmng
 jipisxgulAbi; sekxVbiyasf kulakekx seVrida oMdu hUgiDa. 
\emng
\eentry

\bentry
\wordnospeech{gypsy's warning}{gypsy's warning}
\pron{?}
\gl{\nA}
\bmng
 jipisx ecacxrike; nigUDhavAda yA aniSaTxda ecacxrike. 
\emng
\eentry

\bentry
\wordnospeech{gypsy table}{gypsy table}
\pron{?}
\gl{\nA}
\bmng
 jipisxmeVju; mugAgxlu meVju; oMdara meVloMdu hAyidxruva mUru kAlugaLa meVle niMtiruva, haguravAda duMDumeVju. 
\emng
\eentry

\bentry
\word[gyrate(1)]{gyrate}
\pron{jeYaraTf}
\gl{\gu}
\bmng
 (\savi) valayita; uMguruMguragaLAgi, sututxsututxgaLAgi vayxvasethxgoMDiruva. 
\emng
\eentry

\bentry
\word[gyrate(2)]{gyrate}
\pron{jeYareVTf}
\gl{\akirx}
\bmng
 pariBarxmisu; oMdu biMduvina sutatx vaqtAtxkAravAgi yA akaSxda sutatx suruLiyAkAravAgi calisu. 
\emng
\eentry

\bentry
\word{gyration}
\pron{jeYareVSanf}
\gl{\nA}
\bmng
 pariBarxmaNe; oMdu biMduvina sutatx vaqtAtxkAravAgi yA akaSxda sutatx suruLiyAkAravAgi calisuvudu. 
\emng
\eentry

\bentry
\word{gyratory}
\pron{jeYaraTari}
\gl{\gu}
\bmng
 pariBArxmaka; pariBarxmisuva. 
\emng
\eentry

\bentry
\word[gyre(1)]{gyre}
\pron{jeYarf}
\gl{\akirx}
\bmng
 (\kanmu\ \kAparx)  = \hyperlink{gyrate(2)}{$^2$gyrate}. 
\emng
\eentry

\bentry
\word[gyre(2)]{gyre}
\pron{jeYarf}
\gl{\nA}
\bmng
 (\kanmu\ \kAparx)  = \hyperlink{gyration}{gyration}. 
\emng
\eentry

\bentry
\word{gyrfalcon}
\pron{japhARlakxnf}
\gl{\nA}
\bmng
utatxravalayada, phAlokxrasiTxkalasf kulada, oMdu doDaDx giDuga, DeVge. \imglink{gyrfalconfigure}{\raisebox{-0.15cm}[0pt][0pt]{\pdfimage width 0.6cm height 0.7cm {G_Pictures/gyrfalcon.jpg}}} 
\emng
\eentry

\bentry
\word{gyro}
\pron{jeYaroV}
\gl{\nA}
\expl{(\bava\ \eng{gyros}).}
\bmng
(\AmA) 
\bnum
\num{1}  = \hyperlink{gyroscope}{gyroscope}. 
\num{2}  = \hyperlink{gyrocompass}{gyrocompass}. 
\enum
\emng
\eentry

\bentry
\word{gyro-}
\pron{jeYaroV-}
\gl{\sapUpa}
\bmng
 pariBarxmaNa eMbathaRdalilx baLasuva \sapUpa. 
\emng
\eentry

\bentry
\word{gyrocompass}
\pron{jeYaroVkaMpasf}
\gl{\nA}
\bmng
 jeYrokaMpAsu; jeYroVdikUsxci; BArxmaka dikUsxci; (eDebiDade) sututx(titxru)va jeYroVsokxVpu; oMdara BarxmaNacakarxda akaSxvu sadA BUmiya meVlemxYge samAMtaravAgiruvaMte aLavaDisi, adara sahAyadiMda BwgoVLika utatxravanUnx, adariMda itara dikukxgaLanUnx niNaRyisuva upakaraNa. 
\emng
\eentry

\bentry
\word{gyrograph}
\pron{jeYaroVgArxphf}
\gl{\nA}
\bmng
 jeYroVgArxphu; pariBarxmaNaleVKi; pariBarxmaNagaLa saMKeyxyanunx dAKalisuva upakaraNa. 
\emng
\eentry

\bentry
\word{gyromagnetic}
\pron{jeYaroVmAYxgenxTikf}
\gl{\gu}
\bmng
 BarxmaNakAMtiVya; giraki hoDeyutitxruva viduyxtakxNada kAMtiVya guNalakaSxNagaLa. 
\emng
\eentry

\bentry
\word{gyropilot}
\pron{jeYaroVpeYlaTf}
\gl{\nA}
\bmng
 jeYroVcAlaka; savxyaMcAlaneyalilx baLasuva jeYroV dikUsxci. 
\emng
\eentry

\bentry
\word{gyroplane}
\pron{jeYaroVpelxVnf}
\gl{\nA}
\bmng
 BarxmaNa vimAna; girigaTeTx vimAna; meVlABxgadalilx sututxtitxruva girigaTeTxya sahAyadiMda Eruva vimAna. 
\emng
\eentry

\bentry
\word{gyroscope}
\pron{jeYarasokxVpf}
\gl{\nA}
\bmng
 jeYroVsokxVpu; BarxmaNa dashaRka: 
\banum
\alnum{a} giriki hoDeyuva kAyagaLa calaneya lakaSxNagaLanunx nidashiRsuva upakaraNa. 
\alnum{b} giriki hoDeyutitxruva cakarxvoMdara akaSxvu saliVsAgi dikukx badalisuvaMtidudx, adanunx hotitxruva kAyavu yAva kaDege tirugikoMDarU tAnu mAtarx sadA oMdeV dikikxge tirugikoMDiruvaMte EpARTu mADiruva oMdu upakaraNa. \imglink{gyroscopefigure}{\raisebox{-0.30cm}[0pt][0pt]{\pdfimage width 0.6cm height 0.7cm {G_Pictures/gyroscope.jpg}}} 
\eanum
\emng
\eentry

\bentry
\word{gyroscopic}
\pron{jeYarasAkxpikf}
\gl{\gu}
\bmng
\bnum
\num{1} BarxmaNadashaRkada; jeYroVsokxVpina; BarxmaNadashaRkakekx saMbaMdhisida. 
\num{2} pariBarxmaNada. 
\enum
\emng
\eentry

\bentry
\word{gyrose}
\pron{jeYroVsf}
\gl{\gu}
\bmng
 (\savi) 
\bnum
\num{1} nirigeyuLaLx. 
\num{2} alegaLaMtha geregaLuLaLx. 
\enum
\emng
\eentry

\bentry
\word{gyrostabilizer}
\pron{jeYaroVseTxVbileYsaZrf}
\gl{\nA}
\bmng
 jeYroVseTxbileYsaZrf; haDagu, vimAna, \mo vugaLa samatoVlanavanunx kApADalu baLasuva, jeYroVsokxVpinaMtha sAdhana. 
\emng
\eentry

\bentry
\word{gyrostat}
\pron{jeYarasATxTf}
\gl{\nA}
\bmng
  = \hyperlink{gyrostabilizer}{gyrostabilizer}. 
\emng
\eentry

\bentry
\word{gyrotiller}
\pron{jeYaroVTilarf}
\gl{\nA}
\bmng
 girakikuMTe; jeYroV kuMTe; oMdu acicxna sutatxlU sututxva halulxgaLiruva kuMTe. 
\emng
\eentry

\bentry
\word{gyrus}
\pron{jeYarasf}
\gl{\nA}
\expl{(\bava\ \eng{gyri} \ucAcx- jeYareY).}
\bmng
(\kanmu\ miduLina) maDike; sututx. 
\emng
\eentry

\bentry
\word{gyttja}
\pron{yica}
\gl{\nA}
\bmng
 (\BUvi) yica; jeYvika padAthaRgaLiMda tuMbida, himayugada naMtara saroVvaragaLalilx nikeSxVpagoMDa, \kanmu\ kapupx baNaNxda maNuNx. 
\emng
\eentry

\bentry
\word[gyve(1)]{gyve}
\pron{jeYvf}
\gl{\nA}
\bmng
 (\sA\ \bava dalilx) (\kAparx) saMkoVle; beVDi; koVLa. 
\emng
\eentry

\bentry
\word[gyve(2)]{gyve}
\pron{jeYvf}
\gl{\sakirx}
\bmng
 (\kAparx) saMkoVle toDisu; beVDihAku; koVLahAku. 
\emng
\eentry

%\newpage%\input figs_list
